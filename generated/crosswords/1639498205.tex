
\documentclass[10pt]{article}
\usepackage[utf8]{inputenc}

\usepackage[unboxed]{cwpuzzle}
\usepackage[top=15mm,bottom=15mm,left=15mm,right=3cm,twoside]{geometry}
\usepackage{adjustbox}

\newcommand\drarr{$\rightarrow \!\!\!\!\! \downarrow$}
\newcommand\rarr{$\rightarrow$}
\newcommand\darr{$\downarrow$}

\begin{document}



\newpage\section*{Krzyżówka 1}

\noindent\begin{Puzzle}{22}{24}|*	|*	|*	|[1][S]\darr	|[2][S]\darr	|*	|*	|*	|*	|*	|*	|[3][S]\darr	|*	|*	|*	|*	|*	|[4][S]\darr	|*	|*	|[5][S]\darr	|*	|*	|.
|*	|*	|*	|m	|h	|[6][S]\darr	|*	|[7][S]\rarr	|b	|o	|b	|o	|w	|i	|a	|n	|k	|a	|*	|*	|h	|*	|*	|.
|*	|*	|*	|a	|o	|n	|[8][S]\drarr	|a	|z	|o	|t	|a	|n	|[][,]{ }	|g	|l	|i	|n	|u	|*	|u	|*	|[9][S]\darr	|.
|*	|[10][S]\darr	|[11][S]\rarr	|k	|r	|u	|p	|y	|*	|*	|*	|z	|[12][S]\darr	|[13][S]\rarr	|b	|a	|ł	|a	|m	|u	|t	|*	|b	|.
|*	|p	|[14][S]\darr	|a	|d	|r	|i	|*	|*	|[15][S]\drarr	|ł	|a	|s	|k	|u	|n	|*	|s	|*	|*	|n	|*	|ą	|.
|*	|ł	|m	|r	|a	|o	|ł	|*	|*	|z	|[16][S]\darr	|*	|z	|*	|*	|*	|*	|t	|*	|*	|i	|*	|k	|.
|*	|a	|i	|t	|*	|*	|k	|*	|*	|o	|k	|[17][S]\rarr	|p	|i	|ł	|k	|a	|r	|z	|y	|k	|i	|*	|.
|*	|s	|l	|*	|[18][S]\rarr	|n	|a	|d	|z	|o	|r	|c	|a	|[][,]{ }	|s	|ą	|d	|o	|w	|y	|*	|*	|*	|.
|[19][S]\drarr	|k	|a	|g	|e	|l	|*	|*	|*	|*	|y	|*	|d	|*	|[20][S]\rarr	|m	|u	|f	|k	|a	|*	|*	|*	|.
|g	|o	|[][,]{ }	|[21][S]\darr	|*	|[22][S]\drarr	|f	|r	|u	|s	|t	|r	|a	|c	|j	|a	|*	|a	|*	|*	|*	|*	|*	|.
|ł	|s	|n	|h	|[23][S]\drarr	|p	|o	|l	|o	|n	|e	|z	|*	|[24][S]\rarr	|s	|e	|r	|*	|*	|*	|*	|*	|*	|.
|u	|z	|a	|o	|i	|u	|*	|*	|*	|[25][S]\rarr	|r	|u	|m	|i	|a	|n	|e	|k	|*	|*	|*	|*	|[26][S]\darr	|.
|p	|*	|[][,]{ }	|m	|n	|c	|*	|*	|[27][S]\rarr	|p	|i	|ą	|t	|a	|[][,]{ }	|c	|z	|ę	|ś	|ć	|*	|*	|p	|.
|k	|*	|g	|e	|t	|e	|[28][S]\darr	|*	|[29][S]\rarr	|b	|u	|r	|g	|e	|r	|*	|*	|*	|*	|*	|*	|*	|r	|.
|o	|*	|o	|r	|e	|k	|s	|*	|[30][S]\rarr	|s	|m	|u	|k	|w	|o	|w	|a	|t	|e	|*	|[31][S]\darr	|*	|z	|.
|w	|*	|d	|y	|r	|*	|o	|*	|*	|*	|[][,]{ }	|*	|*	|[32][S]\drarr	|r	|o	|m	|f	|o	|r	|d	|*	|y	|.
|a	|[33][S]\rarr	|z	|d	|r	|a	|d	|z	|i	|e	|c	|t	|w	|o	|*	|*	|*	|*	|*	|*	|o	|[34][S]\darr	|c	|.
|t	|*	|i	|a	|e	|[35][S]\darr	|a	|[36][S]\drarr	|z	|j	|a	|w	|i	|s	|k	|o	|*	|*	|*	|*	|m	|r	|h	|.
|o	|*	|n	|*	|g	|h	|*	|m	|*	|*	|ł	|*	|[37][S]\rarr	|ł	|u	|s	|k	|a	|*	|*	|o	|o	|a	|.
|ś	|*	|ę	|[38][S]\rarr	|n	|o	|g	|a	|*	|*	|k	|*	|[39][S]\rarr	|o	|k	|r	|e	|s	|*	|*	|w	|d	|c	|.
|ć	|*	|*	|*	|u	|n	|[40][S]\rarr	|d	|y	|s	|o	|n	|a	|n	|s	|*	|*	|*	|*	|*	|i	|r	|i	|.
|*	|*	|*	|*	|m	|g	|*	|d	|*	|*	|w	|*	|[41][S]\rarr	|k	|a	|ł	|k	|a	|n	|*	|n	|i	|e	|.
|*	|*	|*	|*	|*	|z	|*	|e	|*	|*	|e	|[42][S]\rarr	|w	|a	|t	|e	|r	|s	|z	|t	|a	|g	|*	|.
|*	|[43][S]\rarr	|d	|z	|i	|e	|l	|n	|i	|k	|*	|*	|*	|*	|*	|*	|*	|*	|*	|*	|*	|o	|*	|.
|[44][S]\rarr	|b	|a	|e	|z	|*	|*	|*	|*	|*	|*	|*	|*	|*	|*	|*	|*	|*	|*	|*	|*	|*	|*	|.\end{Puzzle}

\newpage

\begin{PuzzleClues}{\textbf{Poziome}\\}\Clue{7}{}{mieszkanka Bobowej}
\Clue{8}{}{sól kwasu azotowego i glinu na III stopniu utlenienia}
\Clue{11}{}{staropolska nazwa kaszy}
\Clue{13}{}{uwodziciel - mężczyzna, który uwodzi kobiety}
\Clue{15}{}{nadrzewny lub naziemny ssak z rodziny łasz}
\Clue{17}{}{gra w piłkarzyki - przedmiot}
\Clue{18}{}{organ postępowania upadłościowego z możliwością zawarcia układu, w którym ustanowiono zarząd własny upadłego lub postępowania naprawczego, ustanawiany przez sąd; funkcjonariusz publiczny}
\Clue{19}{}{argentyński kompozytor i dyrygent ur. w 1931r., utwory kameralne wokalno-instrumentalne, sceniczne, elektroniczne}
\Clue{20}{}{mała mufa; łącznik rur}
\Clue{22}{}{zespół przykrych emocji związanych z niemożliwością realizacji potrzeby lub osiągnięcia określonego celu}
\Clue{23}{}{model samochodu osobowego produkowany przez Fabrykę Samochodów Osobowych w Warszawie od 3 maja 1978 roku do 22 kwietnia 2002 roku}
\Clue{24}{}{porcja sera, zazwyczaj kostka, ale też opakowanie; określona ilość sera}
\Clue{25}{}{roślina zielna ze złożonych, aromatyczny chwast polny, uprawiany także jako roślina lecznicza}
\Clue{27}{}{inaczej: jedna piąta, jedna z pięciu części czegoś podzielnego}
\Clue{29}{}{kanapka z miękkiej bułki z wieloma dodatkami w środku, z których to dodatków najbardziej charakterystycznym dla tej potrawy jest gruby kotlet o takiej samej nazwie}
\Clue{30}{}{Scoliidae - rodzina owadów z grupy żądłówek}
\Clue{32}{}{miasto w Anglii w hrabstwie Essex na płn.-wsch. od Londynu}
\Clue{33}{}{cecha człowieka, który jest zdradziecki}
\Clue{36}{}{widzenie senne lub mara}
\Clue{37}{}{element struktury, cienka warstwa stałej substancji często podobna do łuski zwierzęcej - kostnej lub rogowej płytki, która wraz z innymi łuskami tworzy pokrywę ciała}
\Clue{38}{}{część jakiegoś sprzętu, która służy jako wspornik, podstawa}
\Clue{39}{}{miesiączka}
\Clue{40}{}{rym niedokładny, nie obejmuje przedostatniej akcentowanej samogłoski wersu}
\Clue{41}{}{lekka, okrągła, silnie wypukła tarcza, typowa dla krajów wschodnich, takich jakich Persja bądź Turcja, znana także w Polsce oraz na Węgrzech}
\Clue{42}{}{lina łącząca nok bukszprytu z dziobem statku, usztywnia bukszpryt}
\Clue{43}{}{łącznik, dywiz, tiret - drukarski znak graficzny, krótka pozioma kreska umieszczona ponad podstawową linią pisma; służy do oznaczenia łączenia wyrazów złożonych, nazw podwójnych lub przeniesienia części wyrazu do następnego wiersza}
\Clue{44}{}{ur. w 1941 r., piosenkarka amerykańska; występuje przeciw wojnie, rasizmowi}\end{PuzzleClues}

\begin{PuzzleClues}{\textbf{Pionowe}\\}\Clue{1}{}{malarz austriacki (1840-84) kompozycje historyczne i alegoryczne o charakterze eklektycznym}
\Clue{2}{}{u Turków, Tatarów: wojsko, obóz wojskowy, orda}
\Clue{3}{}{teren o bardzo bujnej roślinności na obszarze pustyń i półpustyń}
\Clue{4}{}{rodzaj inwersji, przestawienie wyrazów w związku frazeologicznym}
\Clue{5}{}{pracownik huty szkła lub metalu, specjalista w dziedzinie hutnictwa}
\Clue{6}{}{miasto we Włoszech (Sardynia), ośrodek turystyczny}
\Clue{8}{}{przedmiot służący do gry i zabawy; sprężysta kula różnej wielkości. }
\Clue{9}{}{dziecko, zwłaszcza małe}
\Clue{10}{}{Ramaria Bonord. - rodzaj wielkoowocnikowych grzybów podstawkowych należący do rodziny siatkolistowatych (Gomphaceae), którego gatunkiem typowym jest koralówka czerwonowierzchołkowa (Ramaria botrytis)}
\Clue{12}{}{sportowa broń kolna, posiada klingę o trójkątnym przekroju i duży kosz, osłaniający dłoń zawodnika}
\Clue{14}{}{stosowana w krajach anglosaskich jednostka prędkości, oznaczana mph}
\Clue{15}{}{teren udostępniony odwiedzającym, na którym hodowane są zwierzęta, najczęściej pochodzące z różnych obszarów geograficznych}
\Clue{16}{}{metoda sprawdzania, czy nieskończony szereg liczbowy o nieujemnych wyrazach jest zbieżny}
\Clue{19}{}{pobłażliwie, żartobiliwie lub z ironią o czyjejś głupocie; cecha kogoś, kto postrzegany jest jako głupkowaty}
\Clue{21}{}{rapsod rozpowszechniający poezję Homera}
\Clue{22}{}{pyzaty mężczyzna, chłopiec}
\Clue{23}{}{hist. polit}
\Clue{26}{}{miejsce przy chacie, w najbliższym otoczeniu domu, obszar otaczający chatę}
\Clue{28}{}{wodorowęglan sodu, soda oczyszczona (NaHCO3) - nieorganiczny związek chemiczny z grupy wodorowęglanów, wodorosól kwasu węglowego i sodu, która stosowana jest głównie jako jeden ze składników proszku do pieczenia i dodatek do żywności regulujący pH}
\Clue{31}{}{trumna - specjalna skrzynia, w której spoczywa ciała zmarłego}
\Clue{32}{}{struktura rozwijająca się początkowo u nasady zalążka roślin nasiennych i w miarę rozwoju stopniowo go obrastająca}
\Clue{34}{}{ur. w 1902 r., kompozytor hiszpański; w twórczości nawiązuje do rodzinnego folkloru; koncerty, utwory kameralne, fortepianowe, gitarowe}
\Clue{35}{}{jezioro w Chinach na Nizinie Chińskiej}
\Clue{36}{}{jezioro w Panamie, z jeziora wypływa rzeka Chagres}\end{PuzzleClues}\newpage\section*{Krzyżówka 2}

\noindent\begin{Puzzle}{23}{33}|*	|*	|*	|*	|*	|*	|*	|*	|*	|*	|*	|*	|*	|*	|*	|*	|*	|*	|*	|*	|*	|*	|[1][S]\darr	|*	|.
|*	|*	|*	|*	|*	|*	|*	|*	|*	|*	|*	|*	|*	|*	|*	|*	|[2][S]\darr	|*	|*	|[3][S]\darr	|*	|*	|w	|*	|.
|*	|*	|*	|*	|*	|*	|*	|*	|*	|*	|*	|*	|*	|*	|*	|*	|p	|*	|*	|p	|*	|*	|e	|*	|.
|*	|*	|*	|*	|*	|*	|*	|*	|*	|*	|[4][S]\rarr	|c	|z	|e	|c	|h	|o	|s	|ł	|o	|w	|a	|k	|*	|.
|*	|*	|*	|*	|*	|[5][S]\darr	|*	|*	|*	|*	|*	|*	|*	|*	|*	|*	|b	|[6][S]\darr	|*	|t	|*	|*	|t	|*	|.
|*	|*	|*	|*	|*	|t	|*	|*	|*	|*	|*	|*	|*	|*	|*	|*	|o	|s	|[7][S]\darr	|a	|*	|*	|o	|*	|.
|*	|[8][S]\darr	|*	|*	|*	|ł	|*	|*	|*	|*	|*	|*	|*	|*	|*	|*	|ż	|k	|r	|ż	|*	|*	|r	|*	|.
|*	|a	|*	|*	|*	|u	|*	|*	|*	|*	|*	|*	|*	|*	|*	|*	|n	|r	|e	|[][,]{ }	|*	|*	|[][,]{ }	|*	|.
|*	|n	|*	|*	|*	|s	|*	|[9][S]\darr	|*	|*	|*	|*	|*	|*	|*	|*	|o	|z	|z	|ż	|*	|*	|k	|*	|.
|*	|a	|*	|*	|*	|t	|*	|k	|*	|*	|*	|*	|*	|*	|*	|*	|ś	|y	|y	|r	|*	|*	|o	|*	|.
|*	|t	|*	|*	|*	|o	|*	|r	|*	|*	|[10][S]\darr	|*	|*	|*	|[11][S]\darr	|*	|ć	|p	|d	|ą	|*	|*	|l	|*	|.
|*	|o	|*	|*	|*	|g	|*	|o	|*	|[12][S]\rarr	|a	|m	|b	|o	|n	|a	|*	|i	|e	|c	|*	|*	|u	|*	|.
|*	|m	|*	|*	|*	|o	|*	|k	|*	|*	|n	|*	|*	|[13][S]\darr	|i	|*	|*	|c	|n	|y	|*	|*	|m	|*	|.
|*	|i	|*	|*	|*	|n	|*	|o	|*	|*	|d	|*	|[14][S]\rarr	|s	|e	|r	|v	|e	|t	|*	|*	|*	|n	|*	|.
|*	|a	|*	|*	|*	|[][,]{ }	|*	|d	|*	|*	|o	|*	|*	|y	|c	|*	|*	|*	|u	|*	|*	|*	|o	|*	|.
|*	|[][,]{ }	|*	|*	|*	|a	|*	|y	|*	|*	|r	|*	|*	|l	|h	|[15][S]\darr	|*	|*	|r	|*	|*	|*	|w	|*	|.
|*	|r	|*	|*	|[16][S]\darr	|f	|[17][S]\darr	|l	|*	|*	|k	|*	|*	|w	|l	|p	|*	|[18][S]\darr	|a	|*	|*	|*	|y	|*	|.
|*	|a	|*	|*	|p	|r	|u	|[][,]{ }	|*	|*	|a	|*	|*	|e	|u	|r	|*	|n	|*	|*	|*	|*	|*	|*	|.
|*	|d	|*	|*	|r	|y	|n	|b	|*	|*	|*	|*	|*	|t	|j	|e	|*	|a	|*	|*	|*	|*	|*	|*	|.
|*	|i	|*	|*	|ą	|k	|i	|ł	|*	|[19][S]\drarr	|b	|e	|z	|a	|n	|m	|a	|s	|z	|t	|*	|*	|*	|*	|.
|*	|o	|*	|*	|d	|a	|w	|o	|[20][S]\darr	|k	|*	|*	|*	|*	|o	|i	|*	|z	|*	|*	|*	|*	|*	|*	|.
|*	|l	|*	|*	|[][,]{ }	|ń	|e	|t	|k	|o	|*	|*	|*	|*	|ś	|a	|*	|[][,]{ }	|*	|*	|*	|*	|*	|*	|.
|*	|o	|*	|*	|g	|s	|r	|n	|a	|z	|*	|*	|*	|*	|ć	|[][,]{ }	|*	|c	|*	|*	|*	|*	|*	|*	|.
|*	|g	|[21][S]\rarr	|m	|a	|k	|s	|y	|m	|a	|l	|i	|z	|m	|*	|o	|*	|z	|*	|*	|*	|*	|*	|*	|.
|*	|i	|*	|*	|l	|i	|y	|*	|y	|*	|*	|*	|*	|*	|*	|p	|*	|ł	|*	|*	|*	|*	|*	|*	|.
|*	|c	|*	|*	|w	|*	|t	|*	|k	|*	|*	|[22][S]\rarr	|p	|s	|z	|c	|z	|o	|l	|i	|n	|k	|i	|*	|.
|[23][S]\rarr	|z	|ę	|b	|a	|t	|e	|k	|*	|*	|*	|*	|*	|*	|*	|y	|*	|w	|*	|*	|*	|*	|*	|*	|.
|*	|n	|*	|*	|n	|*	|c	|*	|*	|*	|*	|*	|*	|*	|*	|j	|*	|i	|*	|*	|*	|*	|*	|*	|.
|*	|a	|*	|*	|i	|*	|k	|*	|*	|*	|*	|*	|*	|*	|*	|n	|*	|e	|*	|*	|*	|*	|*	|*	|.
|*	|*	|*	|*	|c	|*	|o	|*	|*	|*	|*	|*	|*	|*	|*	|a	|*	|k	|*	|*	|*	|*	|*	|*	|.
|*	|*	|*	|*	|z	|*	|ś	|*	|*	|*	|*	|*	|*	|*	|*	|*	|*	|*	|*	|*	|*	|*	|*	|*	|.
|*	|*	|*	|*	|n	|*	|ć	|*	|*	|*	|*	|*	|*	|*	|*	|*	|*	|*	|*	|*	|*	|*	|*	|*	|.
|[24][S]\rarr	|l	|i	|r	|y	|k	|*	|*	|*	|*	|*	|*	|*	|*	|*	|*	|*	|*	|*	|*	|*	|*	|*	|*	|.
|[25][S]\rarr	|b	|ó	|b	|*	|*	|*	|*	|*	|*	|*	|*	|*	|*	|*	|*	|*	|*	|*	|*	|*	|*	|*	|*	|.\end{Puzzle}

\newpage

\begin{PuzzleClues}{\textbf{Poziome}\\}\Clue{4}{}{mieszkaniec Czechosłowacji, państwa w Europie (1945-1990)}
\Clue{12}{}{KAZALNICA}
\Clue{14}{}{hiszpański teolog i lekarz (1511-53); odkrywca małego (płucnego) krążenia krwi, za kwestionowanie dogmatu Trójcy Św. spalony na stosie}
\Clue{19}{}{nazwa tylnego żagla na jednostce żaglowej trzy- lub więcej-masztowej; jeżeli statek ma 2 maszty, to ostatni może być nazwany bezanmasztem tylko wtedy, gdy pierwszy to grotmaszt}
\Clue{21}{}{koncepcja filozoficzna opierająca się na dążeniu do wyjaśnienia całej rzeczywistości, zmierzająca do systematycznego rozwinięcia wszystkich działów filozofii}
\Clue{22}{}{Andreninae - podrodzina błonkówek występujących prawie na całym świecie przy czym największą różnorodność prezentuje jeden rodzaj - pszczolinka (Andrena), która zawiera 1300 gatunków}
\Clue{23}{}{australijski ptak z rzędu wróblowatych, rodzina altanników}
\Clue{24}{}{poeta, twórca poezji, wierszy}
\Clue{25}{}{Vicia faba - gatunek jednorocznej rośliny uprawnej z rodziny bobowatych}\end{PuzzleClues}

\begin{PuzzleClues}{\textbf{Pionowe}\\}\Clue{1}{}{macierz o wymiarze mx1}
\Clue{2}{}{cecha człowieka, który żyje pobożnie, jest wierzący i spełnia praktyki religijne}
\Clue{3}{}{mieszanina substacji, o dominującej zawartości wodorotlenku potasu, powstająca w procesie kaustyfikacji z potażu}
\Clue{5}{}{Pachyuromys duprasi - gatunek niewielkiego gryzonia z podrodziny suwaków, jedyny przedstawiciel rodzaju Pachyuromys; występuje w Afryce Północnej w północnej części pustyni leżącej na zachód od delty Nilu w Egipcie, a literatura opisuje rownież jego występowanie na terenach Algierii, Libii, Tunezji}
\Clue{6}{}{instrument ludowy, prototyp skrzypiec}
\Clue{7}{}{forma doskonalenia zawodowego umożliwiająca zdobycie określonej specjalizacji lekarskiej finansowana przez Ministerstwo Zdrowia}
\Clue{8}{}{dział anatomii, pierwotnie opisujący organizm ludzki przy użyciu zdjęć rentgenowskich układu kostnego}
\Clue{9}{}{Crocodylus palustris - gatunek gada z rodziny krokodylowatych podobny do krokodyla nilowego, występujący w Iranie, Pakistanie, Indiach, Nepalu, Bangladeszu i Sri Lance}
\Clue{10}{}{mieszkanka Andory, kobieta pochodzenia andorskiego}
\Clue{11}{}{to, że coś wygląda niechlujnie, jest zrobione, przeprowadzone niechlujnie - bez dbałości o poprawność, wygląd, szczegóły; to, że coś świadczy o czyjejś niechlujności w wykonywaniu czegoś}
\Clue{13}{}{masywna figura człowieka lub zwierzęcia, niezbyt dobrze widoczna}
\Clue{15}{}{kwota, jaką dostaje sprzedawca opcji od jej kupca}
\Clue{16}{}{stały prąd charakteryzujący się niskim napięciem i małym natężeniem; nie wywołuje skurczu mięśni, znajduje zastosowanie w medycynie}
\Clue{17}{}{to, że coś jest uniwersytetem, cecha uczelni wyższej wyrażająca się w używaniu przez nią nazwyUniwersytet, wynikająca z charakteru instytucji jako jednostki prowadzącej działalność naukową, badawczą, mającej uprawnienia do nadawania tytułów naukowych}
\Clue{18}{}{osoba, która należy do naszego środowiska lub ugrupowania, która stoi po naszej stronie}
\Clue{19}{}{dawny pogański obrzęd, który w tradycji ludowej przetrwał do połowy XX wieku}
\Clue{20}{}{orzech ziemny w kolorowej polewie}\end{PuzzleClues}\newpage\section*{Krzyżówka 3}

\noindent\begin{Puzzle}{22}{31}|*	|*	|*	|*	|*	|*	|*	|*	|[1][S]\darr	|[2][S]\drarr	|s	|p	|o	|t	|*	|*	|*	|*	|[3][S]\drarr	|l	|u	|z	|*	|.
|*	|*	|*	|*	|*	|*	|*	|[4][S]\darr	|f	|r	|*	|*	|*	|*	|*	|[5][S]\drarr	|ł	|a	|s	|k	|u	|n	|*	|.
|*	|*	|*	|*	|*	|[6][S]\darr	|*	|k	|e	|a	|*	|*	|*	|*	|*	|p	|[7][S]\darr	|*	|z	|*	|*	|[8][S]\darr	|*	|.
|*	|*	|*	|[9][S]\darr	|*	|s	|*	|a	|l	|c	|*	|*	|*	|*	|*	|o	|b	|*	|o	|*	|*	|s	|*	|.
|*	|*	|*	|k	|*	|a	|*	|w	|d	|h	|*	|*	|*	|*	|*	|l	|o	|[10][S]\darr	|p	|*	|*	|z	|*	|.
|*	|*	|*	|r	|*	|ł	|*	|a	|j	|u	|[11][S]\darr	|*	|*	|[12][S]\drarr	|m	|i	|r	|i	|*	|*	|*	|k	|*	|.
|*	|*	|*	|a	|*	|a	|*	|l	|e	|n	|ł	|[13][S]\darr	|*	|p	|[14][S]\darr	|c	|o	|s	|[15][S]\darr	|*	|*	|a	|*	|.
|*	|*	|*	|ś	|*	|t	|*	|e	|g	|k	|u	|w	|*	|u	|f	|j	|w	|e	|c	|*	|*	|r	|*	|.
|*	|*	|*	|n	|*	|a	|[16][S]\rarr	|r	|e	|i	|k	|i	|*	|c	|r	|a	|c	|o	|w	|*	|*	|ł	|*	|.
|*	|*	|*	|i	|*	|[][,]{ }	|*	|*	|r	|*	|*	|w	|[17][S]\darr	|h	|o	|n	|e	|*	|a	|*	|*	|a	|*	|.
|*	|*	|*	|k	|*	|m	|*	|*	|*	|[18][S]\rarr	|d	|e	|s	|a	|n	|t	|*	|*	|n	|*	|[19][S]\darr	|t	|*	|.
|[20][S]\drarr	|f	|u	|[][,]{ }	|z	|o	|n	|g	|*	|*	|*	|r	|p	|c	|t	|*	|*	|*	|o	|*	|w	|k	|*	|.
|d	|[21][S]\drarr	|o	|p	|ó	|r	|*	|[22][S]\rarr	|o	|ł	|t	|a	|r	|z	|*	|*	|*	|[23][S]\darr	|ś	|*	|a	|a	|*	|.
|z	|p	|*	|i	|*	|s	|*	|[24][S]\rarr	|d	|e	|j	|*	|z	|y	|*	|*	|*	|f	|ć	|[25][S]\darr	|r	|[][,]{ }	|*	|.
|i	|r	|*	|ę	|[26][S]\rarr	|k	|u	|j	|a	|[][,]{ }	|b	|ł	|ę	|k	|i	|t	|n	|a	|*	|n	|t	|k	|*	|.
|e	|e	|*	|c	|*	|a	|*	|*	|*	|*	|*	|*	|t	|[][,]{ }	|*	|*	|*	|z	|*	|e	|o	|r	|*	|.
|r	|c	|*	|i	|*	|*	|[27][S]\rarr	|k	|l	|a	|n	|g	|*	|g	|*	|*	|*	|a	|[28][S]\darr	|g	|ś	|ó	|*	|.
|z	|j	|*	|o	|[29][S]\rarr	|h	|i	|p	|o	|t	|e	|l	|o	|r	|y	|z	|m	|*	|s	|a	|c	|l	|*	|.
|b	|o	|[30][S]\drarr	|p	|y	|ł	|k	|o	|j	|a	|d	|[][,]{ }	|s	|z	|a	|r	|y	|*	|k	|c	|i	|e	|*	|.
|i	|z	|w	|l	|*	|*	|*	|[31][S]\rarr	|s	|z	|a	|c	|h	|y	|*	|*	|*	|*	|a	|y	|o	|w	|*	|.
|k	|a	|a	|a	|*	|[32][S]\rarr	|m	|i	|e	|j	|s	|c	|o	|w	|y	|*	|*	|[33][S]\darr	|k	|j	|w	|s	|*	|.
|[][,]{ }	|*	|c	|m	|*	|*	|*	|*	|*	|*	|*	|[34][S]\rarr	|v	|i	|p	|*	|*	|p	|u	|n	|o	|k	|*	|.
|z	|[35][S]\drarr	|h	|e	|l	|l	|e	|n	|i	|s	|t	|y	|k	|a	|*	|[36][S]\darr	|*	|e	|n	|o	|ś	|a	|*	|.
|b	|m	|l	|k	|*	|*	|*	|*	|[37][S]\darr	|*	|[38][S]\rarr	|d	|o	|s	|t	|o	|j	|n	|o	|ś	|ć	|*	|*	|.
|r	|a	|a	|*	|[39][S]\drarr	|s	|e	|w	|a	|n	|*	|*	|*	|t	|*	|b	|*	|i	|w	|ć	|*	|*	|*	|.
|o	|g	|r	|*	|b	|*	|*	|*	|s	|[40][S]\rarr	|p	|a	|s	|y	|w	|i	|s	|t	|a	|*	|*	|*	|*	|.
|c	|g	|z	|*	|u	|*	|*	|[41][S]\rarr	|p	|o	|p	|o	|w	|*	|*	|*	|*	|e	|t	|*	|*	|*	|*	|.
|z	|i	|*	|*	|s	|*	|*	|[42][S]\rarr	|a	|n	|d	|r	|o	|g	|e	|n	|*	|n	|e	|*	|*	|*	|*	|.
|n	|o	|[43][S]\rarr	|l	|a	|s	|[][,]{ }	|d	|z	|i	|e	|w	|i	|c	|z	|y	|*	|t	|*	|*	|*	|*	|*	|.
|y	|r	|*	|*	|*	|*	|*	|*	|j	|*	|*	|*	|*	|*	|*	|*	|*	|*	|*	|*	|*	|*	|*	|.
|*	|e	|[44][S]\rarr	|k	|o	|ń	|[][,]{ }	|k	|a	|b	|a	|r	|d	|y	|ń	|s	|k	|i	|*	|*	|*	|*	|*	|.
|*	|*	|[45][S]\rarr	|k	|l	|i	|p	|s	|*	|*	|*	|*	|*	|*	|*	|*	|*	|*	|*	|*	|*	|*	|*	|.\end{Puzzle}

\newpage

\begin{PuzzleClues}{\textbf{Poziome}\\}\Clue{2}{}{lampa, która daje oświetlenie w postaci snopu światła zbliżonego w kształcie do stożka}
\Clue{3}{}{różnica między wymiarem otworu a wymiarem wałka przednich połączeniem}
\Clue{5}{}{ssak z podrodziny łaskunów, niewielkich ssaków drapieżnych w rodzinie łaszowatych; występuje w południowej i południowo-wschodniej Azji}
\Clue{12}{}{miasto w Malezji na płn. wybrzeżu Borneo (Sarawak), ważny port naftowy}
\Clue{16}{}{duchowy wpływ, energia duchowa}
\Clue{18}{}{wojska biorące udział w takiej operacji}
\Clue{20}{}{ur. w 1934 r., pianista chiński; laureat Konkursu im. F. Chopina w 1955 r}
\Clue{21}{}{siła działająca na poruszające się ciało fizyczne, która przeciwdziała poruszaniu się tego ciała}
\Clue{22}{}{w wolnomularstwie: stół z istotnymi dla masonerii przedmiotami, symbolami, który stoi w loży masońskiej}
\Clue{24}{}{miasto w Rumunii, okręg Kluż; przemysł drzewny}
\Clue{26}{}{Coua caerulea - gatunek ptaka z rodziny kukułkowatych (Cuculidae), z podrodziny kuji (Couinae)}
\Clue{27}{}{uderzenie w dzwon okrętowy}
\Clue{29}{}{wada wrodzona, objawiająca się zmniejszeniem rozstawu gałek ocznych}
\Clue{30}{}{Conopophila whitei - gatunek ptaka z rodziny miodojadów (Meliphagidae) występujący w Australii i na Nowej Gwinei}
\Clue{31}{}{gra prowadzona na szachownicy przez dwóch przeciwników, z których każdy dysponuje 16 figurami}
\Clue{32}{}{człowiek, który mieszka w tym miejscu, człowiek, który jest stąd}
\Clue{34}{}{peptydowy hormon składający się z 28 reszt aminokwasowych; u człowieka produkowany w jelitach (komórki D1), trzustce i niektórych strukturach mózgu}
\Clue{35}{}{nauka o kulturze i języku Grecji starożytnej i współczesnej}
\Clue{38}{}{powaga, elegancja, wyraz czyjegoś dostojeństwa}
\Clue{39}{}{jezioro w Armenii, na wysokości 1900 m, powierzchnia 1200 km2 z jeziora wypływa rzeka Razdan}
\Clue{40}{}{zwolennik pasywizmu tj. kierunku politycznego popularnego w czasie I wojny światowej i okresie międzywojennym w Polsce}
\Clue{41}{}{fizyk rosyjski (1859-1906); pionier radiotechniki, zbudował radiotelegraf i uzyskał łączność radiową na odległość 5 km}
\Clue{42}{}{hormon płciowy o budowie steroidowej o działaniu maskulinizującym, fizjologicznie występujący u mężczyzn; w małych stężeniach występuje również u kobiet}
\Clue{43}{}{las, w którego kształtowaniu człowiek nie brał udziału}
\Clue{44}{}{stara rasa koni wywodząca się z Kaukazu, charakteryzująca się wytrzymałością ,doskonałą równowagą i stabilnym chodem w trudnym terenie, zdolnością do pokonywania długich dystansów, odwagą i inteligencją; wytworzyła się w wyniku mieszania wielu ras koni stepowych (nogajskie, kałmuckie, baszkirskie, donieckie) i szlachetnych (karabachskie, perskie, achałtekińskie, arabskie) z rasami mongolskimi}
\Clue{45}{}{biżuteria damska, przyczepiana do płatków uszu uchwytem w formie klamerki}\end{PuzzleClues}

\begin{PuzzleClues}{\textbf{Pionowe}\\}\Clue{1}{}{XIX w. żołnierz batalionu strzelców w armii austro-węgierskiej i w Rosji, pełniący służbę żandarma}
\Clue{2}{}{lekcje matematyki, na których uczy się liczenia, arytmetyki}
\Clue{3}{}{rodzaj ssaków z rodziny szopowatych}
\Clue{4}{}{tytuł przysługujący odznaczonym niektórymi orderami}
\Clue{5}{}{osoba należąca do służb policji}
\Clue{6}{}{błonica sałatowa, ulwa sałatowa, Ulva lactuca - gatunek roślin z gromady zielenic (Chlorophyta)}
\Clue{7}{}{Nyctalus - rodzaj nietoperzy z rodziny mroczkowatych}
\Clue{8}{}{Alisterus scapularis - gatunek ptaka z rodziny papugowatych (Psittacidae), z podrodziny papug wschodnich (Psittaculinae)}
\Clue{9}{}{Zygaena trifolii - gatunek motyla z rodziny kraśnikowatych; występuje na większości obszaru Europy; brak go w Norwegii, Finlandii, Irlandii i na Bałkanach}
\Clue{10}{}{jezioro we Włoszech, w Alpach Lombardzkich, powierzchnia 65 km2, głębokość do 251 m, przez Iseo przepływa rzeka Oglio}
\Clue{11}{}{symbol graficzny w notacji muzycznej służący do wydłużania wartości rytmicznych; ligatura}
\Clue{12}{}{Jubula lettii - gatunek ptaka drapieżnego z rodziny puszczykowatych (Strigidae), z podrodziny puszczyków (Striginae)}
\Clue{13}{}{ŁASZA WODNIK TOPIK}
\Clue{14}{}{przednia część budynku, od ulicy}
\Clue{15}{}{cecha czynu, zachowania itp}
\Clue{17}{}{przedmiot użytkowy, np. mebel, kuchenka}
\Clue{19}{}{wysoka wartość pieniężna}
\Clue{20}{}{Malaconotus cruentus - gatunek ptaka  z rodziny dzierzbików (Malaconotidae)}
\Clue{21}{}{kosztowności, które świecą i ładnie wyglądają, świecidełka; określenie dotyczy to najczęściej biżuterii i ozdób}
\Clue{23}{}{obwód elektryczny - układ elementów tworzących drogę zamkniętą dla prądu elektrycznego}
\Clue{25}{}{to, że coś ma charakter (znaczenie) negacji, jest negacyjne}
\Clue{28}{}{Salticidae - kosmopolityczna rodzina pająków, które nie tkają sieci, tylko polują skacząc na swoją ofiarę; w skład rodziny, mającej minimum 65 mln lat, wchodzi ok. 5600 tysięcy gatunków przynależących do blisko 600 rodzajów, co czyni je najliczniejszą rodziną wśród pająków z ok. 13 proc. wszystkich gatunków; w Polsce stwierdzono występowanie 59 gatunków}
\Clue{30}{}{łow. ogon u głuszca}
\Clue{33}{}{w Kościele katolickim: osoba przystępująca do spowiedzi}
\Clue{35}{}{skala majorowa czyli durowa}
\Clue{36}{}{wąski pasek papieru zakładany na obwolutę lub okładkę książki, płyty CD, LP, DVD, Laserdisc, powszechnie stosowany w Japonii; służy on do hasłowego zareklamowania utworu, podania ceny i nazwy wydawnictwa}
\Clue{37}{}{mieszkanka Miletu urodzona około 470 roku p.n.e., która zasłynęła swym związkiem z ateńskim mężem stanu - Peryklesem}
\Clue{39}{}{statek słowiańskich plemion}\end{PuzzleClues}\newpage\section*{Krzyżówka 4}

\noindent\begin{Puzzle}{20}{30}|*	|*	|*	|*	|*	|*	|*	|*	|*	|*	|*	|[1][S]\drarr	|f	|o	|r	|y	|ś	|*	|[2][S]\darr	|[3][S]\darr	|*	|.
|*	|*	|*	|*	|[4][S]\darr	|*	|*	|*	|[5][S]\rarr	|b	|a	|t	|t	|l	|e	|d	|r	|e	|s	|s	|*	|.
|*	|*	|*	|*	|k	|*	|[6][S]\rarr	|b	|y	|s	|t	|r	|z	|a	|k	|*	|*	|*	|t	|w	|*	|.
|*	|*	|*	|*	|o	|*	|[7][S]\rarr	|k	|a	|z	|u	|i	|s	|t	|y	|k	|a	|*	|o	|i	|*	|.
|*	|*	|*	|*	|n	|[8][S]\rarr	|p	|o	|z	|o	|s	|t	|a	|ł	|o	|ś	|ć	|*	|p	|n	|*	|.
|[9][S]\drarr	|d	|e	|l	|f	|i	|n	|[][,]{ }	|m	|a	|ł	|y	|*	|*	|*	|*	|*	|*	|n	|g	|*	|.
|l	|*	|*	|*	|e	|[10][S]\drarr	|b	|ł	|ą	|d	|[][,]{ }	|l	|e	|k	|a	|r	|s	|k	|i	|*	|*	|.
|u	|*	|[11][S]\darr	|[12][S]\darr	|s	|k	|[13][S]\darr	|*	|[14][S]\darr	|*	|*	|o	|[15][S]\rarr	|a	|d	|a	|g	|i	|o	|*	|*	|.
|l	|[16][S]\darr	|i	|t	|j	|a	|a	|*	|k	|[17][S]\darr	|*	|d	|*	|*	|*	|*	|*	|*	|w	|*	|*	|.
|o	|c	|g	|e	|a	|r	|g	|[18][S]\drarr	|o	|r	|t	|o	|p	|t	|y	|s	|t	|k	|a	|*	|*	|.
|*	|u	|ł	|o	|*	|t	|a	|p	|ł	|e	|*	|n	|[19][S]\rarr	|h	|e	|n	|g	|e	|l	|o	|*	|.
|[20][S]\drarr	|b	|a	|r	|b	|u	|*	|l	|c	|k	|*	|*	|[21][S]\drarr	|z	|a	|o	|c	|z	|n	|y	|*	|.
|b	|a	|[][,]{ }	|i	|[22][S]\darr	|z	|[23][S]\darr	|a	|h	|i	|*	|*	|m	|*	|*	|[24][S]\darr	|*	|*	|o	|*	|*	|.
|u	|[][,]{ }	|g	|a	|m	|j	|s	|g	|o	|n	|[25][S]\rarr	|f	|i	|l	|l	|e	|r	|*	|ś	|*	|[26][S]\darr	|.
|s	|l	|r	|[][,]{ }	|o	|a	|z	|a	|z	|*	|*	|[27][S]\rarr	|l	|a	|n	|d	|*	|[28][S]\darr	|ć	|*	|c	|.
|z	|i	|a	|h	|m	|*	|u	|*	|*	|*	|[29][S]\drarr	|t	|a	|u	|r	|o	|v	|o	|*	|*	|y	|.
|*	|b	|m	|e	|e	|*	|m	|*	|[30][S]\darr	|*	|s	|*	|[][,]{ }	|*	|*	|*	|*	|r	|*	|*	|f	|.
|[31][S]\drarr	|r	|o	|l	|n	|i	|k	|[][,]{ }	|r	|y	|c	|z	|a	|ł	|t	|o	|w	|y	|*	|[32][S]\darr	|e	|.
|p	|e	|f	|i	|t	|*	|a	|[33][S]\darr	|e	|[34][S]\darr	|y	|*	|n	|*	|*	|*	|*	|k	|*	|w	|r	|.
|r	|*	|o	|o	|[][,]{ }	|*	|*	|ł	|d	|s	|t	|*	|g	|*	|[35][S]\darr	|*	|*	|s	|[36][S]\darr	|ó	|k	|.
|z	|*	|n	|c	|p	|[37][S]\darr	|*	|u	|ł	|t	|*	|*	|i	|*	|t	|*	|*	|*	|p	|z	|a	|.
|e	|*	|o	|e	|ę	|z	|[38][S]\drarr	|b	|o	|o	|g	|i	|e	|[][S]-	|w	|o	|o	|g	|i	|e	|*	|.
|d	|*	|w	|n	|d	|a	|t	|e	|w	|p	|*	|*	|l	|*	|ó	|*	|*	|*	|ż	|k	|*	|.
|p	|*	|a	|t	|u	|c	|y	|k	|o	|a	|[39][S]\rarr	|a	|s	|t	|r	|o	|n	|o	|m	|*	|*	|.
|ł	|*	|*	|r	|*	|h	|p	|*	|*	|*	|*	|*	|k	|*	|*	|*	|*	|*	|a	|*	|*	|.
|a	|*	|*	|y	|*	|y	|*	|[40][S]\rarr	|d	|e	|d	|l	|a	|j	|n	|*	|*	|[41][S]\darr	|k	|*	|*	|.
|t	|*	|*	|c	|[42][S]\rarr	|ł	|ą	|c	|z	|n	|i	|k	|*	|*	|*	|*	|*	|ł	|*	|*	|*	|.
|a	|*	|*	|z	|[43][S]\rarr	|k	|o	|n	|t	|r	|a	|b	|a	|s	|*	|*	|*	|u	|[44][S]\darr	|*	|*	|.
|*	|*	|*	|n	|[45][S]\rarr	|a	|b	|o	|l	|i	|c	|j	|o	|n	|i	|s	|t	|k	|a	|*	|*	|.
|*	|*	|*	|a	|*	|*	|*	|*	|*	|*	|*	|*	|*	|*	|*	|*	|*	|*	|t	|*	|*	|.
|*	|*	|*	|*	|*	|*	|*	|*	|*	|*	|*	|*	|*	|*	|*	|*	|*	|*	|*	|*	|*	|.\end{Puzzle}

\newpage

\begin{PuzzleClues}{\textbf{Poziome}\\}\Clue{1}{}{konny ordynans oficera}
\Clue{5}{}{wojskowa kurtka, rodzaj wiatrówki}
\Clue{6}{}{ktoś, kto jest bystry, inteligentny, szybko kojarzy fakty i wyciąga poprawne wnioski}
\Clue{7}{}{metoda badawcza, metoda prowadzenia obserwacji i rozważań stosowana np. w medycynie, polegająca na omówieniu problemu przez wyliczenie znanych przypadków czy przykładów danego zjawiska}
\Clue{8}{}{to, co powstało w wyniku oddziaływania jakiegoś zjawiska, pozostało i jest trwałym dowodem na zaistnienie tego zjawiska}
\Clue{9}{}{delfin La Platy, Pontoporia blainvillei - gatunek walenia z rodziny Iniidae; żyje w Ameryce Południowej w dorzeczu La Platy}
\Clue{10}{}{nieumyślne działanie, zaniedbanie lub zaniechanie lekarza, lekarza dentysty, pielęgniarki, położnej lub osoby wykonującej inny zawód medyczny powodujące szkodę pacjenta}
\Clue{15}{}{utwór lub fragment melodii w tempie adagio}
\Clue{18}{}{osoba, która zajmuje się zaburzeniami widzenia obuocznego oraz ich leczeniem za pomocą odpowiednio dobranych ćwiczeń}
\Clue{19}{}{miasto we wsch. Holandii; węzeł kolejowy}
\Clue{20}{}{rumuński poeta i matematyk (1895-1961), profesor uniwersytetu w Bukareszcie}
\Clue{21}{}{student, który studiuje w trybie zaocznym}
\Clue{25}{}{jednostka zdawkowa na Węgrzech; 1/100 korony (od 1892), pengő (od 1926) i forinta (od 1946)}
\Clue{27}{}{powierzchnia odczytu w nośnikach optycznych, takich jak CD czy DVD}
\Clue{29}{}{miasto w azjatyckiej części Federacji Rosyjskiej na płn. od Tubolska}
\Clue{31}{}{rolnik dokonujący dostawy produktów rolnych pochodzących z własnej działalności rolniczej lub świadczący usługi rolnicze, korzystając ze zwolnienia od podatku VAT}
\Clue{38}{}{styl gry na fortepianie}
\Clue{39}{}{zajmuje się badaniem ciał niebieskich}
\Clue{40}{}{ostateczny termin wykonania czegoś}
\Clue{42}{}{element przewodu wiertniczego służący do łączenia elementów przewodu o różnych rodzajach gwintów, np. obciążników z rurami płuczkowymi, obciążników ze świdrem itp}
\Clue{43}{}{największy instrument strunowy, mający najniższą skalę tonów, wchodzi w skład orkiestry symfonicznej}
\Clue{45}{}{kobieta, która opowiada się za zniesieniem jakiegoś prawa}\end{PuzzleClues}

\begin{PuzzleClues}{\textbf{Pionowe}\\}\Clue{1}{}{Trityldon - terapsyd z grupy cytodontów, żyjący we wczesnej jurze i prawdopodobnie w późnym triasie (obok dinozaurów); zamieszkiwał południową Afrykę, ale jego skamieniałości znaleziono też na Antarktydzie}
\Clue{2}{}{cecha wyrazu, części mowy: odmiana przez stopnie}
\Clue{3}{}{cios sierpowy}
\Clue{4}{}{dawniej wyznawana religia, wyznanie}
\Clue{9}{}{smaczny owoc (jagoda) psianki lulo}
\Clue{10}{}{klasztor zakonu Kartuzów składający się z domków dla zakonników oraz kościoła klasztornego}
\Clue{11}{}{element przenoszący drgania na wkładkę gramofonową}
\Clue{12}{}{teoria budowy Układu Słonecznego, która mówi, że Słońce jest centrum Wszechświata}
\Clue{13}{}{stopień oficerski w sułtańskiej Turcji}
\Clue{14}{}{formalnie rolnicza spółdzielnia produkcyjna, rodzaj przedsiębiorstwa rolniczego charakterystycznego dla byłego ZSRR}
\Clue{16}{}{koktajl alkoholowy łączący rum, colę i opcjonalnie limonkę}
\Clue{17}{}{ryba spodoustna o długości do 16 m}
\Clue{18}{}{postępujące bardzo szybko i powodujące negatywne skutki zjawisko, na przykład epidemia choroby zakaźnej lub susza}
\Clue{20}{}{formacja roślinna, charakterystyczna dla suchych obszarów podrównikowych}
\Clue{21}{}{pozaukładowa jednostka odległości stosowana w krajach anglosaskich}
\Clue{22}{}{wektorowa wielkość fizyczna opisująca ruch ciała, zwłaszcza jego ruch obrotowy}
\Clue{23}{}{skoczna piosenka ukraińska}
\Clue{24}{}{dawna stolica Japonii, od 1868 r. zwane Tokio}
\Clue{26}{}{zdrobniale o cyfrze - inicjale}
\Clue{28}{}{Oryx - rodzaj antylop z podrodziny antylop końskich (Hippotraginae) w rodzinie krętorogich; występuje w kilku gatunkach w Afryce Wschodniej, Południowej i Północnej}
\Clue{29}{}{przedstawiciel jednego z koczowniczych ludów irańskich wywodzących się z obszarów pomiędzy Ałtajem a dolną Wołgą, zamieszkujących od schyłku VIII lub od VII wieku p.n.e. północne okolice Morza Czarnego}
\Clue{30}{}{Redłowo - nadmorska dzielnica mieszkalna Gdyni, granicząca z następującymi dzielnicami tegoż miasta: Wzgórze Św. Maksymiliana (od północy), Działki Leśne, Mały Kack (obie od zachodu), Orłowo (od południa), a od wschodu także z Morzem Bałtyckim}
\Clue{31}{}{rodzaj rozliczenia transakcji, w którym kupujący pokrywa całą należność z góry}
\Clue{32}{}{część biegowa szynowego złożona z dwu lub więcej zestawów kołowych}
\Clue{33}{}{płyta stanowiąca element złączany przy łączeniu różnorakich konstrukcji na styk}
\Clue{34}{}{(zęba) część zęba koła zębatego zawarta między powierzchnią podziałową i powierzchnią podstaw}
\Clue{35}{}{stworzenie, istota żywa, zwłaszcza człowiek albo zwierzę}
\Clue{36}{}{PIŻMOSZCZUR; północnoamerykański. gryzoń ziemnowodny o cennym futrze}
\Clue{37}{}{cienistka, Gymnocarpium - rodzaj paproci należących do rodziny rozrzutkowatych; występuje powszechnie na półkuli północnej; wykształca podłużne, pierzaste liście; gatunkiem typowym jest Gymnocarpium dryopteris}
\Clue{38}{}{przypuszczenie, informacja (cynk), że jakieś przedsięwzięcie zakończy się właśnie w dany sposób, zazwyczaj gdy mowa o zakładach}
\Clue{41}{}{OSTROŁUK}
\Clue{44}{}{w chemii: symbol astatu}\end{PuzzleClues}\newpage\section*{Krzyżówka 5}

\noindent\begin{Puzzle}{22}{32}|*	|*	|*	|*	|*	|*	|*	|*	|*	|*	|*	|*	|*	|*	|*	|*	|[1][S]\darr	|*	|*	|*	|*	|[2][S]\darr	|*	|.
|*	|*	|*	|*	|*	|*	|*	|*	|[3][S]\drarr	|f	|r	|a	|j	|e	|r	|s	|t	|w	|o	|*	|[4][S]\darr	|k	|*	|.
|*	|*	|*	|*	|*	|[5][S]\rarr	|k	|a	|s	|z	|t	|a	|n	|e	|k	|*	|r	|*	|*	|*	|h	|i	|*	|.
|*	|*	|[6][S]\darr	|[7][S]\darr	|*	|[8][S]\drarr	|j	|ę	|t	|k	|a	|*	|[9][S]\darr	|*	|*	|*	|y	|[10][S]\darr	|*	|*	|e	|l	|*	|.
|*	|*	|p	|n	|*	|o	|*	|*	|ę	|*	|*	|*	|n	|*	|*	|*	|m	|p	|*	|*	|r	|i	|*	|.
|*	|*	|o	|i	|[11][S]\darr	|m	|*	|[12][S]\rarr	|p	|i	|n	|t	|a	|*	|[13][S]\darr	|*	|o	|ł	|*	|*	|s	|m	|*	|.
|*	|*	|w	|e	|p	|a	|[14][S]\rarr	|n	|i	|e	|s	|k	|r	|ę	|p	|o	|w	|a	|n	|i	|e	|*	|*	|.
|*	|[15][S]\rarr	|i	|b	|i	|s	|[][,]{ }	|k	|a	|s	|z	|t	|a	|n	|o	|w	|a	|t	|y	|*	|y	|*	|*	|.
|*	|*	|e	|e	|e	|t	|*	|*	|*	|*	|*	|*	|*	|*	|d	|*	|n	|e	|*	|[16][S]\darr	|*	|*	|[17][S]\darr	|.
|*	|*	|ś	|z	|k	|a	|*	|[18][S]\darr	|*	|*	|[19][S]\darr	|*	|*	|*	|ł	|*	|i	|k	|*	|k	|[20][S]\darr	|*	|p	|.
|*	|*	|ć	|p	|a	|*	|*	|ś	|[21][S]\darr	|*	|c	|*	|*	|*	|o	|[22][S]\darr	|e	|*	|*	|o	|j	|*	|o	|.
|*	|*	|[][,]{ }	|i	|r	|*	|[23][S]\darr	|n	|p	|[24][S]\darr	|y	|*	|*	|*	|ż	|p	|*	|*	|[25][S]\darr	|d	|a	|*	|t	|.
|*	|*	|a	|e	|n	|*	|p	|i	|o	|p	|k	|[26][S]\darr	|[27][S]\rarr	|b	|e	|l	|w	|e	|d	|e	|r	|*	|w	|.
|*	|*	|u	|c	|i	|[28][S]\drarr	|r	|e	|w	|o	|l	|w	|e	|r	|*	|e	|*	|[29][S]\darr	|y	|k	|m	|*	|a	|.
|*	|*	|t	|z	|a	|a	|z	|ż	|o	|d	|[][,]{ }	|i	|*	|*	|*	|b	|*	|z	|s	|s	|a	|*	|l	|.
|*	|*	|o	|e	|*	|b	|e	|k	|ź	|a	|l	|ś	|[30][S]\darr	|[31][S]\rarr	|k	|i	|j	|e	|k	|*	|r	|*	|[][,]{ }	|.
|*	|*	|b	|ń	|[32][S]\drarr	|s	|z	|a	|n	|t	|u	|n	|g	|*	|*	|s	|*	|s	|*	|*	|k	|*	|o	|.
|*	|*	|i	|s	|b	|o	|i	|[][,]{ }	|i	|n	|n	|i	|e	|*	|*	|c	|*	|t	|*	|*	|o	|[33][S]\darr	|l	|.
|*	|*	|o	|t	|o	|l	|e	|c	|k	|o	|a	|e	|o	|*	|*	|y	|*	|r	|[34][S]\darr	|*	|w	|b	|b	|.
|*	|*	|g	|w	|l	|u	|r	|z	|*	|ś	|r	|w	|g	|*	|*	|t	|*	|z	|m	|*	|i	|u	|r	|.
|*	|*	|r	|o	|e	|c	|n	|a	|*	|ć	|n	|*	|r	|*	|*	|*	|*	|a	|a	|*	|c	|r	|o	|.
|*	|*	|a	|*	|s	|j	|i	|r	|*	|*	|y	|*	|a	|*	|*	|*	|*	|ł	|u	|*	|z	|a	|t	|.
|*	|*	|f	|*	|ł	|a	|k	|n	|*	|*	|*	|*	|f	|*	|*	|*	|*	|*	|r	|*	|*	|c	|o	|.
|*	|*	|i	|[35][S]\darr	|a	|*	|*	|o	|*	|*	|*	|*	|*	|[36][S]\rarr	|d	|e	|r	|b	|y	|*	|*	|t	|w	|.
|*	|*	|c	|a	|w	|[37][S]\rarr	|w	|s	|t	|ę	|ż	|n	|i	|c	|e	|*	|*	|*	|t	|*	|*	|w	|i	|.
|*	|*	|z	|p	|c	|[38][S]\rarr	|e	|k	|l	|e	|k	|t	|y	|z	|m	|*	|*	|*	|y	|*	|*	|o	|e	|.
|*	|*	|n	|a	|z	|*	|[39][S]\rarr	|r	|ó	|w	|[][,]{ }	|i	|r	|y	|g	|a	|c	|y	|j	|n	|y	|*	|c	|.
|*	|[40][S]\rarr	|a	|n	|a	|l	|i	|z	|a	|t	|o	|r	|[][,]{ }	|r	|ó	|ż	|n	|i	|c	|o	|w	|y	|*	|.
|*	|*	|*	|a	|n	|*	|*	|y	|[41][S]\rarr	|g	|w	|a	|t	|e	|m	|a	|l	|c	|z	|y	|k	|*	|*	|.
|*	|*	|*	|ż	|k	|[42][S]\rarr	|o	|d	|n	|a	|w	|i	|a	|n	|i	|e	|*	|*	|y	|*	|*	|*	|*	|.
|*	|*	|*	|*	|a	|*	|*	|ł	|*	|*	|*	|*	|[43][S]\rarr	|j	|u	|d	|a	|i	|k	|a	|*	|*	|*	|.
|*	|*	|*	|*	|*	|[44][S]\rarr	|z	|a	|d	|z	|w	|o	|n	|i	|e	|n	|i	|e	|*	|*	|*	|*	|*	|.
|[45][S]\rarr	|w	|y	|k	|u	|s	|z	|*	|*	|*	|*	|*	|*	|*	|*	|*	|*	|*	|*	|*	|*	|*	|*	|.\end{Puzzle}

\newpage

\begin{PuzzleClues}{\textbf{Poziome}\\}\Clue{3}{}{naiwność (cecha człowieka)}
\Clue{5}{}{mały kasztan - owoc kasztanowca (drzewa)}
\Clue{8}{}{belka podtrzymująca krokwie dachu}
\Clue{12}{}{stosowana w dawnej Francji miara objętości cieczy; około 0,9 litra}
\Clue{14}{}{niezależność, swoboda, brak ograniczeń}
\Clue{15}{}{Plegadis falcinellus - gatunek ptaka z rodziny ibisowatych (Threskiornithidae)}
\Clue{27}{}{warszawski pałac, reprezentacyjna rezydencja państwowa}
\Clue{28}{}{wielostrzałowa nieautomatyczna krótka broń powtarzalna, w której funkcję magazynka niewymiennego spełnia obrotowy bęben z kilkoma komorami nabojowymi (zwykle sześcioma)}
\Clue{31}{}{zdrobniale: kij - kawałek drewna o cylindrycznym kształcie, o różnej długości i grubości, posiadający dwa lub więcej końców (rozwidlenia), który został ułamany lub obcięty z drzewa, krzewu, trzciny lub trawy (bambus), ewentualnie wystrugany z drewna}
\Clue{32}{}{SHANDONG}
\Clue{36}{}{miasto w Anglii nad rzeką Dermrent; przemysł lotniczy, maszynowy, samochodowy, chemiczny}
\Clue{37}{}{Nemertea - typ zwierząt bezkręgowych o bardzo długim, robakokształtnym, obłym lub płaskim, nieczłonowanym ciele, w większości wodnych, przeważnie morskich; charakterystyczną cechą (synapomorficzną) wstężnic jest obecność długiego, wysuwanego, chwytnego ryjka (proboscis), oddzielonego od jelita, uzbrojonego lub nie}
\Clue{38}{}{nurt polegający na twórczości kompilacyjnej, łączącej różne elementy i treści z różnych stylów, epok i kierunków artystycznych, nieprowadzący do nowej syntezy, nieoryginalny}
\Clue{39}{}{ręcznie lub mechanicznie wykonane podłużne zagłębienie w ziemi służące do regulowania gospodarki wodnej upraw rolnych}
\Clue{40}{}{analogowy komputer do rozwiązywania równań różniczkowych}
\Clue{41}{}{mieszkaniec Gwatemali, człowiek pochodzenia gwatemalskiego}
\Clue{42}{}{renowacja, odświeżenie czegoś już istniejącego}
\Clue{43}{}{dokumenty, druki, przedmioty dotyczące Żydów, ich religii i kultury}
\Clue{44}{}{dźwięczny i metaliczny odgłos}
\Clue{45}{}{forma architektoniczna wzorowana na budownictwie Bliskiego Wschodu (architektura islamu) stanowiąca wystający z lica elewacji fragment budynku poszerzający przylegające wnętrze, nadwieszony powyżej pierwszego piętra na wysokości jednej lub kilku kondygnacji}\end{PuzzleClues}

\begin{PuzzleClues}{\textbf{Pionowe}\\}\Clue{1}{}{nadawanie żaglowi, kadłubowi jachtu lub linie prawidłowego kształtu czy długości w celu zapewnienia najkorzystniejszych warunków żeglugi}
\Clue{2}{}{dwustronna tkanina dekoracyjna na ścianę lub podłogę, wełniana, której wzór tworzy wątek pokrywający osnowę}
\Clue{3}{}{powolny, miarowy chód zwierząt, zwłaszcza koni}
\Clue{4}{}{amerykański pisarz i publicysta, ur. 1914r, „Hiroszima”, „Sprzysiężenie”}
\Clue{6}{}{typ powieści, której fabuła oparta jest na faktach z życia autora, często przeplatających się z wydarzeniami fikcyjnymi}
\Clue{7}{}{stan, sytuacja, zdarzenie, położenie grożące czymś złym, zagrażające komuś}
\Clue{8}{}{OKRASA}
\Clue{9}{}{miasto w Japonii (środkowe Honsiu), ośrodek administracyjny prefektury Nara w pobliżu Osaki, dawna stolica cesarstwa}
\Clue{10}{}{kryształek lodu lub śniegu, mała drobinka zamarzniętej wody}
\Clue{11}{}{pomieszczenie służące do wypieku pieczywa}
\Clue{13}{}{powód czegoś, czynnik coś wywołujący, będący czegoś przyczyną}
\Clue{16}{}{zbiór przepisów prawnych}
\Clue{17}{}{kaszalot, potwal, Physeter macrocephalus - gatunek walenia z rodziny kaszalotowatych, największy przedstawiciel podrzędu zębowców; występuje we wszystkich oceanach}
\Clue{18}{}{Montifringilla adamsi - gatunek ptaków z rodziny wróblowatych (Passeridae) występujący w Eurazji}
\Clue{19}{}{19 lat, po upływie których te same fazy Księżyca odpowiadają tym samym dniom słonecznej rachuby czasu}
\Clue{20}{}{bywalec jarmarków}
\Clue{21}{}{człowiek zajmujący się zawodowo produkcją powozów}
\Clue{22}{}{głosowanie członków społeczności na określonym terytorium (np. kraju, województwa, gminy) w różnych sprawach mających związek z ich miejscem zamieszkania}
\Clue{23}{}{motyl z rodziny przeziernikowatych, podobny do osy, o czarnym odwłoku w żółte lub czerwone pierścienie}
\Clue{24}{}{cecha przedmiotu lub osoby wskazująca na możliwość ich kształtowania przez czynniki zewnętrzne lub na skłonność do ulegania ich wpływowi}
\Clue{25}{}{twardy dysk, pomocnicza pamięć w komputerze}
\Clue{26}{}{wieś w Polsce położona w województwie mazowieckim, w powiecie siedleckim, w gminie Wiśniew}
\Clue{28}{}{rozgrzeszenie, uwolnienie od kar kościelnych}
\Clue{29}{}{poprzeczna podpórka skrzydeł w samolocie}
\Clue{30}{}{nauczyciel uczący w szkole geografii}
\Clue{32}{}{mieszkanka Bolesławca}
\Clue{33}{}{grupa zachowujących się niekulturalnie, chamsko, prostacko ludzi}
\Clue{34}{}{mieszkaniec Mauritiusa, człowiek pochodzenia maurytyjskiego}
\Clue{35}{}{wynagrodzenie, renta lub inna forma dochodu dla monarchów i ich rodzin, wysokich urzędników państwowych, byłych głów państw itp}\end{PuzzleClues}\newpage\section*{Krzyżówka 6}

\noindent\begin{Puzzle}{18}{29}|*	|*	|*	|*	|*	|*	|*	|*	|*	|*	|*	|*	|*	|*	|[1][S]\darr	|[2][S]\darr	|*	|*	|*	|.
|*	|*	|*	|*	|[3][S]\drarr	|w	|a	|n	|i	|e	|n	|k	|a	|*	|g	|s	|*	|*	|[4][S]\darr	|.
|*	|*	|*	|[5][S]\rarr	|m	|o	|t	|o	|w	|ą	|ż	|*	|*	|[6][S]\darr	|e	|k	|[7][S]\darr	|[8][S]\darr	|l	|.
|*	|*	|[9][S]\rarr	|j	|a	|m	|r	|a	|j	|e	|*	|*	|*	|k	|t	|r	|g	|w	|e	|.
|[10][S]\drarr	|ż	|u	|r	|n	|a	|l	|*	|*	|*	|*	|*	|*	|u	|t	|y	|ł	|e	|w	|.
|r	|[11][S]\rarr	|h	|e	|n	|r	|y	|k	|[][,]{ }	|i	|v	|[][,]{ }	|p	|r	|o	|b	|u	|s	|*	|.
|y	|[12][S]\darr	|*	|*	|*	|*	|*	|*	|*	|*	|*	|*	|*	|i	|*	|a	|c	|z	|[13][S]\darr	|.
|t	|n	|*	|[14][S]\darr	|[15][S]\rarr	|k	|r	|e	|t	|s	|c	|h	|m	|e	|r	|*	|h	|[][,]{ }	|k	|.
|e	|i	|[16][S]\rarr	|c	|y	|l	|i	|n	|d	|e	|r	|*	|*	|r	|[17][S]\darr	|*	|ó	|m	|o	|.
|l	|e	|[18][S]\rarr	|z	|a	|k	|ł	|a	|d	|z	|i	|n	|y	|*	|s	|[19][S]\darr	|w	|o	|r	|.
|*	|s	|[20][S]\darr	|a	|*	|*	|*	|*	|[21][S]\rarr	|n	|a	|r	|y	|b	|e	|k	|*	|r	|d	|.
|*	|y	|k	|p	|*	|*	|*	|*	|*	|*	|[22][S]\drarr	|c	|f	|*	|n	|a	|*	|s	|o	|.
|[23][S]\drarr	|m	|u	|l	|t	|y	|p	|l	|i	|k	|a	|c	|j	|a	|*	|m	|[24][S]\darr	|k	|n	|.
|a	|p	|k	|a	|*	|*	|*	|*	|[25][S]\rarr	|ł	|u	|g	|o	|w	|n	|i	|c	|a	|*	|.
|k	|a	|l	|[][,]{ }	|*	|*	|*	|*	|*	|*	|d	|[26][S]\darr	|*	|*	|*	|e	|z	|*	|*	|.
|c	|t	|i	|z	|*	|*	|[27][S]\rarr	|c	|a	|r	|y	|z	|m	|*	|*	|n	|e	|[28][S]\darr	|*	|.
|e	|y	|k	|i	|[29][S]\rarr	|k	|o	|p	|i	|s	|t	|a	|*	|*	|*	|i	|r	|k	|*	|.
|n	|c	|[][,]{ }	|e	|[30][S]\drarr	|e	|s	|k	|i	|m	|o	|s	|k	|a	|*	|u	|w	|a	|*	|.
|t	|z	|c	|l	|s	|*	|[31][S]\rarr	|m	|o	|d	|r	|z	|e	|w	|*	|s	|o	|r	|*	|.
|u	|n	|h	|o	|p	|*	|[32][S]\drarr	|b	|o	|y	|*	|c	|*	|*	|*	|z	|n	|t	|*	|.
|a	|o	|i	|n	|a	|*	|l	|[33][S]\drarr	|ś	|l	|i	|z	|g	|*	|[34][S]\darr	|k	|e	|a	|*	|.
|c	|ś	|l	|a	|ł	|*	|u	|h	|*	|*	|*	|y	|*	|*	|b	|a	|[][,]{ }	|g	|*	|.
|j	|ć	|i	|w	|a	|*	|z	|u	|[35][S]\rarr	|p	|o	|t	|a	|l	|a	|*	|m	|i	|*	|.
|a	|*	|j	|a	|*	|*	|*	|n	|*	|*	|*	|n	|[36][S]\darr	|*	|r	|[37][S]\darr	|i	|ń	|*	|.
|*	|*	|s	|*	|[38][S]\rarr	|m	|e	|t	|y	|l	|d	|o	|p	|a	|*	|d	|ę	|c	|*	|.
|*	|[39][S]\rarr	|k	|l	|u	|c	|z	|*	|*	|*	|*	|ś	|a	|[40][S]\rarr	|m	|y	|s	|z	|*	|.
|*	|[41][S]\rarr	|i	|l	|u	|m	|i	|n	|a	|t	|*	|ć	|d	|*	|*	|n	|o	|y	|*	|.
|*	|*	|*	|*	|*	|*	|*	|*	|*	|*	|*	|*	|ó	|*	|*	|i	|*	|k	|*	|.
|*	|*	|*	|*	|*	|*	|*	|*	|*	|[42][S]\rarr	|s	|i	|ł	|a	|*	|a	|*	|*	|*	|.
|*	|*	|*	|*	|*	|*	|*	|*	|*	|*	|*	|*	|*	|*	|*	|*	|*	|*	|*	|.\end{Puzzle}

\newpage

\begin{PuzzleClues}{\textbf{Poziome}\\}\Clue{3}{}{zawartość wanienki, laboratoryjnego lub kuchennego naczynia, które jest nieduże i dość płytkie}
\Clue{5}{}{sieć służąca do połowu węgorzy}
\Clue{9}{}{jamrajokształtne, Peramelemorphia - rząd torbaczy obejmujący niewielkie ssaki o wydłużonym pysku, długim ogonie oraz ze zrośniętym 2. i 3. palcem stopy}
\Clue{10}{}{czasopismo poświęcone modzie, obficie ilustrowane modelami wzorów}
\Clue{11}{}{książę wrocławski w latach 1270-1290, książę krakowski w latach 1288-1290}
\Clue{15}{}{twórczość, poglądy i dorobek naukowy Ernsta Kretschmera, niemieckiego lekarza psychiatry i twórcy teorii konstytucjonalnej}
\Clue{16}{}{sztywny męski kapelusz jedwabny z wysoką główką w kształcie walca}
\Clue{18}{}{obyczaj, związany z rozpoczęciem budowy nowego domu, wypełniany, aby zapewnić domostwu i rodzinie powodzenie i spokój}
\Clue{21}{}{młode, małe ryby (do ukończenia pierwszego roku życia), które po zużyciu zawartości woreczka żółtkowego rozpoczęły samodzielne zdobywanie pokarmu}
\Clue{22}{}{skrót/symbol franka komoryjskiego}
\Clue{23}{}{zwielokrotnienie}
\Clue{25}{}{rodzaj naczynia laboratoryjnego, które służy do ługowania}
\Clue{27}{}{monarchiczny ustrój państwa z carem na czele; też: system takich rządów}
\Clue{29}{}{osoba, która przepisywała księgi}
\Clue{30}{}{przedstawicielka grupy rdzennych ludów obszarów arktycznych i subarktycznych Grenlandii, Kanady, Alaski i Syberii}
\Clue{31}{}{długowieczne drzewo iglaste o igłach opadających na zimę i bardzo cennym drewnie}
\Clue{32}{}{pseudonim Tadeusza Kamila Marcjana Żeleńskiego - polskiego tłumacza literatury francuskiej, krytyka literackiego i teatralnego, pisarza, poety-satyryka, kronikarza, eseisty, działacza społecznego, z wykształcenia lekarza}
\Clue{33}{}{pojedynczy zjazd w saneczkarstwie sportowym, wykonany na specjalnie przygotowanym torze}
\Clue{35}{}{siedziba Dalajlamy w Tybecie}
\Clue{38}{}{lek hipotensyjny, który wskutek podobieństwa do dihydroksyfenyloalaniny (DOPA) blokuje powstawanie dopaminy, która jest prekursorem noradrenaliny}
\Clue{39}{}{charakterystyczna formacja tworzona podczas wędrówki przez niektóre migrujące ptaki}
\Clue{40}{}{wszystkożerny, szkodliwy gryzoń długości do 10 cm}
\Clue{41}{}{laminat świecący stosowany do wystroju wnętrz}
\Clue{42}{}{przemoc, sięganie po siłę mięśni lub broni, uciekanie się do nacisku fizycznego}\end{PuzzleClues}

\begin{PuzzleClues}{\textbf{Pionowe}\\}\Clue{1}{}{grupa społeczna izolująca się od otoczenia, posiadająca własne prawa i zasady, według których funkcjonuje}
\Clue{2}{}{dawniej: osoba, która przepisywała teksty, nie tylko zawodowo (lub w związku z powołaniem) oraz spisywała teksty np. ze słuchu, pod dyktando}
\Clue{3}{}{Heinrich (1871-1956), pisarz niemiecki, antyfaszysta; „Profesor Unrat”, „Cesarstwo”, nowele, eseje}
\Clue{4}{}{jednostka walutowa w Bułgarii}
\Clue{6}{}{pracownik firmy kurierskiej, dostarczający przesyłki}
\Clue{7}{}{wieś w Polsce położona w województwie łódzkim, w powiecie skierniewickim, w gminie Głuchów}
\Clue{8}{}{Saduria entomon - drapieżny morski gatunek skorupiaka z rzędu równonogów (Isopoda), dochodzący do 8 cm długości; pancerz koloru żółto-szarego składa się ze względnie małej głowy, 7 segmentów tułowia (z 7 parami odnóży) oraz 4 segmentów odwłoka, zakończonego charakterystyczną trójkątną (klinowatą) płytką ogonową}
\Clue{10}{}{duża wieś borowiacka w Polsce położona w województwie pomorskim, w powiecie chojnickim, w gminie Czersk}
\Clue{12}{}{to, że coś jest odbierane jako niesympatyczne, niemiłe, uciążliwe}
\Clue{13}{}{system posterunków granicznych}
\Clue{14}{}{Butorides striata - gatunek ptaka z rodziny czaplowatych (Ardeidae)}
\Clue{17}{}{marzenie senne - to, co się człowiekowi czasem śni, gdy śpi; serie obrazów, dźwięków, emocji, myśli i innych wrażeń zmysłowych pojawiające się podczas snu}
\Clue{19}{}{Histrionicus histrionicus - gatunek średniego ptaka wodnego z rodziny kaczkowatych (Anatidae), zamieszkujący Syberię na wschód od Bajkału, Aleuty, Alaskę, północną i środkową Kanadę, Grenlandię i Islandię}
\Clue{20}{}{Geum chiloense - gatunek rośliny z rodziny różowatych}
\Clue{22}{}{członek sądu kościelnego (może być świecki) przygotowujący materiał procesowy}
\Clue{23}{}{w psychologii proces poznawczy akcentowania różnic pomiędzy członkami grupy własnej a członkami grupy obcej przy jednoczesnej tendencji do ujednolicania grupy obcej, co prowadzi do powstawania stereotypów}
\Clue{24}{}{mięso, które ma więcej tkanki tłuszczowej i mioglobuliny (białka biorącego udział w magazynowaniu tlenu), niż mięso białe}
\Clue{26}{}{cecha czegoś, co przynosi zaszczyt}
\Clue{28}{}{mieszkaniec Kartaginy}
\Clue{30}{}{część powierzchni pnia drzewa (głównie sosny) poddawana żywicowaniu}
\Clue{32}{}{dużo miejsca, przestrzeni, przestronne miejsce}
\Clue{33}{}{angielski malarz i grafik (1817-1910), obrazy o tematyce religijnej i literackiej}
\Clue{34}{}{pierwiastek z grupy II głównej układu okresowego pierwiastków, z okresu szóstego}
\Clue{36}{}{świat, Ziemia}
\Clue{37}{}{Cucurbita - rodzaj roślin jednorocznych z rodziny dyniowatych obejmujący około 20 gatunków}\end{PuzzleClues}\newpage\section*{Krzyżówka 7}

\noindent\begin{Puzzle}{21}{21}|*	|*	|*	|*	|[1][S]\drarr	|o	|r	|g	|a	|n	|[][,]{ }	|r	|e	|n	|t	|o	|w	|y	|*	|*	|*	|[2][S]\darr	|.
|*	|*	|*	|[3][S]\drarr	|c	|y	|n	|k	|*	|*	|*	|*	|*	|*	|*	|*	|*	|*	|*	|*	|[4][S]\darr	|t	|.
|[5][S]\drarr	|r	|a	|s	|o	|w	|o	|ś	|ć	|*	|*	|*	|*	|*	|*	|*	|*	|*	|*	|[6][S]\darr	|p	|a	|.
|k	|*	|*	|c	|r	|[7][S]\darr	|*	|[8][S]\drarr	|m	|a	|m	|m	|o	|l	|o	|g	|i	|a	|*	|d	|e	|b	|.
|r	|*	|[9][S]\darr	|h	|r	|o	|[10][S]\drarr	|s	|n	|a	|j	|p	|e	|r	|k	|a	|*	|*	|[11][S]\darr	|i	|r	|u	|.
|ą	|[12][S]\drarr	|a	|w	|i	|c	|e	|n	|i	|a	|*	|[13][S]\darr	|*	|[14][S]\darr	|[15][S]\darr	|*	|[16][S]\darr	|[17][S]\darr	|t	|a	|g	|n	|.
|ż	|r	|l	|y	|b	|z	|d	|i	|*	|*	|*	|w	|[18][S]\darr	|r	|s	|[19][S]\drarr	|k	|l	|a	|p	|a	|*	|.
|e	|o	|d	|z	|*	|a	|y	|f	|*	|*	|*	|k	|l	|a	|z	|a	|r	|e	|r	|s	|m	|*	|.
|n	|ś	|r	|*	|[20][S]\darr	|r	|t	|f	|*	|*	|*	|ł	|e	|k	|t	|t	|z	|g	|c	|y	|i	|[21][S]\darr	|.
|i	|l	|i	|*	|y	|*	|o	|e	|*	|*	|*	|a	|w	|i	|u	|a	|t	|i	|z	|d	|n	|p	|.
|e	|i	|n	|[22][S]\drarr	|b	|a	|r	|r	|i	|g	|u	|d	|o	|*	|r	|k	|a	|t	|k	|y	|*	|ł	|.
|[][,]{ }	|n	|*	|s	|*	|[23][S]\darr	|k	|*	|*	|*	|*	|*	|*	|*	|w	|[][,]{ }	|*	|k	|a	|*	|[24][S]\darr	|o	|.
|m	|a	|*	|z	|[25][S]\rarr	|s	|a	|l	|a	|[][,]{ }	|p	|l	|e	|n	|a	|r	|n	|a	|*	|*	|s	|t	|.
|ó	|[][,]{ }	|*	|y	|*	|u	|*	|*	|*	|*	|[26][S]\rarr	|b	|e	|z	|ł	|a	|d	|*	|*	|*	|e	|e	|.
|z	|u	|*	|b	|[27][S]\rarr	|p	|r	|z	|y	|d	|a	|w	|k	|a	|*	|k	|*	|*	|*	|*	|n	|k	|.
|g	|p	|*	|*	|*	|o	|*	|[28][S]\rarr	|n	|e	|u	|t	|r	|a	|l	|i	|z	|a	|c	|j	|a	|*	|.
|o	|r	|*	|[29][S]\rarr	|k	|r	|u	|p	|i	|e	|c	|*	|*	|[30][S]\rarr	|b	|e	|n	|n	|e	|t	|t	|*	|.
|w	|a	|[31][S]\rarr	|h	|u	|t	|y	|r	|a	|*	|*	|*	|[32][S]\rarr	|z	|a	|t	|o	|p	|e	|k	|*	|*	|.
|e	|w	|*	|*	|*	|*	|[33][S]\rarr	|m	|a	|r	|t	|w	|a	|[][,]{ }	|w	|o	|d	|a	|*	|*	|*	|*	|.
|*	|n	|*	|*	|*	|*	|*	|[34][S]\rarr	|t	|o	|m	|a	|s	|z	|e	|w	|s	|k	|i	|*	|*	|*	|.
|*	|a	|*	|[35][S]\rarr	|r	|a	|d	|c	|a	|[][,]{ }	|p	|r	|a	|w	|n	|y	|*	|*	|*	|*	|*	|*	|.
|*	|*	|[36][S]\rarr	|j	|e	|r	|a	|[][,]{ }	|w	|i	|ę	|k	|s	|z	|a	|*	|*	|*	|*	|*	|*	|*	|.\end{Puzzle}

\newpage

\begin{PuzzleClues}{\textbf{Poziome}\\}\Clue{1}{}{instytucja odpowiadająca za wypłatę emerytur, rent i innych świadczeń oraz pobieranie składek od ubezpieczonych osób}
\Clue{3}{}{tajna informacja, ostrzeżenie, wskazówka}
\Clue{5}{}{posługiwanie się rasą jako kategorią klasyfikującą i wartościującą ludzi}
\Clue{8}{}{TERIOLOGIA; dział zoologii zajmujący się badaniem ssaków}
\Clue{10}{}{karabin przeznaczony dla strzelca wyborowego w celu prowadzenia ognia pojedynczego na dużą odległość, z dużą precyzją, używany do likwidowania pojedynczych, istotnych dla przeciwnika celów (głównie osób), w odległości do kilkuset metrów}
\Clue{12}{}{ROZCIĘŻA krzew morskich wybrzeży strefy międzyzwrotnikowej}
\Clue{19}{}{pokrywa służąca do zamykania otworów w instrumencie muzycznym}
\Clue{22}{}{wełniak szary i największa z amerykańskich kapucynek, owocożerna}
\Clue{25}{}{sala do obrad plenarnych odbywających się w gmachu ważnej instytucji}
\Clue{26}{}{zamęt, zamieszanie z jakiegoś powodu; bałagan}
\Clue{27}{}{część zdania określająca rzeczownik lub zaimek rzeczowny}
\Clue{28}{}{likwidacja jakiegoś czynnika, którego wpływ oceniany jest jako negatywny, szkodliwy}
\Clue{29}{}{DZIARNINA; skrystalizowany miód pszczeli}
\Clue{30}{}{James, dziennikarz amerykański, inicjator międzynarodowych zawodników balonowych}
\Clue{31}{}{węgierski lekarz weterynarii (1860-1934); profesor Wyższej Szkoły Weterynaryjnej w Budapeszcie}
\Clue{32}{}{lekkoatleta czechosłowacki, długodystansowiec, zwany 'czeską lokomotywą' czterokrotny mistrz olimpijski z Londynu i Helsinek, 18-krotny rekordzista świata}
\Clue{33}{}{akwen, w którym fale wewnętrzne uniemożliwiają ruch jednostkom pływającym}
\Clue{34}{}{muzykolog i wydawca ur. w 1921 r., profesor A. M. w Krakowie}
\Clue{35}{}{prawnik świadczący pomoc prawną podmiotom gospodarczym, jednostkom organizacyjnym oraz osobom fizycznym, która polega przede wszystkim na udzielaniu porad prawnych, sporządzaniu opinii prawnych, opracowywaniu projektów aktów prawnych oraz występowaniu przed sądami i urzędami}
\Clue{36}{}{Jaera albifrons - gatunek skorupiaka z rzędu równonogów}\end{PuzzleClues}

\begin{PuzzleClues}{\textbf{Pionowe}\\}\Clue{1}{}{jezioro w zachodniej Irlandii, powierzchnia 176 km2, głębokość do 46 m}
\Clue{2}{}{dawniej; koń stepowy}
\Clue{3}{}{kanton w środkowej Szwajcarii, obszar 908 km2}
\Clue{4}{}{potocznie: papier pergaminowy}
\Clue{5}{}{przepływ krwi przez naczynia krwionośne mózgowia, zapewniający czynność ośrodkowego układu nerwowego}
\Clue{6}{}{Diapsida - grupa owodniowców z gromady zauropsydów (Sauropsida), w których czaszce rozwinęły się dwie pary otworów skroniowych (po dwa - górny i dolny - za każdym okiem), przez które przewleczone są silne mięśnie szczęk; dzięki temu czaszka stała się lżejsza a szczęki bardziej ruchliwe}
\Clue{7}{}{ozdobny krzew lub drzewko z Azji Wsch. i Ameryki Północnej, liście owłosione, kwiaty żółte}
\Clue{8}{}{program komputerowy lub urządzenie, którego zadaniem jest przechwytywanie i ewentualnie analizowanie danych przepływających w sieci}
\Clue{9}{}{Edwin, ur. w 1930r. astronauta amerykański, uczestniczył w pierwszej wyprawie załogowej na Księżyc}
\Clue{10}{}{wydawczyni}
\Clue{11}{}{niewielkich rozmiarów tarcza - obracająca się cześć telefonu, służąca do wybierania numeru}
\Clue{12}{}{roślina użytkowa nieeksploatowana ze stanowisk naturalnych, lecz z upraw stworzonych i pielęgnowanych przez człowieka}
\Clue{13}{}{pieniądze, które zostały złożone w banku, np. oszczędności}
\Clue{14}{}{drzewołazy}
\Clue{15}{}{poziomy wał z nawiniętą liną lub łańcuchem sterowym, połączony z kołem sterowym}
\Clue{16}{}{odrobina, kapka, ociupina, mała ilość}
\Clue{17}{}{dokument potwierdzający to, że ktoś ma jakiś status, przynależy gdzieś}
\Clue{18}{}{lewa strona}
\Clue{19}{}{manewr zbliżania samolotu do obiektu przeciwnika w celu zajęcia pozycji i wystrzelenia rakiety lub zrzucenia bomby}
\Clue{20}{}{jednostka informacji w systemie dwójkowym, oznaczająca 2\textasciicircum80 bajtów}
\Clue{21}{}{przeszkoda, przez którą przeskakują lekkoatleci podczas biegów}
\Clue{22}{}{w budownictwie - pionowy kanał w konstrukcji budynku}
\Clue{23}{}{część układu napędowego roweru - ułożyskowany wałek, wkręcany do tulei, zwanej mufą suportu, która znajduje się w miejscu łączenia rury podsiodłowej i rury dolnej ramy}
\Clue{24}{}{budynek rządowy, w którym odbywają się posiedzenia senatu}\end{PuzzleClues}\newpage\section*{Krzyżówka 8}

\noindent\begin{Puzzle}{22}{22}|*	|*	|*	|*	|*	|*	|*	|*	|*	|*	|*	|*	|*	|*	|*	|*	|[1][S]\drarr	|g	|o	|ł	|ą	|b	|*	|.
|*	|*	|[2][S]\darr	|*	|[3][S]\darr	|[4][S]\darr	|*	|*	|[5][S]\rarr	|s	|i	|e	|d	|m	|i	|o	|b	|ó	|j	|*	|*	|*	|*	|.
|*	|*	|p	|*	|g	|ś	|*	|*	|*	|[6][S]\rarr	|k	|l	|o	|n	|i	|n	|a	|*	|*	|*	|*	|*	|*	|.
|*	|[7][S]\darr	|o	|*	|o	|w	|*	|*	|[8][S]\darr	|[9][S]\drarr	|g	|a	|r	|*	|[10][S]\rarr	|t	|r	|u	|c	|h	|ł	|o	|*	|.
|*	|n	|w	|*	|m	|i	|*	|*	|s	|m	|*	|[11][S]\rarr	|t	|r	|z	|c	|i	|n	|i	|a	|k	|*	|*	|.
|*	|o	|i	|*	|ó	|ę	|*	|[12][S]\darr	|e	|i	|[13][S]\darr	|*	|*	|[14][S]\rarr	|l	|a	|b	|i	|r	|y	|n	|t	|*	|.
|*	|c	|e	|*	|ł	|c	|*	|d	|r	|g	|p	|[15][S]\drarr	|a	|n	|i	|m	|a	|l	|n	|o	|ś	|ć	|*	|.
|*	|e	|ś	|[16][S]\rarr	|k	|o	|ł	|o	|w	|r	|o	|t	|e	|k	|*	|[17][S]\rarr	|l	|o	|t	|i	|*	|*	|*	|.
|*	|k	|ć	|*	|a	|n	|[18][S]\drarr	|d	|i	|a	|l	|e	|k	|t	|y	|k	|*	|*	|[19][S]\darr	|*	|*	|[20][S]\darr	|*	|.
|*	|[][,]{ }	|[][,]{ }	|*	|*	|e	|m	|a	|s	|c	|o	|l	|*	|*	|*	|[21][S]\darr	|*	|[22][S]\darr	|r	|[23][S]\darr	|[24][S]\darr	|t	|*	|.
|*	|b	|e	|*	|*	|*	|o	|t	|*	|j	|*	|e	|*	|*	|*	|s	|*	|p	|o	|k	|a	|a	|*	|.
|[25][S]\rarr	|e	|p	|i	|g	|e	|n	|e	|z	|a	|*	|f	|[26][S]\rarr	|p	|o	|m	|p	|a	|d	|o	|u	|r	|*	|.
|*	|c	|i	|*	|*	|*	|o	|k	|*	|[][,]{ }	|*	|o	|[27][S]\darr	|*	|*	|o	|*	|p	|n	|ń	|t	|c	|*	|.
|*	|h	|s	|*	|*	|*	|*	|*	|*	|z	|*	|n	|o	|*	|[28][S]\darr	|l	|*	|a	|i	|[][,]{ }	|o	|z	|*	|.
|*	|s	|t	|[29][S]\rarr	|s	|ł	|o	|m	|i	|a	|k	|*	|b	|*	|l	|u	|*	|t	|a	|d	|m	|o	|*	|.
|*	|t	|o	|*	|*	|*	|[30][S]\rarr	|e	|k	|s	|p	|o	|r	|t	|a	|c	|j	|a	|*	|u	|a	|w	|*	|.
|*	|e	|l	|[31][S]\rarr	|f	|l	|e	|t	|n	|i	|a	|*	|a	|[32][S]\darr	|m	|h	|*	|c	|*	|n	|t	|n	|*	|.
|*	|i	|a	|[33][S]\rarr	|s	|z	|y	|p	|u	|ł	|a	|*	|z	|n	|a	|*	|*	|z	|*	|a	|*	|i	|*	|.
|*	|n	|r	|*	|*	|*	|[34][S]\rarr	|s	|a	|k	|l	|a	|*	|a	|*	|*	|*	|*	|*	|j	|*	|c	|*	|.
|*	|a	|n	|[35][S]\rarr	|m	|i	|a	|s	|t	|o	|[][,]{ }	|o	|t	|w	|a	|r	|t	|e	|*	|s	|*	|a	|*	|.
|*	|*	|a	|*	|*	|*	|*	|*	|[36][S]\rarr	|w	|e	|b	|m	|a	|s	|t	|e	|r	|*	|k	|*	|*	|*	|.
|*	|*	|*	|*	|[37][S]\rarr	|ł	|y	|c	|z	|a	|k	|ó	|w	|*	|*	|[38][S]\rarr	|c	|u	|k	|i	|e	|r	|*	|.
|[39][S]\rarr	|w	|e	|l	|o	|ń	|s	|k	|i	|*	|*	|*	|*	|*	|*	|*	|*	|*	|*	|*	|*	|*	|*	|.\end{Puzzle}

\newpage

\begin{PuzzleClues}{\textbf{Poziome}\\}\Clue{1}{}{ptak; poszczególne gatunki tego ptaka w taksonomii biologicznej klasyfikowane są w obrębie rodziny gołębiowatych (Columbidae), w podrodzinie gołębi (Columbinae)}
\Clue{5}{}{dyscyplina sportowa składająca się z siedmiu konkurencji}
\Clue{6}{}{klony - drzewa}
\Clue{9}{}{garnek pokaźnych rozmiarów}
\Clue{10}{}{pogardliwie o starszym człowieku}
\Clue{11}{}{ptak nawodny z rzędu wróblowatych; buduje koszyczkowe gniazdo zawieszone na trzcinach; Eurazja, płn. Afryka; chroniony}
\Clue{14}{}{gąszcz, plątanina; w znaczeniu przenośnym o skomplikowanej sytuacji, w której trudno się zorientować}
\Clue{15}{}{zwierzęcość; bycie jak zwierzę (czyli istota nieuduchowiona)}
\Clue{16}{}{walec obracający się pod wpływem ruchów umieszczonego wewnątrz stworzenia}
\Clue{17}{}{właściwie Viaud - pisarz francuski (1850-1923), powieści z życia marynarzy i rybaków; „Rybak islandzki”}
\Clue{18}{}{filozof uprawiający dialektykę}
\Clue{25}{}{teoria biologiczna, według której rozwój embrionalny przebiega z niezróżnicowanych początkowo komórek zygoty, które różnicują się dopiero na dalszych etapach wzrostu embrionu, prowadząc do utworzenia narządów i ich układów}
\Clue{26}{}{karmazynowy róż}
\Clue{29}{}{ul słomiany}
\Clue{30}{}{wyprowadzenie zwłok na miejsce, gdzie pozostają do pogrzebu}
\Clue{31}{}{prosty instrument muzyczny składający się z szeregu złączonych razem piszczałek różnej długości}
\Clue{33}{}{część organu, przez którą przechodzi wiązka nerwowa lub naczyniowa}
\Clue{34}{}{dom mieszkalny kaukaskich górali z kamienia, gliny lub drewna}
\Clue{35}{}{miasto, które na czas konfliktu zbrojnego zostało ogłoszone przez organy sprawujące nad nim władzę miastem niebronionym oraz takie miasto, które zgodnie z prawem międzynarodowym nie może zostać zbombardowane}
\Clue{36}{}{osobę zajmującą się projektowaniem, kodowaniem, szatą graficzną oraz aktualizacją witryny internetowej}
\Clue{37}{}{Łyczaków - dzielnica Lwowa}
\Clue{38}{}{słodka w smaku przyprawa, zwykle w postaci krystalizowanej sacharozy}
\Clue{39}{}{włoski malarz i rzeźbiarz (1509-66) obrazy i freski religijne}\end{PuzzleClues}

\begin{PuzzleClues}{\textbf{Pionowe}\\}\Clue{1}{}{niedźwiedź z lasów Ameryki Płd}
\Clue{2}{}{powieść skonstruowana w formie listów (niekiedy przeplatających się z fragmentami pamiętnika), wymienianych między sobą przez bohaterów}
\Clue{3}{}{mała szybka z ołowianym łączeniem; kiedyś szybki takie stosowane powszechnie w oknach, dziś - do celów dekoracyjnych (głównie w witrażach)}
\Clue{4}{}{nazwa pokarmów (głównie mięsa, jaj, chrzanu, chleba etc.) święconych w Wielką Sobotę w kościołach katolickich Polski oraz w graniczących ze Słowenią austriackich regionach Styrii, Karyntii, południowego Tyrolu oraz Bawarii}
\Clue{7}{}{Myotis bechsteinii - gatunek ssaka z rzędu nietoperzy z rodziny mroczkowatych; w Polsce jego zasięg ogranicza się do południowej części kraju, najdalej na północ sięga do Cedyńskiego Parku Krajobrazowego, Wielkopolski, okolic Tomaszowa Mazowieckiego i Polesia Lubelskiego}
\Clue{8}{}{punkt usługowy}
\Clue{9}{}{rodzaj migracji będący społecznym obciążeniem dla krajów przyjmujących imigrantów, ponieważ ponoszą one dodatkowe koszty zapewnienia tym osobom zakwaterowania, pomocy prawnej i usług służby zdrowia}
\Clue{12}{}{akcesorium w postaci np. biżuterii, szala, krawatu, będące ozdobą, dopełnieniem ubioru}
\Clue{13}{}{volkswagen z modelu Polo}
\Clue{15}{}{telefonia, podłączenie do sieci telefonicznej}
\Clue{18}{}{jezioro w USA u podnóża gór Sierra Nevada}
\Clue{19}{}{żeński organ rozmnażania się mszaków, paprotników i roślin nagozalążkowych}
\Clue{20}{}{ZŁOTOROST}
\Clue{21}{}{człowiek o śniadym lub ciemnym kolorze skóry, rozpoznawany jako nienależący do kultury białych ludzi}
\Clue{22}{}{włoskie ciastko z ciasta drożdżowego z rodzynkami i cynamonem}
\Clue{23}{}{jedna z ras konia, ukształtowana dopiero na początku XX wieku, poprzez skrzyżowanie oryginalnego ogiera noniusa z klaczami gidran; używany jako wierzchowiec i jako silny koń do lekkich zaprzęgów}
\Clue{24}{}{maszyna wykonująca cały cykl swej pracy bez udziału człowieka}
\Clue{27}{}{w matematyce - zbiór wszystkich wartości (należących do przeciwdziedziny) przyjmowanych przez funkcję dla każdego elementu danego podzbioru jej dziedziny oraz zbiór wszystkich elementów dziedziny, które są odwzorowywane na elementy danego podzbioru przeciwdziedziny}
\Clue{28}{}{duchowny buddyjski w Tybecie i Mongolii}
\Clue{32}{}{składowa część kościoła położona pomiędzy prezbiterium a kruchtą, przeznaczona dla wiernych}\end{PuzzleClues}\newpage\section*{Krzyżówka 9}

\noindent\begin{Puzzle}{24}{31}|*	|*	|*	|*	|*	|*	|*	|*	|*	|*	|*	|*	|*	|[1][S]\drarr	|m	|ó	|z	|g	|*	|[2][S]\darr	|*	|*	|[3][S]\darr	|[4][S]\darr	|*	|.
|*	|*	|*	|*	|[5][S]\rarr	|f	|o	|t	|o	|c	|h	|r	|o	|m	|y	|*	|*	|[6][S]\darr	|*	|f	|*	|[7][S]\darr	|k	|r	|[8][S]\darr	|.
|*	|*	|*	|*	|*	|*	|*	|*	|*	|*	|[9][S]\darr	|*	|*	|u	|*	|*	|*	|s	|*	|i	|*	|t	|o	|e	|c	|.
|*	|*	|*	|*	|*	|*	|*	|*	|*	|*	|p	|[10][S]\rarr	|s	|s	|a	|k	|*	|z	|[11][S]\darr	|l	|*	|o	|p	|p	|o	|.
|*	|*	|*	|*	|*	|*	|[12][S]\rarr	|k	|o	|m	|i	|n	|*	|z	|[13][S]\darr	|*	|[14][S]\darr	|k	|p	|m	|[15][S]\darr	|l	|u	|r	|l	|.
|*	|*	|*	|*	|*	|*	|*	|*	|[16][S]\drarr	|w	|e	|k	|*	|k	|e	|*	|n	|o	|r	|[][,]{ }	|s	|e	|ł	|e	|u	|.
|*	|*	|*	|[17][S]\rarr	|p	|r	|z	|y	|p	|o	|r	|a	|*	|i	|p	|*	|a	|l	|o	|m	|t	|r	|a	|z	|m	|.
|*	|[18][S]\rarr	|s	|z	|t	|y	|l	|p	|a	|*	|c	|*	|*	|e	|i	|*	|d	|a	|k	|u	|a	|a	|[][,]{ }	|e	|b	|.
|*	|*	|[19][S]\rarr	|k	|o	|s	|o	|d	|r	|z	|e	|w	|*	|t	|p	|*	|ś	|r	|u	|z	|r	|n	|p	|n	|u	|.
|*	|*	|[20][S]\drarr	|l	|o	|d	|ó	|w	|k	|a	|*	|*	|*	|e	|s	|[21][S]\darr	|w	|s	|r	|y	|z	|c	|a	|t	|s	|.
|*	|*	|a	|[22][S]\darr	|*	|*	|[23][S]\drarr	|m	|a	|s	|a	|[][,]{ }	|g	|r	|a	|w	|i	|t	|a	|c	|y	|j	|n	|a	|*	|.
|[24][S]\rarr	|p	|l	|a	|m	|i	|c	|a	|*	|*	|*	|*	|*	|*	|m	|a	|e	|w	|t	|z	|*	|a	|c	|c	|*	|.
|*	|*	|m	|n	|*	|[25][S]\darr	|i	|[26][S]\drarr	|k	|a	|c	|y	|k	|*	|m	|d	|t	|o	|o	|n	|*	|*	|e	|j	|*	|.
|*	|*	|a	|t	|[27][S]\drarr	|g	|a	|l	|o	|i	|s	|*	|*	|*	|o	|l	|l	|*	|r	|y	|[28][S]\darr	|*	|r	|a	|*	|.
|*	|[29][S]\darr	|[][,]{ }	|i	|a	|o	|ł	|i	|*	|[30][S]\drarr	|c	|u	|g	|*	|n	|i	|n	|*	|[][,]{ }	|*	|o	|*	|n	|*	|*	|.
|*	|k	|m	|g	|m	|u	|o	|t	|[31][S]\drarr	|s	|w	|i	|n	|g	|*	|w	|a	|[32][S]\rarr	|g	|i	|k	|*	|a	|[33][S]\darr	|*	|.
|*	|a	|a	|u	|b	|d	|[][,]{ }	|e	|s	|y	|*	|[34][S]\drarr	|ż	|a	|k	|o	|*	|*	|e	|*	|n	|*	|*	|h	|*	|.
|*	|s	|t	|a	|a	|a	|n	|r	|t	|g	|*	|m	|[35][S]\darr	|*	|*	|ś	|*	|*	|n	|*	|o	|*	|[36][S]\darr	|e	|*	|.
|*	|z	|e	|*	|*	|*	|i	|a	|a	|n	|[37][S]\darr	|o	|d	|[38][S]\darr	|*	|ć	|*	|*	|e	|*	|*	|[39][S]\darr	|k	|b	|*	|.
|*	|t	|r	|*	|[40][S]\drarr	|t	|e	|t	|r	|a	|s	|t	|y	|c	|h	|*	|*	|*	|r	|*	|*	|b	|l	|a	|*	|.
|*	|e	|*	|*	|s	|*	|b	|u	|a	|ł	|e	|y	|m	|z	|*	|*	|[41][S]\drarr	|g	|a	|u	|g	|u	|i	|n	|*	|.
|*	|l	|*	|[42][S]\darr	|i	|*	|i	|r	|*	|[][,]{ }	|r	|l	|i	|a	|*	|*	|c	|*	|l	|*	|*	|d	|n	|o	|*	|.
|*	|a	|[43][S]\darr	|o	|e	|*	|e	|a	|*	|c	|p	|i	|s	|s	|*	|*	|h	|*	|n	|*	|*	|y	|*	|w	|*	|.
|*	|ń	|k	|t	|s	|[44][S]\darr	|s	|[][,]{ }	|*	|y	|e	|c	|j	|z	|[45][S]\rarr	|f	|o	|r	|y	|ś	|*	|ń	|*	|i	|*	|.
|*	|s	|a	|t	|i	|t	|k	|l	|[46][S]\darr	|f	|n	|a	|a	|a	|*	|*	|m	|*	|*	|*	|*	|*	|*	|e	|*	|.
|*	|t	|n	|o	|e	|e	|i	|a	|p	|r	|t	|*	|*	|*	|*	|*	|i	|*	|*	|*	|*	|*	|*	|c	|*	|.
|*	|w	|o	|n	|n	|m	|e	|g	|e	|o	|y	|*	|*	|[47][S]\rarr	|p	|e	|c	|h	|s	|t	|e	|i	|n	|*	|*	|.
|[48][S]\rarr	|o	|p	|o	|k	|a	|*	|r	|r	|w	|n	|*	|[49][S]\rarr	|n	|i	|e	|z	|g	|o	|d	|n	|o	|ś	|ć	|*	|.
|*	|*	|a	|w	|i	|t	|*	|o	|r	|y	|*	|*	|[50][S]\rarr	|c	|h	|l	|e	|b	|o	|d	|a	|w	|c	|a	|*	|.
|*	|*	|*	|i	|*	|*	|*	|w	|e	|*	|*	|*	|*	|*	|[51][S]\rarr	|s	|k	|a	|j	|l	|a	|j	|t	|*	|*	|.
|[52][S]\rarr	|n	|i	|e	|o	|s	|t	|a	|t	|e	|c	|z	|n	|o	|ś	|ć	|*	|*	|*	|*	|*	|*	|*	|*	|*	|.
|*	|*	|*	|*	|*	|*	|*	|*	|*	|*	|*	|*	|*	|*	|*	|*	|*	|*	|*	|*	|*	|*	|*	|*	|*	|.\end{Puzzle}

\newpage

\begin{PuzzleClues}{\textbf{Poziome}\\}\Clue{1}{}{osoba o nieprzeciętnym umyśle}
\Clue{5}{}{przyrząd optyczny, którego soczewki zmieniają kolor wpływem światła}
\Clue{10}{}{ssąca głowica pogłębiarki}
\Clue{12}{}{piec kuchenny}
\Clue{16}{}{słoik, który służy do przechowywania przetworów}
\Clue{17}{}{rusztowanie, które ma zabezpieczać ściany przed zawaleniem}
\Clue{18}{}{ochronny mankiet zakładany na rękaw np. munduru}
\Clue{19}{}{zwyczajowa nazwa sosny górskiej}
\Clue{20}{}{gatunek kaczki o biało-brunatno-czarnym upierzeniu; w Polsce na przelotach, zamieszkuje obszary mórz dalekiej północy; łowna}
\Clue{23}{}{masa, która wynika z oddziaływania grawitacyjnego}
\Clue{24}{}{zakaźna choroba larw jedwabników}
\Clue{26}{}{środkowoamerykański ptak z rzędu wróblowatych}
\Clue{27}{}{matematyk francuski (1811-32); twórca nowoczesnej teorii równań algebraicznych oraz teorii gry}
\Clue{30}{}{dość intensywny ruch powietrza}
\Clue{31}{}{jeden ze stylów jazzowych o nieregularnym rytmie}
\Clue{32}{}{bom na małych jachtach śródlądowych}
\Clue{34}{}{afrykański ptak z rzędu papug o szarym upierzeniu i różowym ogonie, umie naśladować mowę - papuga popielata}
\Clue{40}{}{stofa licząca 4 wersy}
\Clue{41}{}{samodzielne muzeum malarstwa, rzeźby lub salon wystawowy połączony ze sprzedażą dzieł sztuki}
\Clue{45}{}{w dawnym wojsku: konny ordynans oficera}
\Clue{47}{}{niemiecki malarz i grafik (1881-1955) reprezentant ekspresjonizmu; kompozycje figuralne. pejzaże, portrety}
\Clue{48}{}{skała mieszana, zbudowana z organogenicznej krzemionki (opal, chalcedon, przeważnie pochodzące z krzemionkowych gąbek) i węglanu wapnia}
\Clue{49}{}{brak zgodności z czymś, sprzeczność}
\Clue{50}{}{pracodawca; osoba zatrudniająca pracownika}
\Clue{51}{}{oszklone okienko w pokładzie statku lub dachu kabiny, które przepuszcza światło z zewnątrz}
\Clue{52}{}{cecha decyzji, wyroku, który można zmienić}\end{PuzzleClues}

\begin{PuzzleClues}{\textbf{Pionowe}\\}\Clue{1}{}{żołnierz, którego podstawowym uzbrojeniem był muszkiet oraz rapier; formacje złożone z muszkieterów występowały w XVI-XVII wieku w niemal wszystkich armiach europejskich i nie tylko}
\Clue{2}{}{gatunek filmowy, w którym przeplatają się sceny mówione, śpiewane i taneczne}
\Clue{3}{}{stała, nieruchoma pancerna czasza, najczęściej wykonana jako odlew ze staliwa lub ze stalowych płyt połączonych nitami, służąca do celów obserwacyjnych lub jako stanowisko ogniowe dział lub karabinów maszynowych}
\Clue{4}{}{grupa sportowców, którzy w rozgrywkach reprezentują dany kraj lub klub}
\Clue{6}{}{opieranie się w swych sądach, decyzjach, postępowaniu na szkolnych, uproszczonych formułach}
\Clue{7}{}{zdolność do znoszenia rzeczy ocenianych jako negatywne, umiejętność lub chęć do przymykania oka na czyny, zachowania, sprawy naganne}
\Clue{8}{}{miasto w USA (Georgia) nad rzeką Chattahooche; przemysł lotniczy, samochodowy, elektroniczny, maszynowy}
\Clue{9}{}{amerykański filozof i logik (1839-1914); jeden z twórców pragmatyzmu}
\Clue{11}{}{naczelny organ prokuratury}
\Clue{13}{}{zespół organizmów żyjących na powierzchni ziarenek piasku dna zbiornika wodnego}
\Clue{14}{}{prędkość większa od prędkości światła}
\Clue{15}{}{PLUR. czyiś rodzice}
\Clue{16}{}{gruba, dłuższa kurtka wypełniona naturalnym lub syntetycznym puchem, z wyściełanym futrem kapturem}
\Clue{20}{}{podniosłe określenie szkoły wyższej, najczęściej uniwersytetu}
\Clue{21}{}{o przedmiotach, ludziach, zwierzętach, zjawiskach naturalnych - niepełność, niekompletność, choroba, także uszkodzenie}
\Clue{22}{}{miasto w płd. Gwatemali; ośrodek handlowy i turystyczny}
\Clue{23}{}{każdy naturalny obiekt fizyczny oraz układ powiązanych ze sobą obiektów lub ich struktur występujący w przestrzeni kosmicznej poza granicą atmosfery ziemskiej}
\Clue{25}{}{gatunek sera twardego, podpuszczkowego, dojrzewającego, produkowanego z mleka krowiego}
\Clue{26}{}{ogół utworów związanych tematycznie z funkcjonowaniem hitlerowskich obozów koncentracyjnych (lagrów) w czasie II wojny światowej}
\Clue{27}{}{trudne położenie, sytuacja bez wyjścia}
\Clue{28}{}{luka lub prześwit w jakiejś ciemniejszej materii, pozwalająca na spojrzenie na drugą stronę}
\Clue{29}{}{urząd lokalny w średniowiecznej Polsce}
\Clue{30}{}{sygnał, którego dziedzina i zbiór wartości są dyskretne w czasie}
\Clue{31}{}{czyjaś małżonka, żona}
\Clue{33}{}{PERSYMONA, KAKI}
\Clue{34}{}{BARCIAK}
\Clue{35}{}{rezygnacja z zajmowanego stanowiska lub urzędu}
\Clue{36}{}{porcja alkoholu, którą pije się rano, gdy ma się kaca w nadziei, że przejdzie}
\Clue{37}{}{minerał zaliczany do krzemianów warstwowych}
\Clue{38}{}{rozpraszacz lub odbłyśnik umieszczony pod lampą}
\Clue{39}{}{porcja budyniu, tj. proszku, z którego po dodaniu mleka otrzymuje się budyń}
\Clue{40}{}{wielkopolski ludowy instrument muzyczny}
\Clue{41}{}{chomik}
\Clue{42}{}{okres panowania Ottonów w Świętym Cesarstwie Rzymskim}
\Clue{43}{}{egipska lub etruska urna do przechowywania wnętrzności zmarłego}
\Clue{44}{}{zagadnienie omawiane w szkole, zazwyczaj w ramach jednej jednostki lekcyjnej}
\Clue{46}{}{architekt francuski (1874-1954), zwany też ojcem żelbetu - ratusz w Hawrze}\end{PuzzleClues}\newpage\section*{Krzyżówka 10}

\noindent\begin{Puzzle}{24}{31}|*	|*	|*	|[1][S]\darr	|*	|*	|*	|*	|*	|*	|*	|*	|*	|*	|*	|*	|*	|*	|[2][S]\darr	|*	|*	|*	|*	|*	|[3][S]\darr	|.
|*	|[4][S]\rarr	|p	|i	|e	|ś	|n	|i	|a	|r	|k	|a	|*	|*	|[5][S]\darr	|*	|*	|*	|b	|[6][S]\darr	|*	|[7][S]\darr	|*	|*	|w	|.
|[8][S]\drarr	|j	|u	|n	|i	|o	|r	|k	|a	|[][,]{ }	|m	|ł	|o	|d	|s	|z	|a	|*	|l	|p	|*	|h	|[9][S]\darr	|*	|a	|.
|o	|*	|[10][S]\darr	|t	|*	|[11][S]\drarr	|p	|r	|u	|s	|k	|i	|[][,]{ }	|d	|r	|y	|l	|*	|e	|r	|*	|a	|n	|*	|r	|.
|g	|*	|l	|e	|[12][S]\rarr	|s	|p	|ó	|ł	|g	|ł	|o	|s	|k	|a	|[][,]{ }	|c	|i	|s	|z	|ą	|c	|a	|*	|i	|.
|r	|*	|e	|r	|*	|ł	|*	|*	|*	|[13][S]\darr	|*	|[14][S]\rarr	|o	|d	|l	|e	|w	|*	|b	|e	|*	|k	|l	|*	|a	|.
|ó	|[15][S]\darr	|m	|e	|[16][S]\darr	|o	|*	|[17][S]\rarr	|m	|o	|u	|n	|d	|o	|u	|*	|*	|[18][S]\darr	|o	|g	|*	|n	|e	|*	|n	|.
|d	|s	|n	|s	|c	|w	|*	|*	|*	|b	|*	|*	|[19][S]\darr	|*	|c	|*	|*	|s	|k	|i	|[20][S]\darr	|e	|ż	|*	|t	|.
|e	|t	|i	|*	|i	|i	|*	|*	|*	|n	|*	|[21][S]\darr	|r	|*	|h	|*	|*	|u	|*	|ę	|p	|y	|n	|*	|y	|.
|k	|y	|s	|*	|s	|k	|[22][S]\darr	|*	|[23][S]\darr	|i	|*	|i	|e	|*	|*	|*	|*	|b	|*	|c	|a	|*	|o	|*	|w	|.
|[][,]{ }	|l	|k	|*	|o	|*	|r	|*	|c	|ż	|*	|r	|f	|*	|*	|[24][S]\darr	|*	|s	|[25][S]\darr	|i	|s	|[26][S]\darr	|ś	|*	|n	|.
|l	|[][,]{ }	|a	|[27][S]\darr	|l	|*	|a	|*	|y	|o	|[28][S]\darr	|a	|o	|*	|*	|k	|*	|k	|k	|e	|k	|p	|ć	|[29][S]\darr	|o	|.
|e	|z	|t	|t	|i	|*	|p	|*	|r	|w	|h	|n	|r	|*	|*	|a	|*	|r	|o	|*	|o	|u	|[][,]{ }	|i	|ś	|.
|t	|a	|a	|e	|s	|*	|t	|*	|a	|a	|e	|i	|m	|*	|*	|s	|*	|y	|p	|[30][S]\darr	|w	|d	|p	|m	|ć	|.
|n	|k	|[][,]{ }	|o	|t	|*	|u	|[31][S]\darr	|n	|t	|l	|s	|a	|*	|*	|a	|[32][S]\rarr	|p	|a	|s	|i	|e	|r	|b	|*	|.
|i	|o	|b	|r	|[][,]{ }	|*	|l	|n	|e	|e	|i	|t	|c	|*	|*	|c	|*	|c	|n	|y	|k	|ł	|z	|i	|*	|.
|*	|p	|e	|i	|g	|*	|a	|i	|c	|*	|n	|a	|j	|*	|*	|j	|*	|j	|i	|t	|[][,]{ }	|e	|y	|r	|*	|.
|*	|i	|r	|a	|ę	|[33][S]\drarr	|r	|e	|z	|y	|g	|n	|a	|c	|j	|a	|*	|a	|c	|n	|s	|c	|w	|[][,]{ }	|*	|.
|*	|a	|n	|[][,]{ }	|s	|d	|z	|k	|k	|*	|*	|*	|*	|*	|*	|*	|*	|*	|a	|i	|z	|z	|o	|m	|*	|.
|*	|ń	|o	|e	|t	|ż	|*	|l	|a	|*	|[34][S]\rarr	|c	|o	|m	|p	|a	|c	|t	|*	|k	|a	|n	|z	|a	|*	|.
|[35][S]\drarr	|s	|u	|w	|n	|i	|c	|a	|[][,]{ }	|o	|d	|l	|e	|w	|n	|i	|c	|z	|a	|*	|r	|i	|o	|l	|*	|.
|p	|k	|l	|o	|o	|n	|[36][S]\drarr	|r	|z	|a	|d	|z	|i	|z	|n	|a	|*	|*	|*	|*	|o	|k	|w	|a	|*	|.
|e	|i	|l	|l	|l	|s	|s	|o	|w	|*	|[37][S]\darr	|*	|*	|*	|*	|*	|[38][S]\darr	|*	|*	|*	|g	|[][,]{ }	|a	|j	|*	|.
|t	|*	|i	|u	|i	|y	|t	|w	|y	|*	|k	|*	|*	|[39][S]\rarr	|m	|e	|t	|a	|l	|*	|ł	|l	|*	|s	|*	|.
|y	|*	|e	|c	|s	|*	|y	|n	|c	|*	|i	|*	|*	|[40][S]\rarr	|z	|a	|r	|k	|a	|*	|o	|i	|[41][S]\darr	|k	|*	|.
|h	|*	|g	|j	|t	|[42][S]\rarr	|k	|o	|z	|i	|b	|r	|ó	|d	|*	|*	|a	|*	|*	|*	|w	|s	|b	|i	|*	|.
|o	|*	|o	|i	|n	|*	|a	|ś	|a	|*	|i	|*	|*	|[43][S]\rarr	|c	|y	|p	|*	|*	|*	|y	|t	|u	|*	|*	|.
|r	|*	|*	|*	|y	|*	|*	|ć	|j	|*	|t	|*	|*	|*	|[44][S]\rarr	|l	|o	|g	|*	|*	|*	|w	|l	|[45][S]\darr	|*	|.
|z	|*	|*	|*	|*	|*	|*	|*	|n	|[46][S]\rarr	|k	|a	|r	|t	|a	|[][,]{ }	|w	|i	|z	|y	|t	|o	|w	|a	|*	|.
|e	|[47][S]\rarr	|p	|r	|z	|y	|s	|t	|a	|w	|a	|n	|i	|e	|*	|*	|y	|*	|*	|*	|*	|w	|a	|n	|*	|.
|c	|*	|[48][S]\rarr	|g	|w	|a	|ł	|t	|*	|*	|*	|*	|*	|*	|*	|*	|*	|*	|*	|*	|*	|y	|*	|a	|*	|.
|*	|[49][S]\rarr	|s	|k	|u	|b	|u	|n	|[][,]{ }	|k	|l	|e	|s	|z	|c	|z	|o	|w	|n	|i	|k	|*	|*	|*	|*	|.\end{Puzzle}

\newpage

\begin{PuzzleClues}{\textbf{Poziome}\\}\Clue{4}{}{poetka}
\Clue{8}{}{zawodniczka sportowa, która rywalizuje w najmłodszej grupie wiekowej}
\Clue{11}{}{powszechna nazwa, jaką obdarza się nową metodę szkolenia i musztry żołnierzy, jaką wprowadził w latach dwudziestych i trzydziestych XVIII w. w Prusach Leopold von Anhalt-Dessau}
\Clue{12}{}{spółgłoska szczelinowa lub zwarto-szczelinowa artykułowana przez wysklepienie języka w kierunku przedniej części podniebienia przy jednoczesnym kontakcie czubka języka z dziąsłami - ś, ź, ć, dź, ń; określenie używane odnośnie do polskich spółgłosek środkowojęzykowych}
\Clue{14}{}{produkt otrzymywany przez wypełnienie formy tworzywem w stanie ciekłym}
\Clue{17}{}{miasto w płd. Czadzie, nad rzeką Logone, ośrodek handlowy regionu uprawy bawełny}
\Clue{32}{}{syn męża lub żony z poprzedniego związku}
\Clue{33}{}{decyja wydana przez kogoś, w której (na piśmie lub ustnie) zostaje wyrażona chęć zrezygnowania z czegoś, zrzeczenia się czegoś}
\Clue{34}{}{płyta CD}
\Clue{35}{}{suwnica pomostowa przeznaczona do pracy w odlewniach stali}
\Clue{36}{}{to, że coś jest bardzo rzadkie, niegęste}
\Clue{39}{}{podgatunek muzyki rockowej powstały na przełomie lat 60. i 70. XX wieku w Wielkiej Brytanii i Stanach Zjednoczonych}
\Clue{40}{}{miasto w płn. Jordanii na płn.-wsch od Ammanu; rafineria ropy naftowej}
\Clue{42}{}{SALSEFIA roślina zielna ze złożonych, kozibród lekarski zwany jest salsefią}
\Clue{43}{}{kod ISO 4217 funta cypryjskiego}
\Clue{44}{}{urządzenie do pomiaru prędkości statku wodnego względem wody}
\Clue{46}{}{cecha, sposób zachowania, charakterystyczny element stanowiący znak rozpoznawczy jakiejś osoby lub grupy osób; coś, co decyduje o tym, jak ktoś jest postrzegany przez otoczenie}
\Clue{47}{}{w geometrii relacja równoważności figur zdefiniowana przez izometrię, rozumianą intuicyjnie jako identyczność kształtu i wielkości figury}
\Clue{48}{}{zaprzeczenie jakiemuś ustalonemu stanowi, normie}
\Clue{49}{}{Ischyropsalis helwigii - gatunek pajęczaka z rzędu kosarzy, z rodziny Ischyropsalididae}\end{PuzzleClues}

\begin{PuzzleClues}{\textbf{Pionowe}\\}\Clue{1}{}{coś do załatwienia, do zrobienia}
\Clue{2}{}{Damaliscus pygargus phillipsi - podgatunek bonteboka, ssaka parzystokopytnego z rodziny krętorogich; zasiedla tereny wschodniej i środkowej części Afryki Południowej}
\Clue{3}{}{to, że coś ma kilka wariantów, istnieje w kilku wariantach, odmianach, wersjach}
\Clue{5}{}{lekceważące i pogardliwe najczęściej zdradzające negatywne nastawienie (ale też dowcipne, może być użyte pieszczotliwie) określenie dziecka}
\Clue{6}{}{zdecydowana, absolutna przesada}
\Clue{7}{}{kuc Hackney - rasa prawdziwych kuców z typowym dla nich charakterem, dzieło hodowcy Christophera Wilsona z Kirkby Lonsdale w Kumbrii; dzięki ciekawemu i bardzo charakterystycznemu sposobowi ruchu można je obecnie często spotkać na różnego rodzaju pokazach}
\Clue{8}{}{zewnętrzna część lokalu gastronomicznego ustawiana na zewnątrz lokalu w okresie wiosenno-letnim}
\Clue{9}{}{cło lub podobna opłata o równoważnym skutku należna przy wwozie towarów do danego kraju}
\Clue{10}{}{krzywa będąca zbiorem punktów, dla której iloczyn odległości od dwóćh ognisk: F(-a, 0) i F (a, 0) wynosi a do potęgi drugiej}
\Clue{11}{}{chrząszcz z rodziny ryjkowców}
\Clue{13}{}{Liocranidae - rodzina pająków z podrzędu Opisthothelae}
\Clue{15}{}{styl architektoniczny wprowadzony przez Stanisława Witkiewicza w latach 90. XIX wzorowany na tradycyjnym budownictwie górali podhalańskich i wzbogacający je elementami secesji}
\Clue{16}{}{Taxiphyllum densifolium - gatunek mchu z rodziny rokietowatych}
\Clue{18}{}{forma tzw. zwykłego podwyższenia kapitału zakładowego w spółce akcyjnej (w odróżnieniu od podwyższenia warunkowego i celowego) - zarząd tej spółki oferuje w drodze ogłoszenia akcje, co do których służy akcjonariuszom prawo poboru}
\Clue{19}{}{ruch religijno-polityczno-społeczny zapoczątkowany przez Marcina Lutra w XVI wieku, mający na celu odnowę chrześcijaństwa}
\Clue{20}{}{Ptiloprora perstriata - gatunek ptaka z rodziny miodojadów (Meliphagidae)}
\Clue{21}{}{fikcyjna kraina stworzona przez Roberta E. Howarda w cyklu powieści fantasy, których bohaterem jest Conan Barbarzyńca i które kontynuowane były przez innych autorów}
\Clue{22}{}{rodzaj księgi parafialnej, w której proboszczowie zapisywali dane o zdarzeniach podlegających rejestracji w chwili ich zgłaszania lub bezpośrednio po nich}
\Clue{23}{}{cyraneczka, Anas crecca - gatunek średniego, wędrownego ptaka wodnego z rodziny kaczkowatych (Anatidae)}
\Clue{24}{}{likwidacja czegoś, unieważnienie}
\Clue{25}{}{wieś w Polsce położona w województwie wielkopolskim, w powiecie wolsztyńskim, w gminie Siedlec}
\Clue{26}{}{Pyxidea mouhotii - gatunek żółwia z rodziny batagurowatych}
\Clue{27}{}{w potocznym rozumieniu - darwinizm}
\Clue{28}{}{stanowisko budowy małych statków wodnych, szybowców itp}
\Clue{29}{}{Zingiber malaysianum - gatunek rośliny należący do rodziny imbirowatych}
\Clue{30}{}{saturator}
\Clue{31}{}{nieprzejrzystość, niejasność czegoś, to, że coś budzi wątpliwości, nie jest jednoznaczne}
\Clue{33}{}{spodnie wzorowane na spodniach farmerskich, płócienne lub drelichowe nabijane metalowymi nitami}
\Clue{35}{}{pancerny żołnierz na Litwie w XVI/XVIII w}
\Clue{36}{}{malarz i poeta ludowy (1849-1922) pochodził z Podhala}
\Clue{37}{}{rodzaj namiotu koczowniczego ludów z Azji}
\Clue{38}{}{marynarz pełniący służbę przy trapie}
\Clue{41}{}{gula, zgrubienie}
\Clue{45}{}{miasto w Mezopotamii nad Eufratem}\end{PuzzleClues}\newpage\section*{Krzyżówka 11}

\noindent\begin{Puzzle}{21}{31}|*	|[1][S]\darr	|*	|[2][S]\darr	|*	|*	|*	|*	|*	|*	|*	|*	|*	|*	|*	|*	|*	|*	|*	|[3][S]\darr	|*	|*	|.
|*	|i	|[4][S]\drarr	|k	|u	|c	|h	|n	|i	|a	|*	|*	|*	|[5][S]\drarr	|d	|e	|k	|i	|e	|l	|*	|*	|.
|[6][S]\drarr	|s	|p	|o	|w	|i	|e	|d	|ź	|[][,]{ }	|p	|o	|w	|s	|z	|e	|c	|h	|n	|a	|*	|[7][S]\darr	|.
|a	|n	|o	|r	|*	|*	|[8][S]\rarr	|h	|u	|b	|a	|[][,]{ }	|m	|a	|ś	|l	|a	|k	|*	|p	|*	|p	|.
|n	|a	|s	|a	|*	|[9][S]\rarr	|g	|ł	|o	|w	|i	|e	|n	|k	|a	|*	|[10][S]\darr	|*	|*	|a	|[11][S]\darr	|r	|.
|c	|*	|t	|n	|*	|*	|[12][S]\rarr	|l	|e	|n	|o	|n	|k	|i	|*	|*	|k	|*	|*	|r	|p	|z	|.
|y	|*	|o	|*	|*	|*	|*	|[13][S]\drarr	|s	|z	|p	|a	|l	|e	|r	|*	|l	|*	|*	|e	|o	|e	|.
|m	|[14][S]\darr	|z	|[15][S]\rarr	|w	|r	|ó	|b	|l	|o	|w	|e	|*	|w	|*	|*	|u	|*	|*	|n	|s	|c	|.
|o	|a	|u	|[16][S]\rarr	|o	|p	|ł	|o	|t	|k	|i	|*	|*	|n	|*	|*	|c	|*	|*	|t	|t	|i	|.
|n	|r	|c	|[17][S]\rarr	|b	|ł	|y	|s	|z	|c	|z	|[][,]{ }	|m	|i	|e	|d	|z	|i	|*	|o	|ę	|w	|.
|*	|r	|h	|*	|*	|*	|[18][S]\rarr	|s	|i	|ł	|a	|*	|*	|k	|*	|*	|*	|[19][S]\darr	|*	|z	|p	|n	|.
|*	|h	|*	|*	|*	|[20][S]\rarr	|c	|u	|r	|w	|o	|o	|d	|*	|*	|[21][S]\darr	|*	|l	|*	|a	|o	|i	|.
|[22][S]\drarr	|e	|o	|t	|r	|i	|c	|e	|r	|a	|t	|o	|p	|s	|*	|b	|*	|u	|*	|u	|w	|k	|.
|b	|n	|*	|[23][S]\rarr	|e	|l	|i	|t	|y	|z	|m	|*	|*	|[24][S]\rarr	|l	|a	|n	|d	|a	|r	|a	|*	|.
|r	|i	|*	|*	|*	|*	|*	|*	|*	|*	|*	|*	|*	|[25][S]\rarr	|u	|s	|m	|a	|ń	|*	|n	|*	|.
|*	|u	|*	|[26][S]\rarr	|t	|u	|l	|i	|ł	|e	|z	|k	|o	|w	|a	|t	|e	|*	|*	|*	|i	|*	|.
|[27][S]\drarr	|s	|y	|s	|t	|e	|m	|[][,]{ }	|i	|n	|s	|t	|a	|n	|c	|y	|j	|n	|y	|*	|e	|*	|.
|a	|*	|*	|[28][S]\rarr	|c	|e	|n	|a	|[][,]{ }	|n	|o	|m	|i	|n	|a	|l	|n	|a	|*	|*	|[][,]{ }	|*	|.
|s	|*	|*	|*	|[29][S]\rarr	|p	|r	|z	|e	|k	|ł	|a	|m	|a	|n	|i	|e	|*	|*	|*	|c	|[30][S]\darr	|.
|t	|*	|*	|*	|[31][S]\drarr	|z	|ł	|o	|t	|o	|p	|i	|ó	|r	|k	|a	|*	|*	|*	|*	|y	|c	|.
|r	|[32][S]\drarr	|ż	|ó	|ł	|w	|[][,]{ }	|m	|a	|l	|o	|w	|a	|n	|y	|*	|*	|*	|*	|[33][S]\darr	|w	|e	|.
|o	|p	|[34][S]\rarr	|n	|u	|m	|e	|r	|y	|c	|z	|n	|o	|ś	|ć	|*	|*	|*	|*	|j	|i	|d	|.
|n	|ó	|[35][S]\rarr	|o	|p	|o	|l	|a	|n	|i	|n	|*	|*	|*	|*	|*	|*	|[36][S]\darr	|*	|a	|l	|e	|.
|a	|ł	|*	|[37][S]\rarr	|n	|i	|e	|s	|p	|o	|r	|c	|z	|a	|k	|i	|*	|w	|[38][S]\darr	|n	|n	|t	|.
|w	|p	|*	|*	|i	|*	|*	|*	|[39][S]\darr	|*	|*	|*	|*	|*	|*	|[40][S]\drarr	|b	|o	|c	|c	|e	|*	|.
|i	|i	|*	|*	|a	|*	|*	|*	|s	|[41][S]\rarr	|ł	|u	|c	|z	|e	|k	|*	|j	|y	|a	|*	|*	|.
|g	|ę	|*	|*	|k	|*	|[42][S]\rarr	|b	|u	|n	|t	|*	|[43][S]\rarr	|d	|n	|o	|*	|s	|d	|r	|*	|*	|.
|a	|t	|*	|*	|*	|*	|*	|[44][S]\rarr	|m	|i	|l	|u	|*	|*	|*	|l	|*	|k	|o	|z	|*	|*	|.
|c	|r	|*	|*	|*	|*	|*	|[45][S]\rarr	|k	|r	|a	|s	|p	|e	|d	|o	|d	|o	|n	|*	|*	|*	|.
|j	|o	|*	|*	|[46][S]\rarr	|b	|a	|h	|a	|r	|i	|a	|z	|a	|u	|r	|*	|*	|i	|*	|*	|*	|.
|a	|*	|*	|*	|*	|*	|*	|*	|*	|*	|*	|*	|*	|*	|*	|*	|*	|*	|a	|*	|*	|*	|.
|*	|[47][S]\rarr	|r	|y	|t	|m	|[][,]{ }	|b	|i	|o	|l	|o	|g	|i	|c	|z	|n	|y	|*	|*	|*	|*	|.\end{Puzzle}

\newpage

\begin{PuzzleClues}{\textbf{Poziome}\\}\Clue{4}{}{przen. tajniki, warsztat przygotowania czegoś}
\Clue{5}{}{pokrywka, przykrywka, zamknięcie czegoś}
\Clue{6}{}{dawniej - spowiedź całego zgromadzenia, mająca charakter sakramentu (odpuszczenia wszystkich grzechów)}
\Clue{8}{}{Suillus luteus - gatunek grzyba z rodziny maślakowatych; jest szeroko rozprzestrzeniony, występuje na całej półkuli północnej na obszarach o klimacie umiarkowanym}
\Clue{9}{}{głowienka zwyczajna, kaczka rdzawogłowa, Aythya ferina - gatunek ptaka z rodziny kaczkowatych (Anatidae); zamieszkuje środkowe szerokości geograficzne Eurazji - Wyspy Brytyjskie, Europę Środkową i Wschodnią i pas w Azji Środkowej po Mandżurię i północną Japonię, poza tym izolowana populacja występuje w Azji Mniejszej}
\Clue{12}{}{wzór okularów o okrągłych oprawkach, których nazwa pochodzi od nazwiska brytyjskiego muzyka Johna Lennona}
\Clue{13}{}{przejście utworzone z dwóch kolumn ustawionyc obok siebie ludzi}
\Clue{15}{}{Passeriformes - rząd ptaków z podgromady Neornithes}
\Clue{16}{}{obszar koło domu czy chałupy, otoczony płotkiem}
\Clue{17}{}{minerał z grupy siarczków}
\Clue{18}{}{energia, którą dysponuje człowiek, witalność, tyle zdrowia i zapału, ile ma w danej chwili}
\Clue{20}{}{(1878-1927), pisarz amerykański, popularne powieści z życia Indian i traperów; „Szara wilczyca”, „Władca skalny”}
\Clue{22}{}{Eotriceratops - rodzaj roślinożernego dinozaura z rodziny ceratopsów; żył w okresie późnej kredy na terenach Ameryki Północnej, mógł dorastać do 12 metrów i ważył 13 ton}
\Clue{23}{}{pogląd społeczno-polityczny, zakładający wyodrębnienie ze społeczeństwa warstwy wyższej - elity}
\Clue{24}{}{duże, wystawne auto}
\Clue{25}{}{miasto w europejskiej części Federacji Rosyjskiej na płn.-wsch od Woroweża}
\Clue{26}{}{Roridulaceae - rodzina roślin z rzędu wrzosowców}
\Clue{27}{}{prowadzenie spraw sądowych przez kilka etapów postępowania w różnych sądach}
\Clue{28}{}{oficjalna wersja ceny}
\Clue{29}{}{stwierdzenie w części prawdziwe}
\Clue{31}{}{Leptosittaca branickii - gatunek ptaka z rodziny papugowatych (Psittacidae), z podrodziny papug neotropikalnych (Arinae)}
\Clue{32}{}{Chrysemys picta - gatunek gada z podrzędu żółwi skrytoszyjnych z rodziny żółwi błotnych, jedyny przedstawiciel rodzaju Chrysemys, charakteryzujący się pancerzem barwy zielono-oliwkowej lub oliwkowej w różnych odcieniach (aż do czarnego) i występowaniem na tym tle na karapaksie i głowie żółtych lub pomarańczowych plam; żyje na obszarze od południowej Kanady do północnego Meksyku}
\Clue{34}{}{liczebność; liczbowość}
\Clue{35}{}{mieszkaniec Opola}
\Clue{37}{}{maleńkie bezkręgowce o nie ustalonej przynależności, zaliczane zwykle do stawonogów}
\Clue{40}{}{gra sportowa ściśle związana z pétanque, bowls i grą prowansalską, która polega na umieszczeniu własnych kul jak najbliżej małej kulki}
\Clue{41}{}{stosowany w zapisie nutowym znak graficzny  łączący kilka nut}
\Clue{42}{}{protest przeciwko czemuś, postawa wyrażająca - najczęściej w sposób spontaniczny i gwałtowny - przeciwny stosunek do czegoś}
\Clue{43}{}{pogardliwie: kompletny ignorant, zero; stosowane także w odniesieniu do osoby zgniłej moralnie}
\Clue{44}{}{wsch. azjatycki ssak z jeleniowatych- obecnie tylko w zoo}
\Clue{45}{}{Craspedodon - rodzaj dinozaura znany jedynie na podstawie trzech zębów datowanych na górną kredę, znalezionych w okolicach wsi Lonzée w Belgii}
\Clue{46}{}{Bahariasaurus - rodzaj dinozaura z grupy teropodów; żył w okresie kredy na terenach Afryki}
\Clue{47}{}{zjawisko polegające na występowaniu w organizmach żywych pewnych przemian o charakterze cyklicznym, związanych z działaniem swoistych oscylatorów, zwanych zegarami biologicznymi}\end{PuzzleClues}

\begin{PuzzleClues}{\textbf{Pionowe}\\}\Clue{1}{}{egipskie miasto nad Nilem}
\Clue{2}{}{egzemplarz Koranu}
\Clue{3}{}{Lapparentosaurus - rodzaj zauropoda o niepewnej pozycji filogenetycznej, dawniej uznawany za przedstawiciela cetiozaurów, brachiozaurów lub tytanokształtnych}
\Clue{4}{}{Postosuchus - rodzaj drapieżnego archozaura żyjącego od środkowego karniku do noryku (późny trias) na terenie obecnej Ameryki Północnej}
\Clue{5}{}{Origma solitaria - gatunek małego ptaka z rodziny buszówkowatych (Acanthizidae); endemit, występuje jedynie na terenie Australii, w Nowej Południowej Walii; jego środowiskiem występowania są lasy klimatu umiarkowanego, zarośla oraz obszary skalne (klify, szczyty górskie)}
\Clue{6}{}{agregat, model, utrapienie, nicpoń}
\Clue{7}{}{rywal, taki jak w rywalizacji sportowej, konkurent, współzawodnik}
\Clue{10}{}{narzędzie do nakręcania mechanizmu sprężynowego}
\Clue{11}{}{postępowanie, w którym rozpatruje się sprawy z zakresu prawa cywilnego, rodzinnego i opiekuńczego, prawa pracy, sprawy dotyczące ubezpieczeń społecznych oraz inne sprawy, do których przepisy kodeksu postępowania cywilnego stosuje się z mocy ustaw szczególnych; termin prawny}
\Clue{13}{}{(1627-1704), francuski pisarz, teolog i historyk; „Uwagi nad historią powszechną”}
\Clue{14}{}{astrofizyk i fizykochemik szwedzki (1839-1927), nagroda Nobla w 1903 r}
\Clue{19}{}{zespół miejski w Chinach, obejmuje miasto Dalian i Lushuin}
\Clue{21}{}{paryska twierdza z XIVw, później więzienie}
\Clue{22}{}{skrót/symbol waluty birr}
\Clue{27}{}{nawigacja opierająca się na pozycjach ciał niebieskich, względem których określa się położenie statku}
\Clue{30}{}{potoczna nazwa największego w Warszawie domu towarowego (Centralny Dom Towarowy, czyli stołeczny Powszechny Dom Towarowy) stosowana obecnie także jako nazwa budynku, w którym ten dom towarowy się mieścił}
\Clue{31}{}{młot kamieniarski z ostrzem do nacinania w kamieniach rysy}
\Clue{32}{}{ANTRESOLA, MEZANIN, MEZZANINO}
\Clue{33}{}{utytułowany żużlowiec, zawodnik Stali Gorzów}
\Clue{36}{}{ARMIA}
\Clue{38}{}{PIGWA}
\Clue{39}{}{określona ilość pieniędzy}
\Clue{40}{}{w pokerze: pięć kart jednego koloru na ręce}\end{PuzzleClues}\newpage\section*{Krzyżówka 12}

\noindent\begin{Puzzle}{14}{33}|*	|*	|*	|*	|*	|*	|*	|[1][S]\darr	|*	|*	|*	|*	|*	|*	|[2][S]\darr	|.
|*	|[3][S]\rarr	|b	|i	|s	|k	|u	|p	|i	|a	|n	|i	|n	|*	|m	|.
|*	|*	|[4][S]\rarr	|f	|u	|l	|t	|o	|n	|*	|*	|*	|*	|*	|e	|.
|[5][S]\drarr	|w	|i	|e	|l	|k	|o	|ś	|ć	|*	|*	|*	|*	|[6][S]\darr	|l	|.
|o	|*	|*	|[7][S]\rarr	|c	|h	|a	|r	|l	|t	|o	|n	|*	|g	|o	|.
|p	|*	|*	|*	|[8][S]\darr	|*	|*	|e	|*	|*	|*	|*	|*	|e	|d	|.
|r	|*	|[9][S]\darr	|*	|s	|*	|[10][S]\rarr	|d	|y	|n	|i	|a	|*	|r	|r	|.
|e	|*	|k	|*	|k	|*	|*	|n	|*	|*	|[11][S]\darr	|*	|[12][S]\darr	|s	|a	|.
|s	|*	|o	|*	|u	|*	|*	|i	|*	|*	|o	|*	|g	|h	|m	|.
|y	|[13][S]\rarr	|t	|u	|r	|z	|y	|c	|a	|*	|r	|*	|e	|w	|a	|.
|j	|*	|l	|*	|c	|*	|*	|a	|*	|*	|l	|*	|t	|i	|t	|.
|n	|*	|e	|*	|z	|*	|*	|*	|[14][S]\drarr	|b	|a	|ż	|a	|n	|*	|.
|o	|*	|t	|*	|*	|[15][S]\rarr	|g	|a	|m	|a	|*	|*	|*	|*	|[16][S]\darr	|.
|ś	|*	|[][,]{ }	|*	|[17][S]\rarr	|d	|e	|k	|a	|m	|e	|t	|r	|*	|f	|.
|ć	|*	|d	|*	|[18][S]\rarr	|c	|a	|b	|r	|e	|r	|a	|*	|*	|o	|.
|*	|*	|e	|[19][S]\drarr	|s	|p	|i	|n	|k	|s	|*	|[20][S]\darr	|[21][S]\darr	|*	|r	|.
|*	|*	|[][,]{ }	|b	|*	|*	|[22][S]\darr	|*	|e	|*	|*	|o	|j	|*	|m	|.
|*	|[23][S]\drarr	|v	|o	|y	|a	|g	|e	|r	|*	|*	|s	|a	|[24][S]\darr	|u	|.
|[25][S]\drarr	|s	|o	|l	|b	|e	|r	|g	|*	|*	|*	|t	|r	|j	|l	|.
|j	|k	|l	|e	|*	|*	|u	|*	|*	|*	|*	|r	|z	|a	|a	|.
|a	|r	|a	|ń	|*	|*	|s	|*	|*	|[26][S]\darr	|*	|y	|y	|ź	|r	|.
|r	|z	|i	|*	|*	|*	|z	|*	|*	|h	|*	|[][,]{ }	|n	|w	|z	|.
|z	|y	|l	|[27][S]\darr	|[28][S]\rarr	|k	|a	|m	|i	|e	|ń	|s	|k	|i	|*	|.
|ę	|n	|l	|c	|*	|*	|[][,]{ }	|[29][S]\darr	|*	|l	|*	|t	|a	|e	|*	|.
|b	|k	|e	|i	|*	|*	|c	|f	|*	|i	|*	|r	|*	|c	|[30][S]\darr	|.
|i	|a	|*	|e	|*	|*	|h	|i	|*	|n	|*	|z	|*	|*	|k	|.
|n	|*	|[31][S]\drarr	|p	|a	|t	|o	|l	|o	|g	|i	|a	|*	|*	|e	|.
|ó	|*	|l	|l	|*	|*	|j	|a	|*	|*	|*	|ł	|*	|*	|j	|.
|w	|*	|e	|i	|*	|*	|u	|m	|*	|*	|*	|*	|*	|*	|a	|.
|k	|*	|w	|k	|[32][S]\drarr	|k	|r	|e	|m	|o	|w	|o	|ś	|ć	|*	|.
|a	|*	|a	|*	|z	|*	|o	|n	|*	|*	|*	|*	|*	|*	|*	|.
|*	|*	|r	|*	|o	|*	|*	|t	|*	|*	|*	|*	|*	|*	|*	|.
|*	|*	|*	|*	|o	|*	|*	|*	|*	|*	|*	|*	|*	|*	|*	|.
|*	|*	|*	|*	|*	|*	|*	|*	|*	|*	|*	|*	|*	|*	|*	|.\end{Puzzle}

\newpage

\begin{PuzzleClues}{\textbf{Poziome}\\}\Clue{3}{}{przedstawiciel grupy etnograficznej zamieszkującej tereny Biskupizny}
\Clue{4}{}{(1765-1815) amerykański budowniczy statków wodnych}
\Clue{5}{}{cecha tego, co jest wielkie (pod względem rozmiaru fizycznego)}
\Clue{7}{}{Jack, piłkarz brytyjski, złoty medalista mistrzostw świata w Anglii, uznany trener}
\Clue{10}{}{Cucurbita - rodzaj roślin jednorocznych z rodziny dyniowatych obejmujący około 20 gatunków}
\Clue{13}{}{łow. sierść zająca lub królika}
\Clue{14}{}{ur. 1904r, poeta ukraiński, działacz ruchu pokoju, „Mickiewicz w Odessie”, „Rozmowa serc”}
\Clue{15}{}{Vasco da (1460-1524); żeglarz portugalski, odkrył drogę morską do Indii}
\Clue{17}{}{wielokrotność metra, podstawowej jednostki długości w układzie SI; jeden dekametr = 10 metrów}
\Clue{18}{}{ur. 1929r, pisarz kubański „.Odpływająca fala”}
\Clue{19}{}{Michael, bokser amerykański, mistrz olimpijski z Montrealu, zawodowy mistrz świata w kategorii półciężkiej}
\Clue{23}{}{seria 2 amerykańskich próbników badających między innymi Jowisza i Saturna}
\Clue{25}{}{dwuboista norweski, dwukrotny mistrz olimpijski z Grenoble, Sapporo}
\Clue{28}{}{Łucjan, muzykolog i kompozytor (1885-1964); zorganizował pierwsze w Polsce folklorystyczne Archiwum etnograficzne}
\Clue{31}{}{dział medycyny, nauka o chorobach}
\Clue{32}{}{cecha tego, co ma jasnobeżowy kolor, co jest w kolorze białym przełamanym beżem}\end{PuzzleClues}

\begin{PuzzleClues}{\textbf{Pionowe}\\}\Clue{1}{}{pośredniczka, kobieta pośrednik}
\Clue{2}{}{jakieś dramatyczne wydarzenie w życiu, przedstawione w sposób przejaskrawiony}
\Clue{5}{}{ucisk, wykluczanie, marginalizowanie}
\Clue{6}{}{ameryk kompozytor i pianista (1898-1937); muzyka do filmów, rewii, utwory symfoniczne; 'Błękitna rapsodia'}
\Clue{8}{}{zaburzenie pracy mięśni, powodujące ból}
\Clue{9}{}{kotlet panierowany w jajku i bułce tartej, smażony na głębokim oleju sporządzony z roztłuczonej piersi kurczaka nadziewanej masłem i przyprawami}
\Clue{11}{}{rzeka w zachodniej Polsce, największy, prawy dopływ Baryczy o długości 88 km i powierzchni dorzecza 1546 km2}
\Clue{12}{}{drewniane obuwie japońskie przypominające chodaki}
\Clue{14}{}{rekwizyt do gry, substytut broni w paintballu}
\Clue{16}{}{formuła, konwencjonalne sformułowanie, powatrzalny schemat}
\Clue{19}{}{RAP; drapieżna ryba z karpiowatych o długości do 80 cm; rzeki wschodniej Europy}
\Clue{20}{}{strzał, który jest oddany nabojem bojowym}
\Clue{21}{}{przyprawa ze sproszkowanych, suszonych warzyw}
\Clue{22}{}{Pyrus pyriflia 'Chojuro' - odmiana uprawna gruszy chińskiej}
\Clue{23}{}{opakowanie do transportu szklanych butelek standardowych kształtów i rozmiarów (z wódką, winem, piwem, wodą mineralną itp.)}
\Clue{24}{}{BORSUK; ssak z rodziny łasicowatych}
\Clue{25}{}{nalewka domowa robiona z owoców jarzębiny}
\Clue{26}{}{stanowisko budowy małych statków wodnych, szybowców itp}
\Clue{27}{}{ptak z rzędu wróblowatych podobny do trznadla, u samca czoło i podgardle czarne; Eurazja, płn-zach. Afryka}
\Clue{29}{}{włókno białkowe znajdujące się w cytoplazmie, które odpowiada za rozmieszczenie organelli w komórce}
\Clue{30}{}{nadbrzeże uzbrojone w urządzenia cumownicze}
\Clue{31}{}{dźwignik}
\Clue{32}{}{teren udostępniony odwiedzającym, na którym hodowane są zwierzęta, najczęściej pochodzące z różnych obszarów geograficznych}\end{PuzzleClues}\newpage\section*{Krzyżówka 13}

\noindent\begin{Puzzle}{17}{26}|*	|[1][S]\drarr	|s	|o	|l	|s	|t	|y	|c	|j	|u	|m	|*	|*	|*	|*	|*	|*	|.
|*	|o	|*	|*	|[2][S]\drarr	|t	|e	|n	|d	|e	|r	|*	|*	|*	|[3][S]\darr	|*	|*	|*	|.
|[4][S]\drarr	|s	|e	|r	|p	|e	|n	|t	|*	|[5][S]\drarr	|r	|u	|*	|[6][S]\darr	|p	|*	|[7][S]\darr	|[8][S]\darr	|.
|c	|z	|*	|[9][S]\darr	|o	|[10][S]\drarr	|p	|r	|o	|z	|e	|l	|i	|t	|a	|*	|p	|z	|.
|h	|o	|*	|p	|w	|k	|[11][S]\drarr	|s	|*	|a	|[12][S]\darr	|[13][S]\darr	|[14][S]\darr	|e	|s	|*	|ó	|ł	|.
|a	|ł	|*	|e	|e	|k	|w	|*	|*	|r	|n	|r	|p	|l	|c	|*	|ł	|o	|.
|n	|o	|[15][S]\darr	|r	|r	|*	|y	|*	|[16][S]\darr	|a	|i	|e	|r	|a	|h	|*	|c	|ż	|.
|*	|m	|p	|r	|b	|[17][S]\rarr	|s	|z	|c	|z	|e	|n	|a	|*	|a	|[18][S]\darr	|i	|e	|.
|*	|*	|ę	|a	|a	|[19][S]\darr	|i	|*	|e	|a	|d	|n	|w	|*	|*	|h	|ę	|n	|.
|*	|[20][S]\drarr	|p	|u	|l	|p	|e	|t	|*	|[][,]{ }	|o	|*	|o	|[21][S]\darr	|[22][S]\darr	|o	|ż	|i	|.
|*	|m	|a	|l	|l	|i	|d	|*	|*	|m	|l	|*	|[][,]{ }	|h	|o	|w	|a	|e	|.
|*	|a	|w	|t	|*	|k	|l	|[23][S]\darr	|*	|o	|i	|[24][S]\drarr	|p	|a	|s	|e	|r	|*	|.
|*	|ł	|a	|*	|[25][S]\darr	|*	|o	|k	|[26][S]\darr	|r	|s	|p	|r	|w	|t	|a	|ó	|[27][S]\darr	|.
|*	|a	|[][,]{ }	|*	|m	|[28][S]\darr	|n	|r	|w	|o	|e	|r	|y	|a	|r	|*	|w	|k	|.
|*	|[][,]{ }	|d	|*	|a	|p	|y	|y	|e	|w	|k	|o	|w	|ń	|o	|[29][S]\darr	|k	|a	|.
|[30][S]\drarr	|w	|a	|r	|g	|a	|*	|p	|k	|a	|*	|t	|a	|c	|ż	|w	|a	|w	|.
|s	|i	|c	|*	|l	|t	|*	|t	|t	|*	|*	|a	|t	|z	|e	|o	|*	|a	|.
|z	|e	|h	|*	|o	|o	|*	|o	|o	|[31][S]\darr	|*	|n	|n	|y	|ń	|j	|*	|ł	|.
|t	|ś	|o	|*	|w	|w	|[32][S]\drarr	|p	|r	|z	|ą	|d	|e	|k	|*	|c	|*	|e	|.
|a	|*	|w	|[33][S]\drarr	|n	|o	|k	|s	|*	|e	|*	|r	|*	|*	|*	|i	|*	|k	|.
|n	|[34][S]\darr	|a	|z	|i	|ś	|a	|*	|*	|s	|[35][S]\drarr	|i	|s	|a	|j	|e	|w	|*	|.
|g	|m	|*	|m	|c	|ć	|k	|*	|[36][S]\drarr	|p	|r	|a	|g	|n	|ą	|c	|y	|*	|.
|a	|u	|*	|k	|a	|*	|a	|*	|n	|ó	|u	|*	|*	|*	|*	|h	|*	|*	|.
|*	|z	|*	|*	|*	|*	|*	|*	|o	|ł	|m	|*	|*	|*	|*	|ó	|*	|*	|.
|[37][S]\rarr	|a	|n	|a	|l	|i	|t	|y	|k	|*	|b	|[38][S]\rarr	|t	|y	|k	|w	|a	|*	|.
|*	|*	|*	|*	|*	|*	|*	|*	|*	|[39][S]\rarr	|a	|n	|a	|n	|d	|*	|*	|*	|.
|*	|*	|*	|*	|*	|*	|*	|*	|*	|*	|*	|*	|*	|*	|*	|*	|*	|*	|.\end{Puzzle}

\newpage

\begin{PuzzleClues}{\textbf{Poziome}\\}\Clue{1}{}{przesilenie}
\Clue{2}{}{okręt-baza}
\Clue{4}{}{instrument muzyczny z rodziny cynków; zrobiony z dwóch wydrążonych drewnianych łusek drzewa kasztanowego sklejonych razem i powleczonych skórą, wygięty czterokrotnie, posiada sześć otworów i miedziany ustnik}
\Clue{5}{}{w chemii: symbol rutenu}
\Clue{10}{}{nowo pozyskany, gorliwy wyznawca jakiejś religii}
\Clue{11}{}{w chemii: symbol siarki}
\Clue{17}{}{szczęka; słowo młodzieżowe}
\Clue{20}{}{ktoś dość gruby, pulchny}
\Clue{24}{}{osoba biorąca udział w handlu kradzionymi rzeczami lub je rozprowadzająca}
\Clue{30}{}{zgrubienie na brzegu otworu muszli}
\Clue{32}{}{mężczyzna przędący lub zatrudniony w przędzalni}
\Clue{33}{}{jednostka natężenia oświetlenia}
\Clue{35}{}{Jegor, poeta rosyjski, ur,1926r; „Sąd pamięci”}
\Clue{36}{}{ten, który odczuwa pragnienie}
\Clue{37}{}{specjalista, który coś analizuje na czyjeś zlecenie}
\Clue{38}{}{Lagenaria - rodzaj jednorocznych, pnących roślin tropikalnych z rodziny dyniowatych.}
\Clue{39}{}{ur. 1905r, indyjski pisarz tworzący w języku angielskim i pendżabskim; „Kulis” - Leninowska Nagroda Pokoju}\end{PuzzleClues}

\begin{PuzzleClues}{\textbf{Pionowe}\\}\Clue{1}{}{pogardliwie o osobie owładniętej jakąś myślą, fanatyku religijnym lub ideologicznym}
\Clue{2}{}{zawierający żyroskop przyrząd do treningu i rehabilitacji palców, stawów, mięśni nadgarstka i przedramienia}
\Clue{3}{}{w języku religijnym (z judeo-chrześcijańskiego kręgu kulturowego): ofiara, poświęcenie czegoś na życzenie, w imię Boga (nazwa wywodzi się od zwyczaju, święta znanego w plemionach semickich jeszcze przed Mojżeszem - zabijania na ofiarę jednego z najbardziej okazałych zwierząt jednorocznych w stadzie przed wyruszeniem na wypas i zjadania go na wspólnej kolacji w imię braterstwa pomiędzy ludźmi i przymierza z Bogiem)}
\Clue{4}{}{jedna z najważniejszych szkół chińskiego buddyzmu, należąca do praktycznej i medytacyjnej tradycji buddyzmu}
\Clue{5}{}{ostra bakteryjna choroba zakaźna gryzoni i (rzadziej) innych drobnych ssaków, a także człowieka}
\Clue{6}{}{miasto i port w Hondurasie nad Morzem Karaibskim}
\Clue{7}{}{samochód półciężarowy z nadwoziem typu pikap}
\Clue{8}{}{w językoznawstwie - derywat, który powstał z połączenia conajmniej dwóch rdzeni za pomocą elementu łączącego - międzyrostka}
\Clue{9}{}{architekt francuski (1613-88), przedstawiciel barokowego klasycznego Ludwika XIV}
\Clue{10}{}{akt normatywny stanowiący zbiór przepisów regulujących odpowiedzialność karną obywateli danego państwa, zawierający definicję przestępstwa, zasady odpowiedzialności za przestępstwo, zasady przedawnienia odpowiedzialności karnej oraz spis kar i reguły ich stosowania}
\Clue{11}{}{człowiek, którego wysiedlono, zmuszono do opuszczenia miejsca zamieszkania}
\Clue{12}{}{łow. młody lis}
\Clue{13}{}{pisarz niemiecki, uczestnik hiszpańskiej wojny domowej (1889-1979), „Wojna”, „Murzynek Nobi”}
\Clue{14}{}{jedna z dwóch podstawowych gałęzi prawa (obok prawa publicznego), skupiająca normy prawne, których zadaniem jest ochrona interesu jednostek i regulacja stosunków pomiędzy nimi}
\Clue{15}{}{Crepis tectorum - gatunek rośliny należący do rodziny astrowatych}
\Clue{16}{}{w chemii: symbol ceru}
\Clue{18}{}{KENCJA palma z wysp Oceanu Spokojnego, uprawiana jako ozdobna roślina doniczkowa}
\Clue{19}{}{jeden z czterech kolorów w kartach, oznaczony małym czarnym listkiem}
\Clue{20}{}{wieś w Polsce położona w województwie mazowieckim, w powiecie grójeckim, w gminie Belsk Duży}
\Clue{21}{}{mieszkaniec Hawany}
\Clue{22}{}{roślina zielna strefy umiarkowanej z rodziny jaskrowatych; kwiaty z ostrogą}
\Clue{23}{}{Kryptops - rodzaj prymitywnego teropoda z rodziny abelizaurów; żył w okresie wczesnej kredy na terenach dzisiejszego Nigru}
\Clue{24}{}{zjawisko występujące w kwiatach obupłciowych u niektórych gatunków roślin, u których pręciki dojrzewają szybciej niż słupki}
\Clue{25}{}{maszyna, służąca do maglowania, czyli prasowania przy użyciu systemu walców}
\Clue{26}{}{obiekt matematyczny, który ma moduł, kierunek oraz zwrot (określający orientację wzdłuż danego kierunku)}
\Clue{27}{}{niewielka część czegoś, odcinek, fragment, urywek}
\Clue{28}{}{to, że sytuacja nie ulega zmianie i nie ma widoków na przezwyciężenie trudności i pozytywne rozwiązanie}
\Clue{29}{}{wieś w województwie dolnośląskim, w powiecie lwóweckim}
\Clue{30}{}{sprzęt sportowy służący do wykonywania nim ćwiczeń siłowych}
\Clue{31}{}{grupa muzyków występujących razem}
\Clue{32}{}{nowozelandzki ptak z rodziny papug}
\Clue{33}{}{kod ISO 4217 kwachy zambiijskiej}
\Clue{34}{}{muzyka śpiewana, grana lub tylko odsłuchiwana przez kogoś}
\Clue{35}{}{latynoamerykański taniec towarzyski pochodzący z Kuby}
\Clue{36}{}{kod ISO 4217 korony norweskiej}\end{PuzzleClues}\newpage\section*{Krzyżówka 14}

\noindent\begin{Puzzle}{23}{32}|*	|*	|*	|*	|*	|*	|*	|*	|*	|*	|[1][S]\darr	|*	|*	|*	|*	|[2][S]\darr	|*	|*	|[3][S]\darr	|[4][S]\darr	|*	|*	|*	|[5][S]\darr	|.
|*	|*	|*	|[6][S]\darr	|*	|*	|*	|*	|[7][S]\drarr	|g	|r	|u	|s	|*	|[8][S]\darr	|s	|[9][S]\darr	|*	|ś	|s	|[10][S]\darr	|[11][S]\darr	|*	|a	|.
|*	|*	|*	|w	|[12][S]\darr	|*	|*	|*	|f	|*	|y	|*	|[13][S]\drarr	|a	|d	|a	|m	|*	|w	|z	|s	|p	|*	|g	|.
|*	|*	|*	|z	|r	|*	|*	|[14][S]\darr	|a	|*	|k	|*	|n	|*	|z	|g	|i	|*	|i	|y	|p	|a	|*	|a	|.
|*	|[15][S]\darr	|*	|ó	|u	|*	|*	|s	|n	|*	|*	|[16][S]\darr	|u	|*	|i	|a	|n	|*	|ę	|b	|a	|p	|[17][S]\darr	|w	|.
|*	|c	|*	|r	|c	|*	|*	|y	|a	|*	|*	|n	|ż	|*	|a	|n	|a	|*	|t	|k	|c	|i	|j	|o	|.
|*	|r	|*	|[][,]{ }	|h	|[18][S]\darr	|*	|g	|t	|[19][S]\darr	|*	|i	|e	|*	|d	|*	|*	|*	|ó	|o	|e	|l	|ę	|w	|.
|*	|u	|*	|c	|y	|a	|*	|n	|y	|p	|*	|h	|n	|*	|z	|[20][S]\darr	|*	|*	|w	|w	|r	|o	|z	|e	|.
|*	|z	|[21][S]\darr	|i	|[][,]{ }	|m	|*	|a	|c	|o	|*	|i	|i	|[22][S]\rarr	|i	|d	|i	|a	|k	|a	|n	|t	|y	|*	|.
|*	|a	|o	|o	|t	|i	|*	|ł	|z	|i	|[23][S]\drarr	|l	|e	|m	|u	|r	|[][,]{ }	|w	|a	|r	|i	|*	|k	|*	|.
|*	|d	|d	|ł	|a	|n	|*	|[][,]{ }	|n	|d	|w	|i	|c	|[24][S]\darr	|*	|z	|[25][S]\rarr	|b	|*	|*	|k	|*	|i	|*	|.
|*	|o	|d	|k	|l	|e	|[26][S]\drarr	|p	|o	|ł	|o	|z	|[][,]{ }	|z	|i	|e	|l	|o	|n	|y	|*	|*	|[][,]{ }	|*	|.
|*	|[][,]{ }	|y	|o	|a	|k	|p	|o	|ś	|o	|d	|m	|k	|ł	|*	|w	|*	|[27][S]\darr	|*	|[28][S]\drarr	|p	|a	|s	|*	|.
|*	|b	|c	|w	|s	|[][,]{ }	|s	|w	|ć	|*	|n	|*	|o	|a	|[29][S]\darr	|i	|*	|g	|*	|p	|[30][S]\darr	|*	|e	|*	|.
|*	|r	|h	|s	|o	|e	|z	|t	|*	|*	|i	|*	|z	|j	|ż	|k	|*	|r	|*	|r	|g	|*	|m	|*	|.
|*	|a	|a	|k	|g	|g	|c	|a	|*	|[31][S]\darr	|k	|[32][S]\darr	|i	|n	|y	|[][,]{ }	|*	|u	|*	|o	|a	|[33][S]\darr	|i	|*	|.
|*	|z	|n	|i	|e	|i	|z	|r	|*	|s	|*	|b	|*	|i	|w	|w	|[34][S]\darr	|s	|[35][S]\darr	|c	|l	|b	|t	|*	|.
|*	|y	|i	|e	|n	|p	|o	|z	|*	|z	|*	|l	|*	|k	|a	|ś	|t	|z	|n	|h	|a	|a	|o	|*	|.
|[36][S]\drarr	|l	|e	|g	|i	|s	|l	|a	|t	|y	|w	|a	|*	|*	|[][,]{ }	|c	|e	|a	|i	|[][,]{ }	|g	|r	|c	|[37][S]\darr	|.
|g	|i	|[][,]{ }	|o	|c	|k	|i	|j	|[38][S]\drarr	|s	|y	|s	|t	|e	|m	|i	|k	|*	|e	|b	|o	|w	|h	|p	|.
|r	|j	|h	|*	|z	|i	|n	|ą	|l	|z	|*	|k	|*	|[39][S]\darr	|o	|b	|t	|[40][S]\darr	|s	|e	|[][,]{ }	|y	|a	|a	|.
|a	|s	|o	|*	|n	|*	|k	|c	|i	|k	|*	|*	|*	|p	|w	|s	|o	|l	|t	|z	|o	|[][,]{ }	|m	|c	|.
|b	|k	|l	|*	|e	|*	|a	|y	|b	|o	|[41][S]\darr	|*	|*	|y	|a	|k	|n	|i	|e	|p	|l	|k	|i	|y	|.
|a	|i	|o	|*	|*	|*	|*	|*	|u	|j	|o	|[42][S]\darr	|*	|r	|*	|i	|i	|s	|r	|ł	|b	|l	|c	|f	|.
|r	|e	|t	|[43][S]\drarr	|k	|o	|n	|t	|r	|a	|p	|u	|n	|k	|t	|*	|k	|i	|o	|o	|r	|u	|k	|i	|.
|z	|*	|r	|s	|[44][S]\darr	|*	|*	|*	|n	|g	|a	|a	|[45][S]\darr	|a	|*	|*	|a	|e	|w	|m	|z	|b	|i	|k	|.
|*	|*	|o	|ó	|k	|*	|*	|*	|a	|o	|r	|k	|b	|*	|*	|*	|*	|c	|n	|i	|y	|o	|e	|a	|.
|*	|[46][S]\drarr	|p	|l	|a	|c	|y	|k	|*	|d	|i	|*	|l	|*	|*	|*	|*	|k	|o	|e	|m	|w	|*	|ł	|.
|*	|o	|o	|*	|u	|[47][S]\rarr	|ż	|ą	|d	|a	|n	|i	|e	|[][,]{ }	|w	|y	|j	|a	|ś	|n	|i	|e	|ń	|*	|.
|*	|b	|w	|[48][S]\rarr	|s	|z	|r	|a	|f	|*	|*	|[49][S]\rarr	|k	|l	|i	|p	|s	|*	|ć	|n	|*	|*	|*	|*	|.
|*	|c	|e	|[50][S]\rarr	|z	|o	|o	|f	|e	|n	|o	|l	|o	|g	|i	|a	|*	|*	|*	|y	|*	|*	|*	|*	|.
|*	|y	|*	|[51][S]\rarr	|a	|b	|a	|*	|[52][S]\rarr	|p	|o	|r	|t	|u	|g	|a	|l	|k	|a	|*	|*	|*	|*	|*	|.
|*	|*	|*	|*	|*	|[53][S]\rarr	|c	|o	|y	|p	|e	|l	|*	|*	|*	|*	|*	|*	|*	|*	|*	|*	|*	|*	|.\end{Puzzle}

\newpage

\begin{PuzzleClues}{\textbf{Poziome}\\}\Clue{7}{}{ŻURAW}
\Clue{13}{}{według tradycji biblijnej: pierwszy człowiek, symbol każdego człowieka}
\Clue{22}{}{Idiacanthus - rodzaj drapieżnych, morskich ryb głębinowych z rodziny wężorowatych (Stomiidae)}
\Clue{23}{}{Varecia variegata - gatunek małpiatki z rodziny lemurowatych, której jest największym przedstawicielem; prawdopodobnie jest to jedyny gatunek naczelnych budujący gniazda wyłącznie na czas porodu i pierwszych tygodni opieki nad młodymi; zamieszkuje wschodni Madagaskar, żyjąc na terenach zalesionych, od brzegu morza do wysokości 1 350 m n.p.m}
\Clue{25}{}{bajt - najmniejsza adresowalna jednostka informacji pamięci komputerowej, składająca się z bitów}
\Clue{26}{}{Gonyosoma oxycephalum - gatunek węża z rodziny połozowatych, występujący na Półwyspie Indochińskim oraz na wyspach Malezji, Filipin, Indonezji oraz w części Indii}
\Clue{28}{}{wąski długi obszar wydzielony ze względu na swoją charakterystykę lub funkcje}
\Clue{36}{}{organ władzy ustawodawczej}
\Clue{38}{}{zasada organizacji czegoś, podstawa funkcjonowania lub konstrukcji czegoś}
\Clue{43}{}{dział teorii muzyki zajmujący się techniką kompozytorską o tej samej nazwie}
\Clue{46}{}{niewielka przestrzeń wydzielona w jakimś celu}
\Clue{47}{}{rodzaj pisma urzędowego, w którym nadawca wzywa odbiorcę do udzielenia wyjaśnień w jakiejś kwestii}
\Clue{48}{}{technika rysowania polegająca na kreśleniu kresek równoległych lub przecinających się, w celu wydobycia bryłowatości rysowanego przedmiotu}
\Clue{49}{}{biżuteria damska, przyczepiana do płatków uszu uchwytem w formie klamerki}
\Clue{50}{}{fenologia zwierząt}
\Clue{51}{}{ABAJA}
\Clue{52}{}{mieszkanka Portugalii, kobieta pochodzenia portugalskiego}
\Clue{53}{}{rodzina malarzy francuski (XVII-XVIII w.) przedstawiciele oficjalnej sztuki dworskiej}\end{PuzzleClues}

\begin{PuzzleClues}{\textbf{Pionowe}\\}\Clue{1}{}{donośny gardłowy głos zwierzęcia}
\Clue{2}{}{ur. 1935r, pisarka francuska, współczesne powieści psychologiczne; „Witaj smutku”, „Zamek w Szwecji”, „Pewien uśmiech”, nowele, sztuki}
\Clue{3}{}{w dawnym kapitalizmie: dzień przymusowo wolny od pracy (z powodu jakiegoś święta), za który nie otrzymywało się wypłaty (rzecz uważana za uciążliwość)}
\Clue{4}{}{zawartość szybkowaru, szczelnego garnka, w którym w czasie gotowania potraw następuje wzrost ciśnienia, który powoduje wzrost temperatury wrzenia wody i przyspieszenie procesu gotowania}
\Clue{5}{}{Agavoideae - podrodrodzina (wg systemu APG III z 2009) rodziny szparagowatych; w niektórych systemach klasyfikowana w randze rodziny agawowatych}
\Clue{6}{}{podstawowy wzór w technice rakietowej określający prędkość rakiety zużywającej podczas lotu paliwo, czyli rakiety zmieniającej masę}
\Clue{7}{}{to, że coś jest wykonywane z fanatyzmem, nadmierną gorliwością, bezrefleksyjnym zaangażowaniem}
\Clue{8}{}{czule, pieszczotliwie o dziadku - ojcu rodzica; forma pierwotnie wyłącznie wołaczowa, dzisiaj także mianownikowa}
\Clue{9}{}{starożytna jednostka wagowo-pieniężna równa 1/60 talenta}
\Clue{10}{}{dziedziniec więzienny; miejsce spacerów osadzonych w zakładzie karnym więźniów}
\Clue{11}{}{zwitek papieru, na który nawija się włosy w celu uzyskania loków}
\Clue{12}{}{ruchy pionowe, które powodują stopniowe obniżanie lądu, a w efekcie przekształcenie go w dno oceaniczne}
\Clue{13}{}{Demodex caprae - gatunek roztocza z rodzaju nużeńca}
\Clue{14}{}{kolejowe urządzenie sygnalizacyjne informujące o tym, jaki sygnał jest wyświetlany przez następny semafor}
\Clue{15}{}{jednostka walutowa Brazylii do 1990 r}
\Clue{16}{}{pogląd filozoficzny całkowicie lub częściowo negujący istnienie pewnych bytów}
\Clue{17}{}{dawna nazwa dla wielkiej rodziny języków, zajmującej obszar od afrykańskich wybrzeży Atlantyku (język hausa, arabski i berberyjski na zachodzie) po Róg Afryki (języki kuszyckie) i Bliski Wschód z językami hebrajskim i arabskim na wschodzie}
\Clue{18}{}{gatunek rośliny zielnej, która rośnie dziko w Azji Zachodniej, Afryce Północnej, na Maderze, Wyspach Kanaryjskich oraz w Europie Południowej}
\Clue{19}{}{zawartość poidła}
\Clue{20}{}{Platypelis milloti - gatunek płaza bezogonowego z rodziny wąskopyskowatych}
\Clue{21}{}{technika psychoterapeutyczna łącząca w sobie przyspieszone oddychanie oraz relaksację i muzykę jako czynniki wspomagające}
\Clue{23}{}{TOPIK}
\Clue{24}{}{pies myśliwski}
\Clue{26}{}{Andrena - rodzaj pszczoły z rodziny pszczolinkowatych (Andrenidae); w Europie występuje około 200 gatunków o różnej wielkości osobników (od drobnych do większych niż pszczoła miodna)}
\Clue{27}{}{Pyrus - rodzaj w większości niewielkich drzew z rodziny różowatych (Rosaceae), uprawianych ze względu na ich słodkie owoce o kulistym zwężającym się ku górze kształcie}
\Clue{28}{}{proch zawierający substancje uniemożliwiające zapłon gazów po wyjściu z lufy}
\Clue{29}{}{język, którym człowiek posługuje się w bezpośredniej komunikacji}
\Clue{30}{}{galago gruboogonowy, Otolemur crassicaudatus - gatunek małpiatki z rodziny galagowatych; żyje w stadzie, w leśnych gęstwinach południowo-zachodniej Afryki}
\Clue{31}{}{rodzaj szyszki, której łuski nasienne są zmięśniałe i ze sobą zrośnięte, tworząc twór przypominający jagodę}
\Clue{32}{}{przejaw, przebłysk}
\Clue{33}{}{zestaw barw symbolizujących dany klub sportowy}
\Clue{34}{}{cecha danego terenu, układ i wzajemne oddziaływanie warstw skalnych, lokalna specyfika budowy skorupy ziemskiej}
\Clue{35}{}{to, że coś jest niesterowne, nie można nad tym zapanować, kierować, zarządzać (np.niesterowność państwa, systemu)}
\Clue{36}{}{padlinożerny chrząszcz; zakopują w ziemi trupy kręgowców, którymi się żywią}
\Clue{37}{}{w Kościele katolickim: relikwiarz w kształcie krzyża lub monstrancji, bogato zdobiony, podawany wiernym do całowania}
\Clue{38}{}{szybki, zwrotny, jednorzędowy okręt wojenny z taranem i jednym masztem}
\Clue{39}{}{wiejska potrawa z gotowanych ziemniaków, mąki i zsiadłego mleka}
\Clue{40}{}{tradycyjna wędzona kiełbasa wieprzowa, wytwarzana w gminach Liszki i Czernichów w województwie małopolskim}
\Clue{41}{}{radziecki biolog i biochemik (1894-1980); autor jednej z hipotez powstania życia na Ziemi}
\Clue{42}{}{kod ISO 4217 karbowańca}
\Clue{43}{}{artykuł spożywczy, będący prawie czystym chlorkiem sodu (NaCl), stosowany jako przyprawa i konserwant}
\Clue{44}{}{CHAMĄTKO; metalowa wkładka wzmacniająca i wiążąca od wewnątrz końcową pętlę liny}
\Clue{45}{}{silnie trujący chwast z rodziny baldaszkowatych, podobny do pietruszki, w Polsce pospolity}
\Clue{46}{}{przybysz z kosmosu}\end{PuzzleClues}\newpage\section*{Krzyżówka 15}

\noindent\begin{Puzzle}{22}{29}|*	|*	|*	|*	|*	|*	|*	|*	|*	|*	|*	|[1][S]\drarr	|k	|a	|n	|t	|a	|r	|y	|d	|a	|*	|*	|.
|*	|*	|*	|*	|*	|*	|*	|*	|*	|*	|[2][S]\rarr	|e	|m	|f	|i	|t	|e	|u	|t	|a	|*	|*	|*	|.
|*	|*	|[3][S]\rarr	|p	|o	|d	|r	|z	|e	|ń	|[][,]{ }	|g	|a	|r	|b	|a	|t	|y	|*	|*	|*	|*	|*	|.
|*	|*	|*	|*	|[4][S]\rarr	|k	|ł	|ę	|b	|o	|s	|z	|[][,]{ }	|n	|a	|d	|w	|o	|d	|n	|y	|*	|*	|.
|[5][S]\drarr	|c	|z	|a	|r	|n	|y	|[][,]{ }	|c	|h	|l	|e	|b	|*	|*	|*	|*	|*	|*	|*	|*	|*	|*	|.
|s	|*	|*	|[6][S]\rarr	|s	|t	|a	|t	|u	|s	|[][,]{ }	|m	|a	|t	|e	|r	|i	|a	|l	|n	|y	|*	|*	|.
|t	|[7][S]\rarr	|w	|s	|p	|ó	|ł	|c	|i	|e	|r	|p	|i	|ą	|c	|y	|*	|*	|*	|*	|*	|*	|[8][S]\darr	|.
|ę	|*	|*	|*	|[9][S]\rarr	|s	|t	|a	|s	|z	|e	|l	|*	|*	|*	|*	|*	|*	|*	|*	|*	|*	|k	|.
|ż	|[10][S]\rarr	|a	|s	|f	|o	|d	|e	|l	|o	|w	|a	|[][,]{ }	|ł	|ą	|k	|a	|*	|*	|*	|*	|*	|o	|.
|e	|*	|*	|[11][S]\rarr	|w	|a	|n	|g	|a	|[][,]{ }	|g	|r	|u	|b	|o	|d	|z	|i	|o	|b	|a	|*	|n	|.
|n	|*	|*	|*	|*	|*	|*	|*	|[12][S]\rarr	|p	|r	|z	|e	|g	|l	|ą	|d	|a	|r	|k	|a	|*	|f	|.
|i	|[13][S]\drarr	|t	|o	|r	|f	|o	|w	|i	|e	|c	|[][,]{ }	|c	|i	|e	|m	|n	|y	|*	|[14][S]\darr	|[15][S]\darr	|[16][S]\darr	|e	|.
|e	|m	|*	|*	|[17][S]\rarr	|k	|a	|n	|o	|p	|u	|s	|*	|*	|*	|[18][S]\darr	|[19][S]\darr	|*	|*	|s	|s	|w	|s	|.
|[][,]{ }	|i	|[20][S]\rarr	|k	|u	|r	|o	|b	|r	|o	|d	|y	|*	|*	|*	|p	|r	|*	|[21][S]\darr	|c	|k	|e	|j	|.
|m	|k	|*	|[22][S]\darr	|*	|*	|*	|*	|*	|[23][S]\drarr	|o	|g	|i	|w	|a	|r	|a	|*	|g	|e	|a	|g	|a	|.
|o	|r	|*	|u	|[24][S]\rarr	|k	|a	|p	|e	|l	|a	|n	|*	|[25][S]\rarr	|t	|a	|j	|w	|a	|n	|k	|a	|*	|.
|l	|o	|*	|l	|*	|[26][S]\rarr	|j	|a	|j	|o	|w	|a	|r	|*	|*	|w	|t	|[27][S]\darr	|l	|a	|a	|n	|*	|.
|o	|p	|*	|g	|*	|[28][S]\darr	|[29][S]\rarr	|m	|e	|t	|y	|l	|*	|*	|*	|o	|u	|k	|e	|r	|n	|i	|*	|.
|w	|o	|[30][S]\darr	|a	|*	|b	|*	|*	|*	|n	|*	|n	|*	|*	|*	|[][,]{ }	|z	|r	|a	|i	|k	|n	|*	|.
|e	|w	|w	|[][,]{ }	|*	|a	|*	|*	|*	|i	|*	|y	|*	|*	|*	|n	|y	|ę	|s	|u	|a	|*	|*	|.
|*	|i	|i	|o	|[31][S]\drarr	|l	|o	|f	|t	|k	|i	|*	|*	|*	|*	|i	|*	|t	|*	|s	|*	|*	|*	|.
|*	|e	|r	|d	|e	|e	|*	|*	|*	|*	|[32][S]\rarr	|w	|y	|b	|i	|e	|l	|a	|c	|z	|*	|*	|*	|.
|*	|ś	|o	|s	|l	|t	|[33][S]\rarr	|m	|a	|d	|a	|p	|o	|l	|a	|m	|*	|r	|*	|*	|*	|*	|*	|.
|*	|ć	|p	|e	|a	|*	|*	|*	|*	|[34][S]\rarr	|k	|a	|d	|i	|r	|i	|*	|z	|*	|*	|*	|*	|*	|.
|*	|*	|ł	|t	|s	|*	|*	|*	|*	|*	|[35][S]\rarr	|r	|a	|f	|a	|e	|l	|*	|*	|*	|*	|*	|*	|.
|*	|[36][S]\rarr	|a	|k	|t	|[][,]{ }	|u	|s	|t	|a	|w	|o	|d	|a	|w	|c	|z	|y	|*	|*	|*	|*	|*	|.
|*	|*	|t	|o	|i	|*	|*	|[37][S]\rarr	|u	|r	|z	|ą	|d	|[][,]{ }	|s	|k	|a	|r	|b	|o	|w	|y	|*	|.
|*	|*	|*	|w	|l	|*	|[38][S]\rarr	|g	|a	|r	|d	|e	|r	|o	|b	|i	|a	|n	|a	|*	|*	|*	|*	|.
|*	|*	|*	|a	|*	|*	|*	|*	|*	|*	|*	|*	|[39][S]\rarr	|g	|i	|e	|r	|y	|m	|s	|k	|i	|*	|.
|*	|*	|*	|*	|*	|*	|*	|*	|*	|[40][S]\rarr	|g	|ł	|o	|w	|a	|*	|*	|*	|*	|*	|*	|*	|*	|.\end{Puzzle}

\newpage

\begin{PuzzleClues}{\textbf{Poziome}\\}\Clue{1}{}{MAJKA LEKARSKA; metalicznie zielony chrząszcz}
\Clue{2}{}{czynszownik na prawie emfiteuzy}
\Clue{3}{}{Blechnum gibbum - gatunek paproci należący do rodziny podrzeniowatych (Blechnaceae); pochodzi z wysp Oceanu Spokojnego: Nowej Kaledonii oraz Fidżi}
\Clue{4}{}{Dolomedes plantarius - gatunek dużego europejskiego pająka z rodziny darownikowatych (Pisauridae)}
\Clue{5}{}{gatunek chleba najczęściej żytniego, który powstaje z mąki razowej (grubo mielonej i najsłabiej oczyszczonej)}
\Clue{6}{}{status społeczny osoby lub grupy osób (np. rodziny) określany na podstawie wysokości zarobków i ilości posiadanych dóbr}
\Clue{7}{}{towarzysz cierpień, ten, kto cierpi razem z kimś}
\Clue{9}{}{narciarz, brązowy medalista mistrzostw świata z 1974 r. w biegu na 30 km}
\Clue{10}{}{miejsce, gdzie według mitologii greckiej odbywają przechadzki cienie zmarłych}
\Clue{11}{}{Xenopirostris polleni - gatunek ptaka z rodziny wang (Vangidae)}
\Clue{12}{}{urządzenie optyczne, które służy do oglądania przezroczy}
\Clue{13}{}{torfowiec brunatny, Sphagnum fuscum - gatunek mszaka z rodziny torfowcowatych; rozpowszechniony na półkuli północnej, dość pospolity na terenie Polski}
\Clue{17}{}{gwiazda w gwiazdozbiorze Kilu}
\Clue{20}{}{koralniki, Callaeidae - rodzina ptaków z rzędu wróblowych, obejmująca kilka gatunków ptaków, występujących wyłącznie w Nowej Zelandii}
\Clue{23}{}{narciarz japoński, mistrz olimpijski w kombinacji norweskiej z Albertville i Lillehammer}
\Clue{24}{}{niewielka ryba ze stynkowatych żyjąca w stadach, stanowiąca pokarm drapieżnych ryb}
\Clue{25}{}{mieszkanka Tajwanu, kobieta pochodzenia tajwańskiego}
\Clue{26}{}{urządzenie służące do gotowania jajek}
\Clue{29}{}{grupa funkcyjna, powstała przez oderwanie atomu wodoru od cząsteczki metanu, o wzorze -CH3}
\Clue{31}{}{LOTKI}
\Clue{32}{}{środek chemiczny służący do wybielania}
\Clue{33}{}{cienka, gęsta tkanina bawełniana o splocie płóciennym, podobna do batystu, bielona albo drukowana, używana na bieliznę lub sukienki}
\Clue{34}{}{pisarz uzbecki (1896-1939), pierwsze nowele w literaturze uzbeckiej, powieści, sztuki teatralne}
\Clue{35}{}{włoski malarz i architekt, najmłodszy z trójki genialnych artystów włoskiego renesansu, znany z licznych przedstawień Madonny}
\Clue{36}{}{akt prawny, który został przyjęty w drodze procedury ustawodawczej}
\Clue{37}{}{państwowa jednostka budżetowa obsługująca naczelnika urzędu skarbowego, który jest organem administracji niezespolonej w terenie podlegającym Ministrowi Finansów, a zarazem organem podatkowym pierwszej instancji}
\Clue{38}{}{służąca mająca pieczę nad garderobą i pomagająca pani w ubieraniu się}
\Clue{39}{}{zbiór gipsowych odlewów rzeźb antycznych}
\Clue{40}{}{antropometryczna miara wysokości, stosowana również w przypadku zwierząt}\end{PuzzleClues}

\begin{PuzzleClues}{\textbf{Pionowe}\\}\Clue{1}{}{pierwszy egzemplarz druku, który jest przesyłany przez drukarnię redakcji w celu ostatecznego sprawdzenia i zatwierdzenia przed rozpoczęciem rozpowszechniania nakładu}
\Clue{5}{}{liczba moli danej substancji chemicznej jaką zawiera 1 dm3 roztworu}
\Clue{8}{}{dawniej wyznawana religia, wyznanie}
\Clue{13}{}{powieść o bardzo małej objętości; dłuższe opowiadanie}
\Clue{14}{}{materiał literacki będący podstawą realizacji fabuły filmowej lub teatralnej, zawierający dialogi oraz opis postaci i miejsc}
\Clue{15}{}{przyrząd w gimnastyce artystycznej}
\Clue{16}{}{osoba stosująca dietę wegańską, niespożywająca produktów zwierzęcych}
\Clue{18}{}{średniowieczne prawo miejskie, wzorowane na prawie Magdeburga}
\Clue{19}{}{rodzaj grubych, zwykle wełnianych rajstop bez stóp}
\Clue{21}{}{mały żaglowiec towarowy używany na przełomie XIX i XX w}
\Clue{22}{}{możliwość odliczenia od dochodu odsetek naliczanych z tytułu kredytu zaciągniętego na realizację własnych celów mieszkaniowych}
\Clue{23}{}{PILOT; osoba odpowiedzialna za kierowanie statkiem kosmicznym}
\Clue{27}{}{u owadów człon nogi łączący biodro z udem}
\Clue{28}{}{taniec klasyczny}
\Clue{30}{}{aerodyna zdolna do lotu dzięki powstawaniu siły nośnej na wirujących powierzchniach nośnych}
\Clue{31}{}{dość puszysta przędza uzyskiwana w wyniku modyfikacji stilonu}\end{PuzzleClues}\newpage\section*{Krzyżówka 16}

\noindent\begin{Puzzle}{19}{31}|*	|*	|*	|*	|*	|*	|[1][S]\drarr	|d	|e	|j	|n	|e	|k	|a	|*	|*	|*	|*	|*	|*	|.
|*	|*	|*	|[2][S]\darr	|*	|*	|f	|*	|*	|*	|*	|[3][S]\drarr	|s	|z	|c	|z	|a	|p	|a	|*	|.
|*	|*	|*	|o	|[4][S]\drarr	|k	|i	|e	|ł	|ż	|[][,]{ }	|b	|r	|z	|e	|g	|o	|w	|y	|*	|.
|*	|*	|*	|k	|h	|*	|l	|[5][S]\darr	|*	|[6][S]\darr	|*	|y	|*	|[7][S]\darr	|[8][S]\darr	|[9][S]\darr	|*	|*	|*	|*	|.
|*	|[10][S]\darr	|[11][S]\rarr	|r	|e	|k	|a	|p	|i	|t	|u	|l	|a	|c	|j	|a	|*	|[12][S]\darr	|*	|*	|.
|*	|d	|[13][S]\darr	|e	|l	|*	|n	|u	|*	|w	|*	|i	|*	|e	|ę	|d	|*	|p	|*	|*	|.
|*	|o	|c	|s	|m	|[14][S]\darr	|d	|n	|*	|a	|*	|c	|*	|o	|z	|i	|*	|r	|*	|*	|.
|*	|c	|h	|[][,]{ }	|o	|p	|e	|a	|[15][S]\darr	|r	|*	|a	|[16][S]\darr	|w	|y	|u	|[17][S]\darr	|a	|*	|*	|.
|*	|h	|o	|l	|n	|i	|r	|*	|s	|z	|*	|[][,]{ }	|b	|n	|k	|n	|l	|c	|*	|*	|.
|*	|o	|r	|i	|d	|e	|[][,]{ }	|*	|i	|y	|*	|m	|a	|i	|[][,]{ }	|k	|a	|e	|*	|*	|.
|*	|d	|o	|t	|*	|r	|r	|*	|ł	|c	|*	|l	|b	|k	|a	|t	|t	|[][,]{ }	|*	|*	|.
|*	|z	|b	|e	|*	|w	|u	|*	|a	|z	|*	|e	|i	|*	|r	|*	|o	|u	|*	|*	|.
|*	|e	|a	|r	|*	|o	|d	|*	|[][,]{ }	|k	|[18][S]\darr	|c	|o	|*	|a	|*	|l	|s	|*	|*	|.
|*	|n	|[][,]{ }	|a	|*	|t	|a	|*	|d	|a	|k	|z	|g	|*	|m	|*	|i	|t	|*	|*	|.
|*	|i	|d	|c	|[19][S]\darr	|k	|w	|[20][S]\darr	|o	|*	|o	|n	|ó	|[21][S]\darr	|e	|*	|s	|a	|*	|*	|.
|*	|o	|w	|k	|d	|o	|y	|n	|ś	|*	|s	|o	|r	|p	|j	|*	|t	|w	|*	|*	|.
|*	|w	|o	|i	|o	|w	|*	|y	|r	|*	|t	|b	|k	|s	|s	|[22][S]\darr	|e	|o	|*	|*	|.
|*	|i	|r	|*	|b	|c	|*	|s	|o	|*	|u	|i	|a	|y	|k	|p	|k	|d	|*	|*	|.
|[23][S]\drarr	|e	|s	|t	|r	|e	|h	|a	|d	|u	|r	|a	|*	|l	|i	|r	|[][,]{ }	|a	|*	|*	|.
|p	|c	|k	|*	|o	|*	|*	|*	|k	|*	|*	|ł	|*	|o	|*	|z	|b	|w	|*	|*	|.
|a	|*	|a	|*	|[][,]{ }	|*	|[24][S]\rarr	|k	|o	|l	|k	|a	|*	|t	|[25][S]\darr	|y	|l	|c	|*	|*	|.
|l	|*	|*	|*	|m	|[26][S]\rarr	|n	|o	|w	|a	|k	|*	|*	|o	|m	|g	|a	|z	|*	|*	|.
|m	|*	|*	|[27][S]\drarr	|a	|l	|i	|*	|a	|*	|*	|[28][S]\darr	|*	|w	|o	|o	|d	|e	|*	|*	|.
|a	|*	|*	|r	|t	|*	|*	|*	|*	|*	|[29][S]\rarr	|p	|l	|e	|n	|t	|y	|*	|*	|*	|.
|*	|[30][S]\rarr	|c	|z	|e	|r	|k	|i	|e	|s	|k	|a	|*	|*	|i	|o	|*	|*	|*	|*	|.
|[31][S]\drarr	|m	|i	|e	|r	|z	|y	|n	|e	|k	|*	|l	|*	|*	|u	|w	|*	|*	|*	|*	|.
|o	|*	|*	|k	|i	|*	|*	|*	|[32][S]\darr	|*	|*	|m	|*	|*	|s	|a	|*	|*	|*	|*	|.
|d	|*	|*	|o	|a	|*	|*	|*	|b	|*	|*	|i	|*	|*	|z	|l	|*	|*	|*	|*	|.
|s	|*	|*	|t	|l	|[33][S]\rarr	|s	|p	|u	|s	|t	|*	|*	|*	|k	|n	|*	|*	|*	|*	|.
|u	|*	|*	|k	|n	|*	|*	|*	|g	|*	|*	|*	|*	|*	|o	|i	|*	|*	|*	|*	|.
|w	|*	|*	|a	|e	|*	|[34][S]\rarr	|m	|a	|t	|r	|y	|c	|a	|*	|a	|*	|*	|*	|*	|.
|*	|*	|*	|*	|*	|*	|*	|*	|*	|[35][S]\rarr	|k	|l	|o	|s	|z	|*	|*	|*	|*	|*	|.\end{Puzzle}

\newpage

\begin{PuzzleClues}{\textbf{Poziome}\\}\Clue{1}{}{francuski malarz, grafik i rzeźbiarz (1808-79) przedstawiciel realizmu; 'Wagon trzeciej klasy'}
\Clue{3}{}{bardzo chudy człowiek lub chude zwierzę}
\Clue{4}{}{Gammarus duebeni - morski gatunek skorupiaka z rzędu obunogów}
\Clue{11}{}{streszczenie}
\Clue{23}{}{kraina historyczna w środk. Portugalii, obszar 7,9 tyś. km2, główne miasto Lisbona}
\Clue{24}{}{ostry miejscowy ból promieniujący od organu wewnętrznego}
\Clue{26}{}{ur. 1930r, poeta i prozaik, wiersze powieści, opowiadania; „A jak królem, a jak katem będziesz”, „W jutrzni”, „Ślepe koło wyobraźni”}
\Clue{27}{}{kalif (600- 661), zięć Mahometa, mąż Fatimy, czczony przez szyitów}
\Clue{29}{}{zatoka Oceanu Spokojnego u wybrzeży Wyspy Północnej w Nowej Zelandii}
\Clue{30}{}{członkini grupy etnicznej z północno-zachodniego Kaukazu, obecnie zamieszkującej głównie Republikę Karaczajo-Czerkieską, gdzie Czerkiesi stanowią ok. 10\% ludności, oraz kilka wiosek Republiki Adygei}
\Clue{31}{}{MIERZYN}
\Clue{33}{}{element broni palnej: metalowy języczek uruchamiający mechanizm w celu oddania strzału}
\Clue{34}{}{forma stosowana w galwanoplastyce jako katoda}
\Clue{35}{}{osłona np. żarówki, która służy do rozpraszania światła lub nadawania mu koloru; często jest zrobiony ze szkła}\end{PuzzleClues}

\begin{PuzzleClues}{\textbf{Pionowe}\\}\Clue{1}{}{filander kosmaty, Lagorchestes hirsutus - torbacz z rodziny kangurowatych; głównym środowiskiem dla niego była pierwotnie pustynia Tanami, obecnie występuje także na wyspach Dorre i Bernier w Zatoce Rekina na zachodnim wybrzeżu Australii}
\Clue{2}{}{faza procesu rozwoju historii literatury, która daje się zamknąć w granicach czasowych}
\Clue{3}{}{Artemisia lactiflora - gatunek rośliny należący do rodziny astrowatych}
\Clue{4}{}{miasto w Holandii nad kanałem Zuid-Willem-Svaart, ważny ośrodek przemysłu włókienniczego}
\Clue{5}{}{miasto w Indiach (Maharasztra) duży ośrodek przemysłowy i naukowo-kulturalny}
\Clue{6}{}{pieszczotliwie o twarzy, buzi; zwłaszcza o dziecięcej lub drobnej}
\Clue{7}{}{kształtownik o przekroju zbliżonym do litery „C”}
\Clue{8}{}{język z grupy semickiej używany na Bliskim Wschodzie od II tysiąclecia p.n.e. do czasów dzisiejszych}
\Clue{9}{}{tytuł zawodowy nadawany pracownikom służby bibliotecznej, leśnictwa i muzealnictwa}
\Clue{10}{}{policjant pracujący w policji dochodzeniowej}
\Clue{12}{}{ogół czynności dokonywanych w parlamencie i poza nim w celu uchwalenia ustawy}
\Clue{13}{}{eufemistyczna nazwa, jaką określano kiłę}
\Clue{14}{}{Protococcales - rząd zielenic należący do klasy zielenic właściwych (Chlorophyceae)}
\Clue{15}{}{siła powodująca zakrzywianie toru ruchu ciała, skierowana wzdłuż normalnej (prostopadle) do toru, w stronę środka jego krzywizny}
\Clue{16}{}{góralka babiogórska, członkini grupy etnograficznej ludności polskiej, zamieszkującej północne stoki Babiej Góry}
\Clue{17}{}{Gonepteryx farinosa - gatunek motyla z rodziny bielinkowatych, występujący na Bałkanach, na Wyspach Jońskich i na wyspach wschodniej części Morza Egejskiego}
\Clue{18}{}{gruba laska służąca do podpierania się w podróży}
\Clue{19}{}{produkt materialny, który ma zaspokoić potrzeby człowieka - posiadacza}
\Clue{20}{}{miasto w woj. opolskim, w powiecie nyskim, siedziba gminy miejsko-wiejskiej Nysa}
\Clue{21}{}{Psilotopsida - grupa roślin o różnej randze w różnych systemach, stanowi najstarszą linię rozwojową w obrębie kladu Monilophyta, obejmującym współcześnie występujące rośliny zaliczane do skrzypów i paproci}
\Clue{22}{}{część zakładu produkcyjnego, gdzie przygotowuje się produkty do dalszego przetwarzania (np. przygotowalnia spożywcza, przygotowalnia poligraficzna)}
\Clue{23}{}{znak w kształcie drzewa lub liścia palmy, który był noszony dawniej przy różnego rodzaju odzieży, np. mundurach, czapkach uczniowskich itp}
\Clue{25}{}{kompozytor (1819-1872); twórca polskiej opery narodowej i liryki pieśniarskiej; opery 'Halka', 'Straszny dwór', 'Flis', 'Hrabina', 'Paria', balety, msze, litanie, kantaty}
\Clue{27}{}{niewielka żaba z rodziny rzekotkowatych, charakteryzująca się zgrubiałymi palcami tworzącymi rodzaj przyssawek}
\Clue{28}{}{miasto we Włoszech (Kalabria) u wybrzeży Morza Tyrreńskiego}
\Clue{31}{}{ruch oddalenia narzędzia od przedmiotu obrabianego na obrabiarce}
\Clue{32}{}{językoznawca litewski (1879-1924); autor słownika litewskiego, badacz języków bałtyckich}\end{PuzzleClues}\newpage\section*{Krzyżówka 17}

\noindent\begin{Puzzle}{24}{19}|*	|*	|*	|*	|*	|*	|*	|*	|[1][S]\drarr	|k	|o	|c	|h	|a	|n	|i	|e	|*	|*	|[2][S]\darr	|*	|[3][S]\darr	|*	|[4][S]\darr	|*	|.
|*	|[5][S]\rarr	|s	|z	|a	|r	|p	|a	|n	|k	|a	|*	|[6][S]\rarr	|p	|l	|a	|m	|i	|a	|k	|*	|k	|*	|m	|*	|.
|*	|[7][S]\darr	|[8][S]\darr	|*	|*	|*	|*	|[9][S]\darr	|o	|*	|*	|[10][S]\rarr	|h	|a	|l	|a	|[][,]{ }	|t	|a	|r	|g	|o	|w	|a	|*	|.
|[11][S]\drarr	|ś	|w	|i	|ę	|t	|o	|k	|r	|a	|d	|c	|z	|y	|n	|i	|*	|*	|*	|ó	|[12][S]\darr	|r	|[13][S]\darr	|k	|*	|.
|c	|c	|a	|*	|*	|[14][S]\drarr	|l	|a	|w	|e	|t	|a	|*	|[15][S]\rarr	|b	|u	|r	|d	|e	|l	|m	|a	|m	|a	|*	|.
|a	|i	|z	|[16][S]\darr	|*	|r	|[17][S]\rarr	|s	|e	|r	|e	|k	|*	|[18][S]\drarr	|w	|i	|ę	|ź	|*	|e	|a	|l	|ą	|r	|*	|.
|r	|g	|k	|k	|[19][S]\darr	|o	|[20][S]\darr	|z	|ż	|[21][S]\drarr	|b	|k	|*	|h	|[22][S]\drarr	|m	|e	|r	|*	|w	|t	|[][,]{ }	|t	|o	|*	|.
|r	|a	|a	|u	|z	|l	|m	|a	|k	|k	|[23][S]\darr	|[24][S]\rarr	|p	|y	|s	|k	|*	|*	|*	|i	|e	|s	|w	|n	|*	|.
|a	|*	|*	|l	|o	|a	|o	|*	|a	|o	|b	|*	|[25][S]\rarr	|d	|i	|e	|t	|a	|*	|e	|m	|z	|i	|*	|*	|.
|r	|[26][S]\drarr	|b	|i	|n	|d	|r	|a	|*	|r	|u	|*	|*	|r	|a	|*	|[27][S]\darr	|*	|*	|c	|a	|l	|k	|*	|*	|.
|a	|s	|*	|k	|n	|a	|d	|[28][S]\rarr	|s	|o	|l	|i	|d	|a	|r	|n	|o	|ś	|ć	|*	|t	|a	|*	|*	|*	|.
|*	|a	|*	|*	|*	|*	|w	|*	|*	|n	|l	|[29][S]\darr	|*	|z	|k	|*	|c	|*	|*	|*	|y	|c	|[30][S]\darr	|*	|*	|.
|[31][S]\drarr	|m	|i	|e	|s	|z	|a	|n	|k	|a	|*	|c	|*	|y	|o	|*	|z	|*	|*	|*	|k	|h	|f	|*	|*	|.
|r	|p	|[32][S]\drarr	|p	|a	|b	|*	|*	|*	|*	|*	|h	|*	|d	|s	|*	|e	|*	|*	|*	|a	|e	|a	|*	|*	|.
|y	|l	|f	|*	|*	|[33][S]\drarr	|s	|e	|t	|u	|b	|a	|l	|*	|ó	|[34][S]\rarr	|p	|r	|ą	|d	|*	|t	|l	|*	|*	|.
|b	|e	|a	|*	|*	|t	|*	|*	|[35][S]\rarr	|s	|i	|n	|g	|e	|l	|*	|*	|[36][S]\rarr	|j	|e	|a	|n	|s	|*	|*	|.
|y	|r	|l	|[37][S]\rarr	|k	|o	|ż	|u	|s	|z	|y	|s	|k	|o	|*	|*	|*	|*	|*	|*	|*	|y	|e	|*	|*	|.
|*	|*	|a	|[38][S]\rarr	|t	|n	|d	|*	|*	|[39][S]\rarr	|k	|o	|r	|a	|l	|o	|d	|z	|i	|ó	|b	|*	|t	|*	|*	|.
|*	|*	|*	|*	|*	|*	|[40][S]\rarr	|z	|a	|k	|o	|n	|n	|i	|c	|z	|e	|k	|*	|*	|*	|*	|*	|*	|*	|.
|[41][S]\rarr	|r	|o	|z	|k	|r	|u	|s	|z	|k	|i	|*	|*	|*	|*	|*	|*	|*	|*	|*	|*	|*	|*	|*	|*	|.\end{Puzzle}

\newpage

\begin{PuzzleClues}{\textbf{Poziome}\\}\Clue{1}{}{serdeczny, poufały zwrot do kogoś}
\Clue{5}{}{materiał do budowy gniazd dla ptaków i gryzoni, mający postać skrawków, pasków, nitek, wykonany najczęściej z tworzywa naturalnego (bawełny, włókien kosowa itp.)}
\Clue{6}{}{ŁUPACZ}
\Clue{10}{}{handlowy budynek, w którym znajdują się stoiska z branży spożywczej, przemysłowej i usługowej}
\Clue{11}{}{kobieta, która coś profanuje, bezcześci przedmioty i miejsca kultu lub znieważa powszechnie uznawane wartości}
\Clue{14}{}{podstawa broni palnej służąca do mocowania lufy i manewrowania działem}
\Clue{15}{}{właścicielka domu publicznego}
\Clue{17}{}{dekolt o trójątnym kształcie}
\Clue{18}{}{to, co jednoczy, zespala, łączy ludzi ze sobą}
\Clue{21}{}{w chemii: symbol berkelu}
\Clue{22}{}{najprostszy, jaki da się wyróżnić, stale powtarzający się fragment cząsteczek polimerów}
\Clue{24}{}{twarz człowieka}
\Clue{25}{}{produkty jedzone przez kogoś najczęściej, jadłospis}
\Clue{26}{}{stanowiąca element beczki klepka z drewna dębowego}
\Clue{28}{}{to, że ktoś jest solidarny w ramach jakiejś grupy, wobec kogoś; zgoda i wzajemne wsparcie w gronie osób, które współdziałają, bo łączą je wspólne interesy, poglądy, wartości, dążenia}
\Clue{31}{}{różne fizyczne obiekty (np. ciecze, gazy, nasiona) pomieszane celowo w określonych proporcjach, z myślą o praktycznym zastosowaniu; materia, substancja, która jest zrobiona z więcej niż jednego składnika}
\Clue{32}{}{kod ISO 4217 waluty balboa}
\Clue{33}{}{zatoka Oceanu Atlantyckiego u wybrzeży Portugalii, główny port Setubal}
\Clue{34}{}{ruch cieczy, zwłaszcza wody}
\Clue{35}{}{często mniejsza rozmiarem płyta, która zawiera jeden utwór, służy do promocji nowej, większej płyty}
\Clue{36}{}{astronom angielski (1877-1946), przepływowa teoria powstania Układu Słonecznego}
\Clue{37}{}{zgrubiale: kożuch - twarda, lekko sucha warstwa tworząca się na mleku}
\Clue{38}{}{kod ISO 4217 dinara tunezyjskiego}
\Clue{39}{}{Hypositta corallirostris - gatunek ptaka z rodziny wang (Vangidae)}
\Clue{40}{}{hodowlana rasa gołębia lotnego}
\Clue{41}{}{nazwa roztoczy, groźnych szkodników magazynów żywnościowych}\end{PuzzleClues}

\begin{PuzzleClues}{\textbf{Pionowe}\\}\Clue{1}{}{mieszkanka Norwegii, kobieta pochodzenia norweskiego}
\Clue{2}{}{stolica obwodu kaliningradzkiego - eksklawy Federacji Rosyjskiej, u ujścia Pregoły do Bałtyku, w historycznej krainie Sambii; obecnie Kaliningrad}
\Clue{3}{}{Corallium rubrum - gatunek koralowca z rodziny Coralliidae występujący w Morzu Adriatyckim i zachodnim Morzu Śródziemnym}
\Clue{4}{}{produkt żywnościowy wytwarzany na bazie ciasta z mąki, wody i niekiedy jaj, który miewa różne rozmiary i kształty, jest dość trwały: można go suszyć i przechowywać}
\Clue{7}{}{chrząszcz z rodziny kózek}
\Clue{8}{}{zawartość wazki, małej, ozdobnej wazy}
\Clue{9}{}{porcja kaszy - ziaren zbóż, które po zagotowaniu lub zalaniu wodą nadają się do spożycia; określona ilość tego produktu, zazwyczaj papierowa torebka lub kartonowe pudełko}
\Clue{11}{}{miasto we Włoszech (Toskania), słynny ośrodek wydobycia i obróbki marmurów}
\Clue{12}{}{nauka o liczbach, figurach i stosunkach przestrzennych, posługująca się metodą dedukcyjną}
\Clue{13}{}{drobny wicień pasożytujący w korzeniach roślin, szkodnik}
\Clue{14}{}{potrawa z mięsa zawiniętego w rulon z jakimś nadzieniem (mięso w roladzie rzadko bywa jednym kawałkiem, częściej jest ono mielone, wtedy rolada jest rodzajem pasztetu)}
\Clue{16}{}{kosmopolityczny ptak podmokłych terenów z rzędu mew siewek, dziób długi, łukowaty, w Polsce 3 gatunki, chronione}
\Clue{18}{}{organiczny związek chemiczny, otrzymywany w reakcji kwasów karboksylowych i ich pochodnych z hydrazyną}
\Clue{19}{}{astronom (1905-75), dyrektor obserwatorium Uniwersytetu Warszawskiego}
\Clue{20}{}{obecna Mordowia}
\Clue{21}{}{w muzyce: fermata}
\Clue{22}{}{tiosól - sól siarkowego odpowiednika kwasu karboksylowego, w którego cząsteczkach 1 lub 2 atomy tlenu zostały zastąpione atomami siarki}
\Clue{23}{}{bullowaty, pies wyhodowany poprzez mieszanie terierów z buldogami}
\Clue{26}{}{klawiszowy instrument muzyczny z grupy elektrofonów elektronicznych, odgrywający wcześniej nagrane próbki dźwięków (ang. samples) instrumentów akustycznych, innych dźwięków muzycznych lub niemuzycznych dźwięków wykorzystywanych w muzyce}
\Clue{27}{}{pozioma belka wiążąca górne końce słupów budowy}
\Clue{29}{}{obecnie francuska piosenka kabaretowa}
\Clue{30}{}{głos wydobywany z silnie napiętych wiązadeł głosowych, stosowany często w jodłowaniu - FISTUŁA}
\Clue{31}{}{PISCES; zodiakalny gwiazdozbiór równikowy, również znak Zodiaku}
\Clue{32}{}{pasmo włosów, które się skręca}
\Clue{33}{}{w muzyce: miara odległości pomiędzy dźwiękami w skali}\end{PuzzleClues}\newpage\section*{Krzyżówka 18}

\noindent\begin{Puzzle}{23}{27}|*	|*	|*	|*	|*	|*	|*	|*	|*	|*	|*	|*	|*	|*	|*	|[1][S]\drarr	|r	|a	|d	|a	|*	|*	|[2][S]\darr	|[3][S]\darr	|.
|*	|*	|*	|[4][S]\darr	|*	|[5][S]\darr	|*	|*	|*	|*	|*	|[6][S]\rarr	|m	|a	|n	|g	|o	|*	|[7][S]\darr	|[8][S]\darr	|*	|*	|t	|p	|.
|*	|*	|*	|d	|*	|s	|*	|*	|*	|[9][S]\drarr	|s	|a	|m	|o	|t	|o	|k	|*	|w	|p	|*	|[10][S]\darr	|r	|l	|.
|*	|*	|*	|y	|*	|i	|[11][S]\rarr	|s	|m	|u	|k	|l	|i	|k	|*	|l	|*	|*	|a	|r	|*	|a	|z	|u	|.
|[12][S]\rarr	|f	|e	|n	|o	|m	|e	|n	|o	|l	|o	|g	|i	|a	|*	|k	|[13][S]\darr	|[14][S]\darr	|g	|o	|[15][S]\darr	|n	|m	|j	|.
|*	|*	|*	|*	|*	|a	|*	|*	|*	|t	|[16][S]\drarr	|g	|s	|*	|*	|i	|n	|b	|a	|t	|p	|a	|i	|k	|.
|*	|*	|[17][S]\darr	|*	|[18][S]\darr	|*	|*	|*	|[19][S]\darr	|r	|c	|*	|*	|*	|*	|p	|u	|a	|n	|e	|e	|p	|e	|a	|.
|*	|*	|p	|*	|g	|*	|*	|*	|n	|a	|h	|*	|*	|[20][S]\drarr	|k	|e	|r	|s	|t	|s	|t	|o	|l	|*	|.
|*	|*	|o	|[21][S]\darr	|a	|*	|[22][S]\drarr	|c	|o	|b	|a	|n	|*	|k	|[23][S]\darr	|r	|n	|k	|*	|t	|r	|l	|a	|*	|.
|*	|[24][S]\darr	|k	|s	|r	|*	|m	|[25][S]\drarr	|w	|o	|ł	|y	|n	|o	|w	|*	|i	|i	|[26][S]\darr	|*	|e	|i	|k	|*	|.
|*	|f	|r	|z	|b	|*	|i	|s	|o	|o	|u	|[27][S]\darr	|*	|h	|ł	|[28][S]\darr	|c	|j	|p	|*	|l	|s	|*	|*	|.
|*	|a	|z	|t	|a	|*	|e	|y	|ś	|k	|p	|n	|*	|o	|o	|g	|z	|k	|r	|*	|[][,]{ }	|*	|*	|*	|.
|[29][S]\rarr	|l	|y	|o	|t	|*	|j	|n	|ć	|*	|n	|a	|*	|r	|m	|a	|e	|a	|o	|[30][S]\darr	|w	|*	|[31][S]\darr	|*	|.
|*	|a	|w	|r	|e	|*	|s	|e	|*	|*	|i	|g	|*	|t	|*	|z	|k	|*	|t	|h	|u	|*	|b	|*	|.
|*	|n	|o	|t	|k	|*	|c	|k	|*	|*	|c	|o	|*	|a	|[32][S]\darr	|e	|[][,]{ }	|*	|o	|[][S]ä	|l	|*	|o	|*	|.
|*	|s	|w	|*	|*	|*	|ó	|*	|*	|*	|z	|z	|[33][S]\darr	|*	|i	|l	|k	|[34][S]\darr	|g	|n	|k	|[35][S]\darr	|h	|*	|.
|*	|t	|a	|*	|*	|[36][S]\darr	|w	|*	|*	|*	|k	|a	|h	|[37][S]\darr	|z	|a	|r	|i	|a	|d	|a	|z	|r	|*	|.
|*	|e	|t	|*	|[38][S]\darr	|o	|k	|*	|[39][S]\rarr	|g	|a	|l	|a	|g	|o	|[][,]{ }	|a	|l	|l	|e	|n	|a	|*	|*	|.
|[40][S]\drarr	|r	|e	|m	|b	|r	|a	|n	|d	|t	|*	|ą	|i	|ę	|o	|a	|s	|o	|a	|l	|i	|j	|*	|*	|.
|c	|*	|*	|*	|y	|t	|*	|*	|*	|*	|*	|ż	|t	|ś	|k	|r	|n	|k	|k	|*	|c	|ą	|*	|*	|.
|h	|[41][S]\drarr	|l	|a	|s	|o	|w	|i	|a	|c	|z	|k	|a	|*	|t	|a	|o	|e	|t	|*	|z	|k	|[42][S]\darr	|*	|.
|o	|p	|*	|*	|z	|l	|*	|*	|*	|*	|*	|o	|ń	|*	|a	|b	|d	|l	|y	|*	|n	|n	|f	|*	|.
|l	|r	|[43][S]\rarr	|r	|e	|a	|l	|*	|*	|*	|*	|w	|c	|*	|n	|s	|z	|e	|k	|*	|y	|i	|r	|*	|.
|e	|ó	|*	|*	|w	|n	|*	|*	|*	|*	|*	|e	|z	|*	|*	|k	|i	|z	|a	|*	|*	|ę	|a	|*	|.
|r	|g	|*	|*	|o	|*	|*	|*	|*	|*	|*	|*	|y	|*	|*	|a	|o	|j	|*	|*	|*	|c	|n	|*	|.
|a	|*	|*	|*	|*	|[44][S]\rarr	|k	|o	|l	|p	|o	|s	|k	|o	|p	|*	|b	|a	|*	|[45][S]\rarr	|p	|i	|k	|*	|.
|*	|*	|*	|[46][S]\rarr	|g	|ł	|o	|w	|i	|z	|n	|a	|*	|*	|*	|*	|y	|*	|*	|*	|*	|e	|o	|*	|.
|[47][S]\rarr	|t	|r	|a	|n	|s	|p	|a	|r	|e	|n	|t	|n	|o	|ś	|ć	|*	|*	|*	|*	|*	|*	|*	|*	|.\end{Puzzle}

\newpage

\begin{PuzzleClues}{\textbf{Poziome}\\}\Clue{1}{}{lokalne zgromadzenie robotnicze, chłopskie i żołnierskie, na początku dwudziestego wieku, w okresie ostatnich lat dynastii Romanowów}
\Clue{6}{}{jadalny owoc (pestkowiec) mango indyjskiego}
\Clue{9}{}{wałkowy lub krążkowy przenośnik hutniczy}
\Clue{11}{}{Halictus - rodzaj pszczoły z rodziny smuklikowatych (Halictidae) i podrodziny smuklikowatych właściwych (Halictinae); zakłada gniazda w ziemi, a także zapyla rośliny uprawne}
\Clue{12}{}{XX-wieczny kierunek filozoficzny, którego głównymi twórcami i reprezentantami są Edmund Husserl oraz Max Scheler (w dużej mierze niezależny od Husserla), a także wytworzona przez ten kierunek metoda badań filozoficznych, stosowana przez filozofów egzystencji}
\Clue{16}{}{Gminna SpółdzielniaSamopomoc Chłopska - spółdzielnia produkcyjno-handlowo-usługowa istniejąca w większości gmin wiejskich i miejsko-wiejskich}
\Clue{20}{}{rodzaj słodkiego chleba spożywanego tradycyjnie podczas świąt Bożego Narodzenia głównie w Holandii i Niemczech}
\Clue{22}{}{miasto w środkowej Gwatemali, ośrodek administracyjny departamentu Alta Verapaz}
\Clue{25}{}{kosmonauta radziecki, uczestnik lotu w 1969r. podczas którego doszło do połączenia statków}
\Clue{29}{}{astronom francuski (1879-1952), wynalazca koronografu}
\Clue{39}{}{Galago alleni - ssak z rodziny galagowatych, występujący w lasach deszczowych centralno-zachodniej Afryki}
\Clue{40}{}{twórczość artystyczna Rembrandta}
\Clue{41}{}{przedstawicielka grupy etnograficznej zamieszkującej głównie Równinę Tarnobrzeską i Płaskowyż Kolbuszowski; w węższym ujęciu widły Wisły i Sanu, w szerszym także prawy brzeg Sanu}
\Clue{43}{}{historyczna moneta (głównie srebrna) bita od XIV wieku w Hiszpanii, a następnie w Portugalii}
\Clue{44}{}{endoskop służący do wziernikowania powierzchni szyjki macicy, dolnej części jej kanału oraz pochwy i sromu}
\Clue{45}{}{szczyt na wykresie, ekstremum fali}
\Clue{46}{}{mięso otrzymywane po oddzieleniu od głowy zwierzęcia: mózgu, uszu, języka, oczu}
\Clue{47}{}{cecha substancji (najczęściej kosmetyku), która rozświetla, rozjaśnia podłoże}\end{PuzzleClues}

\begin{PuzzleClues}{\textbf{Pionowe}\\}\Clue{1}{}{bramkarz - członek zespołu, którego zadaniem jest uniemożliwienie zdobycia bramki (gola) zawodnikom drużyny przeciwnej}
\Clue{2}{}{ptak z rodziny kolibrów}
\Clue{3}{}{herbata lub kawa z fusami, podana mało elegancko, np. w szklance}
\Clue{4}{}{jednostka siły, pochodna w układzie miar CGS; jest to siła nadająca ciału o masie 1 grama przyspieszenie równe 1 cm/s2}
\Clue{5}{}{czeski malarz i grafik (1891-1971) kompozycje o tematyce mitycznej, portrety, ilustracje}
\Clue{7}{}{średniowieczny śpiewak, aktor, poeta wędrujący w większej grupie}
\Clue{8}{}{sprzeciw, postawa wyrażająca brak zgody na coś, opowiadanie się przeciwko czemuś}
\Clue{9}{}{odmiana notebooka, która ma być lekka, bardzo wydajna energetycznie, niedroga, najczęściej wyposażona w ekran dotykowy}
\Clue{10}{}{miasto w Brazylii (Goias); przemysł spożywczy}
\Clue{13}{}{Aethia psittacula - gatunek ptaka morskiego z rodziny alk (Alcidae)}
\Clue{14}{}{płaskie, okrągłe nakrycie głowy bez daszka, beret baskijski}
\Clue{15}{}{Pterodroma baraui - gatunek ptaka z rodziny burzykowatych (Procellariidae); endemiczny dla wyspy Reunion}
\Clue{16}{}{kobieta bezrolna, która miała obowiązek odrabiać pańszczyznę}
\Clue{17}{}{Urticaceae - rodzina roślin zielnych z rzędu różowców (Rosales); należą tu 54 rodzaje występujące na obszarach międzyzwrotnikowych oraz o klimacie umiarkowanym i zimnym}
\Clue{18}{}{ŁOKAŚ}
\Clue{19}{}{coś nowego, coś, czego wcześniej nie było}
\Clue{20}{}{gromadza, grupa, banda ludzi ustawionych w określonym szyku, najczęściej uzbrojonych i zachowujących się agresywnine}
\Clue{21}{}{dawny dęty instrument muzyczny, prototyp fagotu, rozpowszechniony głównie w XVI w}
\Clue{22}{}{bilet z zarezerwowanym konkretnym miejscem, np. bilet na pociąg, który oprócz możliwości przejazdu zapewnia także rezerwację miejsca siedzącego}
\Clue{23}{}{w górnictwie: wyrwa w caliźnie wykonana robotami strzelniczymi}
\Clue{24}{}{projekt wspólnoty równych i wolnych ludzi zarządzającej własnym miastem}
\Clue{25}{}{chłopiec, młody mężczyzna; określenie pieszczotliwe, żartobliwe, ironiczne, najczęściej w zwrocie do tego chłopca}
\Clue{26}{}{galaktyka w początkowym stadium ewolucji}
\Clue{27}{}{nagonasienne, Gymnospermae - jedna z dwóch obok okrytonasiennych grup siostrzanych współczesnych roślin nasiennych; do nagozalążkowych należy ok. 700 gatunków drzew i krzewów}
\Clue{28}{}{Gazella gazella gazella - podgatunek gazeli górskiej, ssaka z rodziny krętorogich; występuje w Izraelu}
\Clue{30}{}{niemiecki kompozytor i organista (1685-1759); opery w stylu włoskim ('Julius Cezar') utwory orkiestrowe, organowe, klawesynowe, kameralne; jeden z największych kompozytorów baroku}
\Clue{31}{}{fizyk duński (1885-1962); jeden z twórców teorii kwantów, laureat nagrody Nobla}
\Clue{32}{}{izomer oktanu}
\Clue{33}{}{mieszkaniec Haiti, człowiek pochodzenia haitańskiego}
\Clue{34}{}{Ilokelesia - rodzaj teropoda z rodziny abelizaurów; żył w okresie późnej kredy na terenach Ameryki Południowej}
\Clue{35}{}{oznaka braku pewności, zawahanie, coś, co może wydać czyjeś prawdziwe emocje podczas mówienia}
\Clue{36}{}{chroniony gatunek trznadla; Eurazja}
\Clue{37}{}{idiotka, głupia, naiwna kobieta}
\Clue{38}{}{wieś w Polsce położona w województwie kujawsko-pomorskim, w powiecie bydgoskim, w gminie Koronowo}
\Clue{40}{}{złość, wściekłość}
\Clue{41}{}{przenośnie coś, co stanowi przeszkodę, symbolicznie dzieli}
\Clue{42}{}{(1856-1916), ukraiński pisarz i działacz społeczny; „Lasy i pastwiska”, „Zachar Berkut”, pisał także po polsku}\end{PuzzleClues}\newpage\section*{Krzyżówka 19}

\noindent\begin{Puzzle}{19}{25}|*	|*	|*	|*	|*	|*	|*	|*	|*	|[1][S]\drarr	|f	|l	|o	|k	|u	|ł	|y	|*	|[2][S]\darr	|[3][S]\darr	|.
|*	|*	|[4][S]\darr	|[5][S]\darr	|*	|*	|[6][S]\drarr	|k	|o	|s	|t	|u	|r	|e	|k	|*	|*	|*	|u	|t	|.
|*	|[7][S]\darr	|t	|b	|*	|*	|g	|*	|[8][S]\darr	|i	|*	|*	|*	|*	|[9][S]\drarr	|s	|e	|*	|c	|a	|.
|*	|a	|e	|e	|*	|*	|i	|*	|g	|l	|*	|[10][S]\rarr	|k	|a	|k	|a	|o	|*	|z	|m	|.
|[11][S]\rarr	|p	|l	|a	|t	|e	|r	|*	|u	|n	|[12][S]\darr	|[13][S]\rarr	|u	|m	|o	|w	|a	|*	|e	|i	|.
|*	|r	|i	|u	|*	|*	|o	|*	|s	|i	|m	|*	|[14][S]\drarr	|g	|r	|a	|d	|*	|l	|l	|.
|*	|i	|g	|v	|[15][S]\darr	|[16][S]\drarr	|s	|e	|t	|k	|a	|*	|k	|[17][S]\darr	|z	|[18][S]\darr	|[19][S]\darr	|*	|n	|k	|.
|*	|o	|a	|a	|w	|k	|t	|*	|r	|[][,]{ }	|w	|*	|o	|p	|y	|p	|g	|[20][S]\darr	|i	|a	|.
|*	|r	|*	|i	|y	|l	|a	|[21][S]\darr	|o	|p	|s	|[22][S]\drarr	|p	|o	|s	|ł	|o	|w	|a	|*	|.
|*	|y	|*	|s	|s	|a	|t	|m	|w	|n	|o	|t	|i	|d	|t	|a	|t	|y	|[][,]{ }	|*	|.
|*	|c	|*	|*	|t	|d	|*	|e	|*	|e	|n	|a	|d	|r	|n	|k	|t	|s	|a	|*	|.
|*	|z	|*	|*	|ę	|o	|*	|s	|*	|u	|*	|n	|ó	|y	|o	|s	|l	|p	|k	|*	|.
|*	|n	|*	|*	|p	|f	|*	|s	|*	|m	|[23][S]\darr	|d	|ł	|w	|ś	|a	|i	|y	|a	|*	|.
|*	|o	|*	|*	|n	|o	|*	|i	|*	|a	|ł	|e	|*	|k	|ć	|*	|e	|[][,]{ }	|d	|*	|.
|*	|ś	|[24][S]\darr	|[25][S]\rarr	|o	|r	|i	|e	|n	|t	|a	|c	|j	|a	|*	|*	|b	|s	|e	|*	|.
|*	|ć	|d	|*	|ś	|a	|[26][S]\darr	|r	|*	|y	|ł	|i	|*	|*	|[27][S]\darr	|[28][S]\darr	|*	|a	|m	|*	|.
|*	|*	|a	|*	|ć	|*	|t	|*	|[29][S]\darr	|c	|o	|a	|*	|[30][S]\darr	|c	|k	|*	|m	|i	|*	|.
|*	|*	|l	|*	|*	|*	|e	|*	|k	|z	|k	|r	|*	|s	|h	|o	|*	|o	|c	|*	|.
|*	|[31][S]\rarr	|e	|c	|h	|n	|a	|t	|o	|n	|*	|z	|*	|ł	|l	|m	|*	|a	|k	|*	|.
|*	|*	|*	|[32][S]\darr	|[33][S]\darr	|*	|t	|*	|l	|y	|*	|*	|*	|o	|o	|u	|*	|*	|a	|*	|.
|[34][S]\drarr	|w	|e	|c	|k	|e	|r	|*	|e	|*	|*	|*	|*	|m	|r	|n	|*	|*	|*	|*	|.
|m	|*	|*	|s	|o	|*	|o	|[35][S]\rarr	|b	|e	|ł	|t	|*	|i	|y	|i	|*	|*	|*	|*	|.
|m	|*	|*	|*	|p	|*	|m	|*	|k	|[36][S]\rarr	|i	|g	|u	|a	|n	|a	|*	|*	|*	|*	|.
|*	|*	|*	|*	|e	|[37][S]\rarr	|a	|k	|a	|d	|e	|m	|i	|k	|*	|*	|*	|*	|*	|*	|.
|[38][S]\rarr	|m	|e	|e	|r	|a	|n	|e	|*	|*	|*	|*	|*	|*	|*	|*	|*	|*	|*	|*	|.
|*	|*	|*	|*	|*	|*	|*	|*	|*	|*	|*	|*	|*	|*	|*	|*	|*	|*	|*	|*	|.\end{Puzzle}

\newpage

\begin{PuzzleClues}{\textbf{Poziome}\\}\Clue{1}{}{jasne i ciemne strumienie gazów w atmosferze słonecznej}
\Clue{6}{}{kij służący do podpierania się}
\Clue{9}{}{w chemii: symbol selenu}
\Clue{10}{}{napój z proszku kakaowego przyrządzany na wodzie lub mleku}
\Clue{11}{}{sztuciec powlekany srebrem lub złotem}
\Clue{13}{}{pojęcie prawne, które przeszło do języka ogólnego}
\Clue{14}{}{jednostka miary kąta płaskiego równa 1/100 kąta prostego}
\Clue{16}{}{nagranie, podczas którego jednocześnie nagrywa się obraz i dźwięk}
\Clue{22}{}{żona posła}
\Clue{25}{}{położenie czegoś względem punktu odniesienia (zwłaszcza w przestrzeni i przy wzięciu pod uwagę wektorów kierunkowych)}
\Clue{31}{}{władca starożytnego Egiptu (faraon) z XVIII dynastii, syn Amenhotepa III i królowej Teje, autor religijnej reformy amarneńskiej}
\Clue{34}{}{gimnastyk niemiecki, złoty medalista z Atlanty w ćwiczeniach na drążku}
\Clue{35}{}{substancja płynna lub półpłynna, kleista, często powtała ze zmieszania innych substancji, odbierana jako nieprzyjemna}
\Clue{36}{}{LEGWAN; jaszczurka o skórze pokrytej łuskami, ceniona ze względu na jaja i mięso, żyje w lasach Ameryki}
\Clue{37}{}{dom studencki}
\Clue{38}{}{miasto w Niemczech (Saksonia) na przedgórzu Rudaw; przemysł włókienniczy, obuwniczy, maszynowy}\end{PuzzleClues}

\begin{PuzzleClues}{\textbf{Pionowe}\\}\Clue{1}{}{maszyna pneumatyczna, przetwarzająca energię sprężonego powietrza lub innego gazu na ruch obrotowy lub postępowy}
\Clue{2}{}{uczelnia, w której przynajmniej jedna jednostka organizacyjna posiada uprawnienie do nadawania stopnia naukowego doktora}
\Clue{3}{}{członkini ludu Tamilów, narodu z grupy ludów drawidyjskich}
\Clue{4}{}{(1919-1970) żeglarz, lotnik, dziennikarz; na jachcie 'OPTY' odbył samotny rejs dookoła świata}
\Clue{5}{}{miasto w płn. Francji (Basen Paryski), ośrodek administracyjny departamentu Ois}
\Clue{6}{}{przyrząd zwiększający równowagę statku}
\Clue{7}{}{właściwość tego, co aprioryczne}
\Clue{8}{}{miasto w Niemczech (Meklemburgia) na Pojezierzu Meklemburskim; przemysł spożywczy drzewny, odzieżowy}
\Clue{9}{}{to, że coś ma pozytywny wpływ na coś lub kogoś innego}
\Clue{12}{}{austriacki badacz polarny (1882-1958); pierwszy dotarł do bieguna magnetycznego}
\Clue{14}{}{kopacz}
\Clue{15}{}{cecha złego, niezgodnego z prawem postępowania}
\Clue{16}{}{GAŁĘZATKA}
\Clue{17}{}{sieć, która ma kształt wydłużonego stożka i jest przeznaczona do połowów ryb przy sztucznym oświetleniu}
\Clue{18}{}{KAPUCYNKA}
\Clue{19}{}{Maurycy, brat Leopolda (1856-78) obrazy o tematyce żydowskiej}
\Clue{20}{}{archipelag złożony z 14 wysp w Polinezji na Oceanie Spokojnym}
\Clue{21}{}{astronom francuski (1730-1817), opracował pierwszy katalog obiektów mgławicowych}
\Clue{22}{}{handlarz tandetą - niewielkimi produktami o niskiej jakości}
\Clue{23}{}{obwisła skóra na szyi żubra w gwarze łowieckiej}
\Clue{24}{}{fizjolog i biochemik angielski (1875-1968); odkrywca zjawisk chemicznych związanych z przekazywaniem impulsów nerwowych, laureat nagrody Nobla}
\Clue{26}{}{osoba zafascynowana teatrem, często bywająca na spektaklach teatralnych}
\Clue{27}{}{sól kwasu chlorowego (III); jest środkiem bielącym używanym np. do wybielania papieru}
\Clue{28}{}{w Kościele katolickim sakrament; akt spożycia Hostii}
\Clue{29}{}{pojazd konny rozpowszechniony wśród warstw wyższych zwłaszcza w XVI wieku; jego podstawowym elementem było zawieszone na rzemiennych pasach pudło, w którym zasiadali podróżni}
\Clue{30}{}{ul słomiany}
\Clue{32}{}{w chemii: symbol cezu}
\Clue{33}{}{miasto i port w Słowenii nad Morzem Adriatyckim}
\Clue{34}{}{mila morska - jednostka odległości stosowana w nawigacji morskiej oraz lotnictwie; jednej mili morskiej odpowiadają 1852 metry, czyli uśredniona długość łuku południka ziemskiego odpowiadająca jednej minucie kątowej koła wielkiego}\end{PuzzleClues}\newpage\section*{Krzyżówka 20}

\noindent\begin{Puzzle}{22}{33}|*	|*	|*	|*	|*	|*	|*	|*	|*	|*	|*	|*	|*	|*	|*	|*	|*	|*	|[1][S]\darr	|*	|*	|*	|*	|.
|*	|*	|*	|[2][S]\drarr	|o	|w	|e	|n	|*	|[3][S]\darr	|*	|*	|*	|[4][S]\drarr	|d	|i	|u	|s	|z	|e	|s	|a	|*	|.
|*	|*	|*	|g	|*	|[5][S]\drarr	|p	|o	|e	|m	|a	|t	|*	|b	|[6][S]\darr	|*	|*	|[7][S]\darr	|i	|*	|[8][S]\darr	|*	|*	|.
|*	|*	|*	|ł	|[9][S]\drarr	|s	|e	|t	|b	|o	|l	|*	|*	|i	|k	|[10][S]\darr	|*	|m	|e	|*	|r	|*	|*	|.
|*	|*	|*	|o	|r	|z	|[11][S]\rarr	|t	|o	|n	|*	|*	|*	|e	|o	|c	|*	|a	|m	|*	|y	|*	|*	|.
|*	|*	|*	|s	|y	|o	|*	|[12][S]\rarr	|b	|o	|l	|a	|*	|d	|l	|e	|*	|t	|i	|*	|n	|*	|*	|.
|*	|*	|[13][S]\darr	|k	|n	|r	|[14][S]\rarr	|h	|u	|m	|o	|r	|*	|n	|i	|g	|*	|c	|a	|*	|e	|*	|*	|.
|*	|[15][S]\darr	|k	|a	|e	|t	|*	|*	|[16][S]\darr	|e	|*	|*	|*	|i	|n	|l	|*	|z	|n	|*	|k	|*	|*	|.
|*	|b	|o	|[][,]{ }	|k	|y	|*	|[17][S]\rarr	|b	|r	|y	|j	|k	|a	|*	|a	|*	|y	|k	|*	|[][,]{ }	|*	|[18][S]\darr	|.
|*	|a	|s	|e	|[][,]{ }	|*	|*	|*	|l	|*	|*	|*	|*	|k	|*	|r	|*	|s	|a	|*	|f	|[19][S]\darr	|l	|.
|*	|w	|t	|k	|p	|[20][S]\rarr	|b	|l	|a	|d	|o	|ś	|ć	|*	|*	|s	|*	|k	|*	|[21][S]\darr	|i	|m	|a	|.
|*	|ó	|i	|s	|i	|[22][S]\drarr	|l	|i	|c	|o	|*	|*	|*	|*	|*	|t	|*	|o	|*	|a	|n	|u	|t	|.
|[23][S]\drarr	|ł	|u	|p	|e	|k	|[][,]{ }	|c	|h	|l	|o	|r	|y	|t	|o	|w	|y	|*	|*	|p	|a	|z	|a	|.
|k	|*	|m	|i	|n	|a	|*	|*	|ó	|*	|*	|*	|[24][S]\drarr	|o	|s	|o	|b	|a	|*	|o	|n	|u	|r	|.
|a	|*	|*	|r	|i	|m	|*	|*	|w	|*	|*	|*	|l	|[25][S]\darr	|*	|*	|*	|*	|*	|s	|s	|ł	|k	|.
|c	|*	|[26][S]\darr	|a	|ę	|i	|*	|*	|k	|*	|[27][S]\rarr	|b	|o	|c	|z	|e	|k	|*	|*	|t	|o	|m	|a	|.
|*	|*	|c	|c	|ż	|o	|[28][S]\rarr	|m	|a	|c	|i	|e	|r	|z	|[][,]{ }	|d	|y	|s	|k	|o	|w	|a	|*	|.
|[29][S]\drarr	|c	|z	|y	|n	|n	|i	|k	|*	|[30][S]\drarr	|w	|o	|d	|a	|[][,]{ }	|n	|a	|[][,]{ }	|m	|ł	|y	|n	|*	|.
|k	|*	|a	|j	|y	|k	|*	|[31][S]\darr	|[32][S]\darr	|b	|*	|*	|o	|r	|*	|*	|*	|*	|[33][S]\darr	|*	|*	|i	|*	|.
|e	|*	|r	|n	|*	|a	|[34][S]\rarr	|w	|o	|r	|e	|c	|z	|n	|i	|c	|a	|*	|k	|[35][S]\darr	|[36][S]\darr	|n	|*	|.
|t	|*	|n	|a	|*	|*	|*	|s	|s	|o	|*	|*	|a	|o	|*	|[37][S]\drarr	|d	|r	|a	|g	|a	|*	|*	|.
|o	|[38][S]\darr	|a	|*	|*	|*	|*	|p	|t	|w	|*	|*	|[][,]{ }	|k	|*	|s	|*	|[39][S]\darr	|c	|o	|n	|*	|*	|.
|t	|f	|[][,]{ }	|*	|*	|*	|*	|ó	|r	|i	|*	|*	|l	|s	|*	|t	|*	|o	|y	|d	|d	|*	|*	|.
|e	|l	|m	|*	|*	|*	|*	|ł	|o	|e	|*	|*	|ę	|i	|*	|r	|*	|p	|k	|y	|r	|*	|*	|.
|t	|a	|a	|*	|[40][S]\darr	|*	|*	|p	|b	|c	|*	|*	|d	|ę	|*	|z	|[41][S]\darr	|i	|*	|*	|e	|*	|*	|.
|r	|d	|g	|[42][S]\darr	|s	|*	|*	|r	|o	|*	|*	|*	|ź	|ż	|[43][S]\darr	|e	|n	|s	|[44][S]\darr	|*	|a	|*	|*	|.
|o	|r	|i	|s	|v	|*	|*	|a	|k	|*	|*	|*	|w	|n	|m	|l	|i	|o	|s	|*	|*	|*	|*	|.
|z	|o	|a	|i	|o	|*	|*	|c	|*	|*	|*	|*	|i	|i	|e	|e	|s	|w	|t	|*	|*	|*	|*	|.
|a	|w	|*	|s	|r	|[45][S]\rarr	|d	|a	|w	|c	|a	|*	|o	|k	|l	|c	|k	|o	|ó	|*	|*	|*	|*	|.
|*	|a	|*	|l	|a	|*	|*	|*	|[46][S]\rarr	|o	|ł	|ó	|w	|*	|i	|*	|o	|ś	|ł	|*	|*	|*	|*	|.
|[47][S]\rarr	|n	|i	|e	|d	|e	|f	|i	|n	|i	|o	|w	|a	|l	|n	|o	|ś	|ć	|*	|*	|*	|*	|*	|.
|*	|i	|*	|y	|a	|[48][S]\rarr	|t	|y	|g	|i	|e	|l	|*	|*	|a	|*	|ć	|*	|*	|*	|*	|*	|*	|.
|*	|e	|*	|*	|*	|*	|*	|*	|*	|*	|*	|*	|*	|*	|*	|*	|*	|*	|*	|*	|*	|*	|*	|.
|*	|*	|*	|*	|*	|*	|*	|*	|*	|*	|*	|*	|*	|*	|*	|*	|*	|*	|*	|*	|*	|*	|*	|.\end{Puzzle}

\newpage

\begin{PuzzleClues}{\textbf{Poziome}\\}\Clue{2}{}{angielski anatom i zoolog (1804-92); prace z anatomii porównawczej i paleontologii kręgowców}
\Clue{4}{}{znana od XIX wieku odmiana uprawna gruszy}
\Clue{5}{}{jednoczęściowy utwór na orkiestrę symfoniczną o swobodnej budowie opartej na programie literackim}
\Clue{9}{}{piłka decydująca o wygraniu seta; piłka setowa}
\Clue{11}{}{zasady, które obowiązują w jakiejś dziedzinie, przede wszystkim w sposobie zachowania}
\Clue{12}{}{broń myśliwska miotana złożona z dwóch lub więcej kul połączonych cięgnem używana przez Indian i Inuitów}
\Clue{14}{}{dobry humor, pozytywne nastawienie, pogodny nastrój}
\Clue{17}{}{potrawa mączna regionalnej kuchni podhalańskiej, która ma półpłynną formę}
\Clue{20}{}{cecha światła - to, że jest słabe, mało wyraźne}
\Clue{22}{}{w spawalnictwie: zewnętrzna powierzchnia spoiny}
\Clue{23}{}{skała metamorficzna powstała w wyniku przeobrażenia łupka ilastego bądź mułowca w warunkach niskich temperatur (200-400 °C) i niezbyt wysokich ciśnień}
\Clue{24}{}{postać, która występuje w utworze literackim, zwłaszcza w dramacie}
\Clue{27}{}{zdrobniale: prawa lub lewa strona ciała ludzkiego lub zwierzęcego}
\Clue{28}{}{urządzenie zawierające zbiór od kilku do kilkuset dysków fizycznych, które pogrupowane są w kilka do kilkudziesięciu grup RAID}
\Clue{29}{}{czynnik iloczynu - element mnożony w działaniu mnożenia}
\Clue{30}{}{coś, co napędza działania, często bezmyśle, wykonywane odruchowo}
\Clue{34}{}{SAKULINA}
\Clue{37}{}{pogłębiarka}
\Clue{45}{}{czlowiek, od którego pobierane są narządy lub tkanki do przeszczepu}
\Clue{46}{}{przedmiot wykonany z ołowiu}
\Clue{47}{}{to, że coś jest niedefiniowalne, cecha tego, czego nie można zdefiniować}
\Clue{48}{}{tyle, ile mieści się w tyglu - rondlu z rączką lub dwoma uszkami}\end{PuzzleClues}

\begin{PuzzleClues}{\textbf{Pionowe}\\}\Clue{1}{}{pomieszczenie wykopane w ziemi}
\Clue{2}{}{głoska artykułowana na wydechu}
\Clue{3}{}{związek chemiczny o małej masie cząsteczkowej, który w wyniku reakcji polimeryzacji może tworzyć różnej długości polimery}
\Clue{4}{}{chłop należący do grupy najbiedniejszych; określenie stosowane od 1944 roku}
\Clue{5}{}{krótkie spodnie sportowe, plażowe itp}
\Clue{6}{}{miasto w Czechach (kraj środkowoczeski) port nad Łabą}
\Clue{7}{}{pieszczotliwie, ze współczuciem o matce}
\Clue{8}{}{miejsce, w którym są zawierane transakcje kupna i sprzedaży różnych form kapitału pieniężnego}
\Clue{9}{}{segment rynku finansowego, na którym dokonywane są transakcje o okresie zapadalności do jednego roku}
\Clue{10}{}{rzemiosło ceglarskie; to czym zajmuje się ceglarz}
\Clue{13}{}{specjalny strój używany tylko w pewnych sytuacjach, np. w teatrze, jako przebranie karnawałowe, w sporcie itp}
\Clue{15}{}{przodek bydła domowego, ssak z rodziny krętorogich}
\Clue{16}{}{półwyrób, przeznaczony do dalszego wyrobu blachy}
\Clue{18}{}{mała latarnia, szczególnie ręczna}
\Clue{19}{}{wyznawca islamu, mahometanin}
\Clue{21}{}{każdy z dwunastu wybranych uczniów Jezusa; także propagator chrześcijaństwa}
\Clue{22}{}{południowoamerykański ptak z rodziny garncarzy}
\Clue{23}{}{złe samopoczucie kilka godzin po wypiciu zbyt dużej ilości alkoholu}
\Clue{24}{}{lordoza występująca na odcinku lędźwiowym kręgosłupa}
\Clue{25}{}{postać mityczna wspólna wielu kulturom, czarodziej, który zajmuje się czarnoksięstwem, tajemniczą gałęzią magii najczęściej uważaną za niebezpieczną i raczej podporządkowaną złu}
\Clue{26}{}{magia, która ma na celu robienie złych rzeczy}
\Clue{29}{}{tetroza należąca do ketoz}
\Clue{30}{}{browar, piwo - porcja piwa}
\Clue{31}{}{wspólne działanie}
\Clue{32}{}{ryba morska z rodziny ostrobokowatych}
\Clue{33}{}{środkowoamerykański ptak z rzędu wróblowatych}
\Clue{35}{}{rykowisko u jeleni}
\Clue{36}{}{Castagno}
\Clue{37}{}{gwiazdozbiór zodiakalny nieba południowego, również znak Zodiaku}
\Clue{38}{}{mazerowanie, słojowanie - malowanie na powierzchni drewna słojów mających imitować drogie gatunki drewna}
\Clue{39}{}{cecha czegoś, co ma charakter opisowy}
\Clue{40}{}{kolarz czeski, zwycięzca Wyścigu Pokoju w 1990 r}
\Clue{41}{}{niskie położenie, zawieszenie}
\Clue{42}{}{malarz francuski pochodzenia angielskiego (1839-99) przedstawiciel impresjonizmu; pejzaże okolic Paryża, brzegu Sekwany i innych}
\Clue{43}{}{podrzędny, nieprzyjemny lokal, gromadzący ludzi z marginesu społecznego}
\Clue{44}{}{(wiertniczy) element wiertnicy umieszczony nad otworem wiertniczym, przeznaczony do obracania przewodu wiertniczego}\end{PuzzleClues}\newpage\section*{Krzyżówka 21}

\noindent\begin{Puzzle}{22}{22}|*	|*	|[1][S]\drarr	|a	|k	|t	|*	|[2][S]\drarr	|m	|ł	|y	|n	|a	|r	|z	|ó	|w	|n	|a	|*	|*	|*	|*	|.
|*	|*	|k	|*	|*	|[3][S]\drarr	|a	|s	|f	|a	|l	|t	|o	|g	|u	|m	|a	|*	|*	|*	|*	|*	|*	|.
|*	|*	|i	|*	|*	|e	|*	|e	|[4][S]\darr	|[5][S]\darr	|*	|[6][S]\drarr	|e	|c	|u	|*	|[7][S]\drarr	|k	|l	|a	|n	|*	|*	|.
|*	|[8][S]\drarr	|t	|r	|a	|k	|*	|k	|t	|r	|[9][S]\drarr	|k	|u	|p	|a	|*	|o	|*	|*	|[10][S]\darr	|*	|*	|*	|.
|*	|n	|a	|[11][S]\darr	|[12][S]\darr	|s	|[13][S]\darr	|t	|ł	|o	|s	|a	|[14][S]\rarr	|g	|l	|o	|s	|a	|*	|u	|[15][S]\darr	|*	|*	|.
|*	|p	|*	|p	|s	|a	|k	|*	|u	|g	|e	|p	|*	|*	|*	|*	|t	|*	|*	|n	|p	|*	|*	|.
|*	|r	|*	|u	|i	|b	|a	|[16][S]\rarr	|m	|o	|r	|k	|a	|*	|[17][S]\drarr	|v	|a	|t	|*	|i	|ł	|*	|*	|.
|*	|*	|*	|d	|e	|i	|p	|*	|*	|w	|i	|a	|*	|[18][S]\darr	|a	|*	|t	|*	|*	|ż	|o	|*	|*	|.
|*	|[19][S]\rarr	|h	|u	|r	|t	|o	|w	|n	|i	|a	|*	|[20][S]\darr	|r	|p	|*	|n	|[21][S]\darr	|[22][S]\darr	|a	|m	|*	|*	|.
|*	|*	|*	|*	|o	|*	|k	|*	|[23][S]\darr	|e	|l	|*	|d	|e	|o	|*	|i	|i	|w	|n	|y	|*	|*	|.
|*	|[24][S]\rarr	|a	|l	|t	|*	|*	|*	|j	|c	|i	|*	|a	|s	|l	|[25][S]\darr	|a	|t	|y	|i	|c	|[26][S]\darr	|*	|.
|*	|[27][S]\rarr	|s	|z	|a	|ł	|w	|i	|a	|*	|z	|*	|c	|z	|o	|b	|[][,]{ }	|i	|p	|e	|z	|k	|*	|.
|*	|*	|*	|*	|[][,]{ }	|*	|*	|*	|h	|*	|m	|[28][S]\darr	|h	|t	|g	|ę	|w	|h	|i	|[][,]{ }	|e	|o	|*	|.
|*	|[29][S]\rarr	|t	|o	|n	|i	|k	|a	|*	|*	|*	|r	|ó	|k	|e	|b	|o	|a	|e	|s	|k	|ń	|*	|.
|*	|[30][S]\drarr	|d	|i	|a	|l	|o	|g	|[][,]{ }	|o	|b	|y	|w	|a	|t	|e	|l	|s	|k	|i	|*	|[][,]{ }	|*	|.
|[31][S]\drarr	|p	|r	|e	|t	|e	|n	|s	|j	|a	|*	|b	|k	|*	|y	|n	|a	|a	|i	|ę	|[32][S]\darr	|w	|*	|.
|s	|a	|*	|[33][S]\rarr	|u	|r	|n	|a	|*	|*	|*	|i	|a	|*	|k	|*	|*	|*	|*	|*	|k	|i	|*	|.
|k	|l	|*	|[34][S]\rarr	|r	|u	|d	|z	|i	|k	|*	|t	|*	|*	|*	|*	|[35][S]\drarr	|l	|i	|b	|r	|a	|*	|.
|u	|b	|[36][S]\rarr	|c	|a	|ł	|u	|s	|e	|k	|*	|w	|*	|[37][S]\rarr	|k	|u	|b	|e	|k	|*	|w	|c	|*	|.
|l	|a	|*	|[38][S]\rarr	|l	|e	|p	|i	|d	|o	|z	|a	|u	|r	|o	|m	|o	|r	|f	|y	|*	|k	|*	|.
|l	|*	|*	|[39][S]\rarr	|n	|a	|w	|a	|ł	|a	|*	|*	|[40][S]\rarr	|p	|r	|a	|l	|k	|a	|*	|*	|i	|*	|.
|*	|*	|[41][S]\rarr	|m	|y	|d	|l	|e	|n	|i	|e	|c	|*	|*	|[42][S]\rarr	|t	|a	|r	|o	|*	|*	|*	|*	|.
|[43][S]\rarr	|t	|o	|r	|*	|*	|*	|*	|[44][S]\rarr	|l	|e	|l	|k	|o	|w	|e	|*	|*	|*	|*	|*	|*	|*	|.\end{Puzzle}

\newpage

\begin{PuzzleClues}{\textbf{Poziome}\\}\Clue{1}{}{czyn, działanie, gest}
\Clue{2}{}{córka młynarza}
\Clue{3}{}{mieszanka asfaltu i gumy, służąca jako materiał do budowy dróg}
\Clue{6}{}{jednostka monetarna Europejskiej Wspólnoty Gospodarczej, przyjęta w międzynarodowych rozliczeniach finansowych; w 1999 r. została zastąpiona przez wspólną walutę europejską}
\Clue{7}{}{polski serial telewizyjny}
\Clue{8}{}{mieszkaniec Tracji, człowiek pochodzenia trackiego}
\Clue{9}{}{odchody stałe, kał}
\Clue{14}{}{głos w dyskuji na jakiś temat, często w formie artystycznej, gdy dyskusja odbywa się np. w mediach lub jej tłem jest czyjaś twórczość}
\Clue{16}{}{wiatr wiejący od strony morza, najczęściej z towarzyszącą mu mżawką}
\Clue{17}{}{podatek pośredni, pobierany na każdym kolejnym etapie obrotu towarami lub usługami (podatek obrotowy), którego konstrukcja zakłada brak kaskadowego nakładania się podatku poprzez zastosowanie mechanizmu odliczenia podatku pobranego w poprzednich etapach obrotu}
\Clue{19}{}{rodzaj przedsiębiorstwa zajmującego się handlem hurtowym}
\Clue{24}{}{niski głos kobiecy lub chłopięcy; głos środkowy między sopranem a tenorem}
\Clue{27}{}{napar z liści szałwii}
\Clue{29}{}{główny dźwięk gamy}
\Clue{30}{}{kontakt między władzą państwową a organizacjami trzeciego sektora, który polega na wzajemnym przekazywaniu sobie informacji czy ustaleń dotyczących celów, instrumentów oraz strategii wdrażania polityki publicznej}
\Clue{31}{}{żal kogoś odczuwany z jakiegoś konkretnego powodu; przeważnie l.mn}
\Clue{33}{}{naczynie służące do przechowywania prochów zmarłego}
\Clue{34}{}{raszka; ptak leśno-parkowy z rzędu wróblowatych, żywi się owadami i dżdżownicami. chroniony; Eurazja, płn. Afryka}
\Clue{35}{}{WAGA}
\Clue{36}{}{zdrobniale: całus}
\Clue{37}{}{plastikowy lub papierowy pojemnik, w który pakuje się produkty spożywcze}
\Clue{38}{}{Lepidosauromorpha - infragromada gadów z podgromady Diapsida; znane są w zapisie kopalnym od permu}
\Clue{39}{}{nagłe zdarzenie, zjawisko gwałtowne i niebezpieczne}
\Clue{40}{}{zawartość pralki, tyle, ile mieści się w pralce}
\Clue{41}{}{ZAPIAN}
\Clue{42}{}{jadalne bulwy kolokazji}
\Clue{43}{}{dwie szyny podtrzymujące i prowadzące koła pojazdów szynowych, ułożone na podkładach lub wlane w specjalną płytę betonową służą jako droga kolejowa, tramwajowa lub metro, w określonej odległości od siebie}
\Clue{44}{}{kozodoje, Caprimulgiformes - rząd ptaków z podgromady Neornithes}\end{PuzzleClues}

\begin{PuzzleClues}{\textbf{Pionowe}\\}\Clue{1}{}{ogon lisa}
\Clue{2}{}{wino musujące}
\Clue{3}{}{jednostka informacji równa 10\textasciicircum18 bitów}
\Clue{4}{}{lekceważąco: masy, lud}
\Clue{5}{}{małż z rodziny rogowcowatych, z podgromady Heterodonta}
\Clue{6}{}{odrobina, bardzo mała ilość czegoś (zwykle płynnego)}
\Clue{7}{}{ostatnie życzenia, rozporządzenia umierającego, również wyrażone w testamencie}
\Clue{8}{}{kod ISO 4217 rupii nepalskiej}
\Clue{9}{}{kierunek w muzyce współczesnej wywodzący się z muzyki dodekafonicznej; polega na całkowitym uporządkowaniu wszelkich aspektów tworzywa muzycznego: rytmu, dynamiki, artykulacji itp. za pomocą serii}
\Clue{10}{}{przyznawanie komuś wyższości}
\Clue{11}{}{najmniejszy ssak południowoamerykański z rodziny jeleniowatych}
\Clue{12}{}{chłopiec, którego oboje rodzice zmarli; forma rzadsza}
\Clue{13}{}{nazwa łatwo łamiących się, nieprzędnych włókien nasiennych (w formie puchu) o długości do 35 mm, otrzymywanych z różnych gatunków drzew z podrodziny wełniakowatych}
\Clue{15}{}{zdrobniale: płomyk - coś, co ma charakterystyczny, podłużny, zwężający się ku górze kształt przywodzący na myśl język ognia}
\Clue{17}{}{przemowa (wygłoszona lub napisana) zawierająca apologię}
\Clue{18}{}{kawałek tkaniny sprzedawany po obniżonej cenie; pozostałość z większej ilości, której jest za mało, by szyć z niej większe rzeczy}
\Clue{20}{}{wyrób z wypalonej gliny lub mieszaniny cementu i piasku, używany do krycia dachów}
\Clue{21}{}{rodzaj sagi, eposu indyjskiego (zwykle w lm)}
\Clue{22}{}{zaczerwienienie skóry na twarzy, często będące następstwem emocji, zmęczenia, gorąca albo mrozu}
\Clue{23}{}{wśród rastafarian i twórców reggae: bóg, stwórca}
\Clue{25}{}{część maszyny w kształcie obrotowego walca}
\Clue{26}{}{wiatka - rasa małego, rosyjskiego, prymitywnego konia rolniczego pochodzącego od tarpana; używany do  zaprzęgu, transportu, przy wyrębie drzew oraz różnorakich prac rolniczych}
\Clue{28}{}{ptak wodny z podrodziny rybitw, rodziny mewowatych}
\Clue{30}{}{gęsta strzelanina}
\Clue{31}{}{skul - jednoosobowa łódź półwyścigowa}
\Clue{32}{}{kod ISO 4217 wona południowokoreańskiego}
\Clue{35}{}{broń myśliwska miotana złożona z dwóch lub więcej kul połączonych cięgnem używana przez Indian i Inuitów}\end{PuzzleClues}\newpage\section*{Krzyżówka 22}

\noindent\begin{Puzzle}{22}{32}|*	|[1][S]\drarr	|t	|v	|*	|[2][S]\drarr	|s	|o	|c	|j	|o	|p	|s	|y	|c	|h	|o	|l	|o	|g	|i	|a	|*	|.
|*	|b	|*	|*	|*	|p	|[3][S]\darr	|[4][S]\darr	|*	|*	|[5][S]\darr	|[6][S]\darr	|*	|*	|*	|[7][S]\drarr	|k	|o	|t	|e	|l	|a	|*	|.
|*	|e	|[8][S]\darr	|*	|[9][S]\darr	|a	|c	|a	|[10][S]\darr	|[11][S]\rarr	|s	|k	|a	|r	|b	|n	|i	|c	|a	|*	|*	|[12][S]\darr	|[13][S]\darr	|.
|*	|ł	|b	|[14][S]\darr	|w	|r	|f	|l	|r	|[15][S]\darr	|k	|o	|*	|[16][S]\rarr	|m	|o	|t	|e	|l	|a	|*	|h	|d	|.
|[17][S]\drarr	|t	|a	|w	|d	|a	|*	|l	|z	|d	|o	|n	|[18][S]\drarr	|d	|y	|w	|a	|n	|i	|k	|*	|i	|o	|.
|m	|*	|l	|i	|ó	|p	|*	|e	|e	|e	|r	|o	|b	|[19][S]\rarr	|l	|e	|e	|u	|w	|*	|*	|o	|w	|.
|a	|*	|o	|j	|w	|e	|*	|g	|ź	|b	|z	|p	|y	|*	|[20][S]\rarr	|l	|a	|k	|*	|[21][S]\darr	|*	|b	|ó	|.
|t	|[22][S]\drarr	|n	|e	|k	|t	|a	|r	|*	|i	|o	|n	|s	|[23][S]\rarr	|h	|i	|r	|o	|l	|a	|*	|o	|d	|.
|e	|r	|*	|*	|a	|*	|*	|e	|*	|u	|n	|i	|t	|[24][S]\darr	|[25][S]\rarr	|s	|o	|b	|o	|l	|e	|w	|*	|.
|r	|e	|*	|*	|*	|*	|*	|t	|*	|t	|e	|c	|r	|k	|[26][S]\darr	|t	|*	|*	|[27][S]\darr	|b	|*	|a	|*	|.
|i	|a	|[28][S]\darr	|*	|*	|*	|[29][S]\darr	|t	|[30][S]\darr	|[][,]{ }	|r	|a	|z	|u	|w	|y	|*	|*	|d	|a	|[31][S]\darr	|[][,]{ }	|*	|.
|a	|k	|l	|*	|*	|*	|n	|o	|t	|n	|a	|*	|y	|s	|r	|k	|*	|*	|u	|t	|d	|w	|[32][S]\darr	|.
|ł	|c	|a	|[33][S]\darr	|*	|[34][S]\darr	|i	|*	|r	|i	|*	|[35][S]\darr	|c	|z	|o	|a	|*	|*	|l	|r	|ż	|i	|g	|.
|[][,]{ }	|j	|s	|d	|*	|a	|d	|*	|a	|m	|[36][S]\drarr	|m	|a	|n	|n	|*	|*	|[37][S]\darr	|c	|o	|e	|e	|o	|.
|j	|a	|o	|o	|*	|n	|e	|*	|c	|z	|c	|l	|*	|i	|i	|*	|*	|k	|y	|s	|r	|ś	|r	|.
|ą	|[][,]{ }	|n	|w	|*	|t	|r	|*	|z	|o	|h	|e	|*	|k	|e	|*	|[38][S]\darr	|u	|n	|*	|e	|ć	|b	|.
|d	|k	|ó	|ó	|*	|r	|l	|*	|[][,]{ }	|w	|r	|c	|*	|*	|[][,]{ }	|[39][S]\rarr	|p	|l	|e	|ś	|ń	|*	|u	|.
|r	|s	|g	|d	|*	|o	|a	|[40][S]\darr	|a	|i	|y	|z	|[41][S]\rarr	|b	|o	|r	|u	|t	|a	|*	|*	|[42][S]\darr	|s	|.
|o	|a	|[][,]{ }	|[][,]{ }	|[43][S]\darr	|p	|n	|m	|m	|t	|z	|a	|*	|[44][S]\darr	|k	|*	|n	|*	|*	|*	|*	|s	|z	|.
|w	|n	|j	|p	|m	|o	|d	|e	|e	|s	|m	|r	|[45][S]\drarr	|p	|o	|d	|k	|o	|w	|a	|*	|k	|a	|.
|y	|t	|e	|o	|i	|n	|z	|l	|r	|c	|a	|z	|s	|i	|*	|[46][S]\rarr	|t	|u	|r	|n	|e	|r	|*	|.
|*	|o	|z	|ś	|l	|i	|k	|o	|y	|h	|t	|*	|z	|ó	|*	|[47][S]\darr	|[][,]{ }	|[48][S]\darr	|[49][S]\darr	|*	|*	|z	|*	|.
|[50][S]\drarr	|p	|i	|r	|a	|m	|i	|d	|k	|a	|*	|*	|c	|r	|[51][S]\rarr	|b	|a	|j	|a	|*	|*	|e	|*	|.
|b	|r	|o	|e	|[][,]{ }	|i	|*	|r	|a	|*	|*	|*	|z	|n	|*	|r	|p	|a	|s	|[52][S]\darr	|*	|l	|*	|.
|o	|o	|r	|d	|l	|a	|*	|a	|ń	|*	|*	|*	|u	|i	|*	|a	|t	|r	|y	|t	|*	|o	|*	|.
|r	|t	|n	|n	|ą	|*	|*	|m	|s	|*	|*	|*	|r	|k	|*	|u	|e	|z	|l	|r	|*	|n	|*	|.
|o	|e	|y	|i	|d	|[53][S]\rarr	|t	|a	|k	|a	|m	|i	|*	|*	|*	|n	|c	|m	|a	|o	|*	|o	|*	|.
|k	|i	|*	|*	|o	|[54][S]\rarr	|e	|t	|i	|u	|d	|a	|*	|*	|*	|*	|z	|o	|b	|g	|[55][S]\darr	|g	|*	|.
|r	|n	|[56][S]\rarr	|ś	|w	|i	|t	|*	|*	|*	|*	|[57][S]\rarr	|d	|a	|r	|o	|n	|*	|i	|o	|c	|i	|*	|.
|z	|o	|[58][S]\drarr	|f	|a	|l	|o	|w	|ó	|d	|[][,]{ }	|a	|k	|u	|s	|t	|y	|c	|z	|n	|y	|*	|*	|.
|e	|w	|d	|*	|*	|*	|[59][S]\drarr	|n	|a	|d	|n	|e	|r	|c	|z	|e	|*	|*	|m	|y	|d	|*	|*	|.
|m	|a	|y	|[60][S]\rarr	|r	|e	|s	|e	|t	|*	|[61][S]\rarr	|r	|a	|d	|[][S]ž	|k	|o	|t	|*	|*	|r	|*	|*	|.
|*	|*	|*	|*	|*	|*	|*	|*	|*	|*	|*	|*	|*	|*	|*	|*	|*	|*	|*	|*	|*	|*	|*	|.\end{Puzzle}

\newpage

\begin{PuzzleClues}{\textbf{Poziome}\\}\Clue{1}{}{telewizja, dział telekomunikacji zajmujący się przekazywaniem ruchomego obrazu oraz dźwięku na odległość}
\Clue{2}{}{dział psychologii zajmujący się procesami zachodzącymi w grupach ludzkich; psychologia społeczna}
\Clue{7}{}{ur.1924r. architekt - plan ogólny Częstochowy}
\Clue{11}{}{obfite źródło czegoś, co jest uznawane za cenne, wartościowe}
\Clue{16}{}{ryba morska z rodziny dorszowatych o wydłużonym ciele mająca dwie płetwy grzbietowe}
\Clue{17}{}{miasto w azjatyckiej części Federacji Rosyjskiej na płn.-wsch. od Jekaterynburga, nad rzeką Tawda}
\Clue{18}{}{sytuacja, w której podwładny otrzymuje reprymendę od zwierzchnika, najczęściej u niego w gabinecie; upominająca rozmowa z szefem}
\Clue{19}{}{ur. w 1926 r., kompozytor holenderski; utwory orkiestrowe, kameralne, fortepianowe, opera telewizyjna 'Alceste'}
\Clue{20}{}{kod ISO 4217 waluty kip}
\Clue{22}{}{napój owocowy, przecierowy lub sokowy otrzymany ze świeżego lub utrwalonego metodami fizycznymi kremogenu lub zagęszczonego soku rozcieńczonego wodą z dodatkiem cukrów oraz kwasów spożywczych}
\Clue{23}{}{antylopa Huntera, Beatragus hunteri - gatunek ssaka z rodziny krętorogich, zamieszkujący sawanny w Kenii i Somalii; introdukowany w Tsavo National Park w Kenii}
\Clue{25}{}{wieś w Polsce położona w województwie mazowieckim, w powiecie garwolińskim, w gminie Sobolew}
\Clue{36}{}{dzieła Thomasa Manna}
\Clue{39}{}{przestawiciel saprofitycznych grzybów z różnych grup systematycznych; ich grzybnia rozwija się na różnych związkach organicznych (np. pokarmach roślinnych, nawozie, kompoście, skórze), pokrywając je gęstym, białym lub barwnym kożuszkiem (szkodliwe zjawisko pleśnieni)}
\Clue{41}{}{brak porządku, chaos, zamieszanie}
\Clue{45}{}{siniec pod okiem, zwykle wynik zmęczenia lub choroby}
\Clue{46}{}{angielski malarz i grafik (1775-1851 ) wybitny pejzażysta}
\Clue{50}{}{coś o kształcie małej piramidy lub pieszczotliwie o czymś, co ma kształt piramdy}
\Clue{51}{}{miasto w płd. Węgrzech, port nad Dunajem, ważny węzeł komunikacyjny}
\Clue{53}{}{(1907-65), japoński pisarz i historyk literatury}
\Clue{54}{}{utwór muzyczny o walorach dydaktycznych, zawierający trudności techniczne, by ćwiczyć grę na instrumencie lub śpiew}
\Clue{56}{}{początek czegoś (np. świt nowej ery)}
\Clue{57}{}{francuski filozof, socjolog i publicysta (1905-83); jeden z głównych przedstawicieli teorii konwergencji}
\Clue{58}{}{kanał służący do prowadzenia fal akustycznych}
\Clue{59}{}{parzysty, niewielki (waga od 10 do 18 gramów) gruczoł wydzielania wewnętrznego położony zaotrzewnowo na górnym biegunie nerki}
\Clue{60}{}{stan człowieka polegający na całkowitym odpoczynku, odcięciu się od sytuacji stresowych, drażniących sytuacji}
\Clue{61}{}{miasto w Indiach (Gudżarat) na Płw. Kathijawar}\end{PuzzleClues}

\begin{PuzzleClues}{\textbf{Pionowe}\\}\Clue{1}{}{cieśnina morska}
\Clue{2}{}{żartobliwie lub pogardliwie: elektryczny lub elektroniczny instrument klawiszowy, który pozwala grającemu na wykonywanie utworów z automatycznym akompaniamentem}
\Clue{3}{}{skrót oznaczający kaliforn}
\Clue{4}{}{umiarkowane tempo, które jest szybsze niż andante oraz wolniejsze niż allegro}
\Clue{5}{}{Scorzonera hispanica - gatunek rośliny warzywnej z rodziny astrowatych}
\Clue{6}{}{Datisca - rodzaj bylin z rodziny konopnicowatych; obejmuje dwa gatunki}
\Clue{7}{}{dział literatury związany z tworzeniem krótkich utworów pisanych prozą}
\Clue{8}{}{rodzaj pneumatycznej opony o szerokim przekroju}
\Clue{9}{}{Scabiosa caucasica, driakiew kaukaska - gatunek rośliny należący do rodziny szczeciowatych}
\Clue{10}{}{przenośnie, żartobliwie: bardzo trudny egzamin, którego nie zdało wiele osób, też: trudne przesłuchanie, którego nie przeszło wiele osób}
\Clue{12}{}{informacja zapowiadająca tragedię, nieszczęście}
\Clue{13}{}{w logice, skończony ciąg zdań uzasadnijący prawdziwość jakiegoś twierdzenia}
\Clue{14}{}{gromada stawonogów lądowych o walcowatym ciele}
\Clue{15}{}{otwarcie szachowe, powstające po posunięciach: 1.e4 Sc6}
\Clue{17}{}{substancje, które mogą uwolnić znaczącą ilość energii jądrowej}
\Clue{18}{}{fragment biegu rzeki, w którym następuje przyspieszenie prądu wskutek załamania bądź zwężenia koryta rzeki}
\Clue{21}{}{duży ptak oceaniczny z rzędu rurkonosych, rybożerny, długie, wąskie skrzydła, Ocean Spokojny i Ocean Atlantycki}
\Clue{22}{}{reakcja charakterystyczna białek zawierających aminokwasy z pierścieniami aromatycznymi (np. tryptofan, tyrozyna, fenyloalanina) ze stężonym kwasem azotowym}
\Clue{24}{}{średniowieczny żołnierz uzbrojony w kuszę}
\Clue{26}{}{Strychnos nux-vomica - gatunek wiecznie zielonego drzewa lub krzewu z rodziny loganiowatych (Loganiaceae)}
\Clue{27}{}{dama serca, ukochana, kochanka}
\Clue{28}{}{Mysis relicta - gatunek skorupiaka z rzędu lasonogów; relikt polodowcowy}
\Clue{29}{}{język indoeuropejski z grupy języków germańskich zaliczany do języków dolnoniemieckich}
\Clue{30}{}{kapturnik, Lophodytes cucullatus - gatunek ptaka z rodziny kaczkowatych (Anatidae), jedyny przedstawiciel rodzaju Lophodytes; gniazduje w pobliżu słodkich wód w Ameryce Północnej}
\Clue{31}{}{gazela mongolska, Procapra gutturosa - gatunek ssaka parzystokopytnego z rodziny krętorogich; zamieszkuje wschodnią Mongolię i przylegające do niej tereny Rosji oraz północnych Chin}
\Clue{32}{}{gatunek pacyficznego łososia o długości do 70 cm; Pacyfik, Morze Arktyczne}
\Clue{33}{}{forma dowodu logicznego, w którym z założenia o nieprawdziwości tezy wyprowadza się sprzeczność ze zdaniem prawdziwym}
\Clue{34}{}{dział onomastyki zajmujący się badaniem nazw osobowych (imion, przezwisk, nazwisk itd.)}
\Clue{35}{}{specjalista zajmujący się mleczarstwem}
\Clue{36}{}{według wierzeń chrześcijańskich mistyczne piętno jakie otrzymuje ochrzczona osoba}
\Clue{37}{}{zespół wierzeń i związanych z nich działań podejmowanych, aby uczcić bóstwo lub jego artefakt}
\Clue{38}{}{placówka spełniająca podobną funkcję jak apteka, która może zostać usytuowana jedynie na terenie wiejskim, o ile na terenie danej wsi nie jest już prowadzona apteka ogólnodostępna}
\Clue{40}{}{gatunek sztuki teatralnej, popularny we Francji i Włoszech w XVI i XVII wieku, opierający się na deklamowaniu utworu przy akompaniamencie muzyki}
\Clue{42}{}{Branchiopoda - gromada skorupiaków słodkowodnych; ciało wyraźnie segmentowane - duża liczba segmentów; odnóża liczne, dwugałęziste, służą jednocześnie do poruszania się, oddychania i do pobierania pokarmu}
\Clue{43}{}{pozaukładowa jednostka odległości stosowana w krajach anglosaskich, równa 1609,344 metry}
\Clue{44}{}{zawartość piórnika, to co się mieści w piórniku}
\Clue{45}{}{Rattus - rodzaj gryzonia z rodziny myszowatych obejmujący około 50 gatunków}
\Clue{47}{}{Volker - ur. 1939, pisarz niemiecki, poezje, opowiadania, dramaty; „Prowokacje”, „Nieskrępowany żywot Kasta”}
\Clue{48}{}{w elektrotechnice: nieuzwojony element przeznaczony do łączenia rdzeni elektromagnesu, transformatora itp}
\Clue{49}{}{system wersyfikacji, który nie zakłada stałej liczby sylab w wersie, dopuszcza ich niejednakową ilość}
\Clue{50}{}{szkło o wysokiej wytrzymałości chemicznej i temperaturowej zawierające telenek krzemu}
\Clue{52}{}{Trogoniformes - rząd ptaków z podgromady Neornithes}
\Clue{55}{}{JABŁECZNIK; francuskie, słabe, wytrawne wino otrzymywane z jabłek}
\Clue{58}{}{w chemii: symbol dysprozu}
\Clue{59}{}{w chemii: symbol siarki}\end{PuzzleClues}\newpage\section*{Krzyżówka 23}

\noindent\begin{Puzzle}{25}{19}|*	|*	|*	|*	|*	|*	|*	|*	|*	|*	|[1][S]\darr	|*	|*	|*	|*	|*	|*	|[2][S]\darr	|*	|*	|*	|*	|*	|*	|*	|*	|.
|*	|*	|*	|*	|*	|*	|*	|*	|*	|*	|n	|*	|*	|*	|*	|*	|*	|k	|*	|*	|*	|*	|*	|*	|*	|*	|.
|*	|*	|*	|*	|*	|*	|*	|*	|[3][S]\rarr	|s	|u	|w	|a	|n	|y	|*	|*	|w	|*	|*	|*	|*	|*	|*	|*	|*	|.
|*	|*	|*	|*	|*	|*	|*	|*	|*	|*	|n	|*	|*	|*	|*	|*	|*	|a	|*	|*	|*	|*	|*	|*	|[4][S]\darr	|*	|.
|*	|*	|*	|*	|*	|*	|*	|*	|[5][S]\darr	|*	|e	|*	|*	|*	|*	|*	|*	|s	|*	|*	|*	|*	|*	|*	|g	|*	|.
|*	|*	|*	|*	|*	|*	|*	|*	|h	|*	|z	|*	|*	|*	|*	|*	|*	|[][,]{ }	|*	|*	|*	|*	|*	|*	|r	|*	|.
|*	|*	|*	|*	|*	|*	|[6][S]\rarr	|t	|u	|k	|*	|*	|*	|*	|*	|*	|*	|m	|*	|*	|*	|*	|*	|*	|z	|*	|.
|*	|*	|*	|[7][S]\darr	|*	|*	|[8][S]\rarr	|o	|s	|o	|b	|ó	|w	|k	|a	|*	|*	|r	|*	|*	|*	|[9][S]\darr	|*	|*	|y	|*	|.
|[10][S]\rarr	|c	|e	|r	|a	|t	|o	|r	|y	|t	|*	|*	|*	|*	|*	|[11][S]\darr	|*	|ó	|*	|*	|*	|c	|*	|*	|b	|*	|.
|*	|*	|*	|o	|*	|*	|*	|*	|t	|*	|*	|*	|*	|*	|*	|h	|*	|w	|*	|*	|*	|z	|*	|*	|n	|*	|.
|*	|*	|*	|y	|*	|*	|*	|*	|k	|*	|*	|*	|*	|*	|*	|e	|*	|k	|*	|*	|*	|o	|*	|*	|i	|*	|.
|*	|*	|*	|c	|*	|[12][S]\rarr	|m	|i	|a	|z	|g	|a	|[][,]{ }	|k	|o	|r	|k	|o	|t	|w	|ó	|r	|c	|z	|a	|*	|.
|*	|[13][S]\rarr	|w	|e	|l	|l	|e	|r	|*	|*	|*	|*	|*	|*	|*	|o	|*	|w	|*	|*	|*	|t	|*	|*	|*	|*	|.
|*	|*	|*	|*	|*	|*	|*	|*	|*	|*	|*	|*	|*	|*	|*	|d	|*	|y	|*	|*	|*	|*	|*	|*	|*	|*	|.
|*	|*	|*	|*	|*	|*	|*	|*	|*	|*	|*	|*	|*	|*	|*	|[][S]-	|*	|*	|*	|*	|*	|*	|*	|*	|*	|*	|.
|*	|*	|*	|*	|*	|*	|*	|*	|*	|*	|*	|*	|*	|*	|*	|b	|*	|*	|*	|*	|*	|*	|*	|*	|*	|*	|.
|*	|*	|*	|*	|*	|*	|*	|*	|*	|*	|*	|*	|[14][S]\rarr	|c	|h	|a	|c	|o	|n	|n	|e	|*	|*	|*	|*	|*	|.
|*	|*	|*	|*	|*	|*	|*	|*	|*	|*	|*	|*	|*	|*	|*	|b	|*	|*	|*	|*	|*	|*	|*	|*	|*	|*	|.
|*	|*	|*	|*	|*	|*	|*	|*	|*	|*	|*	|*	|*	|*	|*	|a	|*	|*	|*	|*	|*	|*	|*	|*	|*	|*	|.
|*	|*	|*	|*	|*	|*	|*	|*	|*	|*	|*	|*	|*	|*	|*	|*	|*	|*	|*	|*	|*	|*	|*	|*	|*	|*	|.\end{Puzzle}

\newpage

\begin{PuzzleClues}{\textbf{Poziome}\\}\Clue{3}{}{rodzaj tańca ludowego, odmiana krakowiaka}
\Clue{6}{}{w gwarze poznańskiej: tłuszcz zwierzęcy, który wytopiono; niekiedy również szpik kostny albo mieszanka tych obu}
\Clue{8}{}{potoczne określenie pociągu osobowego}
\Clue{10}{}{malarz holenderski. ur. w 1922 r., działał głównie w Paryżu, współzałożyciel grupy Cobra i Reflex}
\Clue{12}{}{tkanka twórcza wtórna rośliny naczyniowej, której komórki dzieląc się peryklinalnie wytwarzają korek po zewnętrznej stronie felogenu i fellodermę po stronie wewnętrznej, przy czym podziały prowadzące do powstania korka są liczniejsze niż podziały związane z produkcją felodermy}
\Clue{13}{}{amerykański pediatra i bakteriolog ur. w 1915 r., współtwórca metody hodowli wirusa choroby Heinego-Medina, nagroda Nobla}
\Clue{14}{}{utwór instrumentalny w takcie trójdzielnym złożony z szeregu wariacji}\end{PuzzleClues}

\begin{PuzzleClues}{\textbf{Pionowe}\\}\Clue{1}{}{portugalski astronom i kartograf (1492-1577)}
\Clue{2}{}{organiczny związek chemiczny, najprostszy kwas karboksylowy, występuje np. w jadzie mrówek}
\Clue{4}{}{Mycelium - forma plechy grzybów, stanowiąca wegetatywne ciało grzybów zbudowane z wyrośniętej i rozgałęzionej strzępki lub wielu strzępek skupionych w jednym miejscu}
\Clue{5}{}{zwolenniczka nauczania Jana Husa, zwłaszcza uczestniczka jego ruchu religijnego i społecznego z XV wieku}
\Clue{7}{}{samochód marki Rolls-Royce}
\Clue{9}{}{z podziwem o człowieku sprytnym, zaradnym}
\Clue{11}{}{kobieta o silnym charakterze, stanowcza i energiczna; może być postrzegana pozytywnie ze względu na odwagę czy upór, lub negatywnie z powodu despotyzmu}\end{PuzzleClues}\newpage\section*{Krzyżówka 24}

\noindent\begin{Puzzle}{19}{22}|*	|*	|[1][S]\darr	|[2][S]\darr	|*	|*	|*	|[3][S]\drarr	|t	|a	|h	|i	|t	|a	|ń	|c	|z	|y	|k	|*	|.
|*	|*	|o	|s	|[4][S]\darr	|*	|[5][S]\rarr	|p	|r	|z	|y	|b	|y	|t	|e	|k	|*	|*	|*	|*	|.
|*	|[6][S]\rarr	|p	|ł	|a	|s	|k	|o	|n	|o	|s	|*	|[7][S]\darr	|[8][S]\darr	|[9][S]\darr	|*	|*	|*	|*	|*	|.
|*	|*	|o	|u	|n	|[10][S]\darr	|*	|d	|*	|*	|*	|[11][S]\drarr	|p	|a	|j	|a	|c	|y	|k	|*	|.
|*	|[12][S]\darr	|n	|ż	|t	|k	|[13][S]\rarr	|k	|u	|*	|[14][S]\darr	|c	|u	|r	|e	|*	|[15][S]\darr	|*	|*	|*	|.
|*	|k	|a	|b	|e	|a	|*	|o	|[16][S]\darr	|*	|t	|e	|d	|g	|r	|*	|g	|*	|*	|*	|.
|*	|u	|[][,]{ }	|a	|n	|o	|*	|w	|k	|[17][S]\darr	|r	|c	|ł	|y	|o	|[18][S]\darr	|a	|*	|[19][S]\darr	|*	|.
|*	|l	|r	|*	|o	|l	|*	|i	|a	|n	|a	|h	|o	|r	|m	|u	|e	|[20][S]\darr	|s	|*	|.
|*	|i	|a	|[21][S]\drarr	|r	|i	|v	|e	|r	|i	|n	|a	|*	|o	|e	|k	|l	|p	|ą	|*	|.
|*	|k	|d	|b	|*	|a	|[22][S]\darr	|c	|a	|e	|s	|[][,]{ }	|*	|z	|*	|o	|i	|a	|d	|*	|.
|*	|[][,]{ }	|i	|a	|*	|n	|e	|[][,]{ }	|f	|l	|g	|f	|*	|a	|*	|ś	|c	|m	|e	|*	|.
|*	|m	|a	|r	|[23][S]\darr	|*	|u	|d	|i	|e	|r	|i	|*	|u	|*	|n	|k	|i	|c	|*	|.
|*	|n	|l	|r	|s	|*	|f	|u	|n	|g	|e	|z	|*	|r	|*	|i	|i	|ę	|z	|*	|.
|*	|i	|n	|a	|e	|[24][S]\darr	|o	|ż	|k	|a	|s	|y	|*	|*	|*	|k	|[][,]{ }	|ć	|a	|*	|.
|*	|e	|a	|m	|n	|n	|r	|y	|a	|l	|j	|c	|*	|*	|*	|o	|s	|[][,]{ }	|n	|*	|.
|*	|j	|*	|u	|t	|i	|i	|*	|*	|*	|a	|z	|*	|*	|[25][S]\darr	|w	|z	|u	|k	|*	|.
|[26][S]\rarr	|s	|o	|n	|a	|t	|a	|*	|*	|[27][S]\darr	|*	|n	|*	|*	|k	|a	|k	|k	|a	|*	|.
|*	|z	|*	|d	|*	|k	|*	|*	|*	|d	|[28][S]\rarr	|a	|e	|r	|a	|t	|o	|r	|*	|*	|.
|*	|y	|*	|a	|[29][S]\rarr	|a	|m	|m	|e	|r	|s	|*	|*	|*	|f	|e	|c	|y	|*	|*	|.
|*	|*	|*	|*	|*	|*	|[30][S]\rarr	|p	|t	|a	|s	|z	|n	|i	|k	|*	|k	|t	|*	|*	|.
|[31][S]\rarr	|f	|e	|t	|y	|s	|z	|y	|z	|m	|*	|*	|[32][S]\rarr	|k	|a	|n	|i	|a	|*	|*	|.
|*	|*	|*	|*	|*	|[33][S]\rarr	|d	|y	|n	|a	|m	|i	|z	|m	|*	|*	|*	|*	|*	|*	|.
|*	|[34][S]\rarr	|o	|p	|p	|i	|d	|u	|m	|*	|*	|*	|*	|*	|*	|*	|*	|*	|*	|*	|.\end{Puzzle}

\newpage

\begin{PuzzleClues}{\textbf{Poziome}\\}\Clue{3}{}{mieszkaniec Tahiti, człowiek pochodzący z Tahiti}
\Clue{5}{}{miejsce, w którym jest dużo ludzi, w którym dzieje się coś (pozornie) ważnego lub atrakcyjnego; słowo używane przenośnie, często ironicznie}
\Clue{6}{}{gatunek kaczki, dziób szeroki, łyżkowaty, żywi się planktonem; Eurazja, Ameryka Płn. płn. Afryka, płd. Azja, chroniony}
\Clue{11}{}{często w liczbie mnogiej: ćwiczenie, które polega na podskakiwaniu i wymachiwaniu rękami i nogami w taki sposób, że na zmianę albo ręce łączą się nad głową, a nogi są w rozkroku, albo nogi są połączone, a ręce przylegają do tułowia po bokach; ruch ćwiczącego przypomina sposób, w jaki porusza się drewniany pajacyk - marionetka}
\Clue{13}{}{heterodimer białka Ku70 (masa 69 kDa) i Ku86 (masa 83 kDA, zwane również Ku80), będący ważnym czynnikiem łączącym końce nici DNA w procesie niehomologicznego scalania końców DNA}
\Clue{21}{}{rolniczy region w Australii (Nowa Południowa Walia)}
\Clue{26}{}{pierwotnie każda kompozycja instrumentalna od XVII w, utwór instrumentalny, składająca się z czterech części}
\Clue{28}{}{urządzenie do napowietrzania}
\Clue{29}{}{Kuller- (1884-1966), pisarka holenderska; „Pochód krzyżowych kobiet”, „Jabłko i Ewa”}
\Clue{30}{}{myśliwy polujący na ptaki}
\Clue{31}{}{rodzaj parafilii seksualnej polegający na uzyskiwaniu satysfakcji seksualnej głównie lub wyłącznie w wyniku kontaktu z obiektem pobudzającym - fetyszem}
\Clue{32}{}{ludowy poeta z Opolszczyzny (1872-1957), uczestnik III powstania śląskiego; „Wiersze śląskie”}
\Clue{33}{}{cecha czegoś, co jest prężne, pełne energii, sprawia wrażenie szybkiego, dynamicznego, żywego}
\Clue{34}{}{miejsce obwarowane, osiedle obronne zakładane najczęściej w miejscu chronionym ukształtowaniem terenu}\end{PuzzleClues}

\begin{PuzzleClues}{\textbf{Pionowe}\\}\Clue{1}{}{opona, której kord ułożony jest promieniowo, używana głównie w samochodach osobowych}
\Clue{2}{}{wojsko lub inna instytucja użyteczności publicznej, która zajmuje się jakąś określoną dziedziną}
\Clue{3}{}{Rhinolophus ferrumequinum - nietoperz z rodziny podkowcowatych; występuje w zachodniej i środkowej Europie (najdalej na północ sięga do Anglii i Walii), w środkowej Azji po Japonię; w Polsce stwierdzony tylko siedmiokrotnie na terenie Wyżyny Krakowsko-Częstochowskiej i Karpat, gdzie prawdopodobnie zalatuje z sąsiedniej Słowacji}
\Clue{4}{}{gra świateł i cieni (np. w malarstwie) dająca wrażenie reliefu}
\Clue{7}{}{podium - miejsce dekoracji zwycięzców}
\Clue{8}{}{Argyrosaurus - rodzaj zauropoda z grupy tytanozaurów; żył w epoce późnej kredy na terenach Ameryki Południowej}
\Clue{9}{}{(1859-1927), pisarz angielski; „Trzech panów w łódce (nie licząc psa)”}
\Clue{10}{}{Sorghum japonicum - gatunek rośliny z rodziny wiechlinowatych}
\Clue{11}{}{cecha obiektów fizycznych}
\Clue{12}{}{Numenius phaeopus phaeopus - nominatywny podgatunek ptaka wyróżniony w obrębie gatunku kulik mniejszy (Numenius phaeopus); występuje na obszarze od Islandii, Wysp Owczych i północnej Szkocji przez Półwysep Skandynawski po południowo-zachodni Tajmyr}
\Clue{14}{}{przemiana, przekraczanie ustalonych, obowiązujących wcześniej zasad}
\Clue{15}{}{język z grupy goidelskiej języków celtyckich}
\Clue{16}{}{zawartość karafki oceniana jako mała ilość; tyle, ile się mieści w karafce}
\Clue{17}{}{rodzaj albumu lub utworu muzycznego, który został wyprodukowany z użyciem sampli uzyskanych bez zgody i wiedzy właściciela praw autorskich, a także uiszczenia pochodzących z tego tytułu opłat}
\Clue{18}{}{Thomisidae - kosmopolityczna rodzina pająków, obejmująca ponad 2150 gatunków, sklasyfikowanych w ok. 170 rodzajach; pająki te mają trójkątny odwłok i maskujące kolory}
\Clue{19}{}{planowana od wielu lat (obecnie obiecana na 2022 rok) droga szybkiego ruchu łącząca autostradę A4 z Nowym Sączem, Brzeskiem i granicą Polski}
\Clue{20}{}{pamięć bez identyfikacji pamiętanych treści jako wcześniej poznanych}
\Clue{21}{}{ROGOZĄB}
\Clue{22}{}{stan nienaturalnie dobrego samopoczucia z tendencją do śmiechu, radości, płytkiej wesołości i dowcipkowania, doznawany nawet pomimo rzeczywistych niedomagań organizmu}
\Clue{23}{}{miasto w Jugosławii (Serbia) w okręgu autonomicznym Wojwodina, port nad Cisą}
\Clue{24}{}{część układu (sieci) łącząca dwa punkty}
\Clue{25}{}{twórczość Kafki, zbiór jego powieści i opowiadań, także: zbiór tematów jego powieści (rzadko mówi się tak o jednym utworze)}
\Clue{27}{}{rodzaj gry fabularnej na pograniczu zabawy i sztuki; polega na wcielaniu się w postacie i odgrywanie  ról}\end{PuzzleClues}\newpage\section*{Krzyżówka 25}

\noindent\begin{Puzzle}{22}{32}|*	|*	|*	|*	|*	|*	|*	|*	|*	|*	|*	|*	|[1][S]\drarr	|l	|i	|m	|f	|o	|c	|y	|t	|*	|*	|.
|*	|*	|*	|*	|*	|*	|*	|[2][S]\rarr	|w	|y	|c	|i	|e	|r	|a	|c	|z	|k	|a	|*	|*	|*	|*	|.
|*	|*	|*	|*	|[3][S]\darr	|[4][S]\darr	|[5][S]\drarr	|c	|z	|e	|k	|o	|l	|a	|d	|z	|i	|a	|r	|n	|i	|a	|*	|.
|*	|*	|*	|*	|k	|b	|g	|*	|[6][S]\drarr	|n	|a	|j	|e	|m	|n	|i	|c	|t	|w	|o	|*	|*	|*	|.
|*	|*	|*	|*	|a	|r	|i	|*	|o	|[7][S]\drarr	|s	|a	|m	|o	|t	|n	|i	|c	|z	|o	|ś	|ć	|*	|.
|*	|*	|*	|*	|s	|u	|r	|[8][S]\darr	|ś	|r	|*	|*	|e	|*	|[9][S]\darr	|[10][S]\drarr	|l	|e	|k	|a	|r	|z	|*	|.
|*	|*	|*	|*	|z	|k	|l	|s	|m	|u	|*	|[11][S]\darr	|n	|*	|n	|t	|*	|*	|*	|*	|[12][S]\darr	|*	|*	|.
|*	|*	|*	|*	|a	|[][,]{ }	|a	|m	|l	|c	|*	|k	|t	|*	|o	|o	|*	|[13][S]\darr	|[14][S]\rarr	|g	|o	|*	|[15][S]\darr	|.
|*	|*	|*	|*	|[][,]{ }	|m	|n	|ó	|n	|h	|*	|m	|a	|*	|o	|w	|*	|s	|*	|[16][S]\darr	|d	|*	|s	|.
|*	|*	|*	|[17][S]\drarr	|j	|o	|d	|ł	|a	|[][,]{ }	|b	|o	|r	|y	|s	|a	|*	|t	|*	|o	|w	|*	|t	|.
|*	|[18][S]\darr	|*	|t	|a	|r	|a	|k	|*	|w	|*	|t	|z	|*	|f	|r	|*	|e	|*	|r	|r	|[19][S]\darr	|r	|.
|*	|w	|*	|y	|g	|e	|*	|a	|*	|s	|*	|r	|*	|*	|e	|z	|*	|e	|*	|a	|ó	|w	|o	|.
|*	|e	|*	|m	|l	|n	|*	|*	|*	|t	|[20][S]\darr	|*	|*	|*	|r	|y	|[21][S]\darr	|l	|*	|t	|c	|y	|p	|.
|*	|ł	|*	|a	|a	|o	|*	|*	|*	|e	|s	|*	|[22][S]\rarr	|f	|a	|s	|t	|[][,]{ }	|f	|o	|o	|d	|*	|.
|*	|n	|*	|l	|n	|w	|[23][S]\rarr	|m	|a	|c	|a	|n	|k	|a	|*	|t	|r	|p	|*	|r	|n	|a	|*	|.
|*	|a	|*	|o	|a	|y	|*	|*	|*	|z	|b	|[24][S]\darr	|*	|*	|*	|w	|y	|a	|*	|i	|a	|w	|*	|.
|*	|[][,]{ }	|*	|n	|*	|*	|*	|*	|*	|n	|i	|r	|*	|*	|*	|o	|g	|n	|[25][S]\darr	|a	|[][,]{ }	|n	|*	|.
|*	|m	|*	|e	|[26][S]\rarr	|m	|u	|r	|z	|y	|n	|e	|k	|*	|*	|[][,]{ }	|e	|*	|p	|n	|h	|i	|*	|.
|*	|e	|[27][S]\rarr	|k	|u	|l	|f	|o	|n	|*	|*	|*	|*	|*	|*	|k	|r	|*	|o	|i	|i	|c	|*	|.
|*	|r	|[28][S]\darr	|[][,]{ }	|[29][S]\rarr	|a	|l	|l	|o	|p	|u	|r	|y	|n	|o	|l	|*	|*	|t	|n	|p	|t	|*	|.
|*	|y	|k	|r	|*	|*	|[30][S]\rarr	|s	|t	|e	|r	|y	|l	|i	|z	|a	|c	|j	|a	|*	|o	|w	|*	|.
|*	|n	|a	|d	|*	|*	|*	|*	|*	|*	|[31][S]\darr	|[32][S]\rarr	|w	|e	|i	|s	|s	|*	|ż	|*	|t	|o	|*	|.
|*	|o	|s	|z	|[33][S]\darr	|*	|*	|*	|[34][S]\rarr	|k	|r	|a	|w	|c	|z	|y	|n	|i	|*	|*	|e	|[][,]{ }	|*	|.
|[35][S]\rarr	|s	|z	|a	|m	|p	|a	|n	|*	|*	|a	|*	|*	|*	|*	|f	|[36][S]\darr	|*	|*	|*	|k	|z	|*	|.
|*	|o	|t	|w	|o	|*	|*	|*	|*	|*	|d	|[37][S]\rarr	|d	|u	|b	|i	|s	|k	|a	|*	|a	|w	|*	|.
|*	|w	|a	|y	|c	|[38][S]\rarr	|p	|ł	|y	|t	|a	|[][,]{ }	|k	|o	|r	|k	|o	|w	|a	|*	|*	|a	|*	|.
|*	|a	|n	|*	|*	|*	|*	|*	|*	|*	|*	|*	|*	|*	|[39][S]\rarr	|a	|k	|e	|r	|s	|*	|r	|*	|.
|*	|*	|*	|*	|*	|[40][S]\rarr	|n	|a	|r	|o	|d	|o	|w	|i	|e	|c	|*	|*	|[41][S]\rarr	|k	|e	|t	|*	|.
|*	|*	|*	|[42][S]\rarr	|p	|e	|n	|i	|t	|e	|n	|c	|j	|a	|r	|y	|s	|t	|k	|a	|*	|e	|*	|.
|[43][S]\rarr	|c	|y	|k	|l	|[][,]{ }	|i	|n	|w	|e	|s	|t	|y	|c	|y	|j	|n	|y	|*	|*	|*	|*	|*	|.
|[44][S]\rarr	|g	|r	|u	|p	|a	|[][,]{ }	|d	|y	|s	|k	|u	|s	|y	|j	|n	|a	|*	|*	|*	|*	|*	|*	|.
|[45][S]\rarr	|o	|g	|n	|i	|w	|o	|[][,]{ }	|w	|z	|o	|r	|c	|o	|w	|e	|*	|*	|*	|*	|*	|*	|*	|.
|*	|*	|*	|*	|*	|*	|*	|*	|*	|*	|[46][S]\rarr	|a	|ż	|u	|r	|*	|*	|*	|*	|*	|*	|*	|*	|.\end{Puzzle}

\newpage

\begin{PuzzleClues}{\textbf{Poziome}\\}\Clue{1}{}{komórka układu odpornościowego należąca do agranulocytów z grupy leukocytów, uczestnicząca i będąca podstawą odpowiedzi odpornościowej swoistej}
\Clue{2}{}{rzecz, o którą wyciera się podeszwy butów, przed przejściem przez drzwi; zbudowana zazwyczaj z materiału, gumy lub tworzywa sztucznego}
\Clue{5}{}{kawiarnia, w której sprzedaje się m.in. różne desery z czekolady i z czekoladą, na przykład czekolady do picia na gorąco, ciasta czekoladowe, napoje z czekoladą, czekoladowe słodycze}
\Clue{6}{}{wynajmowanie się do pracy, paranie się najemnictwem}
\Clue{7}{}{bycie samotnikiem, cecha istoty, która żyje sama, chce być sama}
\Clue{10}{}{osoba posiadająca wiedzę i uprawnienia do leczenia ludzi; tytuł zawodowy nadawany absolwentom studiów medycznych na kierunku lekarskim}
\Clue{14}{}{starochińska gra planszowa popularna również w Korei i Japonii, a w ostatnich latach zdobywająca rosnącą popularność na całym świecie (w tym także w Polsce)}
\Clue{17}{}{jodła bułgarska, Abies borisii-regis - gatunek drzew z rodziny sosnowatych; występuje na terenach górskich, w Bułgarii, Albanii i w północnej Grecji}
\Clue{22}{}{szybkie w przygotowaniu jedzenie, które dostarcza bardzo niewiele wartości odżywczych}
\Clue{23}{}{włoska przystawka w formie naleśnika}
\Clue{26}{}{w Krakowie i okolicach: czarna kawa}
\Clue{27}{}{noga człowieka określana z niechęcią, niezgrabna}
\Clue{29}{}{pochodna puryny, izomer hipoksantyny, inhibitor oksydazy ksantynowej, zmniejsza wytwarzanie ksantyn i kwasu moczowego}
\Clue{30}{}{proces technologiczny polegający na zniszczeniu wszystkich, zarówno wegetatywnych, jak i przetrwalnikowych oraz zarodnikowych, form mikroorganizmów}
\Clue{32}{}{Peter, ur. 1916r, pisarz niemiecki, dramaty, proza, eseista; „Męczeństwo i śmierć Jean Paul Marata”, „Dochodzenie”}
\Clue{34}{}{krawcowa}
\Clue{35}{}{białe wino wytrawne produkowane w Szampanii}
\Clue{37}{}{skrzypaczka (1899-1989); pedagog, profesor Akademii Muzycznej w Warszawie}
\Clue{38}{}{płyta, którą wytwarza się przez zlepianie pod ciśnieniem lub przez ekspandowanie surowca pozyskanego z drzew dających korek (m.in. dębu korkowego)}
\Clue{39}{}{astronauta amerykański na pokładzie Discovery}
\Clue{40}{}{zwolennik nacjonalizmu}
\Clue{41}{}{rodzaj ożaglowania łodzi lub jachtu}
\Clue{42}{}{kobieta penitencjarysta; specjalista do spraw więziennictwa, reedukacji więźniów oraz stosowania odpowiedniego systemu kar}
\Clue{43}{}{okres obejmujący trzy etapy działalności inwestycyjnej: przygotowanie inwestycji, jej wykonanie oraz odbiór}
\Clue{44}{}{forma wymiany poglądów w sieci}
\Clue{45}{}{ogniwo o napięciu bliskim 1 V stosowane jako wzorzec jednostki miary napięcia elektrycznego}
\Clue{46}{}{ozdobny układ otworów stosowany w architekturze}\end{PuzzleClues}

\begin{PuzzleClues}{\textbf{Pionowe}\\}\Clue{1}{}{podstawy, elementarne wiadomości, ABC}
\Clue{3}{}{kasza z łuskanego ziarna prosa}
\Clue{4}{}{nagromadzenie głazów pozostałe po wypłukaniu przez wodę drobnych składników moreny lodowcowej (residuum)}
\Clue{5}{}{ornament w formie wieńca lub wiązanki kwiatów}
\Clue{6}{}{ÓSMAK, ACHTEL}
\Clue{7}{}{obrót obiektu orbitującego większe ciało niebieskie w kierunku przeciwnym do ruchu obrotowego większego ciała}
\Clue{8}{}{pierwsze stolce w życiu noworodka o dużej gęstości, lepkości i ciemnym zabarwieniu zanikające około czwartej doby życia}
\Clue{9}{}{sfera ludzkiego rozumu; pojęcie wprowadzone przez Vernadsky'ego jako trzecia - po geosferze i biosferze faza rozwoju Ziemi}
\Clue{10}{}{organizacja rzeczoznawcza zajmująca się głównie klasyfikacją jednostek pływających}
\Clue{11}{}{ojciec chrzestny, mężczyzna przedstawiający do chrztu osobę chrzczoną (najczęściej dziecko)}
\Clue{12}{}{rodzaj kredytu hipotecznego, w którym kredytobiorca otrzymuje wypłatę w formie dożywotniej renty}
\Clue{13}{}{instrument muzyczny wywodzący się z wysp Trynidad i Tobago na Karaibach; ma formę blaszanych bębnów posiadających szereg wgłębień, które umożliwiają wydobycie różnych dźwięków}
\Clue{15}{}{górna warstwa pokładu}
\Clue{16}{}{członek Kongregacji Oratorium św. Filipa Neri}
\Clue{17}{}{Timeliopsis griseigula - gatunek ptaka z rodziny miodojadów (Meliphagidae) występujący na Nowej Gwinei}
\Clue{18}{}{wełna wytwarzana z runa owiec merynosowych}
\Clue{19}{}{publikacja ukazująca się jednokrotnie, zaplanowana przez wydawcę lub autora jako całość wydawnicza składająca się z określonej (z góry zaplanowanej) liczby części (wydanych jednocześnie lub niejednocześnie) i rozpowszechniana w dowolnej formie produktu}
\Clue{20}{}{Sabińczyk - członek jednego z plemion sabelskich, zamieszkujących środkową Italię}
\Clue{21}{}{elektroniczny układ spustowy}
\Clue{24}{}{w chemii: symbol renu}
\Clue{25}{}{zanieczyszczona postać węglanu potasu K2CO3, rozpuszczalna w wodzie część popiołu pochodzącego ze spalania węgla drzewnego, zawierająca również zmienne ilości innych związków potasu}
\Clue{28}{}{drewno pozyskiwane z drzewa o tej samej nazwie, zazwyczaj jest to kasztan amerykański; używane do produkcji drewna opałowego}
\Clue{31}{}{grupa ludzi, wybrana, żeby radzić, rządzić (np. rada starszych)}
\Clue{33}{}{skalarna wielkość fizyczna określająca pracę wykonaną w jednostce czasu przez układ fizyczny}
\Clue{36}{}{porcja soku, tyle, ile mieści się w jakimś naczyniu lub opakowaniu}\end{PuzzleClues}\newpage\section*{Krzyżówka 26}

\noindent\begin{Puzzle}{19}{23}|*	|*	|*	|*	|*	|*	|*	|*	|*	|*	|*	|*	|*	|*	|*	|*	|[1][S]\darr	|*	|*	|*	|.
|*	|[2][S]\drarr	|o	|c	|z	|k	|o	|*	|[3][S]\drarr	|w	|a	|ł	|*	|[4][S]\darr	|*	|[5][S]\drarr	|r	|a	|j	|*	|.
|*	|s	|[6][S]\drarr	|t	|o	|m	|b	|a	|k	|*	|*	|[7][S]\darr	|*	|n	|[8][S]\darr	|g	|o	|[9][S]\darr	|*	|*	|.
|*	|i	|p	|*	|*	|[10][S]\darr	|*	|*	|i	|*	|[11][S]\darr	|p	|[12][S]\darr	|i	|k	|o	|z	|p	|*	|*	|.
|*	|l	|r	|*	|*	|d	|[13][S]\darr	|*	|t	|*	|n	|r	|n	|e	|w	|g	|ś	|i	|*	|*	|.
|*	|n	|z	|*	|*	|e	|y	|[14][S]\darr	|e	|*	|e	|z	|a	|p	|i	|h	|w	|e	|*	|*	|.
|*	|i	|e	|*	|[15][S]\drarr	|m	|a	|s	|s	|[][,]{ }	|m	|e	|d	|i	|a	|*	|i	|r	|*	|[16][S]\darr	|.
|*	|k	|n	|[17][S]\darr	|p	|o	|k	|y	|u	|*	|e	|z	|r	|ś	|t	|*	|t	|ś	|[18][S]\darr	|g	|.
|*	|[][,]{ }	|i	|r	|a	|*	|u	|r	|r	|[19][S]\darr	|g	|i	|e	|m	|u	|*	|*	|c	|p	|o	|.
|*	|p	|k	|e	|u	|[20][S]\darr	|z	|t	|f	|a	|t	|e	|n	|i	|s	|*	|*	|i	|e	|l	|.
|*	|y	|l	|l	|l	|i	|a	|y	|e	|r	|o	|r	|i	|e	|z	|*	|*	|e	|n	|o	|.
|*	|ł	|i	|a	|o	|z	|*	|k	|r	|n	|m	|n	|a	|n	|e	|*	|[21][S]\darr	|ń	|i	|n	|.
|*	|o	|w	|c	|w	|b	|*	|a	|*	|o	|a	|i	|*	|n	|k	|[22][S]\darr	|a	|[][,]{ }	|c	|e	|.
|[23][S]\drarr	|w	|o	|j	|n	|a	|r	|*	|*	|u	|j	|k	|*	|y	|*	|a	|p	|z	|y	|c	|.
|s	|y	|ś	|a	|i	|[][,]{ }	|[24][S]\rarr	|p	|u	|l	|a	|*	|*	|*	|*	|s	|o	|e	|l	|z	|.
|z	|*	|ć	|*	|a	|p	|*	|*	|*	|d	|*	|*	|*	|*	|*	|e	|s	|r	|i	|k	|.
|t	|*	|*	|*	|*	|o	|*	|*	|*	|*	|*	|*	|*	|*	|*	|p	|t	|o	|n	|a	|.
|a	|*	|*	|*	|[25][S]\rarr	|s	|t	|a	|w	|k	|a	|[][,]{ }	|k	|w	|o	|t	|o	|w	|a	|*	|.
|j	|*	|[26][S]\rarr	|b	|i	|e	|l	|a	|w	|i	|a	|n	|k	|a	|*	|y	|l	|y	|z	|*	|.
|e	|[27][S]\rarr	|t	|h	|u	|l	|e	|*	|*	|*	|*	|*	|*	|*	|*	|k	|a	|*	|a	|*	|.
|r	|*	|[28][S]\rarr	|a	|k	|s	|a	|m	|i	|t	|n	|i	|k	|o	|w	|a	|t	|e	|*	|*	|.
|e	|*	|*	|[29][S]\rarr	|u	|k	|r	|a	|i	|ń	|s	|k	|o	|ś	|ć	|*	|*	|*	|*	|*	|.
|k	|*	|*	|[30][S]\rarr	|z	|a	|k	|w	|a	|t	|e	|r	|o	|w	|a	|n	|i	|e	|*	|*	|.
|*	|[31][S]\rarr	|c	|e	|z	|*	|*	|*	|*	|*	|*	|*	|*	|*	|*	|*	|*	|*	|*	|*	|.\end{Puzzle}

\newpage

\begin{PuzzleClues}{\textbf{Poziome}\\}\Clue{2}{}{kropla tłuszczu na powierzchni innego płynu}
\Clue{3}{}{frajer, głupek}
\Clue{5}{}{miejce sprzyjające np. interesom}
\Clue{6}{}{stop miedzi z cynkiem zawierający powyżej 80\% miedzi}
\Clue{15}{}{środki społecznego komunikowania o szerokim zasięgu, czyli prasa, radio, telewizja, Internet, a w szerszym znaczeniu także książka, film, plakat, kino}
\Clue{23}{}{saneczkarz i pilot szybowcowy, mistrz świata w saneczkach z 1951 i 61 r., rekordzista świata w szybowcowych przelotach}
\Clue{24}{}{miasto w Chorwacji, ważny port nad Morzem Adriatyckim, na półwyspie Istra, katedra z V-VI w}
\Clue{25}{}{stawka do zapłaty obliczana od wysokości podatku}
\Clue{26}{}{mieszkanka Bielawy}
\Clue{27}{}{miasto w Grenlandii u wybrzeży Morza Baffina}
\Clue{28}{}{Clubionidae - rodzina pająków z podrzędu Opisthothelae}
\Clue{29}{}{fakt, że coś jest ukraińskie, zwłaszcza: pochodzi z Ukrainy, należy do Ukrainy, przynależy do ukraińskiej kultury}
\Clue{30}{}{miejsce krótkotrwałego zamieszkania}
\Clue{31}{}{pierwiastek chemiczny, metal alkaliczny}\end{PuzzleClues}

\begin{PuzzleClues}{\textbf{Pionowe}\\}\Clue{1}{}{chwila przed wschodem słońca}
\Clue{2}{}{silnik spalinowy na paliwo stałe doprowadzane w postaci pyłu}
\Clue{3}{}{osoba uprawiająca kitesurfing}
\Clue{4}{}{osoba dorosła, która nie umie pisać ani czytać}
\Clue{5}{}{malarz holenderskierski , postimpresjonista (1853-90), portrety, pejzaże, sceny rodzajowe, martwe natury; 'Słoneczniki'}
\Clue{6}{}{właściwa człowiekowi bystrość, inteligencja, domyślność, umiejętność dostrzegania istoty zjawisk, ważnych aspektów problemów}
\Clue{7}{}{mały otwór w drzwiach lub bramie, który służy do wyglądania}
\Clue{8}{}{mały cięty (lub urwany) kwiat}
\Clue{9}{}{pierścień, którego działanie multiplikatywne (mnożenie) daje zawsze w wyniku element neutralny działania addytywnego (zero)}
\Clue{10}{}{demonstracyjna wersja programu komputerowego}
\Clue{11}{}{Nemegtomaia - rodzaj teropoda z rodziny owiraptorów; żył w epoce późnej kredy na terenach centralnej Azji}
\Clue{12}{}{PALATYNAT- kraj związkowy w zach części Niemiec, obszar 19,8tyś. km2, główne miasta: Moguncja (stolica) Koblencja, Trewir}
\Clue{13}{}{rodzaj tradycyjnych grup przestępczych w Japonii}
\Clue{14}{}{kraina w Libii, u wybrzeży Morza Śródziemnego}
\Clue{15}{}{drewno pozyskiwane z drzewa o tej samej nazwie; wykorzystywane do budowy instrumentów i przedmiotów codziennego użytku oraz przemysłu papierniczego}
\Clue{16}{}{element mięsa wieprzowego, fragment nogi odciętej od szynki na wysokości 1/3 kości goleni}
\Clue{17}{}{opowiadanie, sprawozdanie}
\Clue{18}{}{enzym należący do beta-laktamaz, warunkujących oporność niektórych bakterii na naturalne penicyliny}
\Clue{19}{}{światowej sławy śpiewaczka francuska (1740-1802)}
\Clue{20}{}{rodzaj niższej izby parlamentu}
\Clue{21}{}{działalność apostolska, krzewienie wiary; apostolstwo}
\Clue{22}{}{postępowanie mające na celu dążenie do jałowości pomieszczeń, narzędzi, materiałów opatrunkowych i innych przedmiotów w celu niedopuszczenia drobnoustrojów do określonego środowiska, np. otwartej rany}
\Clue{23}{}{rodzaj żywej polki}\end{PuzzleClues}\newpage\section*{Krzyżówka 27}

\noindent\begin{Puzzle}{23}{23}|*	|*	|*	|*	|*	|*	|*	|*	|*	|*	|*	|*	|*	|*	|*	|[1][S]\darr	|*	|*	|*	|[2][S]\drarr	|u	|*	|*	|[3][S]\darr	|.
|*	|[4][S]\drarr	|s	|p	|o	|i	|w	|o	|[][,]{ }	|m	|a	|l	|a	|r	|s	|k	|i	|e	|*	|p	|*	|[5][S]\darr	|[6][S]\darr	|o	|.
|[7][S]\drarr	|k	|o	|m	|i	|n	|o	|w	|e	|*	|[8][S]\drarr	|n	|a	|m	|i	|a	|r	|*	|*	|u	|*	|d	|d	|b	|.
|z	|o	|*	|[9][S]\rarr	|s	|k	|ł	|a	|d	|o	|w	|a	|*	|[10][S]\drarr	|f	|r	|a	|k	|*	|s	|[11][S]\darr	|z	|i	|r	|.
|ł	|n	|*	|*	|[12][S]\rarr	|p	|l	|a	|s	|t	|y	|k	|*	|d	|*	|a	|*	|*	|*	|z	|o	|i	|e	|ę	|.
|o	|s	|[13][S]\drarr	|l	|e	|d	|a	|*	|[14][S]\darr	|*	|r	|[15][S]\rarr	|w	|r	|ó	|b	|e	|l	|*	|k	|d	|e	|t	|b	|.
|t	|t	|d	|[16][S]\drarr	|k	|o	|m	|ó	|r	|k	|a	|[][,]{ }	|f	|a	|g	|o	|c	|y	|t	|a	|r	|n	|a	|*	|.
|l	|a	|a	|ż	|[17][S]\rarr	|v	|e	|l	|a	|*	|ż	|*	|*	|g	|*	|n	|[18][S]\rarr	|h	|u	|r	|o	|n	|*	|[19][S]\darr	|.
|i	|b	|w	|e	|*	|[20][S]\rarr	|b	|u	|n	|d	|e	|s	|r	|a	|t	|*	|*	|*	|*	|z	|b	|i	|[21][S]\darr	|d	|.
|n	|l	|k	|r	|*	|*	|*	|*	|w	|*	|n	|*	|*	|*	|[22][S]\rarr	|k	|e	|n	|t	|*	|i	|k	|d	|r	|.
|[][,]{ }	|*	|a	|d	|[23][S]\rarr	|b	|e	|b	|e	|*	|i	|*	|*	|*	|[24][S]\rarr	|a	|y	|r	|t	|o	|n	|*	|u	|y	|.
|j	|*	|[][,]{ }	|z	|[25][S]\darr	|[26][S]\drarr	|u	|p	|r	|z	|e	|j	|m	|o	|s	|t	|k	|a	|*	|*	|k	|*	|f	|l	|.
|a	|*	|ś	|i	|a	|s	|*	|*	|s	|*	|[][,]{ }	|*	|*	|*	|[27][S]\rarr	|s	|e	|r	|e	|n	|a	|d	|a	|*	|.
|p	|[28][S]\drarr	|m	|o	|l	|t	|o	|n	|*	|[29][S]\rarr	|p	|u	|*	|*	|*	|[30][S]\rarr	|h	|e	|j	|t	|*	|*	|y	|*	|.
|o	|w	|i	|w	|l	|e	|*	|*	|[31][S]\rarr	|t	|r	|a	|g	|i	|g	|r	|o	|t	|e	|s	|k	|a	|*	|*	|.
|ń	|ę	|e	|i	|e	|n	|*	|*	|[32][S]\rarr	|s	|z	|c	|z	|u	|r	|[][,]{ }	|ś	|n	|i	|a	|d	|y	|*	|*	|.
|s	|g	|r	|n	|l	|*	|[33][S]\rarr	|r	|e	|z	|y	|d	|e	|n	|t	|[][,]{ }	|w	|y	|w	|i	|a	|d	|u	|*	|.
|k	|i	|t	|a	|o	|*	|*	|*	|[34][S]\rarr	|m	|i	|ę	|k	|k	|i	|e	|[][,]{ }	|s	|e	|r	|c	|e	|*	|[35][S]\darr	|.
|i	|e	|e	|*	|p	|[36][S]\rarr	|k	|o	|ń	|[][,]{ }	|m	|o	|r	|s	|k	|i	|*	|*	|*	|*	|*	|*	|*	|u	|.
|*	|r	|l	|*	|a	|*	|[37][S]\rarr	|n	|i	|s	|k	|o	|ś	|ć	|*	|*	|*	|*	|*	|*	|*	|*	|*	|f	|.
|*	|k	|n	|*	|t	|*	|*	|[38][S]\rarr	|n	|e	|o	|k	|a	|p	|i	|t	|a	|l	|i	|z	|m	|*	|*	|a	|.
|*	|a	|a	|*	|i	|*	|*	|[39][S]\rarr	|n	|o	|w	|o	|g	|r	|ó	|d	|[][,]{ }	|w	|i	|e	|l	|k	|i	|*	|.
|*	|*	|*	|*	|a	|*	|[40][S]\rarr	|p	|r	|z	|e	|w	|o	|d	|n	|i	|c	|z	|k	|a	|*	|*	|*	|*	|.
|*	|*	|*	|*	|*	|[41][S]\rarr	|b	|a	|t	|h	|*	|*	|*	|*	|*	|*	|*	|*	|*	|*	|*	|*	|*	|*	|.\end{Puzzle}

\newpage

\begin{PuzzleClues}{\textbf{Poziome}\\}\Clue{2}{}{w chemii: symbol uranu}
\Clue{4}{}{zawiesina, która łączy ze sobą cząsteczki pigmentu, tworząc farbę}
\Clue{7}{}{opłata za usługę czyszczenia komina}
\Clue{8}{}{lokalizacja kogoś lub czegoś względem namierzającego}
\Clue{9}{}{część składająca się na jakąś większą całość, stanowiąca integralny komponent całości}
\Clue{10}{}{uroczysty wieczorowy strój męski - rodzaj surduta o obciętych z przodu i wydłużonych z tyłu połach}
\Clue{12}{}{Ksawery, brat Henryka (1843-1902) malarz i rysownik; ilustracje do książek, sceny rodzajowe, obrazy religijne, portrety}
\Clue{13}{}{satelita Jowisza}
\Clue{15}{}{potoczna nazwa wróbla zwyczajnego, najczęściej spotykanego w faunie Polski}
\Clue{16}{}{komórka w przedniej części jamy ciała nicieni wzdłuż bocznych wałków wydalniczych, zatrzymująca nierozpuszczalne produkty przemiany materii, które dostały się do organizmu}
\Clue{17}{}{ŻAGIEL}
\Clue{18}{}{jezioro w Kanadzie i USA, drugie pod względem wielkości w grupie Wielkich Jezior, powierzchnia 59,6 tyś. km2}
\Clue{20}{}{izba parlamentu w Niemczech, konstytucyjne przedstawicielstwo krajów związkowych}
\Clue{22}{}{angielska owca mięsno-wełnista}
\Clue{23}{}{kołnierzyk mający zaokrąglone rogi}
\Clue{24}{}{angielski fizyk, elektrotechnik (1847-1908); konstruktor elektrycznego pojazdu drogowego oraz przyrządów pomiarowych}
\Clue{26}{}{drobna uprzejmość}
\Clue{27}{}{pieśń miłosna śpiewana przy akompaniamencie np. gitary o zmroku pod oknem ukochanej kobiety}
\Clue{28}{}{miękka tkanina bawełniana, drapana i strzyżona, używana na zimową bieliznę}
\Clue{29}{}{w chemii: symbol plutonu}
\Clue{30}{}{osoba, która hejtuje - wyraża złe i złośliwe opinie o czymś lub o kimś w internecie}
\Clue{31}{}{odmiana groteski}
\Clue{32}{}{Rattus rattus - gatunek gryzonia z rodziny myszowatych; prawdopodobnie pochodzi z południowo-wschodniej Azji, w Polsce występuje na nielicznych stanowiskach w dorzeczu Odry i w niektórych portach na wybrzeżu Bałtyku, ale jego zasięg nie jest dokładnie znany}
\Clue{33}{}{funkcjonariusz wywiadu kierujący rezydenturą wywiadu za granicą pod przykrywką dyplomatyczną, jeżeli jest to rezydentura legalna}
\Clue{34}{}{cecha kogoś, kto jest dobroduszny, łatwo ustępuje, łatwo poddaje się pewnym wpływom, jest altruistą}
\Clue{36}{}{gatunek dużego drapieżnego ssaka morskiego, jedyny współcześnie żyjący przedstawiciel dawniej licznej w gatunki rodziny morsowatych (Odobenidae) i jedyny gatunek z rodzaju Odobenus}
\Clue{37}{}{mała ilość, np. dochodów}
\Clue{38}{}{współczesna odmiana kapitalizmu nieakcentująca wyraźnie podziału na klasy przedsiębiorców i pracowników}
\Clue{39}{}{miasto w północno-zachodniej Rosji nad rzeką Wołchow; stolica obwodu nowogrodzkiego (przydomek „Wielki” przywrócono oficjalnie w 1998 roku)}
\Clue{40}{}{kobieta, która zawodowo zajmuje się oprowadzaniem (np. turystów) albo niezawodowo wskazuje komuś drogę}
\Clue{41}{}{miasto w Anglii nad rzeką Avon, uzdrowisko znane już od starożytnych czasów (gorące źródła)}\end{PuzzleClues}

\begin{PuzzleClues}{\textbf{Pionowe}\\}\Clue{1}{}{zaprzęgowy pojazd wieloosobowy przeznaczony do dalekich podróży, z  krytym nadwoziem, używany w Polsce przez szlachtę w XVII-XVIII w}
\Clue{2}{}{rzemieślnik produkujący broń palną także artylerzysta}
\Clue{3}{}{przestrzeń ograniczona dobrze uchwytną granicą}
\Clue{4}{}{stopień lub stanowisko w wymiarze sprawiedliwości wielu krajów, m.in. w policji w Anglii i Australii}
\Clue{5}{}{zbiór publikacji prasowych ukazujący się codziennie}
\Clue{6}{}{produkty jedzone przez kogoś najczęściej, jadłospis}
\Clue{7}{}{gatunek krzewu z rodziny różowatych (Rosaceae)}
\Clue{8}{}{połączenie przyimka z rzeczownikiem, liczebnikiem, przysłówkiem, przymiotnikiem (w użyciu rzeczownym) albo zaimkiem}
\Clue{10}{}{jednostka pływająca, której przeznaczeniem jest pogłębianie akwenów morskich i śródlądowych}
\Clue{11}{}{zdrobniale: odrobina - bardzo mało}
\Clue{13}{}{taka dawka promieniowania, która powoduje chorobę popromienną i w jej efekcie śmierć}
\Clue{14}{}{figura akrobacji lotniczej, polegająca na wykonaniu obrotu wokół końcówki skrzydła w locie wznoszącym (pionowym) o 180°, z przejściem do lotu nurkowego i wyprowadzeniem samolotu (szybowca) do lotu poziomego}
\Clue{16}{}{żerdzie zrobione z drewna o tej samej nazwie}
\Clue{19}{}{afrykańska małpa wąskonosa}
\Clue{21}{}{kompozytor franko-fiamandzki (1400-1474); msze motety, ronda, ballady}
\Clue{25}{}{szkodliwy lub korzystny wpływ substancji chemicznych, wydzielanych przez rośliny lub grzyby danego gatunku lub pochodzących z rozkładu tych roślin na organizmy w otoczeniu}
\Clue{26}{}{jednostka siły; 1000 niutonów}
\Clue{28}{}{rodzaj śliwki, owoc z drzewa odmiany o tej samej nazwie}
\Clue{35}{}{miasto w europejskiej części Federacji Rosyjskiej; stolica Baszkirii}\end{PuzzleClues}\newpage\section*{Krzyżówka 28}

\noindent\begin{Puzzle}{21}{31}|*	|*	|*	|*	|*	|*	|*	|*	|*	|*	|*	|*	|*	|*	|*	|*	|*	|*	|*	|[1][S]\darr	|[2][S]\darr	|*	|.
|*	|*	|*	|*	|*	|*	|*	|*	|*	|*	|*	|*	|*	|*	|*	|*	|*	|[3][S]\darr	|*	|n	|b	|*	|.
|*	|*	|[4][S]\darr	|*	|*	|*	|[5][S]\rarr	|k	|a	|l	|e	|n	|d	|a	|r	|i	|u	|m	|*	|i	|a	|*	|.
|*	|[6][S]\rarr	|k	|ą	|p	|i	|e	|l	|[][,]{ }	|s	|ł	|o	|n	|e	|c	|z	|n	|a	|*	|e	|n	|*	|.
|*	|[7][S]\rarr	|o	|d	|t	|y	|l	|c	|ó	|w	|k	|a	|*	|*	|*	|*	|*	|k	|*	|w	|t	|*	|.
|*	|*	|ś	|[8][S]\darr	|[9][S]\rarr	|d	|e	|p	|u	|t	|a	|t	|[][,]{ }	|w	|ę	|g	|l	|o	|w	|y	|*	|*	|.
|*	|[10][S]\drarr	|c	|z	|a	|p	|a	|[][,]{ }	|p	|o	|l	|a	|r	|n	|a	|*	|[11][S]\darr	|w	|*	|t	|[12][S]\darr	|*	|.
|[13][S]\drarr	|b	|i	|e	|d	|o	|t	|a	|*	|*	|*	|*	|[14][S]\darr	|*	|*	|[15][S]\darr	|k	|i	|[16][S]\darr	|r	|k	|*	|.
|p	|o	|ó	|y	|*	|*	|*	|*	|*	|*	|[17][S]\darr	|*	|d	|*	|*	|b	|o	|n	|p	|z	|o	|*	|.
|e	|l	|ł	|e	|*	|*	|[18][S]\darr	|[19][S]\drarr	|t	|h	|a	|l	|e	|*	|*	|o	|t	|a	|a	|y	|ł	|*	|.
|t	|e	|[][,]{ }	|r	|*	|*	|p	|s	|*	|*	|l	|*	|z	|*	|*	|ż	|e	|*	|p	|m	|o	|*	|.
|r	|s	|g	|*	|*	|*	|e	|i	|*	|[20][S]\drarr	|b	|l	|a	|c	|h	|o	|w	|n	|i	|a	|*	|*	|.
|e	|ł	|r	|*	|*	|*	|n	|ó	|[21][S]\darr	|a	|u	|*	|s	|*	|*	|d	|k	|*	|n	|ł	|*	|[22][S]\darr	|.
|l	|a	|e	|*	|*	|*	|t	|d	|w	|m	|m	|[23][S]\darr	|e	|*	|*	|r	|a	|*	|*	|o	|*	|j	|.
|[][,]{ }	|w	|k	|*	|*	|*	|a	|e	|y	|a	|i	|k	|m	|*	|[24][S]\darr	|z	|*	|*	|*	|ś	|*	|u	|.
|c	|i	|o	|*	|*	|*	|p	|m	|b	|p	|k	|o	|b	|*	|r	|e	|*	|*	|[25][S]\darr	|ć	|*	|j	|.
|z	|a	|k	|*	|*	|*	|l	|k	|r	|a	|*	|ź	|l	|*	|e	|w	|[26][S]\darr	|*	|l	|*	|*	|u	|.
|a	|n	|a	|*	|*	|*	|o	|a	|a	|*	|*	|l	|e	|[27][S]\darr	|k	|*	|t	|[28][S]\darr	|u	|[29][S]\darr	|[30][S]\darr	|y	|.
|r	|i	|t	|*	|*	|*	|i	|*	|n	|[31][S]\rarr	|n	|a	|r	|k	|o	|l	|e	|p	|s	|j	|a	|*	|.
|n	|n	|o	|*	|[32][S]\darr	|[33][S]\rarr	|d	|o	|k	|t	|o	|r	|*	|a	|m	|*	|r	|a	|i	|a	|l	|*	|.
|o	|*	|l	|*	|p	|*	|*	|*	|a	|*	|*	|z	|[34][S]\darr	|r	|p	|*	|m	|r	|t	|s	|u	|*	|.
|s	|*	|i	|*	|a	|*	|*	|*	|*	|*	|*	|[][,]{ }	|m	|a	|i	|[35][S]\darr	|o	|a	|a	|t	|m	|*	|.
|k	|*	|c	|*	|d	|[36][S]\darr	|*	|*	|*	|*	|*	|c	|o	|m	|l	|b	|j	|f	|n	|r	|i	|*	|.
|r	|*	|k	|*	|ó	|c	|*	|*	|*	|*	|*	|z	|s	|a	|a	|e	|o	|i	|o	|z	|n	|[37][S]\darr	|.
|z	|[38][S]\rarr	|i	|g	|ł	|a	|[][,]{ }	|m	|a	|g	|n	|e	|t	|y	|c	|z	|n	|a	|*	|ę	|i	|j	|.
|y	|*	|*	|*	|*	|l	|*	|*	|[39][S]\rarr	|m	|a	|r	|e	|*	|j	|a	|i	|*	|*	|b	|u	|a	|.
|d	|*	|*	|*	|*	|v	|*	|*	|*	|*	|*	|w	|k	|*	|a	|n	|z	|*	|*	|i	|m	|k	|.
|ł	|[40][S]\drarr	|b	|r	|i	|a	|n	|c	|o	|n	|*	|o	|*	|*	|*	|*	|a	|*	|*	|[][,]{ }	|*	|o	|.
|y	|c	|[41][S]\drarr	|a	|l	|d	|e	|r	|a	|m	|i	|n	|*	|*	|*	|*	|c	|*	|*	|n	|*	|ś	|.
|*	|a	|p	|*	|*	|o	|[42][S]\rarr	|p	|r	|e	|z	|y	|d	|e	|n	|c	|j	|a	|*	|o	|*	|ć	|.
|*	|p	|t	|*	|*	|s	|*	|*	|[43][S]\rarr	|k	|w	|*	|*	|*	|[44][S]\rarr	|c	|a	|b	|a	|s	|a	|*	|.
|*	|*	|*	|*	|*	|*	|[45][S]\rarr	|o	|d	|n	|a	|l	|a	|z	|c	|a	|*	|*	|*	|*	|*	|*	|.\end{Puzzle}

\newpage

\begin{PuzzleClues}{\textbf{Poziome}\\}\Clue{5}{}{spis chronologiczny jakichś zdarzeń}
\Clue{6}{}{opalanie się, świadome przebywanie w nasłonecznionym miejscu w celu poddania się działaniu promieni słonecznych}
\Clue{7}{}{strzelba myśliwska ładowana od tyłu}
\Clue{9}{}{przyznawane z budżetu państwa określonym grupom społecznym uposażenie w postaci węgla do ogrzania domu lub ekwiwalenu pieniężnego}
\Clue{10}{}{nagromadzenie lodu znajdujące się na biegunach planet}
\Clue{13}{}{człowiek, który ma mało, jest biedny}
\Clue{19}{}{miasto w Niemczech (Saksonia Anhart) u podnóża Harzu; uzdrowisko (Solanki)}
\Clue{20}{}{warsztat lub pracownia, gdzie dokonywana jest obróbka blachy}
\Clue{31}{}{zespół chorobowy z grupy dyssomni o nieznanej etiologii, klasyfikowany jednak w grupie dyssomni o podłożu organicznym, na obraz którego składa się tetrada objawów: nadmierna senność w ciągu dnia i napady snu, katapleksja, porażenie przysenne oraz omamy hipnagogiczne i omamy hipnopompiczne}
\Clue{33}{}{naukowiec, który ma tytuł doktora}
\Clue{38}{}{magnes trwały, zazwyczaj w kształcie wydłużonej linii, zamocowany tak by mógł się obracać wokół pionowej osi, używany do wskazywania kierunku linii pola magnetycznego}
\Clue{39}{}{angielski poeta i prozaik (1873-1956), liryka przesączona baśniową fantastyką}
\Clue{40}{}{miasto w płd.-wsch. Francji w Alpach, ośrodek turystyki i sportów zimowych}
\Clue{41}{}{najjaśniejsza gwiazda w obrębie gwiazdozbioru Cefeusza}
\Clue{42}{}{Przewodzenie, stanie na czele zebrania lub instytucji}
\Clue{43}{}{skrót, symbol jednostki - kilowata}
\Clue{44}{}{brazylijski instrument perkusyjny}
\Clue{45}{}{znalzaca - człowiek, który coś odnalazł lub odnajdzie}\end{PuzzleClues}

\begin{PuzzleClues}{\textbf{Pionowe}\\}\Clue{1}{}{słabość psychiczna, łatwe poddawanie się działaniu czegoś, nieodporność}
\Clue{2}{}{kawał płótna naszyty w poprzek i na rogach żagla w celu zwiększenia jego wytrzymałości}
\Clue{3}{}{wysuszona i opróżniona z nasion torebka nasienna maku}
\Clue{4}{}{Kościoły (Cerkwie) katolickie tradycji wschodniej}
\Clue{8}{}{pisarz czeski, neoromantyk (1841-1901), powieści, opowiadania, poezje, nowele, dramaty}
\Clue{10}{}{mieszkaniec Bolesławia albo człowiek pochodzący z Bolesławia}
\Clue{11}{}{zdrobniale o kotwie; ciężki metalowy hak, który wbija się w dno i utrzymuje statek w miejscu}
\Clue{12}{}{w dawnej Polsce: zebranie, sejmik}
\Clue{13}{}{Pterodroma nigripennis - gatunek ptaka z rodziny burzykowatych (Procellariidae)}
\Clue{14}{}{program komputerowy, który tłumaczy język maszynowy lub kod bajtowy na język asemblera; disasembler jest niskopoziomowym odpowiednikiem dekompilatora}
\Clue{15}{}{AILANT}
\Clue{16}{}{francuski fizyk i wynalazca (1614-1714); wynalazł autoklaw, piec do topienia szkła}
\Clue{17}{}{mały zeszyt, rodzaj albumu, w którym zbierało się pamiątkowe wpisy (zwykle wierszyki, obrazki itp.) od przyjaciół}
\Clue{18}{}{organizm, u którego zestaw chromosomów jest powielony pięciokrotnie}
\Clue{19}{}{coś lub ktoś oznaczone siódemką, noszące taki numer}
\Clue{20}{}{terytorium federalne w północnej Brazylii, nad Oceanem Atlantyckim ośrodek administracyjny Macapa}
\Clue{21}{}{ukochana kobieta lub dziewczyna}
\Clue{22}{}{miasto w Argentynie, w Andach, stolica prowincji Jujuy}
\Clue{23}{}{Leccinum aurantiacum -  gatunek grzybów z rodziny borowikowatych (Boletaceae)}
\Clue{24}{}{ponowienie kompilacji, które wymagane zawsze, gdy kod źródłowy programu został zmieniony bądź gdy program jest przenoszony na inną architekturę procesora lub inny system operacyjny}
\Clue{25}{}{koń luzytański - jedna z gorącokrwistych ras konia domowego pochodząca z terenów Ribatejo i Alentejo w Portugalii; powstała na skutek inwazji Maurów na Hiszpanię, gdy kuce Sorraia połączyły się z krwią koni berberyjskich dosiadanych przez zbrojnych z Afryki Północnej}
\Clue{26}{}{jonizacja atomów bądź cząsteczek gazu lub pary spowodowana działaniem wysokiej temperatury}
\Clue{27}{}{miasto w Chinach (Xinjang) w Kotlinie Dżungarskiej}
\Clue{28}{}{osiedle lub jego część, miejsce, gdzie mieszka jakieś środowisko, grupa ludzi zżytych ze sobą}
\Clue{29}{}{nos, który jest lekko zakrzywiony i wydatny}
\Clue{30}{}{nazwa glinu używana w technice}
\Clue{32}{}{świat, Ziemia}
\Clue{34}{}{rodzaj małej protezy dentystycznej, która jest mocowana na sąsiadujących z ubytkiem, zdrowych zębach}
\Clue{35}{}{nazwa tylnego żagla na jednostce żaglowej trzy- lub więcej-masztowej; jeżeli statek ma 2 maszty, to ostatni może być nazwany bezanem tylko wtedy, gdy pierwszy to grotmaszt}
\Clue{36}{}{francuska wódka produkowana z jabłek; winiak jabłkowy}
\Clue{37}{}{coś; pewne cechy, które składają się na byt, wyróżniają go spośród innych}
\Clue{40}{}{czop - rodzaj kołka}
\Clue{41}{}{w chemii: symbol platyny}\end{PuzzleClues}\newpage\section*{Krzyżówka 29}

\noindent\begin{Puzzle}{22}{27}|*	|*	|[1][S]\darr	|*	|*	|*	|*	|*	|*	|*	|*	|*	|*	|*	|*	|*	|*	|*	|*	|*	|[2][S]\darr	|[3][S]\darr	|[4][S]\darr	|.
|[5][S]\drarr	|w	|k	|ł	|a	|d	|k	|a	|*	|[6][S]\drarr	|m	|o	|d	|e	|r	|u	|n	|e	|k	|*	|d	|c	|d	|.
|s	|*	|o	|*	|*	|*	|*	|*	|[7][S]\rarr	|b	|r	|y	|g	|a	|n	|t	|*	|[8][S]\darr	|*	|*	|o	|h	|ę	|.
|i	|*	|m	|[9][S]\drarr	|c	|a	|p	|s	|t	|r	|z	|y	|k	|*	|*	|*	|*	|m	|*	|*	|m	|e	|b	|.
|e	|[10][S]\darr	|o	|t	|[11][S]\drarr	|p	|i	|s	|t	|o	|l	|e	|t	|[][,]{ }	|g	|a	|z	|o	|w	|y	|*	|m	|n	|.
|ć	|m	|d	|o	|p	|*	|*	|*	|*	|y	|*	|*	|[12][S]\drarr	|k	|a	|b	|a	|r	|e	|t	|*	|i	|i	|.
|*	|i	|o	|j	|a	|*	|*	|[13][S]\darr	|*	|*	|*	|[14][S]\rarr	|k	|o	|r	|a	|*	|f	|*	|*	|*	|k	|k	|.
|[15][S]\drarr	|k	|r	|a	|s	|n	|o	|p	|i	|ó	|r	|k	|a	|[][,]{ }	|t	|i	|m	|o	|r	|s	|k	|a	|*	|.
|k	|u	|*	|d	|j	|[16][S]\rarr	|s	|a	|n	|c	|h	|e	|z	|*	|[17][S]\drarr	|l	|e	|n	|*	|[18][S]\darr	|*	|l	|*	|.
|a	|l	|[19][S]\darr	|[][,]{ }	|a	|*	|*	|n	|*	|*	|[20][S]\drarr	|l	|a	|m	|e	|n	|t	|o	|s	|o	|*	|i	|*	|.
|b	|s	|p	|f	|*	|*	|*	|t	|*	|*	|u	|*	|r	|*	|d	|*	|*	|l	|*	|k	|*	|o	|*	|.
|i	|k	|o	|i	|*	|*	|[21][S]\darr	|e	|*	|*	|s	|*	|k	|*	|y	|*	|*	|o	|*	|r	|*	|w	|*	|.
|n	|i	|r	|s	|[22][S]\rarr	|o	|k	|o	|l	|c	|z	|*	|a	|*	|l	|*	|*	|g	|[23][S]\darr	|ę	|*	|i	|*	|.
|d	|*	|z	|c	|[24][S]\rarr	|p	|u	|n	|i	|c	|k	|i	|*	|*	|*	|*	|*	|i	|k	|g	|*	|e	|*	|.
|a	|*	|ą	|h	|*	|*	|l	|*	|[25][S]\rarr	|g	|o	|l	|e	|m	|*	|*	|[26][S]\darr	|a	|r	|*	|*	|c	|*	|.
|*	|[27][S]\drarr	|d	|e	|k	|*	|a	|[28][S]\rarr	|p	|o	|d	|r	|ó	|ż	|n	|i	|k	|*	|a	|*	|*	|*	|*	|.
|[29][S]\drarr	|h	|e	|r	|t	|z	|*	|[30][S]\rarr	|k	|s	|z	|t	|a	|ł	|t	|*	|a	|*	|j	|*	|*	|*	|*	|.
|c	|o	|k	|a	|[31][S]\drarr	|g	|o	|r	|ą	|c	|e	|[][,]{ }	|k	|r	|z	|e	|s	|ł	|a	|*	|*	|*	|*	|.
|h	|t	|[][,]{ }	|*	|r	|[32][S]\rarr	|s	|t	|a	|d	|n	|i	|a	|k	|i	|*	|z	|[33][S]\darr	|k	|*	|*	|*	|*	|.
|ł	|e	|d	|*	|e	|[34][S]\drarr	|p	|r	|o	|m	|i	|e	|ń	|[][,]{ }	|b	|e	|t	|a	|*	|*	|*	|*	|*	|.
|o	|n	|z	|*	|t	|g	|[35][S]\drarr	|p	|r	|z	|e	|ż	|y	|c	|i	|e	|*	|l	|*	|*	|*	|*	|*	|.
|n	|t	|i	|*	|o	|i	|e	|*	|*	|*	|*	|[36][S]\rarr	|l	|e	|ż	|a	|n	|i	|n	|a	|*	|*	|*	|.
|n	|o	|e	|*	|r	|g	|l	|*	|*	|*	|*	|*	|[37][S]\rarr	|s	|t	|r	|ą	|c	|z	|y	|n	|a	|*	|.
|o	|c	|n	|*	|y	|a	|i	|*	|*	|*	|*	|*	|*	|[38][S]\rarr	|p	|i	|l	|a	|t	|e	|s	|*	|*	|.
|ś	|k	|n	|*	|k	|n	|n	|*	|*	|*	|*	|[39][S]\rarr	|b	|y	|t	|o	|w	|n	|i	|t	|*	|*	|*	|.
|ć	|i	|y	|*	|a	|t	|g	|*	|*	|*	|*	|*	|*	|[40][S]\rarr	|e	|l	|i	|t	|y	|z	|m	|*	|*	|.
|*	|*	|*	|*	|*	|*	|*	|*	|*	|*	|*	|[41][S]\rarr	|d	|o	|d	|a	|t	|e	|k	|*	|*	|*	|*	|.
|*	|*	|*	|*	|*	|*	|*	|*	|*	|*	|*	|*	|*	|*	|*	|*	|*	|*	|*	|*	|*	|*	|*	|.\end{Puzzle}

\newpage

\begin{PuzzleClues}{\textbf{Poziome}\\}\Clue{5}{}{element, który się wkłada do jakiegoś większego przedmiotu}
\Clue{6}{}{uprząż konia}
\Clue{7}{}{członek oddziału niszczącego majątki magnatów w okresie rewolucji francuskiej}
\Clue{9}{}{wieczorny przemarsz wojska ulicami miasta}
\Clue{11}{}{broń na naboje wypełnione gazem obezwładniającym}
\Clue{12}{}{forma widowiska, często o charakterze satyrycznym, na którą składają się skecze, piosenki i monologi}
\Clue{14}{}{nepalska broń sieczna; rodzaj szabli o zakrzywionej jednosiecznej głowni}
\Clue{15}{}{Aprosmictus jonquillaceus - gatunek ptaka z rodziny papugowatych (Psittacidae), z podrodziny papug wschodnich (Psittaculinae)}
\Clue{16}{}{Hugo; piłkarz meksykański, napastnik m.in. Realu Madryt}
\Clue{17}{}{Linum - rodzaj dość pospolitych roślin zielnych krótkowiecznych (przeważnie trzyletnich) z rodziny lnowatych}
\Clue{20}{}{określenie wykonawcze; żałośnie}
\Clue{22}{}{południowoamerykańska roślina z traw}
\Clue{24}{}{wymarły język semicki z grupy kananejskiej, wywodzący się z języka fenickiego}
\Clue{25}{}{w tradycji żydowskiej istota utworzona z gliny na kształt człowieka, ale pozbawiona duszy rozumiejącej neszama, a zatem również zdolności mowy}
\Clue{27}{}{pokład okrętu, kryty pomost na statku}
\Clue{28}{}{osoba, która podróżuje i zwiedza różne miejsca}
\Clue{29}{}{Jan (1878-1943), pisarz, utwory poetyckie i głośny dramat o strajku szkockim 1095 r; „Młody las”}
\Clue{30}{}{przejaw, postać czegoś; ujęta abstrakcyjnie forma czegoś}
\Clue{31}{}{zabawa weselna, podczas której goście weselni krążą wokół zestawu krzeseł, aż nie umilknie muzyka - wtedy szybko zajmują wolne miejsca, a osoba, która nie zdąży zająć miejsca, odpada (na końcu ma zostać jeden zwycięzca)}
\Clue{32}{}{Pomatostomidae - rodzina ptaków z rzędu wróblowych (Passeriformes), do której należy 5 gatunków; występują w Australii i Nowej Gwinei}
\Clue{34}{}{promień wysyłany przez promieniotwórcze jądra atomowe podczas przemiany jądrowej}
\Clue{35}{}{przetrwanie, niestracenie życia w wyniku jakichś okoliczności, np. wypadku}
\Clue{36}{}{leżące nadpsute martwe drewno}
\Clue{37}{}{wyłuskane strąki (odpadki po łuskaniu ziaren) niektórych roślin strączkowych, które są używane jako pasza dla bydła}
\Clue{38}{}{zajęcia pilatesu, trening pilatesu, zwykle taki, który odbywa się regularnie, jest częścią czyjegoś regularnego rozkładu np. tygodnia}
\Clue{39}{}{pospolity minerał z gromady krzemianów, grupy plagioklazów, o białej, szarej lub żółtawej barwie, może być też bezbarwny, bywa stosowany w jubilerstwie}
\Clue{40}{}{promowanie elit w obsadzaniu wysokich stanowisk}
\Clue{41}{}{akcesorium w postaci np. biżuterii, szala, krawatu, będące ozdobą, dopełnieniem ubioru}\end{PuzzleClues}

\begin{PuzzleClues}{\textbf{Pionowe}\\}\Clue{1}{}{tytułu grzecznościowy używany w stosunku do najstarszego kapitana danego armatora, na statku pasażerskim lub handlowym}
\Clue{2}{}{wszystkie sprawy rodzinne, domowe, gospodarstwo domowe}
\Clue{3}{}{statek, zbiornikowiec służący do przewozu chemikaliów}
\Clue{4}{}{kora dębowa wykorzystywana w celu garbowania skóry}
\Clue{5}{}{ogół placówek obejmujących swym zasięgiem jakiś duży teren, także zespół organizacji, jednostek administracyjnych}
\Clue{6}{}{długowłosy chart rosyjski}
\Clue{8}{}{dyscyplina naukowa, dział morfologii badający wykorzystanie środków fonologicznych w systemie morfologicznym danego języka}
\Clue{9}{}{Aconitum carmichaelii Debeaux - uprawiany gatunek tojada pochodzący z Chin}
\Clue{10}{}{ur. w 1918 r. malarz i scenograf; obrazy metaforyczno-surrealistyczne, grafika książkowa}
\Clue{11}{}{utwór wokalny lub wokalne - instrumentalny oparty na ewangelicznym tekście o męce Chrystusa}
\Clue{12}{}{gatunek kaczki, wszystkożerna, w Polsce rzadka, zalatująca, łowna}
\Clue{13}{}{grupa osób uznanych za najwybitniejsze w danej dziedzinie}
\Clue{15}{}{miasto i port w Angoli (enklawa Kabinda)}
\Clue{17}{}{urzędnik starożytnego Rzymu, który wchodził w skład kolegium zajmującego się porządkiem publicznym, urządzaniem igrzysk oraz wykonywaniem poleceń trybunów}
\Clue{18}{}{obszar kraju o określonym profilu przemysłowym, gospodarczym}
\Clue{19}{}{lista spraw, które stanowią przedmiot obrad oraz ich kolejność}
\Clue{20}{}{rezultat uszkodzenia}
\Clue{21}{}{przedmiot o kształcie kulistym}
\Clue{23}{}{narzędzie służące do nacinania drewna w głowicach frezowych}
\Clue{26}{}{kara śmierci przez spalenie na stosie}
\Clue{27}{}{najważniejszy język khoisański, używany przez około 250 tys. Hotentotów, głównie w Namibii i Republice Południowej Afryki}
\Clue{29}{}{wchłanianie, absorpcyjność, wsysanie}
\Clue{31}{}{sztuka pięknego i skutecznego perswazyjnie przemawiania}
\Clue{33}{}{miasto i port w Hiszpanii (Walencja) nad Morzem Śródziemnym}
\Clue{34}{}{w mitologii greckiej olbrzym o wężowych splotach zamiast nóg i uskrzydlonym torsie; syn Gai i Uranosa}
\Clue{35}{}{urządzenie linowe do podnoszenia statków}\end{PuzzleClues}\newpage\section*{Krzyżówka 30}

\noindent\begin{Puzzle}{22}{18}|*	|*	|*	|*	|*	|*	|*	|*	|[1][S]\darr	|*	|[2][S]\drarr	|r	|o	|f	|e	|k	|o	|k	|s	|y	|b	|*	|*	|.
|*	|*	|[3][S]\rarr	|n	|o	|t	|a	|[][,]{ }	|p	|r	|o	|t	|e	|s	|t	|a	|c	|y	|j	|n	|a	|*	|*	|.
|*	|*	|*	|*	|[4][S]\drarr	|u	|c	|h	|o	|*	|b	|*	|*	|*	|*	|[5][S]\darr	|*	|*	|[6][S]\darr	|[7][S]\darr	|*	|[8][S]\darr	|*	|.
|*	|*	|[9][S]\darr	|[10][S]\darr	|d	|*	|*	|*	|c	|*	|i	|*	|*	|*	|*	|i	|*	|*	|p	|l	|[11][S]\darr	|o	|*	|.
|*	|*	|s	|p	|o	|[12][S]\darr	|*	|*	|h	|*	|e	|*	|[13][S]\rarr	|k	|o	|c	|k	|*	|i	|u	|t	|t	|*	|.
|*	|*	|a	|r	|b	|s	|*	|*	|ł	|*	|g	|[14][S]\rarr	|m	|a	|c	|h	|*	|*	|l	|t	|ł	|o	|*	|.
|*	|[15][S]\drarr	|g	|o	|r	|y	|c	|z	|a	|k	|*	|[16][S]\rarr	|t	|r	|e	|n	|d	|*	|e	|n	|u	|l	|*	|.
|*	|b	|a	|s	|z	|l	|*	|*	|n	|*	|*	|*	|[17][S]\rarr	|p	|ł	|e	|ć	|*	|s	|i	|s	|o	|*	|.
|*	|r	|n	|i	|y	|w	|[18][S]\rarr	|p	|i	|e	|s	|*	|[19][S]\rarr	|l	|a	|u	|d	|a	|*	|k	|z	|g	|*	|.
|*	|o	|*	|ę	|s	|e	|[20][S]\rarr	|r	|a	|j	|a	|[][,]{ }	|k	|o	|s	|m	|a	|t	|a	|*	|c	|i	|*	|.
|*	|w	|*	|*	|k	|t	|*	|*	|c	|[21][S]\rarr	|k	|a	|l	|e	|s	|o	|n	|y	|*	|*	|z	|a	|*	|.
|*	|n	|*	|*	|i	|a	|[22][S]\rarr	|c	|z	|a	|s	|[][,]{ }	|s	|ł	|o	|n	|e	|c	|z	|n	|y	|*	|*	|.
|*	|*	|*	|*	|*	|*	|*	|*	|[][,]{ }	|*	|[23][S]\rarr	|b	|i	|e	|g	|*	|*	|[24][S]\rarr	|b	|u	|k	|*	|*	|.
|[25][S]\rarr	|f	|r	|e	|i	|b	|e	|r	|g	|e	|r	|*	|*	|*	|*	|*	|*	|*	|*	|*	|*	|*	|*	|.
|*	|*	|*	|*	|*	|*	|[26][S]\rarr	|g	|a	|b	|a	|r	|d	|y	|n	|a	|*	|*	|*	|*	|*	|*	|*	|.
|[27][S]\rarr	|m	|e	|c	|h	|a	|n	|i	|z	|m	|[][,]{ }	|u	|d	|e	|r	|z	|e	|n	|i	|o	|w	|y	|*	|.
|*	|*	|*	|*	|[28][S]\rarr	|w	|s	|p	|ó	|ł	|p	|r	|a	|c	|a	|*	|*	|*	|*	|*	|*	|*	|*	|.
|*	|[29][S]\rarr	|n	|i	|e	|r	|o	|z	|w	|i	|ą	|z	|a	|l	|n	|o	|ś	|ć	|*	|*	|*	|*	|*	|.
|[30][S]\rarr	|w	|i	|n	|d	|s	|o	|r	|*	|*	|*	|*	|*	|*	|*	|*	|*	|*	|*	|*	|*	|*	|*	|.\end{Puzzle}

\newpage

\begin{PuzzleClues}{\textbf{Poziome}\\}\Clue{2}{}{przeciwbólowy i przeciwzapalny lek z grupy inhibitorów COX-2 produkowany przez koncern farmaceutyczny Merck \& Co}
\Clue{3}{}{dokument dużej wagi urzędowej, w którym przedstawiony jest protest wobec jakiegoś zjawiska, działania, które odpowiednie służby państwowe, międzynarodowe lub zarząd organizacji mogą powstrzymać}
\Clue{4}{}{uchwyt zwykle półkolisty, miękki (jak przy np. torbie) lub sztywny (jak np. przy garnku), który pomaga w chwytaniu, trzymaniu i noszeniu rzeczy, bo można przez niego przełożyć dłoń}
\Clue{13}{}{pisarz francuski (1793-1871), powieści, melodramaty, wodewile; „Pan Tapin”}
\Clue{14}{}{austriacki fizyk i filozof (1838-1916); prace z akustyki, optyki i mechaniki płynów, jeden z twórców empiriokrytycyzmu}
\Clue{15}{}{Tylopilus felleus (Bull.) P. Karst. - gatunek grzyba z rodziny borowikowatych; występuje od czerwca do końca października w lasach iglastych, lubi kwaśne gleby}
\Clue{16}{}{w ekonomii - kierunek zmian wartości indeksów}
\Clue{17}{}{narządy płciowe}
\Clue{18}{}{ssak z rodziny psowatych}
\Clue{19}{}{malarz włoski (1751-1830) nadworny malarz cesarskich dworów w Wiedniu i Petersburgu}
\Clue{20}{}{Leucoraja fullonica - gatunek ryby chrzęstnoszkieletowej z rodziny rajowatych (Rajidae); raja kosmata występuje w północnym Atlantyku od Norwegii, Islandii i Wysp Owczych do Madery, w Morzu Śródziemnym i sporadycznie w Kanale Sueskim}
\Clue{21}{}{część męskiej bielizny, spodenki z wydłużonymi nogawkami noszone pod spodniami}
\Clue{22}{}{czas wynikający bezpośrednio z pozycji Słońca na niebie}
\Clue{23}{}{przekładnia skrzyni biegów o określonym przełożeniu odpowiadającym pewnemu zakresowi prędkości jazdy}
\Clue{24}{}{drewno bukowe}
\Clue{25}{}{rasa konia lekkiego zimnokrwistego pochodząca ze Szwajcarii}
\Clue{26}{}{jednobarwna tkanina wełniana, zwykle z wełny czesankowej, tkana w skośne prążki; używana jako materiał na ubrania i płaszcze}
\Clue{27}{}{rodzaj mechanizmu odpalającego, powodującego strzał poprzez mechaniczne zbicie spłonki iglicą}
\Clue{28}{}{wspólne działanie}
\Clue{29}{}{bycie nierozwiązywalnym, niemożliwym do rozwiązania, rozwikłania, odgadnięcia (np. nierozwiązalność problemu, nierozwiązalność zadania)}
\Clue{30}{}{miasto w Anglii w regionie Londynu, z królewskim zamkiem Windsor Castle z XI w}\end{PuzzleClues}

\begin{PuzzleClues}{\textbf{Pionowe}\\}\Clue{1}{}{substancja chemiczna, której zadaniem jest utrzymywanie w danym przyrządzie próżni przez pochłanianie resztek gazów}
\Clue{2}{}{(silnika) dag kolejnych powtarzających się procesów zachodzących w cylindrze silnika podczas jego pracy}
\Clue{4}{}{kompozytor (1807-1867); symfonie, utwory kameralne, opery, pieśni}
\Clue{5}{}{mangusta - ssak z rodziny mangustowatych}
\Clue{6}{}{cykl obrazów przedstawiających mękę Chrystusa}
\Clue{7}{}{rzemieślnik budujący skrzypce i inne pokrewne instrumenty}
\Clue{8}{}{dział otorynolaryngologii zajmujący się fizjologią, rozpoznawaniem, diagnozowaniem i leczeniem chorób uszu}
\Clue{9}{}{duże naczynie kuchenne z miedzi lub żelaza}
\Clue{10}{}{świnia, człowiek, który postępuje niegodnie, jest paskudny (słowo często używane żartobliwie, stąd rzadko uważany za inwektywę)}
\Clue{11}{}{zdrobniale o tkance tłuszczowej gromadzącej się w organizmie}
\Clue{12}{}{kształt postaci lub obiektu rysujący się na kontrastowym tle}
\Clue{15}{}{chemik amerykański ur. w 1912 r., badania mechanizmu przemian cholesterolu, laureat nagrody Nobla}\end{PuzzleClues}\newpage\section*{Krzyżówka 31}

\noindent\begin{Puzzle}{17}{33}|*	|*	|*	|*	|[1][S]\darr	|*	|*	|*	|*	|*	|*	|*	|*	|*	|*	|*	|*	|*	|.
|*	|*	|*	|*	|k	|*	|*	|[2][S]\darr	|*	|*	|[3][S]\darr	|*	|*	|*	|*	|*	|*	|*	|.
|*	|*	|*	|*	|a	|*	|*	|e	|*	|[4][S]\darr	|o	|*	|*	|[5][S]\darr	|*	|*	|*	|*	|.
|*	|*	|*	|*	|r	|*	|*	|d	|*	|p	|k	|[6][S]\darr	|*	|k	|*	|*	|*	|*	|.
|*	|*	|*	|[7][S]\drarr	|t	|o	|r	|u	|*	|a	|u	|k	|*	|a	|*	|*	|*	|*	|.
|*	|*	|*	|w	|a	|*	|*	|k	|*	|n	|l	|a	|*	|s	|*	|*	|*	|*	|.
|*	|[8][S]\rarr	|m	|i	|n	|i	|w	|a	|n	|*	|i	|r	|*	|k	|*	|*	|*	|*	|.
|[9][S]\drarr	|k	|o	|r	|a	|l	|*	|c	|*	|*	|s	|m	|*	|a	|*	|*	|*	|*	|.
|l	|*	|*	|k	|*	|*	|*	|j	|*	|*	|t	|a	|*	|d	|*	|*	|*	|*	|.
|a	|*	|*	|i	|*	|[10][S]\rarr	|k	|a	|s	|t	|a	|n	|i	|e	|t	|y	|*	|*	|.
|m	|*	|*	|*	|*	|*	|*	|[][,]{ }	|*	|*	|*	|i	|*	|r	|*	|*	|*	|*	|.
|e	|*	|*	|*	|*	|*	|*	|d	|*	|[11][S]\darr	|*	|o	|*	|*	|*	|*	|*	|*	|.
|n	|*	|*	|*	|*	|*	|*	|l	|*	|w	|*	|l	|*	|*	|*	|*	|*	|*	|.
|t	|*	|[12][S]\darr	|[13][S]\darr	|*	|[14][S]\rarr	|s	|a	|n	|i	|t	|a	|r	|i	|u	|s	|z	|*	|.
|a	|*	|z	|s	|*	|*	|*	|[][,]{ }	|*	|e	|*	|*	|*	|*	|*	|*	|*	|*	|.
|c	|*	|a	|t	|*	|*	|*	|b	|[15][S]\darr	|r	|*	|*	|*	|*	|*	|*	|*	|*	|.
|j	|[16][S]\rarr	|p	|r	|z	|y	|l	|e	|p	|n	|o	|ś	|ć	|*	|*	|*	|*	|*	|.
|a	|*	|r	|e	|*	|*	|*	|z	|a	|a	|*	|*	|*	|*	|*	|*	|*	|*	|.
|*	|*	|a	|f	|*	|*	|*	|p	|r	|[][,]{ }	|*	|*	|*	|*	|*	|*	|*	|*	|.
|*	|*	|w	|a	|*	|*	|*	|i	|t	|p	|*	|*	|*	|*	|*	|*	|*	|*	|.
|*	|*	|a	|[][,]{ }	|*	|*	|*	|e	|i	|e	|*	|*	|*	|*	|*	|*	|*	|*	|.
|*	|*	|[][,]{ }	|r	|*	|*	|*	|c	|a	|n	|*	|*	|*	|*	|*	|*	|*	|*	|.
|*	|*	|k	|e	|*	|*	|*	|z	|[][,]{ }	|e	|*	|*	|*	|*	|*	|*	|*	|*	|.
|*	|*	|l	|l	|*	|*	|*	|e	|w	|l	|*	|*	|*	|*	|*	|*	|*	|*	|.
|*	|*	|e	|a	|*	|*	|*	|ń	|i	|o	|*	|*	|*	|*	|*	|*	|*	|*	|.
|*	|*	|j	|k	|*	|*	|*	|s	|e	|p	|*	|*	|*	|*	|*	|*	|*	|*	|.
|*	|*	|o	|s	|*	|*	|*	|t	|d	|a	|*	|*	|*	|*	|*	|*	|*	|*	|.
|*	|*	|w	|a	|*	|*	|*	|w	|e	|*	|*	|*	|*	|*	|*	|*	|*	|*	|.
|*	|*	|a	|c	|*	|*	|*	|a	|ń	|*	|*	|*	|*	|*	|*	|*	|*	|*	|.
|*	|*	|*	|y	|*	|*	|*	|*	|s	|*	|*	|*	|*	|*	|*	|*	|*	|*	|.
|*	|*	|*	|j	|*	|*	|*	|*	|k	|*	|*	|*	|*	|*	|*	|*	|*	|*	|.
|*	|*	|*	|n	|*	|*	|*	|*	|a	|*	|*	|*	|*	|*	|*	|*	|*	|*	|.
|*	|*	|*	|a	|*	|*	|*	|*	|*	|*	|*	|*	|*	|*	|*	|*	|*	|*	|.
|*	|*	|*	|*	|*	|*	|*	|*	|*	|*	|*	|*	|*	|*	|*	|*	|*	|*	|.\end{Puzzle}

\newpage

\begin{PuzzleClues}{\textbf{Poziome}\\}\Clue{7}{}{drewniany lub kamienny łuk nad bramą świątyni sintoistycznej}
\Clue{8}{}{minivan - samochód, który powstał jako wersja osobowa samochodu dostawczego, przypominająca go wyglądem, choć mniejsza}
\Clue{9}{}{koralik, mała kulka, paciorek}
\Clue{10}{}{perkusyjny instr. używany w tańcach hiszpańskich do podkreślenia rytmu}
\Clue{14}{}{specjalnie przeszkolony żołnierz udzielający pierwszej pomocy rannym żołnierzom}
\Clue{16}{}{cecha powierzchni przylepnej - takiej, która przylepia się do czegoś}\end{PuzzleClues}

\begin{PuzzleClues}{\textbf{Pionowe}\\}\Clue{1}{}{lekkie odprzodowe działo większego kalibru, o krótkiej lufie, używane od XVI do XVIII wieku jako działo oblężnicze lub do walki na bliską odległość}
\Clue{2}{}{przedmiot realizowany od września 2009 w gimnazjach i szkołach ponadgimnazjalnych, w których zastąpił przysposobienie obronne; w ramach tego przedmiotu naucza się o szeroko pojętej obronie cywilnej, metodach ochrony przed różnymi zagrożeniami, przygotowaniu do postępowania w wypadku katastrof, a także o pierwszej pomocy}
\Clue{3}{}{lekarz, który zajmuje się narządem wzroku człowieka, bada i leczy zmiany w oku}
\Clue{4}{}{człowiek władający czymś, panujący nad kimś, mający poddanych, władca}
\Clue{5}{}{osoba zawodowo wykonująca jako dubler w filmach lub przedstawieniach teatralnych niebezpieczne sceny takie jak: wypadki samochodowe, sceny walki, upadki z dużych wysokości, palenie, ewolucje konne itp}
\Clue{6}{}{taniec z okresu rewolucji francuskiej tańczony do piosenki o tej samej nazwie}
\Clue{7}{}{Turbellaria - grupa zwierząt należąca do płazińców; żyją w środowisku wodnym (wody słodkie lub słone), są drapieżnikami, a ich ciała okrywają nabłonki z rzęskami, w nabłonkach tych umieszczone są rabdity wyrzucane w czasie ataku}
\Clue{9}{}{śpiewy liturgiczne, których teksty to fragmenty lamentacji proroka Jeremiasza, wykonywane w kościele katolickim w czasie jutrzni w okresie Triduum Paschalnego}
\Clue{11}{}{wierna, oddana żona}
\Clue{12}{}{substancja stosowana podczas prac budowlanych, mająca właściwości podkładu i kleju}
\Clue{13}{}{w akupresurze - miejsce na ciele człowieka, w którym zgromadzone są nerwy czuciowe}
\Clue{15}{}{otwarcie szachowe, które charakteryzuje się posunięciami: 1. e4 e5, 2. Sc3}\end{PuzzleClues}\newpage\section*{Krzyżówka 32}

\noindent\begin{Puzzle}{17}{32}|*	|*	|[1][S]\darr	|*	|*	|*	|*	|*	|*	|*	|*	|*	|*	|*	|*	|*	|[2][S]\darr	|*	|.
|*	|[3][S]\drarr	|r	|a	|m	|a	|n	|*	|*	|[4][S]\drarr	|f	|u	|r	|c	|z	|a	|k	|*	|.
|[5][S]\drarr	|w	|e	|r	|s	|j	|a	|[][,]{ }	|b	|e	|t	|a	|*	|*	|*	|*	|i	|*	|.
|c	|a	|m	|*	|[6][S]\rarr	|r	|o	|z	|r	|u	|s	|z	|n	|i	|k	|*	|b	|*	|.
|h	|s	|i	|[7][S]\darr	|*	|[8][S]\rarr	|s	|k	|ó	|r	|n	|i	|k	|*	|[9][S]\darr	|*	|i	|*	|.
|o	|z	|z	|g	|[10][S]\drarr	|s	|e	|c	|c	|o	|*	|*	|[11][S]\darr	|*	|d	|[12][S]\darr	|t	|*	|.
|r	|y	|a	|e	|k	|[13][S]\rarr	|c	|o	|u	|p	|[][S]é	|*	|g	|*	|o	|t	|k	|[14][S]\darr	|.
|o	|n	|*	|n	|a	|*	|*	|*	|*	|e	|*	|*	|r	|*	|s	|u	|a	|b	|.
|b	|g	|*	|e	|u	|*	|*	|*	|[15][S]\darr	|j	|*	|*	|z	|*	|k	|j	|*	|o	|.
|a	|t	|*	|r	|l	|*	|*	|*	|r	|c	|*	|*	|e	|[16][S]\darr	|o	|o	|[17][S]\darr	|n	|.
|[][,]{ }	|o	|*	|a	|i	|*	|*	|*	|e	|z	|*	|[18][S]\darr	|j	|p	|n	|w	|n	|d	|.
|d	|ń	|*	|l	|f	|*	|*	|[19][S]\darr	|w	|y	|*	|o	|n	|s	|a	|i	|o	|i	|.
|w	|c	|*	|i	|l	|[20][S]\darr	|*	|p	|i	|k	|[21][S]\drarr	|b	|i	|t	|l	|e	|s	|*	|.
|o	|z	|[22][S]\drarr	|c	|o	|m	|b	|e	|r	|*	|b	|r	|k	|r	|e	|c	|o	|*	|.
|r	|y	|p	|j	|r	|e	|*	|j	|*	|*	|o	|o	|[][,]{ }	|o	|n	|[][,]{ }	|r	|*	|.
|s	|k	|i	|a	|i	|t	|[23][S]\darr	|s	|*	|*	|l	|ń	|e	|k	|i	|w	|o	|*	|.
|k	|*	|c	|*	|a	|o	|p	|z	|*	|[24][S]\darr	|i	|c	|l	|a	|e	|ł	|ż	|*	|.
|a	|*	|*	|*	|*	|d	|u	|a	|*	|l	|m	|a	|e	|c	|[][,]{ }	|o	|e	|*	|.
|*	|*	|[25][S]\darr	|[26][S]\drarr	|p	|a	|n	|n	|e	|a	|u	|*	|k	|z	|z	|s	|c	|*	|.
|*	|*	|s	|w	|*	|[][,]{ }	|k	|z	|*	|m	|s	|*	|t	|k	|a	|k	|[][,]{ }	|*	|.
|*	|*	|z	|r	|*	|s	|r	|a	|*	|e	|z	|*	|r	|i	|w	|o	|s	|*	|.
|*	|*	|k	|a	|*	|h	|o	|u	|*	|l	|k	|*	|y	|*	|o	|l	|z	|*	|.
|*	|*	|l	|k	|*	|e	|c	|r	|[27][S]\darr	|k	|a	|*	|c	|*	|d	|i	|e	|*	|.
|*	|*	|i	|*	|*	|p	|k	|*	|t	|a	|*	|*	|z	|*	|o	|s	|r	|*	|.
|[28][S]\rarr	|z	|w	|o	|r	|a	|*	|*	|e	|*	|*	|*	|n	|*	|w	|t	|o	|*	|.
|[29][S]\rarr	|k	|o	|n	|t	|r	|a	|k	|t	|*	|*	|*	|y	|*	|e	|n	|k	|*	|.
|*	|*	|*	|*	|*	|d	|[30][S]\darr	|[31][S]\rarr	|r	|ó	|w	|*	|*	|*	|*	|y	|o	|*	|.
|[32][S]\drarr	|g	|e	|r	|i	|a	|t	|r	|i	|a	|*	|*	|*	|*	|*	|*	|p	|*	|.
|d	|[33][S]\drarr	|j	|a	|m	|*	|o	|[34][S]\rarr	|s	|k	|o	|t	|*	|*	|*	|*	|y	|*	|.
|u	|l	|*	|*	|*	|*	|n	|*	|*	|*	|*	|[35][S]\rarr	|j	|a	|n	|u	|s	|*	|.
|g	|y	|*	|*	|*	|*	|*	|[36][S]\rarr	|p	|r	|o	|m	|i	|o	|n	|e	|k	|*	|.
|a	|d	|[37][S]\rarr	|p	|o	|t	|o	|p	|[][,]{ }	|s	|z	|w	|e	|d	|z	|k	|i	|*	|.
|*	|*	|*	|[38][S]\rarr	|s	|p	|r	|o	|s	|t	|o	|w	|a	|n	|i	|e	|*	|*	|.\end{Puzzle}

\newpage

\begin{PuzzleClues}{\textbf{Poziome}\\}\Clue{3}{}{fizyk indyjski (1888-1970); odkrywca zjawiska kombinacyjnego rozpraszania światła, laureat nagrody Nobla w 1930 r}
\Clue{4}{}{południowoamerykański ptak z rodziny kolibrów (Andy)}
\Clue{5}{}{wersja oprogramowania, w której program ma już pierwszych użytkowników, zwanych często beta-testerami, wyłapywane są błędy związane z różnymi środowiskami i warunkami pracy programu}
\Clue{6}{}{rozrusznik silnika spalinowego; urządzenie do uruchamiania silnika spalinowego}
\Clue{8}{}{Stereum Hill ex Pers. - rodzaj grzybów z rodziny skórnikowatych; grzyby saprotroficzne lub pasożytnicze, rozwijające się w drewnie i powodujące jego białą zgniliznę drewna}
\Clue{10}{}{określenie wykonawcze: sucho, oschle}
\Clue{13}{}{konny pojazd dwuosobowy}
\Clue{21}{}{żartobliwie lub z ironią o mężczyźnie, który nosi fryzurę wzorowaną na uczesaniach członków zespołu The Beatles}
\Clue{22}{}{część zwierzęcej półtuszy z części lędźwiowej grzbietu bez nerki, najczęściej ze zwierząt łownych, ale także owiec i królików}
\Clue{26}{}{motyw dekoracyjny w formie rozłożonego wachlarzowato, stylizowanego liścia palmy}
\Clue{28}{}{element łączący prowadnice łożysk osiowych pojazdów szynowych}
\Clue{29}{}{doroczny zjazd szlachty w dawnej Polsce}
\Clue{31}{}{polowa fortyfikacja ziemna, obronna lub oblężnicza, w postaci wykopu o głębokości chroniącej przed ostrzałem nieprzyjaciela na wprost, redukująca przy okazji skutki ostrzału od góry i bombardowań, usytuowana na froncie walk, służąca głównie do prowadzenia ognia z broni osobistej i zespołowej oraz obserwacji przedpola}
\Clue{32}{}{dziedzina medycyny zajmująca się schorzeniami wieku podeszłego}
\Clue{33}{}{DŻEM; rodzaj konfitur o galaretowatej konsystencji zawierający także całe, nie rozgotowane owoce}
\Clue{34}{}{kołowy transporter opancerzony}
\Clue{35}{}{jedno z najważniejszych bóstw staroitalskich czczone w starożytnym Rzymie; był bogiem wszelkich początków, a także opiekunem drzwi, bram, przejść i mostów, patronem umów i układów sojuszniczych}
\Clue{36}{}{z koncepcji filozoficznej Tomasza Zana - pierwiastek duchowy w człowieku, który emanuje z niego na otoczenie}
\Clue{37}{}{najazd Szwecji na Rzeczpospolitą w 1655 w czasie II wojny północnej (1655-1660)}
\Clue{38}{}{informacja korygująca błąd}\end{PuzzleClues}

\begin{PuzzleClues}{\textbf{Pionowe}\\}\Clue{1}{}{zajezdnia dla wozów strażackich, rodzaj budynku służący do magazynowania sprzętu przeciwpożarowego, osobistego sprzętu ochronny, węży pożarniczych, gaśnic i innych sprzętów gaśniczych}
\Clue{2}{}{rodzaj namiotu koczowniczego ludów z Azji}
\Clue{3}{}{mieszkaniec stanu Waszyngton}
\Clue{4}{}{mieszkaniec Europy, człowiek pochodzący z państwa europejskiego}
\Clue{5}{}{eufemistyczna nazwa, jaką określano kiłę}
\Clue{7}{}{wszyscy generałowie w armii danego kraju; także złożony z nich korpus}
\Clue{9}{}{proces aktualizowania i pogłębiania wiedzy oraz umiejętności związanych z wykonywanym zawodem}
\Clue{10}{}{wyrastanie kwiatów bezpośrednio na pniu i konarach}
\Clue{11}{}{urządzenie grzewcze działające na prąd elektryczny}
\Clue{12}{}{Thuidium philibertii - gatunek mchu należący do rodziny tujowcowatych}
\Clue{14}{}{Hermann, ur. w 1919r. kosmolog i matematyk - budowa wewnętrzna gwiazd, budowa i ewolucja Wszechświata}
\Clue{15}{}{barak szpitalny w hitlerowskich obozach koncentracyjnych}
\Clue{16}{}{Stictonettinae - podrodzina ptaków wyróżniona w obrębie rodziny kaczkowatych (Anatidae)}
\Clue{17}{}{nosorożec biały, nosorożec afrykański, nosorożec tęponosy, Ceratotherium simum - gatunek ssaka nieparzystokopytnego z rodziny nosorożców, największy z żyjących obecnie nosorożców, a także największe - oprócz słoni - współczesne zwierzę lądowe; występuje głównie na terenach Demokratycznej Republiki Konga, Namibii, Republiki Południowej Afryki, Botswany, Zambii i Kenii}
\Clue{18}{}{w prawie: pełnomocnik występujący w imieniu oskarżonego na podstawie udzielonego pełnomocnictwa lub decyzji sądu}
\Clue{19}{}{Peishansaurus - rodzaj roślinożernego dinozaura z rzędu dinozaurów ptasiomiednicznych; żył w okresie późnej kredy na terenach centralno-wschodniej Azji}
\Clue{20}{}{sposób aproksymacji wielowymiarowej dla rozproszonych zbiorów znanych punktów aproksymacyjnych}
\Clue{21}{}{mucha nieco mniejsza od domowej; kłuje boleśnie i żywi się krwią ssaków}
\Clue{22}{}{kłamstwo, udawanie, oszustwo}
\Clue{23}{}{gatunek w obrębie muzyki rockowej lub całokształt muzyki tworzonej przez grupy nawiązujące do ideologii ruchu punk}
\Clue{24}{}{pasek z blachy, cienka blaszana płytka}
\Clue{25}{}{warstwa, nakładana na wyroby ceramiczne przez zanurzenie, natrysk lub polanie i wypala w temperaturze kilkuset stopni Celsjusza, na skutek czego tworzy się cieńka powłoka}
\Clue{26}{}{pojazd, który wskutek uszkodzeń lub starości przestał nadawać się do użytku}
\Clue{27}{}{komputerowa gra logiczna stworzona przez Aleksieja Pażytnowa i jego współpracowników, Dimitrija Pawłowskiego i Wadima Geriasimowa}
\Clue{30}{}{zabarwienie emocjonalne, stylistyczne}
\Clue{32}{}{DUHA; kabłąk w zaprzęgu jednokonnym służący do przymocowania chomąta do hołobu}
\Clue{33}{}{kod ISO 4217 dinara libijskiego}\end{PuzzleClues}\newpage\section*{Krzyżówka 33}

\noindent\begin{Puzzle}{24}{17}|*	|*	|*	|*	|*	|*	|*	|*	|*	|*	|*	|*	|*	|*	|*	|*	|*	|*	|[1][S]\drarr	|t	|o	|w	|o	|t	|*	|.
|*	|*	|*	|*	|*	|[2][S]\drarr	|l	|o	|t	|o	|k	|o	|t	|[][,]{ }	|f	|i	|l	|i	|p	|i	|ń	|s	|k	|i	|*	|.
|*	|*	|*	|*	|*	|f	|*	|*	|[3][S]\rarr	|g	|u	|z	|[][,]{ }	|z	|ł	|o	|ś	|l	|i	|w	|y	|*	|*	|*	|*	|.
|*	|*	|*	|[4][S]\drarr	|d	|a	|v	|i	|d	|*	|*	|*	|*	|*	|[5][S]\darr	|*	|[6][S]\drarr	|s	|t	|r	|u	|ś	|*	|*	|*	|.
|*	|*	|[7][S]\drarr	|d	|e	|l	|f	|i	|n	|a	|t	|*	|*	|*	|l	|*	|ó	|*	|o	|*	|*	|[8][S]\darr	|*	|*	|*	|.
|*	|[9][S]\rarr	|i	|z	|b	|a	|*	|*	|*	|*	|*	|*	|*	|[10][S]\darr	|a	|*	|s	|*	|s	|*	|*	|z	|*	|*	|*	|.
|*	|*	|s	|i	|*	|i	|*	|*	|[11][S]\rarr	|p	|r	|o	|p	|o	|r	|z	|e	|c	|*	|*	|*	|e	|*	|[12][S]\darr	|*	|.
|*	|*	|a	|a	|[13][S]\rarr	|s	|m	|o	|l	|e	|r	|*	|*	|g	|y	|[14][S]\darr	|m	|*	|*	|*	|[15][S]\darr	|z	|[16][S]\darr	|k	|*	|.
|*	|*	|j	|d	|*	|e	|*	|*	|[17][S]\rarr	|b	|o	|g	|o	|r	|*	|f	|k	|*	|*	|*	|k	|n	|t	|u	|*	|.
|*	|*	|e	|z	|*	|*	|*	|*	|*	|[18][S]\rarr	|t	|u	|t	|o	|r	|i	|a	|l	|*	|*	|ą	|a	|r	|t	|*	|.
|*	|*	|w	|i	|*	|*	|*	|*	|*	|*	|[19][S]\drarr	|r	|e	|m	|i	|z	|*	|*	|*	|*	|p	|w	|a	|a	|*	|.
|[20][S]\rarr	|k	|*	|e	|[21][S]\rarr	|p	|o	|k	|ł	|o	|s	|i	|e	|*	|*	|y	|*	|*	|*	|*	|i	|a	|n	|s	|*	|.
|[22][S]\rarr	|k	|a	|n	|a	|n	|g	|a	|*	|*	|a	|[23][S]\rarr	|c	|m	|o	|k	|n	|o	|n	|s	|e	|n	|s	|*	|*	|.
|*	|*	|*	|i	|*	|*	|*	|[24][S]\rarr	|c	|z	|ł	|a	|p	|a	|k	|*	|*	|*	|*	|*	|l	|i	|e	|*	|*	|.
|*	|*	|[25][S]\rarr	|e	|u	|r	|y	|d	|y	|k	|a	|*	|[26][S]\rarr	|u	|b	|o	|g	|o	|ś	|ć	|*	|e	|p	|*	|*	|.
|*	|*	|*	|*	|*	|*	|*	|[27][S]\rarr	|o	|s	|t	|a	|t	|n	|i	|a	|[][,]{ }	|w	|o	|l	|a	|*	|t	|*	|*	|.
|*	|[28][S]\rarr	|k	|o	|n	|d	|o	|r	|[][,]{ }	|k	|a	|l	|i	|f	|o	|r	|n	|i	|j	|s	|k	|i	|*	|*	|*	|.
|*	|*	|*	|*	|[29][S]\rarr	|g	|a	|m	|m	|a	|*	|*	|*	|*	|*	|*	|*	|*	|*	|*	|*	|*	|*	|*	|*	|.\end{Puzzle}

\newpage

\begin{PuzzleClues}{\textbf{Poziome}\\}\Clue{1}{}{smar maszynowy do łożysk i powierzchni ślizgowych}
\Clue{2}{}{kaguan, lotokot, kolugo, Cynocephalus volans - ssak łożyskowy z rzędu latawców; zamieszkuje lasy deszczowe na Filipinach, spotykany jest również na plantacjach bananowych, kokosowych i kauczukowych}
\Clue{3}{}{nowotwór utworzony z komórek o niskim zróżnicowaniu (niedojrzałych), o budowie znacznie odbiegającej od obrazu prawidłowych tkanek, charakteryzujący się atypią i szybkim wzrostem}
\Clue{4}{}{miasto w Panamie w pobliżu Oceanu Spokojnego, ośrodek administracyjny prowincji Chiriqui}
\Clue{6}{}{wybitny lekarz epoki odrodzenia (1510-68); lekarz Zygmunta Augusta}
\Clue{7}{}{kraina historyczna w płn-wsch. Francji, w dorzeczu Isere, w środkowej części Alp, główne miasto Grenoble}
\Clue{9}{}{organ władzy w dwuizbowym parlamencie, sądzie (np. Sądzie Najwyższym) lub innej instytucji wykonującej władzę}
\Clue{11}{}{w marynarce}
\Clue{13}{}{etnograf, publicysta, pisarz górnołużycki (1816-84), autor słownika niemiecko-łużyckiego, zbiór ludowych pieśni}
\Clue{17}{}{miasto w Indonezji, na Jawie, w pobliżu Dżakarty, ośrodek turystyczny i wypoczynkowy}
\Clue{18}{}{program pozwalający łatwo nauczyć się obsługi aplikacji, programowania czy tworzenia grafiki albo fragment będący częścią większego programu (np. gry komputerowej), który jest formą wprowadzenia, samouczka pozwalającego na oswojenie się z programem, poznanie zasad i naukę podstaw}
\Clue{19}{}{ptak zaroślowy podmokłych terenów, z rzędu wróblowatych, gniazdo workowate, wiszące; Europa, zach. Azja - chroniony}
\Clue{20}{}{w chemii: symbol potasu}
\Clue{21}{}{wynik czegoś, rezultat otrzymany w konsekwencji jakichś innych wydarzeń}
\Clue{22}{}{miasto w Zairze, nad rzeką Lulua ośrodek handlowy}
\Clue{23}{}{pocałunek w rękę - pogardliwie}
\Clue{24}{}{człowiek, który wolno się porusza}
\Clue{25}{}{Eurydice pulchra - gatunek skorupiaka z rzędu równonogów}
\Clue{26}{}{to, że gdzieś jest zła sytuacja materialna}
\Clue{27}{}{ostatnie życzenia, rozporządzenia umierającego, również wyrażone w testamencie}
\Clue{28}{}{Gymnogyps californianus - gatunek dużego ptaka padlinożernego z rodziny kondorowatych (Cathartidae), jedyny przedstawiciel rodzaju Gymnogyps; zamieszkuje góry północnej Kalifornii, a niegdyś całe Góry Skaliste}
\Clue{29}{}{wysokoenergetyczna forma promieniowania elektromagnetycznego}\end{PuzzleClues}

\begin{PuzzleClues}{\textbf{Pionowe}\\}\Clue{1}{}{starogreckie gliniane naczynie używane do przechowywania żywności a zwłaszcza zboża}
\Clue{2}{}{miasto w płn. Francji (Normandia), zwycięska bitwa wojsk alianckich z udziałem dywizji gen. Maczka nad Niemcami w 1944 r}
\Clue{4}{}{to, że ktoś staje się słabym fizycznie i psychicznie pod wpływem wieku; to, że ktoś starzeje się w bardzo niekorzystny sposób pod względem sprawności fizycznej i umysłowej}
\Clue{5}{}{dusze zmarłych czczone w Rzymie jako bóstwa opiekuńcze domu i szczęścia domowego, chroniące od nieszczęść}
\Clue{6}{}{rodzaj wiosłowej łodzi regatowej na 8 osób}
\Clue{7}{}{Jegor, poeta rosyjski, ur,1926r; „Sąd pamięci”}
\Clue{8}{}{oficjalne składanie zeznań}
\Clue{10}{}{o odniesieniu do stanów psychicznych, uczuć, emocji: bardzo duża intensywność, wielkie natężenie, np. ogrom winy, ogrom miłości, ogrom rozpaczy}
\Clue{12}{}{penis, członek męski}
\Clue{14}{}{nauczyciel fizyki}
\Clue{15}{}{środek używany do przeprowadzenia kąpieli - poddawanie wyrobów obróbce poprzez wystawienie ich na równomierne działanie jakiegoś czynnika czy składnika}
\Clue{16}{}{nawa poprzeczna między korpusem nawowym a prezbiterium}
\Clue{19}{}{Lactuca - rodzaj roślin jednorocznych z rodziny astrowatych, z których wiele gatunków to rośliny uprawne}\end{PuzzleClues}\newpage\section*{Krzyżówka 34}

\noindent\begin{Puzzle}{23}{26}|*	|*	|[1][S]\darr	|[2][S]\drarr	|k	|r	|z	|y	|ż	|o	|w	|y	|[][,]{ }	|o	|g	|i	|e	|ń	|*	|*	|*	|*	|*	|*	|.
|[3][S]\drarr	|w	|o	|d	|o	|r	|o	|s	|t	|*	|*	|*	|*	|*	|[4][S]\drarr	|a	|b	|d	|e	|r	|y	|t	|a	|*	|.
|w	|[5][S]\drarr	|b	|r	|a	|n	|i	|e	|[][,]{ }	|r	|o	|z	|w	|o	|d	|u	|*	|[6][S]\drarr	|o	|s	|a	|d	|*	|[7][S]\darr	|.
|o	|p	|ł	|o	|*	|[8][S]\darr	|*	|[9][S]\rarr	|k	|o	|k	|s	|i	|a	|r	|z	|*	|h	|[10][S]\darr	|*	|*	|*	|*	|j	|.
|z	|o	|ę	|b	|*	|p	|*	|[11][S]\drarr	|j	|e	|d	|l	|i	|n	|a	|*	|*	|a	|p	|[12][S]\darr	|*	|*	|*	|a	|.
|ó	|j	|k	|i	|*	|r	|[13][S]\drarr	|t	|r	|e	|k	|*	|*	|*	|p	|*	|*	|n	|u	|g	|*	|[14][S]\darr	|*	|g	|.
|w	|a	|*	|n	|*	|z	|k	|r	|*	|*	|*	|[15][S]\darr	|[16][S]\drarr	|n	|i	|ć	|*	|i	|b	|w	|[17][S]\darr	|c	|[18][S]\darr	|ł	|.
|k	|z	|*	|k	|*	|y	|s	|a	|*	|*	|[19][S]\drarr	|o	|s	|i	|e	|d	|l	|e	|*	|a	|w	|h	|b	|a	|.
|a	|d	|*	|a	|*	|g	|a	|c	|*	|*	|w	|t	|t	|[20][S]\darr	|ż	|*	|[21][S]\drarr	|b	|a	|r	|y	|ł	|a	|*	|.
|*	|[][,]{ }	|[22][S]\darr	|*	|*	|r	|n	|j	|*	|*	|ę	|o	|e	|p	|n	|*	|k	|n	|*	|a	|ż	|o	|k	|*	|.
|*	|n	|b	|*	|*	|ó	|t	|a	|*	|[23][S]\darr	|ż	|k	|r	|a	|o	|*	|o	|o	|*	|*	|s	|d	|c	|[24][S]\darr	|.
|*	|i	|ł	|[25][S]\darr	|*	|d	|o	|*	|*	|d	|o	|*	|o	|s	|ś	|*	|n	|ś	|*	|*	|z	|n	|y	|ż	|.
|*	|e	|ą	|s	|*	|e	|s	|*	|*	|r	|w	|[26][S]\darr	|w	|z	|ć	|*	|d	|ć	|*	|*	|e	|i	|l	|o	|.
|*	|k	|d	|t	|[27][S]\drarr	|k	|o	|m	|p	|o	|n	|e	|n	|t	|*	|*	|y	|*	|*	|*	|[][,]{ }	|c	|o	|ł	|.
|*	|o	|[][,]{ }	|a	|a	|*	|m	|*	|[28][S]\darr	|b	|i	|t	|i	|e	|*	|[29][S]\darr	|c	|*	|*	|*	|n	|a	|f	|ą	|.
|*	|ł	|e	|r	|g	|[30][S]\darr	|a	|*	|o	|i	|k	|e	|k	|t	|*	|s	|j	|*	|*	|*	|a	|*	|o	|d	|.
|*	|o	|k	|o	|e	|b	|*	|*	|d	|a	|*	|r	|*	|n	|*	|i	|o	|*	|*	|*	|c	|*	|b	|k	|.
|[31][S]\drarr	|w	|o	|d	|n	|a	|[][,]{ }	|w	|s	|z	|a	|*	|*	|i	|*	|c	|n	|*	|*	|*	|z	|*	|i	|ó	|.
|p	|y	|l	|a	|c	|n	|*	|*	|ł	|g	|*	|*	|[32][S]\rarr	|k	|o	|z	|a	|*	|*	|[33][S]\darr	|e	|*	|a	|w	|.
|a	|*	|o	|w	|j	|n	|*	|*	|o	|*	|*	|*	|*	|*	|*	|*	|l	|*	|*	|n	|l	|*	|*	|k	|.
|n	|*	|g	|n	|a	|v	|*	|*	|n	|*	|[34][S]\rarr	|m	|e	|g	|a	|h	|i	|t	|*	|e	|n	|*	|*	|a	|.
|t	|*	|i	|o	|*	|*	|[35][S]\rarr	|m	|i	|r	|a	|ż	|*	|[36][S]\rarr	|b	|e	|z	|b	|r	|z	|e	|ż	|e	|*	|.
|e	|*	|c	|ś	|*	|*	|*	|*	|e	|[37][S]\rarr	|s	|o	|n	|g	|n	|i	|m	|*	|*	|v	|*	|*	|*	|*	|.
|r	|*	|z	|ć	|[38][S]\rarr	|p	|i	|a	|n	|i	|s	|t	|y	|k	|a	|*	|*	|[39][S]\rarr	|z	|a	|b	|ó	|r	|*	|.
|k	|*	|n	|*	|[40][S]\rarr	|ł	|a	|z	|i	|k	|*	|*	|[41][S]\rarr	|t	|r	|i	|a	|z	|o	|l	|a	|m	|*	|*	|.
|a	|*	|y	|*	|*	|[42][S]\rarr	|a	|m	|e	|r	|y	|k	|a	|ń	|s	|k	|o	|ś	|ć	|*	|*	|*	|*	|*	|.
|*	|*	|*	|[43][S]\rarr	|b	|u	|r	|t	|*	|*	|*	|*	|*	|*	|*	|*	|*	|*	|*	|*	|*	|*	|*	|*	|.\end{Puzzle}

\newpage

\begin{PuzzleClues}{\textbf{Poziome}\\}\Clue{2}{}{szereg szybko i gwałtownie następujących po sobie działań}
\Clue{3}{}{popularna nazwa rośliny wodnej (mówi się tak także o niektórych roślinnych protistach wielokomórkowych) o miękkim ciele}
\Clue{4}{}{człowiek ograniczony umysłowo, tępak, kiep}
\Clue{5}{}{rozwodzenie się, rozchodzenie}
\Clue{6}{}{ogólne określenie utworów luźnych zgromadzonych na powierzchni Ziemi, powstałych wskutek gromadzenia się materiału w procesie sedymentacji}
\Clue{9}{}{koks, mięśniak; kulturysta, człowiek posiadający rozbudowaną tkankę mięśniową (często wskutek stosowania koksu - anabolików)}
\Clue{11}{}{gałęzie drzewa iglastego}
\Clue{13}{}{turystyczna wędrówka, zwykle długodystansowa, w trudnym, często egzotycznym terenie; trekking}
\Clue{16}{}{włókno powstałe z krzepnącej wydzieliny gruczołów niektórych owadów (czasem też mięczaków)}
\Clue{19}{}{mieszkańcy osiedla}
\Clue{21}{}{miara pojemności płynów równa 4 kwarty}
\Clue{27}{}{integralny element jakiejś większej całości}
\Clue{31}{}{Calla palustris - gatunek rośliny z rodziny obrazkowatych}
\Clue{32}{}{SIUTA bezroga samica sarny}
\Clue{34}{}{wielki przebój, bardzo znana i rozpowszechniona piosenka}
\Clue{35}{}{przenośnie o marzeniu, które nie może się ziścić, jest tylko złudną nadzieją na coś}
\Clue{36}{}{obszar, który wydaje się nie mieć granic; niezmierzona przestrzeń czegoś, ale też nieskończony ogrom czasu (granice czasu, których nie można pojąć ani ustalić)}
\Clue{37}{}{miasto w KRL-D nad rzeką Tedong-gang; hutnictwo żelaza, przemysł chemiczny, wydobycie rud żelaza}
\Clue{38}{}{dział muzyki obejmujący utwory na fortepian (niekiedy też na klawesyn), ich komponowanie i wykonywanie}
\Clue{39}{}{terytorium państwa, które jest okupowane przez inne państwo}
\Clue{40}{}{nowozelandzki ptak z rodziny barglików}
\Clue{41}{}{benzodiazepina stosowana w leczeniu bezsenności}
\Clue{42}{}{cecha czegoś typowego dla Amerykanina (obywatela Stanów Zjednoczonych) lub czegoś amerykańskiego}
\Clue{43}{}{lamówka - taśma pasmanteryjna lub pasek materiału do naszywania na brzegach odzieży}\end{PuzzleClues}

\begin{PuzzleClues}{\textbf{Pionowe}\\}\Clue{1}{}{przedmiot zakrzywiony w półkole; przedmiot w kształcie łuku}
\Clue{2}{}{zdrobniale, pieszczotliwie o dziecku, zwykle małym}
\Clue{3}{}{wozownia, powozownia - budynek gospodarczy, przeznaczony na wozy}
\Clue{4}{}{wyraz tego, że ktoś jest agresywny, drapieżny}
\Clue{5}{}{pojazd poruszający się w sposób niewymagający posiadania kół, np. czołg, wodolot, sanie}
\Clue{6}{}{to, że coś jest haniebne - przynosi hańbę, okrywa kogoś hańbą}
\Clue{7}{}{objaw jaglicy; grudka jagliczna zbudowana z limfocytów i plazmocytów}
\Clue{8}{}{mały gród, zależny od większego}
\Clue{10}{}{w polskich warunkach: miejsce, gdzie można się czegoś napić; zazwyczaj piwiarnia, lokal otwierany zwykle w godzinach popołudniowych}
\Clue{11}{}{historyczna kraina na Płw. Bałkańskim, do 1923rw granicach Grecji, Bułgarii i Turcji}
\Clue{12}{}{środowiskowa lub zawodowa odmiana języka}
\Clue{13}{}{Xanthosoma - rodzaj rośliny z rodziny obrazkowatych}
\Clue{14}{}{urządzenie służące do obniżania temperatury w jakimś mechanizmie}
\Clue{15}{}{kształt, linia; rzecz naokoło czegoś}
\Clue{16}{}{kosmopolityczny ptak z rodziny zimorodków}
\Clue{17}{}{Haplorrhini - podrząd ssaków naczelnych obejmujący wyraki oraz małpy właściwe, czyli szerokonose małpy Nowego Świata, wąskonose małpy Starego Świata i małpy człekokształtne; charakterysytyczny dla wyższych naczelnych jest występujący w oku, nieobecny u niższych naczelnych, dołek środkowy siatkówki oka (fovea centralis)}
\Clue{18}{}{lęk przed zarazkami}
\Clue{19}{}{gatunek rośliny z rodziny rdestowatych}
\Clue{20}{}{człowiek (najczęściej kucharz), który zajmuje się wytwarzaniem pasztetu i musów z mięsa oraz dań, które je zawierają, np. pasztecików}
\Clue{21}{}{doktryna chrześcijańska zakładająca, że dusza ludzka jest ze swej natury śmiertelna, a nieśmiertelność będzie dana od Boga dopiero podczas zmartwychwstania}
\Clue{22}{}{nieuzasadnione przenoszenie wniosków z korelacji grupowych (dokonywanych na danych opisujących zagregowane zbiorowości) na zależności na poziomie jednostkowym}
\Clue{23}{}{niewielki utwór literacki}
\Clue{24}{}{porcja żołądkówki; określona ilość tego produktu, zazwyczaj butelka, karafka lub kieliszek}
\Clue{25}{}{to, że coś jest starodawne, pochodzi z dawnych czasów}
\Clue{26}{}{domyślne określenie eteru dietylowego}
\Clue{27}{}{przedstawicielstwo firmy działające na zasadzie dzierżawy}
\Clue{28}{}{czynienie widocznym; odkrywanie; pokazywanie}
\Clue{29}{}{warowny obóz Kozaków zaporoskich}
\Clue{30}{}{miasto w Pakistanie w płn.zach. prowincji przygranicznej}
\Clue{31}{}{Chromileptes altivelis - gatunek ryby okoniokształtnej z rodziny strzępielowatych}
\Clue{33}{}{poeta czeski (1900-58), przedstawiciel poetyzmu i surrealizmu; „Pieśń pokoju”}\end{PuzzleClues}\newpage\section*{Krzyżówka 35}

\noindent\begin{Puzzle}{17}{24}|*	|*	|*	|*	|*	|*	|*	|[1][S]\drarr	|j	|a	|p	|o	|n	|k	|a	|*	|*	|[2][S]\darr	|.
|*	|*	|*	|*	|*	|[3][S]\rarr	|c	|h	|y	|t	|r	|u	|s	|k	|a	|*	|*	|c	|.
|*	|*	|[4][S]\rarr	|v	|e	|j	|l	|e	|*	|[5][S]\darr	|*	|*	|*	|*	|*	|*	|[6][S]\darr	|s	|.
|*	|*	|*	|*	|*	|[7][S]\rarr	|i	|m	|i	|e	|n	|n	|i	|c	|t	|w	|o	|*	|.
|*	|[8][S]\rarr	|g	|r	|i	|n	|g	|o	|*	|f	|*	|*	|[9][S]\darr	|*	|*	|[10][S]\darr	|g	|*	|.
|*	|*	|[11][S]\rarr	|i	|n	|t	|e	|r	|m	|e	|z	|z	|o	|*	|[12][S]\darr	|ś	|o	|*	|.
|*	|*	|*	|[13][S]\rarr	|o	|p	|ł	|o	|t	|k	|i	|*	|d	|*	|o	|w	|ń	|*	|.
|*	|*	|*	|[14][S]\darr	|[15][S]\darr	|*	|*	|i	|*	|t	|*	|[16][S]\darr	|ł	|*	|s	|i	|c	|*	|.
|*	|*	|*	|m	|m	|*	|*	|d	|*	|*	|*	|t	|u	|[17][S]\darr	|t	|e	|z	|[18][S]\darr	|.
|*	|[19][S]\darr	|*	|a	|i	|[20][S]\rarr	|ż	|y	|d	|*	|*	|u	|p	|c	|e	|r	|y	|k	|.
|*	|m	|*	|s	|k	|[21][S]\darr	|*	|*	|[22][S]\darr	|*	|*	|b	|e	|e	|o	|z	|k	|i	|.
|*	|a	|[23][S]\darr	|a	|r	|h	|*	|*	|k	|*	|*	|u	|k	|n	|t	|b	|[][,]{ }	|r	|.
|*	|t	|w	|[][,]{ }	|o	|e	|[24][S]\rarr	|p	|o	|l	|*	|l	|*	|t	|o	|[][,]{ }	|t	|k	|.
|*	|e	|y	|k	|k	|k	|*	|[25][S]\darr	|s	|[26][S]\drarr	|k	|o	|l	|u	|m	|n	|a	|*	|.
|*	|m	|d	|a	|o	|s	|*	|l	|o	|c	|[27][S]\darr	|p	|*	|r	|i	|o	|r	|*	|.
|*	|a	|z	|ł	|s	|o	|*	|i	|g	|y	|t	|a	|*	|i	|a	|r	|n	|*	|.
|*	|t	|i	|o	|m	|l	|[28][S]\rarr	|b	|o	|g	|a	|t	|k	|a	|*	|w	|i	|*	|.
|*	|y	|e	|w	|o	|i	|*	|u	|n	|a	|r	|i	|[29][S]\darr	|*	|*	|e	|n	|*	|.
|[30][S]\rarr	|k	|l	|a	|s	|t	|e	|r	|*	|n	|a	|a	|s	|*	|*	|s	|o	|*	|.
|*	|*	|i	|*	|*	|*	|*	|n	|*	|*	|*	|*	|t	|*	|*	|k	|w	|*	|.
|*	|*	|n	|*	|[31][S]\rarr	|s	|z	|a	|l	|o	|t	|k	|a	|*	|*	|i	|i	|*	|.
|*	|*	|a	|[32][S]\rarr	|b	|ą	|k	|*	|[33][S]\rarr	|c	|a	|c	|c	|i	|a	|*	|e	|*	|.
|*	|*	|*	|*	|*	|*	|*	|*	|*	|[34][S]\rarr	|r	|a	|j	|a	|*	|*	|c	|*	|.
|*	|*	|[35][S]\rarr	|k	|a	|n	|i	|b	|a	|l	|i	|z	|a	|c	|j	|a	|*	|*	|.
|[36][S]\rarr	|s	|t	|a	|t	|y	|c	|z	|n	|o	|ś	|ć	|*	|*	|*	|*	|*	|*	|.\end{Puzzle}

\newpage

\begin{PuzzleClues}{\textbf{Poziome}\\}\Clue{1}{}{mieszkanka Japonii, kobieta pochodzenia japońskiego}
\Clue{3}{}{cwaniarka, spryciula}
\Clue{4}{}{miasto i port w Danii na Półwyspie Jutlandzkim nad zatoką Vejle Fjord}
\Clue{7}{}{zasób imion osobowych}
\Clue{8}{}{cudzoziemiec, turysta, osoba niemówiąca po hiszpańsku, jankes, blondas}
\Clue{11}{}{wstawka muzyczno-sceniczna o charakterze komicznym wykonywana od XVI w. między aktami sztuki teatralnej, z której rozwinęły się opera, opera buffa i balet jako samodzielne formy teatralne}
\Clue{13}{}{przejście, najczęściej niewielkiej szerokości, mieszczące sie między ogrodzeniami położonych obok siebie działek}
\Clue{20}{}{przedstawiciel ludu semickiego zamieszkującego w starożytności Palestynę}
\Clue{24}{}{poeta, geograf i etnograf (1807-72), „Pieśń Janusza”, „Pieśń o ziemi naszej”, „Mohort”}
\Clue{26}{}{pojedynczy pionowy pas tekstu w obrębie zapisanej strony}
\Clue{28}{}{DZIUPLAK; gatunek sikory owadożerna, chroniona, chętnie przebywa w sąsiedztwie człowieka; Eurazja, płn. zach. Afryka}
\Clue{30}{}{struktura nadcząsteczkowa, w której istnieją dwie grupy cząsteczek tworzące wspólną sieć powiązań}
\Clue{31}{}{Allium ascalonicum - gatunek rośliny uprawnej należący do rodziny czosnkowatych}
\Clue{32}{}{cyga}
\Clue{33}{}{wokalny kanon dwugłosowy, będący popularną włoską formą muzyczną do tekstów opisujących zwłaszcza polowanie}
\Clue{34}{}{PŁASZCZKA; morska ryba o długości do 2,5 m}
\Clue{35}{}{proces polegający na pozyskiwaniu części zamiennych z uszkodzonych używanych maszyn, często stosowany w celu naprawy starych nie produkowanych już urządzeń lub utylizacji rozbitych samochodów lub samolotów}
\Clue{36}{}{niezmienność, bycie statycznym, np. statyczność przepisów}\end{PuzzleClues}

\begin{PuzzleClues}{\textbf{Pionowe}\\}\Clue{1}{}{stan chorobowy związany ze strukturami naczyniowymi odbytu o tej samej nazwie}
\Clue{2}{}{w chemii: symbol cezu}
\Clue{5}{}{wrażenie wywarte na kimś}
\Clue{6}{}{Satyrium spini - motyl dzienny z rodziny modraszkowatych, należący do podrodziny ogończyków; na terenie Polski występuje lokalnie w środkowej, południowej i wschodniej części kraju}
\Clue{9}{}{ułamany fragment czegoś, coś, co zostało odłupane}
\Clue{10}{}{ciężka postać świerzbu}
\Clue{12}{}{w ortopedii i chirurgii urazowej działanie polegające na przecięciu kości. Może ono mieć na celu poprawę kształtu kości, jej skrócenie lub wydłużenie (np. w aparacie Ilizarowa lub orthofix), poprawę mechaniki znajdującego się w pobliżu stawu lub korekcję osi kończyny}
\Clue{14}{}{odchody zbierające się w jelitach}
\Clue{15}{}{od czasów Demokryta świat człowieka przeciwstawny makrokosmosowi}
\Clue{16}{}{schorzenie charakteryzujące się występowaniem upośledzenia czynności resorpcyjnej lub wydzielniczej cewek nerkowych, przy prawidłowym lub tylko nieznacznie zmniejszonym przesączaniu kłębkowym}
\Clue{17}{}{jednostka powierzchni gruntu w starożytnym Rzymie}
\Clue{18}{}{pisarz duński (1898-1962), powieści współczesne o tematyce społecznej oraz historyczne; „Gra cieni”}
\Clue{19}{}{nauczyciel matematyki}
\Clue{21}{}{materia wybuchowy o podobnych właściwościach i zastosowaniu co heksogen}
\Clue{22}{}{lis morski, rekin o długości do 5 m}
\Clue{23}{}{wydzielina organizmu - substancja wydzielana przez komórki lub gruczoły w organizmie}
\Clue{25}{}{szybki, zwrotny, jednorzędowy okręt wojenny z taranem i jednym masztem}
\Clue{26}{}{ktoś, kto ma śniadą cerę i ciemne włosy, ale etnicznie nie jest Cyganem}
\Clue{27}{}{płaski przedmiot o pofałdowanej powierzchni, dającej większe tarcie, używany do prania}
\Clue{29}{}{zespół urządzeń służący do przetwarzania lub rozdzielania zasobów}\end{PuzzleClues}\newpage\section*{Krzyżówka 36}

\noindent\begin{Puzzle}{23}{28}|*	|*	|*	|*	|*	|*	|*	|*	|*	|*	|*	|*	|*	|*	|*	|*	|*	|*	|*	|[1][S]\darr	|*	|[2][S]\darr	|*	|*	|.
|*	|*	|[3][S]\drarr	|p	|o	|s	|t	|r	|o	|m	|a	|n	|t	|y	|z	|m	|*	|*	|[4][S]\darr	|f	|*	|z	|*	|*	|.
|*	|[5][S]\drarr	|k	|u	|c	|[][,]{ }	|ż	|e	|m	|a	|j	|t	|u	|k	|a	|*	|*	|*	|e	|a	|*	|i	|*	|[6][S]\darr	|.
|[7][S]\drarr	|p	|o	|d	|a	|t	|e	|k	|[][,]{ }	|o	|b	|r	|o	|t	|o	|w	|y	|*	|l	|r	|[8][S]\darr	|e	|*	|p	|.
|j	|a	|d	|*	|*	|*	|*	|*	|*	|*	|*	|*	|*	|*	|*	|*	|*	|*	|i	|g	|g	|m	|*	|i	|.
|o	|l	|[][,]{ }	|*	|*	|*	|*	|*	|*	|*	|*	|*	|*	|*	|*	|*	|*	|*	|n	|o	|ł	|n	|*	|ę	|.
|d	|m	|p	|*	|*	|*	|*	|*	|*	|*	|*	|[9][S]\drarr	|m	|a	|d	|ż	|o	|n	|g	|*	|a	|i	|[10][S]\darr	|t	|.
|ł	|a	|o	|[11][S]\drarr	|w	|i	|e	|l	|k	|i	|[][,]{ }	|p	|o	|s	|t	|*	|[12][S]\darr	|[13][S]\darr	|*	|*	|d	|a	|p	|a	|.
|a	|*	|c	|f	|*	|*	|*	|[14][S]\rarr	|c	|z	|a	|r	|n	|a	|[][,]{ }	|s	|k	|r	|z	|y	|n	|k	|a	|*	|.
|[][,]{ }	|*	|z	|r	|[15][S]\drarr	|c	|h	|o	|w	|a	|c	|z	|*	|*	|*	|*	|o	|a	|*	|*	|i	|i	|z	|*	|.
|n	|[16][S]\drarr	|t	|o	|p	|ó	|r	|[][,]{ }	|w	|o	|j	|e	|n	|n	|y	|*	|m	|m	|*	|*	|k	|[][,]{ }	|u	|*	|.
|i	|c	|o	|m	|a	|*	|[17][S]\darr	|[18][S]\rarr	|w	|ą	|s	|k	|o	|ś	|ć	|*	|p	|l	|*	|*	|[][,]{ }	|p	|r	|*	|.
|k	|z	|w	|e	|n	|[19][S]\drarr	|s	|e	|r	|p	|u	|l	|a	|*	|*	|*	|e	|a	|*	|*	|m	|o	|o	|*	|.
|k	|a	|y	|*	|i	|h	|a	|*	|*	|[20][S]\rarr	|c	|e	|w	|k	|a	|*	|t	|*	|*	|[21][S]\darr	|r	|[][,]{ }	|g	|*	|.
|o	|j	|*	|*	|e	|i	|r	|[22][S]\darr	|[23][S]\darr	|*	|*	|ń	|*	|*	|*	|*	|e	|*	|*	|g	|ó	|c	|o	|*	|.
|*	|k	|*	|[24][S]\darr	|ń	|d	|a	|e	|n	|*	|*	|s	|*	|*	|[25][S]\darr	|*	|n	|*	|*	|r	|w	|a	|n	|*	|.
|*	|a	|[26][S]\darr	|f	|s	|ż	|n	|u	|a	|[27][S]\darr	|*	|t	|*	|*	|p	|*	|c	|*	|*	|a	|k	|b	|[][,]{ }	|*	|.
|*	|[][,]{ }	|m	|i	|k	|a	|*	|f	|n	|g	|*	|w	|*	|*	|e	|[28][S]\darr	|j	|*	|[29][S]\darr	|c	|o	|a	|u	|*	|.
|[30][S]\drarr	|p	|a	|z	|i	|b	|r	|o	|d	|a	|*	|o	|*	|*	|d	|s	|e	|*	|p	|j	|l	|ń	|z	|*	|.
|o	|ł	|r	|j	|[][,]{ }	|*	|[31][S]\drarr	|n	|i	|m	|b	|*	|*	|*	|e	|u	|*	|*	|o	|a	|u	|s	|d	|*	|.
|d	|a	|t	|o	|k	|*	|d	|i	|*	|z	|[32][S]\darr	|[33][S]\darr	|[34][S]\darr	|*	|r	|b	|*	|*	|m	|l	|b	|k	|o	|*	|.
|b	|t	|w	|n	|o	|*	|e	|u	|[35][S]\drarr	|a	|t	|a	|s	|z	|a	|t	|*	|*	|i	|i	|*	|u	|p	|*	|.
|o	|k	|i	|o	|r	|*	|c	|m	|c	|*	|ł	|u	|o	|[36][S]\drarr	|s	|e	|l	|v	|a	|s	|*	|*	|r	|*	|.
|j	|o	|c	|m	|z	|*	|r	|*	|o	|*	|u	|r	|b	|h	|t	|l	|*	|*	|n	|t	|*	|*	|ę	|*	|.
|n	|l	|a	|i	|e	|*	|o	|*	|m	|*	|m	|a	|o	|e	|a	|n	|*	|*	|o	|a	|*	|*	|g	|*	|.
|i	|i	|*	|a	|ń	|*	|l	|*	|o	|*	|a	|*	|t	|r	|*	|o	|*	|*	|w	|*	|*	|*	|i	|*	|.
|k	|c	|*	|*	|*	|*	|y	|*	|*	|*	|c	|*	|a	|t	|*	|ś	|[37][S]\rarr	|d	|o	|n	|n	|a	|*	|*	|.
|*	|a	|*	|*	|*	|*	|*	|*	|*	|*	|z	|*	|*	|z	|*	|ć	|*	|*	|*	|*	|*	|*	|*	|*	|.
|*	|*	|*	|*	|*	|*	|*	|*	|*	|*	|*	|*	|*	|*	|*	|*	|*	|*	|*	|*	|*	|*	|*	|*	|.\end{Puzzle}

\newpage

\begin{PuzzleClues}{\textbf{Poziome}\\}\Clue{3}{}{okres po epoce romantyzmu}
\Clue{5}{}{konik żmudzki - rasa koni w typie kuca, pochodząca z Litwy (Żmudzi), hodowana w krajach bałtyckich; konie wszechstronne użytkowo}
\Clue{7}{}{podatek pośredni, płacony przez osoby i instytucje prowadzące działalność gospodarczą, stanowiący alternatywę dla podatku od wartości dodanej (VAT)}
\Clue{9}{}{komplet do gry w madżonga}
\Clue{11}{}{w różnych obrządkach chrześcijańskich - czas pokuty, przygotowujący do świąt wielkanocnych, obwarowany różnymi zakazami, mającymi swe źródło w tradycji tych obrządków}
\Clue{14}{}{urządzenie rejestrujące parametry lotu samolotu; jego zapis wykorzystywany jest w celu ustalenia przyczyn i przebiegu nienormalnego zachowania samolotu}
\Clue{15}{}{Ceutorrhynchus - rodzaj chrząszcza z rodziny ryjkowcowatych}
\Clue{16}{}{metafora konfliktu, walka z kimś o coś}
\Clue{18}{}{szczupłość, niewielka liczebność (najczęściej) osób}
\Clue{19}{}{gatunek grzyba z rodziny stroczkowatych}
\Clue{20}{}{element elektroniczny bierny, który tworzy pewna liczba zwojów przewodnika, wewnątrz lub na zewnątrz których może znajdować się namagnetyzowany rdzeń}
\Clue{30}{}{regionalna wiejska potrawa kielecka; przygotowywana z utłuczonych gotowanych ziemniaków zmieszanych z ugotowaną kapustą kiszoną, omaszczonych skwarkami świeżo wytopionymi ze słoniny}
\Clue{31}{}{świecące się koło nad głową świętego (w potocznym wyobrażeniu)}
\Clue{35}{}{budynek, w którym mieści się siedziba ataszatu}
\Clue{36}{}{kraina w północnej Brazylii, w dorzeczu rzek Madeira, Purys, Jurya}
\Clue{37}{}{włoski tytuł grzecznościowy używany w stosunku do kobiet; odpowiednik polskiego zwrotupani}\end{PuzzleClues}

\begin{PuzzleClues}{\textbf{Pionowe}\\}\Clue{1}{}{miasto w USA (Dakota Płn.) nad rzeką Red}
\Clue{2}{}{tradycyjna chrzanowska potrawa jednogarnkowa z ziemniaków, jarzyn i wędlin}
\Clue{3}{}{ciąg cyfr (rzadziej liter i cyfr) dodawany do adresu, mający ułatwiać sortowanie przesyłek}
\Clue{4}{}{urządzenie linowe do podnoszenia statków}
\Clue{5}{}{przedstawiciel rodziny arekowatych, z kladu jednoliściennych}
\Clue{6}{}{fragment stopy (części elementu garderoby: pończoch, śpiochów, rajstop, którą okrywa się stopę), także: skarpet i butów, który po ubraniu znajduje się w okolicach pięty}
\Clue{7}{}{jodła nikkońska, Abies homolepis - gatunek zimozielonego drzewa z rodziny sosnowatych; pochodzi z Japonii}
\Clue{8}{}{Callilepis schuszteri - niewystępujący w Polsce gatunek pająka z rodziny worczakowatych}
\Clue{9}{}{jakaś cecha człowieka, której trudno się pozbyć (człowiek musi z nią żyć), która przysparza wielu kłopotów}
\Clue{10}{}{pazurogon pręgoudy, pazurogon rudopręgi, pazurogon udopręgi, Onychogalea fraenata - gatunek torbacza z rodziny kangurowatych, żyjący we wschodniej Australii; powodem zamieszania w ustaleniu polskiej nazwy zwyczajowej tego torbacza był chochlik drukarski}
\Clue{11}{}{słone jezioro okresowe w południowej Australii, powierzchnia 2,4 tyś. km2}
\Clue{12}{}{umiejętności, uprawnienia}
\Clue{13}{}{miasto w środkowym Izraelu, ośrodek administracyjny Okręgu Centralnego}
\Clue{15}{}{Potentilla erecta - gatunek rośliny z rodziny różowatych; nazwa ludowa}
\Clue{16}{}{Vanellus miles miles - nominatywny podgatunek ptaka wyróżniony w obrębie gatunku czajka płatkolica (Vanellus miles)}
\Clue{17}{}{włókno syntetyczne o dużej wytrzymałości na rozerwanie, służące do wyrobu tkanin tapicerskich, sieci rybackich}
\Clue{19}{}{zasłona, która okrywa głowę i piersi, noszona przez wiele muzułmanek po osiągnięciu dojrzałości w obecności dorosłych mężczyzn spoza ich najbliższej rodziny oraz kobiet nie muzułmanek}
\Clue{21}{}{sługa pozostający do końca życia na łasce pana, mimo że nie jest już w stanie wykonywać wcześniejszych prac}
\Clue{22}{}{instrument dęty blaszany o niskim dźwięku zaliczany do rodziny bugelhornów, podobnie jak skrzydłówka, sakshorn tenorowy i tuba}
\Clue{23}{}{miasto w Fidżi na wyspie Viti Levu}
\Clue{24}{}{określenie czyjegoś charakteru}
\Clue{25}{}{często pejoratywnie: mężczyzna utrzymujący homoseksualne kontakty z dojrzewającymi chłopcami}
\Clue{26}{}{rodzaj skały osadowej pochodzenia chemicznego (skała chemiczna)}
\Clue{27}{}{wytrawne wino bułgarskie}
\Clue{28}{}{cecha czegoś - sprawy, zagadnienia - które wymaga delikatności i taktu, gdyż może być drażliwe}
\Clue{29}{}{wieś sołecka w Polsce położona w województwie zachodniopomorskim, w powiecie białogardzkim, w gminie Białogard}
\Clue{30}{}{blokada, która może być wykonana z różnych materiałów; służy zabzepieczeniu sprzętów przed obijaniem ich}
\Clue{31}{}{belgijski psycholog, pedagog i lekarz psychiatra (1871-1932); przedstawiciel kierunku naturalistycznego i ruchu szkoły aktywnej w wychowaniu}
\Clue{32}{}{człowiek, który dokonuje interpretacji czegoś, objaśnia, odczytuje jakieś treści, doszukuje się w czymś znaczeń, sensu}
\Clue{33}{}{w medycynie: zespół objawów, które poprzedają atak jakiejś przypadłości neurologicznej (np. padaczki, migreny)}
\Clue{34}{}{nazwa dnia tygodnia między piątkiem a niedzielą}
\Clue{35}{}{miasto we Włoszech (Lombardia) nad jeziorem Como, ośrodek handlowy i turystyczny}
\Clue{36}{}{biolog niemiecki (1849-1922); prekursor embriologii doświadczalnej, dokonał sztucznego zapłodnienia jaja jeżowców}\end{PuzzleClues}\newpage\section*{Krzyżówka 37}

\noindent\begin{Puzzle}{25}{25}|*	|*	|*	|*	|*	|*	|*	|[1][S]\drarr	|m	|i	|e	|s	|i	|ą	|c	|z	|n	|i	|c	|a	|*	|*	|*	|*	|*	|*	|.
|*	|*	|*	|[2][S]\darr	|*	|[3][S]\drarr	|z	|d	|r	|a	|d	|l	|i	|w	|o	|ś	|ć	|*	|*	|*	|*	|*	|*	|[4][S]\darr	|*	|*	|.
|*	|[5][S]\rarr	|t	|u	|b	|m	|a	|r	|y	|n	|a	|*	|*	|*	|*	|*	|*	|*	|*	|*	|*	|*	|*	|s	|*	|*	|.
|*	|*	|[6][S]\darr	|s	|[7][S]\darr	|a	|*	|ż	|*	|*	|*	|*	|*	|[8][S]\rarr	|s	|c	|e	|n	|a	|r	|i	|u	|s	|z	|*	|*	|.
|*	|*	|d	|t	|p	|g	|[9][S]\drarr	|e	|k	|s	|t	|r	|e	|m	|i	|s	|t	|a	|*	|*	|*	|*	|*	|p	|*	|*	|.
|*	|*	|e	|e	|r	|n	|o	|n	|[10][S]\darr	|*	|*	|*	|*	|*	|*	|*	|*	|[11][S]\drarr	|n	|u	|b	|i	|r	|a	|*	|*	|.
|*	|[12][S]\drarr	|p	|r	|z	|e	|s	|i	|l	|e	|n	|i	|e	|[][,]{ }	|w	|i	|o	|s	|e	|n	|n	|e	|*	|l	|*	|*	|.
|*	|k	|u	|z	|y	|t	|i	|e	|e	|*	|*	|[13][S]\darr	|*	|*	|*	|*	|[14][S]\rarr	|p	|e	|t	|a	|b	|i	|t	|*	|*	|.
|*	|i	|t	|e	|m	|y	|o	|[][,]{ }	|w	|*	|*	|d	|*	|[15][S]\rarr	|h	|o	|l	|a	|r	|k	|t	|y	|k	|a	|*	|*	|.
|*	|n	|a	|n	|i	|c	|ł	|p	|a	|*	|*	|o	|[16][S]\darr	|*	|*	|[17][S]\darr	|[18][S]\darr	|d	|*	|*	|*	|*	|*	|*	|*	|*	|.
|*	|e	|t	|i	|o	|z	|[][,]{ }	|ę	|r	|*	|*	|s	|k	|*	|[19][S]\rarr	|z	|b	|a	|w	|i	|e	|n	|i	|e	|*	|*	|.
|*	|z	|[][,]{ }	|e	|t	|n	|a	|c	|*	|[20][S]\darr	|*	|t	|a	|[21][S]\darr	|*	|a	|o	|j	|*	|[22][S]\darr	|*	|*	|*	|[23][S]\darr	|*	|*	|.
|*	|y	|w	|[][,]{ }	|n	|o	|s	|z	|*	|t	|[24][S]\darr	|y	|c	|z	|*	|d	|ś	|ą	|*	|w	|[25][S]\darr	|*	|*	|d	|*	|*	|.
|*	|t	|ę	|p	|o	|ś	|i	|k	|*	|e	|d	|c	|z	|a	|*	|a	|n	|c	|*	|i	|l	|*	|*	|y	|*	|*	|.
|*	|e	|g	|ł	|[][,]{ }	|ć	|n	|o	|[26][S]\rarr	|t	|r	|z	|e	|b	|i	|n	|i	|a	|n	|k	|a	|*	|*	|p	|*	|*	|.
|*	|r	|l	|y	|n	|*	|a	|w	|*	|u	|a	|n	|n	|y	|*	|i	|a	|[][,]{ }	|*	|i	|g	|[27][S]\darr	|*	|l	|[28][S]\darr	|*	|.
|*	|a	|o	|t	|a	|*	|r	|e	|*	|a	|g	|a	|i	|t	|*	|e	|c	|g	|*	|p	|u	|m	|*	|o	|h	|*	|.
|*	|p	|w	|o	|g	|*	|a	|*	|*	|n	|o	|*	|c	|e	|*	|*	|k	|w	|[29][S]\darr	|e	|n	|e	|*	|m	|y	|*	|.
|*	|e	|y	|w	|i	|*	|*	|*	|*	|*	|n	|*	|a	|k	|*	|*	|o	|i	|b	|d	|a	|t	|*	|a	|d	|*	|.
|*	|u	|*	|e	|e	|[30][S]\drarr	|a	|t	|a	|r	|*	|*	|*	|*	|*	|*	|ś	|a	|e	|i	|*	|a	|*	|t	|r	|*	|.
|[31][S]\rarr	|t	|o	|*	|*	|z	|*	|*	|*	|*	|*	|*	|*	|*	|*	|*	|ć	|z	|r	|a	|*	|d	|*	|k	|o	|*	|.
|*	|a	|*	|[32][S]\rarr	|k	|a	|s	|a	|[][,]{ }	|c	|h	|o	|r	|y	|c	|h	|*	|d	|y	|*	|*	|a	|*	|a	|f	|*	|.
|*	|*	|*	|*	|*	|j	|*	|*	|*	|*	|*	|*	|*	|*	|*	|[33][S]\rarr	|m	|a	|l	|d	|i	|n	|i	|*	|o	|*	|.
|*	|*	|*	|*	|*	|o	|*	|*	|*	|[34][S]\rarr	|p	|o	|d	|c	|i	|e	|p	|*	|*	|[35][S]\rarr	|r	|e	|d	|a	|n	|*	|.
|*	|*	|*	|*	|[36][S]\rarr	|b	|a	|r	|s	|z	|c	|z	|*	|*	|*	|*	|*	|*	|*	|*	|*	|*	|*	|*	|*	|*	|.
|*	|*	|*	|*	|*	|*	|*	|*	|*	|*	|*	|*	|*	|*	|*	|*	|*	|*	|*	|*	|*	|*	|*	|*	|*	|*	|.\end{Puzzle}

\newpage

\begin{PuzzleClues}{\textbf{Poziome}\\}\Clue{1}{}{roślina zielna z rodziny krzyżowych, w Polsce w lasach górskich, uprawiana też jako ozdobna}
\Clue{3}{}{cecha zdradliwego działania lub czynu}
\Clue{5}{}{dawny instrument smyczkowy o długim korpusie rezonansowym z jedną struną}
\Clue{8}{}{materiał literacki będący podstawą realizacji fabuły filmowej lub teatralnej, zawierający dialogi oraz opis postaci i miejsc}
\Clue{9}{}{radykał; człowiek o skrajnych, radykalnych poglądach}
\Clue{11}{}{model samochodu osobowego klasy średniej, a od 2003 roku klasy kompaktowej produkowany przez koncern GM Daewoo, a następnie przez FSO}
\Clue{12}{}{ostatnie tygodnie zimy i początek wiosny często łączone z odczuwalnym spadkiem formy psychicznej i fizycznej organizmu}
\Clue{14}{}{jednostka informacji równa 10\textasciicircum15 bitów}
\Clue{15}{}{kraina zoogeograficzna obejmująca paleartykę i nearktykę}
\Clue{19}{}{uchronienie, oszczędzenie, wybawienie}
\Clue{26}{}{mieszkanka Trzebini}
\Clue{30}{}{miasto w zach. Mauretanii, ważny ośrodek handlowy}
\Clue{31}{}{struktura osobowości w modelu psychoanalitycznym}
\Clue{32}{}{instytucja ubezpieczeniowo-finansowa, której celem jest zapewnienie finansowania opieki zdrowotnej jej członkom}
\Clue{33}{}{Paolo, syn Cesare, obrońca AC Milan, jeden z najlepszych defensorów świata}
\Clue{34}{}{w dawnych wierzeniach Słowian: dziecko, które zostało podrzucone, podmienione przez mamunę - charakteryzowało się tym, że płakało i śmierdziało, co wskazywało na to, że na pewno musiało być podmienione}
\Clue{35}{}{próg w dnie ślizgowej łodzi motorowej zmniejszający powierzchnię styku dna z wodą}
\Clue{36}{}{zupa z buraków ćwikłowych zakwaszana octem lub kwaskiem cytrynowym}\end{PuzzleClues}

\begin{PuzzleClues}{\textbf{Pionowe}\\}\Clue{1}{}{szybkie, drobne skurcze grup włókienek w komókrach tkanki mięśniowej}
\Clue{2}{}{typ usterzenia samolotów, nieposiadający podziału na statecznik i ster - ruchoma jest cała, jednolita powierzchnia}
\Clue{3}{}{posiadanie tajemnej siły, przyciągającej do kogoś innych ludzi}
\Clue{4}{}{strona zadrukowana tekstem, będąca częścią książki lub czasopisma}
\Clue{6}{}{przyznawane z budżetu państwa określonym grupom społecznym uposażenie w postaci węgla do ogrzania domu lub ekwiwalenu pieniężnego}
\Clue{7}{}{Erigeron droebachiensis - gatunek roślin z rodziny astrowatych}
\Clue{9}{}{bardzo rzadka rasa osła - w 2008 r. liczba wynosiła ok. 400 egzemplarzy, w tym jeden w Polsce; prawdopodobnie pochodzi od austro-węgierskiej rasy osłów barokowych, przewiezionych na wyspę Asinara}
\Clue{10}{}{rura w kształcie odwróconej litery U służąca do przelewania cieczy z jednego zbiornika do drugiego}
\Clue{11}{}{okruchy skalne (mniejsze od planetoid) poruszające się po orbitach wokół Słońca}
\Clue{12}{}{osoba zajmująca się kinezyterapią}
\Clue{13}{}{nazwa cotangensa zaproponowana w XVIII w. przez Jana Śniadeckiego}
\Clue{16}{}{osiadły, morski skorupiak wąsonogi długości do 5 cm}
\Clue{17}{}{zbiór instrukcji programu załadowanych do pamięci RAM i wykonywanych przez procesor}
\Clue{18}{}{fakt, że coś jest bośniackie, zwłaszcza: pochodzi z Bośni i Hercegowiny, należy do Bośni i Hercegowiny, przynależy do bośniackiej kultury}
\Clue{20}{}{prowincja w północnym Maroku, główne miasto Tetuan}
\Clue{21}{}{budowla o wartości historycznej, naukowej lub artystycznej}
\Clue{22}{}{wielojęzyczna encyklopedia internetowa działająca w oparciu o zasadę otwartej treści}
\Clue{23}{}{kobieta, która zawsze postępuje politycznie, taktownie}
\Clue{24}{}{żartobliwie o potężnej, despotycznej kobiecie}
\Clue{25}{}{część morza odcięta od niego przez lido, rafę barierową lub atol, która jest zazwyczaj dość płytka}
\Clue{27}{}{ustrukturalizowane informacje stosowane do opisu zasobów informacji lub obiektów informacji}
\Clue{28}{}{mikrofon służący do odbierania dźwięków rozchodzących się w wodzie lub innych cieczach; podstawowy element konstrukcyjny sonarów pasywnych}
\Clue{29}{}{minerał należący do grupy krzemianów pierścieniowych}
\Clue{30}{}{osoba, która ma zajoba, korbę, jest nakręcona, napalona na coś}\end{PuzzleClues}\newpage\section*{Krzyżówka 38}

\noindent\begin{Puzzle}{24}{25}|*	|*	|*	|[1][S]\darr	|[2][S]\drarr	|f	|u	|n	|k	|c	|j	|a	|n	|a	|*	|[3][S]\drarr	|u	|w	|e	|r	|t	|u	|r	|a	|*	|.
|*	|*	|[4][S]\rarr	|s	|t	|e	|k	|*	|[5][S]\rarr	|p	|s	|o	|t	|n	|i	|k	|*	|*	|*	|[6][S]\darr	|*	|*	|*	|[7][S]\darr	|*	|.
|*	|*	|[8][S]\drarr	|t	|r	|a	|s	|z	|k	|a	|[][,]{ }	|ż	|a	|b	|i	|o	|z	|ę	|b	|n	|a	|*	|*	|d	|*	|.
|*	|*	|f	|a	|z	|*	|*	|*	|*	|*	|*	|*	|*	|*	|[9][S]\drarr	|r	|z	|y	|g	|a	|c	|z	|*	|ż	|*	|.
|*	|*	|a	|n	|ę	|[10][S]\darr	|*	|[11][S]\drarr	|ś	|l	|i	|z	|g	|*	|c	|d	|*	|*	|*	|p	|[12][S]\darr	|*	|[13][S]\darr	|o	|*	|.
|*	|*	|r	|*	|s	|z	|[14][S]\rarr	|w	|a	|n	|*	|*	|*	|*	|h	|o	|*	|*	|[15][S]\darr	|r	|k	|*	|k	|k	|*	|.
|*	|*	|b	|*	|i	|b	|*	|y	|*	|[16][S]\rarr	|l	|e	|z	|g	|i	|n	|*	|[17][S]\darr	|r	|ę	|a	|*	|a	|e	|*	|.
|*	|[18][S]\darr	|a	|*	|d	|i	|[19][S]\rarr	|m	|y	|s	|z	|o	|w	|ó	|r	|*	|*	|b	|y	|ż	|n	|*	|w	|r	|[20][S]\darr	|.
|*	|t	|[][,]{ }	|*	|ł	|ó	|*	|i	|*	|[21][S]\rarr	|r	|e	|t	|r	|o	|g	|r	|a	|d	|a	|c	|j	|a	|*	|a	|.
|*	|r	|p	|[22][S]\darr	|a	|r	|[23][S]\rarr	|a	|y	|d	|i	|n	|*	|*	|p	|*	|*	|j	|z	|c	|o	|*	|[][,]{ }	|[24][S]\darr	|m	|.
|*	|ó	|r	|m	|*	|[][,]{ }	|[25][S]\drarr	|n	|a	|r	|o	|ż	|e	|*	|t	|*	|*	|a	|*	|z	|n	|*	|p	|s	|i	|.
|*	|j	|o	|u	|*	|w	|d	|a	|[26][S]\rarr	|j	|a	|h	|n	|k	|e	|*	|[27][S]\darr	|*	|*	|*	|a	|*	|o	|p	|n	|.
|[28][S]\drarr	|e	|s	|t	|e	|y	|a	|*	|*	|*	|*	|*	|[29][S]\drarr	|t	|r	|a	|m	|p	|e	|k	|*	|*	|[][,]{ }	|e	|o	|.
|s	|d	|z	|u	|*	|p	|g	|*	|[30][S]\rarr	|m	|u	|r	|a	|k	|o	|z	|i	|*	|[31][S]\darr	|*	|[32][S]\darr	|[33][S]\darr	|t	|c	|f	|.
|p	|n	|k	|a	|[34][S]\darr	|u	|l	|*	|[35][S]\darr	|*	|*	|*	|f	|*	|l	|*	|n	|*	|p	|*	|b	|g	|u	|y	|e	|.
|ó	|i	|o	|l	|m	|k	|e	|*	|m	|[36][S]\darr	|*	|*	|l	|*	|o	|*	|u	|*	|o	|*	|e	|l	|r	|f	|n	|.
|ł	|k	|w	|i	|i	|ł	|z	|[37][S]\darr	|a	|m	|*	|*	|*	|*	|g	|*	|c	|*	|k	|[38][S]\darr	|z	|a	|e	|i	|o	|.
|k	|[][,]{ }	|a	|z	|j	|y	|j	|r	|ł	|e	|*	|*	|*	|[39][S]\darr	|i	|*	|j	|*	|[][S]é	|d	|j	|u	|c	|k	|l	|.
|a	|p	|*	|m	|a	|*	|a	|u	|p	|g	|[40][S]\rarr	|p	|ę	|p	|a	|w	|a	|[][,]{ }	|m	|i	|ę	|k	|k	|a	|*	|.
|[][,]{ }	|o	|*	|*	|n	|*	|[][,]{ }	|s	|k	|a	|*	|*	|*	|l	|*	|*	|*	|*	|o	|a	|z	|o	|u	|c	|*	|.
|j	|w	|*	|[41][S]\rarr	|k	|o	|s	|z	|a	|l	|i	|n	|i	|a	|n	|k	|a	|*	|n	|p	|y	|z	|*	|j	|*	|.
|a	|a	|*	|*	|a	|*	|i	|t	|*	|o	|*	|*	|[42][S]\rarr	|m	|o	|h	|e	|r	|*	|a	|k	|a	|*	|a	|*	|.
|w	|b	|*	|*	|*	|*	|n	|*	|*	|d	|[43][S]\rarr	|b	|l	|i	|n	|d	|a	|ż	|*	|z	|o	|u	|*	|*	|*	|.
|n	|n	|[44][S]\rarr	|ś	|w	|i	|a	|t	|ł	|o	|[][,]{ }	|z	|i	|e	|l	|o	|n	|e	|*	|o	|w	|r	|*	|*	|*	|.
|a	|y	|[45][S]\rarr	|b	|o	|t	|*	|*	|*	|n	|[46][S]\rarr	|k	|o	|c	|i	|e	|ł	|*	|*	|n	|e	|*	|*	|*	|*	|.
|*	|*	|*	|*	|*	|*	|*	|*	|*	|*	|*	|*	|*	|*	|*	|*	|*	|*	|*	|*	|*	|*	|*	|*	|*	|.\end{Puzzle}

\newpage

\begin{PuzzleClues}{\textbf{Poziome}\\}\Clue{2}{}{funkcja przyjmująca jako swoje wartości wszystkie elementy przeciwdziedziny, przy czym każdemu z tych elementów odpowiada co najmniej jeden element dziedziny}
\Clue{3}{}{muzyka instrumentalna skomponowana jako wstęp do opery, oratorium, kantaty itp}
\Clue{4}{}{KLOAKA; końcowy odcinek przewodu pokarmowego u gadów i stekowców}
\Clue{5}{}{owad uskrzydlony z rzędu psotników}
\Clue{8}{}{żaboząb ałatauski, Ranodon sibiricus - gatunek płaza ogoniastego z rodziny kątozębnych, występujący w Azji}
\Clue{9}{}{GARGULEC; ozdobne zakończenie rynny dachowej często w formie paszcz; lub fantastycznych zwierząt}
\Clue{11}{}{pojedynczy zjazd w saneczkarstwie sportowym, wykonany na specjalnie przygotowanym torze}
\Clue{14}{}{sieć komputerowa znajdująca się na obszarze wykraczającym poza jedno miasto (bądź kompleks miejski)}
\Clue{16}{}{członek kaukaskiej grupy etnicznej, żyjącej głównie w południowo-wschodnim Dagestanie (gdzie stanowi ok. 10-11\% ludności) i północnym Azerbejdżanie}
\Clue{19}{}{TAFA}
\Clue{21}{}{zjawisko polegające na przemianie formy spiralnej skrobi w liniową i porządkowaniu się wyprostowanych łańcuchów amylozy w zwarte micele, których strukturę stabilizują wiązania wodorowe tworzące się między grupami hydroksylowymi cząsteczek skrobi położonych blisko siebie}
\Clue{23}{}{miasto w płd.-zach. Turcji w dolinie rzeki Menderes}
\Clue{25}{}{narożnik, róg jakiegoś przedmiotu}
\Clue{26}{}{skrzypek i pedagog (1895-1972); zorganizował Filharmonię w Poznaniu}
\Clue{28}{}{amerykańskie organy}
\Clue{29}{}{sportowy, sznurowany but płócienny z gumową podeszwą}
\Clue{30}{}{rasa zimnokrwistego konia domowego, pochodząca z okolic rzeki Mura w południowych Węgrzech; powstała poprzez krzyżowanie miejscowych węgierskich i polskich klaczy z Ardenami, Perszeronami i Norikerami, a na początku XX wieku do rasy została wprowadzona czysta krew arabska}
\Clue{40}{}{Crepis mollis - gatunek rośliny należący do rodziny astrowatych}
\Clue{41}{}{mieszkanka Koszalina}
\Clue{42}{}{wełna kóz angorskich}
\Clue{43}{}{opancerzenie okrętu wojennego}
\Clue{44}{}{element sygnalizacji świetlnej, którego zapalenie się oznacza możliwość przejechania lub przejścia przed drogę bądź skrzyżowanie}
\Clue{45}{}{program przeznaczony do wykonywania czynności w zastępstwie człowieka}
\Clue{46}{}{duży, metalowy garnek, najczęściej zamykany pokrywą}\end{PuzzleClues}

\begin{PuzzleClues}{\textbf{Pionowe}\\}\Clue{1}{}{talia, wcięcie w pasie, kibić}
\Clue{2}{}{kita z piór zdobiąca końskie głowy w czasie parad lub innych uroczystości}
\Clue{3}{}{system posterunków granicznych}
\Clue{6}{}{mięsień naprężacz powięzi szerokiej, łac. Musculus tensor fasciae latae - mięsień znajdujący się na kończynie dolnej, zaliczany do grupy tylnej mięśni obręczy kończyny dolnej; jego głównym zadaniem jest napinanie pasma biodrowo-piszczelowego, przez co stabilizuje wyprostowany staw kolanowy}
\Clue{7}{}{karta do gry z wizerunkiem trefnisia}
\Clue{8}{}{termoutwardzalna substancja na bazie polimerów, służąca do malowania proszkowego}
\Clue{9}{}{gałąź zoologii (teriologii) zajmująca się badaniem nietoperzy}
\Clue{10}{}{podzbiór pewnej przestrzeni zawierający wraz dowolnymi dwoma jego punktami odcinek je łączący}
\Clue{11}{}{porozumiewanie się, przekazywanie myśli, informowanie}
\Clue{12}{}{włoska pieśń świecka, wielogłosowa z XVI w przeniesiona do muzyki instrumentalnej, wstępna forma fugi i suity}
\Clue{13}{}{czarna kawa parzona na swoisty sposób (gotowana w wodzie, od razu z dodatkiem cukru) w tygielku}
\Clue{15}{}{Camelina - roślina z rodziny kapustowatych}
\Clue{17}{}{pulchna, miękka tkanina bawełniana o luźno skręconym podwójnym wątku, grubsza flanela}
\Clue{18}{}{Calochortus venustus - gatunek roślin zielnych z rodziny liliowatych}
\Clue{20}{}{organiczny związek chemiczny z grupy fenoli zawierający aminową grupę funkcyjną (-NH2)}
\Clue{22}{}{jedna z interakcji protekcjonistycznych między populacjami, charakteryzująca się obopólnymi korzyściami o takim stopniu, który praktycznie wzajemnie uzależnia istnienie obu populacji}
\Clue{24}{}{wyszczególnienie kosztów przedsiębiorstwa}
\Clue{25}{}{jedlica sina, Pseudotsuga menziesii subsp. glauca - podgatunek jedlicy zielonej, drzewa z rodziny sosnowatych; od podgatunku nominatywnego różni się pokrojem, igłami i szyszkami, ma także większą odporność na mrozy}
\Clue{27}{}{pisany prozą lub wierszem kalendarz z przepowiedniami lub prognozami meteorologicznymi zawierający cechy parodii}
\Clue{28}{}{osobowa spółka prowadząca przedsiębiorstwo pod własną firmą i niebędąca inną spółką handlową}
\Clue{29}{}{skrót/symbol florina arubiańskiego}
\Clue{31}{}{nazwa serii gier konsolowych firmy Nintendo tworzonych od 1996 przez Satoshiego Tajiri}
\Clue{32}{}{grzbietorodowate, żaby bezjęzyczne, grzbietorody, Pipidae - rodzina płazów należąca do podrzędu (w zależności od systematyki): Pipoidea (według starszej) albo Mesobatrachia (według nowszej), z rzędu płazów bezogonowych, w obrębie której wyróżnia się 5 rodzajów i 33 gatunki}
\Clue{33}{}{Glaukozaur - rodzaj gada ssakokształtnego z rodziny edafozaurów; żył we wczesnym permie na terenie dzisiejszej Ameryki Północnej}
\Clue{34}{}{fragment trasy, miejsce specjalnie dostosowane, by mogły się tam mijać pojazdy jadące w przeciwnych kierunkach}
\Clue{35}{}{zdrobniale: małpa - zwierzę}
\Clue{36}{}{Carcharodon megalodon - wymarły gatunek ryby chrzęstnoszkieletowej, prehistoryczny rekin, największa ze znanych ryb drapieżnych; megalodon żył 25-1,5 mln lat temu w morzach oligocenu, miocenu, pliocenu i plejstocenu}
\Clue{37}{}{żelazne pręty, które są częścią paleniska pieca i są ułożone w taki sposób, by umożliwiać pieczenie na nich czegoś}
\Clue{38}{}{KAMERTON}
\Clue{39}{}{motyl nocny, gąsienica żeruje na agreście; agreściak}\end{PuzzleClues}\newpage\section*{Krzyżówka 39}

\noindent\begin{Puzzle}{21}{19}|*	|*	|*	|*	|*	|*	|*	|*	|*	|*	|*	|*	|*	|[1][S]\drarr	|l	|i	|p	|i	|c	|a	|n	|*	|.
|*	|*	|*	|[2][S]\darr	|*	|[3][S]\rarr	|g	|e	|o	|t	|e	|c	|h	|n	|i	|k	|a	|*	|*	|*	|*	|*	|.
|*	|*	|*	|g	|*	|[4][S]\rarr	|n	|a	|r	|z	|ę	|d	|z	|i	|a	|r	|n	|i	|a	|*	|*	|*	|.
|[5][S]\rarr	|p	|o	|r	|o	|ś	|l	|e	|*	|*	|[6][S]\rarr	|g	|r	|e	|e	|n	|*	|*	|*	|*	|*	|*	|.
|*	|[7][S]\rarr	|d	|o	|k	|t	|o	|r	|[][,]{ }	|h	|a	|b	|i	|l	|i	|t	|o	|w	|a	|n	|y	|*	|.
|*	|[8][S]\darr	|*	|s	|[9][S]\darr	|*	|*	|*	|[10][S]\rarr	|e	|m	|b	|r	|i	|o	|p	|a	|t	|i	|a	|*	|*	|.
|[11][S]\rarr	|p	|r	|z	|e	|j	|e	|ż	|d	|ż	|a	|j	|ą	|c	|y	|*	|*	|*	|*	|*	|*	|*	|.
|*	|o	|*	|e	|g	|*	|*	|*	|*	|*	|*	|*	|*	|z	|[12][S]\darr	|*	|*	|*	|*	|*	|*	|*	|.
|*	|z	|*	|k	|k	|*	|*	|*	|*	|*	|*	|[13][S]\darr	|*	|n	|p	|*	|*	|*	|*	|*	|*	|*	|.
|*	|y	|*	|*	|*	|*	|*	|*	|*	|*	|*	|t	|[14][S]\darr	|o	|i	|*	|*	|*	|[15][S]\darr	|*	|*	|[16][S]\darr	|.
|[17][S]\rarr	|c	|h	|e	|r	|u	|b	|i	|n	|*	|*	|u	|s	|ś	|c	|*	|*	|*	|e	|[18][S]\darr	|*	|r	|.
|[19][S]\rarr	|j	|u	|n	|k	|e	|r	|s	|*	|*	|*	|s	|p	|ć	|t	|[20][S]\darr	|[21][S]\darr	|[22][S]\darr	|u	|d	|*	|e	|.
|*	|a	|*	|*	|[23][S]\darr	|*	|*	|*	|[24][S]\rarr	|a	|s	|z	|a	|*	|e	|b	|k	|b	|r	|o	|*	|s	|.
|*	|*	|*	|[25][S]\rarr	|p	|t	|a	|s	|z	|ę	|*	|*	|r	|*	|t	|r	|o	|e	|o	|j	|[26][S]\darr	|e	|.
|*	|*	|*	|*	|u	|*	|*	|*	|[27][S]\rarr	|s	|l	|o	|t	|y	|*	|ą	|b	|l	|l	|ś	|c	|t	|.
|[28][S]\rarr	|c	|z	|u	|l	|e	|n	|t	|*	|[29][S]\rarr	|z	|n	|i	|e	|c	|z	|u	|l	|a	|c	|z	|*	|.
|*	|[30][S]\rarr	|t	|o	|m	|i	|*	|[31][S]\rarr	|p	|o	|l	|l	|a	|c	|k	|*	|z	|o	|n	|i	|o	|*	|.
|*	|*	|*	|*	|a	|*	|[32][S]\rarr	|n	|a	|p	|i	|ę	|t	|e	|k	|*	|*	|c	|d	|e	|p	|*	|.
|[33][S]\rarr	|s	|i	|t	|n	|i	|k	|[][,]{ }	|p	|ł	|y	|w	|a	|j	|ą	|c	|y	|*	|*	|*	|*	|*	|.
|*	|*	|*	|*	|*	|[34][S]\rarr	|c	|i	|e	|p	|ł	|o	|*	|*	|*	|*	|*	|*	|*	|*	|*	|*	|.\end{Puzzle}

\newpage

\begin{PuzzleClues}{\textbf{Poziome}\\}\Clue{1}{}{rasa koni gorącokwistych powstała z rodzimych koni Karster krzyżowanych z końmi hiszpańskimi i neapolitańskimi; rasa wzięła nazwę od stadniny, w której została wyhodowana - Lipicy w Słowenii, założonej w 1580}
\Clue{3}{}{gałąź inżynierii ściśle związana z inżynierią lądową, nauka o pracy i badaniach ośrodka gruntowego dla celów projektowania i wykonawstwa budowli ziemnych i podziemnych oraz fundamentów budynków i nawierzchni drogowych}
\Clue{4}{}{pomieszczenie zakładu przemysłowego służące jako magazyn narzędzi}
\Clue{5}{}{roślina rosnąca na innej roślinie, ale zwykle nieprowadząca pasożytniczego trybu życia; korzysta z innego gatunku jako podpory, a odżywia się najczęściej samodzielnie}
\Clue{6}{}{obszar pola golfowego z bardzo krótko przystrzyżoną trawą, na którym znajduje się dołek}
\Clue{7}{}{najwyższy stopień naukowy w Polsce}
\Clue{10}{}{zaburzenia rozwoju płodu pomiędzy 3 a 8 tygodniem, prowadzące do chorób i wad rozwojowych}
\Clue{11}{}{ten, kto przejeżdża przez jakieś miejsce, kto jest gdzieś przejazdem}
\Clue{17}{}{piękny młodzieniec}
\Clue{19}{}{samolot, używany podczas II wojny światowej, niemieckiej marki Junkers}
\Clue{24}{}{miejscowość w Federacji Rosyjskiej u podnóża Uralu Południowego, węzeł kolejowy}
\Clue{25}{}{młoda osoba, która ma przed sobą piękną przyszłość}
\Clue{27}{}{SKRZELA; dodatkowe małe skrzydełka umieszczane na przedniej krawędzi skrzydeł samolotu}
\Clue{28}{}{tradycyjna potrawa kuchni żydowskiej}
\Clue{29}{}{środek znieczulający, znieczulenie - środek powodujący przerwanie przewodzenia impulsów nerwowych i nieodczuwanie bólu}
\Clue{30}{}{gazela Thomsona, Gazella thomsoni, Gazella thomsonii - przedstawiciel rodziny krętorogich z rzędu parzystokopytnych; występuje w Kenii i Tanzanii}
\Clue{31}{}{Sydney Pollack - amerykański reżyser, aktor i producent filmowy}
\Clue{32}{}{jedna z kości stępu, na której opiera się u góry piszczel, u dołu łącząca się z kością piętową, po bokach z kostkami goleni, a z przodu z kością łódkowatą}
\Clue{33}{}{Isolepis fluitans, Scirpus fluitans - rodzaj roślin z rodziny ciborowatych}
\Clue{34}{}{jeden z dwóch, obok pracy, sposobów przekazywania energii wewnętrznej układowi termodynamicznemu; przekazywanie energii chaotycznego ruchu cząstek}\end{PuzzleClues}

\begin{PuzzleClues}{\textbf{Pionowe}\\}\Clue{1}{}{to, że coś jest nieliczne, mało liczne}
\Clue{2}{}{warzywo, zielone niesuszone nasiona groszku cukrowego}
\Clue{8}{}{miejsce, status, znaczenie}
\Clue{9}{}{kompozytor niemiecki (1901-83); opery, balety, utwory orkiestrowe}
\Clue{12}{}{fizyk szwajcarski (1846-1929); badacz niskich temperatur, twórca projektu pierwszej maszyny chłodniczej}
\Clue{13}{}{kosmetyk służący do podkreślenia i malowania rzęs, dopełniający makijażu oka}
\Clue{14}{}{pełnoprawny obywatel starożytnej Sparty}
\Clue{15}{}{grupa państw, których wspólną walutą jest euro}
\Clue{16}{}{stan człowieka polegający na całkowitym odpoczynku, odcięciu się od sytuacji stresowych, drażniących sytuacji}
\Clue{18}{}{o listach, przesyłkach: zostać dostarczonym}
\Clue{20}{}{stop miedzi z cyną}
\Clue{21}{}{drapieżny ptak wielkości gołębia, ceniony w sokolnictwie, w Polsce rzadki, chroniony}
\Clue{22}{}{(1870-1953), angielski pisarz katolicki pochodzenia francuskiego, powieści satyryczne, eseistyka, biografie, wiersze}
\Clue{23}{}{wagon pasażerski, który ma długi, wąski korytarz, do którego prowadzą wejścia do przedziałów}
\Clue{26}{}{klejek - grzyb z rodziny klejówkowatych}\end{PuzzleClues}\newpage\section*{Krzyżówka 40}

\noindent\begin{Puzzle}{24}{24}|*	|*	|[1][S]\drarr	|p	|o	|z	|n	|a	|n	|i	|a	|n	|k	|a	|*	|*	|[2][S]\drarr	|b	|l	|o	|k	|i	|n	|g	|*	|.
|*	|[3][S]\darr	|t	|[4][S]\rarr	|ł	|ą	|c	|z	|n	|i	|k	|[][,]{ }	|a	|u	|t	|o	|m	|a	|t	|y	|c	|z	|n	|y	|*	|.
|*	|a	|r	|[5][S]\rarr	|t	|i	|t	|i	|[][,]{ }	|c	|z	|e	|r	|w	|o	|n	|y	|*	|*	|*	|*	|*	|*	|[6][S]\darr	|*	|.
|[7][S]\rarr	|r	|z	|ę	|s	|k	|a	|*	|*	|*	|*	|*	|*	|*	|[8][S]\drarr	|w	|s	|t	|r	|z	|ą	|s	|*	|k	|*	|.
|[9][S]\rarr	|c	|y	|r	|k	|i	|e	|l	|*	|*	|*	|[10][S]\darr	|*	|*	|p	|*	|z	|*	|*	|*	|[11][S]\darr	|*	|*	|u	|*	|.
|*	|h	|d	|[12][S]\drarr	|k	|e	|s	|*	|*	|*	|[13][S]\darr	|s	|*	|[14][S]\darr	|r	|[15][S]\darr	|[][,]{ }	|*	|[16][S]\darr	|*	|s	|[17][S]\darr	|*	|r	|*	|.
|*	|i	|z	|p	|*	|*	|*	|[18][S]\darr	|*	|*	|w	|i	|*	|s	|z	|s	|p	|*	|b	|*	|k	|z	|[19][S]\darr	|z	|*	|.
|*	|t	|i	|o	|*	|*	|*	|k	|*	|*	|y	|ł	|*	|a	|y	|k	|a	|*	|i	|[20][S]\drarr	|r	|a	|f	|a	|*	|.
|*	|e	|e	|j	|*	|[21][S]\rarr	|f	|a	|r	|b	|k	|a	|*	|m	|k	|w	|n	|*	|e	|b	|ó	|m	|i	|w	|*	|.
|*	|k	|s	|a	|[22][S]\rarr	|o	|d	|r	|o	|ś	|l	|*	|*	|o	|o	|a	|c	|[23][S]\darr	|g	|o	|t	|e	|l	|k	|*	|.
|*	|t	|t	|z	|*	|*	|[24][S]\darr	|c	|[25][S]\darr	|*	|i	|[26][S]\rarr	|s	|u	|p	|r	|e	|m	|u	|m	|*	|k	|t	|a	|*	|.
|[27][S]\rarr	|k	|o	|d	|[][,]{ }	|r	|o	|z	|w	|i	|n	|i	|ę	|t	|y	|*	|r	|y	|n	|b	|*	|[][,]{ }	|r	|*	|*	|.
|*	|a	|p	|[][,]{ }	|*	|*	|r	|o	|y	|*	|a	|[28][S]\darr	|*	|w	|*	|[29][S]\darr	|n	|s	|[][,]{ }	|a	|*	|b	|[][,]{ }	|*	|*	|.
|*	|[][,]{ }	|a	|l	|*	|*	|g	|w	|p	|*	|[][,]{ }	|p	|[30][S]\darr	|i	|[31][S]\darr	|w	|a	|z	|m	|[][,]{ }	|*	|ł	|b	|*	|*	|.
|*	|w	|r	|a	|*	|*	|a	|n	|y	|[32][S]\drarr	|t	|e	|l	|e	|k	|s	|*	|[][,]{ }	|a	|g	|*	|y	|a	|*	|*	|.
|*	|n	|o	|t	|*	|*	|n	|i	|c	|g	|a	|t	|i	|r	|r	|i	|*	|w	|g	|ł	|*	|s	|r	|*	|*	|.
|*	|ę	|l	|a	|*	|*	|*	|k	|h	|o	|t	|a	|s	|d	|y	|u	|*	|o	|n	|ę	|*	|k	|w	|*	|*	|.
|*	|t	|a	|j	|*	|*	|*	|*	|a	|p	|r	|f	|t	|z	|m	|r	|*	|r	|e	|b	|*	|a	|n	|*	|*	|.
|*	|r	|t	|ą	|*	|*	|*	|*	|c	|u	|z	|l	|k	|e	|i	|*	|*	|k	|t	|i	|*	|w	|y	|*	|*	|.
|*	|z	|k	|c	|*	|*	|*	|*	|z	|r	|a	|o	|o	|n	|n	|*	|*	|o	|y	|n	|*	|i	|*	|*	|*	|.
|*	|*	|a	|y	|*	|*	|*	|*	|*	|a	|ń	|p	|w	|i	|a	|*	|*	|w	|c	|o	|*	|c	|*	|*	|*	|.
|*	|*	|*	|*	|[33][S]\rarr	|b	|y	|ł	|y	|*	|s	|s	|i	|e	|ł	|*	|*	|a	|z	|w	|*	|z	|*	|*	|*	|.
|*	|[34][S]\rarr	|c	|y	|b	|e	|r	|a	|t	|a	|k	|*	|e	|*	|e	|*	|*	|t	|n	|a	|*	|n	|*	|*	|*	|.
|*	|*	|[35][S]\rarr	|s	|e	|k	|r	|e	|c	|j	|a	|*	|c	|*	|k	|*	|*	|a	|y	|*	|*	|y	|*	|*	|*	|.
|[36][S]\rarr	|ż	|y	|w	|o	|t	|n	|o	|ś	|ć	|*	|*	|*	|*	|*	|*	|*	|*	|*	|*	|*	|*	|*	|*	|*	|.\end{Puzzle}

\newpage

\begin{PuzzleClues}{\textbf{Poziome}\\}\Clue{1}{}{mieszkanka Poznania}
\Clue{2}{}{gra blokiem w siatkówce}
\Clue{4}{}{łącznik zdolny do samodzielnego działania}
\Clue{5}{}{Callicebus moloch - gatunek małpy szerokonosej; zamieszkuje północną Amerykę Południową, zasiedlając lasy w pobliżu bagien i niewielkich zbiorników wodnych}
\Clue{7}{}{mała rzęsa - włos na powiece}
\Clue{8}{}{gwałtowne drgnięcie, np. wstrząs sejsmiczny}
\Clue{9}{}{CIRCINUS; gwiazdozbiór nieba południowego}
\Clue{12}{}{kod ISO 4217 szylinga kenijskiego}
\Clue{20}{}{struktura podwodna powstała przez nagromadzenie szkieletów organizmów rafotwórczych (m.in. koralowców)}
\Clue{21}{}{zdrobniale o farbie - substancji barwiącej}
\Clue{22}{}{odrost - pęd rozwijający się z pąku śpiącego lub przybyszowego na korzeniach lub w dolnej części pnia roślin drzewiastych}
\Clue{26}{}{w matematyce - największe z ograniczeń górnych danego zbioru}
\Clue{27}{}{bogatsza, uniwersalna forma kodu językowego umożliwiająca porozumiewanie się z każdym innym użytkownikiem języka, który ją zna; według Bernsteina kod ten był typowy dla klasy średniej i wyższej}
\Clue{32}{}{abonencka usługa telegraficzna, polegająca na przesyłaniu informacji w postaci alfanumerycznej, z repertuarem znaków określonym alfabetem telegraficznym}
\Clue{33}{}{były partner - osoba, która była kiedyś w związku, lecz teraz już nie jest}
\Clue{34}{}{napad, atak, które wykonywane są za pomocą sieci Internet}
\Clue{35}{}{proces wytwarzania i uwalniania substancji chemicznych z tkanek lub gruczołów w określonym celu}
\Clue{36}{}{ważność, aktualność czegoś}\end{PuzzleClues}

\begin{PuzzleClues}{\textbf{Pionowe}\\}\Clue{1}{}{kobieta mająca więcej niż 30 lat, przed 40. rokiem życia}
\Clue{2}{}{pancernik karłowaty, Chlamyphorus truncatus - gatunek ssaka łożyskowego z rodziny pancerników, najmniejszy przedstawiciel rodziny pancerników; występuje tylko w Argentynie (gatunek endemiczny), na odkrytych, piaszczystych terenach tuż pod cienką warstwą piasku, zwykle w pobliżu mrowisk}
\Clue{3}{}{kobieta zajmująca się architekturą wnętrz}
\Clue{6}{}{drobnoziarnisty luźny osad, np. piasek lub muł nasycony wodą pod znacznym ciśnieniem}
\Clue{8}{}{aprosze: zygzakowate rowy umożliwiające oblegającym podejście do murów twierdzy}
\Clue{10}{}{poważna, zauważalna moc o dużym natężeniu i dynamice, która jest w stanie w dostrzegalny sposób oddziaływać na otoczenie; para}
\Clue{11}{}{fakt skrócenia, np. dokonać skrótu}
\Clue{12}{}{urządzenie zdolne do poruszania się w atmosferze lub w przestrzeni kosmicznej}
\Clue{13}{}{Poa nobilis - gatunek rośliny z rodziny wiechlinowatych}
\Clue{14}{}{nabranie przekonania o własnej wartości}
\Clue{15}{}{upał, bardzo wysoka temperatura powietrza}
\Clue{16}{}{miejsce igły magnetycznej, magnesu trwałego lub elektromagnesu, w którym natężenie pola magnetycznego ma największą wartość}
\Clue{17}{}{rodzaj zapięcia złożonego z dwóch rzędów ząbków i części do przesuwania}
\Clue{18}{}{eurazjatycki drobny gryzoń, szkodnik sadów i upraw leśnych}
\Clue{19}{}{płaska nasadka na obiektyw zmieniająca wygląd fotografowanego obrazu, nakładana najczęściej na przód obiektywu, a jeśli jest to niemożliwe, to na jego tył}
\Clue{20}{}{broń morska, przeznaczona do rażenia okrętów podwodnych znajdujących się w zanurzeniu}
\Clue{23}{}{myszowór, tafa, tuan, Phascogale tapoatafa - gatunek niewielkiego mięsożernego torbacza z rodziny niełazowatych; występuje we wszystkich stanach Australii z wyjątkiem Tasmanii}
\Clue{24}{}{wyodrębniony w celu wykonywania określonych zadań podmiot (osoba lub grupa osób), którego kompetencje określają normy zewnętrzne albo wewnętrzne (najczęściej normy prawne); termin używany przede wszystkim w prawie, administracji i zarządzaniu}
\Clue{25}{}{w rzemiośle: narzędzie służące do wypychania czegoś}
\Clue{28}{}{jednostka mocy obliczeniowej komputerów wynosząca10\textasciicircum15 flopsów}
\Clue{29}{}{prostak, wieśniak, osoba, która swoim zachowaniem przynosi wstyd, żenuje innych, bo jest niewychowana}
\Clue{30}{}{CYTRYNEK}
\Clue{31}{}{zdrobniale: kryminał - powieść kryminalna}
\Clue{32}{}{indyjska brama w kształcie wielokondygnacyjnej wieży}\end{PuzzleClues}\newpage\section*{Krzyżówka 41}

\noindent\begin{Puzzle}{21}{33}|*	|*	|[1][S]\darr	|*	|*	|*	|*	|*	|*	|*	|*	|*	|*	|*	|*	|*	|*	|*	|*	|*	|*	|*	|.
|*	|*	|s	|*	|*	|[2][S]\darr	|*	|*	|*	|*	|[3][S]\darr	|*	|[4][S]\darr	|*	|*	|[5][S]\darr	|*	|*	|*	|[6][S]\darr	|*	|*	|.
|[7][S]\rarr	|s	|a	|l	|u	|m	|a	|e	|*	|*	|p	|[8][S]\darr	|g	|*	|[9][S]\drarr	|b	|u	|s	|z	|k	|o	|*	|.
|*	|*	|l	|*	|*	|a	|[10][S]\darr	|*	|*	|*	|i	|g	|e	|[11][S]\darr	|h	|r	|[12][S]\darr	|*	|*	|o	|*	|*	|.
|[13][S]\rarr	|w	|a	|l	|o	|n	|k	|i	|*	|*	|e	|i	|m	|ż	|a	|z	|h	|[14][S]\darr	|*	|n	|*	|*	|.
|*	|*	|*	|*	|*	|d	|r	|*	|*	|*	|t	|b	|i	|y	|l	|u	|a	|t	|*	|t	|*	|[15][S]\darr	|.
|*	|*	|[16][S]\darr	|*	|*	|a	|e	|*	|*	|*	|r	|o	|n	|r	|i	|s	|s	|r	|*	|a	|*	|k	|.
|*	|*	|ł	|*	|[17][S]\rarr	|p	|o	|c	|h	|ł	|a	|n	|i	|a	|c	|z	|*	|z	|*	|k	|*	|o	|.
|[18][S]\drarr	|m	|a	|n	|g	|a	|l	|u	|r	|*	|s	|[][,]{ }	|*	|f	|z	|e	|*	|c	|*	|c	|*	|p	|.
|b	|*	|k	|*	|*	|*	|k	|*	|*	|*	|*	|s	|*	|a	|*	|k	|*	|i	|*	|i	|*	|e	|.
|r	|[19][S]\darr	|n	|*	|[20][S]\drarr	|l	|a	|w	|a	|t	|e	|r	|z	|*	|*	|*	|[21][S]\darr	|n	|*	|k	|*	|r	|.
|o	|p	|i	|*	|v	|*	|*	|*	|*	|*	|*	|e	|[22][S]\rarr	|k	|o	|ł	|b	|a	|ń	|*	|*	|c	|.
|m	|o	|e	|[23][S]\darr	|o	|*	|*	|*	|*	|*	|[24][S]\rarr	|b	|r	|o	|n	|t	|e	|*	|*	|*	|[25][S]\darr	|z	|.
|o	|ł	|n	|z	|t	|*	|*	|[26][S]\rarr	|s	|z	|e	|r	|y	|n	|g	|*	|s	|*	|[27][S]\darr	|*	|r	|a	|.
|l	|u	|i	|b	|u	|*	|*	|*	|[28][S]\drarr	|m	|s	|z	|y	|c	|a	|*	|n	|*	|b	|[29][S]\darr	|e	|k	|.
|e	|d	|e	|i	|m	|[30][S]\darr	|*	|[31][S]\darr	|k	|[32][S]\rarr	|s	|y	|n	|t	|e	|z	|a	|*	|u	|s	|t	|i	|.
|j	|n	|*	|e	|*	|z	|*	|n	|a	|*	|*	|s	|*	|[33][S]\drarr	|w	|e	|r	|y	|s	|t	|a	|*	|.
|*	|i	|[34][S]\darr	|g	|[35][S]\drarr	|d	|i	|a	|l	|e	|k	|t	|y	|k	|a	|*	|d	|*	|z	|o	|b	|*	|.
|*	|o	|p	|[][,]{ }	|u	|r	|[36][S]\darr	|b	|i	|*	|*	|y	|*	|o	|*	|*	|*	|*	|ó	|l	|u	|*	|.
|[37][S]\drarr	|w	|r	|o	|t	|o	|w	|i	|s	|k	|o	|*	|[38][S]\rarr	|t	|y	|k	|a	|*	|w	|i	|l	|*	|.
|m	|o	|z	|k	|w	|j	|a	|e	|z	|*	|*	|*	|[39][S]\rarr	|w	|o	|l	|e	|*	|k	|k	|u	|*	|.
|e	|a	|y	|o	|a	|o	|r	|g	|*	|*	|*	|[40][S]\rarr	|v	|i	|g	|n	|o	|l	|a	|*	|m	|*	|.
|t	|f	|w	|l	|r	|w	|z	|u	|*	|[41][S]\rarr	|s	|a	|l	|c	|h	|o	|w	|*	|[][,]{ }	|*	|[][,]{ }	|*	|.
|a	|r	|r	|i	|d	|i	|ę	|n	|[42][S]\rarr	|k	|r	|y	|s	|z	|t	|a	|ł	|e	|k	|*	|o	|*	|.
|l	|y	|y	|c	|z	|s	|c	|n	|*	|*	|*	|*	|*	|n	|[43][S]\rarr	|h	|e	|n	|r	|*	|ł	|*	|.
|i	|k	|*	|z	|a	|k	|h	|i	|*	|[44][S]\rarr	|k	|a	|p	|i	|t	|a	|ł	|*	|e	|*	|t	|*	|.
|c	|a	|*	|n	|c	|o	|a	|k	|*	|[45][S]\rarr	|b	|i	|s	|k	|w	|i	|t	|*	|s	|*	|a	|*	|.
|z	|n	|*	|o	|z	|*	|*	|*	|*	|[46][S]\rarr	|c	|h	|r	|o	|b	|o	|t	|e	|k	|*	|r	|*	|.
|n	|k	|*	|ś	|*	|*	|[47][S]\rarr	|c	|i	|e	|ś	|l	|e	|w	|s	|k	|i	|*	|o	|*	|z	|*	|.
|o	|a	|*	|c	|*	|*	|*	|[48][S]\rarr	|m	|a	|ś	|l	|a	|c	|z	|e	|k	|*	|w	|*	|o	|*	|.
|ś	|*	|*	|i	|*	|*	|*	|*	|*	|[49][S]\rarr	|e	|s	|z	|e	|w	|e	|r	|i	|a	|*	|w	|*	|.
|ć	|*	|*	|*	|*	|*	|*	|[50][S]\rarr	|k	|u	|s	|a	|k	|*	|*	|*	|*	|*	|n	|*	|e	|*	|.
|*	|*	|*	|*	|*	|*	|*	|*	|*	|*	|*	|*	|[51][S]\rarr	|f	|r	|e	|y	|t	|a	|g	|*	|*	|.
|*	|*	|[52][S]\rarr	|m	|e	|t	|o	|d	|a	|[][,]{ }	|s	|t	|r	|z	|a	|ł	|ó	|w	|*	|*	|*	|*	|.\end{Puzzle}

\newpage

\begin{PuzzleClues}{\textbf{Poziome}\\}\Clue{7}{}{radziecka mistrzyni olimpijska z Seulu w kolarskim sprincie na torze, w Barcelonie zdobyła pierwsze złoto dla Estonii}
\Clue{9}{}{ur. w 1924r architekt - Torkat? - Pawilon Ślubów w Chorzowie}
\Clue{13}{}{wysokie buty wojłokowe noszone głównie w Rosji}
\Clue{17}{}{urządzenie, które coś pochłania, najczęściej zapachy, opary}
\Clue{18}{}{miasto w Indiach (Karnataka) nad Morzem Arabskim, główny port morski stanu}
\Clue{20}{}{naczynie do obmywania rąk; obecnie używane w kościołach, złożone z dzbanka i podstawki na ściekającą wodę}
\Clue{22}{}{legowisko sumów}
\Clue{24}{}{Annę (1816-55), pisarka angielska, powieści obyczajowe; „Agnes Gray”}
\Clue{26}{}{skrzypek polski (1918-1988); po II wojnie osiadły w Meksyku; wirtuoz światowej sławy}
\Clue{28}{}{owad z rzędu piersiodziobych, nadrzędu pluskwiaków}
\Clue{32}{}{rezultat połączenia w jedną całość wielu elementów: obiektów, zjawisk itp}
\Clue{33}{}{przedstawiciel lub zwolennik weryzmu w sztuce}
\Clue{35}{}{umiejętność dowodzenia, argumentacji}
\Clue{37}{}{obiekt sportowy przeznaczony do treningów jazdy na wrotkach, zwłaszcza wrotkarstwa figurowego}
\Clue{38}{}{zgrubiale: tyczka - długi, cienki pręt, palik, słupek}
\Clue{39}{}{powiększenie tarczycy}
\Clue{40}{}{architekt włoski (1507-73) - pałac w Capraroli-Barozzi}
\Clue{41}{}{szwedzki łyżwiarz figurowy, pierwszy mistrz olimpijski z 1908 r}
\Clue{42}{}{element jubilerski lub pasmanteryjny imitujący kamień szlachetny}
\Clue{43}{}{jednostka indukcyjności oraz permeancji (przewodności magnetycznej) w układzie SI}
\Clue{44}{}{środowisko osób (np. bankierów, finansistów), które posiadają znaczną sumę pieniędzy i są zaangażowane w działalność gospodarczą, ekonomiczną}
\Clue{45}{}{wstępnie spieczona masa ceramiczna}
\Clue{46}{}{pałeczkowaty lub krzaczasty porost naziemny, niektóre gatunki paszą dla reniferów}
\Clue{47}{}{Józef (1872-1947) malarz i architekt, współzałożyciel 'Łady'; malarstwo dekoracyjne i sztalugowe}
\Clue{48}{}{Chalciporus Bataille - rodzaj grzybów należący do rodziny maślakowatych (Suillaceae); niewielkich rozmiarów grzyby mikoryzowe, naziemne, występujące wyłącznie pod drzewami iglastymi}
\Clue{49}{}{kamienna róża - amerykańskie ozdobne drzewo lub półkrzew z rodziny gruboszowatych o mięsistych liściach}
\Clue{50}{}{ptak naziemny wyglądem zbliżony do kuraków; Ameryka Środk. i Płd.; łowny}
\Clue{51}{}{(1816-95), pisarz niemiecki, komedie, powieści, szkice literackie}
\Clue{52}{}{metoda rozwiązywania zagadnienia brzegowego przez zastąpienie go zagadnieniem początkowym}\end{PuzzleClues}

\begin{PuzzleClues}{\textbf{Pionowe}\\}\Clue{1}{}{duże, przestronne pomieszczenie}
\Clue{2}{}{pomieszczenie poprzedzające świątynię hinduistyczną}
\Clue{3}{}{ur. w 1944 r., popularyzator muzyki od 1991 r. dyrektor naczelny Teatru Wielkiego w Warszawie}
\Clue{4}{}{seria amerykański dwuosobowych statków kosmicznych}
\Clue{5}{}{zdrobniale: brzuch - dolna część tułowia pomiędzy klatką piersiową a miednicą}
\Clue{6}{}{gniazdo wtyczkowe; złącze stanowiące część instalacji elektrycznej}
\Clue{8}{}{Hylobates moloch - ssak naczelny z rodziny gibonowatych; jest endemitem wyspy Jawa w Archipelagu Malajskim, zamieszkującym lasy deszczowe}
\Clue{9}{}{miasto rejonowe w obwodzie iwanofrankiwskim Ukrainy, nad Dniestrem}
\Clue{10}{}{kobieta pochodzenia mieszanego - europejsko-tubylczego zwykle wywodząca się z terenów dawnych kolonii europejskich państw}
\Clue{11}{}{CAMELOPARDALIS; gwiazdozbiór nieba północnego}
\Clue{12}{}{ur. w 1914 r., kompozytor i dyrygent m.in. orkiestry Polskiego Radia i Telewizji: utwory orkiestrowe, kameralne, muzyka rozrywkowa}
\Clue{14}{}{Phragmites - rodzaj rośliny należącej do rodziny wiechlinowatych}
\Clue{15}{}{zaloty, umizgiwanie się do kogoś}
\Clue{16}{}{fizjologiczna potrzeba organizmu wywołująca zachowania ukierunkowane na pobieranie pokarmu}
\Clue{18}{}{w fotografii jedna z technik graficznych opartych na wykorzystaniu światłoczułości; opiera się na zjawisku „garbowania” (utraty rozpuszczalności i zdolności pęcznienia pod wpływem wody) żelatyny uczulonej chromianem potasu w obecności naświetlonych i wywołanych soli srebra tworzących obraz w klasycznej, czarno-białej technice fotograficznej}
\Clue{19}{}{mieszkanka Republiki Południowej Afryki, kobieta pochodzenia południowoafrykańskiego}
\Clue{20}{}{dar dziękczynny lub błagalny; wotum}
\Clue{21}{}{włoski rzeźbiarz, malarz, rysownik i architekt (1598-1680) rzeźby mitologiczne, religijne, portrety, nagrobki, fontanny}
\Clue{23}{}{przypadek, nieoczekiwany, zaskakujący splot faktów lub wydarzeń}
\Clue{25}{}{dekoracja ołtarza w kościele}
\Clue{27}{}{Acanthiza lineata - gatunek ptaka z rodziny buszówkowatych (Acanthizidae) występujący w Australii, Nowej Gwinei i Tasmanii}
\Clue{28}{}{miasto na prawach powiatu w środkowo-zachodniej Polsce, położone na Wysoczyźnie Kaliskiej, nad Prosną, u ujścia Swędrni; historyczna stolica Wielkopolski, stolica Kaliskiego, drugi co do wielkości ośrodek województwa wielkopolskiego, siedziba powiatu kaliskiego, główny ośrodek aglomeracji kalisko-ostrowskiej i Kalisko-Ostrowskiego Okręgu Przemysłowego; siedziba kurii diecezji kaliskiej}
\Clue{29}{}{chrząszcz z rodziny ryjkowców, szkodnik lasów iglastych, larwy drążą chodniki pod korą drzew}
\Clue{30}{}{miejscowość uzdrowiskowa, kurort}
\Clue{31}{}{skrajna część magnetowodu (np. rdzenia elektromagnesu), wyznaczająca dzięki odpowiedniemu uformowaniu kierunek i rozkład linii sił pola magnetycznego}
\Clue{33}{}{Apodida - rząd szkarłupni z gromady strzykw; kotwicznikowce mają robakowate ciało o długości do 2 metrów, są pozbawione nóżek ambulakralnych; zwierzęta te poruszają się, pełzając, za pomocą drobnych kotwiczek szkieletu wystających na powierzchni ich ciała; żyją w pełnosłonych morzach, głównie strefy tropikalnej}
\Clue{34}{}{gromada płazińców; pasożyty wewnętrzne}
\Clue{35}{}{substancja zwiększająca twardość materiału, którego jest składnikiem}
\Clue{36}{}{warzęcha zwyczajna, Platalea leucorodia - gatunek ptaka brodzącego z rodziny ibisowatych (Threskiornithidae)}
\Clue{37}{}{cecha fizyczna czegoś, co jest w całości lub w części zrobione  z metalu, również tego, co wygląda jak zrobione z metalu}\end{PuzzleClues}\newpage\section*{Krzyżówka 42}

\noindent\begin{Puzzle}{20}{25}|*	|*	|*	|*	|*	|[1][S]\drarr	|s	|k	|r	|a	|*	|[2][S]\darr	|[3][S]\darr	|[4][S]\drarr	|i	|b	|s	|e	|n	|*	|*	|.
|[5][S]\drarr	|l	|a	|t	|e	|r	|a	|n	|*	|*	|*	|p	|k	|p	|*	|[6][S]\darr	|[7][S]\darr	|[8][S]\darr	|*	|*	|*	|.
|c	|[9][S]\drarr	|u	|z	|i	|o	|m	|*	|*	|*	|*	|r	|u	|r	|*	|m	|b	|s	|*	|*	|*	|.
|i	|c	|*	|*	|*	|ż	|*	|*	|*	|*	|*	|o	|r	|z	|[10][S]\darr	|c	|i	|i	|*	|*	|*	|.
|ą	|o	|*	|*	|*	|e	|*	|[11][S]\darr	|*	|*	|[12][S]\darr	|m	|t	|ę	|w	|f	|c	|l	|*	|[13][S]\darr	|[14][S]\darr	|.
|g	|k	|*	|*	|[15][S]\rarr	|k	|o	|b	|[][,]{ }	|ś	|n	|i	|a	|d	|y	|*	|i	|n	|*	|k	|s	|.
|*	|*	|*	|*	|*	|*	|*	|o	|*	|*	|o	|n	|c	|z	|p	|*	|e	|i	|[16][S]\darr	|r	|z	|.
|[17][S]\rarr	|c	|w	|i	|k	|i	|e	|r	|*	|*	|r	|e	|z	|i	|r	|*	|[][,]{ }	|k	|k	|y	|t	|.
|*	|[18][S]\drarr	|s	|t	|a	|r	|t	|e	|r	|*	|a	|n	|e	|w	|a	|*	|p	|[][,]{ }	|l	|t	|u	|.
|*	|m	|[19][S]\drarr	|l	|i	|p	|i	|k	|a	|n	|*	|c	|k	|o	|w	|[20][S]\darr	|o	|s	|i	|y	|r	|.
|*	|e	|k	|[21][S]\rarr	|ł	|a	|k	|*	|*	|*	|*	|j	|*	|*	|k	|b	|k	|y	|k	|c	|m	|.
|[22][S]\drarr	|t	|r	|ó	|j	|p	|o	|l	|ó	|w	|k	|a	|*	|*	|a	|o	|ł	|n	|a	|z	|ó	|.
|p	|r	|ę	|[23][S]\rarr	|w	|e	|r	|y	|s	|t	|a	|*	|*	|*	|*	|k	|o	|c	|l	|n	|w	|.
|*	|*	|t	|*	|*	|*	|*	|*	|*	|*	|*	|*	|[24][S]\darr	|*	|*	|s	|n	|h	|n	|o	|k	|.
|[25][S]\drarr	|k	|a	|c	|z	|k	|a	|[][,]{ }	|c	|z	|u	|b	|a	|t	|a	|*	|ó	|r	|o	|ś	|a	|.
|p	|[26][S]\drarr	|r	|ó	|ż	|n	|o	|z	|a	|r	|o	|d	|n	|i	|k	|o	|w	|o	|ś	|ć	|*	|.
|o	|p	|z	|*	|*	|*	|[27][S]\drarr	|g	|r	|e	|n	|a	|d	|y	|n	|a	|*	|n	|ć	|*	|*	|.
|n	|a	|[][,]{ }	|[28][S]\rarr	|e	|g	|z	|e	|m	|p	|l	|a	|r	|z	|*	|*	|*	|i	|*	|*	|[29][S]\darr	|.
|t	|s	|w	|*	|*	|[30][S]\darr	|a	|*	|*	|*	|[31][S]\darr	|*	|u	|[32][S]\rarr	|b	|e	|r	|c	|k	|*	|l	|.
|y	|z	|i	|*	|[33][S]\darr	|k	|k	|[34][S]\darr	|*	|*	|o	|[35][S]\rarr	|t	|c	|h	|ó	|r	|z	|*	|*	|i	|.
|f	|c	|ę	|[36][S]\drarr	|k	|a	|r	|p	|i	|ń	|s	|k	|*	|[37][S]\rarr	|p	|i	|o	|n	|*	|*	|b	|.
|i	|z	|k	|k	|u	|s	|z	|a	|*	|[38][S]\rarr	|n	|i	|e	|l	|o	|t	|*	|y	|*	|*	|r	|.
|k	|ę	|s	|a	|p	|a	|ó	|w	|*	|[39][S]\rarr	|o	|g	|i	|ń	|s	|k	|i	|*	|*	|*	|a	|.
|a	|k	|z	|n	|a	|k	|w	|ę	|*	|[40][S]\rarr	|w	|y	|m	|i	|e	|n	|i	|a	|c	|z	|*	|.
|t	|a	|y	|t	|*	|*	|*	|ż	|*	|[41][S]\rarr	|a	|u	|t	|o	|s	|y	|f	|o	|n	|*	|*	|.
|*	|*	|*	|*	|*	|*	|*	|*	|*	|*	|*	|*	|*	|*	|*	|*	|*	|*	|*	|*	|*	|.\end{Puzzle}

\newpage

\begin{PuzzleClues}{\textbf{Poziome}\\}\Clue{1}{}{delikatny błysk, blask czegoś migotliwego}
\Clue{4}{}{dramatopisarz norweski (1828-1906), dramaty społeczno-psychologiczne, historyczne; „Nora”, „Wróg ludu”, „Dzika kaczka”, „Hedda Gabler”, „Peer Gunt”}
\Clue{5}{}{zespół pałacowo-kościelny w Rzymie, dawna rezydencja papieża}
\Clue{9}{}{metalowa elektroda lub zespół elektrod umieszczona w wilgotnej warstwie gruntu, zapewniający połączenie przedmiotów uziemianych i gruntu (ziemi) z możliwie małą rezystancją}
\Clue{15}{}{Kobus ellipsiprymnus - gatunek ssaka z rodziny krętorogich, występujący w Afryce zachodniej, centralnej i południowej na terenach podmokłych, sawannie i innych terenach trawiastych}
\Clue{17}{}{odmiana okularów, możliwych do utrzymania na nosie dzięki specjalnie wygiętemu sprężynującemu elementowi łączącemu oba szkła}
\Clue{18}{}{zestaw startowy, służący użytkownikowi produktu lub usługi na początku korzystania z niego}
\Clue{19}{}{szlachetny koń białej maści}
\Clue{21}{}{słaby, żenujący, nieautentyczny raper, zwykle odbierany jako pozer}
\Clue{22}{}{sposób uprawy roli polegający na podzieleniu pola na trzy części i uprawianiu kolejno tylko dwóch z nich}
\Clue{23}{}{przedstawiciel lub zwolennik weryzmu w sztuce}
\Clue{25}{}{Lophonetta specularioides - gatunek średniego ptaka wodnego z rodziny kaczkowatych (Anatidae), zasiedlającego Amerykę Południową; to jedyny członek monotypowego rodzaju Lophonetta, bywa włączany do rodzaju Anas}
\Clue{26}{}{wytwarzanie przez sporofit części roślin dwóch rodzajów zarodników}
\Clue{27}{}{dawniej: napój z owoców granatu}
\Clue{28}{}{okaz, organizm, osobnik, jednostka}
\Clue{32}{}{miejscowość we Francji nad Kanałem La Manche; znane uzdrowisko oraz kąpielisko morskie}
\Clue{35}{}{wspólna nazwa określająca kilka gatunków drapieżnych ssaków z rodziny łasicowatych}
\Clue{36}{}{miasto w azjatyckiej części Federacji Rosyjskiej nad Turią; wydobycie węgla brunatnego}
\Clue{37}{}{dział, sekcja, jednostka organizacyjna organizacji lub instytucji}
\Clue{38}{}{KIWI}
\Clue{39}{}{kompozytor, podskarbi wielki litewski (1765-1833); romanse, mazurki, polonezy; 'Pożegnanie ojczyzny'}
\Clue{40}{}{ktoś, kto wymienia}
\Clue{41}{}{rodzaj syfonu, metalowa butla do samodzielnego sporządzania wody sodowej}\end{PuzzleClues}

\begin{PuzzleClues}{\textbf{Pionowe}\\}\Clue{1}{}{zdrobniale: róg - miejsce, gdzie stykają się linie wyznaczające brzeg czegoś i powstaje kąt}
\Clue{2}{}{osoba zajmująca wpływowe stanowisko}
\Clue{3}{}{leśny ptak o kolorowym upierzeniu i krótkim ogonie; zamieszkuje Australię i Afrykę}
\Clue{4}{}{materiał roślinny lub zwierzęcy, z którego wyrabia się przędzę}
\Clue{5}{}{nieprzerwana ciągłość, tok czegoś w czasie, trwanie, dzianie się, bieg, przebieg}
\Clue{6}{}{kod ISO 4217 franka monakijskiego}
\Clue{7}{}{oddawanie czci, kłanianie się}
\Clue{8}{}{silnik prądu przemiennego}
\Clue{9}{}{kod ISO 4217 dolara Wysp Cooka}
\Clue{10}{}{komplet rzeczy dla ucznia danej klasy, które są potrzebne w szkole}
\Clue{11}{}{związek boru z metalem}
\Clue{12}{}{podziemna kryjówka zwierzęca, jama mająca zaspokoić elementarną potrzebę schronienia, będącą biologicznie ukształtowanym warunkiem rozwoju organizmów}
\Clue{13}{}{podważanie wyników badań lub twierdzeń uważanych wcześniej za oczywiste i pewne}
\Clue{14}{}{pałka policyjna}
\Clue{16}{}{liczba wejść na daną stroną internetową rozumiana jako liczba kliknięć w link, który do niej prowadzi}
\Clue{18}{}{w literaturoznawstwie - metrum: oznaczenie rymu regularnego, bądź też wzorcowej miary rytmicznej, która każdorazowo aktualizuje się w materiale fonicznym konkretnego utworu}
\Clue{19}{}{wyniosłość kostna w sąsiedztwie kości udowej, do której przyczepione są mięśnie pośladkowe}
\Clue{20}{}{wyprawiona skóra bydlęca stosowana do wyrobu cholewek obuwia}
\Clue{22}{}{puaz - jednostka lepkości dynamicznej w układzie jednostek miar CGS, nazwana na cześć francuskiego fizyka i lekarza Jeana L. M. Poiseuille'a. 1 P = 1 dyn·s/cm2 = 1 g·/(cm·s)}
\Clue{24}{}{łamliwe ciasto jako foremka do lodów, kremów}
\Clue{25}{}{okres rządów papieża}
\Clue{26}{}{otwór gębowy u zwierząt; określenie stosowane zwłaszcza w przypadku zwierząt, które mają zęby}
\Clue{27}{}{wrocławskie osiedle na terenach byłej dzielnicy Psie Pole, na północno-wschodnim krańcu Wrocławia}
\Clue{29}{}{miara ilości papieru, równa 24 lub 25 arkuszom, później 100 arkuszom}
\Clue{30}{}{jedwabna kurtka noszona przez dżokejów}
\Clue{31}{}{podstawowy szkielet opony dźwigający obciążenie}
\Clue{33}{}{odchody stałe, kał}
\Clue{34}{}{prostokątna, duża tarcza drewniana pokryta skórą lub blachą, używana przez piechotę w XIV/XVI w}
\Clue{36}{}{niemiecki filozof oświeceniowy, profesor logiki i metafizyki na Uniwersytecie Królewieckim}\end{PuzzleClues}\newpage\section*{Krzyżówka 43}

\noindent\begin{Puzzle}{21}{33}|*	|[1][S]\darr	|*	|*	|*	|*	|*	|*	|*	|*	|*	|*	|*	|*	|*	|*	|*	|*	|*	|*	|*	|*	|.
|*	|p	|*	|*	|*	|*	|[2][S]\drarr	|l	|a	|s	|k	|a	|[][,]{ }	|j	|a	|k	|u	|b	|a	|*	|*	|*	|.
|*	|i	|*	|[3][S]\rarr	|e	|n	|s	|o	|r	|*	|[4][S]\drarr	|s	|t	|e	|r	|e	|o	|g	|r	|a	|m	|*	|.
|*	|c	|*	|*	|[5][S]\drarr	|h	|a	|s	|e	|ł	|k	|o	|*	|*	|*	|*	|*	|*	|*	|*	|*	|*	|.
|*	|a	|*	|*	|m	|*	|m	|[6][S]\rarr	|m	|e	|l	|a	|s	|a	|*	|*	|[7][S]\darr	|[8][S]\darr	|*	|*	|*	|*	|.
|[9][S]\drarr	|s	|o	|t	|e	|l	|o	|*	|*	|[10][S]\rarr	|a	|l	|i	|m	|e	|n	|t	|a	|c	|j	|a	|*	|.
|s	|s	|[11][S]\rarr	|s	|t	|a	|c	|j	|a	|*	|u	|[12][S]\darr	|*	|*	|*	|*	|l	|l	|*	|*	|*	|*	|.
|p	|o	|*	|*	|o	|[13][S]\darr	|h	|[14][S]\rarr	|b	|e	|z	|s	|e	|n	|s	|*	|e	|u	|[15][S]\darr	|[16][S]\darr	|[17][S]\darr	|*	|.
|r	|*	|*	|*	|d	|n	|o	|*	|[18][S]\darr	|*	|u	|z	|*	|*	|*	|[19][S]\darr	|n	|m	|i	|c	|u	|*	|.
|z	|*	|*	|[20][S]\darr	|a	|o	|d	|*	|s	|[21][S]\darr	|l	|m	|*	|[22][S]\darr	|*	|c	|e	|i	|b	|h	|k	|*	|.
|e	|*	|[23][S]\darr	|y	|[][,]{ }	|ś	|z	|*	|a	|m	|a	|a	|*	|m	|*	|y	|k	|n	|i	|o	|ł	|*	|.
|n	|[24][S]\drarr	|b	|o	|t	|n	|i	|c	|k	|a	|*	|j	|*	|r	|*	|n	|[][,]{ }	|o	|s	|i	|a	|*	|.
|i	|a	|e	|r	|e	|i	|k	|*	|*	|c	|*	|s	|[25][S]\darr	|ó	|*	|a	|a	|g	|[][,]{ }	|n	|d	|*	|.
|e	|r	|z	|k	|r	|k	|*	|*	|*	|i	|*	|e	|s	|w	|*	|m	|z	|r	|c	|a	|[][,]{ }	|*	|.
|w	|m	|w	|s	|m	|*	|*	|*	|[26][S]\darr	|c	|*	|r	|z	|k	|*	|o	|o	|a	|z	|[][,]{ }	|f	|*	|.
|i	|i	|o	|h	|i	|[27][S]\drarr	|t	|e	|k	|a	|*	|*	|t	|a	|*	|ń	|t	|f	|u	|r	|i	|*	|.
|e	|a	|d	|i	|c	|u	|*	|*	|l	|[][,]{ }	|*	|*	|a	|[][,]{ }	|*	|c	|u	|i	|b	|ó	|z	|*	|.
|r	|*	|n	|r	|z	|k	|[28][S]\darr	|*	|e	|d	|*	|*	|j	|ł	|*	|z	|*	|a	|a	|ż	|y	|*	|.
|z	|[29][S]\darr	|i	|e	|n	|ł	|z	|*	|j	|w	|*	|*	|m	|ą	|*	|y	|[30][S]\darr	|*	|t	|n	|c	|*	|.
|e	|n	|k	|*	|a	|a	|o	|*	|ó	|u	|[31][S]\darr	|*	|e	|k	|*	|k	|t	|*	|y	|o	|z	|*	|.
|n	|i	|[][,]{ }	|*	|*	|d	|o	|[32][S]\drarr	|w	|r	|z	|o	|s	|o	|w	|i	|e	|c	|*	|i	|n	|*	|.
|i	|e	|k	|*	|*	|[][,]{ }	|f	|m	|k	|o	|m	|*	|*	|w	|*	|*	|o	|*	|*	|g	|y	|*	|.
|e	|i	|w	|*	|*	|c	|e	|i	|a	|ż	|i	|[33][S]\drarr	|z	|a	|s	|t	|r	|z	|a	|ł	|*	|*	|.
|*	|n	|a	|[34][S]\darr	|*	|a	|n	|k	|*	|n	|a	|c	|*	|*	|*	|*	|i	|*	|*	|o	|*	|*	|.
|[35][S]\drarr	|w	|s	|p	|ó	|ł	|o	|r	|g	|a	|n	|i	|z	|a	|c	|j	|a	|*	|*	|w	|*	|*	|.
|k	|a	|o	|o	|*	|k	|l	|o	|*	|*	|a	|s	|*	|*	|*	|*	|[][,]{ }	|*	|*	|a	|*	|*	|.
|a	|z	|w	|d	|*	|u	|o	|m	|*	|*	|*	|*	|*	|[36][S]\rarr	|f	|a	|n	|k	|a	|*	|*	|*	|.
|s	|y	|y	|c	|*	|j	|g	|i	|*	|*	|*	|*	|*	|*	|[37][S]\rarr	|r	|a	|b	|i	|e	|c	|*	|.
|z	|j	|*	|h	|*	|ą	|i	|e	|[38][S]\rarr	|b	|ó	|l	|[][,]{ }	|b	|r	|z	|u	|c	|h	|a	|*	|*	|.
|a	|n	|*	|w	|*	|c	|a	|r	|[39][S]\rarr	|o	|ś	|m	|i	|o	|r	|a	|k	|*	|*	|*	|*	|*	|.
|l	|o	|*	|y	|*	|y	|*	|z	|*	|*	|*	|[40][S]\rarr	|b	|r	|e	|t	|o	|ń	|s	|k	|i	|*	|.
|o	|ś	|*	|t	|*	|*	|*	|*	|*	|*	|*	|*	|[41][S]\rarr	|o	|k	|o	|w	|i	|t	|a	|*	|*	|.
|t	|ć	|*	|*	|[42][S]\rarr	|c	|y	|p	|r	|y	|ś	|n	|i	|k	|o	|w	|a	|t	|e	|*	|*	|*	|.
|*	|*	|*	|*	|*	|*	|*	|*	|*	|[43][S]\rarr	|j	|a	|g	|n	|i	|ę	|*	|*	|*	|*	|*	|*	|.\end{Puzzle}

\newpage

\begin{PuzzleClues}{\textbf{Poziome}\\}\Clue{2}{}{przyrząd nawigacyjny w postaci prostej listwy z ruchomą poprzeczką służący do mierzenia wysokości ciał niebieskich nad horyzontem, a także kątów poziomych i pionowych pomiędzy obiektami widocznymi na Ziemi}
\Clue{3}{}{niemiecki malarz, grafik i teoretyk sztuki (1471-1528) drzeworyty, miedzioryty, portrety 'Apokalipsa'-'Czterej apostołowie'}
\Clue{4}{}{obraz przedstawiony na płaszczyźnie w taki sposób, aby sprawiał wrażenie trójwymiarowego}
\Clue{5}{}{hasło-wyraz}
\Clue{6}{}{ciemnobrązowy, gęsty syrop o odczynie słabo alkalicznym, powstaje jako produkt uboczny podczas produkcji cukru spożywczego}
\Clue{9}{}{meksykanka, pierwsza kobieta, która zapaliła znicz olimpijski podczas I.O. w Meksyku}
\Clue{10}{}{w hydrologii: zasilanie, uzupełnianie zasobów wodnych}
\Clue{11}{}{ustalone i oznaczone miejsce zatrzymywania się pociągów, gdzie odbywa się wsiadanie i wysiadani pasażerów lub rozładunek i załadunek towarów}
\Clue{14}{}{brak sensu, to, że coś nie ma sensu, jest nielogiczne, nieprawdziwe, absurdalne; bezsensowność}
\Clue{24}{}{zatoka Morza Bałtyckiego, między Szwecją a Finlandią, pow. 111 tyś. km2}
\Clue{27}{}{teczka dużych rozmiarów, zazwyczaj wykorzystywana do przechowywania i transportu prac plastycznych}
\Clue{32}{}{wrzosiec bagienny, Erica tetralix - gatunek rośliny z rodziny wrzosowatych}
\Clue{33}{}{w lotnictwie: element konstrukcyjny statków powietrznych, głównie samolotów - profilowany pręt łączący płat, statecznik poziomy lub podwozie z kadłubem, stosowany zwłaszcza w starszych konstrukcjach dwupłatów i górnopłatów zastrzałowych}
\Clue{35}{}{wspólne organizowanie czegoś}
\Clue{36}{}{miłośniczka kogoś/czegoś}
\Clue{37}{}{młody ptak łowczy, jeszcze pstrokaty}
\Clue{38}{}{skurcze przepony, wywołane atakiem śmiechu, powodujące odczucie bólu na wysokości żołądka}
\Clue{39}{}{wielorodzinny budynek o ośmiu mieszkaniach, zwykle przeznaczony dla służby folwarcznej}
\Clue{40}{}{język z grupy brytańskiej (p-celtyckiej) języków celtyckich, którym posługują się Bretończycy w dolnej Bretanii; język uznany przez państwo francuskie za tzw. język regionalny}
\Clue{41}{}{mocny alkohol spirytusowy (wódka bądź nalewka) wyrabiany tradycyjnymi metodami; tradycja wyrobu okwoity wywodzi się z okresu staropolskiego}
\Clue{42}{}{Taxodiaceae - rodzina z rzędu cyprysowców}
\Clue{43}{}{młody muflon w gwarze łowieckiej}\end{PuzzleClues}

\begin{PuzzleClues}{\textbf{Pionowe}\\}\Clue{1}{}{Pablo Ruiz Picasso - hiszpański malarz, rzeźbiarz, grafik i ceramik, uznawany za jednego z najwybitniejszych artystów XX wieku}
\Clue{2}{}{mały samochód - napędzany silnikiem pojazd mechaniczny przeznaczony do przewożenia po drogach ludzi i różnego rodzaju ładunków}
\Clue{4}{}{kompozycja wielogłosowa w okresie ars antiqua, w paryskiej szkole Notre Dame}
\Clue{5}{}{metoda określania płodności polegająca na dokonywaniu codziennych pomiarów temperatury ciała kobiety (pomiar podstawowej temperatury ciała - PTC) i zapisywaniu otrzymywanych wyników - powstaje w ten sposób wykres, który charakteryzuje się tym, że w pierwszych dniach po miesiączce temperatura ciała jest niska, następnie następuje gwałtowny wzrost temperatury i faza utrzymywania się temperatury (przynajmniej przez trzy dni) na podwyższonym poziomie, gdzie 6 dni wstecz i 3 dni po dniu wzrostu temperatury przyjmuje się jako dni płodne, pozostałe to dni niepłodne kobiety}
\Clue{7}{}{związek nieorganiczny, w którym azot występuje na II stopniu utlenienia}
\Clue{8}{}{obraz uzyskany przy użyciu techniki aluminografii}
\Clue{9}{}{postępowanie wbrew ideałom, pożytkowi państwa, narodu itp}
\Clue{12}{}{niemiecki pistolet maszynowy używany w czasie II wojny światowej}
\Clue{13}{}{maszyna robocza do transportu}
\Clue{15}{}{Nipponia nippon - gatunek ptaka z rodziny ibisowatych (Threskiornithidae)}
\Clue{16}{}{Tsuga diversifolia - gatunek z rodziny sosnowatych}
\Clue{17}{}{układ  w postaci obiektu fizycznego lub zbioru takich obiektów}
\Clue{18}{}{SIDŁO; potrzask na ptaki}
\Clue{19}{}{Hypocryptadiinae - rodzaj ptaków z rodziny wróblowatych (Passeridae)}
\Clue{20}{}{hrabstwo historyczne i kraina w północno-wschodniej Anglii}
\Clue{21}{}{uterus bicornis - jedna z malformacji macicy}
\Clue{22}{}{mrówka czarniawka, Formica pratensis - gatunek mrówki z podrodziny Formicinae; zamieszkuje głównie suche środowiska Europy i Azji, takie jak stepy, łąki i pastwiska}
\Clue{23}{}{związek chemiczny, który w wyniku reakcji z wodą tworzy kwas}
\Clue{24}{}{WOJSKO; ogół sił zbrojnych państwa}
\Clue{25}{}{człowiek, żyjący na marginesie społeczeństwa, menel}
\Clue{26}{}{Gomphidius - rodzaj grzybów z rodziny klejówkowatych; ich charakterystyczną cechą jest występowanie osłony częściowej w postaci śluzowatej błonki}
\Clue{27}{}{urządzenie do całkowania numerycznego lub graficznego}
\Clue{28}{}{fenologia zwierząt}
\Clue{29}{}{np. badania, zabiegu}
\Clue{30}{}{teoria złożona z logicznych twierdzeń, zrodzona na gruncie naukowym i stosowana w nauce}
\Clue{31}{}{pracownicy, część załogi, która pracuje razem w systemie zmianowym}
\Clue{32}{}{przyrząd pomiarowy służący do mierzenia wymiarów geometrycznych przedmiotów z materiałów twardych z rozdzielczością rzędu 0,01 mm lub 0,001 mm}
\Clue{33}{}{drewno pozyskiwane z drzew o tej samej nazwie}
\Clue{34}{}{w gimnastyce na przyrządach: chwyt, przy którym ręce zwrócone są dłońmi do ćwiczącego, a kciukami na zewnątrz}
\Clue{35}{}{POTWAL; gatunek wieloryba, dostarcza ambry i olbrotu}\end{PuzzleClues}\newpage\section*{Krzyżówka 44}

\noindent\begin{Puzzle}{22}{29}|*	|[1][S]\drarr	|b	|o	|s	|c	|h	|*	|[2][S]\drarr	|f	|e	|r	|r	|o	|m	|a	|g	|n	|e	|t	|y	|k	|*	|.
|*	|p	|[3][S]\rarr	|n	|i	|e	|d	|y	|p	|l	|o	|m	|a	|t	|y	|c	|z	|n	|o	|ś	|ć	|*	|*	|.
|*	|o	|*	|*	|*	|[4][S]\drarr	|s	|z	|a	|f	|e	|r	|*	|*	|*	|*	|*	|*	|[5][S]\darr	|[6][S]\darr	|[7][S]\darr	|[8][S]\darr	|*	|.
|[9][S]\rarr	|m	|i	|k	|r	|o	|s	|a	|t	|e	|l	|i	|t	|a	|*	|*	|*	|[10][S]\darr	|a	|k	|a	|c	|*	|.
|*	|a	|[11][S]\rarr	|g	|ó	|r	|a	|l	|e	|k	|[][,]{ }	|p	|r	|z	|y	|l	|ą	|d	|k	|o	|w	|y	|*	|.
|*	|r	|*	|*	|[12][S]\darr	|g	|*	|[13][S]\darr	|l	|[14][S]\drarr	|a	|s	|c	|e	|z	|a	|*	|a	|c	|l	|a	|k	|*	|.
|*	|a	|*	|[15][S]\darr	|b	|a	|*	|p	|n	|f	|[16][S]\darr	|*	|*	|*	|*	|*	|*	|m	|j	|e	|n	|l	|*	|.
|*	|ń	|*	|k	|r	|n	|*	|a	|i	|i	|d	|*	|*	|*	|*	|[17][S]\darr	|*	|a	|a	|j	|t	|[][,]{ }	|*	|.
|*	|c	|*	|i	|u	|o	|*	|p	|a	|l	|u	|*	|*	|[18][S]\darr	|*	|k	|*	|[][,]{ }	|[][,]{ }	|k	|u	|r	|*	|.
|*	|z	|[19][S]\drarr	|e	|k	|l	|e	|r	|*	|t	|i	|*	|*	|k	|*	|i	|*	|c	|a	|a	|r	|o	|*	|.
|*	|o	|m	|ł	|[][,]{ }	|o	|*	|o	|*	|r	|s	|[20][S]\darr	|*	|a	|*	|e	|*	|z	|p	|[][,]{ }	|n	|z	|*	|.
|*	|w	|ę	|b	|m	|g	|*	|t	|*	|[][,]{ }	|b	|c	|*	|n	|*	|r	|*	|e	|o	|d	|y	|r	|*	|.
|*	|y	|k	|a	|o	|i	|*	|k	|*	|f	|u	|i	|*	|i	|*	|u	|*	|r	|r	|r	|[][,]{ }	|o	|*	|.
|*	|[][,]{ }	|a	|*	|r	|a	|[21][S]\drarr	|a	|g	|o	|r	|a	|*	|u	|[22][S]\darr	|n	|*	|w	|t	|e	|n	|d	|*	|.
|*	|k	|[][,]{ }	|*	|e	|*	|w	|[][,]{ }	|*	|t	|g	|s	|*	|k	|k	|k	|*	|o	|o	|w	|a	|c	|*	|.
|*	|a	|p	|*	|n	|*	|y	|z	|*	|o	|*	|t	|[23][S]\darr	|[][,]{ }	|r	|o	|*	|n	|w	|n	|s	|z	|*	|.
|*	|r	|a	|*	|o	|*	|z	|ł	|*	|g	|[24][S]\drarr	|o	|d	|z	|e	|w	|*	|o	|a	|i	|t	|y	|*	|.
|*	|z	|ń	|[25][S]\darr	|w	|[26][S]\darr	|ł	|o	|[27][S]\darr	|r	|s	|[][,]{ }	|e	|w	|o	|o	|*	|b	|*	|a	|r	|*	|*	|.
|*	|e	|s	|z	|y	|b	|a	|c	|e	|a	|u	|b	|s	|y	|d	|ś	|*	|r	|[28][S]\darr	|n	|ó	|*	|*	|.
|*	|ł	|k	|a	|*	|ą	|c	|i	|n	|f	|k	|i	|y	|c	|o	|ć	|*	|z	|p	|a	|j	|*	|*	|.
|*	|*	|a	|d	|*	|b	|a	|s	|d	|i	|n	|s	|k	|z	|n	|*	|*	|u	|e	|*	|*	|*	|*	|.
|*	|*	|*	|o	|*	|e	|n	|t	|u	|c	|o	|z	|a	|a	|t	|*	|*	|c	|r	|*	|*	|*	|*	|.
|[29][S]\drarr	|h	|o	|m	|i	|l	|i	|a	|r	|z	|*	|k	|n	|j	|*	|*	|*	|h	|u	|*	|[30][S]\darr	|*	|*	|.
|a	|[31][S]\darr	|*	|k	|*	|e	|e	|*	|o	|n	|*	|o	|t	|n	|*	|*	|*	|a	|k	|*	|n	|*	|*	|.
|t	|s	|*	|a	|*	|k	|*	|*	|*	|y	|*	|p	|*	|y	|*	|*	|*	|*	|a	|*	|a	|*	|*	|.
|*	|o	|*	|*	|*	|*	|*	|*	|*	|*	|*	|t	|*	|*	|*	|*	|*	|*	|r	|*	|b	|*	|*	|.
|[32][S]\rarr	|w	|s	|p	|ó	|ł	|w	|ł	|a	|s	|n	|o	|ś	|ć	|[][,]{ }	|ł	|ą	|c	|z	|n	|a	|*	|*	|.
|*	|i	|[33][S]\rarr	|c	|h	|l	|e	|b	|[][,]{ }	|p	|o	|w	|s	|z	|e	|d	|n	|i	|*	|*	|b	|*	|*	|.
|*	|ę	|*	|*	|[34][S]\rarr	|t	|r	|a	|g	|a	|n	|e	|k	|[][,]{ }	|d	|u	|ń	|s	|k	|i	|*	|*	|*	|.
|*	|*	|[35][S]\rarr	|p	|r	|z	|y	|s	|t	|a	|w	|*	|*	|*	|*	|*	|*	|*	|*	|*	|*	|*	|*	|.\end{Puzzle}

\newpage

\begin{PuzzleClues}{\textbf{Poziome}\\}\Clue{1}{}{popiersie; rzeźba lub płaskorzeźba przedstawiająca górną część postaci}
\Clue{2}{}{substancja wykazująca właściwości ferromagnetyczne, czyli własne, spontaniczne namagnesowanie}
\Clue{3}{}{to, że coś jest nietaktowne, coś świadczy o tym, że ktoś nie wie, jak postąpić, zachować się właściwie, grzecznie}
\Clue{4}{}{botanik, profesor UJ (1886-1970); twórca polskiej szkoły paleobotanicznej}
\Clue{9}{}{satelita o mniejszych wymiarach niż standarowe, wykonany przez człowieka, służący np. do precyzyjnych pomiarów niewielkich zmian, celów technologiczno-edukacyjnych}
\Clue{11}{}{góralek abisyński, Procavia capensis - gatunek ssaka z rodziny góralkowatych; zamieszkuje kontynent afrykański, Półwysep Arabski, Bliski Wschód, aż po Turcję}
\Clue{14}{}{dobrowolne ograniczanie potrzeb życiowych, umartwianie się w celu osiągnięcia doskonałości np. w braminizmie lub dla zbawienia duszy w chrześcijaństwie}
\Clue{19}{}{eklerka, eklerek - podłużne ciastko z parzonego ciasta, przecięte na dwie części i napełnione kremem lub bitą śmietaną, z wierzchu oblane czekoladą}
\Clue{21}{}{główny plac, rynek w miastach starożytnej Grecji, centrum wokół którego toczyło się życie polityczne, religijne, a czasami także handlowe}
\Clue{24}{}{odpowiedź na czyjeś wezwanie}
\Clue{29}{}{zbiór kazań}
\Clue{32}{}{współwłasność, w której nie wyróżnia się udziałów; jest zawsze stosunkiem prawnym wynikającym z innego stosunku prawnego (może wynikać np. z małżeństwa lub wspólności majątku w spółce cywilnej)}
\Clue{33}{}{coś zwyczajnego, codziennego}
\Clue{34}{}{Astragalus danicus - gatunek z rodziny bobowatych właściwych}
\Clue{35}{}{kiedyś nadzorca}\end{PuzzleClues}

\begin{PuzzleClues}{\textbf{Pionowe}\\}\Clue{1}{}{karzeł o typie widmowym K}
\Clue{2}{}{płaskie naczynie, najczęściej z jedną rączką, służące do smażenia i duszenia jedzenia}
\Clue{4}{}{dział zoologii zajmujący się budową i funkcjonowaniem narządów w organizmach}
\Clue{5}{}{akcja pokryta wkładem niepieniężnym}
\Clue{6}{}{zabawka dziecięca będąca minaturową kolejką zrobioną z drewna}
\Clue{7}{}{skłonność do kłótni, chwilowa agresja}
\Clue{8}{}{zmiany, które są regulowane hormonalnie i zachodzą okresowo w organizmie kobiet i samic ssaków, głównie w jajnikach i drogach rodnych}
\Clue{10}{}{Lorius domicella - gatunek ptaka z rodziny papugowatych (Psittacidae), z podrodziny papug wschodnich (Psittaculinae)}
\Clue{12}{}{nagromadzenie głazów pozostałe po wypłukaniu przez wodę drobnych składników moreny lodowcowej (residuum)}
\Clue{13}{}{flebodium złociste, Phlebodium aureum - nazwa zwyczajowa}
\Clue{14}{}{płaska nasadka na obiektyw zmieniająca wygląd fotografowanego obrazu, nakładana najczęściej na przód obiektywu}
\Clue{15}{}{określana z żartem lub niechęcią, zgrubiale, ekpresywnie kiełbasa}
\Clue{16}{}{miasto w Niemczech (Nadrenia Płn.  Westfalia) w Zagłębiu Ruhry, przy ujściu Ruhry do Renu}
\Clue{17}{}{to, że coś jest ukierunkowane na jakiś cel}
\Clue{18}{}{Elanus caeruleus caeruleus - nominatywny podgatunek ptaka wyróżniony w obrębie gatunku kaniuk zwyczajny (Elanus caeruleus); obszar występowania obejmuje Półwysep Iberyjski, niemal całą Afrykę i południowo-zachodnią Arabię}
\Clue{19}{}{ogromny kłopot, strapienie}
\Clue{20}{}{rodzaj lekkiego ciasta przyrządzanego na bazie jaj z małym dodatkiem mąki pszennej i cukru - bez dodatku tłuszczu}
\Clue{21}{}{złocenie, pozłacanie - pokrywanie warstwą złota }
\Clue{22}{}{przedstawiciel wymarłego rzędu kreodontów}
\Clue{23}{}{środek chemiczny stosowany w rolnictwie w celu wywołania obumierania i zasychania zielonych części roślin}
\Clue{24}{}{gruba i szorstka wełniana tkanina}
\Clue{25}{}{PERSAK}
\Clue{26}{}{zdrobniale: bąbel - pęcherz, rodzaj niewielkiej wypukłości na ciele, która jest wypełniona płynem i powstaje w wyniku oparzenia lub innego urazu skóry}
\Clue{27}{}{rodzaj motocykla, wykorzystywanego głownie w rajdach enduro}
\Clue{28}{}{kapucynka; ozdobna rasa gołębia domowego, obfite upierzenie na szyi i karku tworzy puszysty kołnierz zakrywający głowę}
\Clue{29}{}{jednostka zdawkowa w Laosie; 1/100 kipa}
\Clue{30}{}{bardzo bogaty człowiek}
\Clue{31}{}{pisklę sowy}\end{PuzzleClues}\newpage\section*{Krzyżówka 45}

\noindent\begin{Puzzle}{18}{27}|*	|*	|*	|*	|*	|*	|*	|*	|*	|*	|*	|*	|*	|*	|*	|[1][S]\darr	|*	|*	|*	|.
|*	|*	|*	|*	|*	|*	|*	|*	|*	|*	|*	|*	|*	|*	|*	|p	|[2][S]\darr	|*	|*	|.
|*	|*	|*	|*	|*	|*	|*	|*	|*	|*	|*	|[3][S]\drarr	|k	|i	|j	|e	|k	|*	|*	|.
|*	|*	|*	|*	|*	|*	|[4][S]\darr	|*	|[5][S]\rarr	|t	|o	|b	|a	|*	|*	|o	|o	|*	|[6][S]\darr	|.
|*	|[7][S]\rarr	|g	|a	|t	|u	|n	|e	|k	|*	|*	|ą	|*	|*	|*	|n	|p	|[8][S]\darr	|h	|.
|*	|[9][S]\darr	|[10][S]\rarr	|s	|y	|j	|a	|m	|c	|z	|y	|k	|*	|[11][S]\darr	|*	|i	|i	|k	|e	|.
|*	|s	|[12][S]\rarr	|m	|i	|e	|d	|n	|i	|c	|a	|*	|*	|u	|*	|a	|a	|u	|g	|.
|*	|z	|*	|*	|*	|[13][S]\darr	|a	|*	|[14][S]\rarr	|f	|i	|o	|r	|d	|*	|*	|*	|r	|e	|.
|*	|a	|*	|*	|*	|p	|w	|[15][S]\drarr	|s	|o	|s	|*	|[16][S]\darr	|k	|[17][S]\darr	|[18][S]\darr	|*	|t	|l	|.
|[19][S]\drarr	|t	|ę	|t	|n	|i	|c	|a	|[][,]{ }	|w	|ą	|t	|r	|o	|b	|o	|w	|a	|*	|.
|j	|r	|*	|[20][S]\drarr	|t	|r	|a	|n	|z	|y	|c	|j	|a	|*	|e	|d	|[21][S]\darr	|c	|*	|.
|ę	|a	|*	|m	|*	|a	|*	|a	|*	|*	|*	|*	|d	|*	|l	|w	|p	|z	|*	|.
|z	|*	|*	|a	|*	|n	|*	|l	|*	|*	|*	|*	|c	|*	|w	|o	|r	|e	|*	|.
|y	|*	|[22][S]\darr	|r	|*	|*	|*	|i	|*	|*	|*	|*	|a	|[23][S]\darr	|e	|d	|z	|k	|*	|.
|k	|*	|c	|m	|*	|*	|[24][S]\rarr	|z	|a	|j	|ą	|c	|*	|z	|d	|o	|e	|*	|*	|.
|[][,]{ }	|[25][S]\drarr	|z	|u	|c	|h	|w	|a	|ł	|o	|ś	|ć	|*	|o	|e	|r	|b	|[26][S]\darr	|*	|.
|u	|b	|u	|r	|*	|*	|*	|[][,]{ }	|*	|*	|*	|*	|*	|ł	|r	|n	|i	|t	|*	|.
|g	|a	|j	|*	|*	|*	|[27][S]\rarr	|p	|l	|z	|*	|*	|*	|z	|*	|i	|e	|a	|*	|.
|r	|h	|k	|*	|*	|[28][S]\rarr	|k	|o	|r	|p	|u	|s	|*	|y	|*	|e	|r	|b	|*	|.
|o	|r	|a	|*	|*	|[29][S]\rarr	|b	|r	|u	|b	|e	|c	|k	|*	|[30][S]\darr	|n	|a	|u	|*	|.
|f	|a	|*	|*	|[31][S]\rarr	|o	|s	|t	|r	|o	|w	|c	|z	|a	|n	|i	|n	|*	|*	|.
|i	|j	|*	|*	|*	|*	|[32][S]\rarr	|f	|a	|n	|o	|n	|*	|*	|a	|e	|i	|*	|*	|.
|ń	|ń	|[33][S]\rarr	|z	|ł	|o	|c	|e	|n	|i	|e	|*	|*	|*	|c	|*	|e	|*	|*	|.
|s	|c	|*	|[34][S]\rarr	|k	|a	|r	|l	|i	|s	|t	|a	|*	|*	|h	|*	|c	|*	|*	|.
|k	|z	|[35][S]\rarr	|ś	|l	|u	|b	|o	|j	|a	|w	|k	|a	|*	|ó	|*	|*	|*	|*	|.
|i	|y	|*	|*	|*	|*	|*	|w	|*	|[36][S]\rarr	|p	|a	|r	|o	|d	|i	|a	|*	|*	|.
|*	|k	|*	|*	|*	|[37][S]\rarr	|m	|a	|l	|u	|c	|z	|c	|y	|*	|*	|*	|*	|*	|.
|*	|*	|[38][S]\rarr	|a	|g	|a	|t	|*	|*	|*	|*	|*	|*	|*	|*	|*	|*	|*	|*	|.\end{Puzzle}

\newpage

\begin{PuzzleClues}{\textbf{Poziome}\\}\Clue{3}{}{zdrobniale: kij - przyrząd sportowy, wydłużony przedmiot, który ma określony kształt i który w jakimś sporcie służy do odbijania, przemieszczania piłki, podpierania się itp}
\Clue{5}{}{jezioro w Indonezji, największe na wyspie Sumatra, pow. 1940 km2, gł. do 529 m}
\Clue{7}{}{rodzaj czegoś; kategoria, do której przynależą elementy o określonych cechach}
\Clue{10}{}{obywatel Syjamu, obecnie Tajlandii}
\Clue{12}{}{zawartość miednicy, tyle, ile się mieści w miednicy}
\Clue{14}{}{FIORDYNG: bardzo silny, norweski koń zaprzęgowy}
\Clue{15}{}{zawiesisty płyn podawany jako dodatek do np. mięs}
\Clue{19}{}{jedna z dwóch tętnic (wątrobowej wspólnej oraz wątrobowej właściwej) doprowadzających utlenioną krew do wątroby}
\Clue{20}{}{zmiana płci}
\Clue{24}{}{LEP; gwiazdozbiór nieba południowego}
\Clue{25}{}{oznaka tego, że ktoś jest zuchwały}
\Clue{27}{}{kod ISO 4217 dla waluty polskiej - złotego - sprzed denominacji z 1 stycznia 1995}
\Clue{28}{}{grupa żołnieży o takim samym stopniu wojskowym (tzw. korpus osobowy)}
\Clue{29}{}{amerykański pianista i kompozytor jazzowy ur. w 1920 r., modern jazz}
\Clue{31}{}{mieszkaniec Ostrowca Świętokrzyskiego}
\Clue{32}{}{jedwabna szata liturgiczna papieża oraz - na mocy szczególnego przywileju - patriarchy Lizbony}
\Clue{33}{}{pozłacanie, wyzłacanie - pokrywanie warstwą złota}
\Clue{34}{}{przedstawiciel konserwatywnego nurtu politycznego powstałego w Hiszpanii w latach 30. XIX w., zmierzającego do odtworzenia mocarstwowej Hiszpanii poprzez reformy konserwatywne, prokatolickie, decentralizacyjne, antyabsolutystyczne, antydemokratyczne i antyliberalne w duchu legitymistycznym}
\Clue{35}{}{Hosta - rodzaj rośliny z rodziny liliowatych}
\Clue{36}{}{utwór lub wykonanie będące ośmieszającym naśladowaniem}
\Clue{37}{}{prości, zwykli ludzie, godni pożałowania, przeciętni}
\Clue{38}{}{półszlachetny minerał, wielobarwna, wstęgowa odmiana chalcedonu}\end{PuzzleClues}

\begin{PuzzleClues}{\textbf{Pionowe}\\}\Clue{1}{}{PIWONIA}
\Clue{2}{}{dany egzemplarz czegoś produkowanego seryjnie, zwykle używane w odniesieniu do nośnika tekstu kultury}
\Clue{3}{}{żyroskop o osi podpartej na jednym końcu}
\Clue{4}{}{firma bądź instytucja, która nadaje treści za pomocą fal; np. nadawca radiowy, nadawca telewizyjny}
\Clue{6}{}{twórczość Hegla, zbiór jego myśli i poglądów}
\Clue{8}{}{ptak z rzędu wróblowatych; Archipelag Malajski, Australia, Afryka}
\Clue{9}{}{płócienny namiot}
\Clue{11}{}{mięso z nóżki drobiowej z kostką}
\Clue{13}{}{miasto w Słowienii nad Morzem Adriatyckim na płw. Istria}
\Clue{15}{}{analiza, której celem jest dobór optymalnych akcji do portfela inwestora giełdowego, co ma zapewnić możliwie najwyższy zysk przy najmniejszym możliwym ryzyku}
\Clue{16}{}{osoba, której działalność polega na udzielaniu porad; od doradcy różni się instytucjonalizacją zawodu i związanym z nim prestiżem}
\Clue{17}{}{budowla ogrodowa położona na wzgórzu, w ciekawym otoczeniu, z której roztaczał się rozległy widok na kompozycję ogrodu}
\Clue{18}{}{reakcja, w której wyniku następuje zmniejszenie liczby atomów wodoru w cząsteczce}
\Clue{19}{}{język z grupy języków ugrofińskich}
\Clue{20}{}{skała metamorficzna powstała z przeobrażenia wapieni, rzadziej dolomitów}
\Clue{21}{}{transwestyta - osoba, która upodabnia się do przedstawicieli płci przeciwnej poprzez strój i zachowanie, celem osiągnięcia satysfakcji emocjonalnej lub seksualnej}
\Clue{22}{}{w wojsku: żołnierz lub grupa żołnierzy wysyłana ze składu placówki, czaty bądź sił głównych pododdziału}
\Clue{23}{}{ostra, zakaźna, zaraźliwa choroba koni}
\Clue{25}{}{mieszkaniec Bahrajnu, człowiek pochodzenia bahrajńskiego}
\Clue{26}{}{przedmiot, zwierzę, temat objęty zakazem związanym z kultem}
\Clue{30}{}{zbrojne wtargnięcie na terytorium innego państwa, najazd}\end{PuzzleClues}\newpage\section*{Krzyżówka 46}

\noindent\begin{Puzzle}{24}{25}|*	|*	|*	|[1][S]\darr	|*	|*	|*	|*	|[2][S]\drarr	|r	|u	|c	|h	|[][,]{ }	|o	|p	|ó	|ź	|n	|i	|o	|n	|y	|*	|[3][S]\darr	|.
|*	|[4][S]\darr	|*	|k	|*	|*	|*	|[5][S]\rarr	|s	|t	|r	|u	|k	|t	|u	|r	|a	|l	|i	|s	|t	|k	|a	|*	|z	|.
|[6][S]\drarr	|m	|i	|o	|d	|ó	|w	|k	|a	|[][,]{ }	|k	|a	|r	|m	|a	|z	|y	|n	|o	|w	|a	|*	|*	|*	|n	|.
|a	|e	|*	|ń	|*	|*	|*	|[7][S]\rarr	|m	|ł	|y	|n	|o	|w	|y	|*	|*	|*	|*	|*	|[8][S]\darr	|*	|*	|*	|a	|.
|r	|c	|*	|[][,]{ }	|[9][S]\darr	|[10][S]\rarr	|m	|i	|o	|d	|z	|i	|a	|r	|e	|k	|*	|*	|[11][S]\darr	|*	|ż	|*	|*	|*	|k	|.
|y	|h	|*	|w	|ż	|*	|[12][S]\darr	|*	|l	|[13][S]\drarr	|n	|i	|e	|p	|o	|r	|z	|ą	|d	|n	|o	|ś	|ć	|*	|[][,]{ }	|.
|t	|[][,]{ }	|*	|i	|e	|[14][S]\darr	|r	|[15][S]\rarr	|o	|p	|a	|r	|i	|n	|*	|*	|*	|*	|e	|[16][S]\darr	|ł	|[17][S]\darr	|[18][S]\darr	|[19][S]\darr	|o	|.
|m	|i	|[20][S]\drarr	|a	|r	|g	|e	|n	|t	|a	|n	|*	|*	|[21][S]\darr	|*	|[22][S]\darr	|*	|*	|s	|d	|ę	|k	|p	|n	|r	|.
|e	|r	|t	|c	|d	|o	|n	|*	|[][,]{ }	|s	|*	|[23][S]\darr	|[24][S]\darr	|ż	|[25][S]\darr	|o	|*	|[26][S]\darr	|k	|y	|d	|l	|r	|i	|i	|.
|t	|l	|o	|k	|n	|r	|t	|*	|s	|t	|*	|k	|w	|o	|f	|c	|*	|d	|a	|s	|n	|a	|a	|e	|e	|.
|y	|a	|p	|i	|i	|y	|a	|*	|z	|o	|*	|o	|k	|n	|l	|h	|[27][S]\darr	|y	|[][,]{ }	|p	|i	|p	|w	|z	|n	|.
|k	|n	|ó	|*	|k	|c	|*	|*	|e	|f	|*	|r	|ł	|a	|a	|l	|p	|n	|k	|e	|c	|a	|o	|r	|t	|.
|a	|d	|r	|*	|*	|z	|[28][S]\rarr	|g	|r	|o	|ź	|b	|a	|[][,]{ }	|k	|a	|r	|a	|l	|n	|a	|*	|r	|ę	|a	|.
|[][,]{ }	|z	|*	|*	|*	|k	|*	|*	|o	|r	|*	|a	|d	|l	|i	|j	|o	|r	|o	|s	|*	|[29][S]\darr	|z	|c	|c	|.
|p	|k	|[30][S]\drarr	|a	|r	|a	|l	|s	|k	|i	|e	|*	|*	|o	|*	|t	|f	|*	|z	|a	|*	|d	|ą	|z	|y	|.
|r	|i	|m	|*	|*	|*	|*	|*	|o	|a	|*	|*	|*	|t	|[31][S]\darr	|u	|i	|*	|e	|*	|[32][S]\darr	|i	|d	|n	|j	|.
|e	|*	|e	|*	|*	|[33][S]\rarr	|b	|u	|k	|*	|*	|*	|[34][S]\darr	|a	|r	|s	|t	|*	|t	|*	|p	|a	|n	|o	|n	|.
|s	|*	|t	|[35][S]\drarr	|s	|a	|l	|w	|a	|*	|*	|*	|m	|*	|y	|*	|e	|*	|o	|*	|u	|g	|o	|ś	|y	|.
|b	|*	|a	|p	|*	|*	|*	|*	|d	|*	|*	|*	|ł	|[36][S]\rarr	|p	|o	|r	|y	|w	|c	|z	|o	|ś	|ć	|*	|.
|u	|[37][S]\rarr	|f	|r	|e	|s	|k	|*	|ł	|*	|*	|*	|o	|*	|s	|*	|o	|*	|a	|[38][S]\darr	|z	|n	|ć	|*	|*	|.
|r	|*	|r	|a	|[39][S]\rarr	|k	|o	|m	|u	|n	|i	|s	|t	|a	|*	|*	|l	|*	|*	|g	|e	|a	|*	|*	|*	|.
|g	|*	|a	|c	|*	|*	|[40][S]\rarr	|o	|b	|i	|e	|g	|*	|*	|*	|*	|k	|*	|*	|r	|l	|l	|*	|*	|*	|.
|e	|*	|z	|a	|*	|[41][S]\rarr	|v	|i	|o	|l	|a	|[][,]{ }	|d	|a	|[][,]{ }	|g	|a	|m	|b	|a	|*	|*	|*	|*	|*	|.
|r	|*	|a	|*	|*	|[42][S]\rarr	|p	|o	|w	|i	|d	|z	|*	|*	|*	|*	|*	|*	|*	|*	|*	|*	|*	|*	|*	|.
|a	|*	|*	|[43][S]\rarr	|n	|a	|d	|w	|y	|ż	|k	|a	|[][,]{ }	|b	|u	|d	|ż	|e	|t	|o	|w	|a	|*	|*	|*	|.
|*	|[44][S]\rarr	|a	|t	|e	|u	|s	|z	|*	|*	|*	|*	|*	|*	|*	|*	|*	|*	|*	|*	|*	|*	|*	|*	|*	|.\end{Puzzle}

\newpage

\begin{PuzzleClues}{\textbf{Poziome}\\}\Clue{2}{}{ruch niejednostajny, w którym przyspieszenie ma wartość ujemną}
\Clue{5}{}{artystka, która tworzy swe dzieła w oparciu o założenia strukturalizmu}
\Clue{6}{}{Myzomela chermesina - gatunek ptaka z rodziny miodojadów (Meliphagidae)}
\Clue{7}{}{człowiek, który jest odpowiedzialny za koordynację zachowania kibiców znajdujących się w młynie na trybunach stadionu, często wyposażony w megafon, bębny, intonujący hymn, nawołujący do klaskania, okrzyków, śpiewów itd}
\Clue{10}{}{Plectorhyncha lanceolata - gatunek ptaka z rodziny miodojadów (Meliphagidae)}
\Clue{13}{}{cecha człowieka: brak ucziwości, moralności}
\Clue{15}{}{radziecki biolog i biochemik (1894-1980); autor jednej z hipotez powstania życia na Ziemi}
\Clue{20}{}{stop metali (zawierający 40-70\% miedzi, 10-20\% niklu i 5-40\% cynku), mosiądz wysokoniklowy, podobny w kolorze do srebra}
\Clue{28}{}{przestępstwo groźby karalnej polega na grożeniu innej osobie popełnieniem przestępstwa na jej szkodę lub szkodę jej najbliższego. Ten występek ścigany jest z oskarżenia publicznego}
\Clue{30}{}{bezodpływowe, słone jezioro w Uzbekistanie i Kazachstanie, powierzchnia 40 tyś. km2, głębokość do 20-25 m}
\Clue{33}{}{drewno bukowe}
\Clue{35}{}{jednoczesny wystrzał na komendę z wielu karabinów lub armat}
\Clue{36}{}{cecha zachowania, które pokazuje, że ktoś jest porwyczy}
\Clue{37}{}{technika malarstwa ściennego polegająca na malowaniu na mokrym tynku farbami odpornymi na alkaliczne działanie zawartego w zaprawie wapna}
\Clue{39}{}{wyznawca ideologii komunistycznej, działacz}
\Clue{40}{}{powszechna obecność czegoś, funkcjonowanie}
\Clue{41}{}{chordofon smyczkowy, instrument dawny o kształcie zbliżonym do dzisiejszej wiolonczeli}
\Clue{42}{}{wieś w Wielkopolsce, siedziba gminy o tej samej nazwie}
\Clue{43}{}{suma dochodów budżetowych przewyższająca wydatki budżetowe; jeden z rodzajów nadwyżki finansowej}
\Clue{44}{}{osoba niewierząca, ateista}\end{PuzzleClues}

\begin{PuzzleClues}{\textbf{Pionowe}\\}\Clue{1}{}{wiatka - rasa małego, rosyjskiego, prymitywnego konia rolniczego pochodzącego od tarpana; używany do  zaprzęgu, transportu, przy wyrębie drzew oraz różnorakich prac rolniczych}
\Clue{2}{}{samolot zwykle używany do lotów długodystansowych: transkontynentalnych i transoceanicznych, cechujący się szerokim kadłubem - zwykle od 5 do 6 m}
\Clue{3}{}{fizyczny znak, symbol, który ma pomóc w orientacji w terenie}
\Clue{4}{}{chrząstnica kędzierzawa, Chondrus crispus - gatunek glonu należącego do typu (gromady) krasnorostów spotykany w północnej części Oceanu Atlantyckiego (jego irlandzka nazwa to Carrageen, co oznaczaskałka)}
\Clue{6}{}{układ aksjomatyczny liczb naturalnych z dodawaniem}
\Clue{8}{}{drobny, nadrzewny gryzoń z rodziny popielicowatych, w Polsce bardzo rzadka, chroniona}
\Clue{9}{}{przedstawiciel dawnej grupy ludności służebnej, która zajmowała się transportem i rozstawianiem namiotów dla władcy podczas podróży}
\Clue{11}{}{element wyposażenia ubikacji, umożliwiający wypróżnianie się lub oddawanie moczu w pozycji siedzącej, bądź stojącej}
\Clue{12}{}{świadczenie pieniężne}
\Clue{13}{}{w kościołach starochrześcijańskich: dwa pomieszczenia w przedłużeniu naw bocznych, służące jako dwudzielna zakrystia}
\Clue{14}{}{smak zbliżony do goryczy, lekko gorzki}
\Clue{16}{}{zwolnienie od obowiązujących przepisów prawa kościelnego, udzielane zwykle przez biskupa lub papieża}
\Clue{17}{}{zawieradło zaworu klapowego w postaci odchylnej płyty}
\Clue{18}{}{cecha; to, że ktoś (człowiek lub grupa ludzi) przestrzega prawa i domaga sie jego przestrzegania}
\Clue{19}{}{zrobienie czegoś nietaktownego, nieodpowiedniego}
\Clue{20}{}{broń obuchowa, rodzaj ciężkiej siekiery, służył do rzucania i walki wręcz}
\Clue{21}{}{kobieta nadmiernie ciekawa, którą ta ciekawość gubi, sprowadza na nią nieszczęście, staje się przyczyną cierpień}
\Clue{22}{}{pogardliwie o osobie, która nadużywa alkoholu}
\Clue{23}{}{dźwignia na końcu wału służąca do wprowadzenia go w ruch obrotowy}
\Clue{24}{}{wnętrze wyrobu papierniczego, czyli wszystkie kartki znajdujące się wewnątrz okładki}
\Clue{25}{}{potrawa z pokrojonych w cienkie paski przedżołądków wołowych lub cielęcych gotowanych w wywarze z włoszczyzny}
\Clue{26}{}{historyczna arabska złota moneta, bita w VII-XV wieku}
\Clue{27}{}{profiterol - niewielkie ciastko zrobione z ciasta parzonego nadzianego kremem}
\Clue{29}{}{rodzaj cienkiej tkaniny wełnianej, półwełnianej lub bawełnianej o splocie skośnym z bardzo dużym nachyleniem prążków}
\Clue{30}{}{dosłowne tłumaczenie (przekład) z jednego języka na inny słowo po słowie i linia po linii dokonany w celu dokładnego oddania treści, bez uwzględniania walorów artystycznych}
\Clue{31}{}{tkanina bawełniana, półbawełniana lub jedwabna o wypukłych, podłużnych lub poprzecznych prążkach używana na suknie, płaszcze, obicia mebli}
\Clue{32}{}{element układanki obrazkowej o charakterystycznym kształcie}
\Clue{34}{}{ciężki młotek ręczny}
\Clue{35}{}{funkcjonowanie jakiejś maszyny, jej elementu lub elementu organizmu żywego}
\Clue{38}{}{zmienność, ruchliwość jakiegoś zjawiska}\end{PuzzleClues}\newpage\section*{Krzyżówka 47}

\noindent\begin{Puzzle}{23}{22}|*	|*	|[1][S]\drarr	|a	|d	|o	|r	|a	|c	|j	|a	|*	|[2][S]\drarr	|d	|z	|i	|o	|b	|ó	|w	|k	|a	|*	|*	|.
|*	|*	|t	|[3][S]\rarr	|k	|o	|s	|z	|a	|r	|n	|i	|a	|k	|*	|*	|[4][S]\drarr	|p	|l	|a	|n	|k	|a	|*	|.
|*	|[5][S]\darr	|r	|[6][S]\drarr	|k	|a	|t	|a	|l	|o	|g	|*	|u	|[7][S]\rarr	|m	|ł	|o	|d	|z	|i	|k	|*	|*	|*	|.
|[8][S]\rarr	|k	|i	|p	|a	|*	|[9][S]\drarr	|r	|e	|g	|e	|n	|t	|*	|*	|[10][S]\rarr	|d	|i	|o	|r	|a	|m	|a	|*	|.
|*	|o	|a	|e	|[11][S]\darr	|*	|p	|[12][S]\darr	|[13][S]\darr	|*	|*	|*	|o	|[14][S]\drarr	|b	|a	|w	|e	|ł	|n	|a	|*	|*	|*	|.
|*	|m	|d	|t	|s	|*	|r	|p	|z	|*	|*	|*	|s	|c	|[15][S]\drarr	|k	|a	|z	|u	|i	|s	|t	|a	|*	|.
|*	|u	|a	|r	|k	|*	|z	|o	|ł	|*	|*	|*	|k	|z	|b	|[16][S]\drarr	|c	|h	|r	|u	|s	|t	|*	|[17][S]\darr	|.
|*	|n	|*	|e	|r	|[18][S]\darr	|e	|l	|o	|*	|*	|*	|l	|o	|o	|g	|h	|*	|*	|*	|*	|*	|*	|r	|.
|*	|i	|[19][S]\darr	|l	|z	|a	|p	|i	|t	|*	|[20][S]\drarr	|l	|e	|r	|k	|a	|*	|*	|*	|*	|[21][S]\darr	|*	|[22][S]\darr	|a	|.
|*	|k	|c	|[][,]{ }	|y	|n	|s	|n	|a	|*	|f	|*	|p	|t	|*	|r	|[23][S]\darr	|*	|[24][S]\darr	|*	|d	|[25][S]\darr	|s	|j	|.
|*	|a	|z	|u	|d	|g	|z	|e	|[][,]{ }	|[26][S]\darr	|*	|*	|*	|*	|*	|n	|i	|[27][S]\rarr	|k	|l	|o	|s	|z	|*	|.
|[28][S]\drarr	|t	|a	|b	|l	|i	|c	|z	|k	|a	|[][,]{ }	|m	|n	|o	|ż	|e	|n	|i	|a	|*	|m	|a	|w	|*	|.
|b	|o	|r	|o	|i	|e	|z	|y	|s	|r	|*	|*	|*	|*	|*	|l	|t	|*	|f	|*	|i	|l	|a	|[29][S]\darr	|.
|a	|r	|n	|g	|c	|l	|e	|j	|i	|c	|*	|*	|*	|*	|*	|o	|e	|*	|a	|*	|n	|o	|b	|m	|.
|n	|*	|y	|i	|a	|c	|l	|s	|ę	|t	|*	|*	|*	|*	|*	|w	|g	|*	|r	|*	|o	|w	|*	|i	|.
|k	|*	|[][,]{ }	|*	|[][,]{ }	|z	|e	|k	|g	|*	|[30][S]\rarr	|s	|z	|e	|d	|a	|r	|*	|*	|*	|*	|a	|*	|n	|.
|i	|*	|k	|*	|r	|y	|n	|i	|a	|*	|*	|*	|*	|*	|[31][S]\rarr	|t	|a	|l	|i	|c	|h	|*	|*	|a	|.
|e	|*	|o	|*	|o	|k	|i	|*	|*	|[32][S]\rarr	|f	|o	|t	|o	|h	|e	|l	|i	|o	|g	|r	|a	|f	|*	|.
|r	|*	|ń	|*	|g	|*	|e	|[33][S]\rarr	|ś	|c	|i	|ę	|c	|i	|e	|*	|n	|*	|*	|*	|*	|*	|*	|*	|.
|*	|*	|*	|*	|a	|*	|*	|*	|[34][S]\rarr	|d	|u	|p	|e	|k	|[][,]{ }	|ż	|o	|ł	|ę	|d	|n	|y	|*	|*	|.
|*	|*	|*	|*	|t	|*	|[35][S]\rarr	|i	|r	|o	|n	|i	|c	|z	|n	|o	|ś	|ć	|*	|*	|*	|*	|*	|*	|.
|*	|[36][S]\rarr	|d	|r	|a	|b	|i	|n	|k	|i	|*	|*	|*	|*	|*	|*	|ć	|*	|*	|*	|*	|*	|*	|*	|.
|[37][S]\rarr	|e	|d	|e	|*	|*	|*	|*	|[38][S]\rarr	|s	|k	|w	|i	|e	|r	|k	|*	|*	|*	|*	|*	|*	|*	|*	|.\end{Puzzle}

\newpage

\begin{PuzzleClues}{\textbf{Poziome}\\}\Clue{1}{}{podziw, uwielbienie dla kogoś lub czegoś, okazywane w sposób widoczny dla otoczenia}
\Clue{2}{}{tropikalny ptak z rzędu siewkowatych}
\Clue{3}{}{w wojsku kara dyscyplinarna polegająca na zakazie opuszczania koszar przez określony czas}
\Clue{4}{}{deska z poszycia okrętu}
\Clue{6}{}{uporządkowany spis, wykaz zawartości jakiegoś zbioru}
\Clue{7}{}{w sporcie: zawodnik młodszy od juniora (do 13 roku życia)}
\Clue{8}{}{kierowcy ciężarówek mówią tak o pace samochodu}
\Clue{9}{}{w dawnej Polsce: sekretarz królewski, osoba sprawująca pieczę nad kancelarią królewską}
\Clue{10}{}{obraz w głębokim obramowaniu, którego pewne części są nieprzezroczyste a inne malowane techniką laserunkową odpowiednio oświetlony daje wrażenie trójwymiarości}
\Clue{14}{}{włókno okrywające nasiona drzew i krzewów tej rośliny stanowiące surowiec do wyrobu przędzy, nici i celulozy, także tkanina z tego włókna}
\Clue{15}{}{jezuita z XVI-XVII wieku, zwolennik kazuistyki - pewnego sposobu myślenia o zasadach moralnych, metody rozpatrywania grzechów typowej dla probablizmu}
\Clue{16}{}{faworek - cienkie chrupkie ciastko w kształcie kokardki, smażone w głębokim tłuszczu i posypywane cukrem pudrem, najczęściej przygotowywyane w czasie karnawału, na tłusty czwartek lub na ostatki}
\Clue{20}{}{skowronek borowy}
\Clue{27}{}{krój ubrań (rozszerzony dół ubrania)}
\Clue{28}{}{tabelaryczny sposób zestawienia wyników mnożenia przez siebie liczb naturalnych}
\Clue{30}{}{gwiazda w gwiazdozbiorze Kasjopei}
\Clue{31}{}{dyrygent czeski (1883-1961); dyrygent Filharmonii Czeskiej}
\Clue{32}{}{przyrząd do fotografowania Słońca, używany do rejestracji plam słonecznych}
\Clue{33}{}{spowodowanie zamarznięcia, krzepnięcia jakiejś cieczy}
\Clue{34}{}{obraźliwe określenie człowieka niezaradnego lub negatywnie postrzeganego}
\Clue{35}{}{cecha sytuacji, życia, zdarzenia - złośliwość losu występująca pod przykrywką innych, pozytywnych zdarzeń}
\Clue{36}{}{przyrząd gimnastyczny złożony z wielu równoległych drążków przymocowanych z obu stron do długich drewnianych listew, które z kolei przymocowane są do ściany; kształtem przypomina drabinę}
\Clue{37}{}{miasto w środkowej Holandii, 87,8 tys. mieszkańców (1985 r.), przemysł chemiczny, maszynowy i mleczarski}
\Clue{38}{}{płacz, żale, narzekanie}\end{PuzzleClues}

\begin{PuzzleClues}{\textbf{Pionowe}\\}\Clue{1}{}{zespół trzech sąsiadujących ze sobą wysepek ekrany mozaikowego}
\Clue{2}{}{sklep, który mieści się w samochodzie i jest przez to mobilny}
\Clue{4}{}{areszt w wartowni}
\Clue{5}{}{program komputerowy pozwalający na przesyłanie natychmiastowych komunikatów pomiędzy dwoma lub większą liczbą komputerów poprzez sieć komputerową, zazwyczaj Internet}
\Clue{6}{}{Pterodroma deserta - gatunek ptaka z rodziny burzykowatych (Procellariidae)}
\Clue{9}{}{w pszczelarstwie: zbyt duża liczba rodzin pszczół przypadających na dany teren (zwykle na 1 km kwadratowy)}
\Clue{11}{}{Pterois antennata - gatunek morskiej ryby z rodziny skorpenowatych występujący w pobliżu Australii, Mikronezji, Japonii i wschodnich wybrzeży Afryki}
\Clue{12}{}{któryś z języków używanych w tzw. trójkącie polinezyjskim}
\Clue{13}{}{księga, w której zapisuje się ważne wydarzenia}
\Clue{14}{}{człowiek (zwłaszcza młody człowiek), który jest niegrzeczny}
\Clue{15}{}{odcinek łączący dwa leżące obok siebie wierzchołki wielokąta}
\Clue{16}{}{Crangonidae - rodzina skorupiaków z infrarzędu krewetek}
\Clue{17}{}{miejce sprzyjające np. interesom}
\Clue{18}{}{muzyka, do której tańczy się angielczyka}
\Clue{19}{}{as, atut, osoba typowana na zwycięzce, najsilniejsza w danych rozgrywkach}
\Clue{20}{}{dźwięk muzyczny, którego częstotliwość w oktawie razkreślnej wynosi 349,6 Hz.}
\Clue{21}{}{płaszcz z kapturem noszony przez mnichów, członków bractw religijnych}
\Clue{22}{}{z pogardą, niechęcią o Niemcu}
\Clue{23}{}{to że coś jest integralne, coś jest niezbywalne dla czegoś innego, coś innego nie może bez tego istnieć}
\Clue{24}{}{koks, mięśniak; kulturysta, człowiek posiadający rozbudowaną tkankę mięśniową}
\Clue{25}{}{pracownik służby zdrowia, zajmujący się utrzymaniem czystości w placówkach ochrony zdrowia, w szczególności w salach szpitalnych}
\Clue{26}{}{(1914-73) pisarz i lotnik „W pogoni za Luftwaffe”, „Cena życia” - powieści lotnicze}
\Clue{28}{}{krupier}
\Clue{29}{}{rodzaj broni, ładunek wybuchowy, np. mina przeciwpiechotna}\end{PuzzleClues}\newpage\section*{Krzyżówka 48}

\noindent\begin{Puzzle}{23}{31}|*	|*	|*	|*	|*	|*	|*	|*	|*	|*	|*	|*	|*	|*	|*	|*	|*	|*	|[1][S]\darr	|*	|[2][S]\darr	|*	|*	|*	|.
|*	|*	|*	|*	|*	|*	|*	|*	|*	|*	|*	|*	|*	|*	|*	|*	|[3][S]\darr	|[4][S]\drarr	|n	|i	|k	|t	|*	|[5][S]\darr	|.
|*	|*	|*	|*	|[6][S]\darr	|*	|*	|*	|*	|*	|*	|*	|*	|*	|*	|[7][S]\darr	|p	|m	|e	|*	|o	|*	|*	|m	|.
|*	|*	|*	|*	|k	|[8][S]\darr	|*	|*	|*	|*	|*	|*	|*	|*	|*	|i	|e	|i	|k	|*	|ł	|*	|*	|e	|.
|*	|[9][S]\darr	|[10][S]\drarr	|j	|a	|b	|ł	|e	|c	|z	|n	|i	|k	|*	|*	|n	|r	|ó	|r	|*	|o	|*	|*	|t	|.
|[11][S]\drarr	|s	|t	|o	|p	|o	|t	|o	|n	|a	|*	|*	|*	|*	|*	|s	|m	|d	|o	|*	|w	|*	|*	|o	|.
|a	|k	|r	|*	|o	|d	|*	|[12][S]\rarr	|d	|e	|k	|a	|d	|e	|n	|t	|*	|[][,]{ }	|m	|*	|r	|*	|*	|d	|.
|u	|a	|w	|*	|k	|e	|*	|*	|[13][S]\darr	|*	|[14][S]\drarr	|o	|r	|a	|w	|a	|*	|s	|a	|*	|ó	|*	|*	|a	|.
|g	|l	|o	|*	|*	|*	|[15][S]\darr	|[16][S]\drarr	|p	|r	|e	|c	|e	|l	|*	|l	|*	|e	|n	|*	|t	|*	|*	|[][,]{ }	|.
|u	|a	|ż	|*	|*	|*	|s	|c	|r	|*	|l	|[17][S]\rarr	|a	|u	|l	|a	|*	|k	|t	|*	|*	|[18][S]\darr	|*	|e	|.
|s	|*	|n	|*	|*	|[19][S]\darr	|z	|z	|o	|*	|o	|*	|*	|*	|*	|t	|*	|c	|a	|*	|[20][S]\darr	|a	|*	|d	|.
|t	|*	|i	|[21][S]\drarr	|s	|u	|m	|a	|k	|*	|p	|*	|*	|*	|*	|o	|[22][S]\darr	|y	|*	|*	|k	|n	|[23][S]\darr	|u	|.
|y	|*	|c	|p	|*	|f	|a	|p	|s	|[24][S]\rarr	|s	|c	|*	|*	|*	|r	|g	|j	|*	|*	|o	|i	|s	|k	|.
|n	|*	|a	|i	|*	|a	|c	|l	|e	|*	|o	|*	|*	|*	|*	|s	|ó	|n	|*	|*	|m	|o	|k	|a	|.
|i	|[25][S]\darr	|[][,]{ }	|a	|*	|*	|i	|a	|n	|[26][S]\rarr	|w	|y	|ż	|*	|*	|t	|r	|y	|*	|*	|e	|ł	|r	|c	|.
|z	|b	|h	|ś	|*	|*	|a	|[][,]{ }	|o	|[27][S]\rarr	|a	|n	|t	|y	|k	|w	|a	|*	|*	|*	|d	|e	|z	|y	|.
|m	|i	|a	|n	|[28][S]\darr	|*	|k	|m	|s	|*	|t	|[29][S]\rarr	|s	|a	|m	|o	|l	|o	|t	|*	|i	|c	|y	|j	|.
|*	|o	|b	|i	|c	|*	|*	|o	|*	|*	|e	|*	|*	|*	|*	|*	|e	|*	|*	|*	|a	|z	|d	|n	|.
|[30][S]\drarr	|s	|u	|c	|h	|y	|[][,]{ }	|d	|o	|k	|*	|*	|[31][S]\rarr	|p	|i	|e	|k	|ł	|o	|*	|n	|e	|l	|a	|.
|w	|*	|*	|a	|m	|[32][S]\rarr	|k	|r	|y	|n	|i	|c	|z	|n	|i	|k	|[][,]{ }	|g	|i	|ę	|t	|k	|i	|*	|.
|*	|*	|*	|*	|u	|[33][S]\rarr	|k	|a	|m	|l	|o	|t	|*	|[34][S]\rarr	|j	|ą	|d	|r	|o	|*	|*	|*	|c	|*	|.
|[35][S]\rarr	|f	|l	|o	|r	|e	|n	|*	|*	|*	|[36][S]\rarr	|s	|t	|w	|i	|e	|r	|d	|z	|e	|n	|i	|e	|*	|.
|[37][S]\rarr	|p	|u	|ł	|a	|p	|*	|*	|*	|*	|*	|*	|[38][S]\darr	|[39][S]\drarr	|j	|a	|z	|z	|ó	|w	|k	|a	|*	|*	|.
|*	|*	|*	|[40][S]\darr	|[][,]{ }	|*	|*	|[41][S]\drarr	|l	|o	|g	|o	|w	|a	|n	|i	|e	|*	|*	|*	|*	|*	|*	|*	|.
|*	|[42][S]\darr	|*	|a	|ś	|*	|*	|k	|*	|*	|*	|*	|i	|d	|*	|[43][S]\rarr	|w	|y	|j	|e	|c	|*	|*	|*	|.
|*	|s	|[44][S]\darr	|s	|r	|*	|*	|o	|*	|*	|*	|*	|d	|a	|[45][S]\rarr	|g	|n	|i	|o	|t	|*	|*	|*	|*	|.
|[46][S]\drarr	|a	|s	|t	|e	|r	|[][,]{ }	|l	|o	|w	|r	|i	|e	|g	|o	|*	|y	|*	|*	|*	|*	|*	|*	|*	|.
|a	|l	|o	|e	|d	|*	|*	|c	|*	|*	|*	|*	|t	|i	|*	|*	|*	|*	|*	|*	|*	|*	|*	|*	|.
|s	|s	|n	|r	|n	|[47][S]\rarr	|r	|z	|e	|ź	|*	|*	|a	|o	|*	|*	|*	|*	|*	|*	|*	|*	|*	|*	|.
|*	|a	|a	|i	|i	|[48][S]\rarr	|z	|a	|w	|i	|s	|*	|*	|*	|*	|*	|*	|*	|*	|*	|*	|*	|*	|*	|.
|*	|*	|r	|x	|a	|*	|[49][S]\rarr	|k	|o	|ł	|o	|[][,]{ }	|s	|t	|e	|r	|o	|w	|e	|*	|*	|*	|*	|*	|.
|*	|*	|*	|*	|*	|*	|*	|*	|*	|*	|*	|*	|*	|*	|*	|*	|*	|*	|*	|*	|*	|*	|*	|*	|.\end{Puzzle}

\newpage

\begin{PuzzleClues}{\textbf{Poziome}\\}\Clue{4}{}{ktoś nic nieznaczący}
\Clue{10}{}{ciasto, placek z jabłkami}
\Clue{11}{}{jednostka pracy, która musi być wykonana, żeby podnieść na wysokość stopy ciało o masie tony w polu grawitacyjnym Ziemi (g = 9,81 m/s2)}
\Clue{12}{}{rodzaj pesymisty}
\Clue{14}{}{kraina w Słowacji i w Polsce, w dorzeczu rzeki Orawy}
\Clue{16}{}{obwarzanek o kształcie zbliżonym do ósemki}
\Clue{17}{}{wysoka sala reprezentacyjna}
\Clue{21}{}{przyprawa charakterystyczna dla kuchni kaukaskiej i azjatyckiej; suszone i zmielone owoce (pestkowce) sumaku garbarskiego}
\Clue{24}{}{w chemii: symbol skandu}
\Clue{26}{}{obszar wysokiego ciśnienia atmosferycznego, w którym najwyższe panuje w centrum układu, a prądy powietrza skierowane są na zewnątrz ku obszarom o niższym ciśnieniu}
\Clue{27}{}{typ kroju pisma opartego na alfabecie łacińskim, który współcześnie jest krojem dominującym wśród pism drukarskich}
\Clue{29}{}{statek powietrzny cięższy od powietrza (aerodyna), utrzymujący się w powietrzu dzięki wytwarzanej sile nośnej za pomocą nieruchomych, w danych warunkach względem statku, skrzydeł, utrzymujący się w locie poziomym dzięki ciągowi silnika (silników)}
\Clue{30}{}{rodzaj budowli hydrotechnicznej w porcie wodnym, najczęściej w stoczni; wąski basen portowy ze szczelnymi wrotami oraz urządzeniami wypompowującymi z jego wnętrza wodę}
\Clue{31}{}{stan duszy ludzkiej po śmierci ciała, w którym pozostaje ona w ciągłej teraźniejszości odłączona od jakiegokolwiek kontaktu z Bogiem, co jest dla niej źródłem cierpienia}
\Clue{32}{}{Nitella flexilis - kosmopolityczny gatunek ramienicy z rodzaju krynicznik}
\Clue{33}{}{ciepła tkanina z wielbłądziej wełny używana głównie do szycia okryć wierzchnich}
\Clue{34}{}{centrum ciała niebieskiego; charakteryzuje się dużą gęstością}
\Clue{35}{}{inna nazwa guldena holenderskiego}
\Clue{36}{}{zdanie, wypowiedź, w której wyraża się jakąś myśl, stwierdza się jakiś fakt, stan rzeczy}
\Clue{37}{}{powała, warstwa desek mocowana na belkach stropowych lub między nimi}
\Clue{39}{}{but wyglądem nawiązujący do specjalnego buta używanego w tańcu jazzowym}
\Clue{41}{}{proces uwierzytelniania i autoryzacji użytkownika komputera, polegający w większości przypadków na podaniu identyfikatora użytkownika i hasła uwierzytelniającego w celu uzyskania dostępu w wyniku ściśle zdefiniowanych uprawnień do korzystania z określonego systemu informatycznego, systemu komputerowego, komputera czy sieci komputerowej}
\Clue{43}{}{nazwa małpy z grupy szerokonosych, żyjącej w lasach Ameryki Południowej; małpa ta swoją nazwę zawdzięcza wydawaniu głosów przypominających ryki i przeciągłe wycia}
\Clue{45}{}{ubytek przekroju ciągnionego lub walcowanego materiału}
\Clue{46}{}{Aster lowrieanus - gatunek rośliny należący do rodziny astrowatych}
\Clue{47}{}{przenośnie, żartobliwie: bardzo trudny egzamin, którego nie zdało wiele osób, też: trudne przesłuchanie, którego nie przeszło wiele osób}
\Clue{48}{}{rodzaj lotu statku powietrznego, podczas którego jest on nieruchomy względem powietrza}
\Clue{49}{}{element urządzenia sterowego, uchwyt o kształcie dużej obręczy, za pomocą której sternik kontroluje ustawienie płetwy sterowej}\end{PuzzleClues}

\begin{PuzzleClues}{\textbf{Pionowe}\\}\Clue{1}{}{czarnoksiężnik, który przyzywa cienie zmarłych w celu poznania prawdopodobnych wersji przyszłości lub w innych celach własnych (np. po to, by duchy były na jego usługach)}
\Clue{2}{}{maszyna wykorzystywana do wciągania, przyciągania lub opuszczania ciężarów}
\Clue{3}{}{kraina historyczna w północnej Rosji przeduralskiej, między Peczorą, Wyczegdą i Kamą a Uralem}
\Clue{4}{}{miód pszczeli sprzedawany w ramkach wraz z woszczyną}
\Clue{5}{}{metoda, wykorzystywana w nauczaniu; niekiedy utożsamiana z 'techniką dydaktyczną'}
\Clue{6}{}{nazwa łatwo łamiących się, nieprzędnych włókien nasiennych (w formie puchu) o długości do 35 mm, otrzymywanych z różnych gatunków drzew z podrodziny wełniakowatych}
\Clue{7}{}{instalowanie urządzeń technicznych, przewodów itp}
\Clue{8}{}{astronom niemiecki (1747-1826) rozpowszechnił prawidłowość w odległościach kolejnych planet od Słońca}
\Clue{9}{}{zakres dźwięków jakimi rozporządza dany instrument lub rodzaj głosu}
\Clue{10}{}{Protobothrops flavoviridis - gatunek węża z podrodziny grzechotnikowatych, z rodziny żmijowatych}
\Clue{11}{}{filozofia św. Augustyna z Hippony i  jej wpływ na europejską myśl teologiczno-filozoficzną}
\Clue{13}{}{w starożytnej Grecji obywatel polis pełniący funkcję opiekuna dla przybyszów z innego polis, którego był proksenosem}
\Clue{14}{}{Elopidae - rodzina ławicowych ryb elopsokształtnych (Elopiformes); zamieszkują ciepłe i tropikalne wody morskie, rzadko wpływają do estuariów i rzek}
\Clue{15}{}{szmaciany but}
\Clue{16}{}{Ardea herodias herodias - nominatywny podgatunek czapli modrej (Ardea herodias); występuje w Ameryce Północnej, od południowej Kanady po środkową część Stanów Zjednoczonych}
\Clue{18}{}{przenośnie lub czule o małym dziecku}
\Clue{19}{}{miasto w Federacji Rosyjskiej, stolica Baszkirii}
\Clue{20}{}{błazen, głupek, człowiek, który zachowuje się niepoważnie, budzi politowanie; artysta, człowiek, który bierze udział w splocie okoliczności, który można nazwać farsą}
\Clue{21}{}{dopływ Cybiny, ciek zlokalizowany na terenie Poznania}
\Clue{22}{}{Dendrohyrax arboreus - gatunek ssaka z rodziny góralkowatych, zamieszkujący Mozambik, Zambię, Tanzanię i Kenię}
\Clue{23}{}{Pterois - rodzaj ryb skorpenokształtnych z rodziny skorpenowatych}
\Clue{25}{}{mieszanka substancji używana do stymulacji wzrostu drożdży}
\Clue{28}{}{chmura mieszcząca się na wysokości pomiędzy 2 a 5 km}
\Clue{30}{}{węzeł - jednostka miary, równa jednej mili morskiej na godzinę; stosowana do określania prędkości morskich jednostek pływających, a w części państw i w ruchu międzynarodowym także statków powietrznych (samolotów, śmigłowców, szybowców, balonów)}
\Clue{38}{}{CZATA, CZUJKA, WEDETA, PIKIETA}
\Clue{39}{}{utwór lub fragment melodii w tempie adagio}
\Clue{40}{}{bohater komiksowy, odważny Gal, który broni swojej wioski przed Rzymianami}
\Clue{41}{}{przedstawiciel rodziny ssaków z rzędu gryzoni; występuje w Ameryce Południowej}
\Clue{42}{}{popularny kubański taniec}
\Clue{44}{}{urządzenie do wykrywania i określania położenia obiektów znajdujących się pod wodą}
\Clue{46}{}{w muzyce: dźwięk A obniżony o półton}\end{PuzzleClues}\newpage\section*{Krzyżówka 49}

\noindent\begin{Puzzle}{21}{24}|*	|*	|[1][S]\drarr	|p	|o	|r	|o	|b	|n	|i	|c	|a	|[][,]{ }	|o	|p	|y	|l	|o	|n	|a	|*	|*	|.
|*	|[2][S]\drarr	|c	|e	|c	|h	|o	|w	|n	|i	|a	|*	|*	|*	|*	|*	|*	|[3][S]\darr	|*	|*	|*	|*	|.
|*	|k	|h	|*	|*	|[4][S]\darr	|*	|*	|*	|*	|*	|[5][S]\drarr	|t	|a	|w	|u	|ł	|a	|*	|*	|[6][S]\darr	|*	|.
|*	|o	|ó	|[7][S]\darr	|*	|d	|*	|[8][S]\drarr	|r	|z	|a	|d	|k	|o	|ś	|ć	|*	|r	|*	|*	|z	|*	|.
|*	|m	|r	|o	|*	|y	|*	|m	|*	|[9][S]\darr	|[10][S]\rarr	|w	|e	|n	|u	|s	|j	|a	|n	|k	|a	|*	|.
|*	|i	|*	|g	|[11][S]\rarr	|d	|o	|u	|g	|h	|n	|u	|t	|*	|*	|[12][S]\darr	|[13][S]\darr	|b	|*	|*	|s	|*	|.
|*	|t	|*	|n	|*	|e	|*	|n	|*	|y	|*	|d	|[14][S]\darr	|[15][S]\rarr	|r	|a	|m	|i	|ę	|*	|k	|*	|.
|*	|e	|*	|i	|*	|l	|[16][S]\darr	|g	|*	|r	|*	|z	|m	|[17][S]\darr	|*	|r	|l	|k	|[18][S]\darr	|[19][S]\darr	|r	|*	|.
|*	|t	|[20][S]\darr	|w	|*	|f	|e	|o	|*	|a	|*	|i	|a	|s	|*	|a	|e	|a	|c	|t	|o	|*	|.
|*	|[][,]{ }	|n	|o	|*	|[][,]{ }	|k	|*	|*	|d	|*	|e	|k	|h	|*	|r	|c	|*	|z	|w	|n	|*	|.
|*	|k	|o	|[][,]{ }	|*	|p	|s	|*	|*	|*	|*	|s	|i	|r	|*	|a	|z	|*	|ł	|a	|i	|*	|.
|*	|o	|c	|p	|*	|o	|p	|[21][S]\drarr	|ż	|ó	|ł	|t	|n	|i	|k	|*	|a	|[22][S]\darr	|o	|r	|e	|*	|.
|*	|o	|e	|a	|*	|ł	|l	|z	|*	|*	|*	|o	|t	|v	|*	|*	|r	|d	|n	|d	|c	|*	|.
|*	|r	|k	|l	|*	|u	|o	|i	|*	|*	|*	|l	|o	|e	|*	|*	|z	|i	|[][,]{ }	|z	|[][,]{ }	|*	|.
|*	|d	|[][,]{ }	|i	|*	|d	|a	|e	|*	|*	|*	|e	|s	|r	|[23][S]\drarr	|f	|*	|o	|d	|i	|r	|*	|.
|*	|y	|m	|w	|[24][S]\darr	|n	|t	|m	|*	|*	|*	|c	|z	|*	|c	|*	|*	|e	|y	|e	|y	|*	|.
|*	|n	|y	|o	|p	|i	|a	|i	|*	|*	|*	|i	|*	|*	|i	|*	|*	|c	|n	|l	|b	|*	|.
|*	|a	|s	|w	|o	|o	|t	|a	|[25][S]\rarr	|o	|b	|e	|d	|i	|e	|n	|c	|j	|a	|*	|o	|*	|.
|[26][S]\drarr	|c	|z	|e	|r	|w	|o	|n	|i	|e	|c	|*	|*	|*	|n	|*	|*	|a	|m	|*	|ż	|*	|.
|a	|y	|o	|*	|ę	|y	|r	|i	|*	|[27][S]\rarr	|d	|j	|e	|r	|n	|i	|s	|*	|i	|*	|e	|*	|.
|l	|j	|u	|*	|k	|*	|*	|n	|[28][S]\rarr	|g	|o	|t	|t	|l	|i	|e	|b	|*	|c	|*	|r	|*	|.
|u	|n	|c	|*	|a	|*	|*	|*	|*	|[29][S]\rarr	|b	|y	|c	|z	|k	|i	|*	|*	|z	|*	|*	|*	|.
|m	|y	|h	|*	|*	|*	|[30][S]\rarr	|f	|a	|l	|k	|o	|n	|a	|*	|*	|*	|*	|n	|*	|*	|*	|.
|n	|*	|y	|*	|[31][S]\rarr	|s	|p	|l	|o	|t	|[][,]{ }	|r	|a	|m	|i	|e	|n	|n	|y	|*	|*	|*	|.
|*	|*	|*	|*	|*	|*	|[32][S]\rarr	|z	|w	|i	|e	|r	|z	|y	|n	|i	|e	|c	|*	|*	|*	|*	|.\end{Puzzle}

\newpage

\begin{PuzzleClues}{\textbf{Poziome}\\}\Clue{1}{}{Anthophora pubescens - gatunek pszczoły właściwej z plemienia porobnicowatych}
\Clue{2}{}{miejsce, w którym oznakowuje się przedmioty, np. narzędzia}
\Clue{5}{}{SPIREA}
\Clue{8}{}{cecha czegoś, co składa się ze stosunkowo niewielu elementów lub elementów będących w dużych odstępach od siebie}
\Clue{10}{}{kobieta, która należy do klubu kobiet aktywnych}
\Clue{11}{}{rodzaj wysmażanego na głębokim oleju wyrobu cukierniczego z charakterystyczną dziurką w środku}
\Clue{15}{}{część śmigła, śruby, element wykorzystywany do wprawiania w ruch mechanizmu (w jego funkcjonowaniu jest istotne, żeby miał on znaczną powierzchnię)}
\Clue{21}{}{glistnik jaskółcze ziele, Chelidonium majus - gatunek byliny z rodziny makowatych; nazwa regionalna używana na Mazowszu}
\Clue{23}{}{w chemii: symbol fluoru}
\Clue{25}{}{w czasie schizmy w Kościele katolickim (1378-1417) podział na dwa obozy: obediencja papieży rzymskich i obediencja papieży awiniońskich; w tym czasie także zakony dzieliły się na obediencje, np. zakon karmelitów czy zakon dominikanów, a podziały te ulegały zatarciu po przywróceniu jedności w świecie katolicyzmu}
\Clue{26}{}{moneta złota, floren, dukat, odróżniająca monetę obiegową (ok. 3,5 g złota) od ówczesnej jednostki obrachunkowej, czyli złotego polskiego}
\Clue{27}{}{duński kolarz przełajowy, mistrz świata z 1993 r}
\Clue{28}{}{Leopold (1879-1934) malarz, rysownik i grafik; rysunki o tematyce żołnierskiej, akwaforty, litografie}
\Clue{29}{}{konserwa rybna w ostrej zalewie pomidorowej}
\Clue{30}{}{działo okrętowe używane w XVI-XVII w}
\Clue{31}{}{twór powstały z gałązek przednich (brzusznych) nerwów rdzeniowych biegnących od rdzenia kręgowego}
\Clue{32}{}{dzielnica Wilna}\end{PuzzleClues}

\begin{PuzzleClues}{\textbf{Pionowe}\\}\Clue{1}{}{wiele dźwięków, głosów wykonywanych lub słyszanych naraz; wielogłos}
\Clue{2}{}{struktura organizacji, która zajmuje się organizacją jej działań i nadzorem nad nimi}
\Clue{3}{}{kawa, napój z parzonej arabiki}
\Clue{4}{}{Didelphis albiventris, Didelphis azarae - ssak z rodziny dydelfowatych; zamieszkuje Amerykę Południową, od północnych wybrzeży Kolumbii po Patagonię}
\Clue{5}{}{okres 20 lat}
\Clue{6}{}{Xenochrophis piscator - gatunek węża z rodziny połozowatych, podrodziny zaskrońcowatych, występujący w południowo-wschodniej Azji}
\Clue{7}{}{ogniwo generujące energię elektryczną z reakcji utleniania stale dostarczanego do niego z zewnątrz paliwa}
\Clue{8}{}{rodzaj włókna, otrzymywany z wełny wtórnej}
\Clue{9}{}{polietylen stosowany jako izolacja przewodów i drutów nawojowych}
\Clue{12}{}{ptak z rodziny papug, występuje na terenach Australii i Nowej Gwinei}
\Clue{13}{}{specjalista zajmujący się mleczarstwem}
\Clue{14}{}{nieprzemakalna, lekka tkanina używana do wyrobu kurtek lub płaszczy}
\Clue{16}{}{pracodawca wyzyskujący pracowników, wyzyskiwacz}
\Clue{17}{}{astronauta amerykański; Discovery 1990r}
\Clue{18}{}{układ opisany równaniem różniczkowym}
\Clue{19}{}{osoba niezłomna, odporna na trudy}
\Clue{20}{}{Myotis lucifugus - gatunek ssaka z rodziny mroczkowatych; występuje w lasach i na terenach zabudowanych Ameryki Północnej od 62° N do Meksyku}
\Clue{21}{}{szlachcic, który posiadał ziemię}
\Clue{22}{}{u roślin występowanie żeńskich i męskich organów rozrodczych na różnych osobnikach; dwupienność}
\Clue{23}{}{duże drzewo o rozłożystej koronie, które pozostawia się w lesie w czasie wycinki, aby rzucało cień na młode drzewka i sadzonki, które rosną w jego pobliżu}
\Clue{24}{}{zapewnienie o czymś, gwarancja, poręczenie}
\Clue{26}{}{kleryk, uczeń katolickiego seminarium duchownego}\end{PuzzleClues}\newpage\section*{Krzyżówka 50}

\noindent\begin{Puzzle}{20}{22}|*	|*	|*	|*	|*	|*	|*	|*	|*	|[1][S]\drarr	|ł	|a	|c	|z	|a	|*	|[2][S]\drarr	|t	|y	|p	|*	|.
|*	|*	|*	|*	|*	|*	|*	|*	|*	|b	|[3][S]\rarr	|b	|i	|s	|k	|u	|p	|k	|a	|*	|*	|.
|*	|*	|*	|[4][S]\rarr	|c	|e	|n	|t	|u	|r	|i	|a	|[][,]{ }	|n	|a	|d	|o	|b	|n	|a	|*	|.
|*	|*	|*	|[5][S]\rarr	|m	|o	|d	|e	|r	|u	|n	|e	|k	|*	|*	|[6][S]\drarr	|ł	|o	|n	|o	|*	|.
|*	|*	|*	|*	|*	|*	|*	|*	|[7][S]\darr	|s	|*	|[8][S]\drarr	|m	|a	|k	|s	|u	|r	|a	|*	|*	|.
|*	|*	|*	|*	|*	|*	|[9][S]\darr	|*	|ł	|c	|*	|w	|[10][S]\drarr	|k	|w	|a	|d	|r	|a	|t	|*	|.
|*	|*	|*	|*	|*	|[11][S]\darr	|k	|*	|e	|h	|[12][S]\darr	|ą	|p	|*	|*	|r	|n	|*	|[13][S]\darr	|[14][S]\darr	|*	|.
|*	|*	|*	|*	|*	|f	|r	|*	|b	|e	|l	|s	|r	|*	|*	|*	|i	|*	|h	|b	|*	|.
|*	|*	|*	|[15][S]\rarr	|k	|r	|o	|k	|*	|t	|e	|k	|z	|*	|[16][S]\darr	|*	|k	|[17][S]\darr	|a	|a	|*	|.
|*	|[18][S]\rarr	|g	|i	|g	|a	|w	|o	|l	|t	|*	|i	|e	|*	|b	|*	|[][,]{ }	|k	|k	|j	|*	|.
|*	|*	|[19][S]\darr	|*	|*	|n	|a	|*	|*	|a	|[20][S]\darr	|e	|c	|*	|o	|[21][S]\darr	|z	|l	|a	|c	|[22][S]\darr	|.
|*	|*	|c	|*	|*	|c	|*	|*	|[23][S]\darr	|*	|b	|[][,]{ }	|h	|[24][S]\darr	|i	|ż	|e	|i	|t	|a	|r	|.
|[25][S]\rarr	|k	|a	|n	|i	|u	|k	|[][,]{ }	|d	|ł	|u	|g	|o	|s	|t	|e	|r	|n	|y	|*	|o	|.
|*	|[26][S]\rarr	|b	|e	|n	|z	|i	|*	|e	|*	|t	|a	|w	|e	|o	|g	|o	|g	|s	|*	|z	|.
|[27][S]\rarr	|k	|r	|o	|s	|*	|*	|*	|r	|[28][S]\darr	|a	|r	|a	|k	|*	|a	|w	|e	|t	|*	|m	|.
|*	|[29][S]\darr	|e	|*	|*	|*	|*	|*	|g	|r	|*	|d	|n	|r	|*	|d	|y	|r	|a	|*	|n	|.
|[30][S]\drarr	|d	|r	|o	|m	|o	|n	|a	|*	|u	|*	|ł	|i	|e	|*	|ł	|*	|*	|*	|*	|o	|.
|k	|u	|a	|[31][S]\rarr	|h	|a	|n	|g	|a	|r	|*	|o	|e	|t	|[32][S]\rarr	|o	|t	|o	|k	|*	|ż	|.
|o	|r	|*	|*	|*	|*	|[33][S]\rarr	|ś	|l	|a	|d	|*	|*	|n	|*	|*	|*	|*	|*	|*	|a	|.
|n	|a	|*	|*	|*	|*	|*	|*	|*	|*	|[34][S]\rarr	|c	|h	|o	|r	|i	|j	|a	|m	|b	|*	|.
|t	|l	|[35][S]\rarr	|d	|o	|b	|r	|o	|t	|l	|i	|w	|o	|ś	|ć	|*	|*	|*	|*	|*	|*	|.
|o	|*	|*	|[36][S]\rarr	|s	|u	|b	|t	|e	|l	|n	|o	|ś	|ć	|*	|*	|*	|*	|*	|*	|*	|.
|*	|*	|*	|*	|*	|[37][S]\rarr	|p	|r	|z	|y	|m	|u	|s	|*	|*	|*	|*	|*	|*	|*	|*	|.\end{Puzzle}

\newpage

\begin{PuzzleClues}{\textbf{Poziome}\\}\Clue{1}{}{jezioro w Federacji Rosyjskiej przez jezioro przepływa Onega}
\Clue{2}{}{przypuszczenie, informacja (cynk), że jakieś przedsięwzięcie zakończy się właśnie w dany sposób, zazwyczaj gdy mowa o zakładach}
\Clue{3}{}{kobieta o święceniach biskupich, sprawująca funkcje biskupa w niektórych kościołach, np. metodystycznych, starokatolickich, niektórych anglikańskich, luterańskich, kalwińskich, a także w Kościele Adwentystów Dnia Siódmego}
\Clue{4}{}{centuria czerwona, tysięcznik czerwony, Centaurium pulchellum - gatunek z rodziny goryczkowatych (Gentianaceae)}
\Clue{5}{}{żołnierski ekwipunek}
\Clue{6}{}{przenośnie: obszar, środowisko, które budzi poczucie bezpieczeństwa}
\Clue{8}{}{wydzielone miejsce w meczecie przeznaczone dla władcy}
\Clue{10}{}{w języku młodzieżowym: dzielnica, najbliższe otoczenie}
\Clue{15}{}{krocze, część ciała pomiędzy nogami}
\Clue{18}{}{wielokrotność wolta (V) - jednostki potencjału elektrycznego, napięcia elektrycznego i siły elektromotorycznej; 1 GV = 10E+9 V}
\Clue{25}{}{Chelictinia riocourii - gatunek ptaka drapieżnego z rodziny jastrzębiowatych (Accipitridae), z podrodziny kaniuków (Elaninae); jedyny przedstawiciel rodzaju Chelictinia}
\Clue{26}{}{ur. 1937 r., dyrygent włoski, prowadził paryską 'Grand Opera'}
\Clue{27}{}{w piłce nożnej lub tenisie: przerzut piłki na przeciwległą część boiska lub kortu}
\Clue{30}{}{wojenny, wiosłowy okręt bizantyjski uzbrojony w taran}
\Clue{31}{}{portowy budynek do składowania towarów}
\Clue{32}{}{rzemień do prowadzenia psa myśliwskiego}
\Clue{33}{}{przen. to, co jest pozostałością po wpływie jakiegoś czynnika (ale nie fizycznego) na kogoś/coś}
\Clue{34}{}{stopa metryczna sześciomorowa, która składa się z jambu i choreja}
\Clue{35}{}{cecha osób kierujących się życzliwością, pomocnością, otwartością}
\Clue{36}{}{cecha czegoś, co jest mało wyraziste, o słabej intensywności}
\Clue{37}{}{niezbędność dla dalszego działania, konieczność}\end{PuzzleClues}

\begin{PuzzleClues}{\textbf{Pionowe}\\}\Clue{1}{}{rodzaj pochodzącej z Włoch przekąski, grzanka posmarowana oliwą i roztartym czosnkiem}
\Clue{2}{}{południk, od 1984 określony przez Ministerstwo Obrony Stanów Zjednoczonych na podstawie globalnej geodezyjnej siatki geograficznej WGS 84}
\Clue{6}{}{kod ISO 4217 riala saudyjskiego}
\Clue{7}{}{głowa zwierzęcia, szczególnie dużego}
\Clue{8}{}{coś, co sprawia trudność, utrudnia przejście przez jakiś problem, opóźnia coś, powoduje niedopełnienie czegoś}
\Clue{9}{}{dorosła samica niektórych innych - niezaliczających się do bydła - ssaków (przeżuwaczy, parzystokopytnych)}
\Clue{10}{}{to, że ktoś coś lub kogoś przechowuje, ukrywa}
\Clue{11}{}{mieszkaniec Francji, człowiek pochodzenia francuskiego}
\Clue{12}{}{skrót/symbol funta egipskiego}
\Clue{13}{}{członek hakaty, niemieckiej organizacji nacjonalistycznej}
\Clue{14}{}{klawiatura do wpisywania emotikonów}
\Clue{16}{}{włoski poeta i kompozytor; opery i libretta operowe}
\Clue{17}{}{pisarz niemiecki (1752-1831), dramaty, powieści}
\Clue{19}{}{lekkoatleta argentyński, mistrz olimpijski z Londynu w maratonie}
\Clue{20}{}{miasto w Republice Środkowoafrykańskiej; węzeł kolejowy}
\Clue{21}{}{narzędzie medyczne stosowane w przyżeganiu termicznym (kauteryzacji termicznej) - zabiegu polegającym na koagulowaniu żywej, patologicznie zmienionej tkanki, wykonywanym najczęściej na powierzchni błony śluzowej czy skóry, a także w celu przyspieszenia gojenia ziarninujących ran}
\Clue{22}{}{czas lęgu zwierząt}
\Clue{23}{}{jezioro w Irlandii, przez Derg przepływa rzeka Shannon}
\Clue{24}{}{właściwość działań, sytuacji, stanów itp., polegająca na byciu sekretnym (utajonym, niejawnym)}
\Clue{28}{}{kobieta, postrzegana pod względem seksualnym, zwłaszcza rozwiązła, mająca wielu partnerów}
\Clue{29}{}{skrótowe określenie duraluminium}
\Clue{30}{}{zbiór dokonań, zarówno pozytywnych (szczególnie zawodowych, sportowych, arytystycznych), jak i negatywnych, np. konfliktów z prawem}\end{PuzzleClues}\newpage\section*{Krzyżówka 51}

\noindent\begin{Puzzle}{20}{24}|*	|*	|*	|*	|[1][S]\drarr	|b	|o	|o	|g	|i	|e	|[][S]-	|w	|o	|o	|g	|i	|e	|*	|*	|*	|.
|*	|*	|*	|*	|s	|*	|[2][S]\darr	|*	|*	|*	|*	|*	|*	|*	|*	|*	|[3][S]\darr	|[4][S]\darr	|[5][S]\darr	|*	|*	|.
|*	|*	|*	|*	|a	|*	|s	|*	|[6][S]\darr	|*	|*	|*	|*	|*	|*	|*	|i	|f	|p	|[7][S]\darr	|[8][S]\darr	|.
|*	|*	|[9][S]\darr	|*	|l	|*	|v	|[10][S]\rarr	|s	|n	|i	|k	|e	|r	|s	|*	|m	|e	|i	|f	|n	|.
|*	|[11][S]\rarr	|s	|ł	|o	|n	|e	|c	|z	|n	|i	|c	|a	|*	|*	|*	|i	|z	|ę	|i	|i	|.
|*	|*	|s	|*	|p	|*	|v	|*	|m	|[12][S]\rarr	|s	|z	|e	|w	|i	|o	|t	|*	|c	|l	|e	|.
|[13][S]\drarr	|w	|a	|r	|a	|n	|o	|z	|a	|u	|r	|*	|[14][S]\darr	|*	|*	|[15][S]\darr	|a	|*	|i	|o	|r	|.
|c	|*	|k	|[16][S]\darr	|*	|*	|*	|[17][S]\darr	|c	|[18][S]\drarr	|s	|t	|a	|r	|*	|w	|c	|*	|o	|l	|u	|.
|z	|*	|i	|s	|[19][S]\darr	|*	|*	|c	|i	|d	|*	|*	|p	|[20][S]\rarr	|k	|i	|j	|*	|b	|o	|c	|.
|a	|*	|*	|k	|s	|*	|*	|z	|a	|y	|[21][S]\rarr	|t	|r	|*	|*	|z	|a	|*	|ó	|g	|h	|.
|r	|*	|[22][S]\darr	|ó	|m	|*	|[23][S]\drarr	|e	|k	|s	|p	|l	|o	|z	|j	|a	|*	|*	|j	|i	|o	|.
|n	|[24][S]\drarr	|f	|r	|a	|j	|e	|r	|*	|t	|*	|*	|s	|*	|*	|[][,]{ }	|*	|*	|[][,]{ }	|a	|m	|.
|y	|m	|o	|a	|k	|*	|l	|w	|[25][S]\drarr	|r	|o	|z	|z	|i	|e	|w	|*	|*	|a	|[][,]{ }	|o	|.
|[][,]{ }	|e	|r	|*	|*	|[26][S]\rarr	|k	|o	|r	|o	|w	|i	|e	|c	|*	|y	|[27][S]\darr	|[28][S]\darr	|n	|h	|ś	|.
|m	|n	|d	|[29][S]\darr	|[30][S]\darr	|*	|o	|n	|a	|f	|*	|*	|*	|*	|*	|j	|a	|a	|t	|i	|ć	|.
|a	|d	|*	|s	|w	|[31][S]\darr	|*	|i	|j	|e	|*	|*	|*	|[32][S]\drarr	|k	|a	|p	|r	|y	|s	|*	|.
|r	|e	|*	|y	|i	|n	|[33][S]\darr	|e	|z	|z	|*	|*	|*	|k	|[34][S]\darr	|z	|s	|t	|c	|z	|*	|.
|s	|*	|*	|b	|e	|a	|z	|c	|e	|*	|*	|*	|*	|o	|z	|d	|y	|y	|z	|p	|*	|.
|z	|[35][S]\rarr	|v	|a	|l	|l	|a	|*	|r	|*	|*	|*	|*	|d	|a	|o	|d	|s	|n	|a	|*	|.
|*	|*	|[36][S]\darr	|r	|k	|u	|g	|*	|*	|*	|*	|*	|*	|*	|m	|w	|a	|t	|y	|ń	|*	|.
|[37][S]\drarr	|ś	|w	|i	|a	|t	|ł	|o	|[][,]{ }	|m	|i	|j	|a	|n	|i	|a	|*	|k	|*	|s	|*	|.
|i	|*	|z	|s	|n	|*	|a	|[38][S]\drarr	|f	|e	|l	|i	|c	|i	|a	|*	|*	|a	|*	|k	|*	|.
|s	|*	|ó	|*	|o	|*	|d	|z	|*	|[39][S]\rarr	|g	|e	|h	|e	|n	|n	|a	|*	|*	|a	|*	|.
|e	|*	|r	|[40][S]\rarr	|c	|h	|a	|r	|a	|k	|t	|e	|r	|*	|a	|[41][S]\rarr	|m	|r	|o	|*	|*	|.
|*	|*	|*	|*	|*	|*	|*	|*	|[42][S]\rarr	|d	|u	|d	|e	|k	|*	|*	|*	|*	|*	|*	|*	|.\end{Puzzle}

\newpage

\begin{PuzzleClues}{\textbf{Poziome}\\}\Clue{1}{}{styl gry na fortepianie}
\Clue{10}{}{bardzo słodkie ciasto z karmelem, orzechami i czekoladą (przywodzące na myśl znany baton - Snickers)}
\Clue{11}{}{Eurypyga helias - gatunek dużego ptaka, będącego jedynym przedstawicielem rodziny słonecznic (Eurypygidae) w rzędzie słonecznicowych (Eurypygiformes), który zamieszkuje strefę międzyzwrotnikową Ameryki kontynentalnej - od Meksyku po Brazylię}
\Clue{12}{}{tkanina ubraniowa, z owczej wełny, która nazwę zawdzięcza szkockiej rasie owiec Cheviot}
\Clue{13}{}{Varanosaurus - nazwa rodzajowa drapieżnego pelykozaura, żyjącego w permie (286-260 milionów lat temu); jego szczątki odkryto na terenie Teksasu}
\Clue{18}{}{klasa balastowego jachtu żaglowego}
\Clue{20}{}{kawałek drewna o cylindrycznym kształcie, o różnej długości i grubości, posiadający dwa lub więcej końców (rozwidlenia), który został ułamany lub obcięty z drzewa, krzewu, trzciny lub trawy (bambus), ewentualnie wystrugany z drewna}
\Clue{21}{}{tor, milimetr słupa rtęci - pozaukładowa jednostka miary ciśnienia, równa ciśnieniu słupa rtęci o wysokości jednego milimetra w temperaturze 273,15 K (0 °C) przy normalnym przyspieszeniu ziemskim}
\Clue{23}{}{dużo silnych, nagłych wrażeń}
\Clue{24}{}{nowicjusz w jakiejś dziedzinie}
\Clue{25}{}{brak proporcji, równowagi, symetrii między rzeczami lub zjawiskami, które występują równocześnie lub łącznie}
\Clue{26}{}{ROZWAŁKA; drobny pluskwiak różnoskrzydły żyjący na sośnie wysysa soki i powoduje pękanie kory}
\Clue{32}{}{utwór instrumentalny, polifoniczny, oparty na technice imitacyjnej, mający swobodną budowę, pogodny charakter}
\Clue{35}{}{humanista włoski (1407-1457); zajmował się filozofia filologia historiografią}
\Clue{37}{}{w komunikacji drogowej: żółte światło, o niewielkim zasięgu, używane do zwykłej jazdy}
\Clue{38}{}{skoda z modelu Felicia}
\Clue{39}{}{w chrześcijaństwie miejsce przebywania osób odrzuconych przez Boga oraz cierpiących katusze w niegasnącym ogniu}
\Clue{40}{}{zespół cech psychicznych najczęściej człowieka (także zwierzęcia)}
\Clue{41}{}{kod ISO 4217 waluty ugija}
\Clue{42}{}{ptak z rzędu kraskowatych z rozkładanym czubem na głowie, dziuplak, chroniony}\end{PuzzleClues}

\begin{PuzzleClues}{\textbf{Pionowe}\\}\Clue{1}{}{długie, obszerne wierzchnie okrycie kobiece z rękawami i pelerynką, zwykle watowane lub podbite futrem, noszone w XVIII i XIX w}
\Clue{2}{}{(1861-1928), pisarz włoski, opisywał życie społeczeństwa triesteńskiego; „Jedno życie”, „Zeno Cosini”}
\Clue{3}{}{rzecz, która jest podobna, ma naśladować inną rzecz}
\Clue{4}{}{miasto w Maroku, dawna stolica kraju, ok. 426 tys. mieszkańców}
\Clue{5}{}{dyscyplina sportu w starożytnej Grecji obejmująca bieg na jeden stadion olimpijski, skok w dal, rzut dyskiem, rzut oszczepem oraz zapasy}
\Clue{6}{}{szmaciany but}
\Clue{7}{}{kierunek studiów, który swoim zakresem obejmuje naukę o języku, kulturze i literaturze hiszpańskiej}
\Clue{8}{}{stan, kiedy nie ma ruchu, bezruch, np. trwać przez chwilę w nieruchomości}
\Clue{9}{}{najwyżej uorganizowane kręgowce stałocieplne}
\Clue{13}{}{marsz protestacyjny}
\Clue{14}{}{PRZYKOPY}
\Clue{15}{}{dokument uprawniający swojego posiadacza do wyjazdu z terenu jakiegoś państwa lub, rzadziej, innej jednostki terytorialnej}
\Clue{16}{}{powłoka ciał zwierząt i ludzi}
\Clue{17}{}{moneta złota, floren, dukat, odróżniająca monetę obiegową (ok. 3,5 g złota) od ówczesnej jednostki obrachunkowej, czyli złotego polskiego}
\Clue{18}{}{Dystrophaeus - rodzaj zauropoda z rodziny diplodoków; żył w epoce późnej jury na terenach Ameryki Północnej}
\Clue{19}{}{upodobanie do czegoś}
\Clue{22}{}{marka samochodu; amerykański koncern motoryzacyjny założony 16 czerwca 1903 roku przez Henry'ego Forda w Detroit}
\Clue{23}{}{miasto w Stanach Zjednoczonych w stanie Nevada}
\Clue{24}{}{miasto we Francji nad rzeką Lot (Masyw Centralny)}
\Clue{25}{}{osoba, która nieustannie zmienia miejsce pobytu}
\Clue{27}{}{punkt największego przybliżenia lub oddalenia orbity jednego ciała niebieskiego w stosunku do drugiego}
\Clue{28}{}{kobieta zajmująca się na co dzień określoną dziedziną sztuki}
\Clue{29}{}{w starożytności kolonia grecka założona w południowej Italii}
\Clue{30}{}{Niedziela Wielkanocna - najstarsze i najważniejsze święto chrześcijańskie upamiętniające zmartwychwstanie Jezusa Chrystusa, obchodzone przez Kościoły chrześcijańskie wyznające Nicejskie Credo (325 r.)}
\Clue{31}{}{miasto w Libii w pobliżu granicy z Tunezją}
\Clue{32}{}{ciąg składników sygnału (kombinacji sygnałów elementarnych, np. kropek i kresek, impulsów prądu, symboli) oraz reguła ich przyporządkowania składnikom wiadomości (np. znakom pisma)}
\Clue{33}{}{absolutne zniszczenie, unicestwienie}
\Clue{34}{}{przekształcenie się, przejście z jednej formy lub postaci w inną}
\Clue{36}{}{rysunek lub projekt w innej postaci służący do odtworzenia czegoś, zrobiony po to, by powielać jakiś kształt}
\Clue{37}{}{miasto w Japonii na wyspie Honsiu; hodowla perłopławów, przemysł włókienniczy i spożywczy}
\Clue{38}{}{w chemii: symbol cyrkonu}\end{PuzzleClues}\newpage\section*{Krzyżówka 52}

\noindent\begin{Puzzle}{22}{27}|*	|*	|*	|*	|*	|*	|*	|*	|[1][S]\drarr	|s	|t	|r	|ó	|ż	|ó	|w	|k	|a	|*	|*	|*	|[2][S]\darr	|[3][S]\darr	|.
|*	|[4][S]\darr	|*	|[5][S]\darr	|*	|[6][S]\rarr	|p	|a	|d	|e	|m	|e	|l	|o	|n	|*	|*	|*	|*	|*	|*	|d	|z	|.
|[7][S]\drarr	|g	|r	|a	|n	|a	|t	|*	|u	|*	|*	|*	|*	|*	|*	|*	|*	|*	|*	|*	|[8][S]\darr	|i	|r	|.
|g	|u	|*	|r	|*	|*	|[9][S]\drarr	|a	|m	|u	|n	|i	|c	|j	|a	|*	|*	|*	|*	|*	|c	|i	|ą	|.
|a	|e	|[10][S]\rarr	|d	|ź	|w	|i	|g	|a	|r	|*	|*	|*	|*	|*	|*	|*	|*	|*	|*	|f	|k	|b	|.
|z	|v	|[11][S]\darr	|e	|*	|*	|c	|*	|s	|*	|*	|[12][S]\rarr	|m	|e	|l	|o	|d	|r	|a	|m	|a	|t	|*	|.
|e	|a	|m	|n	|*	|*	|h	|*	|*	|[13][S]\drarr	|p	|o	|d	|p	|a	|l	|e	|n	|i	|e	|*	|o	|*	|.
|t	|r	|e	|*	|*	|*	|t	|[14][S]\rarr	|s	|k	|l	|a	|w	|i	|n	|o	|w	|i	|e	|*	|*	|d	|*	|.
|a	|a	|r	|*	|[15][S]\drarr	|w	|i	|e	|l	|o	|f	|a	|z	|o	|w	|o	|ś	|ć	|*	|*	|*	|o	|*	|.
|*	|*	|c	|*	|j	|*	|o	|*	|[16][S]\rarr	|t	|r	|a	|n	|z	|y	|s	|t	|o	|r	|*	|*	|n	|*	|.
|*	|*	|e	|*	|e	|*	|f	|[17][S]\rarr	|r	|e	|s	|p	|o	|n	|s	|y	|w	|n	|o	|ś	|ć	|*	|*	|.
|*	|*	|d	|*	|ż	|*	|a	|[18][S]\rarr	|k	|r	|ó	|l	|*	|*	|*	|*	|*	|*	|[19][S]\darr	|*	|[20][S]\darr	|*	|*	|.
|*	|[21][S]\drarr	|e	|w	|o	|l	|u	|c	|j	|a	|[][,]{ }	|n	|a	|r	|c	|i	|a	|r	|s	|k	|a	|*	|[22][S]\darr	|.
|*	|d	|s	|[23][S]\darr	|w	|*	|n	|[24][S]\darr	|[25][S]\rarr	|p	|i	|n	|g	|w	|i	|n	|*	|*	|z	|*	|p	|*	|b	|.
|*	|w	|*	|p	|i	|[26][S]\darr	|a	|w	|*	|e	|*	|*	|*	|*	|*	|*	|[27][S]\drarr	|w	|a	|g	|a	|*	|o	|.
|*	|a	|*	|r	|e	|m	|*	|i	|*	|u	|*	|[28][S]\darr	|[29][S]\darr	|[30][S]\darr	|*	|*	|s	|*	|ł	|*	|n	|*	|l	|.
|*	|[][,]{ }	|[31][S]\darr	|e	|c	|a	|*	|c	|[32][S]\darr	|t	|*	|t	|k	|n	|*	|*	|t	|*	|o	|*	|a	|*	|i	|.
|*	|o	|l	|a	|*	|g	|[33][S]\drarr	|e	|l	|a	|b	|o	|r	|a	|c	|j	|a	|*	|t	|*	|ż	|[34][S]\darr	|m	|.
|[35][S]\drarr	|g	|e	|m	|i	|n	|i	|d	|y	|*	|*	|k	|z	|s	|*	|*	|r	|*	|*	|*	|e	|t	|o	|.
|k	|n	|w	|b	|*	|e	|z	|z	|n	|[36][S]\darr	|*	|a	|a	|a	|[37][S]\rarr	|n	|a	|j	|e	|ż	|*	|r	|w	|.
|a	|i	|a	|u	|*	|t	|b	|i	|c	|l	|*	|j	|n	|d	|*	|*	|[][,]{ }	|*	|*	|*	|[38][S]\darr	|y	|i	|.
|m	|e	|r	|l	|*	|y	|a	|e	|h	|e	|*	|*	|o	|k	|[39][S]\rarr	|s	|k	|a	|j	|*	|d	|m	|a	|.
|e	|*	|e	|u	|*	|k	|*	|k	|*	|m	|[40][S]\rarr	|k	|w	|a	|z	|a	|r	|*	|*	|*	|u	|e	|n	|.
|d	|*	|k	|m	|*	|*	|[41][S]\rarr	|a	|n	|a	|n	|a	|s	|*	|*	|*	|y	|*	|*	|*	|c	|r	|k	|.
|u	|*	|*	|*	|*	|*	|*	|n	|*	|n	|[42][S]\rarr	|s	|k	|r	|z	|y	|p	|ł	|o	|c	|z	|*	|a	|.
|ł	|*	|*	|*	|*	|*	|*	|*	|*	|*	|*	|*	|i	|*	|[43][S]\rarr	|b	|a	|r	|a	|n	|e	|k	|*	|.
|*	|*	|[44][S]\rarr	|p	|s	|i	|e	|[][,]{ }	|p	|o	|l	|e	|*	|*	|*	|*	|*	|*	|*	|*	|k	|*	|*	|.
|*	|*	|*	|*	|*	|*	|*	|*	|*	|*	|*	|*	|*	|*	|*	|*	|*	|*	|*	|*	|*	|*	|*	|.\end{Puzzle}

\newpage

\begin{PuzzleClues}{\textbf{Poziome}\\}\Clue{1}{}{miejsce zamieszkania stróża, dozorcy}
\Clue{6}{}{pademelon czerwonoszyi, Thylogale thetis - gatunek torbacza z rodziny kangurowatych; zamieszkuje wilgotne zarośla i lasy w Australii oraz wschodniej Nowej Zelandii}
\Clue{7}{}{Punica granatum, granatowiec - gatunek drzewa lub ciernistego krzewu należący do rodziny krwawnicowatych}
\Clue{9}{}{pociski, granaty, bomby, miny, rakiety, itp.; ogół środków bojowych}
\Clue{10}{}{poziomy lub pochyły element konstrukcyjny; rozróżniamy dźwigary stropowe i mostowe}
\Clue{12}{}{gatunek powieści, nasycony patetyczno-sentymentalnymi efektami oraz wątkiem miłosnym}
\Clue{13}{}{spowodować zapalenie się czegoś, rozniecić ogień}
\Clue{14}{}{zachodni odłam Słowian; ludy zamieszkujące w V -VII obszary na przedpolu gór - na wyżynach i nizinach od środkowej i górnej Łaby po środkowy Dniepr}
\Clue{15}{}{cecha budowy czegoś, co jest wielofazowe, tj. zbudowane w sposób niejednorodny, z ciał w różnych stanach skupienia}
\Clue{16}{}{trój- lub czteroelektrodowy, półprzewodnikowy element elektroniczny, służący do wzmacniania sygnału elektrycznego}
\Clue{17}{}{postawa odpowiadania zachowaniem na działania drugiej osoby}
\Clue{18}{}{tytuł osoby sprawującej najwyższą władzę w państwie o ustroju monarchicznym}
\Clue{21}{}{trudna figura, skręt czy akrobacja wykonywana przez narciarza z użyciem sprzętu narciarskiego - nart, a także kijków}
\Clue{25}{}{nielatający ptak przystosowany do środowiska wodnego, wiosłowate skrzydła, palce spięte błoną, upierzenie gęste, łuskowate}
\Clue{27}{}{przyrząd pomiarowy służący do wyznaczania masy ciał na zasadzie równoważenia sił lub wykorzystania zjawisk fizycznych}
\Clue{33}{}{zespół czynności, mających na celu przygotowanie rakiet do użycia}
\Clue{35}{}{rój meteorów o najkrótszym obiegu wokół Słońca}
\Clue{37}{}{Diodon holocanthus - ryba występująca niemal we wszystkich tropikalnych strefach mórz, głównie wzdłuż zachodniej strefy Atlantyku}
\Clue{39}{}{sztuczne tworzywo imitujące skórę używane do wyrobu przedmiotów galanteryjnych, wierzchów obuwia itp}
\Clue{40}{}{NIBYGWIAZDA; obiekt pozagalaktyczny o niemal punktowym obrazie}
\Clue{41}{}{gagatek, numer, ptaszek, hultaj}
\Clue{42}{}{MIECZOGON}
\Clue{43}{}{zdrobniale: baran - samiec owcy}
\Clue{44}{}{dzielnica Wrocławia}\end{PuzzleClues}

\begin{PuzzleClues}{\textbf{Pionowe}\\}\Clue{1}{}{chemik francuski (1800-84); metoda oznaczania azotu w związkach organicznych}
\Clue{2}{}{Diictodon - nazwa rodzajowa terapsyda należącego do dicynodontów, żyjącego w późnym permie (około 255 milionów lat temu); jego szczątki odkryto na terenie Afryki i Azji}
\Clue{3}{}{główna część kostrukcji budynków z drewna}
\Clue{4}{}{(1480-1545), pisarz hiszpański, franciszkanin}
\Clue{5}{}{koń ardeński - rasa zimnokrwistych koni, wywodząca się z górzystej prowincji Ardennes, na pograniczu Belgii i Francji}
\Clue{7}{}{grupa ludzi, którzy tworzą gazetę, zespół zajmujący się wydawaniem gazety}
\Clue{8}{}{symbol, skrótowa nazwa franka CFA}
\Clue{9}{}{ogół gatunków ryb zbiornika wodnego, rzeki lub ich części}
\Clue{11}{}{Mercedes-Benz - znana ekskluzywna marka samochodowa}
\Clue{13}{}{teraputa pracujący w zespole terapeutów}
\Clue{15}{}{jeż morski - przedstawiciel gromady morskich zwierząt zaliczanych do typu szkarłupni (Echinodermata), charakteryzujących się kulistym, mniej lub bardziej spłaszczonym, różnorodnie ubarwionym ciałem gęsto pokrytym wapiennymi, ruchomo osadzonymi kolcami; zwierzęta te zamieszkują strefę denną wód słonych o zasoleniu powyżej 20‰ wszystkich stref geograficznych kuli ziemskiej}
\Clue{19}{}{potrawa sporządzana na zimno z pokrojonego śledzia oraz ugotowanych ziemniaków, jaj, ogórka kiszonego, cebuli, skwarek boczku wędzonego i przypraw z dodatkiem musztardy}
\Clue{20}{}{dochód, najczęściej wysoki, związany z piastowanym stanowiskiem}
\Clue{21}{}{sytuacja, z której trudno się uwolnić, taka, w której ktoś jest atakowany z dwóch stron}
\Clue{22}{}{mieszkanka Bolimowa, kobieta pochodząca z Bolimowa}
\Clue{23}{}{dłuższe, bardziej rozbudowane preludium}
\Clue{24}{}{zastępca dziekana - przewodniczącego rady adwokackiej}
\Clue{26}{}{materiał biorący udział w odziaływaniach magnetycznych}
\Clue{27}{}{pejoratywnie o starej, przeciekającej łodzi}
\Clue{28}{}{miejscowość w płn.-wsch. Węgrzech, znany ośrodek regionu uprawy winorośli i produkcji wina}
\Clue{29}{}{kompozytor i akordeonista (1951-1990); utwory orkiestrowe, kameralne, akordeonowe; 'Alkagran'}
\Clue{30}{}{wymienna część palnika laboratoryjnego lub palnika do spawania, którą zakłada się na rękojeść palnika}
\Clue{31}{}{zdrobniale: lewar - dźwignia, która przeznaczona jest do podnoszenia ciężkich rzeczy}
\Clue{32}{}{(1885-1951), pisarz argentyński, realistyczne powieści z życia gauchos}
\Clue{33}{}{organizacja zawodowa, zrzeszająca przestawicieli danego zawodu, starająca się regulować sprawy z nim związane, bronić interesów tej grupy zawodowej lub dbająca o ich interesy w inny sposób}
\Clue{34}{}{dodatkowa płetwa sterowa w jachcie ułatwiająca utrzymanie go na kursie}
\Clue{35}{}{członek katolickiego zakonu kontemplacyjnego założonego w 1012 r. jako reformowany odłam benedyktynów, o surowej regule}
\Clue{36}{}{GENEWSKIE Jezioro; jezioro na granicy szwajcarsko-francuskiej, największe w Alpach, powierzchnia 581,3 km2, głębokość do 310 m}
\Clue{38}{}{ironicznie o młodej osobie noszącej francuski tytuł arystokratyczny najwyższej klasy}\end{PuzzleClues}\newpage\section*{Krzyżówka 53}

\noindent\begin{Puzzle}{21}{29}|*	|*	|[1][S]\drarr	|v	|*	|[2][S]\drarr	|c	|z	|a	|r	|n	|y	|[][,]{ }	|p	|i	|o	|t	|r	|u	|ś	|*	|[3][S]\darr	|.
|*	|[4][S]\rarr	|r	|e	|a	|k	|t	|o	|r	|[][,]{ }	|p	|r	|ę	|d	|k	|i	|*	|*	|*	|[5][S]\darr	|*	|b	|.
|*	|*	|a	|[6][S]\rarr	|m	|o	|d	|r	|a	|s	|z	|k	|i	|*	|*	|[7][S]\darr	|*	|[8][S]\darr	|[9][S]\darr	|ł	|*	|ą	|.
|*	|*	|d	|[10][S]\rarr	|w	|ś	|c	|i	|e	|k	|ł	|o	|ś	|ć	|*	|d	|[11][S]\darr	|n	|b	|o	|*	|k	|.
|*	|*	|*	|*	|*	|ć	|*	|*	|*	|*	|[12][S]\rarr	|m	|o	|d	|g	|i	|l	|i	|a	|n	|i	|*	|.
|*	|[13][S]\drarr	|s	|ą	|d	|[][,]{ }	|w	|o	|j	|s	|k	|o	|w	|y	|*	|e	|u	|e	|r	|o	|*	|*	|.
|*	|ż	|[14][S]\rarr	|p	|a	|s	|s	|e	|[][S]-	|p	|a	|r	|t	|o	|u	|t	|*	|k	|i	|*	|[15][S]\darr	|[16][S]\darr	|.
|*	|ó	|*	|*	|[17][S]\rarr	|k	|a	|w	|i	|a	|r	|k	|a	|*	|*	|a	|*	|o	|e	|*	|s	|p	|.
|*	|ł	|*	|[18][S]\rarr	|u	|r	|o	|l	|o	|g	|i	|a	|*	|*	|*	|[][,]{ }	|*	|n	|r	|[19][S]\darr	|i	|r	|.
|[20][S]\rarr	|w	|ę	|g	|l	|o	|w	|o	|d	|a	|n	|*	|*	|*	|*	|c	|*	|w	|a	|m	|n	|o	|.
|*	|[][,]{ }	|[21][S]\drarr	|j	|e	|n	|e	|r	|a	|ł	|*	|[22][S]\drarr	|m	|i	|l	|u	|*	|e	|[][,]{ }	|u	|d	|i	|.
|*	|z	|p	|[23][S]\drarr	|p	|i	|l	|ź	|n	|i	|a	|n	|k	|a	|*	|d	|*	|n	|k	|s	|i	|n	|.
|*	|ą	|i	|p	|*	|o	|[24][S]\rarr	|o	|k	|u	|p	|a	|c	|j	|a	|*	|*	|c	|r	|z	|n	|n	|.
|*	|b	|ł	|o	|*	|w	|[25][S]\rarr	|t	|y	|p	|[][,]{ }	|d	|z	|i	|k	|i	|*	|j	|e	|k	|g	|o	|.
|[26][S]\drarr	|k	|a	|m	|p	|a	|n	|i	|a	|[][,]{ }	|w	|r	|z	|e	|ś	|n	|i	|o	|w	|a	|*	|w	|.
|t	|o	|*	|i	|*	|*	|*	|*	|*	|*	|*	|e	|*	|*	|[27][S]\darr	|[28][S]\darr	|*	|n	|[][S]-	|*	|*	|a	|.
|y	|w	|*	|e	|*	|*	|*	|*	|[29][S]\darr	|*	|*	|p	|*	|[30][S]\darr	|p	|b	|*	|a	|m	|*	|[31][S]\darr	|c	|.
|s	|a	|*	|s	|[32][S]\rarr	|l	|u	|k	|s	|f	|e	|r	|*	|a	|o	|e	|*	|l	|ó	|*	|p	|y	|.
|i	|n	|*	|z	|[33][S]\drarr	|h	|e	|n	|a	|n	|*	|e	|*	|n	|g	|z	|*	|n	|z	|*	|r	|j	|.
|ę	|y	|*	|a	|o	|[34][S]\drarr	|s	|e	|r	|*	|*	|z	|*	|t	|a	|a	|*	|o	|g	|[35][S]\darr	|z	|n	|.
|c	|*	|*	|n	|b	|w	|[36][S]\drarr	|l	|i	|n	|n	|e	|*	|y	|w	|n	|[37][S]\darr	|ś	|*	|d	|e	|o	|.
|z	|*	|*	|y	|e	|y	|b	|*	|*	|*	|*	|n	|*	|p	|ę	|m	|p	|ć	|[38][S]\darr	|a	|s	|ś	|.
|n	|*	|*	|*	|j	|r	|e	|*	|*	|*	|*	|t	|*	|a	|d	|a	|r	|*	|s	|n	|ł	|ć	|.
|i	|*	|*	|*	|m	|y	|t	|[39][S]\rarr	|l	|w	|i	|a	|[][,]{ }	|p	|a	|s	|z	|c	|z	|k	|a	|*	|.
|k	|*	|*	|[40][S]\drarr	|a	|p	|e	|r	|c	|e	|p	|c	|j	|a	|*	|z	|e	|*	|l	|o	|n	|*	|.
|*	|*	|*	|o	|*	|a	|l	|*	|[41][S]\rarr	|h	|a	|j	|s	|*	|*	|t	|s	|*	|i	|w	|k	|*	|.
|*	|*	|*	|h	|*	|*	|*	|*	|[42][S]\rarr	|b	|ł	|a	|z	|e	|n	|*	|y	|*	|f	|i	|a	|*	|.
|*	|*	|*	|i	|[43][S]\rarr	|ś	|c	|i	|a	|n	|a	|*	|[44][S]\rarr	|l	|a	|m	|p	|y	|*	|c	|*	|*	|.
|*	|[45][S]\rarr	|p	|o	|d	|h	|a	|l	|a	|ń	|c	|z	|y	|k	|*	|*	|*	|*	|*	|e	|*	|*	|.
|*	|*	|*	|*	|*	|*	|*	|*	|*	|*	|*	|*	|*	|*	|*	|*	|*	|*	|*	|*	|*	|*	|.\end{Puzzle}

\newpage

\begin{PuzzleClues}{\textbf{Poziome}\\}\Clue{1}{}{w chemii: symbol wanadu}
\Clue{2}{}{czynnik przyczyniający się do przegranej, przynoszący pecha w rozgrywce; najczęściej w sporcie}
\Clue{4}{}{reaktor, w którym nie ma moderatora, ponieważ reakcje rozszczepienia wywoływane są przez neutrony prędkie}
\Clue{6}{}{Polyommatinae - podrodzina motyli z rodziny modraszkowatych}
\Clue{10}{}{wielka, silna złość, ogromne zdenerwowanie}
\Clue{12}{}{SURREALIZM}
\Clue{13}{}{jeden z rodzajów sądownictwa szczególnego, czyli sądów wyłączonych z sądownictwa powszechnego ze względu na szczególne kategorie podmiotowe (sprawcy) lub przedmiotowe (określone sprawy)}
\Clue{14}{}{kartonowa ramka z wycięciem o powierzchni trochę mniejszej od powierzchni karty z ilustracją, która umieszczana jest najczęściej w ramie obrazu}
\Clue{17}{}{ciśnieniowy czajniczek do parzenia kawy}
\Clue{18}{}{gałąź chirurgii zajmująca się budową, fizjologią i schorzeniami układu układu moczowo-płciowego u mężczyzn oraz układu moczowego u kobiet, a częściowo również żeńskim układem płciowym}
\Clue{20}{}{organiczny związek chemiczny składający się z atomów węgla oraz wodoru i tlenu}
\Clue{21}{}{stopień wojskowy generała, ranga generała}
\Clue{22}{}{wsch. azjatycki ssak z jeleniowatych- obecnie tylko w zoo}
\Clue{23}{}{mieszkanka Pilzna - miasta w Czechach}
\Clue{24}{}{okres okupacji - zajęcia terytorium państwa}
\Clue{25}{}{reprezentatywna, zdrowa forma organizmu, szczepu, genu, genotypu lub fenotypu - taka, jaka po raz pierwszy została zaobserwowana w naturze}
\Clue{26}{}{obrona terytorium Polski przed agresją militarną wojsk III Rzeszy  i ZSRR (Armia Czerwona); pierwszy etap II wojny światowej}
\Clue{32}{}{szklana kształtka osadzona w stropie albo ścianie}
\Clue{33}{}{Honan}
\Clue{34}{}{produkt spożywczy wytwarzany poprzez wytrącenie z mleka tłuszczu i białka w postaci skrzepu, który zostaje poddany dalszej obróbce}
\Clue{36}{}{przyrodnik szwedzki (1707-78); twórca współczesnego systemu klasyfikacji organizmów}
\Clue{39}{}{pieszczotliwa nazwa lwiej paszczy}
\Clue{40}{}{termin psychologiczny oznaczający postrzeganie połączone z odnoszeniem treści spostrzeżenia do tego, co już (uprzednio) znane}
\Clue{41}{}{pieniądze, środki materialne, którymi się płaci}
\Clue{42}{}{na dawnych dworach królewskich i wielkopańskich: człowiek  zawodowo bawiący, rozweselający innych}
\Clue{43}{}{w geometrii: ściana powierzchni wielościennej albo wielościanu - jeden z wielokątów, które tworzą jej/jego brzeg}
\Clue{44}{}{łow. oczy wilka}
\Clue{45}{}{członek polskich oddziałów piechoty górskiej - strzelców podhalańskich}\end{PuzzleClues}

\begin{PuzzleClues}{\textbf{Pionowe}\\}\Clue{1}{}{jednostka dawki absorbowanej (dawki pochłoniętej) promieniowania jonizującego; dawka absorbowana przez napromieniowane ciało o masie 1 g, jeżeli energia przekazana temu ciału przez cząstki jonizujące równa jest 100 erg czyli 10-5 J}
\Clue{2}{}{parzysta, silnie spneumatyzowana kość wchodząca w skład mózgoczaszki}
\Clue{3}{}{bryła sztywna mająca możliwość obrotu wokół dowolnej osi stykającej lub ślizgającej się po powierzchni, mogąca wirować wokół środka masy}
\Clue{5}{}{powierzchnia czegoś}
\Clue{7}{}{dieta, która ma przynieść natychmiastowe i spektakularne efekty}
\Clue{8}{}{oryginalność, cecha czegoś, co jest niekonwencjonalne, nieszablonowe, wyróżnia się}
\Clue{9}{}{fizyczna (uwarunkowana strukturą morfologiczną) i biochemiczna (obejmująca procesy biochemiczne zachodzące w obrębie cytoplazmy komórek tworzących barierę) bariera pomiędzy naczyniami krwionośnymi a tkanką nerwową, mającą zabezpieczać układ nerwowy przed szkodliwymi czynnikami, a także umożliwić selektywny transport substancji z krwi do płynu mózgowo-rdzeniowego}
\Clue{11}{}{w chemii: symbol lutetu}
\Clue{13}{}{Psammobates oculiferus - gatunek gada z odziny żółwi lądowych, mający 15 cm długości, występujący w RPA, Namibii, Bostwanie}
\Clue{15}{}{w malarstwie olejnym stopniowe, łagodne przejścia z partii ciemnych do jasnych}
\Clue{16}{}{wspieranie wprowadzania czegoś nowego}
\Clue{19}{}{sztuczna przynęta wędkarska}
\Clue{21}{}{ryba piła, piła zwyczajna, Pristis pristis - gatunek dwuśrodowiskowej ryby chrzęstnoszkieletowej z rodziny piłowatych (Pristidae); ryba ta występuje w zachodnim i wschodnim Oceanie Spokojnym, zachodnim i wschodnim Oceanie Atlantyckim, jest spotykana również w zachodniej części Morza Śródziemnego; wpływa do estuariów, czasami do rzek; w Amazonce jest spotykana 750 km od ujścia; słodkowodna populacja piły występuje w jeziorze Nikaragua}
\Clue{22}{}{fakt, że obiekty danej klasy występują w badanym zbiorze zbyt często, częściej, niż mogłoby to wynikać z ich proporcji w stosunku do całej badanej grupy, populacji}
\Clue{23}{}{ktoś chory psychicznie, wariat}
\Clue{26}{}{impreza, wydarzenie (szczególnie zawody sportowe), w którym udział bierze tysiąc zawodników}
\Clue{27}{}{rozmowa, pogawędka}
\Clue{28}{}{tylny maszt na wielomasztowym żaglowcu z wyjątkiem statków , z ożaglowaniem rejowym}
\Clue{29}{}{pas materiału, długości 5-6 metrów, którego się nie zszywa}
\Clue{30}{}{dawne określenie antypapieża}
\Clue{31}{}{wydarzenie lub okoliczność, która ma wpływ na przebieg i treść postępowania sądowego}
\Clue{33}{}{element mocujący w postaci taśmy stalowej, którą zaciska się wokół mocowanego przedmiotu}
\Clue{34}{}{w języku przewodników górskich i miłośników gór: wycieczka, wędrówka}
\Clue{35}{}{wieś w południowo-centralnej Polsce położona w województwie śląskim, w powiecie bielskim, w gminie Wilamowice}
\Clue{36}{}{Piper betle - gatunek rośliny z rodziny pieprzowatych; pochodzi z Indii i Półwyspu Malajskiego}
\Clue{37}{}{przesypywanie czegoś, to, że coś się przesypuje, głównie w górnictwie}
\Clue{38}{}{zgład}
\Clue{40}{}{stan w USA, powierzchnia 107 tyś. km2, stolica Columbus, jeden z najważniejszych pod względem gospodarczym stanów USA}\end{PuzzleClues}\newpage\section*{Krzyżówka 54}

\noindent\begin{Puzzle}{22}{25}|*	|*	|*	|*	|*	|*	|[1][S]\drarr	|s	|k	|o	|c	|z	|o	|g	|o	|n	|e	|k	|*	|*	|*	|*	|*	|.
|*	|*	|[2][S]\rarr	|d	|r	|y	|b	|l	|i	|n	|g	|*	|[3][S]\drarr	|p	|o	|k	|ą	|t	|n	|i	|k	|*	|*	|.
|*	|*	|[4][S]\rarr	|l	|o	|k	|a	|t	|a	|[][,]{ }	|t	|e	|r	|m	|i	|n	|o	|w	|a	|*	|*	|*	|*	|.
|*	|*	|*	|*	|*	|*	|s	|[5][S]\rarr	|s	|ł	|o	|w	|e	|n	|k	|a	|*	|*	|*	|*	|*	|*	|[6][S]\darr	|.
|*	|*	|[7][S]\rarr	|w	|ą	|s	|k	|o	|ś	|ć	|*	|*	|z	|*	|[8][S]\darr	|*	|*	|[9][S]\darr	|*	|*	|*	|[10][S]\darr	|w	|.
|*	|*	|[11][S]\darr	|*	|*	|*	|*	|*	|*	|*	|*	|*	|y	|*	|k	|*	|[12][S]\darr	|s	|*	|*	|*	|s	|i	|.
|*	|[13][S]\rarr	|c	|l	|i	|o	|*	|*	|*	|*	|*	|*	|d	|*	|o	|*	|w	|t	|*	|*	|*	|z	|c	|.
|*	|[14][S]\rarr	|z	|e	|s	|t	|a	|w	|i	|e	|n	|i	|e	|*	|n	|*	|i	|a	|*	|[15][S]\darr	|*	|y	|e	|.
|[16][S]\drarr	|l	|o	|g	|*	|[17][S]\rarr	|s	|z	|a	|b	|l	|o	|n	|*	|w	|*	|l	|n	|*	|k	|*	|b	|a	|.
|b	|[18][S]\rarr	|p	|a	|n	|i	|*	|*	|*	|*	|[19][S]\darr	|*	|c	|*	|i	|*	|k	|i	|*	|r	|*	|i	|d	|.
|r	|*	|*	|*	|*	|[20][S]\rarr	|u	|r	|o	|d	|z	|a	|j	|*	|k	|*	|ó	|c	|*	|a	|*	|k	|m	|.
|a	|*	|*	|[21][S]\rarr	|f	|u	|n	|k	|c	|j	|a	|[][,]{ }	|a	|k	|t	|y	|w	|a	|c	|j	|i	|*	|i	|.
|n	|[22][S]\rarr	|g	|i	|m	|n	|a	|z	|j	|a	|d	|a	|*	|*	|*	|*	|*	|*	|*	|a	|*	|*	|r	|.
|k	|[23][S]\rarr	|b	|i	|a	|ł	|o	|s	|t	|e	|r	|e	|k	|[][,]{ }	|z	|i	|e	|l	|o	|n	|y	|*	|a	|.
|a	|[24][S]\drarr	|t	|e	|n	|d	|e	|n	|c	|j	|a	|*	|*	|*	|*	|*	|*	|*	|*	|*	|*	|*	|ł	|.
|r	|s	|*	|*	|[25][S]\rarr	|b	|a	|n	|i	|o	|p	|i	|e	|ń	|[][,]{ }	|l	|[][S]ü	|t	|k	|e	|g	|o	|*	|.
|d	|t	|*	|*	|*	|*	|*	|*	|*	|*	|n	|[26][S]\darr	|[27][S]\drarr	|j	|e	|d	|n	|o	|s	|t	|k	|a	|*	|.
|i	|a	|*	|*	|*	|[28][S]\darr	|*	|*	|*	|*	|i	|k	|d	|*	|*	|[29][S]\drarr	|m	|a	|n	|y	|a	|s	|*	|.
|e	|r	|[30][S]\rarr	|b	|a	|l	|a	|s	|*	|*	|ę	|o	|ż	|*	|*	|n	|*	|*	|[31][S]\darr	|[32][S]\darr	|[33][S]\darr	|*	|*	|.
|r	|t	|*	|*	|*	|e	|*	|*	|*	|*	|c	|n	|i	|[34][S]\drarr	|d	|o	|m	|o	|s	|t	|w	|o	|*	|.
|*	|e	|*	|*	|*	|g	|*	|*	|*	|*	|i	|t	|h	|o	|*	|n	|*	|*	|z	|r	|ę	|*	|*	|.
|*	|r	|*	|[35][S]\rarr	|h	|i	|p	|n	|o	|t	|e	|r	|a	|p	|i	|a	|*	|*	|e	|u	|z	|*	|*	|.
|*	|*	|*	|[36][S]\rarr	|k	|o	|ż	|u	|c	|h	|*	|a	|d	|a	|*	|*	|*	|*	|n	|m	|e	|*	|*	|.
|*	|*	|[37][S]\rarr	|k	|o	|n	|i	|d	|i	|u	|m	|*	|*	|d	|*	|*	|*	|*	|s	|n	|ł	|*	|*	|.
|*	|*	|*	|*	|*	|*	|*	|*	|*	|*	|*	|*	|*	|*	|*	|*	|*	|*	|i	|a	|*	|*	|*	|.
|*	|*	|*	|*	|*	|*	|*	|*	|*	|*	|*	|*	|*	|*	|*	|*	|*	|*	|*	|*	|*	|*	|*	|.\end{Puzzle}

\newpage

\begin{PuzzleClues}{\textbf{Poziome}\\}\Clue{1}{}{stawonóg z podtypu sześcionogów}
\Clue{2}{}{w piłce i hokeju: prowadzenie piłki lub krążka blisko ziemi (lodowiska) z częstą zmianą kierunku dla zmylenia i wyminięcia przeciwnika}
\Clue{3}{}{złowieszczyk: gatunek chrząszcza}
\Clue{4}{}{umowa między bankiem a klientem dotycząca lokowania środków pieniężnych zawierana na czas określony}
\Clue{5}{}{mieszkanka Słowenii, kobieta pochodzenia słoweńskiego}
\Clue{7}{}{wąski zakres czegoś}
\Clue{13}{}{renault z modelu Clio}
\Clue{14}{}{rodzaj kompozycji; zwykle porównanie, przedstawienie dwóch faktów w nieprzypadkowym sąsiedztwie}
\Clue{16}{}{przyrząd pomiarowy określający prędkość poruszania się jednostki pływającej oraz przebytą przez nią drogę}
\Clue{17}{}{wzór, forma, która służy do seryjnego wykonania różnego typu przedmiotów}
\Clue{18}{}{bogaczka - posiadaczka majątku}
\Clue{20}{}{obfity, dobry plon}
\Clue{21}{}{funkcja, według której obliczana jest wartość wyjścia neuronów sieci neuronowej}
\Clue{22}{}{zawody sportowe dla uczniów szkół gimnazjalnych}
\Clue{23}{}{Urosticte benjamini - gatunek ptaka z rzędu jerzykowych (Apodiformes), z rodziny kolibrów (Trochilidae), z podrodziny kolibrów (Trochilinae)}
\Clue{24}{}{potencjalna właściwość}
\Clue{25}{}{Nereocystis luetkeana - gatunek morskiej brunatnicy z rodziny lessoniowatych (Lessoniaceae) występujący w Oceanie Spokojnym w przybrzeżnej strefie Ameryki Północnej, od Alaski (Wyspy Aleuckie) na północy po wybrzeże Półwyspu Kalifornijskiego na południu}
\Clue{27}{}{odrębna jednolita całość, element podziału większej całości, nie tylko w organizacjach, także gdy mowa np. o klasyfikacji}
\Clue{29}{}{jezioro w Turcji}
\Clue{30}{}{tralka}
\Clue{34}{}{obejście, gumno, zagroda; dom z najbliższymi zagospodarowaniami}
\Clue{35}{}{psychoterapia wykorzystująca hipnozę jako środek diagnostyczny i/lub leczniczy}
\Clue{36}{}{ciepła odzież wierzchnia}
\Clue{37}{}{zarodnik grzybów, który nie powstaje w zarodni, ale na szczycie konidioforu przez pączkowanie i odcinanie ścianą nowo powstałej komórki}\end{PuzzleClues}

\begin{PuzzleClues}{\textbf{Pionowe}\\}\Clue{1}{}{mieszkaniec Baskonii - regionu i krainy historycznej w Hiszpanii, człowiek pochodzenia baskijskiego}
\Clue{3}{}{budowla lub zespół architektoniczny stanowiący siedzibę o charakterze reprezentacyjnym}
\Clue{6}{}{wojskowy stopień oficerski w polskiej Marynarce Wojennej, odpowiadający generałowi dywizji w Wojskach Lądowych i Siłach Powietrznych}
\Clue{8}{}{dawny internat lub obecnie dom akademicki dla księży}
\Clue{9}{}{przygraniczna strażnica}
\Clue{10}{}{bodnia; w wiertnictwie wykop o przekroju kwadratowym przeznaczony do pomieszczenia więźby rur okładzinowych}
\Clue{11}{}{zbiór ropy w jamach ciała, przyjmujący kształt stożka}
\Clue{12}{}{wieś w Polsce położona w województwie lubelskim, w powiecie opolskim, w gminie Wilków, w Małopolskim Przełomie Wisły}
\Clue{15}{}{człowiek pochodzący z tej samej okolicy}
\Clue{16}{}{sanitariusz w wojsku Księstwa Warszawskiego}
\Clue{19}{}{uszkodzenie jakiejś powierzchni przez przesunięcie po niej jakiegoś ostrego przedmiotu i spowodowanie powstania rysy}
\Clue{24}{}{rozrusznik}
\Clue{26}{}{kontra oktawa}
\Clue{27}{}{obowiązek szerzenia wiary - w sposób pokojowy lub z użyciem siły - spoczywający na wyznawcy islamu}
\Clue{28}{}{ochotniczy oddział wojsk, np. Legiony Polskie we Włoszech}
\Clue{29}{}{część Oficjum kościelnego odprawiana dawniej o godzinie dziewiątej}
\Clue{31}{}{SHAANXI - prowincja w środkowych Chinach, powierzchnia 195,8 tyś. km2, ośrodek administracyjny Xi'an}
\Clue{32}{}{trapezoidalne lub prostokątne pole na boisku do koszykówki, na którym zawodnik może przebywać nie dłużej niż trzy sekundy w czasie gry}
\Clue{33}{}{jednostka prędkości w żegludze równa 1, 852 km/h}
\Clue{34}{}{zjawisko, zdarzenie, nieintencjonalne spadanie z nieba jakiejś substancji}\end{PuzzleClues}\newpage\section*{Krzyżówka 55}

\noindent\begin{Puzzle}{19}{18}|*	|*	|*	|*	|*	|*	|*	|*	|*	|*	|*	|*	|*	|*	|*	|[1][S]\drarr	|ł	|e	|b	|*	|.
|*	|*	|*	|*	|*	|*	|*	|*	|*	|[2][S]\rarr	|n	|i	|t	|s	|c	|h	|*	|*	|*	|*	|.
|*	|*	|*	|*	|*	|*	|*	|*	|*	|*	|[3][S]\drarr	|s	|u	|k	|n	|o	|*	|[4][S]\drarr	|b	|*	|.
|*	|*	|*	|*	|[5][S]\rarr	|o	|w	|a	|d	|e	|k	|*	|[6][S]\rarr	|t	|o	|t	|e	|m	|*	|[7][S]\darr	|.
|*	|*	|*	|*	|*	|*	|[8][S]\rarr	|s	|c	|h	|u	|l	|z	|*	|[9][S]\darr	|e	|*	|u	|*	|s	|.
|*	|*	|[10][S]\rarr	|p	|o	|m	|a	|g	|i	|e	|r	|*	|*	|*	|w	|n	|*	|r	|*	|o	|.
|*	|*	|[11][S]\darr	|[12][S]\darr	|[13][S]\rarr	|e	|n	|c	|e	|f	|a	|l	|o	|p	|a	|t	|i	|a	|*	|s	|.
|*	|[14][S]\rarr	|s	|t	|e	|p	|ó	|w	|k	|a	|*	|*	|*	|*	|m	|o	|[15][S]\darr	|r	|*	|n	|.
|*	|[16][S]\drarr	|t	|e	|[][,]{ }	|s	|p	|r	|a	|w	|y	|*	|*	|[17][S]\darr	|s	|t	|m	|k	|[18][S]\darr	|ó	|.
|[19][S]\drarr	|k	|o	|r	|e	|k	|*	|*	|[20][S]\rarr	|z	|a	|ł	|a	|m	|*	|k	|u	|a	|k	|w	|.
|z	|a	|p	|y	|*	|*	|*	|*	|*	|*	|[21][S]\rarr	|c	|z	|e	|f	|a	|l	|*	|i	|k	|.
|i	|r	|i	|l	|[22][S]\rarr	|r	|e	|s	|p	|o	|n	|d	|e	|n	|t	|*	|a	|*	|n	|a	|.
|e	|a	|e	|e	|*	|*	|*	|*	|*	|*	|*	|*	|*	|u	|*	|[23][S]\rarr	|k	|b	|d	|*	|.
|m	|c	|ń	|n	|*	|[24][S]\rarr	|k	|a	|r	|c	|z	|o	|c	|h	|*	|*	|*	|*	|e	|*	|.
|i	|e	|*	|*	|*	|*	|*	|*	|*	|[25][S]\rarr	|n	|o	|w	|i	|c	|k	|i	|*	|r	|*	|.
|a	|n	|*	|*	|*	|*	|[26][S]\rarr	|c	|y	|t	|w	|a	|r	|n	|i	|c	|a	|*	|b	|*	|.
|*	|a	|*	|*	|*	|*	|*	|[27][S]\rarr	|g	|o	|ł	|ą	|b	|*	|*	|*	|*	|*	|a	|*	|.
|*	|*	|*	|*	|*	|*	|*	|*	|*	|*	|*	|*	|*	|[28][S]\rarr	|b	|e	|r	|y	|l	|*	|.
|*	|*	|*	|*	|*	|*	|*	|*	|*	|*	|*	|*	|*	|*	|*	|*	|*	|*	|*	|*	|.\end{Puzzle}

\newpage

\begin{PuzzleClues}{\textbf{Poziome}\\}\Clue{1}{}{zdolność myślenia, rozumowania}
\Clue{2}{}{lekarz bakteriolog (1873-1943); prace dotyczące wścieklizny, surowic i szczepionek}
\Clue{3}{}{tkanina z wełny zgrzebnej, spilśniona powierzchniowo, drapana, strzyżona}
\Clue{4}{}{bajt - najmniejsza adresowalna jednostka informacji pamięci komputerowej, składająca się z bitów}
\Clue{5}{}{owad}
\Clue{6}{}{słup reklamowy, na którym umieszczone jest logo firmy (lub firm)}
\Clue{8}{}{twóczość Brunona Schulza, ogół jego dzieł literackich, malarskich, rysowniczych i krytycznych}
\Clue{10}{}{człowiek, który pomaga drugiemu w wykonywaniu jakiejś czynności; często określenie żartobliwe lub ironiczne}
\Clue{13}{}{uszkodzenie mózgu przez czynniki różnego pochodzenia}
\Clue{14}{}{ptak}
\Clue{16}{}{seks, życie seksualne, współżycie; określenie eufemistyczne}
\Clue{19}{}{zator na drodze}
\Clue{20}{}{miejsce załamania czegoś, obszar, który się załamał}
\Clue{21}{}{ryba słodkich mórz}
\Clue{22}{}{osoba, która udziela odpowiedzi na pytania zadawane w wywiadzie lub składające się na ankietę}
\Clue{23}{}{kod ISO 4217 dolara Kiribati}
\Clue{24}{}{ozdoba architektoniczna w kształcie karczocha}
\Clue{25}{}{Jacek, ur. 1921r., architekt - dom handlowy „Merkury”}
\Clue{26}{}{Kaempferia - rodzaj byliny z rodziny imbirowatych}
\Clue{27}{}{ptak; poszczególne gatunki tego ptaka w taksonomii biologicznej klasyfikowane są w obrębie rodziny gołębiowatych (Columbidae), w podrodzinie gołębi (Columbinae)}
\Clue{28}{}{minerał należący do grupy krzemianów pierścieniowych}\end{PuzzleClues}

\begin{PuzzleClues}{\textbf{Pionowe}\\}\Clue{1}{}{przedstawicielka ludu afrykańskiego, rdzenna mieszkanka południa Afryki; dziś nazwa często odbierana jako niegrzeczna i obelżywa}
\Clue{3}{}{ptak domowy pochodzący od bankiwy, udomowiony ok. 2500 p.n.e}
\Clue{4}{}{Osmia - rodzaj pszczoły z rodziny miesierkowatych (Megachilidae) mający silnie owłosione ciało osiągające długość do 15 mm; zakłada gniazda w ziemi lub w pustych muszlach ślimaków}
\Clue{7}{}{eurazjatycki ptak z rodziny sikorowatych, występuje też w płn. Afryce}
\Clue{9}{}{krótki męski kaftan usztywniony, podszyty watą, wykonany z sukna, atłasu lub aksamitu noszony w XVI-XVII w}
\Clue{11}{}{kategoria oraz forma przymiotników i niektórych przysłówków, wyrażająca poziom intensywności jakiejś cechy (najczęściej w porównaniu z innymi obiektami)}
\Clue{12}{}{angielskie włókno poliestrowe, bardzo mocne, odporne na działanie temperatury, światła, kwasów, rozpuszczalników; używane do wyrobu tkanin odzieżowych i bieliźnianych oraz technicznych}
\Clue{15}{}{łowny jeleń Ameryki Północnej}
\Clue{16}{}{oryginalna polska zbroja łuskowa produkowana od XVI do XVIII wieku; składała się ze skórzanego kaftana pokrytego metalowymi płytkami w kształcie łusek nachodzących na siebie}
\Clue{17}{}{ur. w 1916 r., skrzypek amerykański, także dyrygent, osiadł w Wielkiej Brytanii, jeden z najwybitniejszych skrzypków współczesnych}
\Clue{18}{}{bal dla dzieci, podczas którego często odbywają się konkursy i specjalne występy, np. klaunów}
\Clue{19}{}{kort tenisowy o nawierzchni tzw. ziemnej, wykonanej z mączki ceglanej}\end{PuzzleClues}\newpage\section*{Krzyżówka 56}

\noindent\begin{Puzzle}{12}{32}|*	|*	|*	|[1][S]\darr	|*	|[2][S]\drarr	|p	|a	|s	|a	|ż	|*	|*	|.
|*	|*	|*	|m	|[3][S]\drarr	|p	|l	|a	|z	|m	|a	|*	|[4][S]\darr	|.
|*	|[5][S]\rarr	|k	|i	|p	|i	|e	|l	|*	|*	|[6][S]\darr	|*	|r	|.
|*	|*	|[7][S]\darr	|n	|a	|j	|[8][S]\darr	|[9][S]\rarr	|w	|ó	|z	|*	|i	|.
|*	|[10][S]\drarr	|w	|i	|t	|a	|m	|i	|n	|k	|a	|*	|c	|.
|*	|w	|i	|r	|y	|w	|g	|*	|*	|*	|d	|[11][S]\darr	|h	|.
|*	|i	|z	|a	|c	|n	|*	|*	|[12][S]\darr	|*	|a	|k	|t	|.
|*	|r	|a	|d	|z	|i	|*	|*	|c	|*	|s	|o	|e	|.
|*	|u	|w	|i	|a	|k	|*	|*	|z	|*	|z	|t	|r	|.
|*	|s	|a	|o	|k	|i	|[13][S]\drarr	|d	|e	|b	|e	|l	|*	|.
|*	|[][,]{ }	|*	|*	|*	|*	|a	|[14][S]\darr	|r	|[15][S]\darr	|n	|i	|*	|.
|*	|g	|*	|*	|*	|*	|n	|m	|w	|k	|i	|k	|*	|.
|*	|r	|*	|*	|[16][S]\darr	|*	|i	|o	|o	|s	|e	|*	|*	|.
|*	|y	|*	|[17][S]\darr	|i	|*	|o	|d	|ń	|i	|*	|*	|*	|.
|*	|p	|*	|s	|n	|*	|ł	|e	|c	|ę	|*	|[18][S]\darr	|*	|.
|*	|y	|*	|p	|s	|[19][S]\darr	|e	|l	|z	|ż	|*	|s	|*	|.
|*	|[][,]{ }	|*	|a	|t	|w	|k	|[][,]{ }	|y	|y	|*	|t	|*	|.
|*	|t	|*	|c	|r	|ó	|*	|m	|k	|c	|*	|e	|[20][S]\darr	|.
|*	|y	|[21][S]\drarr	|j	|u	|d	|e	|a	|*	|[][,]{ }	|*	|f	|p	|.
|*	|p	|g	|a	|m	|k	|*	|t	|[22][S]\darr	|g	|[23][S]\darr	|f	|k	|.
|*	|u	|i	|*	|e	|a	|*	|e	|r	|a	|p	|e	|r	|.
|*	|[][,]{ }	|ż	|*	|n	|[][,]{ }	|[24][S]\rarr	|m	|o	|l	|o	|s	|*	|.
|*	|a	|y	|*	|t	|w	|[25][S]\drarr	|a	|s	|i	|k	|*	|*	|.
|*	|*	|c	|*	|[][,]{ }	|y	|k	|t	|e	|l	|r	|*	|*	|.
|*	|*	|c	|*	|d	|b	|o	|y	|n	|e	|y	|*	|*	|.
|*	|*	|z	|*	|ę	|o	|ł	|c	|b	|u	|c	|*	|*	|.
|*	|*	|a	|*	|t	|r	|y	|z	|e	|s	|i	|*	|*	|.
|*	|*	|n	|*	|y	|o	|s	|n	|r	|z	|e	|*	|*	|.
|*	|*	|k	|*	|*	|w	|a	|y	|g	|o	|*	|*	|*	|.
|[26][S]\rarr	|m	|a	|j	|d	|a	|n	|*	|*	|w	|*	|*	|*	|.
|*	|*	|*	|*	|*	|*	|k	|*	|*	|y	|*	|*	|*	|.
|[27][S]\rarr	|b	|o	|m	|b	|i	|a	|r	|z	|*	|*	|*	|*	|.
|*	|*	|*	|*	|*	|*	|*	|*	|*	|*	|*	|*	|*	|.\end{Puzzle}

\newpage

\begin{PuzzleClues}{\textbf{Poziome}\\}\Clue{2}{}{wirtuozowski zwrot wykonawczy, oparty zwykle na rozłożonym akordzie, utrzymany w równych, drobnych wartościach muzycznych}
\Clue{3}{}{telewizor, który do tworzenia obrazu wykorzystuje plazmę i luminofor}
\Clue{5}{}{PRZYBÓJ; spiętrzone falowanie, szczególnie silne przy stromym wybrzeżu}
\Clue{9}{}{pojazd zaprzęgowy używany do transportu}
\Clue{10}{}{organiczny związek chemiczny, niezbędny do prawidłowego funkcjonowania organizmu żywego; może być pochodzenia naturalnego lub otrzymywana syntetycznie}
\Clue{13}{}{łódź wiosłowa dla dwóch osób}
\Clue{21}{}{nazwa prowincji rzymskiej odpowiadającej dawnemu terytorium Judy}
\Clue{24}{}{pies o dużych rozmiarach, użytkowany jako pies stróżujący, myśliewski czy bojowy}
\Clue{25}{}{zdrobniale o asie - figurze karcianej}
\Clue{26}{}{w fortecy plac otoczony wałami}
\Clue{27}{}{przestępca, który podkłada ładunki wybuchowe lub wszczyna fałszywy alarm z powodu rzekomo podłożonej bomby}\end{PuzzleClues}

\begin{PuzzleClues}{\textbf{Pionowe}\\}\Clue{1}{}{miniaturowe radio, którego rozmiary pozwalają zwykle zmieścić odbiornik w dłoni}
\Clue{2}{}{Pluvianidae - monotypowa rodzina ptaków z rzędu siewkowych (Charadriiformes)}
\Clue{3}{}{schematycznie naszkicowany ludzik}
\Clue{4}{}{Frantisek (1709-1789); kompozytor czeski, reprezentant szkoły mannhejskiej}
\Clue{6}{}{przykrycie dachem}
\Clue{7}{}{rodzaj czteroosobowego powozu}
\Clue{8}{}{w chemii: symbol magnezu}
\Clue{10}{}{typ wirusa grypy, który jest przyczyną pandemii, m.in. ptasiej grypy, świńskiej grypy i hiszpanki}
\Clue{11}{}{NIEDŹWIEDŹ MORSKI; gatunek foki}
\Clue{12}{}{motyl dzienny z podrodziny czerwończyków}
\Clue{13}{}{byt duchowy w wielu religiach, który służy i na różne sposoby wypełnia zamysły Boga}
\Clue{14}{}{opis zjawisk ze świata rzeczywistego przy użyciu języka matematyki}
\Clue{15}{}{jeden z czterech największych naturalnych satelit Jowisza, odrkyte w 1610 przez Galileusza przy pomocy skonstruowanej przez niego lunety}
\Clue{16}{}{instrument muzyczny, w którym źródłem dźwięku jest drgający wewnątrz instrumentu słup powietrza}
\Clue{17}{}{w tekście drukowanym - przestrzeń, odstęp między wyrazami}
\Clue{18}{}{siatkarz amerykański, złoty medalista z Atlanty (siatkówka plażowa)}
\Clue{19}{}{wódka wytworzona ze spirytusu dwukrotnie rektyfikowanego}
\Clue{20}{}{kod ISO 4217 rupii pakistańskiej}
\Clue{21}{}{mieszkanka Giżycka}
\Clue{22}{}{ur. w 1892 r., szwedzki dyrygent i kompozytor; utwory symfoniczne, kameralne, koncerty, opery, oratoria}
\Clue{23}{}{substancja lub materiał pokrywająca powierzchnię czegoś}
\Clue{25}{}{BERCEUSE}\end{PuzzleClues}\newpage\section*{Krzyżówka 57}

\noindent\begin{Puzzle}{20}{29}|*	|*	|*	|[1][S]\darr	|*	|*	|*	|[2][S]\darr	|[3][S]\drarr	|t	|y	|t	|a	|n	|o	|z	|a	|u	|r	|y	|*	|.
|[4][S]\drarr	|b	|a	|c	|h	|*	|*	|k	|s	|[5][S]\darr	|*	|[6][S]\darr	|[7][S]\darr	|*	|*	|*	|[8][S]\darr	|[9][S]\darr	|*	|*	|[10][S]\darr	|.
|t	|*	|*	|h	|*	|*	|*	|u	|e	|h	|*	|m	|ż	|*	|*	|[11][S]\darr	|p	|b	|*	|*	|p	|.
|u	|*	|*	|ł	|[12][S]\darr	|*	|*	|r	|k	|i	|*	|n	|a	|[13][S]\darr	|[14][S]\darr	|j	|r	|y	|[15][S]\darr	|*	|r	|.
|r	|[16][S]\rarr	|c	|o	|k	|*	|*	|f	|w	|p	|[17][S]\rarr	|i	|r	|o	|k	|e	|z	|k	|a	|*	|i	|.
|b	|*	|*	|p	|w	|*	|*	|i	|o	|o	|*	|c	|t	|f	|i	|l	|y	|o	|m	|*	|z	|.
|i	|*	|*	|a	|a	|*	|*	|r	|j	|d	|*	|h	|*	|i	|p	|e	|c	|w	|f	|*	|r	|.
|n	|[18][S]\darr	|[19][S]\darr	|k	|s	|*	|*	|s	|a	|r	|*	|*	|[20][S]\darr	|a	|a	|c	|z	|i	|i	|*	|e	|.
|a	|s	|m	|*	|[][,]{ }	|*	|*	|z	|d	|o	|[21][S]\darr	|*	|p	|r	|*	|*	|y	|e	|b	|[22][S]\darr	|n	|.
|[][,]{ }	|y	|i	|*	|b	|*	|[23][S]\rarr	|t	|e	|m	|p	|e	|r	|a	|m	|e	|n	|c	|i	|k	|*	|.
|n	|n	|n	|*	|e	|*	|*	|*	|n	|*	|r	|*	|z	|*	|*	|*	|e	|*	|a	|o	|[24][S]\darr	|.
|a	|d	|u	|*	|n	|[25][S]\darr	|[26][S]\drarr	|o	|d	|c	|i	|n	|e	|c	|z	|e	|k	|*	|*	|n	|m	|.
|p	|r	|t	|*	|z	|g	|a	|*	|r	|*	|a	|*	|g	|*	|*	|*	|*	|*	|*	|s	|t	|.
|o	|o	|n	|*	|o	|r	|m	|*	|o	|*	|m	|[27][S]\drarr	|l	|i	|m	|f	|o	|c	|y	|t	|*	|.
|r	|m	|i	|*	|e	|a	|*	|*	|n	|*	|*	|s	|ą	|[28][S]\rarr	|k	|a	|r	|a	|n	|y	|*	|.
|o	|[][,]{ }	|k	|*	|s	|p	|[29][S]\drarr	|d	|*	|*	|*	|z	|d	|*	|[30][S]\drarr	|b	|u	|f	|e	|t	|*	|.
|w	|s	|[][,]{ }	|[31][S]\drarr	|o	|p	|s	|o	|n	|i	|n	|a	|*	|[32][S]\rarr	|k	|a	|d	|ź	|*	|u	|*	|.
|a	|z	|m	|t	|w	|a	|z	|[33][S]\rarr	|k	|o	|p	|r	|o	|f	|a	|g	|i	|a	|*	|t	|*	|.
|*	|t	|e	|a	|y	|*	|c	|*	|[34][S]\darr	|*	|[35][S]\drarr	|m	|ę	|t	|n	|o	|ś	|ć	|*	|y	|*	|.
|*	|o	|c	|r	|*	|[36][S]\darr	|z	|[37][S]\rarr	|d	|u	|b	|a	|j	|*	|t	|*	|*	|*	|[38][S]\darr	|w	|*	|.
|*	|k	|h	|y	|*	|b	|u	|*	|i	|*	|a	|n	|*	|*	|y	|*	|*	|*	|k	|n	|*	|.
|[39][S]\drarr	|h	|a	|d	|ż	|a	|r	|*	|t	|*	|b	|t	|[40][S]\rarr	|a	|l	|l	|e	|g	|r	|o	|*	|.
|l	|o	|n	|a	|*	|r	|[][,]{ }	|*	|*	|*	|i	|*	|*	|*	|e	|*	|*	|*	|e	|ś	|*	|.
|a	|l	|i	|*	|*	|a	|w	|[41][S]\darr	|[42][S]\rarr	|g	|e	|r	|m	|a	|n	|i	|n	|*	|w	|ć	|*	|.
|n	|m	|c	|*	|[43][S]\drarr	|n	|o	|c	|e	|k	|[][,]{ }	|a	|l	|k	|a	|t	|o	|e	|*	|*	|*	|.
|d	|s	|z	|*	|d	|e	|d	|i	|*	|[44][S]\rarr	|l	|i	|p	|a	|*	|*	|*	|*	|*	|*	|*	|.
|s	|k	|n	|*	|a	|k	|n	|e	|[45][S]\rarr	|t	|a	|b	|l	|e	|t	|k	|a	|*	|*	|*	|*	|.
|a	|i	|y	|*	|n	|*	|y	|g	|[46][S]\rarr	|s	|t	|e	|r	|y	|l	|n	|o	|ś	|ć	|*	|*	|.
|t	|*	|*	|*	|a	|*	|*	|*	|[47][S]\rarr	|d	|o	|b	|r	|y	|[][,]{ }	|w	|u	|j	|e	|k	|*	|.
|*	|*	|*	|*	|*	|*	|[48][S]\rarr	|s	|o	|s	|*	|*	|*	|*	|*	|*	|*	|*	|*	|*	|*	|.\end{Puzzle}

\newpage

\begin{PuzzleClues}{\textbf{Poziome}\\}\Clue{3}{}{Titanosauria - grupa zauropodów żyjących od późnej jury do końca kredy, obejmująca jedne z największych lądowych zwierząt w dziejach}
\Clue{4}{}{Jan Sebastian Bach - kompozytor i organista niemiecki epoki baroku, jeden z najwybitniejszych artystów w dziejach muzyki}
\Clue{16}{}{kod ISO 4217 dolara Wysp Cooka}
\Clue{17}{}{przedstawicielka plemienia Indian Ameryki Północnej zamieszkujących tereny na wschód od Wielkich Jezior i na południe od Rzeki św. Wawrzyńca, czyli północną część stanów Nowy Jork, Pensylwanii i Maine, niemal cały stan Vermont oraz południowe obszary kanadyjskich prowincji Quebec i Ontario}
\Clue{23}{}{dość dynamiczna, interesująca osobowość}
\Clue{26}{}{mały, krótki, niewielki odcinek}
\Clue{27}{}{komórka układu odpornościowego należąca do agranulocytów z grupy leukocytów, uczestnicząca i będąca podstawą odpowiedzi odpornościowej swoistej}
\Clue{28}{}{osoba odbywająca karę}
\Clue{29}{}{litera alfabetu używana w numeracji porządkowej}
\Clue{30}{}{część lokalu gastronomicznego, np. restauracji czy pubu, przy której serwuje się napoje alkoholowe i zamówione dania}
\Clue{31}{}{rodzaj przeciwciała; zapoczątkowuje i ułatwia proces fagocytozy}
\Clue{32}{}{zbiornik na ciecz o dużej pojemności}
\Clue{33}{}{konsumpcja kału (zaburzenie łaknienia lub stan chorobowy)}
\Clue{35}{}{cecha tego, co jest niejasne, zagmatwane, nie w pełni zrozumiałe}
\Clue{37}{}{miasto w Zjednoczonych Emiratach Arabskich, stolica emiratu Dubaj}
\Clue{39}{}{Czarny Kamień w świątyni Kaaba czczony przez pielgrzymujących do Mekki muzułmanów}
\Clue{40}{}{szybkie tempo - szybsze niż allegretto, wolniejsze niż presto}
\Clue{42}{}{mieszkaniec starożytnej krainy Germanii}
\Clue{43}{}{nocek Alcathoe, Myotis alcathoe - gatunek ssaka z rzędu nietoperzy, występujący na obszarze południowej, zachodniej i środkowej Europy; opisano go jako nowy dla nauki gatunek z północnej Grecji, następnie stwierdzono go na terenie Węgier, Francji, Szwajcarii, Hiszpanii (Pireneje), Słowacji, Albanii, Bułgarii i Niemiec, w Polsce odnaleziono go dotąd jedynie w Beskidach i Tatrach}
\Clue{44}{}{kiedy plan nie wypali lub kiedy jest nudno}
\Clue{45}{}{doustny środek antykoncepcyjny}
\Clue{46}{}{cecha organizmów żywych, wsytępująca zwłaszcza u mieszańców: niemożność spłodzenia lub urodzenia potomstwa}
\Clue{47}{}{człowiek pobłażliwy i wyrozumiały, życzliwy dla innych}
\Clue{48}{}{kod ISO 4217 szylinga somalijskiego}\end{PuzzleClues}

\begin{PuzzleClues}{\textbf{Pionowe}\\}\Clue{1}{}{syn}
\Clue{2}{}{elektor, książę dawnej Rzeszy Niemieckiej}
\Clue{3}{}{mamutowiec, welingtonia, Sequoiadendron, Wellingtonia - długowieczne drzewo iglaste; rodzaj z rodziny cyprysowatych, obejmujący jeden gatunek}
\Clue{4}{}{wodna}
\Clue{5}{}{stadion, plac przeznaczony do wyścigów koni i zaprzęgów konnych}
\Clue{6}{}{budowla hydrotechniczna służąca do piętrzenia i regulowania przepływu wody w stawach}
\Clue{7}{}{fraszka, nic poważnego}
\Clue{8}{}{wypowiedź, która jest zapowiedzią wydarzenia albo innej wypowiedzi}
\Clue{9}{}{cios zadany bykowcem - biczem}
\Clue{10}{}{miasto w Jugosławii (Serbia), w okręgu autonomicznym Kosowo, u podnóża Szar Płaniny}
\Clue{11}{}{część rękojeści broni siecznej przeznaczona do ochrony ręki}
\Clue{12}{}{organiczny związek chemiczny, najprostszy aromatyczny kwas karboksylowy}
\Clue{13}{}{dar dla bóstwa, składany dla wywołania określonego efektu, np. w celu ochrony przed niezbezpieczeństwem}
\Clue{14}{}{metalowe kółeczko służące do przeciągania szotów, element statku}
\Clue{15}{}{samolot, pojazd wodno-powietrzny, który może startować z wody, a także na niej lądować}
\Clue{18}{}{stan psychiczny, który pojawia się u ofiar porwania lub u zakładników, wyrażający się odczuwaniem sympatii i solidarności z osobami je przetrzymującymi}
\Clue{19}{}{minutnik działający najczęściej na zasadzie wykorzystywania mechanizmu sprężynowego: przekręcenie pokrętła o kąt nastawiany na podziałce wyskalowanej w minutach uruchamia powolny obrót tego pokrętła, który trwa w przybliżeniu tak długo, ile wynika z nastawionej podziałki}
\Clue{20}{}{kontrola czegoś, oględziny, których celem jest ocena stanu czegoś}
\Clue{21}{}{w mitologii greckiej król Troi; bohaterIliady Homera}
\Clue{22}{}{to, że coś jest konstytutywne, ustanawia coś, stanowi o czymś (np. uchwała, zapis, wyrok)}
\Clue{24}{}{w chemii: symbol pierwiastka meitner}
\Clue{25}{}{napój alkoholowy podobny do koniaku}
\Clue{26}{}{w chemii: symbol ameryku}
\Clue{27}{}{człowiek, który charakteryzuje się wyszukanymi manierami zwłasza wobec kobiet; potrafi oczarować, olśnić towarzystwo mową, zachowaniem, wyglądem}
\Clue{29}{}{karczownik, karczownik ziemnowodny, polnik ziemnowodny, Arvicola amphibius - gatunek gryzonia z podrodziny nornikowatych w rodzinie chomikowatych; zamieszkuje brzegi wód Europy oraz północnej i środkowej Azji}
\Clue{30}{}{utwór muzyczny o śpiewnej melodyce}
\Clue{31}{}{mały, śródziemnomorski statek żaglowy o płaskim dnie, budowany w XVIII w. w Wenecji}
\Clue{34}{}{jednostka informacji}
\Clue{35}{}{czas pięknej pogody i ciepłych dni na pewnym etapie jesieni}
\Clue{36}{}{przen. człowiek o potulnym, zgodnym charakterze, cichy i grzeczny}
\Clue{38}{}{śmierć, zwłaszcza tragiczna i okrutna}
\Clue{39}{}{pierwszy amerykański satelita teledetekcyjny}
\Clue{41}{}{warstwa metalu ułożona podczas spawania jednorazowym przejściem elektrody}
\Clue{43}{}{amerykański geolog i mineralog (1813-95); twórca chemicznej systematyki minerałów}\end{PuzzleClues}\newpage\section*{Krzyżówka 58}

\noindent\begin{Puzzle}{15}{30}|*	|*	|*	|*	|*	|*	|*	|*	|*	|*	|*	|[1][S]\drarr	|w	|a	|ł	|*	|.
|*	|*	|*	|*	|*	|*	|[2][S]\rarr	|z	|w	|ó	|j	|k	|a	|*	|*	|[3][S]\darr	|.
|*	|*	|[4][S]\rarr	|s	|o	|l	|i	|d	|a	|r	|n	|o	|ś	|ć	|*	|m	|.
|*	|*	|*	|[5][S]\darr	|[6][S]\rarr	|n	|o	|m	|o	|k	|a	|n	|o	|n	|*	|s	|.
|[7][S]\drarr	|u	|r	|o	|d	|n	|o	|ś	|ć	|*	|[8][S]\darr	|i	|*	|*	|*	|z	|.
|d	|*	|*	|r	|*	|*	|*	|[9][S]\rarr	|r	|y	|b	|k	|a	|*	|*	|a	|.
|i	|*	|[10][S]\rarr	|m	|o	|g	|u	|n	|c	|j	|a	|*	|*	|*	|*	|[][,]{ }	|.
|a	|[11][S]\drarr	|k	|o	|c	|z	|o	|w	|n	|i	|c	|e	|*	|*	|*	|g	|.
|r	|t	|[12][S]\rarr	|c	|a	|n	|n	|o	|n	|*	|ó	|*	|*	|*	|*	|r	|.
|m	|r	|*	|*	|*	|*	|*	|*	|*	|[13][S]\darr	|w	|*	|[14][S]\darr	|*	|*	|e	|.
|u	|y	|[15][S]\darr	|*	|*	|*	|*	|*	|*	|p	|k	|*	|k	|*	|*	|g	|.
|i	|p	|d	|[16][S]\rarr	|k	|o	|n	|s	|o	|l	|a	|*	|o	|*	|*	|o	|.
|d	|o	|a	|[17][S]\rarr	|r	|y	|s	|u	|l	|a	|*	|*	|l	|*	|*	|r	|.
|[][,]{ }	|d	|c	|[18][S]\drarr	|k	|o	|m	|a	|r	|y	|*	|[19][S]\darr	|e	|*	|*	|i	|.
|u	|i	|h	|m	|*	|*	|*	|*	|*	|b	|*	|ł	|c	|*	|*	|a	|.
|a	|a	|[][,]{ }	|i	|[20][S]\drarr	|u	|s	|h	|u	|a	|i	|a	|*	|*	|[21][S]\darr	|ń	|.
|[][,]{ }	|*	|p	|t	|m	|[22][S]\darr	|*	|*	|*	|c	|*	|p	|*	|*	|i	|s	|.
|d	|*	|u	|*	|a	|r	|*	|*	|*	|k	|*	|i	|*	|[23][S]\darr	|n	|k	|.
|u	|[24][S]\drarr	|l	|u	|c	|e	|r	|n	|a	|*	|*	|d	|*	|h	|s	|a	|.
|i	|l	|p	|*	|o	|l	|[25][S]\drarr	|n	|a	|k	|ł	|u	|c	|i	|e	|*	|.
|b	|a	|i	|*	|n	|a	|d	|*	|*	|*	|*	|c	|*	|g	|k	|*	|.
|h	|n	|t	|*	|*	|k	|z	|[26][S]\drarr	|k	|o	|c	|h	|e	|r	|t	|*	|.
|n	|d	|o	|[27][S]\drarr	|p	|s	|i	|a	|n	|k	|a	|*	|*	|o	|y	|*	|.
|e	|l	|w	|ś	|*	|a	|a	|c	|*	|*	|*	|*	|[28][S]\darr	|g	|c	|*	|.
|*	|o	|y	|l	|*	|c	|ł	|h	|*	|[29][S]\drarr	|g	|e	|t	|r	|y	|*	|.
|*	|r	|*	|i	|*	|j	|k	|e	|*	|l	|*	|*	|o	|a	|d	|*	|.
|*	|d	|*	|n	|*	|a	|a	|l	|*	|i	|*	|*	|r	|f	|*	|*	|.
|*	|*	|*	|y	|*	|*	|*	|o	|*	|c	|*	|*	|*	|*	|*	|*	|.
|*	|*	|*	|*	|*	|*	|[30][S]\rarr	|m	|i	|o	|d	|o	|w	|ó	|d	|*	|.
|*	|*	|[31][S]\rarr	|u	|p	|i	|t	|a	|*	|*	|*	|*	|*	|*	|*	|*	|.
|*	|*	|*	|*	|*	|*	|*	|*	|*	|*	|*	|*	|*	|*	|*	|*	|.\end{Puzzle}

\newpage

\begin{PuzzleClues}{\textbf{Poziome}\\}\Clue{1}{}{element maszyny o znacznej długości w porównaniu z wymiarami poprzecznymi, obracający się wokół swej osi i przenoszący moment obrotowy}
\Clue{2}{}{zwojkówka; drobny motyl nocny, szkodnik drzew, krzewów i roślin uprawnych}
\Clue{4}{}{prawnie zadeklarowana współodpowiedzialność jakiejś grupy za podejmowane zobowiązania, wspólne działania (dotycząca dłużników lub wierzycieli)}
\Clue{6}{}{zbiór praw w imperium bizantyjskim}
\Clue{7}{}{to, że ktoś jest urodziwy, urodny}
\Clue{9}{}{Eugeniusz, ur. w 1898r. astronom; prace z fotometrii gwiazdowej i historii astronomii}
\Clue{10}{}{miasto w Niemczech (Nadrenia-Palatynat), port nad Renem, przy ujściu Renu; ośrodek przemysłowy, turystyczny i handlowy}
\Clue{11}{}{Nomadinae - podrodzina owadów z rodziny pszczołowatych w podrzędzie trzonkówek}
\Clue{12}{}{fizjolog amerykański (1871-1945); prace dotyczące fizjologicznych mechanizmów emocji oraz roli autonomicznego układu nerwowego}
\Clue{16}{}{ozdobny wspornik, wykonany z kamienia, cegły lub drewna, podpierający rzeźbę, gzyms, balkon, kolumnę, żebra sklepienia, mający najczęściej formę esownicy lub woluty}
\Clue{17}{}{narciarz, dwudziestokrotny mistrz Polski w biegach w latach 1959-1974}
\Clue{18}{}{Culicidae, komary - występująca na całym świecie rodzina owadów (nadrodzina Culicoidea) z rzędu muchówek}
\Clue{20}{}{miasto w Argentynie nad kanałem Beagle; najdalej na południe położone miasto kuli ziemskiej}
\Clue{24}{}{kanton w środku Szwajcarii, obszar 1,5 tyś. km2, stolica Lucerna}
\Clue{25}{}{nabicie, nadzianie czegoś na szpikulec, ostry przedmiot}
\Clue{26}{}{mieszaniec tulu z dromaderem}
\Clue{27}{}{roślina zielna lub krzew, niektóre gatunki trujące, uprawiana na rabatach i w doniczkach}
\Clue{29}{}{dawniej (głównie w XIX w. i na początku XX w.) często spotykane cholewki z płótna, filcu lub sukna, wkładane zwykle na obuwie jako zabezpieczenie przed chłodem, dziś stosowane w odzieży stylizowanej na taką z dawnej epoki lub przy obuwiu dla współczesnych dandysów (robionym na zamówienie)}
\Clue{30}{}{ptak z rzędu dzięciołowatych, żywi się larwami pszczół i miodem; Afryka, Indie, Archipelag }
\Clue{31}{}{w powieści Henryka SienkiewiczaPotop miasteczko położone na Żmudzi}\end{PuzzleClues}

\begin{PuzzleClues}{\textbf{Pionowe}\\}\Clue{1}{}{osoba, od której można kupić bilety na jakieś wydarzenie, kiedy biletów nie ma już w kasie}
\Clue{3}{}{30 mszy zamówionych w intencji zmarłego lub jedna z 30 tych mszy}
\Clue{5}{}{miasto i port w Filipinach na zach. wybrzeżu wyspy Leyte; przemysł cukrowniczy}
\Clue{7}{}{w mitologii celtyckiej syn (wychowanek) Aengusa i Caer}
\Clue{8}{}{pomieszczenie na halach służące za mieszkanie w okresie wypasu owiec}
\Clue{11}{}{wers trzystopowy w poezji antycznej}
\Clue{13}{}{zapis dźwięku}
\Clue{14}{}{wydłużona, spiczasta, zwykle zaostrzona i kłująca część czegoś}
\Clue{15}{}{dach jednospadowy (o jednej połaci dachowej)}
\Clue{18}{}{symbol, narracja funkcjonujące w świadomości danej społeczności; znaczenie przypisywane czemuś przez daną społeczność, szerzone np. przez środki masowego przekazu}
\Clue{19}{}{kontroler biletów, zazwyczaj w komunikacji miejskiej}
\Clue{20}{}{miasto w USA (Georgia) nad rzeką Ocmulgee; ośrodek handlowy regionu rolniczego}
\Clue{21}{}{środek owadobójczy - substancja z grupy pestycydów używana do zwalczania szkodników w uprawach rolnych, lasach, w magazynach z żywnością, a także w mieszkaniach}
\Clue{22}{}{utrata napięcia w tkankach}
\Clue{23}{}{przyrząd wykorzystywany w meteorologii; higrometr samoczynnie rejestrujący zmiany wilgotności gazów}
\Clue{24}{}{właściciel, szef pubu, baru, restauracji, knajpy, szczególnie na Wyspach Brytyjskich}
\Clue{25}{}{dola, część przypadająca dla kogoś w podziale zysku, pieniędzy}
\Clue{26}{}{Acheloma - rodzaj temnospondyla z rodziny Trematopidae; żył w okresie wczesnego permu na terenach obecnej Ameryki Północnej}
\Clue{27}{}{rogi muflona w gwarze łowieckiej}
\Clue{28}{}{Th - promieniotwórczy pierwiastek chemiczny z grupy aktynowców}
\Clue{29}{}{w garbarstwie: zewnętrzna powierzchnia skóry, o charakterystycznym dla każdego gatunku zwierząt rysunku}\end{PuzzleClues}\newpage\section*{Krzyżówka 59}

\noindent\begin{Puzzle}{22}{26}|*	|*	|*	|*	|[1][S]\drarr	|p	|c	|*	|[2][S]\drarr	|b	|r	|z	|e	|g	|*	|*	|*	|*	|*	|[3][S]\drarr	|b	|u	|*	|.
|*	|*	|[4][S]\darr	|*	|t	|[5][S]\darr	|*	|*	|n	|*	|*	|[6][S]\drarr	|d	|u	|b	|l	|i	|ń	|c	|z	|y	|k	|*	|.
|*	|*	|e	|[7][S]\darr	|r	|e	|[8][S]\rarr	|k	|o	|k	|*	|f	|[9][S]\darr	|[10][S]\rarr	|s	|k	|ó	|r	|z	|a	|k	|*	|*	|.
|*	|*	|n	|o	|e	|p	|*	|*	|w	|[11][S]\darr	|*	|i	|m	|[12][S]\darr	|*	|[13][S]\drarr	|k	|i	|r	|s	|z	|*	|*	|.
|*	|*	|z	|d	|p	|i	|[14][S]\rarr	|k	|a	|n	|a	|l	|i	|k	|[][,]{ }	|n	|e	|r	|k	|o	|w	|y	|*	|.
|*	|*	|y	|l	|a	|l	|*	|*	|[][,]{ }	|a	|*	|a	|r	|u	|*	|i	|*	|*	|*	|b	|*	|*	|*	|.
|*	|[15][S]\drarr	|m	|e	|n	|o	|*	|*	|k	|n	|[16][S]\darr	|b	|*	|b	|*	|e	|[17][S]\rarr	|b	|a	|y	|e	|r	|*	|.
|*	|a	|[][,]{ }	|ż	|a	|g	|*	|[18][S]\darr	|l	|o	|b	|i	|*	|i	|*	|s	|*	|*	|*	|[][,]{ }	|*	|*	|*	|.
|*	|n	|h	|y	|c	|*	|[19][S]\darr	|d	|a	|t	|r	|r	|[20][S]\drarr	|k	|y	|z	|y	|ł	|*	|o	|[21][S]\darr	|*	|*	|.
|*	|d	|y	|n	|j	|*	|v	|ź	|s	|y	|y	|y	|k	|*	|[22][S]\darr	|a	|*	|*	|*	|p	|ś	|*	|*	|.
|*	|o	|d	|a	|a	|*	|i	|a	|y	|r	|n	|s	|o	|*	|p	|b	|*	|[23][S]\darr	|*	|e	|r	|*	|*	|.
|*	|t	|r	|*	|*	|[24][S]\darr	|b	|l	|c	|a	|d	|t	|m	|*	|a	|l	|*	|s	|*	|r	|ó	|*	|*	|.
|*	|u	|o	|*	|*	|o	|r	|g	|z	|n	|z	|a	|e	|[25][S]\rarr	|p	|o	|d	|k	|ł	|a	|d	|*	|*	|.
|*	|k	|l	|*	|*	|m	|a	|a	|n	|*	|a	|*	|d	|*	|a	|n	|*	|a	|*	|t	|p	|*	|*	|.
|*	|a	|i	|*	|[26][S]\drarr	|a	|m	|u	|a	|y	|*	|*	|i	|*	|*	|o	|[27][S]\darr	|f	|*	|y	|o	|*	|*	|.
|*	|n	|t	|*	|s	|c	|*	|n	|*	|*	|*	|*	|a	|*	|[28][S]\darr	|w	|p	|a	|[29][S]\darr	|w	|ś	|*	|*	|.
|*	|[][,]{ }	|y	|*	|c	|n	|*	|*	|*	|*	|*	|*	|n	|*	|d	|o	|u	|n	|p	|n	|c	|*	|*	|.
|*	|n	|c	|*	|*	|i	|*	|*	|[30][S]\darr	|[31][S]\rarr	|l	|i	|t	|*	|i	|ś	|l	|d	|r	|e	|i	|*	|*	|.
|*	|i	|z	|*	|[32][S]\rarr	|c	|r	|u	|m	|b	|*	|*	|k	|*	|e	|ć	|a	|e	|o	|*	|e	|*	|*	|.
|*	|e	|n	|[33][S]\rarr	|d	|a	|n	|d	|a	|k	|o	|z	|a	|u	|r	|*	|r	|r	|f	|[34][S]\darr	|*	|*	|*	|.
|[35][S]\drarr	|b	|y	|ł	|y	|*	|[36][S]\rarr	|s	|z	|y	|k	|*	|*	|*	|e	|*	|d	|*	|i	|r	|*	|*	|*	|.
|b	|i	|*	|*	|*	|[37][S]\rarr	|t	|ł	|u	|s	|t	|y	|[][,]{ }	|c	|z	|w	|a	|r	|t	|e	|k	|*	|*	|.
|a	|e	|[38][S]\rarr	|t	|e	|l	|e	|g	|r	|a	|f	|i	|a	|*	|a	|[39][S]\darr	|*	|*	|*	|a	|*	|*	|*	|.
|ń	|s	|*	|[40][S]\rarr	|p	|r	|e	|z	|e	|s	|ó	|w	|n	|a	|*	|w	|*	|*	|*	|*	|*	|*	|*	|.
|k	|k	|[41][S]\rarr	|f	|i	|o	|ł	|e	|k	|[][,]{ }	|a	|f	|r	|y	|k	|a	|ń	|s	|k	|i	|*	|*	|*	|.
|a	|i	|*	|[42][S]\rarr	|b	|e	|ł	|t	|*	|*	|[43][S]\rarr	|p	|i	|j	|a	|n	|i	|c	|a	|*	|*	|*	|*	|.
|*	|*	|*	|*	|*	|*	|*	|*	|*	|*	|*	|*	|*	|*	|*	|*	|*	|*	|*	|*	|*	|*	|*	|.\end{Puzzle}

\newpage

\begin{PuzzleClues}{\textbf{Poziome}\\}\Clue{1}{}{komputer osobisty, ang. personal computer}
\Clue{2}{}{krawędź przedmiotu lub jego obrysu}
\Clue{3}{}{tradycyjna japońska jednostka miary, wynosząca około 0,303 cm}
\Clue{6}{}{mieszkaniec Dublinu}
\Clue{8}{}{kucharz okrętowy - osoba gotująca dla załogi na jednostce pływającej, legitymująca się świadectwem wydanym przez urząd morski}
\Clue{10}{}{Hebeloma - rodzaj grzybów należący do rodziny zasłonakowatych}
\Clue{13}{}{WIŚNIAK; napój alkoholowy z przefermentowanych wiśni}
\Clue{14}{}{część nefronu, w której mocz pierwotny odprowadzany z ciałka nerkowego ulega resorpcji i sekrecji kanalikowej, przekształcając się w mocz ostateczny}
\Clue{15}{}{określenie wykonawcze: mniej np. mniej głośno (MENO FORTE)}
\Clue{17}{}{astronom niemiecki (1572-1625); opracował atlas nieba}
\Clue{20}{}{miasto w azjatyckiej części Federacji Rosyjskiej nad Jenisejem, w Kyzył znajduje się geograficzny środek Azji}
\Clue{25}{}{podkład kolejowy - element toru kolejowego składający się z poprzecznie ułożonych belek, na których za pomocą specjalnych przytwierdzeń mocuje się szyny}
\Clue{26}{}{port naftowy w Wenezueli (Falcon) nad Zatoką Wenezuelską, na płw. Paraguana}
\Clue{31}{}{jednostka monetarna Litwy}
\Clue{32}{}{kompozytor amerykański ur. 1929 r., utwory wokalno-instrumentalne kameralne, fortepianowe}
\Clue{33}{}{Dandakosaurus - rodzaj dwunożnego, mięsożernego dinozaura z grupy teropodów; żył w okresie wczesnej jury na terenach subkontynentu indyjskiego}
\Clue{35}{}{były partner - osoba, która była kiedyś w związku, lecz teraz już nie jest}
\Clue{36}{}{elegancja, wytworność w stroju i zachowaniu}
\Clue{37}{}{ostatni czwartek karnawału}
\Clue{38}{}{dziedzina telekomunikacji zajmująca się przekazem informacji w postaci znaków pisma przygotowanych do odbioru automatycznego albo bezpośrednio przez człowieka za pomocą przyrządu zwanego telegrafem, zastąpionego przez dalekopis}
\Clue{40}{}{córka prezesa}
\Clue{41}{}{sępolia fiołkowa, Saintpaulia ionantha, gatunek ozdobnej rośliny wieloletniej z rodziny ostrojowatych; uprawiana jako roślina doniczkowa; kwitnie przez cały rok - najobficiej latem}
\Clue{42}{}{substancja płynna lub półpłynna, kleista, często powtała ze zmieszania innych substancji, odbierana jako nieprzyjemna}
\Clue{43}{}{pogardliwie o pijaku}\end{PuzzleClues}

\begin{PuzzleClues}{\textbf{Pionowe}\\}\Clue{1}{}{zabieg chirurgiczny wykonywany zależnie od potrzeb za pomocą pomocą świdra lub trepanu, dłut i kleszczy kostnych}
\Clue{2}{}{gwiazda wybuchowa, w rzeczywistości ciasny układ podwójny złożony z białego karła i gwiazdy ciągu głównego lub nieco odewoluowanej gwiazdy}
\Clue{3}{}{w górnictwie: część zasobów przemysłowych - pomniejszonych o straty ponoszone przy wydobyciu}
\Clue{4}{}{enzym katalizujący rozkładanie związków złożonych do prostych w procesie hydrolizy}
\Clue{5}{}{końcowy odcinek ekspozycji i repryzy w formie sonatowej, niekiedy końcowy fragment utworu muzycznego, zwłaszcza opery}
\Clue{6}{}{kolekcjoner etykiet od piwa}
\Clue{7}{}{ognisko martwicy przechodzące w owrzodzenie skóry, tkanki podskórnej, a nawet mięśni i kości, przeważnie wtórnie zakażone, które powstaje na skutek długotrwałego ucisku i związanego z tym niedotlenienia tkanek}
\Clue{9}{}{ochrona przyznawana przez monarchę pewnym osobom lub miejscom}
\Clue{11}{}{Nanotyrannus - rodzaj dinozaura z rodziny tyranozaurów; żył w późnej kredzie na terenach Ameryki Północnej}
\Clue{12}{}{krzywa płaska algebraiczna}
\Clue{13}{}{to, że ktoś/coś jest nieszablonowe, nieschematyczne}
\Clue{15}{}{gatunek średniego ptaka z rodziny tukanowatych (Ramphastidae)}
\Clue{16}{}{kiepska sytuacja, w której zwykle występuje brak czegoś}
\Clue{18}{}{miasto w Indiach (Maharasztra); przemysł bawełniany}
\Clue{19}{}{mieszanka gumowa produkowana przez włoską firmę Vibram, używana do produkcji podeszw butów}
\Clue{20}{}{kobieta, która zachowuje się niepoważnie, budzi politowanie; artystka, osoba, która bierze udział w splocie okoliczności, który można nazwać farsą}
\Clue{21}{}{środek Wielkiego Postu przypadający na jego czwartą niedzielę}
\Clue{22}{}{tektura nasycona masą smołową stosowana między innymi do pokrycia dachów}
\Clue{23}{}{szczelny, odpowiednio wyposażony ubiór umożliwiający przebywanie człowieka w środowisku, do którego nie jest przystosowany organizm ludzki np. pod wodą, w kosmosie itp}
\Clue{24}{}{motyl nocny, gąsienice roślinożerne}
\Clue{26}{}{w chemii: symbol skandu}
\Clue{27}{}{młoda, nierozwinięta płciowo kura, tuczona w określony sposób celem uzyskania delikatnego i kruchego mięsa}
\Clue{28}{}{cezura wypadająca na granicy stóp}
\Clue{29}{}{korzyść, zysk}
\Clue{30}{}{polski taniec ludowy tańczony w środkowej części Polski}
\Clue{34}{}{ur.w 1921 r nowelista włoski, opowiadania o tematyce społecznej, często z życia Neapolu}
\Clue{35}{}{szczelne naczynie szklane mieszczące żarnik lub elektrody lampy}
\Clue{39}{}{sieć komputerowa znajdująca się na obszarze wykraczającym poza jedno miasto (bądź kompleks miejski)}\end{PuzzleClues}\newpage\section*{Krzyżówka 60}

\noindent\begin{Puzzle}{21}{28}|*	|*	|*	|*	|*	|*	|*	|*	|*	|*	|*	|*	|*	|*	|*	|*	|[1][S]\drarr	|l	|y	|o	|t	|*	|.
|*	|*	|[2][S]\drarr	|p	|r	|z	|e	|d	|o	|d	|w	|ł	|o	|k	|*	|*	|ł	|*	|*	|*	|*	|*	|.
|*	|[3][S]\rarr	|p	|ą	|c	|z	|u	|ś	|*	|*	|*	|*	|*	|[4][S]\rarr	|c	|i	|u	|ć	|m	|a	|*	|*	|.
|*	|[5][S]\drarr	|r	|u	|t	|y	|n	|i	|s	|t	|a	|*	|*	|*	|[6][S]\rarr	|u	|p	|i	|ó	|r	|*	|*	|.
|[7][S]\rarr	|d	|o	|s	|t	|o	|j	|n	|i	|k	|[][,]{ }	|k	|o	|ś	|c	|i	|e	|l	|n	|y	|*	|[8][S]\darr	|.
|*	|e	|z	|*	|*	|*	|*	|[9][S]\rarr	|b	|e	|r	|e	|z	|o	|w	|s	|k	|a	|*	|*	|*	|z	|.
|[10][S]\drarr	|n	|o	|ż	|y	|c	|e	|[][,]{ }	|c	|e	|n	|o	|w	|e	|*	|*	|[][,]{ }	|[11][S]\darr	|*	|*	|*	|b	|.
|z	|i	|d	|*	|*	|*	|*	|*	|*	|*	|[12][S]\darr	|*	|*	|[13][S]\darr	|*	|*	|k	|j	|*	|[14][S]\darr	|*	|ó	|.
|o	|t	|i	|[15][S]\rarr	|l	|e	|g	|i	|t	|y	|m	|a	|c	|j	|a	|*	|w	|u	|[16][S]\darr	|l	|*	|j	|.
|o	|r	|a	|*	|*	|*	|*	|*	|*	|*	|a	|[17][S]\darr	|[18][S]\drarr	|a	|l	|t	|a	|n	|n	|i	|k	|*	|.
|*	|y	|*	|*	|*	|*	|*	|*	|*	|*	|u	|a	|l	|ł	|*	|[19][S]\darr	|r	|a	|i	|m	|*	|*	|.
|[20][S]\rarr	|f	|a	|k	|o	|s	|z	|e	|r	|*	|z	|r	|e	|o	|*	|f	|c	|k	|ć	|b	|[21][S]\darr	|*	|.
|*	|i	|*	|*	|*	|*	|*	|*	|*	|*	|e	|t	|k	|w	|*	|r	|o	|*	|*	|u	|a	|[22][S]\darr	|.
|[23][S]\drarr	|k	|o	|r	|e	|k	|t	|o	|r	|*	|r	|[][,]{ }	|t	|o	|*	|y	|w	|*	|[24][S]\darr	|r	|r	|k	|.
|p	|a	|*	|*	|*	|*	|*	|*	|[25][S]\darr	|*	|*	|r	|u	|ś	|*	|g	|o	|[26][S]\darr	|s	|g	|g	|r	|.
|e	|t	|[27][S]\drarr	|s	|k	|r	|z	|e	|k	|*	|*	|o	|r	|ć	|*	|i	|[][S]-	|k	|a	|i	|e	|y	|.
|t	|o	|s	|*	|*	|[28][S]\rarr	|s	|t	|a	|n	|i	|c	|a	|*	|*	|j	|s	|i	|m	|a	|n	|z	|.
|a	|r	|i	|[29][S]\darr	|[30][S]\rarr	|ś	|w	|i	|s	|t	|a	|k	|*	|[31][S]\darr	|[32][S]\darr	|c	|e	|n	|o	|*	|t	|a	|.
|r	|*	|u	|g	|[33][S]\rarr	|l	|a	|k	|t	|y	|d	|*	|[34][S]\drarr	|b	|e	|z	|r	|o	|l	|n	|y	|*	|.
|d	|*	|s	|o	|*	|*	|*	|*	|y	|*	|*	|[35][S]\darr	|b	|o	|k	|y	|y	|m	|u	|*	|ń	|*	|.
|a	|*	|i	|ł	|*	|*	|*	|*	|l	|*	|*	|p	|o	|l	|s	|c	|c	|e	|b	|*	|s	|*	|.
|*	|*	|u	|ą	|[36][S]\rarr	|f	|a	|k	|i	|r	|*	|u	|a	|i	|p	|y	|y	|c	|n	|*	|k	|*	|.
|*	|*	|m	|b	|[37][S]\rarr	|k	|o	|c	|a	|n	|k	|a	|*	|w	|u	|*	|t	|h	|y	|*	|o	|*	|.
|[38][S]\drarr	|j	|a	|k	|o	|b	|i	|n	|*	|*	|*	|z	|*	|a	|l	|*	|o	|a	|[][,]{ }	|*	|ś	|*	|.
|b	|*	|j	|i	|*	|*	|*	|*	|*	|*	|*	|*	|*	|r	|s	|*	|w	|n	|d	|*	|ć	|*	|.
|ą	|*	|t	|*	|[39][S]\rarr	|ł	|u	|s	|z	|c	|z	|k	|a	|*	|j	|*	|y	|i	|n	|*	|*	|*	|.
|k	|*	|e	|*	|*	|*	|*	|[40][S]\rarr	|f	|o	|t	|o	|g	|r	|a	|f	|*	|k	|a	|*	|*	|*	|.
|*	|[41][S]\rarr	|k	|i	|e	|s	|z	|e	|n	|i	|ó	|w	|k	|a	|*	|*	|*	|*	|*	|*	|*	|*	|.
|*	|*	|*	|*	|*	|*	|*	|*	|*	|*	|*	|*	|*	|*	|*	|*	|*	|*	|*	|*	|*	|*	|.\end{Puzzle}

\newpage

\begin{PuzzleClues}{\textbf{Poziome}\\}\Clue{1}{}{Bernard, ur. w 1897r. astronom francuski - koronograf}
\Clue{2}{}{przednia część odwłoku niektórych stawonogów (m.in. skorpionów)}
\Clue{3}{}{o człowieku, który ma okrągłą, pyzatą buzię}
\Clue{4}{}{obraźliwie o kobiecie nierozgarniętej, głupiej}
\Clue{5}{}{człowiek, który wciąż robi to samo, według schematu, nie wprowadza zmian}
\Clue{6}{}{groźny dla ludzi duch osoby zmarłej}
\Clue{7}{}{osoba piastująca wysokie stanowisko w hierachii kościelnej}
\Clue{9}{}{Pedro, zm. w 1506 r. malarz hiszpański ojciec Alonso; obrazy i freski religijne, portrety}
\Clue{10}{}{relacja cen artykułów sprzedawanych przez dany podmiot (przedsiębiorstwo, kraj, a nawet grupę krajów) do cen towarów kupowanych przez niego (cen środków produkcji)}
\Clue{15}{}{przestarz}
\Clue{18}{}{BUDNIK}
\Clue{20}{}{GUZIEC; łowny gatunek afrykańskiej świni; na głowie rogowe wyrostki}
\Clue{23}{}{przyrząd piśmienniczy, w postaci pióra, płynu w opakowaniu z pędzlem lub w postacipaska; używa się go do korygowania błędów piśmienniczych}
\Clue{27}{}{nieprzyjemny, przypominający zwierzęcy, głos ludzki}
\Clue{28}{}{kozacka osada, na której czele stał ataman}
\Clue{30}{}{chroniony gryzoń górski z rodziny wiewiórek, roślinożerny, przesypia zimę}
\Clue{33}{}{heterocykliczny związek organiczny z grupy estrów}
\Clue{34}{}{chłop, który posiadał tylko chałupę i niewielki ogródek, nie miał własnej ziemi uprawnej}
\Clue{36}{}{indyjski asceta, który poprzez hartowanie swojego ciała i umysłu zyskuje odporność na kontakt z gorącymi i ostrymi przedmiotami, ogniem itp}
\Clue{37}{}{Helichrysum - roślina z rodziny astrowatych, występuje głównie na obszarach o ciepłym klimacie, głównie w południowej Afryce i Australii, w basenie Morza Śródziemnego, w zachodniej i środkowej Azji i w Nowej Zelandii; nazwa zwykle występuje w liczbie mnogiej}
\Clue{38}{}{członek francuskiego, lewicowego klubu jakobinów}
\Clue{39}{}{u gatunków z rodziny wiechlinowatych drobny delikatny listek wewnętrznego okółka okwiatu}
\Clue{40}{}{osoba, która robi zdjęcia (również amatorsko)}
\Clue{41}{}{rodzaj podszewki, która jest wnętrzem niektórych typów kieszeni}\end{PuzzleClues}

\begin{PuzzleClues}{\textbf{Pionowe}\\}\Clue{1}{}{skała metamorficzna z licznymi drobnymi żyłkami kwarcowymi o nieregularnym przebiegu}
\Clue{2}{}{wszelkie brzmieniowe aspekty funkcjonowania języka, które są typowe dla ciągów dźwięków, np. sylab, w toku wypowiedzi}
\Clue{5}{}{filtr akwarystyczny ostatniego etapu oczyszczania wody, który usuwa azotany i część forforanów, pozwalając na rzadsze wymiany wody}
\Clue{8}{}{przestępca siejący postrach, zamieszkujący często w lesie, z dala od siedlisk ludzkich}
\Clue{10}{}{teren udostępniony odwiedzającym, na którym hodowane są zwierzęta, najczęściej pochodzące z różnych obszarów geograficznych}
\Clue{11}{}{polski samolot, dzieło T. Sołtyka}
\Clue{12}{}{typ niemieckiego pistoletu automatycznego}
\Clue{13}{}{czczość, brak celowości, pożytku płynącego z czegoś}
\Clue{14}{}{kraina historyczna w Niderlandach}
\Clue{16}{}{przenośnie: jakieś połączenie pomiędzy dwiema rzeczami lub dwoma osobami, często relacja związana z komunikacją}
\Clue{17}{}{gatunek muzyki rockowej charakteryzujący się różnorodnością i wielowątkowością muzycznych form}
\Clue{18}{}{zajęcia czytania}
\Clue{19}{}{starożytny lud indoeuropejski pochodzący z Bałkanów; przed końcem drugiego tysiąclecia Frygijczycy osiedlili się w środkowej Anatolii, głównie w zakolu rzeki Halys, na terenach poprzednio zajmowanych przez Hetytów}
\Clue{21}{}{zespół cech typowych dla Argentyńczyka lub czegoś argentyńskiego}
\Clue{22}{}{względnie szeroki brzeg tylnej części głowy gadów, który może posiadać szkielet kostny (jak u dinozaurów z grupy marginocefali) lub chrzęstny (jak u agamy kołnierzastej, australijskiej jaszczurki, u której na rusztowaniu chrzęstnym rozpościerają się płaty skórne)}
\Clue{23}{}{wyrób pirotechniczny wykorzystywany do celów widowiskowych, po zapaleniu dający efekty świetlne i akustyczne}
\Clue{24}{}{jednostka DNA, która stanowiąc część genomu nie przynosi korzyści organizmowi, jak ma to miejsce w przypadku większości genów wpływających na fenotyp}
\Clue{25}{}{Stara; region w środkowej Hiszpanii, nad Zatoką Biskajską powierzchnia 66,1 tyś. km2, główne miasta: Valladolid, Santander}
\Clue{26}{}{specjalista w dziedzinie budowy, naprawy i konserwacji aparatury kinowej}
\Clue{27}{}{młody, niedojrzały chłopak lub dzieciak; słowo negatywne}
\Clue{29}{}{potrawa z mielonego mięsa zmieszanego z kaszą lub ryżem zawijanego w liście kapusty}
\Clue{31}{}{stan w Wenezueli}
\Clue{32}{}{wydalenie dyplomaty, uznanego za persona non grata}
\Clue{34}{}{długi, wąski szal z futra, strusich piór lub puchu}
\Clue{35}{}{P - jednostka lepkości dynamicznej w układzie jednostek miar CGS, nazwana na cześć francuskiego fizyka i lekarza Jeana L. M. Poiseuille'a. 1 P = 1 dyn·s/cm2 = 1 g·/(cm·s)}
\Clue{38}{}{cyga}\end{PuzzleClues}\newpage\section*{Krzyżówka 61}

\noindent\begin{Puzzle}{25}{27}|*	|*	|*	|*	|*	|*	|*	|*	|*	|*	|*	|*	|[1][S]\darr	|*	|*	|*	|*	|*	|[2][S]\drarr	|k	|a	|l	|i	|n	|a	|*	|.
|*	|*	|*	|*	|*	|[3][S]\darr	|*	|*	|*	|*	|*	|*	|p	|*	|*	|*	|*	|*	|p	|*	|*	|[4][S]\darr	|[5][S]\darr	|*	|[6][S]\darr	|*	|.
|*	|*	|*	|*	|*	|p	|[7][S]\rarr	|r	|o	|z	|d	|z	|i	|a	|ł	|e	|k	|*	|u	|*	|*	|h	|a	|[8][S]\darr	|l	|*	|.
|*	|*	|*	|*	|*	|a	|*	|*	|*	|[9][S]\drarr	|l	|i	|n	|u	|x	|*	|*	|*	|t	|*	|*	|e	|r	|k	|u	|*	|.
|*	|*	|*	|*	|*	|n	|*	|*	|*	|p	|*	|*	|g	|*	|*	|*	|*	|*	|t	|*	|*	|t	|c	|o	|l	|*	|.
|*	|*	|*	|*	|*	|c	|*	|*	|*	|o	|*	|*	|w	|*	|*	|*	|*	|*	|o	|*	|*	|e	|h	|c	|e	|*	|.
|*	|*	|*	|*	|*	|z	|*	|[10][S]\rarr	|k	|r	|o	|w	|i	|e	|n	|i	|e	|c	|*	|*	|*	|r	|i	|h	|k	|[11][S]\darr	|.
|*	|*	|*	|[12][S]\darr	|*	|e	|*	|[13][S]\rarr	|c	|z	|y	|n	|n	|o	|ś	|ć	|[][,]{ }	|r	|u	|c	|h	|o	|w	|a	|*	|s	|.
|*	|*	|*	|ł	|*	|n	|*	|*	|*	|ą	|*	|*	|y	|*	|*	|*	|*	|[14][S]\darr	|*	|*	|*	|d	|o	|ś	|*	|t	|.
|*	|*	|[15][S]\darr	|o	|*	|i	|*	|*	|*	|d	|*	|*	|*	|*	|[16][S]\darr	|*	|*	|o	|*	|[17][S]\darr	|*	|u	|z	|*	|[18][S]\darr	|e	|.
|*	|*	|r	|w	|[19][S]\rarr	|s	|p	|o	|d	|e	|k	|*	|*	|*	|h	|*	|*	|g	|*	|d	|*	|p	|n	|[20][S]\darr	|b	|p	|.
|*	|*	|y	|c	|*	|t	|*	|*	|[21][S]\rarr	|k	|s	|i	|ę	|g	|a	|[][,]{ }	|h	|o	|d	|o	|w	|l	|a	|n	|a	|*	|.
|*	|[22][S]\drarr	|j	|a	|n	|a	|ć	|e	|k	|*	|*	|*	|*	|*	|m	|*	|[23][S]\darr	|n	|*	|j	|*	|e	|w	|a	|t	|*	|.
|*	|m	|ó	|*	|*	|*	|*	|*	|*	|*	|*	|*	|*	|*	|u	|*	|o	|o	|*	|r	|*	|k	|s	|k	|o	|*	|.
|*	|e	|w	|*	|*	|*	|*	|*	|*	|*	|*	|*	|*	|*	|l	|*	|p	|p	|*	|z	|*	|s	|t	|i	|f	|*	|.
|*	|d	|k	|*	|*	|*	|*	|*	|*	|*	|*	|*	|*	|*	|e	|*	|e	|i	|[24][S]\darr	|a	|*	|*	|w	|e	|o	|*	|.
|*	|i	|i	|*	|[25][S]\rarr	|k	|o	|z	|i	|o	|r	|o	|ż	|e	|c	|*	|r	|ó	|s	|ł	|*	|*	|o	|r	|b	|*	|.
|*	|o	|[][,]{ }	|*	|*	|[26][S]\rarr	|w	|i	|ę	|z	|a	|d	|ł	|o	|[][,]{ }	|k	|a	|r	|k	|o	|w	|e	|*	|o	|i	|*	|.
|*	|z	|n	|*	|*	|*	|*	|*	|*	|*	|*	|*	|*	|*	|p	|*	|c	|[][,]{ }	|r	|ś	|*	|*	|*	|w	|a	|*	|.
|[27][S]\rarr	|n	|a	|b	|ó	|j	|*	|*	|*	|*	|*	|*	|*	|*	|o	|*	|j	|u	|e	|ć	|*	|*	|*	|a	|*	|*	|.
|*	|a	|d	|*	|*	|*	|*	|*	|*	|*	|*	|*	|*	|*	|m	|*	|a	|s	|c	|*	|*	|*	|*	|n	|*	|*	|.
|*	|w	|r	|*	|[28][S]\rarr	|u	|p	|o	|k	|o	|r	|z	|e	|n	|i	|e	|*	|z	|z	|*	|*	|*	|*	|i	|*	|*	|.
|*	|s	|z	|*	|*	|*	|*	|*	|[29][S]\rarr	|p	|a	|r	|a	|s	|a	|n	|g	|a	|*	|*	|*	|*	|*	|e	|*	|*	|.
|*	|t	|e	|*	|*	|*	|*	|*	|*	|*	|[30][S]\rarr	|c	|h	|a	|r	|l	|e	|s	|*	|*	|*	|*	|*	|[][,]{ }	|*	|*	|.
|*	|w	|w	|*	|*	|*	|[31][S]\rarr	|n	|i	|e	|j	|e	|d	|n	|o	|l	|i	|t	|o	|ś	|ć	|*	|*	|s	|*	|*	|.
|*	|o	|n	|*	|*	|*	|*	|*	|*	|*	|*	|*	|*	|*	|w	|*	|*	|y	|*	|*	|*	|*	|*	|i	|*	|*	|.
|*	|*	|e	|*	|*	|*	|*	|*	|*	|*	|*	|*	|*	|*	|y	|*	|*	|*	|*	|*	|*	|*	|*	|ę	|*	|*	|.
|*	|*	|*	|*	|[32][S]\rarr	|t	|r	|a	|n	|z	|y	|c	|j	|a	|*	|*	|*	|*	|*	|*	|*	|*	|*	|*	|*	|*	|.\end{Puzzle}

\newpage

\begin{PuzzleClues}{\textbf{Poziome}\\}\Clue{2}{}{owoc (pestkowiec) kaliny koralowej, jadalny, jeśli zbierany po pierwszym przymrozku}
\Clue{7}{}{pasek, linia, która oddziela części włosów zaczesane lub ułożone w inne strony}
\Clue{9}{}{każdy system z rodziny uniksopodobnych systemów operacyjnych opartych na jądrze Linux}
\Clue{10}{}{kał bydlęcy wymieszany z wodą i poddany procesowi fermentacji; jest szczególnie korzystny w nawożeniu upraw ogrodniczych}
\Clue{13}{}{ruchy wewnętrznych narządów człowieka, np. języka czy układu pokarmowego}
\Clue{19}{}{mały talerzyk; czasem może być podstawką pod filiżankę}
\Clue{21}{}{książka, kartoteka lub informatyczny nośnik danych, do których są wpisywane oraz rejestrowane, zwierzęta hodowlane oraz informacje o ich hodowcach, właścicielach, pochodzeniu oraz wynikach oceny wartości użytkowej lub wartości hodowlanej}
\Clue{22}{}{kompozytor czeski (1854-1928); przedstawiciel narodowej szkoły czeskiej; opery 'Jenufa', 'Katia Kabanowa'}
\Clue{25}{}{znak zodiaku, który przypisuje się osobom urodzonym pomiędzy 22 grudnia a 20 stycznia}
\Clue{26}{}{więzadło biegnące od linii środkowej potylicy, w rozwidleniu wyrostków kolczystych i łączące się z górną częścią mięśnia czworobocznego, przyczepia się do grzebienia potylicznego zewnętrznego, pomaga w stabilizacji głowy}
\Clue{27}{}{jednostka amunicji broni palnej}
\Clue{28}{}{poczucie wstydu na skutek obrazy lub niespełnionej ambicji}
\Clue{29}{}{perska mila}
\Clue{30}{}{francuski fizyk i chemik (1746-1823); zbudowanym przez siebie balonem wzniósł się na wysokość 3000 m (1783 m)}
\Clue{31}{}{różnorodność, zmienność}
\Clue{32}{}{przejście człowieka do innego wymiaru, np. po śmierci jego ciała ziemskiego}\end{PuzzleClues}

\begin{PuzzleClues}{\textbf{Pionowe}\\}\Clue{1}{}{bezlotki, Spheniscidae - rodzina wodnych ptaków nielatających z rzędu pingwinów (Sphenisciformes)}
\Clue{2}{}{Ludwik (1877-1942) rzeźbiarz, malarz i historyk sztuki, zginął w obozie w Oświęcimiu}
\Clue{3}{}{sportowiec uprawiający łyżwiarstwo szybkie}
\Clue{4}{}{dwuniciowa cząsteczka kwasu nukleinowego powstała przez rekombinację genetyczną pojedynczych, komplementarnych nici pochodzących z różnych źródeł}
\Clue{5}{}{dział archiwistyki, nauka o dziejach i zasobach poszczególnych archiwów}
\Clue{6}{}{eurazjatycka trująca roślina zielna o nieprzyjemnej woni, liście zawierają lecznicze alkaloidy}
\Clue{8}{}{protekcjonalny lub żartobliwy sposób zwrócenia się do mężczyzny, chłopaka}
\Clue{9}{}{relacja pomiędzy elementami, która sprawia, że odbieramy coś jako uporządkowane}
\Clue{11}{}{krok taneczny, polegający na wybijaniu rytmu stopami obutymi w buty z blaszkami; jest osią tańca o tej samej nazwie, a jego elementy można spotkać w różnych innych stylach choreograficznych bądź tańcach}
\Clue{12}{}{człowiek, który czegoś poszukuje (np. łowca okazji cenowych)}
\Clue{14}{}{Ptilocercus lowii - gatunek niewielkiego ssaka, jedynego przedstawiciela rodzaju Ptilocercus i rodziny ogonopiórowatych; żyje na Sumatrze, Malezji, Tajlandii, Borneo i na sąsiednich wyspach, zasiedlając lasy, plantacje kauczuku i domostwa w pobliżu terenów zalesionych}
\Clue{15}{}{wiewióreczniki, tupaje, Scandentia - rząd ssaków łożyskowych z nadrzędu euarchontów, blisko spokrewnionych z skóroskrzydłymi i naczelnymi; występują w wilgotnych lasach równikowych Azji Południowo-Wschodniej, m.in. na Archipelagu Malajskim, Filipinach i na Półwyspie Indyjskim}
\Clue{16}{}{dynamometryczny; urządzenie do pomiaru momentu obrotowego działającego na wał maszyny}
\Clue{17}{}{cecha człowieka, który osiągnął dorosłość, pełnoletniość}
\Clue{18}{}{chorobliwy lęk przed głębokością}
\Clue{20}{}{doworzenie, zarządzanie kimś lub czymś wystarczająco długo}
\Clue{22}{}{nauka badająca media}
\Clue{23}{}{w nauce, technice, technologii - działanie osoby lub maszyny, będące elementem bardziej złożonego procesu}
\Clue{24}{}{dźwięk drapania igłą po płycie winylowej}\end{PuzzleClues}\newpage\section*{Krzyżówka 62}

\noindent\begin{Puzzle}{20}{31}|*	|*	|*	|*	|*	|*	|*	|*	|*	|[1][S]\darr	|*	|*	|*	|*	|*	|*	|*	|*	|[2][S]\darr	|*	|*	|.
|[3][S]\rarr	|t	|e	|r	|e	|n	|[][,]{ }	|o	|t	|w	|a	|r	|t	|y	|*	|[4][S]\rarr	|p	|o	|l	|e	|*	|.
|*	|[5][S]\rarr	|c	|u	|k	|i	|e	|r	|[][,]{ }	|z	|ł	|o	|ż	|o	|n	|y	|*	|[6][S]\drarr	|o	|ś	|*	|.
|*	|[7][S]\darr	|*	|[8][S]\rarr	|r	|a	|j	|a	|[][,]{ }	|m	|o	|t	|y	|l	|*	|[9][S]\darr	|[10][S]\darr	|k	|n	|*	|*	|.
|*	|s	|[11][S]\darr	|[12][S]\drarr	|t	|r	|z	|y	|n	|a	|s	|t	|k	|a	|*	|p	|a	|o	|g	|*	|*	|.
|*	|i	|k	|i	|*	|[13][S]\drarr	|f	|a	|r	|c	|i	|a	|r	|a	|*	|a	|n	|n	|s	|*	|*	|.
|[14][S]\rarr	|m	|i	|n	|e	|s	|t	|r	|o	|n	|e	|*	|*	|*	|*	|n	|t	|g	|h	|[15][S]\darr	|*	|.
|*	|c	|c	|f	|*	|k	|*	|*	|*	|i	|*	|*	|*	|*	|[16][S]\darr	|c	|a	|o	|i	|w	|*	|.
|*	|a	|z	|l	|*	|a	|*	|[17][S]\drarr	|f	|a	|r	|t	|u	|s	|z	|e	|k	|*	|p	|s	|*	|.
|*	|*	|a	|a	|*	|r	|*	|k	|*	|c	|*	|*	|[18][S]\rarr	|b	|a	|r	|y	|e	|*	|p	|*	|.
|*	|*	|*	|c	|*	|p	|*	|i	|*	|z	|*	|*	|*	|*	|m	|z	|a	|*	|*	|ó	|*	|.
|*	|*	|*	|j	|[19][S]\darr	|e	|*	|r	|*	|[][,]{ }	|*	|*	|*	|*	|e	|*	|*	|[20][S]\darr	|*	|ł	|[21][S]\darr	|.
|*	|*	|*	|a	|p	|t	|*	|o	|*	|e	|*	|*	|[22][S]\rarr	|s	|k	|r	|a	|w	|*	|w	|m	|.
|*	|*	|*	|[][,]{ }	|o	|k	|[23][S]\drarr	|w	|o	|l	|n	|o	|ś	|ć	|*	|*	|[24][S]\darr	|y	|*	|ł	|ł	|.
|*	|*	|*	|m	|l	|a	|r	|s	|*	|e	|*	|*	|*	|*	|*	|*	|m	|s	|*	|a	|o	|.
|*	|*	|[25][S]\darr	|o	|n	|*	|e	|k	|[26][S]\rarr	|k	|r	|ó	|w	|k	|a	|*	|i	|t	|*	|ś	|d	|.
|*	|*	|i	|n	|i	|*	|u	|*	|[27][S]\darr	|t	|*	|*	|*	|*	|*	|*	|o	|a	|*	|c	|z	|.
|*	|*	|n	|e	|k	|*	|m	|[28][S]\drarr	|t	|r	|z	|o	|n	|e	|k	|*	|d	|w	|*	|i	|i	|.
|*	|*	|t	|t	|[][,]{ }	|[29][S]\rarr	|a	|t	|r	|o	|p	|i	|n	|a	|*	|[30][S]\darr	|o	|a	|[31][S]\darr	|c	|e	|.
|*	|*	|e	|a	|z	|*	|t	|e	|z	|m	|*	|*	|*	|*	|*	|s	|j	|*	|s	|i	|ż	|.
|*	|*	|l	|r	|i	|*	|o	|m	|m	|a	|*	|*	|*	|*	|*	|z	|a	|*	|a	|e	|o	|.
|*	|*	|e	|n	|e	|*	|l	|a	|i	|s	|*	|*	|*	|*	|*	|c	|d	|[32][S]\darr	|g	|l	|w	|.
|*	|*	|k	|a	|m	|*	|o	|t	|e	|z	|*	|*	|*	|*	|*	|z	|[][,]{ }	|f	|i	|s	|i	|.
|*	|*	|t	|*	|n	|*	|g	|*	|l	|y	|*	|*	|*	|*	|*	|e	|w	|a	|n	|t	|e	|.
|*	|[33][S]\darr	|u	|*	|o	|*	|i	|*	|*	|n	|*	|*	|*	|*	|*	|r	|y	|k	|a	|w	|c	|.
|*	|p	|a	|*	|w	|*	|a	|[34][S]\rarr	|p	|o	|d	|c	|h	|l	|e	|b	|s	|t	|w	|o	|*	|.
|*	|l	|l	|*	|o	|*	|*	|*	|[35][S]\rarr	|w	|y	|m	|ó	|g	|*	|a	|p	|*	|*	|*	|*	|.
|*	|a	|i	|[36][S]\rarr	|d	|e	|p	|o	|z	|y	|t	|[][,]{ }	|b	|a	|n	|k	|o	|w	|y	|*	|*	|.
|[37][S]\rarr	|c	|z	|y	|n	|n	|o	|ś	|ć	|*	|[38][S]\rarr	|s	|t	|i	|l	|*	|w	|*	|*	|*	|*	|.
|*	|k	|m	|*	|y	|[39][S]\rarr	|p	|r	|o	|z	|a	|[][,]{ }	|p	|o	|e	|t	|y	|c	|k	|a	|*	|.
|*	|a	|*	|*	|*	|*	|*	|*	|*	|*	|*	|*	|*	|*	|*	|*	|*	|*	|*	|*	|*	|.
|*	|*	|*	|*	|*	|*	|*	|*	|*	|*	|*	|*	|*	|*	|*	|*	|*	|*	|*	|*	|*	|.\end{Puzzle}

\newpage

\begin{PuzzleClues}{\textbf{Poziome}\\}\Clue{3}{}{obszar, który pozbawiony jest dużych obiektów, nieporośnięty lasami, bez wysokiej zabudowy i okazałych form rzeźby terenu}
\Clue{4}{}{zbiorowisko komórek nerwowych leżących w układzie nerwowym ośrodkowym blisko siebie i sprawujących jednakową czynność}
\Clue{5}{}{polisacharyd, wielocukier - węglowodan, biopolimer składający się z merów będących cukrami prostymi połączonych wiązaniami glikozydowymi}
\Clue{6}{}{centrum jakiejś struktury}
\Clue{8}{}{Raja miraletus - gatunek ryby chrzęstnoszkieletowej z rodziny rajowatych (Rajidae); raja motyl występuje we wschodnim Atlantyku od południowej części Zatoki Biskajskiej do południowej Afryki i - bardzo licznie - w Morzu Śródziemnym, żyje również w Oceanie Indyjskim wzdłuż wybrzeża Afryki, Półwyspu Arabskiego i zachodniej części Półwyspu Indyjskiego}
\Clue{12}{}{trzynastolatka}
\Clue{13}{}{kobieta, która ma szczęście, fart}
\Clue{14}{}{zupa na bazie warzyw, głównie cukinii, fasolki szparagowej, marchwi, zielonego groszku oraz szpinaku, podawana na różne sposoby: z drobnym makaronem, ryżem, grzankami lub jako przecierka warzywna}
\Clue{17}{}{rodzaj mundurka szkolnego}
\Clue{18}{}{francuski rzeźbiarz i malarz (1796-1875) nauczyciel Rodina, rzeźby o tematyce animalistycznej}
\Clue{22}{}{zgrubienie od słowa skrawek}
\Clue{23}{}{brak ograniczeń, możliwość swobodnego działania}
\Clue{26}{}{rodzaj cukierka mlecznego z miękkim, ciągliwym nadzieniem}
\Clue{28}{}{członek męski, penis}
\Clue{29}{}{mieszanina racemiczna dwóch izomerów optycznych hioscyjaminy - organicznego związku chemicznego, będącego alkaloidem tropanowym}
\Clue{34}{}{wypowiedź mająca na celu schlebianie komuś i obliczona na przypodobanie się komuś}
\Clue{35}{}{warunek stawiany przy zawieraniu jakiejś umowy}
\Clue{36}{}{transakcja polegająca na powierzeniu wartości majątkowych bankowi}
\Clue{37}{}{ukierunkowany i zorganizowany proces zmierzający do osiągnięcia jakiegoś celu}
\Clue{38}{}{stilo - nośnik w formie taśmy, służący do rejestrowania i odtwarzania dźwięku}
\Clue{39}{}{typ utworów prozaicznych o charakterze lirycznym, bogatych w metafory i inne figury poetyckie}\end{PuzzleClues}

\begin{PuzzleClues}{\textbf{Pionowe}\\}\Clue{1}{}{prądnica w specjalnym wykonaniu, której zadaniem jest wzmocnienie sygnałów prądu, mocy lub napięcia}
\Clue{2}{}{długi okręt Wikingów (IX-X w)}
\Clue{6}{}{rzeka w Afryce Równikowej, uchodzi do Oceanu Atlantyckiego}
\Clue{7}{}{samochód marki Simca}
\Clue{9}{}{część uzbrojenia ochronnego rycerzy; osłona tułowia}
\Clue{10}{}{miasto w płd. Turcji, ośrodek administracyjny ilu Hatay nad rzeką Asi}
\Clue{11}{}{przedstawiciel grupy Indian zamieszkujących zachodnią i środkową Gwatemalę}
\Clue{12}{}{wzrost cen szesnastowiecznej europie spowodowany m.in. gwałtownym napływem dużej ilości kruszcu i srebra}
\Clue{13}{}{zazwyczaj biała, jasna (kontrastowa - być może mówi się tak również o ciemnej, ale rzadko) łatka na nodze (blisko stopy) zwierzęcia}
\Clue{15}{}{odmiana własności polegająca na tym, że dana rzecz należy niepodzielnie do wszystkich współwłaścicieli, zaś każdemu ze współwłaścicieli przysługują wszystkie atrybuty prawa własności}
\Clue{16}{}{mechanizm lub urządzenie do zamykania drzwi, szuflad, walizek}
\Clue{17}{}{miasto we wsch. części Ukrainy nad Ługaniem}
\Clue{19}{}{karczownik, karczownik ziemnowodny, szczur wodny, Arvicola amphibius - gatunek gryzonia z podrodziny nornikowatych w rodzinie chomikowatych; zamieszkuje brzegi wód Europy oraz północnej i środkowej Azji}
\Clue{20}{}{zbiór przedmiotów przeznaczonych do zaprezentowania publiczności}
\Clue{21}{}{działacz organizacji młodzieżowej}
\Clue{23}{}{dziedzina medycyny zajmująca się schorzeniami reumatycznymi (metabolicznymi, zapalnymi, zwyrodnieniowymi) kości, stawów (układu kostno-stawowego), a także ogólnoustrojowymi stanami zapalnymi tkanki łącznej (kolagenozy), ich rozpoznawaniem, leczeniem oraz zapobieganiem}
\Clue{24}{}{Meliphaga vicina - gatunek ptaka z rodziny miodojadów (Meliphagidae), który zamieszkuje wschodnie wybrzeże Australii}
\Clue{25}{}{cecha wypowiedzi, również artystycznej, która ma intelektualny, rozumowy charakter}
\Clue{27}{}{Bombus - rodzaj owada społecznego z rodziny pszczołowatych, obejmujący trzmiele i trzmielce (brzmiki); należą do niego duże owady (do 30 mm), gęsto owłosione, często jaskrawo ubarwione, ale zwykle z przewagą czerni lub też całkiem czarne}
\Clue{28}{}{wydzielona część forum internetowego, w której dyskutuje się o jednej sprawie}
\Clue{30}{}{południowoamerykański owadożerny o puszystym, perłowoszarym futerku, chroniony, hodowany dla cennego futra}
\Clue{31}{}{miasto w USA (Michigan) port nad rzeką Saginaw}
\Clue{32}{}{to, że coś się zdarzy lub zdarzyło w rzeczywistości}
\Clue{33}{}{packa - przyrząd do bicia owadów}\end{PuzzleClues}\newpage\section*{Krzyżówka 63}

\noindent\begin{Puzzle}{21}{25}|*	|*	|*	|*	|*	|*	|*	|*	|*	|*	|[1][S]\darr	|*	|*	|*	|*	|*	|*	|*	|*	|[2][S]\darr	|*	|[3][S]\darr	|.
|*	|*	|*	|*	|*	|[4][S]\rarr	|n	|a	|l	|e	|p	|a	|*	|*	|[5][S]\drarr	|s	|ó	|l	|*	|p	|*	|m	|.
|*	|*	|*	|*	|*	|*	|*	|[6][S]\rarr	|c	|r	|a	|s	|h	|t	|e	|s	|t	|*	|[7][S]\darr	|ł	|*	|i	|.
|*	|*	|*	|*	|*	|[8][S]\drarr	|z	|a	|p	|o	|r	|a	|[][,]{ }	|o	|g	|n	|i	|o	|w	|a	|*	|k	|.
|*	|*	|[9][S]\drarr	|w	|a	|s	|z	|y	|n	|g	|t	|o	|ń	|c	|z	|y	|k	|*	|i	|s	|*	|r	|.
|*	|[10][S]\drarr	|r	|o	|s	|z	|a	|d	|a	|*	|*	|*	|*	|[11][S]\darr	|o	|*	|*	|*	|l	|z	|*	|o	|.
|[12][S]\rarr	|k	|o	|m	|i	|t	|e	|n	|t	|*	|*	|*	|*	|p	|t	|*	|*	|*	|a	|c	|*	|m	|.
|*	|o	|s	|*	|*	|y	|*	|*	|*	|*	|[13][S]\darr	|*	|[14][S]\darr	|u	|y	|*	|[15][S]\darr	|[16][S]\darr	|j	|z	|*	|a	|.
|*	|ł	|t	|*	|*	|c	|*	|[17][S]\darr	|*	|[18][S]\darr	|p	|*	|h	|n	|k	|*	|p	|h	|a	|[][,]{ }	|*	|c	|.
|*	|p	|o	|*	|*	|h	|*	|u	|[19][S]\drarr	|m	|i	|ł	|e	|k	|*	|*	|i	|e	|*	|d	|*	|i	|.
|*	|a	|c	|*	|*	|*	|*	|r	|t	|o	|l	|*	|f	|c	|*	|*	|o	|r	|*	|o	|*	|e	|.
|*	|k	|k	|*	|*	|*	|*	|o	|e	|l	|o	|*	|e	|j	|*	|[20][S]\darr	|n	|m	|*	|l	|*	|r	|.
|*	|*	|*	|*	|[21][S]\rarr	|l	|e	|m	|n	|i	|d	|*	|i	|a	|[22][S]\rarr	|p	|i	|l	|o	|n	|*	|z	|.
|*	|*	|[23][S]\darr	|*	|*	|*	|*	|e	|g	|n	|z	|*	|*	|[][,]{ }	|*	|ó	|e	|i	|*	|y	|*	|[][,]{ }	|.
|*	|*	|c	|*	|*	|*	|*	|t	|e	|o	|i	|[24][S]\darr	|*	|l	|*	|ł	|r	|n	|*	|*	|*	|b	|.
|*	|*	|e	|*	|[25][S]\rarr	|t	|a	|r	|*	|*	|ó	|a	|*	|ę	|*	|p	|*	|*	|[26][S]\darr	|*	|*	|i	|.
|*	|[27][S]\rarr	|l	|ą	|d	|e	|k	|*	|*	|[28][S]\darr	|b	|n	|*	|d	|*	|l	|[29][S]\darr	|[30][S]\darr	|p	|*	|[31][S]\darr	|a	|.
|*	|*	|i	|[32][S]\drarr	|c	|h	|i	|c	|h	|a	|*	|a	|[33][S]\darr	|ź	|*	|a	|p	|k	|a	|[34][S]\darr	|w	|ł	|.
|*	|[35][S]\rarr	|b	|u	|k	|s	|z	|p	|a	|n	|*	|k	|o	|w	|*	|s	|i	|i	|j	|w	|a	|k	|.
|*	|*	|a	|c	|[36][S]\rarr	|b	|u	|l	|w	|a	|*	|o	|n	|i	|*	|t	|e	|n	|ą	|k	|c	|o	|.
|*	|*	|t	|h	|[37][S]\rarr	|s	|a	|m	|u	|m	|*	|l	|t	|o	|*	|e	|t	|g	|c	|ł	|i	|w	|.
|*	|*	|*	|a	|*	|*	|*	|*	|*	|n	|*	|u	|a	|w	|*	|r	|r	|d	|z	|a	|a	|a	|.
|*	|*	|*	|*	|*	|*	|*	|*	|*	|e	|[38][S]\rarr	|t	|r	|a	|p	|e	|z	|o	|e	|d	|r	|*	|.
|[39][S]\rarr	|b	|l	|u	|f	|f	|i	|a	|r	|z	|*	|*	|i	|*	|*	|k	|y	|m	|k	|k	|z	|*	|.
|*	|[40][S]\rarr	|n	|i	|e	|d	|o	|k	|ł	|a	|d	|n	|o	|ś	|ć	|*	|k	|*	|*	|a	|*	|*	|.
|[41][S]\rarr	|b	|u	|r	|a	|c	|t	|w	|o	|*	|*	|*	|*	|*	|*	|*	|*	|*	|*	|*	|*	|*	|.\end{Puzzle}

\newpage

\begin{PuzzleClues}{\textbf{Poziome}\\}\Clue{4}{}{część pieca, przypiecek, na którym można wygrzewać się}
\Clue{5}{}{w chemii: związek chemiczny powstały w wyniku całkowitego lub częściowego zastąpienia w kwasach atomów wodoru innymi atomami bądź grupami o właściwościach elektrofilowych}
\Clue{6}{}{test, w którym w kontrolowanych warunkach doprowadza się do zderzenia pojazdu z przeszkodą, innym pojazdem czy manekinem pieszego, żeby przekonać się, w jakim stopniu chroni on kierowcę, pasażerów czy pieszych; test zderzeniowy}
\Clue{8}{}{oprogramowanie komputerowe, którego celem jest ochrona urządzenia lub sieci przed atakami sieciowymi z zewnątrz}
\Clue{9}{}{mieszkaniec Waszyngtonu - miasta}
\Clue{10}{}{zamiana pracowników na stanowiskach w zakładzie pracy}
\Clue{12}{}{właściciel rzeczy oddanej w komis}
\Clue{19}{}{ADONIS roślina zielna z jaskrowatych, chroniona w Polsce, o żółtych kwiatach, lecznicza}
\Clue{21}{}{pleustofit, który unosi się wolno na powierzchni wody}
\Clue{22}{}{podpora pośrednia mostu wiszącego, na której opierają się pasma nośne}
\Clue{25}{}{TING - łowny ssak z rodziny krętorogich, pokrojem zbliżony do kozy}
\Clue{27}{}{Lądek-Zdrój, miasteczko uzdrowiskowe w województwie dolnośląskim}
\Clue{32}{}{brazylijski, musujący napój alkoholowy otrzymywany przez fermentację kukurydzy, ryżu, manioku i owoców}
\Clue{35}{}{krzew uprawiany w Polsce jako ozdobny głównie na żywopłoty o drobnych skórzastych zimotrwałych liściach}
\Clue{36}{}{gula, zgrubienie}
\Clue{37}{}{gwałtowny, suchy i gorący południowy wiatr wiejący na pustyniach Afryki Północnej i Półwyspu Arabskiego oraz w ich sąsiedztwie}
\Clue{38}{}{wielościan o ścianach czworokątnych parami nierównych i nierównoległych}
\Clue{39}{}{osoba, która oszukuje, robi wrażenie kogoś, kto ma przewagę}
\Clue{40}{}{wyniku, rezultatu; to, że coś jest bardzo ogólne, np. nieścisłość jakiegoś działania}
\Clue{41}{}{grupa zachowujących się niekulturalnie, chamsko, prostacko ludzi}\end{PuzzleClues}

\begin{PuzzleClues}{\textbf{Pionowe}\\}\Clue{1}{}{taśma, plecionka ze szpagatu używana w różnych rzemiosłach na giętkie połączenia pomiędzy sztywnymi elementami}
\Clue{2}{}{warstwa we wnętrzu Ziemi znajdująca się blisko jądra}
\Clue{3}{}{płytka szklana lub plastikowa z naniesionymi w regularnych pozycjach mikroskopowej wielkości polami, zawierającymi białka lub związki wiążące białka}
\Clue{5}{}{przedmiot egzotyczny, osobliwy}
\Clue{7}{}{WILAJAT- jednostka podziału administracyjnego w Algierii i Tunezji}
\Clue{8}{}{rodzaj ataku bronią białą, pchnięcie}
\Clue{9}{}{miasto w Niemczech (Meklemburgia) nad rzeką Wamow, ważny zespół portowy, węzeł komunikacyjny}
\Clue{10}{}{wypukła pokrywa osłaniająca na piastę koła samochodowego}
\Clue{11}{}{inwazyjny zabieg, polegający na wprowadzeniu igły punkcyjnej do przestrzeni podpajęczynkowej w odcinku lędźwiowym kręgosłupa}
\Clue{13}{}{ptak leśny z rzędu kraskowatych o jaskrawym upierzeniu, brzegi dzioba ząbkowane; Ameryka Południowa}
\Clue{14}{}{HOFEJ; miasto w Chinach, ośrodek administracyjny prowincji Anhui}
\Clue{15}{}{nowator, ten, który jako pierwszy coś robił}
\Clue{16}{}{ur. 1915r, pisarz niemiecki, uczestnik wojny domowej w Hiszpanii, „Młodość nieujarzmiona”, opowiadania, poezje}
\Clue{17}{}{przyrząd wykorzystywany w diagnostyce medycznej służący do określania gęstości (ciężaru właściwego moczu}
\Clue{18}{}{miękkie, cienkie płótno bawełniane, bielone, barwione lub drukowane, używane na sukienki, podszewki itp}
\Clue{19}{}{jednostka monetarna Kazachstanu}
\Clue{20}{}{połówka cienkego kawałek czegoś, pół płatka skrojonego z jakiejś całości - najczęściej takiej, która ma przekrój w okrągławym kształcie}
\Clue{23}{}{w niektórych religiach dobrowolna lub przymusowa bezżenność duchownych}
\Clue{24}{}{celowe zniekształcenie składniowe, stosowane jeko zabieg retoryczny}
\Clue{26}{}{rodzaj ażurowego haftu lub delikatnego wzoru koronki, zwykle przypominającego pająka tym, że elementy wzoru promieniście rozchodzą się od jego środka}
\Clue{28}{}{w filozofii: przypomnienie sobie danej rzeczy niezależnie od doświadczenia zmysłowego}
\Clue{29}{}{lekkoatleta, srebrny medalista z Montrealu w sztafecie 4x400 m}
\Clue{30}{}{lekkoatleta amerykański, dwukrotny mistrz olimpijski w biegu na 110 m przez płotki z Los Angeles i Seulu}
\Clue{31}{}{sprzedawca waty cukrowej}
\Clue{32}{}{typowa dla kuchni rosyjskiej zawiesista zupa rybna}
\Clue{33}{}{miasto w USA, w stanie Kalifornia}
\Clue{34}{}{dodatek dodawany np. do książki, czasopisma, wkładany do środka}\end{PuzzleClues}\newpage\section*{Krzyżówka 64}

\noindent\begin{Puzzle}{24}{33}|*	|*	|*	|*	|*	|*	|[1][S]\darr	|*	|*	|*	|*	|*	|*	|*	|*	|*	|*	|*	|*	|*	|*	|*	|*	|*	|*	|.
|*	|*	|*	|*	|*	|*	|c	|*	|*	|*	|*	|*	|*	|*	|*	|*	|*	|*	|[2][S]\drarr	|a	|r	|k	|a	|*	|[3][S]\darr	|.
|*	|[4][S]\darr	|*	|*	|*	|*	|a	|*	|*	|*	|[5][S]\drarr	|k	|u	|r	|z	|e	|j	|*	|h	|*	|*	|*	|*	|*	|p	|.
|[6][S]\drarr	|p	|a	|r	|a	|b	|o	|l	|o	|i	|d	|a	|[][,]{ }	|o	|b	|r	|o	|t	|o	|w	|a	|*	|*	|[7][S]\darr	|r	|.
|g	|a	|[8][S]\drarr	|s	|e	|n	|*	|*	|[9][S]\rarr	|p	|o	|d	|g	|l	|ą	|d	|a	|c	|t	|w	|o	|*	|[10][S]\darr	|s	|ą	|.
|r	|r	|ź	|*	|*	|*	|[11][S]\rarr	|b	|i	|g	|n	|o	|n	|i	|o	|w	|a	|t	|e	|*	|*	|*	|a	|z	|ż	|.
|u	|s	|r	|*	|*	|*	|*	|*	|*	|*	|a	|*	|*	|[12][S]\rarr	|z	|a	|z	|u	|l	|a	|*	|[13][S]\darr	|n	|t	|e	|.
|s	|y	|ó	|[14][S]\rarr	|o	|c	|z	|e	|r	|e	|t	|*	|*	|*	|*	|[15][S]\darr	|[16][S]\darr	|*	|[][,]{ }	|[17][S]\darr	|*	|d	|a	|r	|k	|.
|z	|z	|d	|[18][S]\drarr	|m	|o	|r	|d	|o	|b	|i	|c	|i	|e	|*	|r	|m	|[19][S]\drarr	|r	|o	|z	|e	|t	|a	|*	|.
|a	|m	|ł	|r	|[20][S]\drarr	|s	|z	|p	|i	|l	|*	|*	|*	|*	|*	|a	|d	|z	|o	|b	|[21][S]\darr	|r	|a	|n	|*	|.
|[][,]{ }	|*	|o	|a	|s	|*	|*	|*	|*	|*	|*	|*	|*	|*	|*	|c	|l	|m	|b	|i	|p	|y	|n	|d	|*	|.
|s	|[22][S]\drarr	|w	|s	|p	|ó	|ł	|t	|w	|ó	|r	|c	|z	|y	|n	|i	|*	|y	|o	|e	|i	|w	|a	|o	|*	|.
|y	|m	|o	|z	|ó	|*	|*	|[23][S]\darr	|[24][S]\darr	|*	|*	|*	|*	|*	|*	|c	|[25][S]\darr	|w	|t	|k	|n	|a	|[][,]{ }	|w	|*	|.
|r	|o	|ś	|p	|d	|*	|*	|k	|m	|[26][S]\darr	|[27][S]\darr	|*	|*	|[28][S]\drarr	|s	|a	|w	|a	|n	|t	|*	|c	|i	|a	|*	|.
|y	|ż	|ć	|l	|n	|*	|*	|l	|o	|j	|a	|*	|*	|j	|[29][S]\darr	|*	|i	|c	|i	|*	|[30][S]\darr	|j	|n	|n	|*	|.
|j	|l	|*	|a	|i	|[31][S]\rarr	|z	|a	|p	|o	|r	|a	|*	|e	|j	|*	|e	|z	|c	|*	|k	|a	|d	|i	|*	|.
|s	|i	|[32][S]\darr	|[][,]{ }	|c	|*	|*	|m	|s	|n	|e	|*	|*	|d	|u	|*	|l	|*	|z	|*	|s	|[][,]{ }	|y	|e	|*	|.
|k	|w	|s	|k	|z	|*	|*	|o	|i	|[][,]{ }	|n	|*	|*	|n	|r	|*	|k	|*	|y	|*	|i	|p	|j	|*	|*	|.
|a	|o	|ę	|a	|k	|*	|*	|t	|c	|k	|a	|[33][S]\rarr	|m	|o	|n	|g	|o	|z	|*	|[34][S]\darr	|ę	|o	|s	|*	|*	|.
|*	|ś	|k	|l	|a	|*	|*	|*	|a	|o	|*	|*	|*	|s	|o	|*	|ś	|*	|*	|s	|ż	|s	|k	|*	|*	|.
|*	|ć	|[][,]{ }	|i	|*	|*	|*	|[35][S]\darr	|*	|m	|*	|*	|*	|t	|ś	|*	|ć	|*	|*	|z	|y	|t	|a	|*	|*	|.
|*	|*	|o	|f	|[36][S]\darr	|*	|*	|s	|*	|p	|*	|*	|*	|k	|ć	|*	|*	|*	|*	|u	|n	|f	|*	|*	|*	|.
|[37][S]\drarr	|m	|ł	|o	|d	|o	|h	|e	|g	|l	|i	|s	|t	|a	|*	|*	|*	|*	|[38][S]\darr	|b	|a	|i	|*	|*	|*	|.
|f	|*	|ó	|r	|u	|[39][S]\drarr	|p	|r	|z	|e	|w	|ó	|d	|[][,]{ }	|g	|i	|ę	|t	|k	|i	|*	|k	|*	|*	|*	|.
|o	|*	|w	|n	|p	|k	|*	|p	|*	|k	|*	|*	|*	|t	|*	|*	|*	|*	|r	|n	|*	|s	|*	|*	|*	|.
|r	|*	|k	|i	|a	|o	|*	|e	|*	|s	|*	|[40][S]\drarr	|ł	|a	|n	|i	|a	|*	|ą	|i	|*	|a	|*	|*	|*	|.
|n	|*	|o	|j	|[][,]{ }	|n	|*	|n	|*	|o	|*	|d	|*	|k	|*	|*	|*	|*	|ż	|a	|*	|l	|*	|*	|*	|.
|i	|*	|w	|s	|w	|e	|*	|t	|*	|w	|*	|e	|*	|t	|*	|*	|*	|*	|n	|n	|*	|n	|*	|*	|*	|.
|e	|*	|y	|k	|o	|k	|*	|y	|*	|y	|*	|d	|*	|y	|*	|*	|*	|*	|i	|k	|*	|a	|*	|*	|*	|.
|r	|*	|*	|a	|ł	|t	|*	|n	|*	|*	|*	|e	|*	|c	|[41][S]\rarr	|m	|a	|t	|k	|a	|*	|*	|*	|*	|*	|.
|n	|*	|*	|*	|o	|o	|*	|*	|[42][S]\rarr	|k	|u	|r	|c	|z	|a	|t	|o	|w	|*	|*	|*	|*	|*	|*	|*	|.
|i	|*	|*	|*	|w	|r	|*	|*	|*	|*	|[43][S]\rarr	|k	|o	|n	|t	|r	|o	|l	|k	|a	|*	|*	|*	|*	|*	|.
|a	|*	|*	|*	|a	|*	|*	|*	|*	|[44][S]\rarr	|k	|o	|z	|a	|k	|*	|*	|*	|*	|*	|*	|*	|*	|*	|*	|.
|*	|*	|*	|*	|*	|*	|[45][S]\rarr	|t	|i	|b	|u	|*	|*	|*	|*	|*	|*	|*	|*	|*	|*	|*	|*	|*	|*	|.\end{Puzzle}

\newpage

\begin{PuzzleClues}{\textbf{Poziome}\\}\Clue{2}{}{okręt, na którym według Biblii schronił się Noe wraz z rodziną i zwierzętami na czas potopu}
\Clue{5}{}{zniewieściały, prawdopodobnie homoseksualny kogut, który zachowuje się prawie zupełnie jak kura}
\Clue{6}{}{paraboloida o jednej osi symterii}
\Clue{8}{}{jednostka zdawkowa w Japonii; 1/100 jena (nazwa niezwiązana etymologicznie z angielskimcent)}
\Clue{9}{}{zajmowanie się podglądaniem}
\Clue{11}{}{Bignoniaceae - rodzina roślin z rzędu jasnotowców (Lamiales); należy tu ok. 800 gatunków grupowanych w 110 rodzajach; zaliczane tu rośliny to drzewa i krzewy występujące głównie w strefie tropikalnej, zwłaszcza w Ameryce Południowej}
\Clue{12}{}{ludowe określenie kukułki}
\Clue{14}{}{zarośle roślin, szuwar przybrzeżny; zwykle w liczbie mnogiej}
\Clue{18}{}{gatunek gry komputerowej, w której rozgrywka opiera się głównie na walce fizycznej z przeciwnikami}
\Clue{19}{}{okrągłe okno wypełnione witrażem i ornamentem maswerkowym umieszczone nad głównym portalem kościoła}
\Clue{20}{}{żeglarska przyciągarka}
\Clue{22}{}{kobieta mająca udział w tworzeniu czegoś z kimś innym, kobieta z grona autorów czegoś}
\Clue{28}{}{osoba z zespołem sawanta, niepełnosprawnością umysłową połączoną z wybitnymi uzdolnieniami i doskonałą pamięcią}
\Clue{31}{}{budowla piętrząca dolinę cieku wodnego}
\Clue{33}{}{małpiatka z lemurów - Madagaskar}
\Clue{37}{}{przedstawiciel nurtu szkoły heglowskiej, tzw. lewicy heglowskiej, dążącego do przekształcenia filozofii w krytykę, mającą na celu przemianę świata}
\Clue{39}{}{przewód elektryczny z giętką żyłą wielodrutową}
\Clue{40}{}{samica jelenia}
\Clue{41}{}{termin stosowany w zoologii na określenie samicy niektórych owadów, zdolnej do rozpłodu}
\Clue{42}{}{nazwa pierwiastka 104 w układzie okresowym pierwiastków, która funkcjonowała przez pewien czas w przeszłości w nomenklaturze używanej przez chemików i fizyków radzieckich, obcnie nieużywana}
\Clue{43}{}{notes kontrolny, który przeznaczony jest do prowadzenia rejestów, wykazów itp}
\Clue{44}{}{hopak - muzyczna forma taneczna oparta na tańcu kozak}
\Clue{45}{}{miasto w Kolumbii w pobliżu granicy z Wenezuelą}\end{PuzzleClues}

\begin{PuzzleClues}{\textbf{Pionowe}\\}\Clue{1}{}{zmarły w 1486 r. żeglarz portugalski pierwszy dotarł do ujścia rzeki Kongo}
\Clue{2}{}{budynek, w którym mieszkają robotnicy zamiejscowi}
\Clue{3}{}{wąski, długi kształt, pasek}
\Clue{4}{}{monoteistyczna postać religii Zaratustry (mazdaizmu) występująca obecnie w Indiach; kult ognia, wody, ziemi i powietrza; nakazująca wystawiać zwłoki na pożarcie drapieżnym ptakom}
\Clue{5}{}{Giovanni. ur. w 1826r. astronom włoski; prace ze spektroskopii astronomicznej}
\Clue{6}{}{Pyrus syriaca - gatunek gruszy, drzewa z rodziny różowatych}
\Clue{7}{}{zamierzone, celowe wejście statku na mieliznę co uchronić może przed zatonięciem}
\Clue{8}{}{w filozofii: natura czegoś, co stanowi źródło (będące pojęciem problematycznym, niekoniecznie bowiem oznacza to, z czego coś innego pochodzi), cecha tego, w czym można szukać istoty czegoś}
\Clue{10}{}{Anathana ellioti - ssak z rodziny tupajowatych, jedyny przedstawiciel rodzaju Anathana; zamieszkuje środkowe i południowe Indie, na południe od Gangesu}
\Clue{13}{}{tworzenie wyrazu pochodnego poprzez dodanie postfiksu do podstawy słowotwórczej}
\Clue{15}{}{płytka rogowa u parzystokopytnych np. krowy, jelenia, dzika}
\Clue{16}{}{kod ISO 4217 leja mołdawskiego}
\Clue{17}{}{w informatyce: struktura zawierająca dane i metody, czyli funkcje służące do wykonywania na tych danych określonych zadań}
\Clue{18}{}{Squatina californica - gatunek morskiej ryby chrzęstnoszkieletowej z rodziny raszplowatych (Squatinidae)}
\Clue{19}{}{kosmetyk używany do zmywania lakieru z paznokci}
\Clue{20}{}{zdrobniale - spódnica: wierzchnia część garderoby (w kulturze europejskiej najczęściej garderoby damskiej i dziewczęcej) od pasa w dół}
\Clue{21}{}{określenie wykonawcze, stosowane przy oznaczeniu tempa}
\Clue{22}{}{przyrodzona cecha kogoś/czegoś do robienia czegoś, generowania jakichś sytuacji; zazwyczaj l.mn}
\Clue{23}{}{przedmiot, zwykle stary, zużyty, zawadzający}
\Clue{24}{}{suczka mopsa}
\Clue{25}{}{potęga, wielkie znaczenie; to, że coś jest duże}
\Clue{26}{}{cząsteczka, która ma w swojej budowie jon centralny otoczony jonami lub cząsteczkami pobocznymi (ligandami)}
\Clue{27}{}{obiekt z miejscami dla publiczności, gdzie odbywają się duże widowiska sportowe i kulturalne}
\Clue{28}{}{część większego oddziału lub związku taktycznego nie mająca zwykle samodzielności gospodarczej i administracyjnej}
\Clue{29}{}{cecha istoty żywej, która jest silna, pełna energii i temperamentu seksualnego, zdolna spłodzić liczne potomstwo}
\Clue{30}{}{lekceważąco o osobie duchownej}
\Clue{32}{}{sęk o średnicy od 6 mm do 10 mm}
\Clue{34}{}{mieszkanka Szubina}
\Clue{35}{}{minerał zaliczany do krzemianów warstwowych}
\Clue{36}{}{oferma, ciamajda}
\Clue{37}{}{zakład produkujący forniry}
\Clue{38}{}{ciężki walec stanowiący element roboczy kruszarki}
\Clue{39}{}{inaczej relator, zaimek względny}
\Clue{40}{}{fotografik, inżynier elektryk (1880-1965); twórca nowoczesnego portretu w fotografice polskiej}\end{PuzzleClues}\newpage\section*{Krzyżówka 65}

\noindent\begin{Puzzle}{23}{33}|*	|*	|*	|*	|*	|*	|*	|*	|*	|*	|*	|*	|*	|*	|*	|*	|*	|*	|*	|*	|*	|*	|*	|[1][S]\darr	|.
|*	|*	|*	|*	|*	|*	|*	|*	|*	|*	|*	|*	|*	|*	|*	|*	|[2][S]\darr	|*	|*	|*	|*	|[3][S]\darr	|*	|h	|.
|*	|*	|*	|*	|*	|*	|*	|[4][S]\rarr	|a	|n	|t	|a	|g	|o	|n	|i	|s	|t	|k	|a	|*	|d	|*	|i	|.
|*	|*	|*	|*	|*	|[5][S]\rarr	|s	|k	|o	|c	|z	|n	|i	|a	|[][,]{ }	|m	|a	|m	|u	|c	|i	|a	|*	|p	|.
|*	|*	|*	|*	|*	|*	|*	|*	|*	|*	|*	|*	|*	|*	|[6][S]\darr	|*	|m	|*	|*	|*	|*	|j	|*	|s	|.
|*	|*	|*	|[7][S]\darr	|[8][S]\rarr	|o	|s	|t	|a	|t	|n	|i	|a	|[][,]{ }	|p	|r	|o	|s	|t	|a	|*	|m	|*	|y	|.
|[9][S]\rarr	|l	|e	|m	|o	|n	|i	|a	|d	|a	|*	|*	|*	|*	|y	|*	|i	|*	|*	|*	|*	|o	|*	|b	|.
|*	|*	|*	|o	|*	|[10][S]\darr	|*	|*	|*	|*	|*	|*	|*	|*	|ł	|*	|s	|*	|*	|*	|*	|n	|*	|e	|.
|*	|*	|*	|d	|*	|g	|[11][S]\rarr	|k	|o	|z	|i	|o	|ł	|e	|k	|*	|t	|*	|*	|*	|*	|i	|*	|m	|.
|[12][S]\rarr	|t	|o	|r	|*	|ó	|[13][S]\drarr	|u	|ś	|c	|i	|e	|[][,]{ }	|s	|o	|l	|n	|e	|*	|[14][S]\darr	|*	|o	|*	|a	|.
|*	|*	|*	|z	|[15][S]\darr	|r	|w	|*	|*	|*	|*	|*	|*	|*	|j	|*	|i	|*	|[16][S]\rarr	|b	|u	|n	|t	|*	|.
|*	|*	|*	|y	|p	|n	|i	|*	|*	|*	|*	|[17][S]\rarr	|g	|w	|a	|t	|e	|m	|a	|l	|a	|*	|*	|*	|.
|*	|*	|*	|k	|s	|i	|k	|*	|*	|*	|[18][S]\rarr	|l	|o	|n	|d	|o	|n	|*	|*	|i	|*	|*	|*	|*	|.
|*	|*	|*	|[][,]{ }	|z	|k	|a	|*	|*	|*	|*	|*	|*	|*	|[][,]{ }	|*	|i	|*	|*	|s	|*	|*	|[19][S]\darr	|*	|.
|*	|*	|*	|c	|e	|[][,]{ }	|r	|[20][S]\rarr	|p	|ó	|ł	|p	|l	|a	|s	|t	|e	|r	|e	|k	|*	|[21][S]\darr	|m	|*	|.
|*	|*	|[22][S]\rarr	|i	|n	|d	|y	|w	|i	|d	|u	|a	|l	|i	|z	|m	|*	|[23][S]\drarr	|c	|o	|m	|b	|o	|*	|.
|*	|*	|*	|e	|i	|o	|z	|*	|*	|*	|*	|[24][S]\darr	|*	|*	|a	|*	|*	|c	|*	|ś	|*	|o	|ł	|*	|.
|*	|*	|*	|m	|e	|ł	|m	|*	|*	|[25][S]\darr	|*	|p	|*	|*	|r	|*	|*	|z	|*	|ć	|*	|o	|d	|*	|.
|*	|*	|*	|n	|c	|o	|*	|[26][S]\drarr	|ź	|w	|i	|e	|r	|z	|y	|n	|i	|e	|c	|*	|*	|t	|a	|*	|.
|*	|*	|*	|o	|[][,]{ }	|w	|*	|f	|*	|y	|*	|t	|*	|*	|*	|*	|*	|p	|*	|*	|*	|e	|w	|*	|.
|*	|*	|*	|g	|h	|y	|*	|u	|*	|s	|[27][S]\drarr	|r	|ó	|ż	|a	|n	|i	|e	|c	|*	|*	|s	|i	|*	|.
|*	|*	|*	|r	|e	|*	|*	|g	|*	|y	|s	|e	|*	|*	|*	|*	|*	|k	|*	|*	|*	|*	|a	|*	|.
|*	|[28][S]\darr	|*	|z	|r	|*	|*	|a	|*	|p	|z	|l	|*	|*	|*	|*	|*	|*	|[29][S]\darr	|*	|*	|*	|n	|*	|.
|*	|s	|*	|b	|b	|*	|*	|*	|*	|*	|o	|[][,]{ }	|*	|[30][S]\rarr	|n	|a	|p	|a	|r	|s	|t	|e	|k	|*	|.
|*	|z	|*	|i	|i	|*	|*	|*	|*	|*	|p	|h	|*	|*	|*	|*	|*	|*	|e	|*	|*	|*	|a	|[31][S]\darr	|.
|*	|p	|*	|e	|c	|[32][S]\rarr	|b	|u	|s	|z	|*	|a	|[33][S]\rarr	|z	|a	|s	|i	|l	|a	|n	|i	|e	|*	|s	|.
|*	|r	|*	|t	|h	|*	|*	|*	|*	|*	|*	|w	|*	|*	|*	|*	|*	|*	|*	|*	|*	|*	|*	|p	|.
|*	|i	|*	|y	|a	|*	|[34][S]\rarr	|r	|e	|s	|t	|a	|r	|t	|*	|*	|*	|*	|*	|*	|*	|*	|*	|ł	|.
|*	|n	|*	|*	|*	|*	|*	|*	|*	|*	|[35][S]\rarr	|j	|ę	|z	|y	|k	|[][,]{ }	|g	|u	|r	|*	|*	|*	|y	|.
|*	|g	|*	|*	|[36][S]\rarr	|t	|ę	|p	|o	|l	|i	|s	|t	|k	|a	|[][,]{ }	|w	|o	|d	|n	|a	|*	|*	|w	|.
|*	|b	|*	|[37][S]\rarr	|w	|r	|ó	|b	|e	|l	|[][,]{ }	|k	|a	|s	|z	|t	|a	|n	|o	|w	|a	|t	|y	|*	|.
|*	|o	|*	|*	|*	|[38][S]\rarr	|t	|r	|ó	|j	|l	|i	|s	|t	|[][,]{ }	|ś	|n	|i	|e	|ż	|n	|y	|*	|*	|.
|[39][S]\rarr	|k	|l	|i	|k	|a	|l	|n	|o	|ś	|ć	|*	|*	|*	|*	|*	|*	|*	|*	|*	|*	|*	|*	|*	|.
|*	|*	|*	|*	|*	|*	|*	|*	|*	|*	|*	|*	|*	|*	|*	|*	|*	|*	|*	|*	|*	|*	|*	|*	|.\end{Puzzle}

\newpage

\begin{PuzzleClues}{\textbf{Poziome}\\}\Clue{4}{}{kobieta odgrywająca drugą pod względem ważności rolę po postaci głównej, pierwszoplanowej}
\Clue{5}{}{skocznia narciarska, której punkt konstrukcyjny umiejscowiony jest powyżej 170 metra, a punkt sędziowski powyżej 185m}
\Clue{8}{}{okres w życiu, które wydaje się być końcem jakichś trudności, wymaga od człowieka jeszcze odrobiny wysiłku}
\Clue{9}{}{porcja lemoniady, puszka lub butelka z napojem, ale także np. szklanka, dzbanek}
\Clue{11}{}{na Kujawach: mątewka - przyrząd kuchenny służący do mieszania}
\Clue{12}{}{Th - promieniotwórczy pierwiastek chemiczny z grupy aktynowców}
\Clue{13}{}{wieś (do 1934 miasto) w Polsce położona na prawym brzegu Raby, w województwie małopolskim, w powiecie brzeskim, w gminie Szczurowa}
\Clue{16}{}{protest przeciwko czemuś, postawa wyrażająca - najczęściej w sposób spontaniczny i gwałtowny - przeciwny stosunek do czegoś}
\Clue{17}{}{państwo w Ameryce Środkowej, położone nad Oceanem Atlantyckim i Oceanem Spokojnym}
\Clue{18}{}{Griffith - (1876-1916), pisarz amerykański, powieści i opowiadania; „Wilk morski”, „Zawierucha”, „Przygoda”, „Martin Eden”}
\Clue{20}{}{połówka cienkego kawałek czegoś, pół płatka skrojonego z jakiejś całości - najczęściej takiej, która ma przekrój w okrągławym kształcie}
\Clue{22}{}{zasada filozofii politycznej i etyki, zgodnie z którą jednostka ludzka uznawana jest za najwyższe dobro}
\Clue{23}{}{zespół jazzowy o różnej co do zestawu instrumentów obsadzie sekcji melodycznej}
\Clue{26}{}{zwierzyniec, ogród zoologiczny}
\Clue{27}{}{modlitwa maryjna duchowo jednocząca chrześcijanina z osobą Matki Jezusa i kontemplująca tajemnice zbawienia}
\Clue{30}{}{mały kieliszek}
\Clue{32}{}{formacja roślinna, charakterystyczna dla suchych obszarów podrównikowych}
\Clue{33}{}{układ urządzeń powodujący dopływ prądu do jakiegoś innego urządzenia}
\Clue{34}{}{ponowny start, np. zawodnika w zawodach}
\Clue{35}{}{język z podgrupy języków gur}
\Clue{36}{}{Codriophorus aquaticus - gatunek mchu z rodziny strzechwowatych}
\Clue{37}{}{Passer eminibey - gatunek ptaka z rodziny wróblowatych (Passeridae)}
\Clue{38}{}{Trillium nivale - gatunek rośliny zielnej z rodziny melantkowatych}
\Clue{39}{}{liczba wejść na daną stroną internetową rozumiana jako liczba kliknięć w link, który do niej prowadzi}\end{PuzzleClues}

\begin{PuzzleClues}{\textbf{Pionowe}\\}\Clue{1}{}{Hypsibema - rodzaj roślinożernego dinozaura z nadrodziny hadrozauroidów (Hadrosauroidea); żył w okresie późnej kredy (83-71 mln lat temu) na terenach Ameryki Północnej; długość ciała 9 m, wysokość 4 m, ciężar 2,5 t}
\Clue{2}{}{istnienie niezależnie od niczego, np. samoistnienie Boga, diabła, narodów}
\Clue{3}{}{głos bóstwa, sumienie, ostrzegawczy głos wewnętrzny}
\Clue{6}{}{Conopophila whitei - gatunek ptaka z rodziny miodojadów (Meliphagidae) występujący w Australii i na Nowej Gwinei}
\Clue{7}{}{Porphyrio porphyrio indicus - podgatunek ptaka wyróżniony w obrębie gatunku modrzyk zwyczajny (Porphyrio porphyrio)}
\Clue{10}{}{górnik pracujący pod ziemią}
\Clue{13}{}{zastępowanie taksonu przez takson podobny, ale różniący się wymaganiami ekologicznymi, albo przez takson zajmujący podobne siedlisko na oddzielnym obszarze}
\Clue{14}{}{więź, oparta na przyjaźni lub miłości, zaufaniu itp}
\Clue{15}{}{Melampyrum herbichii - gatunek rośliny należący do rodziny zarazowatych}
\Clue{19}{}{mieszkanka Mołdawii, kobieta pochodzenia mołdawskiego}
\Clue{21}{}{WOLARZ}
\Clue{23}{}{kobiece nakrycie głowy, noszone przez różne grupy społeczne}
\Clue{24}{}{Pterodroma sandwichensis - gatunek ptaka z rodziny burzykowatych (Procellariidae)}
\Clue{25}{}{pojawienie się na rynku w dużej ilości płodów ziemi, głównie owoców}
\Clue{26}{}{spoina murowa}
\Clue{27}{}{rodzaj zakładu produkcyjnego podczas okupacji hitlerowskiej, obecny na terenie getta}
\Clue{28}{}{springbok, antylopa skoczek, Antidorcas marsupialis - gatunek małego brązowo-białego ssaka parzystokopytnego z rodziny krętorogich; żyje na suchych terenach południowej i południowo-zachodniej Afryki, głównie w Namibii, Botswanie, Angoli i Republice Południowej Afryki}
\Clue{29}{}{NANDU; ptak z bezgrzebieniowców zwany też strusiem amerykańskim; hodowane dla ozdobnych piór, wysokości około 1,5 m}
\Clue{31}{}{wycieczka polegająca na przemieszczaniu się po jakimś akwenie kajakiem, kanadyjką, na tratwie lub żaglówce}\end{PuzzleClues}\newpage\section*{Krzyżówka 66}

\noindent\begin{Puzzle}{17}{28}|*	|*	|*	|*	|[1][S]\drarr	|p	|l	|a	|f	|o	|n	|i	|e	|r	|a	|*	|[2][S]\darr	|[3][S]\darr	|.
|*	|[4][S]\rarr	|l	|i	|p	|a	|*	|*	|[5][S]\darr	|*	|*	|*	|*	|*	|[6][S]\darr	|*	|m	|r	|.
|*	|*	|[7][S]\rarr	|d	|u	|g	|a	|*	|g	|*	|[8][S]\rarr	|c	|i	|ą	|g	|*	|a	|y	|.
|*	|*	|*	|*	|l	|*	|[9][S]\rarr	|z	|a	|p	|a	|r	|c	|i	|e	|*	|z	|t	|.
|*	|[10][S]\rarr	|p	|ł	|a	|t	|n	|e	|r	|z	|*	|*	|*	|*	|n	|*	|z	|m	|.
|*	|*	|[11][S]\darr	|*	|*	|*	|*	|*	|d	|[12][S]\rarr	|t	|o	|r	|p	|e	|d	|o	|*	|.
|*	|[13][S]\rarr	|m	|y	|s	|z	|[][,]{ }	|l	|e	|ś	|n	|a	|*	|*	|t	|*	|l	|*	|.
|*	|*	|a	|[14][S]\rarr	|k	|o	|l	|o	|r	|[][,]{ }	|m	|o	|c	|n	|y	|*	|a	|*	|.
|*	|*	|t	|*	|*	|[15][S]\darr	|[16][S]\darr	|[17][S]\rarr	|o	|z	|i	|m	|ó	|w	|k	|a	|*	|*	|.
|*	|*	|i	|*	|[18][S]\darr	|l	|o	|[19][S]\darr	|b	|*	|[20][S]\rarr	|s	|o	|ł	|a	|*	|*	|*	|.
|*	|*	|z	|[21][S]\drarr	|f	|i	|k	|s	|a	|c	|j	|a	|*	|*	|[][,]{ }	|*	|*	|*	|.
|*	|*	|*	|a	|a	|e	|r	|z	|*	|*	|*	|*	|[22][S]\drarr	|t	|e	|k	|a	|*	|.
|*	|*	|*	|z	|m	|n	|ę	|y	|*	|*	|[23][S]\darr	|[24][S]\darr	|b	|[25][S]\darr	|k	|*	|*	|*	|.
|*	|*	|*	|d	|u	|z	|t	|l	|*	|*	|r	|s	|y	|b	|o	|[26][S]\darr	|*	|[27][S]\darr	|.
|*	|*	|*	|y	|ł	|*	|*	|d	|*	|*	|o	|z	|k	|a	|l	|p	|*	|k	|.
|[28][S]\rarr	|w	|i	|k	|a	|r	|y	|z	|m	|*	|z	|c	|[][,]{ }	|r	|o	|r	|*	|o	|.
|*	|[29][S]\darr	|*	|*	|*	|*	|*	|i	|*	|*	|w	|z	|z	|y	|g	|z	|*	|n	|.
|[30][S]\rarr	|j	|a	|p	|o	|ń	|s	|k	|i	|*	|o	|e	|[][,]{ }	|ł	|i	|y	|[31][S]\darr	|t	|.
|*	|i	|[32][S]\rarr	|p	|a	|r	|a	|*	|*	|*	|l	|c	|b	|e	|c	|t	|t	|r	|.
|*	|n	|*	|[33][S]\darr	|*	|*	|*	|*	|[34][S]\darr	|*	|n	|i	|r	|c	|z	|o	|r	|e	|.
|*	|x	|*	|k	|*	|*	|*	|*	|ż	|*	|i	|ń	|ą	|z	|n	|n	|u	|d	|.
|*	|i	|*	|o	|*	|*	|[35][S]\rarr	|p	|a	|ł	|e	|c	|z	|k	|a	|*	|k	|a	|.
|*	|*	|*	|m	|*	|[36][S]\rarr	|r	|ó	|g	|*	|n	|e	|u	|a	|*	|*	|c	|n	|.
|*	|*	|*	|p	|[37][S]\rarr	|t	|y	|r	|i	|*	|i	|*	|*	|*	|*	|[38][S]\darr	|z	|s	|.
|*	|[39][S]\rarr	|r	|o	|t	|a	|*	|[40][S]\rarr	|e	|l	|e	|g	|a	|n	|t	|k	|a	|*	|.
|*	|*	|[41][S]\rarr	|t	|e	|k	|s	|z	|l	|a	|*	|*	|*	|*	|*	|i	|s	|*	|.
|*	|*	|*	|*	|*	|[42][S]\rarr	|p	|l	|e	|r	|e	|z	|a	|*	|*	|j	|z	|*	|.
|[43][S]\rarr	|p	|y	|t	|o	|n	|[][,]{ }	|s	|k	|a	|l	|n	|y	|*	|*	|*	|y	|*	|.
|*	|*	|[44][S]\rarr	|b	|o	|l	|i	|d	|*	|*	|[45][S]\rarr	|k	|i	|s	|c	|h	|*	|*	|.\end{Puzzle}

\newpage

\begin{PuzzleClues}{\textbf{Poziome}\\}\Clue{1}{}{rodzaj płaskiej oprawy oświetleniowej, umieszczanej na suficie}
\Clue{4}{}{długowieczne drzewo o miododajnych kwiatach i miękkim drewnie}
\Clue{7}{}{DUHA; kabłąk w zaprzęgu jednokonnym służący do przymocowania chomąta do hołobu}
\Clue{8}{}{ciągły układ w przestrzeni; droga, ulica, szlak, trasa, pasmo, pas}
\Clue{9}{}{obstrukcja, zatwardzenie - utrudniona lub nieczęsta defekacja}
\Clue{10}{}{rzemieślnik wyrabiający zbroje, tarcze i hełmy a także białą broń}
\Clue{12}{}{rodzaj wolnobiegowej piasty rowerowej wyposażonej w system hamulcowy uruchamiany poprzez cofnięcie pedałów}
\Clue{13}{}{Apodemus flavicollis - gatunek gryzonia z rodziny myszowatych; zamieszkuje Wielką Brytanię, większą część kontynentalnej Europy aż do Ural w Rosji, stwierdzono występowanie tego ssaka również od wschodniej Turcji do zachodniej Armenii, w Iranie i na południe od Syrii, Libanu i Izraela}
\Clue{14}{}{karty w jednym kolorze, które są wysokie}
\Clue{17}{}{muchówka, szkodnik ozimin}
\Clue{20}{}{rzeka w południowej Polsce}
\Clue{21}{}{choroba psychiczna polegająca na uporczywym powtarzaniu pewnych zachowań, mimo że przynoszą one negatywne konsekwencje}
\Clue{22}{}{urząd sprawowany przez członka rady ministrów}
\Clue{28}{}{zastępowanie taksonu przez takson podobny, ale różniący się wymaganiami ekologicznymi, albo przez takson zajmujący podobne siedlisko na oddzielnym obszarze}
\Clue{30}{}{przedmiot szkolny lub uczony w ramach kursu, na którym opanowuje się podstawy języka japońskiego}
\Clue{32}{}{historyczna moneta w Imperium Osmańskim}
\Clue{35}{}{mały drążek wykonany z różnych materiałów; niewielka pałka, kijek}
\Clue{36}{}{miejsce zbiegu dwóch ulic}
\Clue{37}{}{jezioro w Norwegii, położone w pobliżu Oslo}
\Clue{39}{}{w dawnym wojsku polskim pieszy lub konny oddział}
\Clue{40}{}{ironicznie o kobiecie strojącej się; modnisia. (UWAGA! wśród żeńskich nazw człowieka układ leksemów w synsetach i relacji między synsetami może nie być analogiczny do sytuacji wśród nazw mężczyzn)}
\Clue{41}{}{gatunek maliny; bylina o pędach bez kolców i smacznych brunatno-czerwonych owocach}
\Clue{42}{}{pióro strusie zdobiące damski kapelusz}
\Clue{43}{}{Python sebae - gatunek gada z rodziny pytonów, podrzędu węży, występujący w całej Afryce Środkowej i Południowej na południe od Sahary}
\Clue{44}{}{meteor o olbrzymich rozmiarach i znacznej jasności}
\Clue{45}{}{(1885-1948), czeski pisarz i publicysta, tworzył w języku niemieckim; „Szalejący reporter”, „Chiny bez maski”, „Jarmark sensacji”}\end{PuzzleClues}

\begin{PuzzleClues}{\textbf{Pionowe}\\}\Clue{1}{}{miasto w Chorwacji, ważny port nad Morzem Adriatyckim, na półwyspie Istra, katedra z V-VI w}
\Clue{2}{}{francuski malarz i grafik (1869-1954) reprezentant fowizmu; kompozycje figuralne, martwe natury, malowidła ścienne, rysunki}
\Clue{3}{}{regularność w rozkładzie akcentów w tekście poetyckim}
\Clue{5}{}{pomieszczenie w teatrze, gdzie aktorzy charakteryzują się i ubierają przed występem na scenie}
\Clue{6}{}{gałąź genetyki zajmująca się współoddziaływaniem dziedziczenia w ekosystemach i następczą częstotliwością alleli}
\Clue{11}{}{daewoo z modelu Matiz}
\Clue{15}{}{miasto w Austrii (Tyrol) u podnóża Wysokich Taurów, ośrodek turystyczny i sportów zimowych}
\Clue{16}{}{jednostka pływająca przeznaczona do walki na morzu}
\Clue{18}{}{wielorodzinny dom budowany dla pracowników łódzkich fabryk włókienniczych}
\Clue{19}{}{tabliczka z nazwiskiem wieszana na drzwiach}
\Clue{21}{}{hydrolokator do wyrywania okrętów podwodnych}
\Clue{22}{}{jedno z najstarszych znanych narzędzi kaźni, które jako pierwszy używał Falaris, tyran sycylijskiego miasta Akragas (Agrigento) w latach ok. 570-554 p.n.e}
\Clue{23}{}{objaw kliniczny polegający na zwiększonej częstotliwości wypróżnień (według WHO ?3/24h) lub zwiększonej ilości stolca (?200g/24h), wraz ze zmianą konsystencji na płynną bądź półpłynną}
\Clue{24}{}{Thryonomyidae - rodzina ssaków z rzędu gryzoni; obejmuje dwa gatunki występujące w Afryce}
\Clue{25}{}{zawartość baryłeczki, beczki o niewielkich rozmiarach}
\Clue{26}{}{ALIKWOTY}
\Clue{27}{}{zbiorowy, figurowy taniec towarzyski pochodzenia angielskiego, popularny zwłaszcza we Francji u schyłku wieku XVIII}
\Clue{29}{}{miasto w płd.-wsch. Chinach; przemysł chemiczny}
\Clue{31}{}{inaczej stolnik; urzędnik dworski lub ziemski}
\Clue{33}{}{heroina domowej roboty}
\Clue{34}{}{PENDENTYW}
\Clue{38}{}{przyrząd sportowy, wydłużony przedmiot, który ma określony kształt i który w jakimś sporcie służy do odbijania, przemieszczania piłki, podpierania się itp}\end{PuzzleClues}\newpage\section*{Krzyżówka 67}

\noindent\begin{Puzzle}{17}{19}|*	|*	|*	|*	|*	|*	|*	|*	|*	|*	|*	|*	|*	|*	|*	|*	|[1][S]\darr	|*	|.
|*	|*	|[2][S]\darr	|*	|[3][S]\darr	|*	|*	|*	|[4][S]\rarr	|a	|m	|a	|r	|y	|l	|i	|s	|*	|.
|*	|[5][S]\rarr	|m	|a	|k	|ó	|w	|k	|a	|*	|*	|*	|*	|*	|*	|*	|o	|[6][S]\darr	|.
|*	|*	|a	|*	|l	|*	|*	|*	|*	|*	|*	|*	|[7][S]\darr	|*	|*	|*	|s	|d	|.
|*	|*	|j	|*	|u	|*	|[8][S]\rarr	|s	|m	|o	|l	|u	|c	|h	|*	|*	|[][,]{ }	|z	|.
|*	|*	|o	|[9][S]\rarr	|b	|r	|a	|n	|d	|w	|a	|c	|h	|t	|a	|*	|b	|i	|.
|*	|*	|r	|*	|*	|[10][S]\rarr	|k	|o	|r	|y	|f	|e	|u	|s	|z	|*	|e	|u	|.
|*	|*	|*	|*	|*	|*	|*	|*	|*	|[11][S]\darr	|*	|*	|t	|[12][S]\darr	|*	|*	|r	|r	|.
|*	|[13][S]\rarr	|l	|i	|c	|y	|t	|a	|c	|j	|a	|*	|l	|t	|*	|[14][S]\darr	|n	|a	|.
|*	|*	|*	|*	|*	|[15][S]\rarr	|d	|z	|i	|a	|d	|z	|i	|e	|n	|i	|e	|*	|.
|*	|*	|[16][S]\rarr	|b	|a	|n	|d	|u	|n	|g	|*	|*	|w	|r	|[17][S]\darr	|g	|ń	|*	|.
|*	|*	|*	|[18][S]\rarr	|c	|h	|a	|n	|s	|o	|n	|*	|o	|e	|w	|ł	|s	|[19][S]\darr	|.
|*	|*	|*	|[20][S]\rarr	|o	|b	|r	|z	|y	|d	|*	|*	|ś	|n	|y	|a	|k	|s	|.
|*	|*	|*	|*	|*	|*	|[21][S]\drarr	|m	|p	|a	|*	|*	|ć	|ó	|r	|*	|i	|k	|.
|*	|[22][S]\rarr	|p	|r	|e	|s	|t	|o	|n	|*	|*	|*	|*	|w	|o	|*	|*	|o	|.
|[23][S]\rarr	|p	|o	|c	|h	|ł	|a	|n	|i	|a	|c	|z	|*	|k	|b	|*	|*	|k	|.
|*	|*	|*	|*	|*	|*	|b	|[24][S]\rarr	|e	|s	|e	|s	|m	|a	|n	|k	|a	|*	|.
|[25][S]\rarr	|e	|k	|s	|k	|l	|u	|z	|y	|w	|i	|z	|m	|*	|i	|*	|*	|*	|.
|*	|*	|[26][S]\rarr	|s	|z	|y	|n	|k	|a	|*	|[27][S]\rarr	|ł	|y	|ż	|k	|a	|*	|*	|.
|*	|[28][S]\rarr	|p	|r	|y	|k	|*	|[29][S]\rarr	|m	|i	|l	|w	|i	|d	|*	|*	|*	|*	|.\end{Puzzle}

\newpage

\begin{PuzzleClues}{\textbf{Poziome}\\}\Clue{4}{}{AMARYLEK cebulkowa bylina ozdobna o wonnych, lejkowatych kwiatach}
\Clue{5}{}{powszechna nazwa owocu (puszki) roślin z rodzaju mak}
\Clue{8}{}{banknot z górnikiem o nominale 500 zł, który był w obiegu w PRL przed 1977 rokiem}
\Clue{9}{}{czata morska}
\Clue{10}{}{przewodnik chóru w teatrze starogreckim; wygłaszał on najtrudniejsze kwestie oraz wyznaczał chórzystom tempo i wysokość dźwięku}
\Clue{13}{}{w grach karcianych, np. brydżu, tysiącu element gry w karty, podczas którego gracze zabiegają o wzięcie, zagranie, wyjście itp}
\Clue{15}{}{to, że coś traci na aktualności, funkcjonalności lub atrakcyjności, starzeje się}
\Clue{16}{}{miasto w Indonezji, na Jawie ośrodek administracyjny prowincji Jawa Zachodnia}
\Clue{18}{}{odmiana piosenki francuskiej}
\Clue{20}{}{zniesmaczenie, niemiłe uczucie dezaprobaty, pogardy, wstrętu i oburzenia, które nawiedza człowieka, gdy natyka się na coś, co potępia pod względem moralnym}
\Clue{21}{}{jednostka ciśnienia, która jest równa milionowi paskali}
\Clue{22}{}{miasto w Kanadzie (Ontario), uzdrowisko}
\Clue{23}{}{absorber, urządzenie wykorzystywane np. przez fizyków, chemików, które służy do absorpcji gazów lub jego składników przez ciecz}
\Clue{24}{}{kobieta, która była najczęściej nadzorczynią w żeńskim obozie i należała do SS Gefolge (sympatyków SS); nie była jednak członkinią SS}
\Clue{25}{}{odgraniczanie się od społeczeństwa i ograniczanie dostępu do siebie samego albo swojego środowiska}
\Clue{26}{}{tylna część półtuszy wieprzowej z dużymi mięśniami (udowymi), mięso uważane za wyjątkowo szlachetne}
\Clue{27}{}{robocza część koparki w postaci naczynia osadzonego na ruchomym ramieniu}
\Clue{28}{}{lekceważąco, z pogardą: starszy, niedołężny mężczyzna}
\Clue{29}{}{XVIII w. kompozytor; jeden z pierwszych polskich twórców symfonii}\end{PuzzleClues}

\begin{PuzzleClues}{\textbf{Pionowe}\\}\Clue{1}{}{sos przygotowywany na bazie sosu holenderskiego z dodatkiem estragonu}
\Clue{2}{}{oficer noszący tytuł majora}
\Clue{3}{}{organizacja, stowarzyszenie ludzi o podobnych zainteresowaniach, poglądach, zajęciach, które ma na celu wspólne działania, wymianę myśli i nawiązywanie kontaktów towarzyskich}
\Clue{6}{}{wgłębienie w ziemi lub innej powierzchni}
\Clue{7}{}{cecha człowieka pożądliwego, łatwo podniecającego się seksualnie, nadmiernie łaknącego seksu}
\Clue{11}{}{owoc borówki czarnej, jagody}
\Clue{12}{}{samochód przystosowany do pokonywania przeszkód terenowych}
\Clue{14}{}{liść o blaszce, która jest podłużna, wąskaj, sztywna i ostra}
\Clue{17}{}{człowiek, który pracuje jako osoba wynajmowana do wykonywania różnych doraźnie potrzebnych prac}
\Clue{19}{}{nagła (taka, która nie jest stopniowa) zmiana}
\Clue{21}{}{gaz bojowy o działaniu paralityczno-drgawkowym}\end{PuzzleClues}\newpage\section*{Krzyżówka 68}

\noindent\begin{Puzzle}{20}{30}|*	|*	|*	|*	|*	|*	|*	|*	|[1][S]\darr	|*	|*	|*	|*	|*	|*	|[2][S]\darr	|*	|*	|[3][S]\darr	|*	|*	|.
|*	|*	|*	|*	|[4][S]\rarr	|u	|c	|h	|w	|y	|t	|*	|*	|*	|*	|j	|[5][S]\drarr	|s	|e	|x	|*	|.
|*	|*	|*	|*	|*	|[6][S]\darr	|*	|[7][S]\darr	|o	|*	|*	|*	|[8][S]\darr	|*	|*	|u	|m	|[9][S]\darr	|t	|*	|*	|.
|*	|*	|*	|*	|*	|w	|[10][S]\drarr	|k	|l	|o	|c	|*	|ż	|*	|*	|t	|e	|n	|e	|*	|*	|.
|*	|*	|*	|*	|*	|ę	|p	|a	|b	|*	|*	|[11][S]\darr	|ó	|*	|*	|r	|t	|e	|r	|[12][S]\darr	|*	|.
|*	|*	|[13][S]\darr	|*	|*	|z	|a	|d	|r	|*	|*	|z	|ł	|[14][S]\drarr	|f	|o	|r	|t	|*	|j	|[15][S]\darr	|.
|*	|[16][S]\darr	|t	|*	|*	|e	|z	|ź	|o	|*	|*	|a	|w	|s	|[17][S]\darr	|*	|*	|t	|[18][S]\darr	|ę	|k	|.
|*	|n	|r	|*	|*	|ł	|u	|*	|m	|*	|[19][S]\darr	|p	|[][,]{ }	|t	|w	|[20][S]\darr	|*	|o	|z	|z	|w	|.
|*	|a	|y	|[21][S]\darr	|*	|[][,]{ }	|r	|*	|i	|[22][S]\drarr	|f	|a	|b	|r	|y	|k	|a	|*	|b	|y	|a	|.
|*	|p	|c	|r	|[23][S]\darr	|p	|k	|*	|a	|m	|u	|ś	|i	|u	|p	|r	|*	|*	|r	|k	|s	|.
|*	|ę	|h	|e	|e	|o	|o	|*	|n	|ę	|j	|n	|r	|m	|ł	|u	|[24][S]\darr	|[25][S]\darr	|o	|[][,]{ }	|k	|.
|*	|d	|o	|m	|l	|t	|w	|*	|i	|t	|a	|i	|m	|i	|y	|s	|b	|h	|j	|l	|o	|.
|*	|[][,]{ }	|t	|b	|a	|r	|c	|*	|n	|l	|r	|c	|a	|e	|w	|z	|a	|i	|a	|a	|w	|.
|*	|t	|o	|r	|s	|ó	|o	|*	|*	|i	|k	|t	|ń	|n	|*	|y	|r	|p	|[][,]{ }	|t	|a	|.
|*	|a	|m	|a	|t	|j	|w	|*	|*	|k	|a	|w	|s	|i	|*	|n	|y	|s	|p	|y	|t	|.
|*	|ś	|i	|n	|o	|n	|a	|*	|*	|*	|*	|o	|k	|o	|*	|*	|k	|y	|ł	|n	|o	|.
|[26][S]\drarr	|m	|a	|d	|r	|y	|t	|c	|z	|y	|k	|*	|i	|m	|*	|*	|a	|b	|y	|o	|ś	|.
|p	|y	|*	|t	|*	|*	|e	|[27][S]\rarr	|d	|i	|a	|z	|*	|i	|*	|*	|d	|e	|t	|f	|ć	|.
|o	|*	|[28][S]\darr	|*	|*	|*	|*	|*	|*	|*	|[29][S]\rarr	|w	|i	|e	|t	|n	|a	|m	|k	|a	|*	|.
|d	|[30][S]\drarr	|k	|i	|e	|ł	|ż	|[][,]{ }	|j	|e	|z	|i	|o	|r	|n	|y	|*	|a	|o	|l	|*	|.
|w	|g	|*	|*	|*	|[31][S]\darr	|[32][S]\rarr	|w	|p	|i	|e	|r	|d	|z	|i	|e	|l	|*	|w	|i	|*	|.
|o	|r	|*	|*	|[33][S]\rarr	|r	|a	|c	|i	|c	|z	|k	|a	|*	|*	|*	|*	|*	|a	|s	|[34][S]\darr	|.
|d	|o	|*	|[35][S]\rarr	|k	|o	|m	|p	|a	|t	|y	|b	|i	|l	|n	|o	|ś	|ć	|*	|k	|a	|.
|z	|s	|[36][S]\rarr	|w	|i	|z	|y	|t	|a	|[][,]{ }	|s	|t	|u	|d	|y	|j	|n	|a	|*	|i	|l	|.
|i	|z	|*	|*	|[37][S]\rarr	|ł	|u	|s	|z	|c	|z	|*	|*	|*	|*	|*	|*	|[38][S]\darr	|*	|*	|c	|.
|e	|*	|*	|[39][S]\rarr	|b	|u	|d	|u	|a	|r	|e	|k	|*	|*	|*	|*	|*	|m	|*	|*	|o	|.
|*	|[40][S]\drarr	|k	|r	|o	|p	|k	|o	|w	|a	|n	|i	|e	|*	|[41][S]\rarr	|s	|c	|a	|t	|*	|c	|.
|[42][S]\drarr	|j	|ę	|z	|y	|k	|[][,]{ }	|u	|g	|r	|y	|j	|s	|k	|i	|*	|*	|g	|*	|*	|k	|.
|b	|o	|*	|*	|[43][S]\drarr	|a	|n	|a	|c	|h	|r	|o	|n	|i	|c	|z	|n	|o	|ś	|ć	|*	|.
|i	|l	|*	|*	|s	|*	|*	|*	|*	|*	|*	|*	|[44][S]\rarr	|l	|e	|k	|y	|t	|*	|*	|*	|.
|*	|*	|*	|*	|*	|*	|*	|*	|*	|*	|*	|*	|*	|*	|*	|*	|*	|*	|*	|*	|*	|.\end{Puzzle}

\newpage

\begin{PuzzleClues}{\textbf{Poziome}\\}\Clue{4}{}{element jakiegoś przedmiotu służący do trzymania go ręką lub przeznaczony do przytrzymywania czegoś}
\Clue{5}{}{stosunek seksualny, zbliżenie fizyczne dwojga osób; określenie używane w zapisie}
\Clue{10}{}{kał, odchody}
\Clue{14}{}{fortyfikacja (budowla obronna) polowa lub stała, budowana od XVII do początków XX w., przystosowana do obrony okrężnej}
\Clue{22}{}{zakład produkcyjny, wytwarzający produkty na dużą skalę w oparciu o pracę ludzi i maszyn}
\Clue{26}{}{mieszkaniec Madrytu}
\Clue{27}{}{żeglarz portugalski (1450-1500); odkrył Przylądek Dobrej Nadziei}
\Clue{29}{}{mieszkanka Wietnamu, kobieta pochodzenia wietnamskiego}
\Clue{30}{}{Gammarus lacustris - słodkowodny gatunek skorupiaka z rzędu obunogów}
\Clue{32}{}{nauczka, sytuacja, w której ktoś dostaje dotkliwą lekcję, wycisk}
\Clue{33}{}{szczątkowa pokrywa rogowa uwstecznionych palców II i V u parzystokopytnych}
\Clue{35}{}{techniczne dopasowanie, możliwość współpracowania urządzeń lub programów lub ich części}
\Clue{36}{}{rodzaj szkolenia wyjazdowego w celu poznania metod pracy danego ośrodka}
\Clue{37}{}{amerykański ptak z ziarnojadów}
\Clue{39}{}{mały, niewielki buduar - wytwornie umeblowany i wykwintnie ozdobiony, niewielki pokój zajmowany przez panią domu}
\Clue{40}{}{kładzenie (rysowanie, wytłaczanie, pisanie itp.) kropek na jakiejś powierzchni}
\Clue{41}{}{charakterystyczny dla jazzu dźwiękonaśladowczy sposób śpiewania}
\Clue{42}{}{język z podgrupy języków ugryjskich}
\Clue{43}{}{niewłaściwość chronologiczna, niezgodność z rzeczywistymi stosunkami czasowymi}
\Clue{44}{}{starożytne naczynie służące do przechowywania oliwy}\end{PuzzleClues}

\begin{PuzzleClues}{\textbf{Pionowe}\\}\Clue{1}{}{mieszkaniec Wolbromia}
\Clue{2}{}{dzień następny w stosunku do obecnego}
\Clue{3}{}{substancja wypełniająca wszechświat; przez filozofów greckich określany jako pierwotna materia, zdaniem XIX-wiecznych filozofów eter stanowił ośrodek, w którym rozchodzą się fale świetlne}
\Clue{5}{}{podstawowa jednostka długości w układzie SI}
\Clue{6}{}{strefa w skorupie ziemskiej, będąca punktem styku trzech płyt tektonicznych}
\Clue{7}{}{duże otwarte naczynie, dawniej wykonane zwykle z drewna, współcześnie także z metalu lub betonu}
\Clue{8}{}{Geochelone platynota - gatunek gada z rodziny żółwi lądowych, krytycznie zagrożony wyginięciem, występujący w Birmie}
\Clue{9}{}{waga bez opakowania}
\Clue{10}{}{pazurkowce, Callitrichinae, Hapalinae - podrodzina małp z rzędu naczelnych obejmująca między innymi marmozety i tamaryny; wszystkie gatunki zamieszkują Amerykę Południową}
\Clue{11}{}{zapasy - sport walki, polegający na fizycznym zmaganiu dwóch zawodników, których walka odbywa się wręcz przez stosowanie chwytów; początki tego sportu sięgają czasów starożytnych}
\Clue{12}{}{język z podgrupy języków latynofaliskich}
\Clue{13}{}{podział na trzy części}
\Clue{14}{}{przyrząd do pomiaru strumienia indukcji magnetycznej}
\Clue{15}{}{nieco kwaśny posmak lub zapach czegoś}
\Clue{16}{}{najczęściej metalowy wałek będący osią obrotu koła zamachowego, do którego taśma przyciskana jest sprężyną za pośrednictwem gumowej rolki}
\Clue{17}{}{zjawisko polegające na tym, że jakaś substancja wycieka, wypływa z miejsca (np. zbiornika), w którym powinna być}
\Clue{18}{}{zbroja wykonana z płyt metalowych}
\Clue{19}{}{instrument ludowy, piszczałka z kory dębu}
\Clue{20}{}{wieś w Polsce położona w województwie kujawsko-pomorskim, w powiecie bydgoskim, w gminie Sicienko}
\Clue{21}{}{Rembrandt Harmenszoon van Rijn, holenderski malarz, rysownik i grafik}
\Clue{22}{}{zamęt, zamieszanie z jakiegoś powodu; bezład, bałagan}
\Clue{23}{}{polska nazwa przędzy uzyskiwanej przez termiczną modyfikację jedwabiu poliestrowego (torlenu), używany do wyrobu tkanin w przemyśle jedwabniczym}
\Clue{24}{}{doraźna fortyfikacja przegradzająca w całości przejazd danym szlakiem komunikacyjnym, zbudowana w miejscu uniemożliwiającym jego bezpośrednie obejście, zbudowana z materiałów dostępnych na miejscu lub zgromadzonym specjalnie w tym celu}
\Clue{25}{}{Hypsibema - rodzaj roślinożernego dinozaura z nadrodziny hadrozauroidów (Hadrosauroidea); żył w okresie późnej kredy (83-71 mln lat temu) na terenach Ameryki Północnej; długość ciała 9 m, wysokość 4 m, ciężar 2,5 t}
\Clue{26}{}{część kadłuba statku zanurzona w wodzie}
\Clue{28}{}{w chemii: symbol potasu}
\Clue{30}{}{1 monetka o wartości jednego grosza}
\Clue{31}{}{w botanice: rodzaj suchego niepękającego owocu o jednym nasieniu, często nie występuje pojedynczo, lecz jako np. część rozłupni lub wieloskrzydlaka}
\Clue{34}{}{lotnik angielski (1892-1919) dokonał pierwszego w świecie przelotu wraz z Brownem przez Atlantyk}
\Clue{38}{}{małpa wąskonosa; jedyny gatunek małpy w Europie}
\Clue{40}{}{typ dwumasztowego ożaglowania stosowanego przeważnie na jachtach żaglowych średniej wielkości (o długości kadłuba 10 - 20 m)}
\Clue{42}{}{proces przekształcania danych w informacje, a informacji w wiedzę, która może być wykorzystana do zwiększenia konkurencyjności przedsiębiorstwa}
\Clue{43}{}{w chemii: symbol siarki}\end{PuzzleClues}\newpage\section*{Krzyżówka 69}

\noindent\begin{Puzzle}{24}{34}|*	|*	|[1][S]\darr	|*	|*	|*	|*	|*	|*	|*	|*	|*	|*	|*	|*	|*	|*	|*	|*	|*	|*	|*	|*	|*	|*	|.
|*	|*	|w	|*	|*	|*	|*	|*	|*	|*	|*	|*	|*	|*	|*	|*	|*	|*	|*	|*	|*	|*	|*	|*	|*	|.
|*	|*	|a	|*	|*	|*	|*	|*	|*	|*	|*	|*	|*	|*	|*	|*	|*	|*	|*	|*	|*	|*	|*	|*	|*	|.
|*	|*	|l	|*	|*	|*	|*	|*	|*	|*	|*	|*	|*	|*	|*	|*	|*	|*	|*	|*	|*	|*	|*	|*	|*	|.
|*	|*	|t	|*	|*	|*	|*	|*	|*	|*	|*	|*	|*	|*	|*	|*	|*	|*	|*	|*	|*	|*	|*	|*	|*	|.
|*	|*	|o	|*	|*	|*	|*	|*	|*	|*	|*	|*	|*	|*	|*	|*	|*	|*	|*	|*	|*	|*	|*	|*	|*	|.
|*	|*	|r	|*	|*	|[2][S]\darr	|*	|*	|*	|*	|*	|*	|*	|*	|*	|*	|*	|*	|*	|*	|*	|*	|*	|*	|*	|.
|*	|*	|n	|*	|*	|c	|*	|*	|*	|*	|*	|*	|*	|*	|*	|*	|*	|*	|*	|*	|*	|*	|*	|*	|*	|.
|*	|*	|i	|*	|*	|z	|*	|*	|*	|*	|*	|[3][S]\darr	|*	|*	|*	|*	|*	|*	|*	|*	|*	|*	|*	|*	|*	|.
|[4][S]\drarr	|k	|a	|s	|s	|a	|t	|a	|*	|*	|*	|d	|*	|*	|*	|*	|*	|*	|*	|*	|*	|*	|*	|*	|*	|.
|l	|*	|*	|*	|*	|p	|*	|*	|*	|*	|*	|w	|*	|*	|*	|*	|*	|*	|*	|*	|*	|*	|*	|*	|*	|.
|o	|*	|*	|*	|*	|l	|*	|*	|*	|*	|*	|u	|*	|*	|*	|*	|*	|*	|*	|[5][S]\darr	|*	|*	|[6][S]\darr	|*	|*	|.
|t	|[7][S]\rarr	|k	|e	|t	|a	|*	|*	|*	|*	|*	|s	|*	|*	|*	|*	|*	|*	|*	|t	|*	|*	|p	|*	|*	|.
|n	|*	|*	|*	|*	|[][,]{ }	|*	|*	|*	|*	|*	|z	|*	|*	|*	|*	|*	|*	|*	|y	|*	|*	|i	|*	|*	|.
|i	|*	|*	|*	|*	|k	|*	|*	|*	|*	|*	|c	|*	|*	|*	|*	|*	|*	|*	|l	|*	|*	|e	|*	|*	|.
|c	|*	|[8][S]\rarr	|s	|t	|a	|r	|y	|*	|*	|*	|z	|*	|*	|*	|*	|*	|*	|*	|o	|*	|*	|c	|*	|*	|.
|t	|*	|*	|[9][S]\drarr	|t	|r	|z	|y	|d	|z	|i	|e	|s	|t	|y	|[][,]{ }	|p	|i	|e	|r	|w	|s	|z	|y	|*	|.
|w	|*	|*	|d	|*	|a	|*	|*	|*	|*	|*	|b	|*	|*	|*	|*	|*	|*	|*	|*	|*	|*	|y	|*	|*	|.
|o	|*	|*	|o	|*	|i	|*	|*	|*	|*	|*	|l	|*	|*	|*	|*	|*	|*	|*	|*	|*	|*	|w	|*	|*	|.
|[][,]{ }	|*	|*	|k	|*	|b	|*	|*	|[10][S]\drarr	|a	|z	|o	|t	|e	|k	|[][,]{ }	|b	|o	|r	|u	|*	|*	|o	|*	|*	|.
|w	|*	|*	|t	|*	|s	|*	|*	|w	|*	|*	|w	|*	|*	|*	|*	|*	|*	|*	|*	|*	|*	|[][,]{ }	|*	|*	|.
|o	|*	|*	|o	|*	|k	|*	|*	|y	|*	|*	|o	|*	|*	|*	|*	|*	|*	|*	|*	|*	|*	|c	|*	|*	|.
|j	|*	|*	|r	|*	|a	|*	|*	|n	|[11][S]\darr	|*	|ś	|*	|*	|*	|*	|*	|*	|*	|*	|*	|*	|h	|*	|*	|.
|s	|*	|*	|a	|*	|*	|*	|*	|u	|s	|*	|ć	|*	|*	|*	|*	|*	|*	|*	|*	|*	|*	|r	|*	|*	|.
|k	|*	|*	|t	|*	|[12][S]\rarr	|s	|t	|r	|a	|ż	|*	|*	|*	|*	|*	|*	|*	|*	|*	|*	|*	|u	|*	|*	|.
|o	|*	|*	|[][,]{ }	|*	|*	|*	|*	|t	|m	|*	|*	|*	|*	|*	|*	|*	|*	|*	|*	|*	|*	|p	|*	|*	|.
|w	|*	|[13][S]\rarr	|h	|a	|r	|a	|m	|*	|o	|*	|*	|*	|*	|*	|*	|*	|*	|*	|*	|*	|*	|k	|*	|*	|.
|e	|*	|*	|o	|*	|*	|*	|*	|*	|u	|*	|*	|*	|*	|*	|*	|*	|*	|*	|*	|*	|*	|i	|*	|*	|.
|*	|*	|*	|n	|*	|*	|*	|*	|*	|c	|*	|*	|*	|*	|*	|*	|*	|*	|*	|*	|*	|*	|e	|*	|*	|.
|*	|*	|*	|o	|*	|*	|*	|*	|*	|z	|*	|*	|*	|*	|*	|*	|*	|*	|*	|*	|*	|*	|*	|*	|*	|.
|*	|*	|*	|r	|*	|*	|*	|*	|*	|e	|*	|*	|*	|*	|*	|*	|*	|*	|*	|*	|*	|*	|*	|*	|*	|.
|*	|*	|*	|o	|*	|*	|*	|*	|*	|k	|*	|*	|*	|*	|*	|*	|*	|*	|*	|*	|*	|*	|*	|*	|*	|.
|*	|*	|*	|w	|*	|*	|*	|*	|*	|*	|*	|*	|*	|*	|*	|*	|*	|*	|*	|*	|*	|*	|*	|*	|*	|.
|*	|*	|*	|y	|*	|*	|*	|*	|*	|*	|*	|*	|*	|*	|*	|*	|*	|*	|*	|*	|*	|*	|*	|*	|*	|.
|*	|*	|*	|*	|*	|*	|*	|*	|*	|*	|*	|*	|*	|*	|*	|*	|*	|*	|*	|*	|*	|*	|*	|*	|*	|.\end{Puzzle}

\newpage

\begin{PuzzleClues}{\textbf{Poziome}\\}\Clue{4}{}{lody włoskie złożone z kilku warstw o różnych smakach z bakaliami}
\Clue{7}{}{łosoś pacyficzny o długości do 1 m}
\Clue{8}{}{familiarna forma używana przy zwracaniu się do dobrego znajomego}
\Clue{9}{}{trzydziesty pierwszy dzień (najczęściej bieżącego lub przyszłego) miesiąca}
\Clue{10}{}{nieorganiczny związek chemiczny boru i azotu, zsyntetyzowany po raz pierwszy w 1842 roku}
\Clue{12}{}{pełnienie warty}
\Clue{13}{}{sala modlitw w meczecie stanowiąca jego zasadniczą część}\end{PuzzleClues}

\begin{PuzzleClues}{\textbf{Pionowe}\\}\Clue{1}{}{RÓG; instrument dęty, blaszany, mający kształt długiej, wąskiej rury zwiniętej w trzy kręgi}
\Clue{2}{}{Ardea herodias occidentalis - podgatunek czapli modrej (Ardea herodias); występuje od południowo-wschodnich Stanów Zjednoczonych (Floryda) po Karaiby}
\Clue{3}{}{wzajemne relacje pomiędzy poszczególnymi szczeblami, poziomami, oddziałami, np. władzy (centralna i samorządowa), banków (centralny i komercyjne)}
\Clue{4}{}{element powietrznych sił zbrojnych, którego podstawową bronią są różnego rodzaju statki powietrzne}
\Clue{5}{}{angielski etnolog i religioznawca (1832-1917); ewolucjonizm}
\Clue{6}{}{rodzaj pieczywa uważanego za dietetyczne; cienkie i lekkie osobne kromki z różnych przetworów zbożowych}
\Clue{9}{}{stopień naukowy, który uczelnie nadają osobom szczególnie zasłużonym dla nauki}
\Clue{10}{}{owad znajdujący się u nas pod ochroną}
\Clue{11}{}{program pozwalający łatwo nauczyć się obsługi aplikacji, programowania czy tworzenia grafiki albo fragment będący częścią większego programu (np. gry komputerowej), który jest formą wprowadzenia, treningu pozwalającego na oswojenie się z programem, poznanie zasad i naukę podstaw}\end{PuzzleClues}\newpage\section*{Krzyżówka 70}

\noindent\begin{Puzzle}{23}{19}|*	|*	|*	|*	|*	|*	|*	|[1][S]\drarr	|s	|a	|m	|o	|i	|s	|t	|n	|i	|e	|n	|i	|e	|*	|*	|*	|.
|*	|*	|[2][S]\rarr	|b	|u	|d	|o	|w	|n	|i	|c	|t	|w	|o	|[][,]{ }	|l	|ą	|d	|o	|w	|e	|*	|*	|*	|.
|*	|[3][S]\rarr	|a	|m	|l	|o	|d	|y	|p	|i	|n	|a	|*	|*	|[4][S]\darr	|*	|*	|*	|*	|*	|*	|*	|*	|*	|.
|*	|*	|*	|*	|*	|*	|[5][S]\rarr	|s	|k	|a	|r	|b	|[][,]{ }	|p	|a	|ń	|s	|t	|w	|o	|w	|y	|*	|*	|.
|*	|[6][S]\rarr	|t	|w	|a	|r	|d	|z	|i	|e	|l	|*	|*	|[7][S]\rarr	|n	|a	|d	|p	|i	|s	|*	|*	|[8][S]\darr	|*	|.
|*	|*	|*	|[9][S]\rarr	|t	|a	|t	|e	|r	|s	|a	|l	|*	|*	|o	|*	|*	|*	|*	|[10][S]\darr	|*	|[11][S]\darr	|c	|*	|.
|*	|[12][S]\rarr	|p	|r	|z	|y	|c	|h	|ó	|w	|e	|k	|*	|*	|r	|*	|*	|*	|*	|m	|*	|w	|u	|*	|.
|*	|[13][S]\drarr	|k	|o	|l	|i	|b	|r	|z	|y	|k	|[][,]{ }	|b	|i	|a	|ł	|o	|b	|r	|e	|w	|y	|*	|*	|.
|*	|l	|[14][S]\darr	|[15][S]\drarr	|t	|o	|j	|a	|d	|*	|*	|*	|*	|[16][S]\drarr	|k	|u	|r	|a	|*	|j	|[17][S]\darr	|s	|[18][S]\darr	|*	|.
|*	|e	|n	|o	|[19][S]\drarr	|r	|e	|d	|l	|e	|r	|*	|*	|g	|*	|[20][S]\darr	|*	|*	|*	|l	|t	|p	|c	|*	|.
|[21][S]\drarr	|j	|a	|k	|o	|ś	|ć	|*	|*	|*	|*	|[22][S]\drarr	|t	|r	|a	|b	|a	|n	|t	|*	|r	|a	|y	|*	|.
|p	|e	|t	|r	|w	|*	|*	|[23][S]\darr	|*	|*	|*	|d	|[24][S]\drarr	|o	|b	|r	|o	|s	|t	|k	|i	|*	|n	|*	|.
|o	|k	|y	|o	|c	|*	|*	|ć	|*	|*	|*	|e	|d	|t	|*	|z	|[25][S]\rarr	|k	|l	|u	|s	|k	|i	|*	|.
|r	|*	|w	|p	|a	|*	|*	|w	|*	|*	|*	|g	|r	|e	|*	|e	|*	|*	|*	|*	|t	|*	|k	|*	|.
|z	|*	|i	|n	|*	|*	|*	|i	|*	|*	|*	|*	|z	|s	|*	|ż	|*	|*	|[26][S]\rarr	|k	|e	|g	|*	|*	|.
|ą	|[27][S]\rarr	|s	|o	|f	|c	|i	|k	|*	|*	|[28][S]\rarr	|n	|a	|k	|ł	|a	|d	|*	|*	|*	|*	|*	|*	|*	|.
|d	|*	|t	|ś	|*	|*	|*	|*	|*	|*	|[29][S]\rarr	|c	|z	|a	|r	|n	|u	|c	|h	|*	|*	|*	|*	|*	|.
|e	|*	|a	|ć	|*	|[30][S]\rarr	|t	|a	|i	|p	|i	|n	|g	|*	|*	|i	|*	|*	|*	|*	|*	|*	|*	|*	|.
|k	|*	|*	|*	|*	|[31][S]\rarr	|d	|r	|u	|ż	|y	|n	|a	|*	|[32][S]\rarr	|n	|a	|s	|a	|d	|a	|*	|*	|*	|.
|*	|*	|*	|*	|*	|*	|*	|*	|*	|*	|*	|*	|*	|*	|*	|*	|*	|*	|*	|*	|*	|*	|*	|*	|.\end{Puzzle}

\newpage

\begin{PuzzleClues}{\textbf{Poziome}\\}\Clue{1}{}{istnienie niezależnie od niczego, np. samoistnienie Boga, diabła, narodów}
\Clue{2}{}{dział budownictwa zajmujący się projektowaniem, technologią wykonania oraz samym wykonaniem ustrojów budowlanych istniejących na lądzie}
\Clue{3}{}{wielofunkcyjny organiczny związek chemiczny z grupy pirydyn, lek stosowany w leczeniu nadciśnienia tętniczego, o działaniu blokującym wolne kanały wapniowe}
\Clue{5}{}{w dawnej Polsce: finanse wykorzystywane na potrzeby państwa}
\Clue{6}{}{Scleroma - rodzaj przewlekłego, swoistego zapalenia wytwórczego górnych dróg oddechowych}
\Clue{7}{}{nagłówek}
\Clue{9}{}{dawna ujeżdżalnia koni i miejsce ich sprzedaży}
\Clue{12}{}{wzrost liczby zwierząt hodowlanych lub łownych}
\Clue{13}{}{Polytmus guainumbi - gatunek ptaka z rzędu jerzykowych (Apodiformes), z rodziny kolibrów (Trochilidae), z podrodziny kolibrów (Trochilinae)}
\Clue{15}{}{AKONIT trująca i lecznicza bylina o niebieskofioletowych kwiatach, w Polsce w górach, rabatowa, chroniona}
\Clue{16}{}{mięso kury}
\Clue{19}{}{przenośnik do transportu materiałów sypkich}
\Clue{21}{}{coś; pewne cechy, które składają się na byt, wyróżniają go spośród innych}
\Clue{22}{}{żołnierz formacji wojskowej będącej strażą przyboczną wyższych oficerów}
\Clue{24}{}{Dasypodainae - niewielka podrodzina należąca do grupy pszczół właściwych}
\Clue{25}{}{zbiorcze określenie grupy potraw mącznych, często z dodatkiem ziemniaków i jajek}
\Clue{26}{}{zawartość kega, beczki na piwo}
\Clue{27}{}{coś, co (często na płaszczyźnie porównania) postrzegane jest jako błahe, delikatne, niewymagające, charakteryzujące się znacznie mniejszym natężeniem jakiejś cechy niż coś innego}
\Clue{28}{}{liczba egzemplarzy jednego wydania książki, gazety, gry lub wydawnictwa płytowego}
\Clue{29}{}{ktoś, kto ma ciemną lub śniadą karnację, np. dzięki opaleniznie, ale nie należy do czarnej rasy}
\Clue{30}{}{miasto w Malezji (Perak); ośrodek regionu wydobycia rud cyny}
\Clue{31}{}{w Polsce do XII w. trzon sił zbrojnych książąt lub oddział przyboczny wodzów plemiennych}
\Clue{32}{}{to na czym coś jest osadzone: rękojeść, trzonek, oprawa}\end{PuzzleClues}

\begin{PuzzleClues}{\textbf{Pionowe}\\}\Clue{1}{}{historyczne miasto w północnych Węgrzech, 35 km na północ od Budapesztu, w Zakolu Dunaju (naprzeciwko miasta Nagymaros)}
\Clue{4}{}{nieprzemakalna kurtka z kapturem i ściągaczem noszona przez Eskimosów i polarników}
\Clue{8}{}{w chemii: symbol miedzi}
\Clue{10}{}{wiadomość tekstowa otrzymana za pośrednictwem poczty elektronicznej}
\Clue{11}{}{struktura anatomiczna w ludzkim mózgu zaliczana do kresomózgowia}
\Clue{13}{}{Gomphus clavatus (Pers.) Gray - gatunek grzyba należący do rodziny siatkolistowatych (Gomphaceae); w Polsce bardzo rzadki i prawnie chroniony}
\Clue{14}{}{zwolennik natywizmu - poglądu w psychologii}
\Clue{15}{}{coś okropnego, szokującego}
\Clue{16}{}{utwór plastyczny o komicznie przejaskrawionych, nieprawdopodobnych elementach}
\Clue{17}{}{określenie wykonawcze; smutnie, żałośnie}
\Clue{18}{}{osoba złośliwa, kpiąca z wartości i zasad moralnych}
\Clue{19}{}{ssak z  rodziny krętorogich}
\Clue{20}{}{mieszkaniec Brzegu oraz Brzegu Dolnego}
\Clue{21}{}{relacja pomiędzy elementami, która sprawia, że odbieramy coś jako uporządkowane}
\Clue{22}{}{oznaczenie stopnia - jednostki miary kąta płaskiego, równej 1/360 kąta pełnego czyli 1/90 kąta prostego}
\Clue{23}{}{osoba chytra i przebiegła}
\Clue{24}{}{ostry, niewielki kawałek zwykle drewna, rzadziej: metalu, szkła czy innego tworzywa}\end{PuzzleClues}\newpage\section*{Krzyżówka 71}

\noindent\begin{Puzzle}{25}{25}|*	|*	|*	|*	|*	|*	|*	|*	|*	|*	|*	|*	|*	|*	|[1][S]\darr	|*	|*	|*	|[2][S]\drarr	|b	|u	|r	|n	|u	|s	|*	|.
|*	|*	|*	|*	|*	|*	|*	|*	|*	|*	|*	|*	|*	|*	|p	|*	|[3][S]\drarr	|g	|n	|i	|a	|z	|d	|k	|o	|*	|.
|*	|*	|*	|[4][S]\rarr	|p	|r	|a	|w	|o	|s	|ł	|a	|w	|i	|e	|*	|i	|*	|a	|*	|[5][S]\darr	|[6][S]\darr	|*	|*	|*	|*	|.
|*	|*	|*	|*	|*	|*	|[7][S]\drarr	|k	|a	|m	|e	|r	|d	|y	|n	|e	|r	|*	|g	|*	|p	|ł	|*	|[8][S]\darr	|*	|*	|.
|*	|*	|*	|*	|*	|*	|p	|[9][S]\darr	|*	|*	|*	|*	|[10][S]\darr	|*	|i	|*	|g	|*	|a	|*	|o	|u	|[11][S]\darr	|ś	|*	|[12][S]\darr	|.
|*	|*	|*	|*	|[13][S]\drarr	|a	|r	|a	|m	|e	|i	|z	|m	|*	|t	|[14][S]\darr	|a	|*	|[][,]{ }	|*	|w	|p	|k	|c	|*	|p	|.
|*	|*	|*	|*	|p	|*	|a	|g	|*	|*	|*	|*	|i	|*	|e	|p	|[][,]{ }	|*	|p	|*	|i	|e	|a	|i	|*	|o	|.
|*	|*	|*	|*	|a	|*	|s	|e	|[15][S]\darr	|*	|*	|*	|e	|*	|n	|a	|p	|*	|r	|*	|ś	|k	|p	|a	|*	|r	|.
|*	|*	|*	|[16][S]\rarr	|k	|u	|a	|n	|d	|u	|*	|*	|r	|*	|t	|l	|o	|*	|a	|[17][S]\darr	|l	|[][,]{ }	|r	|n	|[18][S]\darr	|y	|.
|*	|*	|*	|*	|s	|*	|*	|t	|e	|*	|*	|*	|z	|*	|*	|u	|z	|*	|w	|l	|a	|p	|y	|a	|d	|w	|.
|*	|*	|*	|*	|e	|*	|*	|*	|k	|*	|*	|*	|e	|*	|*	|c	|i	|*	|d	|e	|n	|l	|s	|[][,]{ }	|u	|c	|.
|*	|*	|*	|*	|*	|*	|*	|*	|l	|*	|[19][S]\drarr	|o	|n	|y	|c	|h	|o	|f	|a	|g	|i	|a	|*	|p	|c	|z	|.
|*	|*	|*	|*	|[20][S]\rarr	|ł	|a	|g	|i	|e	|w	|n	|i	|k	|i	|*	|m	|*	|*	|n	|n	|m	|*	|r	|h	|o	|.
|*	|*	|*	|*	|[21][S]\drarr	|w	|o	|l	|n	|o	|a	|m	|e	|r	|y	|k	|a	|n	|k	|a	|*	|i	|[22][S]\darr	|z	|o	|ś	|.
|[23][S]\rarr	|c	|z	|a	|k	|u	|e	|l	|a	|*	|g	|[24][S]\darr	|*	|*	|*	|*	|*	|*	|*	|g	|*	|s	|g	|e	|w	|ć	|.
|*	|*	|*	|*	|o	|*	|*	|[25][S]\drarr	|c	|h	|o	|r	|ą	|ż	|y	|*	|*	|[26][S]\rarr	|k	|o	|r	|t	|o	|w	|y	|*	|.
|*	|*	|*	|*	|s	|*	|*	|n	|j	|*	|n	|i	|*	|*	|*	|*	|*	|*	|*	|*	|*	|y	|ł	|i	|[][,]{ }	|*	|.
|*	|*	|*	|*	|s	|*	|*	|l	|a	|[27][S]\drarr	|i	|n	|f	|o	|r	|m	|a	|t	|y	|k	|a	|*	|ę	|e	|o	|*	|.
|*	|*	|*	|*	|a	|*	|*	|p	|*	|a	|k	|g	|*	|*	|*	|*	|*	|*	|*	|*	|*	|*	|b	|s	|j	|*	|.
|*	|*	|*	|[28][S]\rarr	|k	|o	|s	|z	|e	|r	|*	|*	|*	|*	|*	|*	|*	|*	|[29][S]\darr	|*	|*	|*	|i	|z	|c	|*	|.
|*	|*	|*	|*	|*	|*	|*	|*	|[30][S]\rarr	|k	|o	|ż	|u	|s	|z	|y	|s	|k	|o	|*	|*	|*	|e	|o	|i	|*	|.
|*	|[31][S]\rarr	|s	|t	|r	|e	|e	|t	|b	|a	|l	|l	|*	|*	|*	|*	|*	|*	|b	|*	|*	|*	|*	|n	|e	|*	|.
|*	|*	|[32][S]\rarr	|k	|o	|c	|z	|k	|o	|d	|a	|n	|i	|k	|*	|*	|*	|*	|r	|*	|*	|*	|*	|a	|c	|*	|.
|*	|*	|[33][S]\rarr	|r	|ó	|w	|n	|a	|n	|i	|e	|[][,]{ }	|r	|ó	|w	|n	|o	|w	|a	|ż	|n	|e	|*	|*	|*	|*	|.
|*	|*	|*	|[34][S]\rarr	|p	|r	|o	|g	|r	|a	|m	|[][,]{ }	|g	|r	|a	|f	|i	|c	|z	|n	|y	|*	|*	|*	|*	|*	|.
|*	|[35][S]\rarr	|r	|a	|j	|t	|u	|z	|y	|*	|*	|*	|*	|*	|*	|*	|*	|*	|*	|*	|*	|*	|*	|*	|*	|*	|.\end{Puzzle}

\newpage

\begin{PuzzleClues}{\textbf{Poziome}\\}\Clue{2}{}{długie obszerne okrycie z kapturem, bez rękawów z gęstego materiału noszone przez Arabów; szeroki płaszcz o kroju przypominającym takie okrycie}
\Clue{3}{}{zdrobniale: gniazdo - schronienie zwierzęce, różnego typu konstrukcja, którą wykonanują zwierzęta jako schronienie, a zwłaszcza jako miejsce wylęgu i odchowywania młodych}
\Clue{4}{}{wschodni odłam chrześcijaństwa}
\Clue{7}{}{lokaj będący na usługach tylko jednej osoby, osobisty służący}
\Clue{13}{}{bezpośrednie zapożyczenie językowe pochodzące z języka aramejskiego lub twór słowny czy konstrukcja językowa podobne do jakichś charakterystycznych części tego języka}
\Clue{16}{}{koendu, Coendou prehensilis - gatunek gryzonia z rodziny ursonowatych; zamieszkuje lasy Ameryki Środkowej i Południowej}
\Clue{19}{}{szkodliwy nawyk polegający na notorycznym obgryzaniu paznokci}
\Clue{20}{}{dzielnica Krakowa}
\Clue{21}{}{styl walk zapaśniczych, gdzie wszystkie chwyty są dozwolone}
\Clue{23}{}{Sauromalus obesus - gatunek gada z rodziny legwanowatych, występujący w południowo-zachodniej części USA oraz w północnym Meksyku}
\Clue{25}{}{żołnierz, który niesie sztandar}
\Clue{26}{}{ten, kto obsługuje kort - plac przystosowany do gry w tenisa}
\Clue{27}{}{gałąź nauki i techniki zajmująca się teorią i technologią przetwarzania informacji}
\Clue{28}{}{potrawy, naczynia lub przedmioty uznawane w judaizmie za rytualnie czyste, także przepisy określające zachowanie rytualnej czystości tych rzeczy}
\Clue{30}{}{zgrubiale: kożuch - ciepła odzież wierzchnia (płaszcz lub kurtka)}
\Clue{31}{}{rekreacyjna gra w koszykówkę o nieprecyzyjnych zasadach, najczęściej rozgrywana na asfalcie, bruku, kostce lub na hali, z użyciem jednego lub dwóch koszów; popularna wśród subkultur młodzieżowych}
\Clue{32}{}{zdrobniale: koczkodan; nazwa młodego koczkodana}
\Clue{33}{}{równanie, które ma ten sam zbiór rozwiązań jak inne}
\Clue{34}{}{użytkowy program komputerowy służący do tworzenia i modyfikacji plików graficznych}
\Clue{35}{}{grube rajstopy dziecięce (wynonane z bawełny)}\end{PuzzleClues}

\begin{PuzzleClues}{\textbf{Pionowe}\\}\Clue{1}{}{w Kościele katolickim: osoba przystępująca do spowiedzi}
\Clue{2}{}{czysta, nieobleczona w zbędne treści i dodatki prawda; szczery, prosty, faktyczny stan czegoś, informacja o czymś}
\Clue{3}{}{Cotoneaster horizontalis - gatunek niskiego krzewu z rodziny różowatych, dorastający do 0,5-1 m wysokości, do terenów występowania krzewu zalicza się głównie zachodnie Chiny}
\Clue{5}{}{mieszkaniec Powiśla (dzielnicy Warszawy)}
\Clue{6}{}{skała metamorficzna powstała w wyniku przeobrażenia łupków ilastych w wyniku kontaktu z intruzją}
\Clue{7}{}{maszyna robocza, której działaniem jest wywieranie nacisku na materiał lub przedmiot}
\Clue{8}{}{wybrzuszony fragment ścian skalnych, stanowiący część ściany nachyloną do poziomu pod kątem przekraczającym 90°}
\Clue{9}{}{osoba zatrudniona przez artystę lub pisarza, zajmująca się jego kontraktami i reklamą, impresario}
\Clue{10}{}{określanie rozmiarów czegoś lub kogoś}
\Clue{11}{}{zachcianka, nagła ochota na coś}
\Clue{12}{}{cecha zachowania, które pokazuje, że ktoś jest porwyczy}
\Clue{13}{}{miasto w płd. Laosie, ośrodek administracyjny prowincji Sedan, port nad Mekongiem}
\Clue{14}{}{duży palec u stopy}
\Clue{15}{}{zboczenie}
\Clue{17}{}{miasto we Włoszech (Wenecja Euganejska) nad Adygą; ośrodek handlowy regionu rolniczego}
\Clue{18}{}{ktoś, kto zainicjował istnienie jakiejś idei lub rzeczy}
\Clue{19}{}{wagon do przewozu ludzi, wchodzący w skład niekolejowego systemu transportu, np.: wagonik tramwajowy, wagonik kolejki linowej lub górskiej}
\Clue{21}{}{Juliusz (1824-99) ojciec Wojciecha, malarz, wybitny akwarelista i ilustrator: sceny batalistyczne, historyczne, rodzajowe, mistrzowskie przedstawianie koni i jeźdźców}
\Clue{22}{}{Columbinae - podrodzina ptaków z rodziny gołębiowatych (Columbidae)}
\Clue{24}{}{miejsce służące do prezentacji koni}
\Clue{25}{}{niesteroidowy lek przeciwzapalny - lek należący do grupy niesteroidowych leków przeciwzapalnych, obejmującej leki przeciwzapalne, przeciwbólowe i przeciwgorączkowe, których działanie polega na hamowaniu cyklooksygenazy prostaglandynowej (COX)}
\Clue{27}{}{kraina historyczna (obecnie nomos) w Grecji, w środkowej, części Peloponezu, główne miasto Tripolis, obszar 4,4 tyś. km2}
\Clue{29}{}{pogląd, jaki ktoś ma o kimś lub o czymś, oparty na posiadanym na ten temat wyobrażeniu}\end{PuzzleClues}\newpage\section*{Krzyżówka 72}

\noindent\begin{Puzzle}{17}{22}|*	|*	|*	|[1][S]\drarr	|l	|a	|k	|*	|*	|[2][S]\drarr	|d	|r	|a	|p	|a	|c	|z	|*	|.
|*	|*	|[3][S]\darr	|p	|*	|*	|[4][S]\rarr	|k	|a	|p	|l	|i	|c	|z	|k	|a	|*	|*	|.
|*	|[5][S]\drarr	|t	|r	|z	|o	|n	|[][,]{ }	|m	|a	|c	|i	|c	|y	|*	|*	|*	|[6][S]\darr	|.
|*	|s	|e	|o	|*	|[7][S]\rarr	|t	|r	|z	|c	|i	|n	|n	|i	|k	|*	|*	|h	|.
|*	|a	|k	|l	|[8][S]\drarr	|n	|i	|e	|c	|z	|y	|n	|n	|o	|ś	|ć	|*	|o	|.
|*	|r	|s	|o	|ś	|[9][S]\darr	|*	|[10][S]\rarr	|q	|u	|e	|b	|e	|c	|*	|*	|[11][S]\darr	|r	|.
|*	|a	|t	|g	|w	|b	|*	|*	|*	|l	|[12][S]\darr	|*	|*	|[13][S]\darr	|*	|*	|z	|d	|.
|*	|n	|*	|*	|i	|i	|*	|[14][S]\rarr	|s	|a	|s	|*	|[15][S]\darr	|k	|[16][S]\darr	|*	|a	|a	|.
|*	|g	|[17][S]\rarr	|o	|t	|u	|n	|i	|t	|*	|t	|[18][S]\drarr	|m	|o	|m	|o	|t	|*	|.
|*	|i	|[19][S]\rarr	|s	|e	|r	|c	|ó	|w	|k	|o	|w	|a	|t	|e	|*	|o	|*	|.
|*	|*	|*	|*	|ź	|o	|*	|*	|*	|*	|p	|y	|ł	|s	|l	|*	|r	|[20][S]\darr	|.
|*	|*	|*	|*	|*	|k	|*	|*	|*	|*	|a	|r	|y	|i	|i	|*	|*	|r	|.
|[21][S]\rarr	|k	|a	|r	|b	|r	|o	|m	|a	|l	|*	|a	|w	|s	|n	|*	|*	|i	|.
|[22][S]\drarr	|m	|u	|l	|d	|a	|*	|*	|*	|*	|*	|z	|ó	|*	|a	|*	|*	|t	|.
|o	|*	|[23][S]\rarr	|a	|r	|t	|y	|k	|u	|ł	|*	|[][,]{ }	|z	|*	|*	|*	|*	|t	|.
|f	|*	|[24][S]\rarr	|d	|i	|a	|l	|e	|k	|t	|*	|z	|*	|*	|*	|*	|*	|e	|.
|i	|*	|*	|*	|*	|*	|[25][S]\drarr	|k	|y	|s	|z	|ł	|a	|k	|*	|*	|*	|r	|.
|a	|*	|[26][S]\drarr	|e	|f	|e	|k	|t	|[][,]{ }	|s	|n	|o	|b	|i	|z	|m	|u	|*	|.
|r	|*	|w	|*	|*	|*	|o	|[27][S]\rarr	|s	|m	|a	|ż	|e	|n	|i	|n	|a	|*	|.
|n	|*	|u	|[28][S]\rarr	|f	|u	|k	|s	|*	|[29][S]\rarr	|d	|o	|j	|ś	|c	|i	|e	|*	|.
|i	|*	|r	|*	|*	|*	|*	|*	|*	|[30][S]\rarr	|k	|n	|o	|t	|n	|i	|k	|*	|.
|k	|[31][S]\rarr	|m	|e	|j	|l	|*	|[32][S]\rarr	|k	|r	|z	|y	|k	|l	|i	|w	|e	|*	|.
|*	|*	|*	|*	|*	|*	|*	|*	|*	|*	|*	|*	|*	|*	|*	|*	|*	|*	|.\end{Puzzle}

\newpage

\begin{PuzzleClues}{\textbf{Poziome}\\}\Clue{1}{}{kod ISO 4217 waluty kip}
\Clue{2}{}{BERNARDYNEK jednoroczna roślina ze złożonych podobna do ostu, uprawiana dla leczniczego ziela}
\Clue{4}{}{niewielka budowla kultowa, wznoszona przy drogach lub rozdrożach w celach wotywnych, dziękczynnych, obrzędowych itp., w najprostszej formie drewniana skrzynka z obrazem lub rzeźbą zawieszona na drzewie lub słupie}
\Clue{5}{}{grubsza i szersza część macicy, wewnątrz której znajduje się jama macicy - miejsce gdzie dochodzi do zagnieżdżenia się zarodka i rozwoju płodu}
\Clue{7}{}{Wąż trzcinowy, Erpetoichthys calabaricus - gatunek słodkowodnej ryby z rodziny wielopłetwcowatych (Polypteridae), jedyny przedstawiciel rodzaju Erpetoichthys; zamieszkuje wolno płynące rzeki i wody stojące Afryki Zachodniej i Środkowej}
\Clue{8}{}{niezdolność do poprawnego funkcjonowania, niesprawność, cecha czegoś, z czego nie można skorzystać, czego nie można użyć}
\Clue{10}{}{prowincja we wsch. Kanadzie, pow. 1,5 min km2, stolica prowincji Ouebec, 6,4 min mieszkańców}
\Clue{14}{}{przedstawiciel ludu germańskiego osiadłego w średniowieczu w Westfalii i Dolnej Saksonii}
\Clue{17}{}{rzadki minerał z gromady minerałów uranylu, występuje tylko w niektórych rejonach Ziemi}
\Clue{18}{}{amerykański ptak z rodziny piłodziobów}
\Clue{19}{}{rodzina słono- i słonowowodnych małży z rzędu Veneroida, z podgromady Heterodonta}
\Clue{21}{}{organiczny związek chemiczny stosowany dawniej w leczeniu farmakologicznym jako środek uspokajający lub nasenny}
\Clue{22}{}{nierówność na powierzchni, wybój na drodze}
\Clue{23}{}{jedna sztuka towaru będącego przedmiotem handlu}
\Clue{24}{}{regionalna odmiana języka, wyróżniająca się pewnymi cechami fonetycznymi, gramatycznymi i leksykalnymi}
\Clue{25}{}{osiedle wiejskie w Azji Środkowej}
\Clue{26}{}{zjawisko ekonomiczne polegające na ograniczaniu zakupu pewnych dóbr lub całkowite zaniechanie ich nabycia, ponieważ są one chętnie nabywane przez innych}
\Clue{27}{}{potrawa z czegoś smażonego}
\Clue{28}{}{koń -  zwycięzca wyścigów, na którego nikt nie liczył}
\Clue{29}{}{uzyskać coś na drodze prawnej}
\Clue{30}{}{borześlad, Pohlia - rodzaj mchów z rodziny prątnikowatych}
\Clue{31}{}{wiadomość tekstowa otrzymana za pośrednictwem poczty elektronicznej}
\Clue{32}{}{amerykańskie ptaki z rzędu wróblowatych, odpowiedniki ptaków śpiewających Starego Świata}\end{PuzzleClues}

\begin{PuzzleClues}{\textbf{Pionowe}\\}\Clue{1}{}{w kolarstwie: jazda, która wyprzedza wieloetapową rywalizację}
\Clue{2}{}{olejek eteryczny otrzymywany z liści paczulki i wykorzystywany do produkcji perfum}
\Clue{3}{}{warstwa tekstowa czegoś, co składa się nie tylko z tekstu}
\Clue{5}{}{hinduski ludowy instrument muzyczny}
\Clue{6}{}{u Turków, Tatarów: wojsko, obóz wojskowy, orda}
\Clue{8}{}{jezioro na Białorusi, w pobliżu Nowogródka, powierzchnia 1,5 km}
\Clue{9}{}{pracownik urzędu będący formalistą, bardzo restrykcyjnie trzymający się przepisów}
\Clue{11}{}{korek na drodze; brak możliwości przejazdu}
\Clue{12}{}{część instrumentu muzycznego, mechanizm umożliwiający grę na centrali w perkusji i hi-hacie}
\Clue{13}{}{malarz (1836-77) wybitny przedstawiciel polskiego realizmu; sceny rodzajowe z życia wsi, pejzaże, portrety}
\Clue{15}{}{Niedźwiedzica Mała - gwiazdozbiór w obrębie nieba północnego}
\Clue{16}{}{podrzędny, nieprzyjemny lokal, gromadzący ludzi z marginesu społecznego}
\Clue{18}{}{wyraz złożony z innych wyrazów; termin językoznawczy}
\Clue{20}{}{geograf niemiecki (1779-1859); obok Humboldta twórca nowoczesnej geografii}
\Clue{22}{}{kapłan, który pełni posługę przy składaniu ofiar}
\Clue{25}{}{hodowany gołąb o jaskrawym upierzeniu}
\Clue{26}{}{STARNBERG}\end{PuzzleClues}\newpage\section*{Krzyżówka 73}

\noindent\begin{Puzzle}{20}{33}|*	|*	|*	|*	|*	|*	|*	|*	|*	|*	|*	|*	|*	|*	|*	|*	|*	|*	|*	|*	|[1][S]\darr	|.
|*	|*	|*	|*	|*	|*	|*	|*	|*	|*	|*	|*	|*	|*	|*	|*	|[2][S]\darr	|*	|*	|*	|r	|.
|*	|*	|*	|*	|*	|*	|*	|*	|*	|*	|*	|*	|*	|*	|*	|*	|m	|*	|*	|*	|e	|.
|*	|*	|*	|*	|*	|*	|*	|*	|*	|*	|*	|*	|*	|*	|*	|*	|i	|*	|*	|*	|d	|.
|*	|*	|*	|*	|*	|*	|*	|*	|*	|*	|*	|*	|*	|*	|*	|*	|ę	|*	|*	|*	|u	|.
|*	|*	|*	|*	|*	|*	|*	|*	|*	|*	|*	|*	|*	|*	|*	|*	|k	|*	|*	|*	|k	|.
|*	|*	|*	|*	|*	|*	|*	|*	|*	|*	|*	|*	|*	|*	|*	|*	|i	|*	|*	|*	|c	|.
|*	|*	|*	|*	|*	|*	|*	|*	|*	|*	|*	|*	|*	|*	|*	|*	|s	|*	|*	|*	|j	|.
|*	|*	|*	|*	|*	|*	|*	|*	|*	|*	|*	|*	|*	|*	|*	|*	|z	|*	|*	|*	|a	|.
|*	|*	|*	|*	|*	|*	|*	|*	|*	|*	|*	|*	|*	|*	|*	|*	|[][,]{ }	|*	|*	|*	|[][,]{ }	|.
|*	|*	|*	|*	|*	|*	|*	|*	|*	|*	|*	|*	|*	|*	|*	|*	|p	|*	|*	|*	|w	|.
|*	|*	|*	|*	|*	|*	|*	|*	|*	|*	|*	|*	|*	|*	|*	|[3][S]\darr	|o	|*	|*	|*	|y	|.
|*	|*	|*	|*	|*	|*	|*	|*	|*	|*	|*	|*	|*	|*	|*	|k	|w	|*	|*	|*	|r	|.
|*	|*	|*	|*	|*	|*	|*	|*	|*	|*	|*	|*	|*	|*	|*	|o	|i	|*	|*	|*	|a	|.
|*	|*	|*	|*	|*	|*	|*	|*	|*	|*	|*	|[4][S]\rarr	|c	|h	|o	|r	|e	|g	|*	|*	|z	|.
|*	|*	|*	|*	|*	|*	|*	|*	|*	|*	|*	|*	|[5][S]\darr	|*	|*	|a	|t	|*	|*	|*	|ó	|.
|*	|*	|*	|*	|*	|*	|*	|*	|*	|*	|*	|*	|b	|*	|*	|n	|r	|*	|*	|*	|w	|.
|*	|*	|*	|*	|*	|*	|*	|*	|*	|*	|[6][S]\rarr	|i	|r	|o	|n	|*	|z	|*	|*	|[7][S]\darr	|[][,]{ }	|.
|*	|*	|*	|*	|*	|*	|*	|*	|*	|*	|*	|*	|y	|*	|*	|*	|n	|*	|*	|h	|p	|.
|*	|*	|*	|*	|*	|*	|*	|*	|*	|*	|*	|*	|c	|*	|*	|*	|y	|*	|*	|e	|o	|.
|*	|*	|*	|*	|*	|*	|*	|*	|*	|*	|*	|*	|z	|*	|*	|*	|*	|*	|*	|l	|d	|.
|*	|*	|*	|*	|*	|*	|*	|*	|*	|*	|*	|[8][S]\darr	|k	|*	|*	|*	|*	|*	|*	|i	|o	|.
|*	|*	|*	|*	|*	|*	|*	|*	|*	|*	|*	|t	|a	|*	|*	|*	|*	|*	|*	|o	|b	|.
|*	|*	|*	|*	|[9][S]\rarr	|w	|i	|w	|e	|n	|d	|a	|*	|*	|*	|*	|*	|*	|*	|s	|n	|.
|*	|*	|*	|*	|*	|*	|*	|[10][S]\rarr	|s	|z	|y	|m	|e	|l	|*	|*	|*	|*	|*	|f	|y	|.
|*	|*	|*	|*	|*	|*	|*	|*	|*	|*	|*	|b	|*	|*	|[11][S]\darr	|*	|*	|*	|*	|e	|c	|.
|*	|*	|*	|*	|*	|*	|*	|*	|*	|*	|*	|u	|*	|*	|g	|*	|*	|*	|*	|r	|h	|.
|*	|*	|*	|*	|*	|*	|*	|*	|*	|*	|[12][S]\rarr	|r	|e	|p	|e	|t	|y	|c	|j	|a	|*	|.
|[13][S]\rarr	|p	|o	|d	|k	|ó	|w	|e	|c	|z	|k	|a	|*	|*	|n	|*	|*	|*	|*	|*	|*	|.
|*	|*	|*	|*	|*	|*	|*	|*	|*	|*	|*	|*	|*	|*	|e	|*	|*	|*	|*	|*	|*	|.
|*	|*	|*	|*	|*	|*	|*	|*	|*	|*	|*	|*	|*	|*	|r	|*	|*	|*	|*	|*	|*	|.
|*	|*	|*	|*	|*	|*	|*	|*	|*	|*	|*	|*	|*	|*	|a	|*	|*	|*	|*	|*	|*	|.
|*	|*	|*	|*	|*	|*	|*	|*	|*	|*	|*	|*	|*	|*	|ł	|*	|*	|*	|*	|*	|*	|.
|*	|*	|*	|*	|*	|*	|*	|*	|*	|*	|*	|*	|*	|*	|*	|*	|*	|*	|*	|*	|*	|.\end{Puzzle}

\newpage

\begin{PuzzleClues}{\textbf{Poziome}\\}\Clue{4}{}{wyznaczony przez archonta obywatel greckiej polis, który miał za zadanie dobrać chór mający występować w widowiskach teatralnych lub muzycznych, finansować jego przygotowanie oraz wyposażać w kostiumy}
\Clue{6}{}{rodzaj kija golfowego służącego do wbijania piłki do dołka}
\Clue{9}{}{żywność, jedzenie}
\Clue{10}{}{wzorzec, szablon formularza, według którego wykonywano powtarzające się czynności urzędowe}
\Clue{12}{}{w muzyce: szybkie powtarzanie tego samego dźwięku}
\Clue{13}{}{zdrobniale: podkówka - metalowe wzmocnienie w kształcie litery U przybijane gwoździami do kopyt końskich w celu zapobiegania ścieraniu się kopyt}\end{PuzzleClues}

\begin{PuzzleClues}{\textbf{Pionowe}\\}\Clue{1}{}{dodawanie (odejmowanie) wyrazów podobnych różniących się jedynie współczynnikiem, na przykład jednomianów, w celu uproszczenia zapisu wyrażenia}
\Clue{2}{}{jedna z odmian tkanki miękiszowej, która charakteryzuje się dużymi przestworami międzykomórkowymi, wypełnionymi powietrzem}
\Clue{3}{}{egzemplarz Koranu}
\Clue{5}{}{lekki, odkryty powóz konny mający z tyłu miękkie siedzenie a z przodu miejsce dla woźnicy}
\Clue{7}{}{obszar wokół Słońca, w którym ciśnienie wiatru słonecznego przeważa nad ciśnieniem wiatrów galaktycznych, tworząc kulę wyrzucanej przez Słońce materii w otaczającym ośrodku międzygwiazdowym}
\Clue{8}{}{instrument strunowy szarpany z pudłem rezonansowym, gryfem i progami na podstrunnicy; wykonywany jest z drewna, ma zwykle cztery struny, obecnie najczęściej metalowe}
\Clue{11}{}{oficerski stopień wyższy od pułkownika}\end{PuzzleClues}\newpage\section*{Krzyżówka 74}

\noindent\begin{Puzzle}{24}{20}|*	|*	|*	|*	|*	|*	|[1][S]\drarr	|s	|i	|ł	|a	|*	|[2][S]\drarr	|t	|e	|r	|i	|o	|d	|o	|n	|t	|y	|*	|*	|.
|*	|*	|*	|*	|*	|*	|k	|[3][S]\rarr	|n	|e	|r	|k	|a	|[][,]{ }	|w	|ę	|d	|r	|u	|j	|ą	|c	|a	|*	|*	|.
|*	|*	|*	|*	|*	|*	|o	|*	|*	|[4][S]\rarr	|i	|o	|u	|i	|o	|u	|e	|*	|*	|*	|*	|*	|*	|*	|*	|.
|*	|*	|*	|*	|[5][S]\darr	|[6][S]\rarr	|r	|o	|k	|[][,]{ }	|l	|i	|t	|u	|r	|g	|i	|c	|z	|n	|y	|*	|*	|*	|*	|.
|*	|[7][S]\darr	|*	|[8][S]\drarr	|p	|e	|d	|a	|l	|s	|t	|w	|o	|*	|*	|*	|*	|*	|[9][S]\darr	|*	|[10][S]\darr	|[11][S]\darr	|*	|*	|*	|.
|*	|r	|*	|h	|a	|*	|o	|*	|*	|[12][S]\rarr	|i	|n	|w	|a	|z	|j	|a	|*	|r	|[13][S]\darr	|g	|c	|*	|*	|*	|.
|*	|u	|*	|a	|m	|[14][S]\rarr	|b	|i	|n	|a	|*	|*	|i	|*	|*	|*	|[15][S]\darr	|*	|u	|m	|e	|z	|*	|*	|*	|.
|*	|l	|*	|k	|p	|*	|a	|*	|[16][S]\drarr	|k	|a	|s	|z	|t	|*	|[17][S]\rarr	|b	|a	|b	|i	|n	|a	|*	|*	|*	|.
|*	|i	|*	|*	|a	|*	|*	|*	|l	|*	|*	|*	|e	|*	|*	|[18][S]\darr	|l	|*	|a	|n	|e	|n	|*	|*	|*	|.
|*	|k	|*	|[19][S]\darr	|s	|[20][S]\rarr	|w	|i	|e	|l	|k	|o	|r	|u	|s	|k	|i	|*	|t	|t	|r	|d	|*	|*	|*	|.
|*	|[][,]{ }	|*	|k	|*	|*	|*	|*	|w	|[21][S]\rarr	|a	|m	|u	|r	|*	|r	|z	|*	|o	|a	|a	|i	|*	|*	|*	|.
|*	|n	|*	|l	|[22][S]\rarr	|r	|u	|m	|*	|*	|*	|*	|n	|*	|*	|y	|n	|*	|*	|j	|ł	|g	|*	|*	|*	|.
|*	|a	|*	|e	|*	|*	|[23][S]\rarr	|w	|i	|l	|c	|z	|e	|[][,]{ }	|s	|t	|a	|d	|o	|*	|[][,]{ }	|a	|*	|*	|*	|.
|*	|d	|[24][S]\drarr	|p	|r	|ę	|d	|k	|o	|ś	|ć	|*	|k	|*	|*	|y	|*	|*	|*	|*	|b	|r	|*	|*	|*	|.
|*	|r	|d	|i	|[25][S]\rarr	|f	|r	|e	|g	|a	|t	|a	|*	|*	|*	|k	|*	|*	|*	|*	|r	|h	|*	|*	|*	|.
|[26][S]\drarr	|z	|e	|s	|p	|ó	|ł	|[][,]{ }	|k	|o	|o	|r	|d	|y	|n	|a	|c	|y	|j	|n	|y	|*	|*	|*	|*	|.
|g	|e	|k	|k	|*	|*	|*	|*	|*	|*	|[27][S]\rarr	|s	|a	|l	|k	|*	|*	|*	|*	|[28][S]\rarr	|g	|ę	|s	|i	|*	|.
|a	|w	|a	|o	|*	|*	|*	|*	|*	|*	|*	|*	|*	|[29][S]\rarr	|n	|e	|k	|r	|o	|m	|a	|n	|t	|a	|*	|.
|p	|n	|d	|*	|[30][S]\rarr	|p	|o	|ż	|y	|c	|z	|k	|a	|[][,]{ }	|l	|o	|m	|b	|a	|r	|d	|o	|w	|a	|*	|.
|*	|y	|a	|*	|*	|*	|[31][S]\rarr	|g	|ł	|ó	|g	|[][,]{ }	|o	|s	|t	|r	|o	|g	|o	|w	|y	|*	|*	|*	|*	|.
|*	|*	|*	|*	|*	|*	|*	|*	|[32][S]\rarr	|a	|g	|l	|o	|m	|e	|r	|a	|c	|j	|a	|*	|*	|*	|*	|*	|.\end{Puzzle}

\newpage

\begin{PuzzleClues}{\textbf{Poziome}\\}\Clue{1}{}{walor, pozytywna strona}
\Clue{2}{}{gady ssakozębne, Theriodontia - grupa gadów z rzędu terapsydów z podgromady Synapsida; bezpośredni przodkowie ssaków, gdyż w ich budowie występuje szereg cech charakterystycznych dla ssaków}
\Clue{3}{}{przemieszczenie się nerki ku dołowi na skutek zwiotczenia tkanek}
\Clue{4}{}{miasto i port w Chile nad Oceanem Spokojnym, ośrodek administracyjny regionu Tarapaca}
\Clue{6}{}{przyjęty w danej religii cykliczny podział roku z wyróżnieniem obchodzonych dni świątecznych, obrzędów i innych, związanych z liturgią i teologią danego wyznania}
\Clue{8}{}{pogardliwie o homoseksualizmie jako zjawisku społecznym, relacjach społecznych}
\Clue{12}{}{w medycynie: zakażenie}
\Clue{14}{}{na Podkarpaciu: bułka paryska}
\Clue{16}{}{rusztowanie podtrzymujące sklepienie w kopalni}
\Clue{17}{}{staruszka - kobieta w podeszłym wieku; słowo wyrażające politowanie}
\Clue{20}{}{język rosyjski; język, którym posługiwali się mieszkańcy właściwej Rosji}
\Clue{21}{}{roślinożerna smaczna ryba z karpiowatych; długość ponad 1 m; rzeki Chin i Rosji}
\Clue{22}{}{napój alkoholowy otrzymywany przez destylację melasy z trzciny cukrowej}
\Clue{23}{}{potoczna nazwa taktyki działania okrętów podwodnych, przy zwalczaniu żeglugi morskiej przeciwnika, której podstawą cechą jest skoordynowany atak dwóch lub większej liczby okrętów podwodnych na jednostkę bądź grupę jednostek morskich przeciwnika}
\Clue{24}{}{cecha charakteru, polegająca na tym, że ktoś jest prędki, jest raptusem}
\Clue{25}{}{duży ptak oceaniczny o mocnych skrzydłach i długim rozwidlonym ogonem; zamieszkuje oceany strefy tropikalnej, gnieździ się gromadnie na wyspach}
\Clue{26}{}{grupa ludzi, którą łączą sformalizowane związki, powołana w celu koordynacji jakiegoś przedsięwzięcia}
\Clue{27}{}{amerykański lekarz bakteriolog i serolog ur. w 1914 r.; szczepionka przeciw chorobie Heinego-Medina}
\Clue{28}{}{domowe ptaki gospodarskie hodowane dla pierza, mięsa i tłuszczu; pochodzą od gęgawy i gęsi garbonosej}
\Clue{29}{}{czarnoksiężnik, który przyzywa cienie zmarłych w celu poznania prawdopodobnych wersji przyszłości lub w innych celach własnych (np. po to, by duchy były na jego usługach)}
\Clue{30}{}{pożyczka, która jest udzielana pod zastaw papierów wartościowych oraz wartościowych przedmiotów o charakterze użytkowym, jak biżuteria czy obrazy}
\Clue{31}{}{Crataegus crus-galli - gatunek rośliny z rodziny różowatych}
\Clue{32}{}{teren, na którym zaludnienie lub działalność gospodarcza są wystarczająco skoncentrowane, aby ścieki komunalne były zbierane i przekazywane do oczyszczalni ścieków komunalnych}\end{PuzzleClues}

\begin{PuzzleClues}{\textbf{Pionowe}\\}\Clue{1}{}{miasto w południowej Hiszpanii, nad rzeką Gwadalkiwir, stolica prowincji Kordoba, w regionie Andaluzja}
\Clue{2}{}{zbiór cech i wyobrażeń, które dana jednostka przypisuje samej sobie}
\Clue{5}{}{PAMPA}
\Clue{7}{}{Lycogala epidendrum - gatunek śluzowca}
\Clue{8}{}{PNIAK - w gwarze łowieckiej górny kieł jelenia}
\Clue{9}{}{określenie chwilowej zmiany tempa}
\Clue{10}{}{stopień wojskowy wprowadzony w Polsce w 1992 roku i oznaczany jedną gwiazdką na naramienniku}
\Clue{11}{}{do 1986r terytorium związkowe w Indiach, włączone do stanu Pendżab}
\Clue{13}{}{jadalna ryba z dorszowatych występująca w północnej części Pacyfiku}
\Clue{15}{}{wada odlewu w postaci długich, rozgałęzionych zagłębień o gładkich ściankach i małym spadku}
\Clue{16}{}{jeden ze znaków zodiaku (tzw. tropikalnego, astrologicznego)}
\Clue{18}{}{dział piśmiennictwa}
\Clue{19}{}{nieruchoma część młocarni, element, o który zespół młócący ociera lub uderza zboże}
\Clue{24}{}{okres 10 lat, miesięcy, tygodni lub dni}
\Clue{26}{}{miasto w płd.-wsch. Francji; 32,1 tys. mieszkańców (1982 r.)}\end{PuzzleClues}\newpage\section*{Krzyżówka 75}

\noindent\begin{Puzzle}{23}{32}|*	|*	|*	|*	|*	|[1][S]\darr	|*	|*	|*	|*	|*	|*	|*	|*	|*	|*	|*	|*	|*	|*	|[2][S]\darr	|*	|*	|*	|.
|*	|*	|[3][S]\drarr	|d	|y	|s	|k	|*	|*	|*	|*	|*	|[4][S]\rarr	|h	|e	|r	|b	|i	|c	|y	|d	|*	|[5][S]\darr	|[6][S]\darr	|.
|*	|*	|t	|*	|*	|t	|[7][S]\rarr	|l	|e	|m	|i	|n	|g	|[][,]{ }	|ś	|n	|i	|e	|ż	|n	|y	|*	|l	|ż	|.
|*	|[8][S]\drarr	|u	|d	|a	|r	|[][,]{ }	|m	|ó	|z	|g	|o	|w	|y	|*	|[9][S]\rarr	|t	|e	|r	|e	|n	|*	|u	|ą	|.
|*	|n	|j	|*	|[10][S]\darr	|o	|[11][S]\rarr	|k	|u	|l	|t	|u	|r	|a	|[][,]{ }	|j	|ę	|z	|y	|k	|a	|*	|t	|d	|.
|*	|a	|o	|*	|w	|i	|*	|*	|*	|[12][S]\darr	|*	|*	|*	|*	|*	|*	|*	|*	|*	|*	|m	|*	|n	|z	|.
|*	|d	|w	|[13][S]\rarr	|s	|k	|r	|y	|p	|t	|o	|r	|i	|u	|m	|*	|*	|*	|*	|*	|i	|*	|i	|a	|.
|*	|g	|i	|*	|p	|*	|*	|[14][S]\drarr	|m	|a	|k	|*	|*	|*	|[15][S]\drarr	|p	|i	|t	|*	|*	|z	|[16][S]\darr	|a	|[][,]{ }	|.
|*	|o	|e	|*	|ó	|*	|[17][S]\darr	|w	|*	|p	|*	|[18][S]\rarr	|k	|o	|r	|e	|k	|*	|*	|*	|m	|m	|r	|s	|.
|*	|r	|c	|[19][S]\rarr	|ł	|ą	|c	|z	|n	|i	|k	|[][,]{ }	|c	|i	|e	|c	|z	|o	|w	|y	|*	|i	|z	|u	|.
|*	|l	|[][,]{ }	|[20][S]\darr	|c	|*	|h	|n	|[21][S]\rarr	|r	|e	|g	|u	|l	|a	|c	|j	|a	|[][,]{ }	|c	|e	|n	|*	|k	|.
|*	|i	|t	|p	|z	|*	|a	|o	|[22][S]\darr	|*	|*	|[23][S]\rarr	|m	|i	|l	|a	|z	|z	|o	|*	|*	|u	|*	|c	|.
|*	|w	|a	|o	|y	|*	|l	|w	|n	|*	|[24][S]\rarr	|o	|k	|u	|p	|a	|c	|j	|a	|*	|*	|t	|*	|e	|.
|*	|o	|m	|c	|n	|[25][S]\darr	|k	|i	|a	|*	|*	|[26][S]\darr	|[27][S]\drarr	|f	|o	|ł	|t	|y	|n	|*	|[28][S]\darr	|a	|*	|s	|.
|*	|ś	|a	|i	|n	|m	|o	|e	|r	|[29][S]\darr	|*	|k	|m	|[30][S]\drarr	|l	|a	|w	|e	|n	|d	|a	|*	|[31][S]\darr	|u	|.
|*	|ć	|r	|ą	|i	|a	|c	|n	|a	|o	|[32][S]\drarr	|r	|u	|p	|i	|e	|c	|i	|a	|r	|n	|i	|a	|*	|.
|*	|*	|y	|g	|k	|g	|h	|i	|m	|t	|g	|z	|s	|o	|t	|*	|[33][S]\darr	|*	|*	|*	|t	|*	|m	|*	|.
|*	|*	|s	|[][,]{ }	|*	|n	|i	|e	|i	|a	|r	|y	|t	|m	|i	|*	|m	|*	|*	|[34][S]\darr	|r	|*	|i	|*	|.
|*	|*	|z	|r	|*	|e	|d	|*	|e	|r	|a	|c	|y	|p	|k	|[35][S]\darr	|u	|[36][S]\darr	|*	|t	|a	|*	|o	|*	|.
|*	|[37][S]\drarr	|k	|o	|c	|z	|o	|w	|n	|i	|c	|z	|k	|a	|*	|z	|l	|p	|*	|u	|c	|*	|k	|*	|.
|*	|p	|o	|b	|*	|j	|n	|[38][S]\darr	|n	|a	|k	|e	|*	|[][,]{ }	|*	|g	|d	|a	|[39][S]\darr	|r	|y	|*	|s	|*	|.
|*	|r	|w	|o	|*	|a	|*	|d	|i	|*	|a	|k	|*	|p	|*	|r	|y	|t	|k	|i	|k	|*	|z	|*	|.
|*	|ę	|y	|c	|*	|*	|[40][S]\rarr	|a	|k	|t	|*	|*	|*	|u	|*	|z	|[][,]{ }	|e	|u	|e	|l	|*	|t	|*	|.
|*	|c	|*	|z	|*	|*	|*	|r	|*	|*	|[41][S]\darr	|*	|*	|s	|[42][S]\darr	|e	|p	|n	|ł	|c	|i	|*	|a	|*	|.
|*	|i	|[43][S]\darr	|y	|*	|*	|*	|ń	|*	|*	|v	|[44][S]\rarr	|s	|z	|a	|b	|o	|t	|a	|*	|n	|*	|ł	|*	|.
|*	|k	|d	|*	|*	|*	|*	|*	|*	|*	|i	|*	|*	|k	|r	|ł	|d	|*	|n	|*	|a	|*	|t	|*	|.
|*	|*	|z	|[45][S]\rarr	|d	|ł	|u	|g	|o	|o	|d	|w	|ł	|o	|k	|o	|w	|e	|*	|*	|*	|[46][S]\darr	|n	|*	|.
|[47][S]\rarr	|d	|i	|e	|s	|e	|l	|*	|*	|*	|i	|*	|*	|w	|a	|*	|ó	|*	|*	|*	|*	|h	|e	|*	|.
|[48][S]\drarr	|b	|e	|n	|e	|f	|i	|c	|j	|e	|n	|t	|k	|a	|*	|*	|j	|*	|*	|*	|*	|a	|*	|*	|.
|r	|[49][S]\rarr	|k	|u	|b	|e	|c	|z	|e	|k	|*	|*	|*	|*	|*	|[50][S]\rarr	|n	|a	|c	|i	|o	|s	|*	|*	|.
|o	|[51][S]\rarr	|a	|r	|c	|h	|i	|d	|i	|a	|k	|o	|n	|i	|a	|*	|e	|*	|*	|*	|*	|ł	|*	|*	|.
|k	|[52][S]\rarr	|n	|i	|e	|o	|p	|a	|t	|r	|z	|n	|o	|ś	|ć	|*	|*	|*	|*	|*	|*	|o	|*	|*	|.
|*	|*	|*	|*	|*	|*	|*	|*	|*	|*	|*	|*	|*	|*	|*	|*	|*	|*	|*	|*	|*	|*	|*	|*	|.\end{Puzzle}

\newpage

\begin{PuzzleClues}{\textbf{Poziome}\\}\Clue{3}{}{sprzęt lekkoatletyczny używany podczas rzutu dyskiem}
\Clue{4}{}{środek chwastobójczy - pestycyd służący do selektywnego lub nieselektywnego zwalczania chwastów w uprawach}
\Clue{7}{}{leming obrożny, Dicrostonyx torquatus - gatunek gryzonia z podrodziny nornikowatych w rodzinie chomikowatych, występujący w strefie tundry od Morza Białego na zachodzie po wybrzeża Pacyfiku na wschodzie, także na wyspach u wybrzeży Syberii}
\Clue{8}{}{zespół objawów klinicznych związanych z nagłym wystąpieniem ogniskowego lub uogólnionego zaburzenia czynności mózgu, powstały w wyniku zaburzenia krążenia mózgowego i utrzymujący się ponad 24 godziny}
\Clue{9}{}{obszar podległy komuś, czemuś, objęty czyimś działaniem (np. pracować w terenie)}
\Clue{11}{}{dział językoznawstwa zajmujący się zachowaniami językowymi i ich odniesieniem do normy językowej}
\Clue{13}{}{pomieszczenie, w którym przepisywano księgi}
\Clue{14}{}{jadalne nasiona rośliny o tej samej nazwie}
\Clue{15}{}{druk urzędowy udostępniany w Polsce przez Ministerstwo Finansów, na którym podatnik składa roczną deklarację rozliczeniową dotyczącą podatku}
\Clue{18}{}{but, na podeszwie którego znajdują się nieostre, zazwyczaj gumowe kolce, używany do gry w piłkę nożną}
\Clue{19}{}{łącznik, w którym do gaszenia łuku zastosowano ciecz}
\Clue{21}{}{ściśle określone działania rządu państwa, które polega na bezpośrednim ustalaniu cen lub na ustalaniu szczegółowych zasad kalkulacji cen przez dane podmioty gospodarcze, które są kontrolowania z tych zasad}
\Clue{23}{}{miasto i port we Włoszech (Sycylia) nad Morzem Tyrreńskim}
\Clue{24}{}{czasowe zajęcie przez siły zbrojne części lub całości terytorium innego państwa}
\Clue{27}{}{ur. w 1925 r., śpiewaczka (sopran); reżyser spektakli operowych}
\Clue{30}{}{śródziemnomorska krzewinka z wargowatych, uprawiana dla kwiatów, z których otrzymuje się aromatyczny olejek}
\Clue{32}{}{skład starych, niepotrzebnych rzeczy}
\Clue{37}{}{Tapinoma - rodzaj owada należący do rodziny mrówkowatych}
\Clue{40}{}{przedstawienie nagiej postaci ludzkiej}
\Clue{44}{}{podstawa, na której umocowane jest kowadło młota maszynowego}
\Clue{45}{}{Macrura - podrząd skorupiaków z rzędu dziesięcionogów}
\Clue{47}{}{wysokoprężny spalinowy silnik tłokowy}
\Clue{48}{}{kobieta korzystająca z przywilejów i korzyści}
\Clue{49}{}{zdrobniale: kubek - tyle, ile się mieści w kubku, pojemniku służącym do pakowania produktów spożywczych; zawartość kubeczka}
\Clue{50}{}{znak na drzewie powstały przez naciosanie, nacięcie}
\Clue{51}{}{dawna jednostka terytorialna w Kościele, część diecezji}
\Clue{52}{}{cecha człowieka, który zachowuje sie nieosrożnie, w sposób nieprzemyślany}\end{PuzzleClues}

\begin{PuzzleClues}{\textbf{Pionowe}\\}\Clue{1}{}{urządzenie muzyczne, pomocne przy strojeniu np. gitary}
\Clue{2}{}{w filozofii: każdy pogląd traktujący ruch, rozwój, zmienność lub stawanie się, w przeciwstawieniu do statyczności, jako pierwotny i podstawowy czynnik bytu}
\Clue{3}{}{tujowiec tamaryszkowaty, Thuidium tamariscinum, Thuidium tamariscifolium - gatunek mchu należący do rodziny tujowcowatych}
\Clue{5}{}{lutnik, rzemieślnik, który wykonuje lub naprawia szyjkowe instrumenty strunowe}
\Clue{6}{}{intensywna chęć osiągnięcia powodzenia, bez względu na koszty i warunki}
\Clue{8}{}{cecha człowieka, który nadmiernie się angażuje, jest nadmiernie gorliwy, ze zbytnią obowiązkowością coś robi, czemuś się oddaje}
\Clue{10}{}{termin nauk przyrodniczych, rozszerzony na inne nauki oraz przejęty przez słownictwo ogólne; miara czegoś wyrażona w sposób ilościowy, obliczana wzorem matematycznym}
\Clue{12}{}{łowny ssak nieparzystokopytny z niewielka trąbka np. anta}
\Clue{14}{}{fakt, że coś się ponownie uruchomiło, zaczęło funkcjonować}
\Clue{15}{}{Polityka uzwględniająca rzeczywisty stosunek sił działających podmiotów}
\Clue{16}{}{jednostka kąta płaskiego}
\Clue{17}{}{spiżowy pancerz grecki odpowiadający kształtowi tułowia człowieka, od pasa w dół zaopatrzony w pionowe skórzane pasy nabijane metalowymi płytkami, tworzącymi formę spódniczki}
\Clue{20}{}{pociąg służący do przewozu materiałów, sprzętu i robotników w do wykonywania robót na szlaku}
\Clue{22}{}{element odzieży (zwłaszcza mundurów), który stanowi pasek materiału wszyty z jednej strony między górny koniec rękawa kurtki mundurowej a samą kurtkę, a drugim końcem przypinany, najczęściej guzikiem, w pobliżu kołnierza, tak że nakrywa ramię wzdłuż obojczyka osoby ubranej w tę kurtkę}
\Clue{25}{}{proszek składający się z pyłu aluminiowego lub magnezowego i substancji utleniającej (na przykład chloranu potasu), podczas spalania której uzyskuje się bardzo jasny błysk białego światła}
\Clue{26}{}{ptak z rodziny jemiołuszek; Afryka, Azja, Indie}
\Clue{27}{}{owad z rodziny meszkowatych}
\Clue{28}{}{rodzaj antybiotyków stosowany jako lek przeciwnowotworowy}
\Clue{29}{}{uchatka patagońska; amerykański ssak z rodziny uchatek}
\Clue{30}{}{pompa wyporowa o cylindrycznym wirniku i elastycznym tłoku stosowana do pompowania niebezpiecznych dla człowieka cieczy}
\Clue{31}{}{Amiiformes - rząd słodkowodnych, drapieżnych ryb promieniopłetwych (Actinopterygii) obejmujący gatunki o prymitywnych cechach budowy; amiokształtne wraz z niszczukokształtnymi są zaliczane do przejściowców (Holostei), jedynym gatunkiem amiokształtnych, który przetrwał do czasów współczesnych jest miękławka (Amia calva)}
\Clue{32}{}{narzędzie rolnicze, składające się z metalowych zębów przytwierdzonych do drewnianego styliska}
\Clue{33}{}{konkurencja narciarska, w której dwójka zawodników zjeżdża na trasach równoległych po stoku 250m, pokrytym muldami}
\Clue{34}{}{rzeka w zachodniej Słowacji, dopływ Wagu}
\Clue{35}{}{druciana szczotka przeznaczona do rozczesywania i prostowania włókien przed przędzeniem}
\Clue{36}{}{ironicznie o tym, że ktoś ma wyłączność na coś}
\Clue{37}{}{światłoczuły receptor siatkówki oka; pręcik umożliwia widzenie czarno-białe przy słabym oświetleniu}
\Clue{38}{}{zwarta okrywa roślinna łąk i pastwisk}
\Clue{39}{}{ssak z rodziny koniowatych, zamieszkuje pustynne obszary środkowej Azji}
\Clue{41}{}{miasto w Rumunii nad Dunajem}
\Clue{42}{}{biblijna łódź, w której Noe uratował się przez potopem}
\Clue{43}{}{dyplomata na czele korpusu dyplomatycznego}
\Clue{46}{}{kod, zabezpieczenie przed czymś}
\Clue{48}{}{grupa ludzi, którzy studiują razem kolejne lata na uczelni}\end{PuzzleClues}\newpage\section*{Krzyżówka 76}

\noindent\begin{Puzzle}{18}{26}|*	|*	|*	|[1][S]\drarr	|z	|n	|a	|m	|i	|ę	|[][,]{ }	|b	|e	|c	|k	|e	|r	|a	|*	|.
|*	|*	|[2][S]\rarr	|k	|v	|a	|r	|n	|e	|r	|*	|*	|*	|[3][S]\darr	|[4][S]\darr	|*	|[5][S]\darr	|[6][S]\darr	|*	|.
|*	|*	|[7][S]\drarr	|i	|r	|b	|i	|t	|*	|*	|*	|*	|*	|k	|n	|*	|s	|b	|*	|.
|*	|[8][S]\rarr	|p	|r	|o	|k	|o	|f	|j	|e	|w	|*	|*	|i	|a	|*	|z	|a	|[9][S]\darr	|.
|[10][S]\drarr	|b	|r	|y	|g	|a	|n	|t	|y	|n	|a	|*	|*	|ś	|w	|*	|p	|s	|v	|.
|e	|*	|z	|s	|*	|*	|*	|*	|*	|*	|*	|*	|*	|ć	|a	|[11][S]\darr	|i	|z	|i	|.
|l	|*	|e	|*	|*	|*	|[12][S]\drarr	|o	|d	|e	|z	|w	|a	|*	|l	|ż	|n	|t	|l	|.
|e	|[13][S]\rarr	|c	|*	|*	|*	|v	|*	|*	|*	|*	|*	|[14][S]\darr	|*	|a	|u	|e	|a	|l	|.
|g	|*	|i	|[15][S]\rarr	|t	|u	|r	|e	|c	|k	|i	|*	|m	|*	|n	|r	|t	|[][,]{ }	|a	|.
|i	|[16][S]\rarr	|w	|o	|d	|z	|i	|s	|ł	|a	|w	|i	|a	|n	|k	|a	|*	|ł	|g	|.
|a	|[17][S]\rarr	|m	|a	|k	|l	|e	|r	|*	|*	|*	|*	|l	|*	|a	|w	|*	|u	|r	|.
|*	|*	|a	|[18][S]\drarr	|b	|a	|s	|i	|s	|t	|a	|*	|i	|[19][S]\darr	|*	|*	|[20][S]\darr	|p	|a	|.
|[21][S]\rarr	|g	|l	|e	|n	|n	|*	|*	|*	|*	|*	|*	|n	|ś	|*	|*	|f	|i	|n	|.
|[22][S]\drarr	|b	|a	|l	|s	|a	|*	|[23][S]\darr	|*	|[24][S]\darr	|[25][S]\rarr	|s	|e	|l	|e	|n	|i	|n	|*	|.
|p	|[26][S]\darr	|r	|o	|*	|*	|*	|b	|*	|s	|*	|[27][S]\darr	|s	|u	|*	|*	|l	|o	|*	|.
|a	|p	|y	|p	|[28][S]\drarr	|o	|o	|l	|i	|t	|y	|t	|*	|b	|*	|[29][S]\darr	|c	|w	|*	|.
|s	|l	|k	|s	|m	|[30][S]\rarr	|m	|o	|d	|r	|z	|e	|w	|n	|i	|k	|*	|a	|*	|.
|*	|a	|*	|o	|o	|*	|*	|k	|*	|ą	|*	|z	|*	|a	|*	|b	|*	|*	|*	|.
|*	|c	|*	|p	|c	|*	|*	|*	|*	|t	|[31][S]\darr	|a	|*	|*	|*	|*	|[32][S]\darr	|*	|*	|.
|[33][S]\drarr	|o	|p	|o	|z	|y	|c	|j	|a	|*	|s	|*	|*	|*	|*	|*	|s	|*	|*	|.
|k	|w	|*	|d	|n	|*	|[34][S]\rarr	|n	|a	|j	|e	|b	|k	|a	|*	|*	|e	|*	|*	|.
|a	|y	|*	|o	|i	|*	|*	|[35][S]\rarr	|k	|u	|c	|[][,]{ }	|m	|e	|r	|e	|n	|s	|*	|.
|l	|*	|[36][S]\rarr	|b	|k	|*	|[37][S]\rarr	|k	|l	|u	|c	|z	|*	|*	|*	|*	|s	|*	|*	|.
|e	|*	|*	|n	|*	|*	|*	|*	|*	|[38][S]\rarr	|o	|d	|c	|z	|u	|c	|i	|e	|*	|.
|b	|[39][S]\rarr	|s	|e	|k	|a	|t	|u	|r	|a	|*	|[40][S]\rarr	|j	|o	|g	|i	|n	|i	|*	|.
|*	|*	|*	|*	|*	|[41][S]\rarr	|z	|a	|p	|o	|ż	|y	|c	|z	|e	|n	|i	|e	|*	|.
|*	|*	|*	|*	|*	|*	|*	|*	|*	|*	|*	|*	|*	|*	|*	|*	|*	|*	|*	|.\end{Puzzle}

\newpage

\begin{PuzzleClues}{\textbf{Poziome}\\}\Clue{1}{}{wrodzona lub pojawiająca się we wczesnym dzieciństwie zmiana skórna występująca częściej u mężczyzn, mająca postać brązowej plamy z nieregularnymi brzegami, niekiedy pokrytej ciemnymi szorstkimi włosami}
\Clue{2}{}{zatoka Morza Adriatyckiego między Półwyspom Istria a wyspą Cies u wybrzeży Chorwacji}
\Clue{7}{}{miasto w azjatyckiej części Federacji Rosyjskiej na płn.-wsch. od Swierdłowska}
\Clue{8}{}{rosyjski kompozytor i pianista (1891-1953); symfonie, opery, 'Wojna i pokój', balety 'Romeo i Julia', kantaty, koncerty, utwory fortepianowe}
\Clue{10}{}{dwumasztowy żaglowiec handlowy o przednim rejowym ożaglowaniu; szkunerbryk}
\Clue{12}{}{apel skierowany do społeczeństwa (najczęściej do narodu) o włączenie się w jakąś aktywność}
\Clue{13}{}{symbol stałej fizycznej prędkości światła w próżni}
\Clue{15}{}{język aglutynacyjny należący do grupy oguzyjskiej języków tureckich, używany przede wszystkim w Turcji}
\Clue{16}{}{mieszkanka Wodzisławia}
\Clue{17}{}{człowiek, który zawodowo pośredniczy w transakcjach giełdowych}
\Clue{18}{}{gitarzysta, który gra na basie (gitarze basowej)}
\Clue{21}{}{John, ur. w 1921 r. astronauta amerykański, odbył jako pierwszy lot na Mercury 6}
\Clue{22}{}{dawna tratwa używana przez Inków}
\Clue{25}{}{sól kwasu selenowego(IV)}
\Clue{28}{}{oolit,  ikrowiec - skała osadowa zbudowana w przeważającej większości z ooidów; tworzy się w płytkich morzach w strefie przybrzeżnej}
\Clue{30}{}{Fomitopsis - rodzaj grzybów z rodziny pniarkowatych; grzyb nadrzewny o owocnikach wieloletnich, zdrewniałych, zazwyczaj dużych i kopytkowatego kształtu}
\Clue{33}{}{w językoznawstwie - zestawienie przeciwstawnych sobie elementów języka, np. głoski dźwięcznej i bezdźwięcznej}
\Clue{34}{}{alkohol, trunek, którym można sie najebać}
\Clue{35}{}{rasa konia z grupy kuców pochodząca z Pirenejów francuskich; początkowo użytkowany jako zwierzę juczne, stał się później koniem rolników, używanym do uprawy wysoko położonych i silnie nachylonych górskich pól, na których traktory byłyby bezużyteczne}
\Clue{36}{}{w chemii: symbol berkelu}
\Clue{37}{}{część eskadry, najmniejszy zespół taktyczny samolotów w powietrzu}
\Clue{38}{}{reakcja psychiczna, intelektualna na dany bodziec}
\Clue{39}{}{dręczenie i prześladowanie kogoś}
\Clue{40}{}{adeptka systemu rozwoju duchowego w hinduizmie i buddyzmie}
\Clue{41}{}{wyraz, związek wyrazowy lub struktura składniowa przejęte z innego języka lub wzorowane na nim}\end{PuzzleClues}

\begin{PuzzleClues}{\textbf{Pionowe}\\}\Clue{1}{}{zbroja, której korpus składająca się z dwóch części: ochraniającego pierś napierśnika i osłaniającego plecy naplecznika, połączonych za pomocą rzemieni w pasie i na ramionach}
\Clue{3}{}{łow. zakończenie ogona żubra}
\Clue{4}{}{film fabularny, w którym duża część akcji związana jest z przemocą, bijatykami, strzelaninami}
\Clue{5}{}{instrument muzyczny, odmiana klawesynu}
\Clue{6}{}{konstrukcja obronna wysunięta poza lico muru i otwarta do wnętrza miasta}
\Clue{7}{}{lek o działaniu przeciwmalarycznym}
\Clue{9}{}{architekt meksykański ur 1901 r, inicjator funkcjonalizmu w architekturze meksykańskiej}
\Clue{10}{}{w tradycji literackiej od XVI w.: utwór liryczny, zwykle o charakterze żałobnym, którego emocjonalną dominantę określają zazwyczaj uczucia smutku, żalu, doświadczenie straty, postawa skargi}
\Clue{11}{}{GRUS; gwiazdozbiór nieba południowego}
\Clue{12}{}{holenderski botanik i genetyk (1848-1935); twórca teorii zmienności mutacyjnej organizmu}
\Clue{14}{}{rodzaj bardzo delikatnej koronki klockowej wykonanej z białej nici lnianej na tiulowym tle, o wzorze wici kwiatowych; produkowane w XVII/XVIII w}
\Clue{18}{}{Elopomorpha - nadrząd ryb promieniopłetwych (Actinopterygii) z infragromady doskonałokostnych (Teleostei), obejmujący rzędy ryb charakteryzujących się larwą typu leptocefala oraz wydłużonym ciałem osobników dorosłych; są to głównie ryby morskie, w tym głębinowe, lub wpływające do estuariów}
\Clue{19}{}{kobieta będąca w związku małżeńskim; słowo żartobliwe}
\Clue{20}{}{wyrób włókienniczy otrzymywany przez spilśnianie}
\Clue{22}{}{(podsadzkowy) murek ze skały płonnej ułożony wzdłuż lub w poprzek wyrobiska jako element podsadzki}
\Clue{23}{}{w siatkówce: gra obronna przy siatce, blokada, bloking}
\Clue{24}{}{nierozpuszczalna substancja wydzielona z roztworu}
\Clue{26}{}{pracownik, który pełni funkcję strażnika i zarządcy otwartego magazynu, składu, placu służącego jakiemuś celowi w danej organizacji, firmie}
\Clue{27}{}{przesłanie - zasadnicza idea, główna myśl zawarta w jakimś tekście kultury bądź czyjejś wypowiedzi}
\Clue{28}{}{organiczny związek chemiczny, diamid kwasu węglowego}
\Clue{29}{}{kilobajt - jednostka używana w informatyce do określenia ilości informacji lub wielkości pamięci}
\Clue{31}{}{określenie wykonawcze: sucho, oschle}
\Clue{32}{}{Nestor piłkarz argentyński, pomocnik Parmy}
\Clue{33}{}{ur. 1905r, pisarz chorwacki, nowele i powieści z życia dalmatyńskiej wsi; „Przydrożny pył”}\end{PuzzleClues}\newpage\section*{Krzyżówka 77}

\noindent\begin{Puzzle}{22}{27}|*	|[1][S]\darr	|*	|*	|*	|*	|*	|*	|*	|*	|*	|*	|*	|*	|*	|*	|*	|[2][S]\darr	|*	|*	|*	|[3][S]\darr	|*	|.
|[4][S]\rarr	|m	|i	|s	|t	|y	|k	|a	|*	|*	|*	|*	|*	|*	|*	|*	|[5][S]\darr	|f	|*	|*	|*	|b	|*	|.
|*	|a	|*	|[6][S]\darr	|[7][S]\darr	|*	|*	|*	|*	|*	|*	|*	|*	|*	|*	|*	|g	|a	|[8][S]\darr	|*	|*	|a	|*	|.
|[9][S]\drarr	|z	|n	|a	|m	|i	|e	|n	|i	|t	|o	|ś	|ć	|*	|[10][S]\darr	|*	|l	|k	|t	|*	|*	|ń	|*	|.
|l	|d	|[11][S]\rarr	|p	|r	|ą	|t	|n	|i	|k	|[][,]{ }	|j	|a	|j	|o	|w	|a	|t	|y	|*	|*	|k	|[12][S]\darr	|.
|e	|a	|[13][S]\darr	|a	|a	|*	|*	|*	|*	|*	|[14][S]\darr	|*	|*	|*	|ś	|[15][S]\darr	|z	|o	|t	|*	|*	|a	|c	|.
|c	|*	|o	|r	|ź	|*	|[16][S]\darr	|*	|[17][S]\darr	|*	|w	|*	|[18][S]\darr	|*	|w	|w	|u	|r	|a	|*	|*	|[][,]{ }	|w	|.
|z	|[19][S]\darr	|d	|a	|n	|[20][S]\rarr	|s	|k	|a	|l	|a	|*	|k	|*	|i	|i	|r	|*	|n	|*	|*	|i	|i	|.
|o	|a	|p	|t	|i	|*	|y	|[21][S]\darr	|n	|*	|ń	|*	|a	|*	|ę	|t	|a	|[22][S]\darr	|i	|*	|[23][S]\darr	|n	|b	|.
|*	|r	|ł	|*	|c	|[24][S]\rarr	|p	|l	|a	|s	|t	|e	|r	|*	|c	|k	|*	|u	|a	|*	|k	|t	|a	|.
|[25][S]\drarr	|c	|y	|n	|a	|*	|i	|u	|k	|[26][S]\darr	|u	|*	|t	|*	|i	|i	|*	|z	|*	|*	|w	|e	|k	|.
|f	|h	|w	|*	|*	|[27][S]\darr	|a	|d	|o	|m	|c	|*	|a	|[28][S]\rarr	|m	|e	|n	|e	|d	|ż	|e	|r	|*	|.
|l	|i	|*	|*	|*	|b	|l	|w	|l	|i	|h	|[29][S]\darr	|*	|*	|i	|w	|[30][S]\darr	|w	|[31][S]\darr	|[32][S]\darr	|s	|n	|*	|.
|i	|t	|*	|*	|*	|a	|n	|i	|u	|n	|*	|h	|*	|*	|a	|i	|z	|n	|a	|m	|t	|e	|*	|.
|n	|e	|[33][S]\darr	|*	|[34][S]\rarr	|w	|i	|k	|t	|o	|r	|i	|i	|*	|n	|c	|l	|ę	|u	|a	|u	|t	|*	|.
|t	|k	|k	|*	|*	|ó	|a	|*	|*	|r	|*	|s	|[35][S]\darr	|*	|i	|z	|e	|t	|t	|k	|r	|o	|*	|.
|p	|t	|o	|*	|*	|ł	|*	|*	|*	|*	|*	|z	|f	|*	|n	|*	|p	|r	|o	|i	|a	|w	|*	|.
|a	|[][,]{ }	|s	|*	|*	|*	|*	|*	|[36][S]\rarr	|w	|a	|p	|e	|r	|*	|*	|n	|z	|p	|a	|*	|a	|*	|.
|s	|w	|o	|*	|*	|[37][S]\rarr	|s	|ą	|d	|[][,]{ }	|g	|a	|r	|n	|i	|z	|o	|n	|o	|w	|y	|*	|*	|.
|*	|n	|d	|*	|*	|[38][S]\rarr	|k	|i	|e	|r	|u	|n	|e	|k	|*	|*	|ś	|i	|r	|e	|*	|*	|*	|.
|*	|ę	|r	|[39][S]\darr	|*	|[40][S]\rarr	|n	|i	|a	|u	|x	|*	|z	|*	|*	|[41][S]\darr	|ć	|e	|t	|l	|*	|*	|*	|.
|*	|t	|z	|l	|[42][S]\drarr	|k	|o	|l	|e	|ń	|[][,]{ }	|u	|j	|a	|t	|o	|*	|n	|r	|i	|*	|*	|*	|.
|*	|r	|e	|a	|b	|[43][S]\rarr	|c	|z	|e	|r	|e	|d	|a	|*	|*	|s	|*	|i	|e	|s	|*	|*	|*	|.
|[44][S]\drarr	|z	|w	|r	|o	|t	|n	|i	|c	|o	|w	|y	|*	|*	|*	|t	|*	|e	|t	|t	|*	|*	|*	|.
|f	|*	|*	|g	|n	|*	|*	|*	|*	|*	|[45][S]\rarr	|h	|a	|z	|a	|r	|d	|*	|*	|a	|*	|*	|*	|.
|b	|[46][S]\rarr	|p	|o	|d	|n	|i	|e	|s	|i	|o	|n	|e	|[][,]{ }	|c	|z	|o	|ł	|o	|*	|*	|*	|*	|.
|*	|*	|*	|*	|*	|*	|*	|*	|*	|[47][S]\rarr	|n	|i	|k	|c	|z	|e	|m	|n	|o	|ś	|ć	|*	|*	|.
|*	|*	|*	|*	|*	|[48][S]\rarr	|j	|a	|r	|z	|y	|n	|i	|a	|k	|*	|*	|*	|*	|*	|*	|*	|*	|.\end{Puzzle}

\newpage

\begin{PuzzleClues}{\textbf{Poziome}\\}\Clue{4}{}{religijność oparta na mistycyzmie, przeżycia mistyczne}
\Clue{9}{}{ktoś znakomity, znamienity, cechujący się znakomitością, niepreciętny, wybitny}
\Clue{11}{}{Bryum subneodamense - gatunek mchu z rodziny prątnikowatych; mech objęty w Polsce ścisłą ochroną gatunkową}
\Clue{20}{}{układ dźwięków uszeregowanych według wysokości o ustalonych odległościach między poszczególnymi stopniami}
\Clue{24}{}{stosunkowo płaski kawałek czegoś}
\Clue{25}{}{Sn - pierwiastek chemiczny, metal z bloku p w układzie okresowym}
\Clue{28}{}{program komputerowy do zarządzania informacjami}
\Clue{34}{}{UKEREWE; największe jezioro w Afryce, powierzchnia 68 tyś. km2, głębokość do 80 m, do jeziora uchodzi Kagera}
\Clue{36}{}{człowiek, który wapuje, chmurzy - używa e-papierosów (zazwyczaj także identyfikuje się ze społecznością wapujących, uważa chmurzenie za istotną część swojego trybu życia, a e-papierosy za swoje zainteresowanie; waperzy mają cechy subkultury)}
\Clue{37}{}{sąd wojskowy orzekający w pierwszej instancji we wszystkich sprawach, z wyjątkiem spraw przekazanych ustawą do właściwości innego sądu}
\Clue{38}{}{strona, w którą ktoś lub coś się zwraca albo porusza}
\Clue{40}{}{miasto w płd. Francji w departamencie Ariege, w pobliżu znana jaskinia z okresu górnego paleolitu}
\Clue{42}{}{Centrophorus uyato - gatunek rekina z rodziny Centrophoridae}
\Clue{43}{}{grupa ludzi, często w odniesieniu do dzieci}
\Clue{44}{}{osoba oodpowiedzialna za obsługę zwrotnic kolejowych}
\Clue{45}{}{gry pieniężne, w których o wygranej w mniejszym lub większym stopniu decyduje przypadek}
\Clue{46}{}{duma, poczucie własnej wartości}
\Clue{47}{}{podły, nikczemny czyn}
\Clue{48}{}{mały nóż do obierania i krojenia warzyw}\end{PuzzleClues}

\begin{PuzzleClues}{\textbf{Pionowe}\\}\Clue{1}{}{samochód marki Mazda}
\Clue{2}{}{podmiot działający (czyniący) za pośrednika}
\Clue{3}{}{okres euforii na giełdach, związany ze spółkami z branży informatycznej i z pokrewnych sektorów}
\Clue{5}{}{cienka powłoka ze szkliwa nakładana na wyroby ceramiczne}
\Clue{6}{}{zespół organów wyspecjalizowany w jakiejś funkcji}
\Clue{7}{}{ogrodzenie stawiane do dojenia owiec}
\Clue{8}{}{Titania, jeden z satelitów planety Uran}
\Clue{9}{}{węgierska potrawa z mięsa i papryki, pomidorów itp}
\Clue{10}{}{mieszkaniec Oświęcimia}
\Clue{12}{}{placek z ciasta biszkoptowego przekładanego bakaliami}
\Clue{13}{}{ilość wody odpływająca z dorzecza, obniżenie się poziomu wód morskich}
\Clue{14}{}{grube płótno konopne, z którego wykonywano worki}
\Clue{15}{}{REPUSOWANIE}
\Clue{16}{}{ŁOŻNICA}
\Clue{17}{}{zniekształcenie konstrukcji składniowej}
\Clue{18}{}{rodzaj formularza w pierwotnej formie karty papieru, na którym klient (petent) wpisuje potrzebne informacje}
\Clue{19}{}{osoba zaowodowo zajmująca się architekturą wnętrz - projektowaniem wystroju i wyposażenia wnętrz}
\Clue{21}{}{złota moneta francuska}
\Clue{22}{}{dostrzegalny znak, symptom podlegający percepcji}
\Clue{23}{}{w starożytności: rzymska administracja skarbu}
\Clue{25}{}{pas do noszenia przez ramię długiej broni strzeleckiej}
\Clue{26}{}{MOLL}
\Clue{27}{}{przodek bydła domowego, ssak z rodziny krętorogich}
\Clue{29}{}{w gwarze młodzieżowej, uczniowskiej: język hiszpański - przedmiot szkolny lub uczony w ramach kursu, na którym opanowuje się podstawy języka hiszpańskiego}
\Clue{30}{}{zdolność krwinek i płytek krwi do łączenia się, zlepiania się ze sobą}
\Clue{31}{}{portret artysty wykonany przez niego samego; może nim być obraz, rzeźba, fotografia}
\Clue{32}{}{człowiek cyniczny, nie cofający się przed żadnym działaniem, dzięki któremu może dojść do celu, nawet idącpo trupach}
\Clue{33}{}{zwyczajowa nazwa sosny górskiej}
\Clue{35}{}{obszerny płaszcz męski, sięgający łydek, wycięty w pasie noszony w Polsce w XVI-XVII w}
\Clue{39}{}{bardzo wolne tempo}
\Clue{41}{}{ostra krawędź tnąca}
\Clue{42}{}{astronom amerykański (1825-65); odkrył księżyc Saturna Hiperion}
\Clue{44}{}{Facebook - popularny portal społecznościowy}\end{PuzzleClues}\newpage\section*{Krzyżówka 78}

\noindent\begin{Puzzle}{21}{22}|*	|*	|*	|*	|*	|*	|*	|*	|*	|*	|*	|[1][S]\drarr	|p	|o	|l	|e	|s	|z	|u	|k	|*	|*	|.
|*	|*	|*	|*	|*	|[2][S]\rarr	|c	|e	|l	|a	|[][,]{ }	|ś	|m	|i	|e	|r	|c	|i	|*	|*	|*	|*	|.
|*	|[3][S]\rarr	|b	|a	|r	|c	|z	|a	|t	|k	|o	|w	|a	|t	|e	|*	|*	|*	|*	|*	|*	|*	|.
|*	|*	|[4][S]\rarr	|p	|u	|ł	|a	|p	|k	|a	|[][,]{ }	|i	|n	|f	|l	|a	|c	|y	|j	|n	|a	|*	|.
|*	|*	|*	|*	|[5][S]\darr	|[6][S]\darr	|[7][S]\rarr	|o	|l	|u	|n	|e	|k	|*	|*	|*	|[8][S]\darr	|*	|*	|*	|[9][S]\darr	|*	|.
|*	|[10][S]\rarr	|t	|r	|a	|n	|s	|p	|o	|z	|y	|c	|j	|a	|*	|*	|l	|[11][S]\darr	|[12][S]\darr	|*	|p	|*	|.
|*	|[13][S]\darr	|*	|[14][S]\darr	|u	|a	|[15][S]\drarr	|g	|l	|i	|n	|a	|[][,]{ }	|m	|o	|r	|e	|n	|o	|w	|a	|*	|.
|*	|t	|*	|d	|k	|t	|d	|*	|*	|*	|*	|[][,]{ }	|*	|*	|*	|*	|w	|i	|g	|[16][S]\darr	|w	|[17][S]\darr	|.
|*	|r	|[18][S]\darr	|a	|c	|a	|e	|*	|*	|*	|[19][S]\drarr	|s	|z	|e	|w	|*	|i	|e	|ó	|r	|i	|a	|.
|*	|u	|k	|t	|j	|l	|t	|*	|*	|*	|n	|t	|*	|*	|*	|*	|s	|w	|r	|o	|l	|k	|.
|*	|s	|r	|o	|a	|*	|a	|*	|*	|*	|o	|a	|*	|*	|*	|*	|*	|y	|e	|g	|o	|c	|.
|[20][S]\rarr	|t	|a	|u	|*	|*	|l	|*	|*	|*	|ś	|n	|*	|[21][S]\darr	|*	|*	|*	|b	|k	|a	|n	|e	|.
|*	|*	|t	|z	|*	|*	|*	|*	|*	|*	|n	|d	|*	|ł	|*	|*	|*	|u	|[][,]{ }	|t	|*	|n	|.
|*	|[22][S]\rarr	|e	|a	|s	|t	|[][,]{ }	|i	|n	|d	|i	|a	|m	|a	|n	|*	|*	|c	|r	|y	|*	|t	|.
|*	|*	|g	|u	|*	|*	|[23][S]\rarr	|z	|e	|g	|a	|r	|[][,]{ }	|s	|z	|a	|c	|h	|o	|w	|y	|*	|.
|[24][S]\rarr	|s	|u	|r	|a	|ż	|*	|*	|*	|*	|*	|d	|*	|k	|*	|*	|[25][S]\darr	|*	|g	|k	|*	|*	|.
|*	|*	|s	|*	|*	|*	|[26][S]\rarr	|s	|z	|y	|n	|o	|b	|u	|s	|*	|ł	|*	|a	|a	|*	|*	|.
|*	|*	|*	|*	|*	|*	|*	|[27][S]\drarr	|s	|c	|h	|w	|e	|n	|k	|*	|y	|*	|t	|*	|*	|*	|.
|[28][S]\rarr	|p	|e	|r	|s	|p	|e	|k	|t	|y	|w	|a	|*	|*	|*	|*	|k	|*	|y	|*	|*	|*	|.
|[29][S]\rarr	|t	|a	|k	|i	|e	|l	|u	|n	|e	|k	|*	|[30][S]\rarr	|p	|i	|z	|a	|*	|*	|*	|*	|*	|.
|*	|*	|*	|[31][S]\rarr	|g	|e	|o	|l	|o	|g	|[][,]{ }	|g	|ó	|r	|n	|i	|c	|z	|y	|*	|*	|*	|.
|*	|*	|[32][S]\rarr	|s	|a	|m	|o	|t	|a	|*	|*	|*	|[33][S]\rarr	|s	|k	|r	|z	|e	|c	|z	|*	|*	|.
|*	|[34][S]\rarr	|s	|e	|n	|a	|t	|*	|*	|*	|*	|*	|*	|*	|*	|*	|*	|*	|*	|*	|*	|*	|.\end{Puzzle}

\newpage

\begin{PuzzleClues}{\textbf{Poziome}\\}\Clue{1}{}{osoba będąca przedstawicielem Poleszuków - grupy etnicznej zamieszkującej tereny Polski, Białorusi i Ukrainy}
\Clue{2}{}{cela, w której skazany na karę śmierci oczekuje na wykonanie wyroku}
\Clue{3}{}{Lasiocampidae - rodzina owadów z rzędu motyli; obejmuje ponad 1000 gatunków nocnych motyli, występujących w lasach i sadach wszystkich kontynentów; w Polsce występuje około 19 gatunków}
\Clue{4}{}{zjawisko gospodarcze, polegające na realnym obniżeniu wartości osiąganych dochodów części płatników podatków na skutek niedopasowania wysokości progów podatkowych do tempa waloryzacji wynagrodzeń}
\Clue{7}{}{w górnictwie: półokrągłe wycięcie w podstropnicowym końcu stojaka, w które wchodzi stropnica}
\Clue{10}{}{przeniesienie utworu muzycznego lub jego części do inne tonacji}
\Clue{15}{}{skała ilasta, rodzaj gliny, niewarstwowany osadowy materiał skalny}
\Clue{19}{}{w budowie kokonu (owadów, ślimaków) fragment, który ma luźniejszą strukturę i przez który w momencie wylęgu wydostają się młode organizmy}
\Clue{20}{}{Chrześcijański znak, symbol krzyża - w kształcie greckiej litery tau (czyli krzyż bez górnej pionowej belki); szczególnie popularny wśród franciszkanów i przez nich propagowany)}
\Clue{22}{}{duży, uzbrojony statek handlowy o napędzie żaglowym umożliwiający samodzielne podróże na długie dystanse}
\Clue{23}{}{urządzenie służące do odmierzania czasu podczas gry w szachy, ale również innych gier planszowych z udziałem dwóch zawodników (np. warcaby, go)}
\Clue{24}{}{Suraż - miasto w Rosji w obwodzie briańskim, stolica administracyjna rejonu suraskiego}
\Clue{26}{}{lekki wagon silnikowy lub zespół trakcyjny o napędzie spalinowym, służący do obsługi ruchu pasażerskiego na mniej uczęszczanych liniach; w porównaniu z tradycyjnym pociągiem, autobus szynowy o podobnej pojemności charakteryzuje się większymi przyspieszeniami, mniejszym zużyciem paliwa i większymi prędkościami osiąganymi na tych samych - zwykle drugorzędnych - szlakach}
\Clue{27}{}{pływak amerykański, srebrny medalista z Atlanty w wyścigu na 200 m stylem grzbietowym}
\Clue{28}{}{otwarty, rozległy widok, panorama}
\Clue{29}{}{OLINOWANIE}
\Clue{30}{}{miasto we Włoszech (Toskania) nad rzeką Arno, międzynarodowe znaczenie turystyczne, tzw. krzywa wieża}
\Clue{31}{}{specjalista z dziedziny geologii górnictwa}
\Clue{32}{}{samotność}
\Clue{33}{}{Paweł, pięściarz wicemistrz olimpijski w kategorii półciężkiej z Moskwy, wicemistrz świata z 1982 r}
\Clue{34}{}{budynek rządowy, w którym odbywają się posiedzenia senatu}\end{PuzzleClues}

\begin{PuzzleClues}{\textbf{Pionowe}\\}\Clue{1}{}{obiekt astronomiczny o znanej absolutnej wielkości gwiazdowej; porównując znaną jasność absolutną z jasnością pozorną (obserwowaną), możemy wyznaczyć odległość do takiego obiektu}
\Clue{5}{}{zorganizowana forma sprzedaży, będąca formą przetargu prowadzonego na żywo}
\Clue{6}{}{prowincja we wschodniej części R.P.A, powierzchnia 87 tyś. km2, ośrodek administracyjny Pietermaritzburg}
\Clue{8}{}{Sinclair (1885-1951), pisarz amerykański, powieści społeczno-obyczajowe; „Babbit”, „Ulica główna” - Nobel 1930}
\Clue{9}{}{wyróżniające się dachem boczne skrzydło budynku}
\Clue{11}{}{przedmiot zawierający materiał wybuchowy w stanie wolnym, który powinien zdetonować, jednak pomimo stworzenia warunków koniecznych do tego procesu nie doszło do wybuchu}
\Clue{12}{}{owoc kiwano}
\Clue{13}{}{przedsiębiorstwo działające na zasadzie trustu - połączenie kilku mniejszych przedsiębiorstw działających dotąd samodzielnie, których dotychczasowi właściciele stają się udziałowcami powstałego tworu}
\Clue{14}{}{Datousaurus - rodzaj zauropoda o bliżej niesprecyzowanej pozycji systematycznej; żył w epoce środkowej jury (ok. 165 mln lat temu) na terenach wschodniej Azji; długość ciała ok. 14-15 m, wysokość ok. 5 m}
\Clue{15}{}{niewielka, mało ważna rzecz, która jest częścią całości}
\Clue{16}{}{KRAKUSKA}
\Clue{17}{}{znak graficzny nad lub pod literą informujący o sposobie jej wymawiania}
\Clue{18}{}{drewno kominkowe i opałowe pozyskiwane z drzewa o tej samej nazwie}
\Clue{19}{}{element urządzenia transportowego, na którym spoczywa przenoszony materiał}
\Clue{21}{}{nadrzewny lub naziemny ssak z rodziny łasz}
\Clue{25}{}{połykacz, człowiek, który daje pokazy tego, jak coś połyka lub udaje, że połyka}
\Clue{27}{}{zespół wierzeń i związanych z nich działań podejmowanych, aby uczcić bóstwo lub jego artefakt}\end{PuzzleClues}\newpage\section*{Krzyżówka 79}

\noindent\begin{Puzzle}{24}{33}|*	|*	|[1][S]\darr	|*	|*	|*	|*	|*	|*	|*	|*	|*	|*	|*	|*	|*	|*	|*	|*	|*	|[2][S]\darr	|*	|*	|*	|*	|.
|*	|*	|z	|*	|*	|*	|*	|*	|*	|*	|*	|*	|*	|*	|*	|*	|*	|*	|*	|*	|h	|*	|*	|*	|*	|.
|*	|*	|a	|*	|*	|*	|*	|*	|*	|*	|*	|*	|*	|*	|*	|*	|*	|*	|*	|*	|e	|*	|*	|*	|*	|.
|*	|*	|ć	|*	|*	|*	|*	|*	|*	|*	|*	|*	|*	|*	|*	|*	|*	|*	|*	|*	|k	|*	|*	|*	|*	|.
|*	|*	|m	|*	|*	|*	|*	|*	|*	|*	|*	|*	|*	|*	|*	|*	|*	|*	|*	|*	|s	|*	|*	|*	|*	|.
|*	|*	|a	|*	|*	|*	|*	|*	|*	|*	|*	|*	|*	|*	|*	|*	|*	|*	|*	|*	|a	|*	|*	|*	|*	|.
|*	|*	|[][,]{ }	|*	|*	|*	|*	|*	|*	|*	|*	|*	|*	|*	|*	|*	|*	|*	|*	|*	|f	|*	|*	|[3][S]\darr	|*	|.
|*	|*	|t	|*	|*	|*	|*	|*	|*	|*	|*	|*	|*	|*	|*	|*	|*	|*	|*	|*	|l	|*	|*	|r	|*	|.
|*	|*	|o	|*	|*	|*	|*	|*	|*	|*	|*	|*	|*	|*	|*	|*	|*	|*	|*	|*	|u	|*	|*	|o	|*	|.
|*	|*	|r	|*	|*	|*	|*	|*	|*	|*	|*	|*	|*	|*	|*	|*	|*	|*	|*	|*	|o	|*	|*	|t	|*	|.
|*	|*	|e	|*	|*	|[4][S]\darr	|*	|*	|*	|*	|*	|*	|*	|*	|*	|*	|*	|[5][S]\darr	|[6][S]\drarr	|k	|r	|e	|d	|a	|*	|.
|[7][S]\rarr	|o	|b	|j	|ę	|t	|o	|ś	|ć	|[][,]{ }	|w	|y	|d	|e	|c	|h	|o	|w	|a	|*	|e	|*	|*	|t	|*	|.
|*	|*	|k	|*	|*	|o	|*	|*	|*	|*	|*	|*	|*	|*	|[8][S]\rarr	|c	|h	|a	|r	|t	|k	|a	|*	|o	|*	|.
|*	|*	|o	|*	|*	|n	|*	|*	|*	|*	|*	|*	|*	|*	|*	|*	|*	|m	|i	|*	|[][,]{ }	|*	|*	|r	|*	|.
|*	|*	|w	|*	|*	|i	|*	|*	|*	|*	|*	|*	|*	|*	|*	|*	|*	|p	|s	|*	|s	|*	|*	|*	|*	|.
|*	|*	|a	|*	|*	|n	|*	|*	|*	|*	|*	|*	|*	|*	|*	|*	|*	|i	|t	|*	|i	|*	|*	|*	|*	|.
|*	|*	|[][,]{ }	|*	|*	|*	|*	|*	|*	|*	|*	|*	|*	|*	|*	|*	|*	|r	|o	|*	|a	|*	|*	|*	|*	|.
|*	|*	|p	|*	|*	|*	|*	|*	|*	|*	|[9][S]\rarr	|k	|a	|d	|e	|t	|t	|*	|z	|*	|r	|*	|*	|*	|*	|.
|*	|*	|r	|*	|*	|*	|*	|*	|*	|*	|*	|*	|*	|*	|*	|*	|*	|*	|u	|*	|k	|*	|*	|*	|*	|.
|*	|*	|z	|*	|*	|*	|*	|*	|*	|*	|*	|*	|*	|[10][S]\darr	|*	|*	|*	|*	|c	|*	|i	|*	|*	|*	|*	|.
|*	|*	|e	|*	|*	|*	|*	|*	|*	|*	|*	|*	|*	|s	|*	|*	|*	|*	|h	|*	|*	|*	|*	|*	|*	|.
|*	|*	|d	|*	|*	|*	|*	|*	|*	|*	|*	|[11][S]\rarr	|m	|a	|ł	|p	|k	|a	|*	|*	|*	|*	|*	|*	|*	|.
|*	|*	|n	|*	|*	|*	|*	|*	|*	|*	|*	|*	|*	|k	|*	|*	|*	|*	|*	|*	|*	|*	|*	|*	|*	|.
|*	|*	|i	|*	|*	|*	|*	|*	|*	|*	|*	|*	|*	|i	|*	|*	|*	|*	|*	|*	|*	|*	|*	|*	|*	|.
|*	|*	|a	|[12][S]\rarr	|z	|w	|i	|ą	|z	|k	|o	|w	|i	|e	|c	|*	|*	|*	|*	|*	|*	|*	|*	|*	|*	|.
|*	|*	|[][,]{ }	|*	|*	|*	|*	|*	|*	|*	|*	|*	|*	|w	|*	|*	|*	|*	|*	|*	|*	|*	|*	|*	|*	|.
|*	|*	|i	|*	|*	|*	|*	|*	|*	|*	|*	|*	|*	|n	|*	|*	|*	|*	|*	|*	|*	|*	|*	|*	|*	|.
|*	|*	|[][,]{ }	|*	|*	|*	|*	|*	|*	|*	|*	|*	|*	|i	|*	|*	|*	|*	|*	|*	|*	|*	|*	|*	|*	|.
|*	|*	|t	|*	|*	|*	|*	|*	|*	|*	|*	|*	|*	|k	|*	|*	|*	|*	|*	|*	|*	|*	|*	|*	|*	|.
|*	|*	|y	|*	|*	|*	|*	|*	|*	|*	|*	|*	|*	|*	|*	|*	|*	|*	|*	|*	|*	|*	|*	|*	|*	|.
|*	|*	|l	|*	|*	|*	|*	|*	|*	|*	|*	|*	|*	|*	|*	|*	|*	|*	|*	|*	|*	|*	|*	|*	|*	|.
|*	|*	|n	|*	|*	|*	|*	|*	|*	|*	|*	|*	|*	|*	|*	|*	|*	|*	|*	|*	|*	|*	|*	|*	|*	|.
|*	|*	|a	|*	|*	|*	|*	|*	|*	|*	|*	|*	|*	|*	|*	|*	|*	|*	|*	|*	|*	|*	|*	|*	|*	|.
|*	|*	|*	|*	|*	|*	|*	|*	|*	|*	|*	|*	|*	|*	|*	|*	|*	|*	|*	|*	|*	|*	|*	|*	|*	|.\end{Puzzle}

\newpage

\begin{PuzzleClues}{\textbf{Poziome}\\}\Clue{6}{}{materiał przeznaczony do pisania po tablicy szkolnej, zbudowany z wapieni, pozyskiwany ze złóż skalistej kredy}
\Clue{7}{}{ilość powietrza usuwanego z płuc podczas normalnego oddychania}
\Clue{8}{}{suczka charta}
\Clue{9}{}{model samochodu osobowego, produkowanego w latach 1936-1940, a następnie 1962-1993, przez firmę Opel}
\Clue{11}{}{zdrobniale: małpa - zwierzę}
\Clue{12}{}{osoba, która należy do związku - organizacji zrzeszającej ludzi o podobnych poglądach, trzymających się wspólnych zasad}\end{PuzzleClues}

\begin{PuzzleClues}{\textbf{Pionowe}\\}\Clue{1}{}{cataracta capsularis anterior et posterior - odmiana zaćmy wrodzonej}
\Clue{2}{}{nieorganiczny związek chemiczny o bardzo dobrych własnościach dielektrycznych}
\Clue{3}{}{namagnesowana wkładka ferrytowa w falowodzie}
\Clue{4}{}{Cephalorhynchus heavisidii - gatunek walenia z rodziny delfinowatych; występuje w morzach południowo-zachodniej Afryki}
\Clue{5}{}{ameryk. nietoperz żywiący się krwią; atakuje ssaki}
\Clue{6}{}{Aristosuchus - rodzaj drapieżnego dinozaura z rodziny kompsognatów; żył w okresie wczesnej kredy na terenach współczesnej Europy}
\Clue{10}{}{Origma solitaria - gatunek małego ptaka z rodziny buszówkowatych (Acanthizidae); endemit, występuje jedynie na terenie Australii, w Nowej Południowej Walii; jego środowiskiem występowania są lasy klimatu umiarkowanego, zarośla oraz obszary skalne (klify, szczyty górskie)}\end{PuzzleClues}\newpage\section*{Krzyżówka 80}

\noindent\begin{Puzzle}{24}{32}|*	|*	|*	|*	|*	|*	|*	|*	|*	|[1][S]\drarr	|n	|i	|e	|u	|n	|i	|k	|n	|i	|o	|n	|o	|ś	|ć	|*	|.
|*	|*	|[2][S]\rarr	|z	|a	|j	|ą	|c	|[][,]{ }	|w	|i	|e	|l	|k	|o	|u	|c	|h	|y	|*	|*	|*	|*	|*	|*	|.
|*	|*	|*	|*	|[3][S]\drarr	|z	|w	|i	|ą	|z	|e	|k	|*	|[4][S]\drarr	|w	|i	|a	|t	|r	|a	|c	|z	|e	|k	|*	|.
|*	|*	|*	|[5][S]\drarr	|w	|y	|c	|i	|ą	|g	|[][,]{ }	|l	|a	|b	|o	|r	|a	|t	|o	|r	|y	|j	|n	|y	|*	|.
|*	|*	|[6][S]\drarr	|b	|i	|d	|w	|e	|l	|l	|*	|[7][S]\darr	|*	|ł	|*	|[8][S]\darr	|[9][S]\darr	|*	|[10][S]\drarr	|w	|e	|n	|u	|s	|*	|.
|*	|[11][S]\drarr	|p	|o	|n	|ę	|t	|a	|*	|ą	|*	|e	|*	|ę	|[12][S]\darr	|u	|n	|*	|r	|*	|[13][S]\darr	|[14][S]\darr	|*	|*	|*	|.
|*	|l	|r	|r	|s	|*	|*	|*	|[15][S]\drarr	|d	|n	|o	|*	|k	|e	|c	|i	|*	|y	|*	|k	|ł	|*	|*	|*	|.
|*	|e	|o	|t	|t	|*	|*	|*	|z	|*	|*	|l	|*	|i	|f	|h	|c	|[16][S]\darr	|n	|[17][S]\rarr	|w	|a	|l	|*	|*	|.
|*	|c	|b	|a	|o	|*	|*	|*	|e	|*	|[18][S]\darr	|*	|*	|t	|e	|t	|i	|s	|e	|[19][S]\darr	|e	|ń	|*	|*	|*	|.
|*	|z	|a	|*	|n	|*	|*	|*	|g	|[20][S]\drarr	|p	|i	|a	|n	|k	|a	|*	|t	|k	|b	|f	|c	|[21][S]\darr	|*	|*	|.
|*	|n	|n	|*	|[][,]{ }	|*	|*	|[22][S]\darr	|a	|p	|r	|[23][S]\drarr	|v	|a	|t	|*	|[24][S]\darr	|o	|[][,]{ }	|a	|*	|u	|l	|*	|*	|.
|*	|i	|t	|*	|c	|*	|*	|w	|r	|e	|o	|e	|*	|[][,]{ }	|[][,]{ }	|*	|p	|p	|f	|l	|*	|c	|i	|*	|*	|.
|*	|c	|k	|*	|h	|*	|*	|y	|[][,]{ }	|r	|m	|n	|*	|k	|z	|*	|n	|i	|i	|o	|*	|h	|c	|*	|*	|.
|*	|t	|a	|*	|u	|*	|[25][S]\rarr	|s	|p	|l	|o	|t	|[][,]{ }	|r	|a	|m	|i	|e	|n	|n	|y	|*	|z	|*	|*	|.
|*	|w	|*	|[26][S]\darr	|r	|*	|[27][S]\darr	|o	|i	|*	|t	|r	|*	|e	|t	|*	|a	|ń	|a	|[][,]{ }	|*	|[28][S]\darr	|e	|*	|*	|.
|[29][S]\drarr	|o	|w	|o	|c	|ó	|w	|k	|a	|*	|o	|e	|*	|w	|ł	|*	|k	|*	|n	|z	|*	|n	|b	|*	|*	|.
|c	|[][,]{ }	|*	|b	|h	|*	|ó	|i	|s	|[30][S]\darr	|r	|c	|*	|*	|o	|*	|*	|[31][S]\rarr	|s	|a	|g	|a	|n	|*	|*	|.
|a	|z	|*	|i	|i	|*	|z	|[][,]{ }	|k	|a	|s	|h	|*	|*	|c	|*	|[32][S]\darr	|[33][S]\darr	|o	|p	|*	|n	|i	|*	|*	|.
|t	|a	|*	|e	|l	|*	|e	|k	|o	|r	|t	|a	|*	|*	|z	|*	|m	|k	|w	|o	|*	|d	|k	|*	|*	|.
|e	|m	|*	|g	|l	|*	|k	|o	|w	|a	|w	|t	|*	|*	|e	|[34][S]\darr	|o	|a	|y	|r	|[35][S]\darr	|u	|[][,]{ }	|*	|*	|.
|r	|k	|*	|*	|*	|*	|*	|m	|y	|b	|o	|*	|*	|*	|n	|k	|s	|b	|*	|o	|p	|*	|z	|*	|*	|.
|i	|n	|*	|*	|[36][S]\darr	|*	|[37][S]\darr	|i	|*	|e	|*	|[38][S]\darr	|*	|*	|i	|o	|t	|o	|*	|w	|i	|*	|b	|*	|*	|.
|n	|i	|*	|*	|m	|[39][S]\darr	|f	|s	|*	|s	|*	|o	|[40][S]\drarr	|t	|a	|r	|o	|t	|*	|y	|e	|*	|i	|*	|*	|.
|g	|ę	|*	|*	|a	|b	|e	|a	|[41][S]\drarr	|k	|u	|b	|a	|s	|*	|a	|w	|a	|*	|*	|s	|*	|o	|*	|*	|.
|*	|t	|*	|*	|n	|e	|r	|r	|w	|a	|*	|l	|m	|*	|*	|*	|n	|ż	|*	|*	|z	|*	|r	|*	|*	|.
|*	|e	|*	|*	|[][,]{ }	|r	|r	|z	|r	|*	|*	|a	|b	|*	|*	|*	|i	|o	|[42][S]\darr	|[43][S]\darr	|c	|*	|o	|*	|*	|.
|*	|*	|*	|*	|s	|i	|y	|*	|i	|*	|*	|t	|i	|*	|*	|*	|c	|w	|p	|d	|z	|*	|w	|*	|*	|.
|*	|*	|[44][S]\rarr	|p	|e	|n	|t	|a	|g	|o	|n	|*	|t	|*	|*	|*	|a	|i	|l	|a	|o	|*	|y	|*	|*	|.
|*	|*	|*	|*	|*	|g	|*	|[45][S]\rarr	|h	|e	|s	|s	|*	|[46][S]\rarr	|a	|l	|*	|e	|a	|i	|t	|*	|*	|*	|*	|.
|[47][S]\drarr	|z	|w	|ó	|d	|*	|*	|[48][S]\rarr	|t	|k	|a	|n	|k	|a	|[][,]{ }	|ł	|ą	|c	|z	|n	|a	|*	|*	|*	|*	|.
|o	|*	|[49][S]\rarr	|v	|i	|d	|i	|n	|*	|[50][S]\rarr	|w	|o	|ł	|y	|n	|o	|w	|*	|m	|a	|*	|*	|*	|*	|*	|.
|ś	|*	|*	|*	|*	|*	|*	|*	|*	|*	|[51][S]\rarr	|ł	|u	|k	|o	|w	|i	|c	|a	|*	|*	|*	|*	|*	|*	|.
|*	|*	|*	|*	|*	|*	|*	|*	|*	|*	|*	|*	|*	|*	|*	|*	|*	|*	|*	|*	|*	|*	|*	|*	|*	|.\end{Puzzle}

\newpage

\begin{PuzzleClues}{\textbf{Poziome}\\}\Clue{1}{}{to, że coś jest nieuniknione, nie można tego uniknąć, dzieje się w sposób nieunikniony}
\Clue{2}{}{Lepus californicus - gatunek ssaka z rodziny zającowatych, nazwę zawdzięczający dużym uszom; występuje w północnej i środkowej Ameryce Północnej, żyje w lasach}
\Clue{3}{}{powiązanie, zależność, stosunek między rzeczami, podmiotami, przedmiotami}
\Clue{4}{}{bardzo mały wiatrak}
\Clue{5}{}{specjalny wyciąg, oszklona komora laboratoryjna instalowana w laboratoriach chemicznych dla usuwania szkodliwych gazów}
\Clue{6}{}{pisarz angielski osiadły w Polsce (1905-89), szkice, wspomnienia, powieści; „Najcenniejszy klejnot”, „Król diamentów”, „Szkoła”}
\Clue{10}{}{druga według kolejności od Słońca planeta Układu Słonecznego}
\Clue{11}{}{pokusa, coś, co budzi wielką chęć do czegoś}
\Clue{15}{}{dolna powierzchnia otworu wiertniczego}
\Clue{17}{}{WALEŃ; wieloryb gładkoskóry}
\Clue{20}{}{dodatek do ciast, robiony przeważnie z białek jajek i cukru}
\Clue{23}{}{podatek pośredni, pobierany na każdym kolejnym etapie obrotu towarami lub usługami (podatek obrotowy), którego konstrukcja zakłada brak kaskadowego nakładania się podatku poprzez zastosowanie mechanizmu odliczenia podatku pobranego w poprzednich etapach obrotu}
\Clue{25}{}{twór powstały z gałązek przednich (brzusznych) nerwów rdzeniowych biegnących od rdzenia kręgowego}
\Clue{29}{}{motyl nocny, szkodnik upraw}
\Clue{31}{}{ur. 1935r, pisarka francuska, współczesne powieści psychologiczne; „Witaj smutku”, „Zamek w Szwecji”, „Pewien uśmiech”, nowele, sztuki}
\Clue{40}{}{talia 78 kart do gry, wróżenia lub medytacji}
\Clue{41}{}{duży kubek}
\Clue{44}{}{pięciokąt foremny -  figura wypukła, pięciokąt o wszystkich bokach równej długości i wszystkich kątach równych}
\Clue{45}{}{rosyjski chemik i mineralog (1802-50); podał podstawowe prawo termochemii zwane prawem Hessa}
\Clue{46}{}{w chemii: symbol glinu}
\Clue{47}{}{przewód elektryczny o małej rezystencji}
\Clue{48}{}{tkanka odpowiedzialna za spajanie innych tkanek w ogranizmie}
\Clue{49}{}{miasto w Rumunii nad Dunajem}
\Clue{50}{}{kosmonauta radziecki, uczestnik lotu w 1969r. podczas którego doszło do połączenia statków}
\Clue{51}{}{wieś w Polsce położona w województwie małopolskim, w powiecie limanowskim, w gminie Łukowica}\end{PuzzleClues}

\begin{PuzzleClues}{\textbf{Pionowe}\\}\Clue{1}{}{jeden z kilku punktów widzenia, jedna z opcji czy aspektów, pod warunkiem których rozpatruje się jakąś sprawę}
\Clue{3}{}{brytyjski polityk, mówca, strateg, pisarz i historyk, dwukrotny premier Zjednoczonego Królestwa, laureat literackiej Nagrody Nobla, honorowy obywatel Stanów Zjednoczonych}
\Clue{4}{}{przynależność do rodziny arystokratycznej, związana z jasnym i bladym odcieniem skóry arystokratów; związek frazeologiczny}
\Clue{5}{}{taśma pasmanteryjna do lamowania, obszywania brzegów odzieży, zasłon itp. lamówka, obszywka}
\Clue{6}{}{zakonnica na jednym z pierwszych etapów życia zakonnego, która złożyła śluby czasowe}
\Clue{7}{}{członek plemienia zasiedlającego w starożytności rejony środkowej Grecji}
\Clue{8}{}{miasto w europejskiej części Federacji Rosyjskiej nad rzeką Uchta: wydobycie ropy naftowej i gazu}
\Clue{9}{}{szkliste zanieczyszczenie w postaci nitek o barwie szkła - lub ciemniejszej, występuje na powierzchni wyrobów ze szkła}
\Clue{10}{}{miejsce, w którym są zawierane transakcje kupna i sprzedaży różnych form kapitału pieniężnego}
\Clue{11}{}{leczenie wymagające pobytu w placówce medycznej (szpitalu lub sanatorium) dłużej niż jedną dobę}
\Clue{12}{}{zjawisko polegające na zmniejszeniu się indywidualnych korzyści z danego dobra wraz ze wzrostem jego użytkowników}
\Clue{13}{}{dawne określenie nakrycia głowy zakonnic i wdów czy starszych pań}
\Clue{14}{}{szereg połączonych ogniw, które tworzą coś na kształt liny}
\Clue{15}{}{przyrząd złożony z dwóch szklanych baniek, pokazujący czas poprzez miarowe przesypywanie piasku pomiędzy nimi}
\Clue{16}{}{ocena, nota wystawiana uczniowi, informująca o poziomie jego wiedzy, najczęściej wyrażana liczbowo}
\Clue{18}{}{ogólnie o opiece naukowej promotora nad swoimi podopiecznymi}
\Clue{19}{}{bezzałogowy balon na uwięzi, który z systemem lin stalowych stosowano jako przeszkodę dla wrogich samolotów bombowych lecących na małych i średnich wysokościach}
\Clue{20}{}{drobna czcionka drukarska o wielkości pięciu punktów, wykorzystywana przy druku notatek i ogłoszeń}
\Clue{21}{}{liczebniki łączące się z rzeczownikami mającymi tylko liczbę mnogą, oznaczającymi osoby różnej płci lub osoby niedorosłe, np. dwoje, pięcioro}
\Clue{22}{}{wysoki urzędnik międzynarodowy}
\Clue{23}{}{poziomy skok z dwu nóg, w trakcie którego nogi są kilka razy rozdzielane i szybko krzyżowane}
\Clue{24}{}{łow. górny kieł u jeleni}
\Clue{26}{}{ruch cieczy lub gazu w układzie termodynamicznie zamkniętym}
\Clue{27}{}{w maszynach do pisania podzespołów się poziomo po prowadnicy, wyposażony w wałek i inne urządzenia}
\Clue{28}{}{Rheidae - rodzina ptaków paleognatycznych z rzędu nandu (Rheiformes)}
\Clue{29}{}{firma cateringowa}
\Clue{30}{}{nowy gatunek prozy artystycznej, niewielki rozmiarami utwór nasycony elementami baśniowej fantastyki}
\Clue{32}{}{belka używana jako podkład pod szyny na moście kolejowym}
\Clue{33}{}{statek wykonujący rejsy między portami jednego kraju, w obrębie tego samego morza}
\Clue{34}{}{nepalska broń sieczna; rodzaj szabli o zakrzywionej jednosiecznej głowni}
\Clue{35}{}{bardzo przyjemne doznanie, coś, co jest odbierane jako rozkosz}
\Clue{36}{}{spółka akcyjna z siedzibą w Monachium; jedno z wiodących przedsiębiorstw produkujących samochody ciężarowe, autobusy, silniki oraz urządzenia przemysłowe pod marką MAN}
\Clue{37}{}{substancja do produkcji magnesów}
\Clue{38}{}{osoba żyjąca przy zakonie, która nie składa ślubów, a jedynie przyrzeczenie wytrwania}
\Clue{39}{}{duński żeglarz i odkrywca (1681-1741); odkrył brzegi Alaski i Aleuty}
\Clue{40}{}{OBEJŚCIE; przedłużenie naw bocznych wokół prezbiterium}
\Clue{41}{}{bracia; konstruktorzy, piloci, pionierzy lotnictwa silnikowego, odbyli w 1908 r. pierwszy lot z pasażerem}
\Clue{42}{}{zasadniczy płynny składnik krwi, w którym są zawieszone elementy morfotyczne (komórkowe); stanowi ok. 55\% objętości krwi}
\Clue{43}{}{łotewska pieśń ludowa}
\Clue{47}{}{abstrakcyjna linia podziału, będąca punktem odniesienia do wyróżniania jakiejś dychotomii}\end{PuzzleClues}\newpage\section*{Krzyżówka 81}

\noindent\begin{Puzzle}{21}{25}|*	|*	|[1][S]\darr	|*	|*	|*	|[2][S]\darr	|*	|[3][S]\darr	|*	|[4][S]\drarr	|f	|a	|z	|a	|*	|*	|*	|[5][S]\drarr	|r	|g	|*	|.
|*	|*	|ż	|*	|*	|[6][S]\darr	|n	|[7][S]\drarr	|t	|r	|a	|n	|s	|p	|a	|r	|e	|n	|c	|j	|a	|*	|.
|*	|*	|y	|*	|*	|z	|i	|o	|a	|*	|g	|*	|[8][S]\darr	|*	|*	|[9][S]\darr	|*	|*	|h	|*	|*	|*	|.
|*	|*	|ł	|*	|*	|a	|ż	|b	|c	|[10][S]\darr	|a	|*	|w	|*	|*	|t	|[11][S]\drarr	|t	|r	|o	|n	|*	|.
|*	|*	|a	|*	|[12][S]\darr	|d	|s	|r	|z	|t	|t	|*	|o	|*	|*	|u	|p	|[13][S]\darr	|o	|[14][S]\darr	|*	|*	|.
|*	|*	|[][,]{ }	|[15][S]\drarr	|m	|u	|z	|y	|k	|a	|[][,]{ }	|k	|a	|m	|e	|r	|a	|l	|n	|a	|*	|*	|.
|*	|*	|g	|g	|i	|f	|e	|s	|a	|c	|m	|*	|l	|*	|[16][S]\darr	|y	|t	|e	|o	|k	|*	|[17][S]\darr	|.
|*	|*	|ł	|a	|e	|a	|[][,]{ }	|i	|r	|h	|s	|*	|*	|[18][S]\darr	|d	|s	|r	|e	|m	|a	|*	|t	|.
|*	|*	|ó	|r	|s	|n	|n	|e	|z	|o	|z	|*	|*	|s	|a	|t	|y	|d	|e	|d	|*	|o	|.
|*	|*	|w	|u	|z	|o	|a	|*	|*	|g	|y	|[19][S]\darr	|*	|p	|n	|y	|c	|s	|t	|e	|*	|r	|.
|*	|*	|n	|d	|a	|ś	|c	|*	|*	|r	|s	|d	|*	|i	|d	|k	|j	|*	|r	|m	|*	|e	|.
|*	|*	|a	|i	|c	|ć	|z	|*	|*	|a	|t	|i	|[20][S]\drarr	|n	|i	|a	|u	|x	|*	|i	|[21][S]\darr	|b	|.
|*	|*	|[][,]{ }	|m	|z	|*	|e	|*	|*	|m	|y	|a	|h	|a	|n	|[][,]{ }	|s	|*	|*	|k	|p	|n	|.
|*	|[22][S]\darr	|d	|i	|*	|[23][S]\darr	|l	|*	|*	|*	|*	|b	|e	|*	|*	|p	|z	|*	|*	|*	|r	|i	|.
|*	|v	|o	|m	|*	|n	|n	|[24][S]\rarr	|n	|u	|m	|e	|r	|*	|*	|r	|k	|*	|[25][S]\darr	|*	|o	|k	|.
|*	|a	|l	|*	|*	|e	|e	|*	|*	|*	|*	|ł	|o	|*	|*	|z	|a	|*	|m	|[26][S]\darr	|f	|[][,]{ }	|.
|*	|l	|n	|*	|*	|f	|*	|*	|*	|*	|*	|e	|d	|*	|*	|y	|*	|*	|l	|b	|o	|p	|.
|[27][S]\drarr	|g	|a	|s	|t	|r	|o	|n	|o	|m	|i	|k	|*	|*	|*	|j	|*	|*	|e	|r	|s	|i	|.
|b	|a	|*	|[28][S]\rarr	|h	|o	|a	|c	|y	|n	|y	|*	|*	|[29][S]\rarr	|m	|a	|n	|i	|c	|a	|*	|ż	|.
|a	|*	|[30][S]\rarr	|p	|a	|s	|m	|a	|n	|t	|e	|r	|i	|a	|*	|z	|*	|*	|z	|n	|[31][S]\darr	|m	|.
|g	|*	|*	|*	|[32][S]\rarr	|t	|o	|r	|o	|s	|*	|*	|[33][S]\rarr	|i	|n	|d	|y	|k	|*	|d	|p	|o	|.
|g	|*	|*	|*	|*	|o	|[34][S]\rarr	|n	|i	|e	|l	|i	|c	|z	|n	|o	|ś	|ć	|*	|*	|ó	|w	|.
|a	|*	|[35][S]\rarr	|b	|e	|m	|a	|r	|*	|*	|[36][S]\rarr	|c	|z	|e	|r	|w	|i	|e	|c	|*	|l	|y	|.
|l	|*	|[37][S]\rarr	|a	|l	|i	|k	|a	|n	|t	|*	|*	|*	|[38][S]\rarr	|z	|a	|w	|i	|s	|a	|k	|*	|.
|a	|*	|*	|[39][S]\rarr	|k	|a	|r	|b	|o	|m	|y	|c	|y	|n	|a	|*	|*	|*	|*	|*	|o	|*	|.
|*	|[40][S]\rarr	|l	|ó	|d	|*	|*	|*	|*	|*	|*	|*	|*	|*	|*	|*	|*	|*	|*	|*	|*	|*	|.\end{Puzzle}

\newpage

\begin{PuzzleClues}{\textbf{Poziome}\\}\Clue{4}{}{źródło napięcia lub prądu elektrycznego pracującego w sieci elektrycznej jedno- lub wielofazowej}
\Clue{5}{}{w chemii: symbol pierwiastka roentgen}
\Clue{7}{}{cecha działań, które są publiczne, ujawnione, ich stan i przebieg jest dostępny dla każdego zainteresowanego, nic w związku z nimi nie jest ukryte, niepewne, niejasne, wątpliwe, nic nie wzbudza podejrzeń}
\Clue{11}{}{mebel, ozdobne siedzisko, np. dla króla, biskupa itp}
\Clue{15}{}{muzyka przeznaczona na niewielkie zespoły, zazwyczaj do 9 wykonawców}
\Clue{20}{}{miasto w płd. Francji w departamencie Ariege, w pobliżu znana jaskinia z okresu górnego paleolitu}
\Clue{24}{}{pomieszczenie opatrzone kolejną liczbą, np. pokój hotelowy}
\Clue{27}{}{każda osoba zatrudniona w gastronomii}
\Clue{28}{}{Opisthocomidae - rodzina ptaków z podgromady Neornithes}
\Clue{29}{}{część zbroi rycerskiej}
\Clue{30}{}{wyrób włókienniczy, najczęściej w formie taśmy, używany do obszywania brzegów ubiorów, tkanin dekoracyjnych, obić tapicerskich, stosowany jako ozdoba, wzmocnienie i ochrona brzegów}
\Clue{32}{}{zwał lodu powstały wskutek ściśnięcia się pokrywy lodowej, może mieć wysokość nawet 10 metrów}
\Clue{33}{}{dość duży ptak hodowlany lub łowny z rodzaju Meleagris, ceniony ze względu na delikatne mięso}
\Clue{34}{}{to, że coś jest nieliczne, mało liczne}
\Clue{35}{}{pojemnik z podgrzewaną wodą, montowany w ladzie bufetu, służy do przechowywania gorących potraw}
\Clue{36}{}{szósty miesiąc w roku, według używanego w Polsce kalendarza gregoriańskiego, który ma 30 dni}
\Clue{37}{}{dawne określenie czerwonego wina produkowanego w Hiszpanii}
\Clue{38}{}{FERTAK}
\Clue{39}{}{słaby antybiotyk makrolidowy pozyskiwany z bakterii Streptomyces halstedii, hamujący rozwój m.in. bakterii Gram-dodatnich}
\Clue{40}{}{woda w stanie stałym}\end{PuzzleClues}

\begin{PuzzleClues}{\textbf{Pionowe}\\}\Clue{1}{}{pień żylny zbierający krew z podprzeponowej połowy ciała i uchodzący do prawego przedsionka serca}
\Clue{2}{}{Strepsirrhini - podrząd ssaków naczelnych obejmujący gatunki o cechach najbardziej prymitywnych wśród naczelnych; wspólną cechą wyróżniającą tę grupę zwierząt jest spiczasty pysk zakończony wilgotnym nosem połączonym z górną wargą}
\Clue{3}{}{robotnik pchający taczkę, przewożący ciężary na taczce}
\Clue{4}{}{odmiana agatu o jasnej barwie, z zielonymi wtronceniami hornblendy lub chlorytu, przypominającymi odrosty mchu}
\Clue{5}{}{zegarek kieszonkowy, który posiadał bardzo precyzyjny mechanizm, dokładnie wskazujący czas}
\Clue{6}{}{bycie zadufanym, przekonanym o własnej wyższości}
\Clue{7}{}{gabaryt}
\Clue{8}{}{przezroczysta, cienka tkanina z silnie skręconej przędzy; głównie zasłony i suknie}
\Clue{9}{}{przyjazdy do danego kraju osób, które na stałe mieszkają za granicą}
\Clue{10}{}{zapis pomiaru wykonywanego przez tachograf}
\Clue{11}{}{obywatelka Rzymu pochodząca z arystokratycznego rodu}
\Clue{12}{}{urządzenie lub maszyna przemysłowa do mieszania substancji sypkich ze sobą lub z płynami}
\Clue{13}{}{miasto w Anglii nad rzeką Aire w hrabstwie metropolitalnym West Yorkshire; ważny region przemysłowy}
\Clue{14}{}{członek akademii - organizacji naukowej lub artystycznej}
\Clue{15}{}{Garudimimus - rodzaj dinozaura z grupy ornitomimozauów; żył w późnej kredzie na terenach Azji}
\Clue{16}{}{VI/VII w , poeta sanskrycki; „Dasiakumaraczarita”}
\Clue{17}{}{Hypsiprymnodon moschatus - gatunek torbacza z rodziny kanguroszczurowatych; występuje w wilgotnych lasach równikowych na wschodzie Australii w północno-wschodnim Queensland}
\Clue{18}{}{kłótnia, zdarzenie konfliktowe}
\Clue{19}{}{dziecko (zwykle chłopiec) niegrzeczne, takie, które psoci; słowo używane często z pobłażliwością lub żartobliwie}
\Clue{20}{}{rzymski wielkorządca Judei wg Ewangelii zarządził rzeź noworodków płci męskiej w celu zgładzenia Jezusa}
\Clue{21}{}{podoficer z etatowej obsługi aresztu wojskowego}
\Clue{22}{}{miasto w Estonii przy granicy z Łotwą przemysł spożywczy, meblarski}
\Clue{23}{}{rodzaj urostomii polegający na odprowadzeniu moczu bezpośrednio z nerki przez wprowadzony do niej cewnik drogą operacyjną lub przy użyciu zestawu punkcyjnego}
\Clue{25}{}{białawy płyn nasienny występujący u samców ryb i mięczaków}
\Clue{26}{}{ur. w 1931 r. astronauta amerykański, uczestnik wspólnego lotu Sojuz-Apollo}
\Clue{27}{}{średniowieczny duży żaglowiec arabski; handlowy lub wojenny; dwumasztowy, o wydłużonym dziobie}
\Clue{31}{}{małe pole; zdrobniale lub pieszczotliwie o polu}\end{PuzzleClues}\newpage\section*{Krzyżówka 82}

\noindent\begin{Puzzle}{23}{31}|*	|*	|*	|[1][S]\darr	|*	|[2][S]\darr	|*	|*	|*	|*	|*	|*	|*	|*	|*	|*	|*	|*	|*	|*	|*	|*	|[3][S]\darr	|*	|.
|*	|*	|*	|m	|*	|g	|*	|*	|*	|*	|*	|*	|*	|*	|*	|*	|*	|*	|*	|*	|[4][S]\darr	|*	|m	|[5][S]\darr	|.
|*	|[6][S]\rarr	|m	|a	|p	|a	|[][,]{ }	|i	|z	|o	|p	|a	|c	|h	|y	|t	|o	|w	|a	|*	|p	|*	|o	|b	|.
|*	|[7][S]\rarr	|k	|r	|y	|p	|t	|o	|k	|o	|m	|u	|n	|i	|z	|m	|*	|*	|*	|*	|a	|[8][S]\darr	|r	|r	|.
|*	|*	|*	|t	|*	|a	|*	|*	|*	|*	|*	|*	|*	|*	|*	|*	|*	|*	|*	|*	|n	|r	|a	|o	|.
|*	|*	|*	|e	|*	|*	|[9][S]\darr	|*	|*	|*	|*	|*	|[10][S]\rarr	|n	|i	|k	|i	|e	|l	|i	|n	|a	|*	|ń	|.
|*	|[11][S]\drarr	|b	|a	|ł	|a	|b	|u	|c	|h	|*	|[12][S]\rarr	|u	|s	|ł	|u	|g	|o	|w	|i	|e	|c	|*	|[][,]{ }	|.
|*	|k	|*	|u	|*	|*	|a	|*	|*	|*	|[13][S]\drarr	|n	|a	|u	|k	|a	|*	|[14][S]\darr	|*	|*	|a	|h	|*	|k	|.
|*	|r	|*	|*	|*	|*	|s	|[15][S]\rarr	|n	|a	|p	|a	|l	|e	|n	|i	|e	|c	|*	|*	|u	|o	|*	|l	|.
|*	|ę	|[16][S]\rarr	|w	|ę	|z	|e	|ł	|[][,]{ }	|l	|i	|m	|f	|a	|t	|y	|c	|z	|n	|y	|*	|ń	|*	|a	|.
|[17][S]\drarr	|p	|r	|ó	|c	|h	|n	|i	|c	|z	|e	|k	|[][,]{ }	|b	|a	|g	|i	|e	|n	|n	|y	|*	|*	|s	|.
|z	|n	|*	|*	|*	|[18][S]\darr	|*	|*	|*	|*	|ś	|*	|*	|*	|[19][S]\rarr	|b	|y	|k	|*	|[20][S]\darr	|*	|[21][S]\darr	|*	|y	|.
|e	|i	|*	|[22][S]\darr	|[23][S]\drarr	|k	|o	|c	|h	|a	|ń	|s	|k	|i	|*	|*	|*	|[][,]{ }	|*	|r	|*	|w	|*	|c	|.
|s	|k	|[24][S]\drarr	|p	|i	|a	|n	|o	|l	|a	|*	|*	|*	|*	|*	|*	|*	|p	|[25][S]\darr	|a	|*	|a	|[26][S]\darr	|z	|.
|t	|[][,]{ }	|w	|r	|z	|m	|*	|*	|[27][S]\darr	|[28][S]\darr	|*	|*	|[29][S]\darr	|*	|*	|[30][S]\darr	|[31][S]\darr	|o	|r	|m	|[32][S]\darr	|g	|c	|n	|.
|a	|c	|a	|ą	|b	|i	|*	|*	|g	|d	|*	|*	|h	|*	|*	|j	|s	|d	|y	|i	|ś	|o	|h	|a	|.
|w	|i	|g	|d	|a	|z	|*	|*	|o	|y	|*	|[33][S]\darr	|a	|*	|[34][S]\drarr	|o	|t	|r	|z	|e	|w	|n	|a	|*	|.
|[][,]{ }	|e	|o	|[][,]{ }	|[][,]{ }	|e	|*	|*	|l	|d	|*	|b	|l	|*	|d	|l	|u	|ó	|y	|n	|i	|[][,]{ }	|r	|*	|.
|g	|m	|n	|e	|w	|l	|*	|*	|[][,]{ }	|k	|*	|e	|e	|*	|o	|*	|d	|ż	|k	|i	|ą	|s	|p	|*	|.
|ł	|n	|[][,]{ }	|l	|y	|k	|*	|[35][S]\drarr	|k	|a	|c	|z	|y	|z	|m	|*	|i	|n	|o	|c	|d	|p	|y	|*	|.
|o	|o	|s	|e	|t	|a	|*	|b	|o	|*	|[36][S]\darr	|b	|*	|[37][S]\darr	|i	|*	|u	|y	|[][,]{ }	|a	|[][,]{ }	|e	|*	|*	|.
|ś	|g	|t	|k	|r	|[][,]{ }	|*	|r	|n	|[38][S]\drarr	|s	|o	|n	|g	|n	|i	|m	|*	|f	|[][,]{ }	|p	|c	|*	|*	|.
|n	|ł	|e	|t	|z	|r	|*	|y	|t	|p	|c	|l	|*	|n	|a	|*	|*	|*	|i	|d	|ł	|j	|*	|*	|.
|o	|o	|r	|r	|e	|a	|*	|g	|a	|a	|h	|e	|*	|o	|n	|*	|[39][S]\darr	|[40][S]\darr	|n	|e	|y	|a	|*	|*	|.
|m	|w	|o	|y	|ź	|t	|*	|a	|k	|p	|e	|s	|*	|j	|t	|[41][S]\darr	|d	|f	|a	|l	|w	|l	|*	|*	|.
|ó	|y	|w	|c	|w	|u	|*	|n	|t	|i	|m	|n	|[42][S]\darr	|o	|a	|w	|z	|l	|n	|i	|a	|n	|*	|*	|.
|w	|*	|n	|z	|i	|n	|*	|t	|o	|l	|a	|o	|m	|w	|*	|y	|w	|i	|s	|k	|k	|y	|*	|*	|.
|i	|*	|i	|n	|e	|k	|*	|y	|w	|o	|t	|ś	|u	|i	|*	|r	|o	|s	|o	|a	|ó	|*	|*	|*	|.
|ą	|*	|c	|y	|ń	|o	|*	|n	|y	|t	|*	|ć	|c	|s	|*	|a	|n	|a	|w	|t	|w	|*	|*	|*	|.
|c	|*	|z	|*	|*	|w	|*	|a	|*	|k	|*	|*	|e	|k	|*	|k	|e	|k	|e	|n	|*	|*	|*	|*	|.
|y	|*	|y	|*	|*	|a	|*	|*	|*	|a	|*	|[43][S]\rarr	|t	|o	|p	|i	|k	|*	|*	|a	|*	|*	|*	|*	|.
|*	|*	|*	|*	|*	|*	|*	|*	|*	|*	|*	|*	|*	|*	|*	|*	|*	|*	|*	|*	|*	|*	|*	|*	|.\end{Puzzle}

\newpage

\begin{PuzzleClues}{\textbf{Poziome}\\}\Clue{6}{}{mapa ilustrująca zmiany miąższości warstw geologicznych}
\Clue{7}{}{poglądy głoszone przez polityków, którzy nie są komunistami, jednak bliskie komunizmowi}
\Clue{10}{}{stop miedzi (około 54\%), z niklem (około 26\%) i dodatkiem cynku lub magnezu}
\Clue{11}{}{pieróg ze słonym nadzieniem z ciasta drożdżowego, kojarzony z kuchnią kresową}
\Clue{12}{}{pracownik branży usługowej}
\Clue{13}{}{morał, przestroga, nauczka}
\Clue{15}{}{mężczyzna chętny do seksu, rozochocony emocjonalnie i erotycznie}
\Clue{16}{}{narząd leżący na przebiegu naczyń limfatycznych, filtrujący limfę i wytwarzający przeciwciała; element układu immunologicznego}
\Clue{17}{}{próchniczek błotny, mochwian błotny, Aulacomnium palustre - gatunek mchu z rodziny próchniczkowatych; występuje w Ameryce Północnej i Środkowej, Eurazji oraz Nowej Zelandii, dość pospolity na obszarze Polski, mech tworzący górą zielonożółte, dołem brązowe darnie, łodyżka wzniesiona, długości do 15 cm, barwy brązowawej, gęsto pokryta rdzawymi chwytnikami, listki lancetowate, długości 3-4 mm, skierowane ku górze, szczytowe wyraźnie zaostrzone, słabo ząbkowane, dolne brzegiem podwinięte, niekiedy na końcach zaokrąglone}
\Clue{19}{}{domowy samiec bydła domowego, bawołów, jaka, żubra, bizona itp}
\Clue{23}{}{skrzypek (1887-1934); wybitny wykonawca dzieł K. Szymanowskiego}
\Clue{24}{}{strunowy młoteczkowy (klawiszowy) instrument muzyczny posiadający mechanizm umożliwiający odtwarzanie utworów ze specjalnych rolek perforowanego papieru; utwór zapisany był na tych rolkach w postaci małych dziurek, przez które przedostawało się powietrze sterujące mechanizmem młoteczków instrumentu}
\Clue{34}{}{cienka, gładka błona surowicza, która wyściela ściany jamy brzusznej i miednicy oraz pokrywa całkowicie lub częściowo położone w tych jamach narządy}
\Clue{35}{}{negatywne określenie rządów braci Jarosława i Lecha Kaczyńskich}
\Clue{38}{}{miasto w KRL-D nad rzeką Tedong-gang; hutnictwo żelaza, przemysł chemiczny, wydobycie rud żelaza}
\Clue{43}{}{wymienny element bezpiecznika ulegający stopieniu przy zwarciu bądź przeciążeniu linii}\end{PuzzleClues}

\begin{PuzzleClues}{\textbf{Pionowe}\\}\Clue{1}{}{malarz francuski (1420-89) tworzył pod wpływem sztuki niderlandzkiej}
\Clue{2}{}{człowiek nierozgarnięty, niezbyt spostrzegawczy, nieostrożny}
\Clue{3}{}{tkanina jedwabna o falisto mieniącym się deseniu używana na suknie wieczorowe i szaty liturgiczne}
\Clue{4}{}{malowidło lub rzeźba, wypełniające pole, obramowane panneau architetonicznym}
\Clue{5}{}{wszystkie rodzaje broni, które nie są bronią masowego rażenia}
\Clue{8}{}{ur. w 1906 r., skrzypek i dyrygent, w latach 1945-1976; kierownik orkiestry Polskiego Radia w Warszawie}
\Clue{9}{}{sztuczny, obudowany zbiornik wodny, przeznaczony do pływania rekreacyjnego i uprawiania niektórych sportów wodnych}
\Clue{11}{}{Batis minor - gatunek ptaka z rodziny krępaczków (Platysteiridae)}
\Clue{13}{}{zazwyczaj podniosły utwór muzyczny z poetyckim tekstem}
\Clue{14}{}{rodzaj polecenia wypłaty, skierowanego przez bank do innych banków na standardową kwotę (np. 100 złotych, 100 euro, 50 funtów) w różnych walutach}
\Clue{17}{}{zespół urządzeń umożliwiający słyszenie rozmówcy telefonicznego i mówienie do niego na pewną (niewielką) odległość, bez konieczności trzymania telefonu przy uchu}
\Clue{18}{}{kamizelka utrzymująca człowieka przez określony czas (zależny od klasy wyporności kamizelki) na powierzchni wody, wypełniona lżejszym od wody materiałem}
\Clue{20}{}{Chara virgata - gatunek ramienicy występujący w większej części Europy (w tym w Bałtyku), na Grenlandii, Alasce i Nowej Zelandii}
\Clue{21}{}{wagon przeznaczony do obsługiwania technicznego infrastruktury i innych zadań}
\Clue{22}{}{uporządkowany ruch ładunków elektrycznych pod wpływem pola elektrycznego}
\Clue{23}{}{instytucja państwowa, stanowiąca miejsce przymusowego pobytu osób nietrzeźwych, zagrażających porządkowi publicznemu, do czasu ich wytrzeźwienia}
\Clue{24}{}{pojazd nietrakcyjny, wyposażony w kabinę maszynisty}
\Clue{25}{}{ryzyko związane ze strukturą kapitałową bilansu kredytobiorcy}
\Clue{26}{}{francuski metaloznawca i chemik (1865-1945); badania struktury stali i mosiądzów}
\Clue{27}{}{bramka zdobyta przez drużynę, która przegrywa różnicą dwóch bramek}
\Clue{28}{}{pierś w gwarze poznańskiej}
\Clue{29}{}{amerykański kompozytor i piosenkarz (1925-1981); pionier rock and rolla}
\Clue{30}{}{żaglowiec 2-masztowy z ożaglowaniem skośnym; także ożaglowanie takiego statku}
\Clue{31}{}{naukowe opracowanie jakiegoś zagadnienia}
\Clue{32}{}{stan zapalny skóry, wywołany wnikaniem cerkarii (postaci inwazyjnych) niektórych robaków Schistosoma podczas zanurzenia ciała w wodzie}
\Clue{33}{}{to, że coś jest bezbolesne - nie sprawia trudności, nie wymaga wysiłku}
\Clue{34}{}{motyw, bardziej widoczny niż inne w danej kompozycji, zespole, układzie}
\Clue{35}{}{rodzaj pancerza, skórzana kurtka z naszytymi lub przynitowanymi od wewnątrz płytkami metalowymi}
\Clue{36}{}{szablon, ustalony wzór, np. schemat myślowy}
\Clue{37}{}{składowisko gnoju mające postać dołu, wgłębienia albo rodzaju zbiornika, pojemnika}
\Clue{38}{}{tyle, ile zmieści się w papilotce - papierowej foremce do ciasta}
\Clue{39}{}{urządzenie elektroniczne zamontowane w czymś do wydawania sygnału}
\Clue{40}{}{przedstawiciel grupy zawodowej zajmującej się w dawnej Polsce flisem, czyli rzecznym spławem (transportem) towarów}
\Clue{41}{}{wyrakowate, tarsjusze, Tarsiidae - rodzina drapieżnych ssaków naczelnych, tradycyjnie zaliczanych - ze względu na zewnętrzne podobieństwo, zwłaszcza do galagowatych - do małpiatek, choć są bliżej spokrewnione z małpami i obecnie klasyfikowane w kladzie wyższych naczelnych (Haplorrhini); stanowią jedyną wśród ssaków naczelnych rodzinę o wyłącznie drapieżnym trybie życia; żyją na Sumatrze i Celebes (w Indonezji) oraz na Filipinach}
\Clue{42}{}{pelerynka sięgająca do łokci z małym kapturkiem w tyle noszona przez wyższych duchownych katolickich}\end{PuzzleClues}\newpage\section*{Krzyżówka 83}

\noindent\begin{Puzzle}{23}{33}|*	|[1][S]\drarr	|h	|e	|ł	|m	|*	|[2][S]\drarr	|l	|i	|s	|t	|[][,]{ }	|i	|n	|t	|e	|n	|c	|y	|j	|n	|y	|*	|.
|[3][S]\drarr	|p	|o	|j	|u	|t	|r	|z	|e	|*	|*	|[4][S]\drarr	|s	|i	|e	|c	|i	|a	|r	|k	|i	|*	|*	|[5][S]\darr	|.
|k	|a	|*	|*	|*	|[6][S]\darr	|*	|e	|*	|[7][S]\rarr	|h	|o	|m	|o	|l	|o	|g	|i	|a	|*	|*	|*	|*	|s	|.
|o	|s	|*	|*	|*	|j	|*	|p	|*	|[8][S]\drarr	|f	|r	|a	|n	|t	|*	|*	|*	|*	|*	|*	|[9][S]\darr	|*	|a	|.
|s	|i	|*	|[10][S]\darr	|*	|ę	|*	|p	|[11][S]\darr	|g	|[12][S]\rarr	|a	|c	|e	|r	|o	|l	|a	|*	|*	|*	|p	|*	|m	|.
|t	|e	|*	|r	|*	|z	|*	|e	|w	|e	|*	|w	|*	|*	|*	|*	|*	|*	|[13][S]\darr	|*	|*	|o	|*	|o	|.
|r	|r	|*	|ó	|*	|y	|*	|l	|i	|k	|*	|a	|*	|*	|*	|*	|*	|*	|i	|*	|[14][S]\darr	|w	|*	|o	|.
|z	|b	|*	|w	|[15][S]\rarr	|c	|h	|i	|n	|o	|l	|*	|*	|*	|*	|*	|[16][S]\darr	|*	|n	|*	|d	|i	|*	|p	|.
|e	|*	|*	|n	|*	|z	|*	|n	|o	|n	|[17][S]\darr	|*	|*	|*	|*	|*	|s	|[18][S]\darr	|w	|*	|y	|ą	|*	|o	|.
|w	|[19][S]\drarr	|k	|o	|r	|e	|k	|*	|d	|*	|p	|*	|*	|*	|[20][S]\darr	|[21][S]\darr	|p	|k	|e	|*	|s	|z	|*	|d	|.
|a	|s	|*	|n	|*	|k	|*	|*	|a	|[22][S]\darr	|ó	|*	|[23][S]\darr	|*	|r	|p	|a	|o	|s	|*	|f	|a	|*	|a	|.
|[][,]{ }	|o	|*	|ó	|[24][S]\darr	|[][,]{ }	|*	|*	|ń	|d	|ł	|*	|k	|*	|e	|t	|d	|r	|t	|[25][S]\darr	|u	|n	|*	|t	|.
|s	|p	|[26][S]\drarr	|g	|l	|u	|c	|k	|*	|e	|b	|*	|a	|*	|p	|a	|o	|a	|y	|u	|n	|i	|*	|k	|.
|z	|l	|w	|*	|o	|[][,]{ }	|*	|[27][S]\darr	|[28][S]\darr	|t	|ó	|*	|s	|*	|e	|k	|c	|l	|c	|k	|k	|e	|[29][S]\darr	|o	|.
|c	|i	|s	|*	|g	|w	|[30][S]\darr	|p	|m	|a	|g	|[31][S]\darr	|z	|*	|r	|i	|h	|o	|j	|ł	|c	|[][,]{ }	|d	|w	|.
|z	|c	|p	|[32][S]\rarr	|g	|a	|w	|i	|a	|l	|*	|s	|l	|[33][S]\darr	|t	|[][,]{ }	|r	|d	|a	|a	|j	|k	|e	|a	|.
|e	|a	|ó	|*	|i	|g	|o	|e	|j	|i	|[34][S]\darr	|e	|n	|u	|u	|r	|o	|z	|[][,]{ }	|d	|a	|a	|l	|n	|.
|c	|[][,]{ }	|l	|*	|a	|i	|l	|r	|ą	|s	|m	|r	|i	|k	|a	|a	|n	|i	|d	|[][,]{ }	|[][,]{ }	|p	|e	|i	|.
|i	|a	|n	|*	|*	|*	|i	|n	|t	|t	|i	|w	|ę	|ł	|r	|j	|i	|ó	|ł	|e	|s	|i	|g	|e	|.
|n	|u	|o	|[35][S]\drarr	|w	|ś	|c	|i	|e	|k	|l	|i	|c	|a	|[][,]{ }	|s	|a	|b	|u	|l	|e	|t	|a	|*	|.
|i	|s	|ś	|ż	|*	|*	|a	|k	|k	|a	|a	|t	|i	|d	|d	|k	|r	|*	|g	|e	|k	|a	|c	|*	|.
|a	|t	|ć	|y	|*	|*	|*	|a	|[][,]{ }	|*	|[][,]{ }	|u	|e	|[][,]{ }	|e	|i	|k	|[36][S]\darr	|o	|k	|s	|ł	|y	|*	|.
|s	|r	|[][,]{ }	|c	|*	|*	|*	|r	|r	|*	|l	|t	|*	|d	|b	|e	|a	|m	|t	|t	|u	|o	|j	|*	|.
|t	|a	|u	|i	|*	|*	|*	|s	|u	|*	|ą	|*	|*	|z	|i	|*	|*	|y	|e	|r	|a	|w	|k	|*	|.
|a	|l	|s	|o	|*	|*	|*	|t	|c	|*	|d	|*	|*	|i	|u	|*	|*	|d	|r	|o	|l	|e	|a	|*	|.
|*	|i	|t	|d	|*	|*	|*	|w	|h	|*	|o	|*	|*	|e	|t	|*	|*	|e	|m	|n	|n	|*	|*	|*	|.
|*	|j	|a	|a	|*	|*	|*	|o	|o	|*	|w	|*	|*	|s	|o	|*	|*	|l	|i	|i	|a	|*	|*	|*	|.
|*	|s	|w	|j	|*	|*	|*	|*	|m	|*	|a	|*	|*	|i	|w	|*	|*	|n	|n	|c	|*	|*	|*	|*	|.
|*	|k	|o	|n	|*	|*	|*	|*	|y	|*	|*	|*	|*	|ę	|y	|*	|*	|i	|o	|z	|*	|*	|*	|*	|.
|*	|a	|w	|o	|*	|*	|*	|*	|*	|*	|*	|*	|*	|t	|*	|*	|*	|c	|w	|n	|*	|*	|*	|*	|.
|*	|*	|a	|ś	|*	|*	|*	|*	|[37][S]\rarr	|k	|u	|z	|y	|n	|*	|*	|*	|z	|a	|y	|*	|*	|*	|*	|.
|*	|*	|*	|ć	|*	|*	|*	|*	|*	|*	|[38][S]\rarr	|z	|b	|y	|t	|n	|i	|k	|*	|*	|*	|*	|*	|*	|.
|*	|*	|*	|*	|*	|*	|*	|[39][S]\rarr	|c	|r	|u	|m	|b	|*	|*	|*	|*	|a	|*	|*	|*	|*	|*	|*	|.
|*	|*	|*	|*	|*	|*	|*	|*	|*	|*	|*	|*	|*	|*	|*	|*	|*	|*	|*	|*	|*	|*	|*	|*	|.\end{Puzzle}

\newpage

\begin{PuzzleClues}{\textbf{Poziome}\\}\Clue{1}{}{bojowa ochrona głowy, chroniąca czaszkę przed urazami, sporządzona z odpornego materiału}
\Clue{2}{}{pisemna, wstępna deklaracja zamierzeń dotyczących swoich działań, składana przez stronę możliwego przyszłego stosunku gospodarczego}
\Clue{3}{}{dzień następujący po dniu jutrzejszym}
\Clue{4}{}{siatkoskrzydłe: rząd ciepłolubnych owadów, drapieżne}
\Clue{7}{}{odpowiedniość między dwoma dowolnymi elementami, zgodność}
\Clue{8}{}{wędrowny komediant}
\Clue{12}{}{Malpighia glabra, malpigia granatolistna - gatunek rośliny z rodziny malpigiowatych}
\Clue{15}{}{pogardliwie, obraźliwie: Chińczyk}
\Clue{19}{}{materiał wytwarzany z kory dębu korkowego lub korkowca amurskiego}
\Clue{26}{}{kompozytor niemiecki (1714-1787); opery 'Orfeusz i Eurydyka', 'Alcesta' 'Ifigenia na Taurydzie' balety, kantaty, pieśni}
\Clue{32}{}{rybożerny krokodyl indyjski}
\Clue{35}{}{Myrmica sabuleti - gatunek mrówki z podrodziny wścieklic; gatunek europejski, występuje pospolicie w lasach i na przestrzeniach otwartych w środowiskach ciepłych}
\Clue{37}{}{dalszy członek rodziny}
\Clue{38}{}{psotnik, ktoś, kto robi drobne żarty}
\Clue{39}{}{kompozytor amerykański ur. 1929 r., utwory wokalno-instrumentalne kameralne, fortepianowe}\end{PuzzleClues}

\begin{PuzzleClues}{\textbf{Pionowe}\\}\Clue{1}{}{boczny, przedwcześnie wyrosły pęd}
\Clue{2}{}{rodzaj sterowca, skonstruowany przez Ferdynanda Zeppelina}
\Clue{3}{}{Festuca trachyphylla - gatunek trawy z rodziny wiechlinowatych}
\Clue{4}{}{kraina w Słowacji i w Polsce, w dorzeczu rzeki Orawy}
\Clue{5}{}{dobrowolny podatek}
\Clue{6}{}{w wagach starego typu (nieelektronicznych) - drobny element, wypustka, która wskazuje na przechylenie się wagi na jedną lub drugą szalę }
\Clue{8}{}{owadożerna jaszczurka; palce z przylgami umożliwiające wspinanie się, głos donośny}
\Clue{9}{}{sytuacja, w której jedna z osób lub jeden z kontrahentów posiada lub dysponuje, bezpośrednio lub pośrednio, prawem głosu wynoszącym co najmniej 5\% wszystkich praw głosu}
\Clue{10}{}{skorupiak morski z rzędu równonogów}
\Clue{11}{}{Borassus - rodzaj drzewa z rodziny arekowatych}
\Clue{13}{}{inwestycja, której czas realizacji przekracza rok, np. inwestycje w nieruchomości, wartości niematerialne i prawne lub długotrwałe aktywa finansowe}
\Clue{14}{}{zaburzenie na tle seksualnym, które polega na nieprawidłowym przebiegu rekacji seksualnych}
\Clue{16}{}{sportsmenka zajmująca się spadochroniarstwem zawodowo lub hobbystycznie}
\Clue{17}{}{w mitologii greckiej postać zrodzona ze związku człowieka i boga, półbóg o nadprzyrodzonych zdolnościach}
\Clue{18}{}{Hypositta corallirostris - gatunek ptaka z rodziny wang (Vangidae)}
\Clue{19}{}{agatis nowozelandzki, soplica kauri, soplica południowa, Agathis australis - gatunek drzewa z rodziny araukariowatych; występuje na Wyspie Północnej w Nowej Zelandii}
\Clue{20}{}{zestaw otwarć, które szachista gra najczęściej}
\Clue{21}{}{rodzina ptaków z rzędu wróblowych (Passeriformes)}
\Clue{22}{}{handlarka detaliczna}
\Clue{23}{}{pojedynczy odgłos wydawany przy kaszlu}
\Clue{24}{}{odrębny budynek lub pomieszczenie otwierające się arkadami na zewnątrz}
\Clue{25}{}{zbiór elementów elektronicznych dyskretnych lub scalonych połączonych elektrycznie tak, aby realizowały określoną funkcję}
\Clue{26}{}{wspólność majątkowa małżonków, która powstaje z chwilą zawarcia małżeństwa, chyba że małżonkowie zawarli wcześniej małżeńską umowę majątkową}
\Clue{27}{}{wyrób i dystrybucja pierników}
\Clue{28}{}{czyjaś własność, która daje się przenosić}
\Clue{29}{}{zdrobniale: delegacja - grupa ludzi, która jest dokądś przez kogoś wysłana w określonym celu}
\Clue{30}{}{poeta, działacz KPP (1909-40), członek „Kwadrygi”; „Młoty w dłoniach”, „Ulica Dorodowa”, „Z kamiennego domu”}
\Clue{31}{}{uprawnienie do użytkowania czyjejś nieruchomości}
\Clue{33}{}{pozycyjny system liczbowy, w którym podstawą pozycji są kolejne wielokrotności liczby 10, a do zapisu potrzeba 10 cyfr (0, 1, 2, 3, 4, 5, 6, 7, 8, 9)}
\Clue{34}{}{pozaukładowa jednostka odległości stosowana w krajach anglosaskich, równa 1609,344 metry}
\Clue{35}{}{cecha czegoś, co jest życiodajne}
\Clue{36}{}{pojemnik, pudełko do przechowywania mydła w kostce, zwłaszcza w podróży}\end{PuzzleClues}\newpage\section*{Krzyżówka 84}

\noindent\begin{Puzzle}{25}{30}|*	|*	|*	|*	|*	|*	|*	|*	|*	|*	|[1][S]\drarr	|e	|c	|h	|i	|n	|o	|c	|e	|r	|e	|u	|s	|*	|*	|*	|.
|*	|[2][S]\drarr	|a	|b	|s	|o	|l	|u	|t	|y	|z	|a	|c	|j	|a	|*	|*	|*	|*	|*	|*	|*	|*	|*	|*	|*	|.
|*	|d	|*	|*	|[3][S]\rarr	|c	|i	|a	|s	|t	|o	|[][,]{ }	|b	|i	|s	|z	|k	|o	|p	|t	|o	|w	|e	|*	|*	|*	|.
|*	|r	|*	|*	|*	|*	|*	|[4][S]\rarr	|t	|u	|b	|i	|e	|t	|i	|e	|j	|k	|a	|*	|*	|*	|*	|*	|*	|*	|.
|*	|o	|*	|*	|*	|*	|*	|*	|[5][S]\rarr	|r	|o	|g	|a	|l	|s	|k	|i	|*	|*	|*	|*	|*	|*	|*	|*	|*	|.
|*	|g	|[6][S]\rarr	|s	|z	|y	|s	|z	|k	|o	|w	|i	|e	|c	|[][,]{ }	|z	|w	|y	|c	|z	|a	|j	|n	|y	|*	|*	|.
|*	|a	|*	|*	|*	|[7][S]\drarr	|t	|e	|o	|r	|i	|a	|[][,]{ }	|p	|e	|r	|s	|p	|e	|k	|t	|y	|w	|y	|*	|*	|.
|*	|*	|*	|[8][S]\darr	|[9][S]\darr	|h	|*	|*	|[10][S]\rarr	|ł	|ą	|c	|z	|n	|i	|k	|[][,]{ }	|g	|w	|i	|n	|t	|o	|w	|y	|*	|.
|*	|[11][S]\darr	|*	|s	|n	|e	|*	|[12][S]\drarr	|p	|r	|z	|e	|m	|y	|s	|ł	|ó	|w	|k	|a	|*	|*	|*	|*	|*	|*	|.
|[13][S]\drarr	|t	|ę	|p	|o	|l	|i	|s	|t	|k	|a	|[][,]{ }	|w	|o	|d	|n	|a	|*	|*	|[14][S]\darr	|*	|*	|*	|*	|*	|*	|.
|k	|e	|*	|y	|r	|s	|*	|i	|*	|*	|n	|*	|*	|*	|*	|[15][S]\drarr	|r	|u	|s	|a	|j	|f	|a	|*	|*	|*	|.
|r	|o	|*	|c	|m	|i	|*	|l	|*	|*	|i	|*	|*	|*	|*	|k	|*	|*	|*	|l	|*	|*	|*	|*	|[16][S]\darr	|*	|.
|o	|r	|*	|h	|o	|n	|[17][S]\darr	|n	|*	|*	|e	|[18][S]\rarr	|g	|a	|w	|r	|o	|n	|*	|a	|*	|*	|[19][S]\darr	|*	|o	|*	|.
|k	|i	|*	|a	|k	|k	|b	|i	|*	|[20][S]\darr	|[][,]{ }	|*	|*	|*	|*	|o	|*	|[21][S]\darr	|*	|n	|*	|*	|z	|*	|b	|*	|.
|o	|a	|*	|r	|s	|i	|e	|k	|*	|c	|w	|*	|[22][S]\darr	|*	|[23][S]\darr	|c	|*	|a	|*	|i	|*	|*	|n	|*	|r	|*	|.
|d	|[][,]{ }	|*	|k	|e	|*	|z	|[][,]{ }	|*	|e	|y	|[24][S]\drarr	|p	|ę	|c	|h	|e	|r	|z	|n	|i	|c	|a	|*	|o	|*	|.
|y	|i	|*	|a	|m	|*	|c	|w	|*	|o	|m	|o	|o	|*	|e	|m	|*	|n	|*	|a	|*	|*	|n	|*	|t	|*	|.
|l	|n	|*	|*	|i	|[25][S]\darr	|z	|o	|[26][S]\rarr	|w	|a	|r	|d	|e	|n	|a	|f	|i	|l	|*	|*	|[27][S]\darr	|o	|*	|ó	|*	|.
|[][,]{ }	|f	|*	|*	|a	|g	|a	|d	|[28][S]\darr	|n	|g	|i	|s	|*	|t	|l	|*	|*	|*	|*	|[29][S]\darr	|i	|ś	|*	|w	|*	|.
|j	|o	|*	|*	|*	|a	|s	|n	|d	|i	|a	|o	|k	|*	|u	|n	|*	|*	|*	|*	|k	|s	|ć	|[30][S]\darr	|k	|*	|.
|o	|r	|*	|*	|*	|z	|z	|y	|z	|k	|l	|n	|o	|*	|r	|i	|[31][S]\drarr	|ż	|u	|p	|a	|n	|*	|m	|a	|*	|.
|h	|m	|*	|*	|*	|i	|k	|*	|i	|*	|n	|*	|k	|[32][S]\darr	|i	|a	|t	|[33][S]\rarr	|s	|p	|r	|a	|w	|a	|*	|*	|.
|n	|a	|[34][S]\drarr	|m	|i	|k	|o	|ł	|a	|j	|e	|k	|*	|a	|a	|*	|c	|*	|*	|*	|e	|*	|*	|l	|*	|*	|.
|s	|c	|n	|*	|*	|*	|w	|*	|ł	|*	|*	|*	|*	|h	|*	|*	|*	|*	|*	|*	|n	|*	|*	|u	|*	|*	|.
|t	|j	|a	|*	|*	|[35][S]\drarr	|c	|i	|a	|ł	|o	|*	|*	|l	|*	|*	|[36][S]\rarr	|s	|e	|r	|c	|e	|*	|c	|*	|*	|.
|o	|i	|s	|*	|*	|t	|e	|*	|n	|[37][S]\rarr	|s	|k	|u	|b	|a	|n	|i	|e	|c	|*	|j	|*	|*	|h	|*	|*	|.
|n	|*	|e	|*	|*	|o	|*	|*	|i	|*	|*	|*	|*	|e	|[38][S]\rarr	|s	|t	|ó	|j	|k	|a	|*	|*	|*	|*	|*	|.
|a	|[39][S]\rarr	|r	|ó	|ż	|n	|o	|w	|a	|r	|t	|o	|ś	|c	|i	|o	|w	|o	|ś	|ć	|*	|*	|*	|*	|*	|*	|.
|*	|*	|*	|*	|*	|a	|*	|*	|*	|[40][S]\rarr	|t	|ę	|s	|k	|n	|o	|t	|a	|*	|*	|*	|*	|*	|*	|*	|*	|.
|*	|*	|*	|*	|*	|ż	|*	|*	|*	|*	|*	|*	|*	|*	|*	|*	|*	|*	|*	|*	|*	|*	|*	|*	|*	|*	|.
|*	|*	|*	|*	|*	|*	|*	|*	|*	|*	|*	|*	|*	|*	|*	|*	|*	|*	|*	|*	|*	|*	|*	|*	|*	|*	|.\end{Puzzle}

\newpage

\begin{PuzzleClues}{\textbf{Poziome}\\}\Clue{1}{}{meksykański kaktus podobny do cereusa o różnobarwnych kwiatach, owoce niektórych gatunków jadalne}
\Clue{2}{}{proces myślowy, w którym pewnym rzeczom (zjawiskom) przyznaje się wyjątkowy status, uzasadniając wypieranie przez nie innych podobnych rzeczy lub zjawisk}
\Clue{3}{}{rodzaj lekkiego ciasta przyrządzanego na bazie jaj z małym dodatkiem mąki pszennej i cukru - bez dodatku tłuszczu}
\Clue{4}{}{czapka noszona przez ludy Wschodu}
\Clue{5}{}{aerodynamik, konstruktor samolotów (1904-1976), wraz z Drzewieckim i Wigurą budował samoloty RWD}
\Clue{6}{}{Cordylus cordylus - gatunek gada z rodziny szyszkowcowatych, występujący w południowej Afryce}
\Clue{7}{}{teoria psychologiczna autorstwa Daniela Kahnemana i Amosa Tversky'ego, tłumacząca podejmowanie przez ludzi decyzji w warunkach ryzyka}
\Clue{10}{}{element konstrukcyjny z gwintem służący do łączenia elementów}
\Clue{12}{}{szkoła przemysłowa}
\Clue{13}{}{Codriophorus aquaticus - gatunek mchu z rodziny strzechwowatych}
\Clue{15}{}{ośrodek wydobycia fosforytów i gipsu w płn. Jordanii}
\Clue{18}{}{Corvus frugilegus - gatunek średniej wielkości ptaka synantropijnego z rodziny krukowatych (Corvidae)}
\Clue{24}{}{Physocarpus - rodzaj roślin należący do rodziny różowatych (Rosaceae)}
\Clue{26}{}{inhibitor fosfodiesterazy typu 5, wykorzystywany w leczeniu zaburzeń erekcji}
\Clue{31}{}{staropolski ubiór męski w formie długiej sukni zapinanej na haftki ze stojącym kołnierzem i wąskimi rękawami noszony przez szlachtę od XVI do XIX w}
\Clue{33}{}{postępowanie sądowe prowadzone przez sąd, którego celem jest ustalenie faktów i wymierzenie sprawiedliwości}
\Clue{34}{}{bylina z baldaszkowatych o niebieskawych, kolczastych liściach i główkowatych kwiatostanach - chroniona}
\Clue{35}{}{martwe ciało ludzkie lub (rzadziej) zwierzęce}
\Clue{36}{}{coś, co ma kształt (w potocznym wyobrażeniu) serca}
\Clue{37}{}{z podziwem o kimś, kto świetnie sobie radzi, jest sprytny, utalentowany, budzi zazdrość}
\Clue{38}{}{stojący kołnierz np. przy koszuli}
\Clue{39}{}{cecha funkcji różnowartościowej}
\Clue{40}{}{uczucie braku czegoś lub kogoś}\end{PuzzleClues}

\begin{PuzzleClues}{\textbf{Pionowe}\\}\Clue{1}{}{zobowiązanie, którego termin płatności minął, a które nie zostało przedawnione lub umorzone; termin prawny}
\Clue{2}{}{wydzielony pas ziemi, przystosowany do komunikacji}
\Clue{7}{}{miasto letnich Igrzysk Olimpijskich w 1952 r}
\Clue{8}{}{buldożer, ciągnik zaopatrzony w lemiesz do odspajania i przemieszczania gruntu, wyrównywania terenu}
\Clue{9}{}{prawidłowa zawartość tlenu parcjalnego (PaO2) we krwi tętniczej}
\Clue{11}{}{dyscyplina zajmująca się problematyką informacji oraz metodami przetwarzania informacji, np. w celu transmisji lub kompresji}
\Clue{12}{}{rodzaj hydraulicznego silnika przetwarzającego energię wody płynącej na pracę mechaniczną}
\Clue{13}{}{Crocodylus johnsoni - gatunek gada z rodziny krokodylowatych, występujący na północy Australii}
\Clue{14}{}{organiczny związek chemiczny z grupy ?-aminokwasów}
\Clue{15}{}{zakład przemysłu rolno-spożywczego, przerabiający surowiec (rośliny bogate w skrobię) na produkt - mączkę skrobiową, zwaną potocznie krochmalem}
\Clue{16}{}{księga handlowa, w której zawarte są zestawienia obrotów, sprawozdania i bilanse za krótki okres}
\Clue{17}{}{strunogłowe, Cephalochordata - niewielkie zwierzęta przybrzeżnej strefy ciepłych mórz o dużym zasoleniu; żyją zagrzebane na dnie, ciało długości 6-8 cm, bocznie spłaszczone, o wydłużnym, lancetowatym kształcie}
\Clue{19}{}{to, że ktoś jest znany, rozpoznawalny, wielu ludzi coś o nim wie}
\Clue{20}{}{kształtownik walcowany o kształcie przekroju zbliżonym do litery C}
\Clue{21}{}{bawół indyjski, bawół wodny, Bubalus arnee - duży ssak z rodziny krętorogich, zwierzę zaadaptowane do klimatu tropikalnego; zamieszkuje Bhutan, Indie, Nepal i Tajlandię}
\Clue{22}{}{skok do góry}
\Clue{23}{}{tysiącznik, Centaurium - rodzaj roślin z rodziny goryczkowatych}
\Clue{24}{}{gwiazdozbiór równikowy: zawiera widoczną gołym okiem nieregularną mgławicę}
\Clue{25}{}{dawniej potocznie o samochodzie terenowym}
\Clue{27}{}{egipskie miasto nad Nilem}
\Clue{28}{}{PLUR. walka zbrojna, ruch wojsk}
\Clue{29}{}{ustalony okres przerwy, po której zawodnik występujący w jednym klubie może wystąpić w barwach innego}
\Clue{30}{}{dziecko, brzdąc}
\Clue{31}{}{w chemii: symbol technetu}
\Clue{32}{}{dzielnica gminy Heringsdorf w Niemczech, w Meklemburgii-Pomorzu Przednim, w powiecie Vorpommern-Greifswald, na wyspie Uznam}
\Clue{34}{}{egipski oficer i polityk, prezydent Egiptu, sprawujący władzę autorytarną w latach 1954-1970}
\Clue{35}{}{wyrażona w tonach ilość dóbr poddawanych obrotowi}\end{PuzzleClues}\newpage\section*{Krzyżówka 85}

\noindent\begin{Puzzle}{25}{21}|*	|*	|[1][S]\drarr	|p	|a	|r	|e	|m	|i	|o	|g	|r	|a	|f	|*	|*	|[2][S]\darr	|*	|*	|*	|*	|*	|[3][S]\darr	|*	|*	|*	|.
|[4][S]\rarr	|i	|n	|t	|e	|r	|m	|e	|d	|i	|u	|m	|*	|*	|*	|[5][S]\drarr	|b	|a	|r	|d	|*	|*	|s	|*	|*	|*	|.
|*	|*	|a	|*	|[6][S]\darr	|*	|*	|*	|*	|*	|*	|[7][S]\darr	|*	|*	|[8][S]\rarr	|p	|i	|e	|t	|r	|u	|s	|z	|k	|a	|*	|.
|*	|*	|n	|[9][S]\drarr	|e	|k	|i	|e	|r	|*	|*	|m	|[10][S]\darr	|[11][S]\rarr	|n	|i	|e	|w	|i	|a	|s	|t	|k	|a	|*	|*	|.
|*	|*	|o	|ś	|l	|*	|*	|*	|*	|*	|*	|i	|t	|*	|*	|s	|l	|*	|*	|*	|*	|*	|o	|*	|*	|*	|.
|*	|*	|t	|w	|e	|*	|*	|*	|*	|*	|*	|c	|z	|*	|*	|m	|i	|*	|[12][S]\darr	|*	|*	|[13][S]\darr	|ł	|*	|*	|*	|.
|*	|*	|y	|i	|k	|*	|*	|*	|*	|*	|[14][S]\rarr	|h	|a	|r	|m	|o	|n	|i	|z	|a	|c	|j	|a	|*	|*	|*	|.
|*	|*	|r	|ę	|t	|*	|*	|[15][S]\rarr	|e	|s	|m	|a	|r	|c	|h	|*	|k	|*	|i	|*	|*	|e	|*	|*	|*	|*	|.
|[16][S]\rarr	|p	|a	|t	|r	|y	|c	|j	|u	|s	|z	|*	|a	|*	|*	|*	|o	|*	|e	|*	|*	|l	|*	|*	|*	|*	|.
|*	|*	|n	|o	|o	|*	|*	|*	|*	|*	|*	|*	|*	|*	|*	|*	|w	|*	|l	|*	|*	|i	|*	|*	|*	|*	|.
|*	|*	|*	|*	|d	|*	|[17][S]\darr	|*	|*	|*	|*	|*	|*	|*	|[18][S]\rarr	|m	|a	|h	|o	|m	|e	|t	|*	|*	|*	|*	|.
|*	|*	|*	|*	|i	|*	|d	|*	|*	|*	|*	|*	|*	|*	|*	|*	|t	|*	|n	|*	|*	|ó	|*	|*	|*	|*	|.
|*	|[19][S]\rarr	|f	|l	|a	|k	|o	|n	|i	|k	|*	|*	|*	|*	|*	|*	|e	|*	|y	|*	|*	|w	|*	|*	|*	|*	|.
|*	|*	|*	|*	|g	|*	|j	|*	|*	|*	|*	|*	|*	|*	|*	|*	|*	|*	|[][,]{ }	|*	|*	|k	|*	|*	|*	|*	|.
|[20][S]\rarr	|h	|a	|r	|n	|a	|ś	|*	|*	|[21][S]\rarr	|k	|a	|w	|a	|[][,]{ }	|m	|i	|e	|l	|o	|n	|a	|*	|*	|*	|*	|.
|*	|*	|*	|*	|o	|*	|c	|*	|*	|*	|*	|*	|*	|*	|*	|*	|*	|*	|u	|*	|*	|*	|*	|*	|*	|*	|.
|*	|*	|*	|*	|s	|*	|i	|*	|*	|*	|*	|*	|*	|*	|*	|*	|*	|*	|d	|*	|*	|*	|*	|*	|*	|*	|.
|[22][S]\rarr	|b	|r	|i	|t	|t	|e	|n	|*	|*	|*	|*	|*	|[23][S]\rarr	|p	|t	|a	|s	|z	|y	|n	|i	|e	|c	|*	|*	|.
|*	|*	|*	|*	|y	|*	|*	|*	|*	|*	|*	|*	|*	|*	|*	|*	|*	|*	|i	|*	|*	|*	|*	|*	|*	|*	|.
|*	|*	|*	|*	|k	|*	|*	|*	|*	|*	|*	|*	|*	|*	|*	|*	|*	|*	|k	|*	|*	|*	|*	|*	|*	|*	|.
|*	|*	|[24][S]\rarr	|f	|a	|c	|e	|c	|j	|o	|n	|i	|s	|t	|a	|*	|*	|*	|*	|*	|*	|*	|*	|*	|*	|*	|.
|*	|*	|*	|*	|*	|*	|*	|*	|*	|*	|*	|*	|*	|*	|*	|*	|*	|*	|*	|*	|*	|*	|*	|*	|*	|*	|.\end{Puzzle}

\newpage

\begin{PuzzleClues}{\textbf{Poziome}\\}\Clue{1}{}{osoba zajmująca się zbieraniem i komentowaniem przysłów}
\Clue{4}{}{INTERMEZZO}
\Clue{5}{}{celtycki poeta i pieśniarz, spotykany od XI do XVII wieku na dworach Irlandii, Szkocji, Walii i Bretanii}
\Clue{8}{}{dwuletnia roślina warzywna i przyprawowa z rodziny baldaszkowatych pochodzenia śródziemnomorskiego}
\Clue{9}{}{ur. w 1913 r., pianista, profesor A.M. w Warszawie}
\Clue{11}{}{synowa, żona syna}
\Clue{14}{}{wzajemne dopasowanie poszczególnych struktur i elementów funkcjonowania państwa}
\Clue{15}{}{chirurg niemiecki (1823-1908); opracował metodę operacji kończyn z założeniem tzw. opaski Esmarcha (opaska uciskająca)}
\Clue{16}{}{honorowy tytuł, który przyznawany jest wysokim urzędnikom albo wodzom zaprzyjaźnionych państw}
\Clue{18}{}{(570-632) twórca i prorok Islamu}
\Clue{19}{}{mały, wysoki i niezbyt pękaty szklany wazon}
\Clue{20}{}{podhalańskie określenie herszta bandy, nadawane przywódcom karpackich zbójników}
\Clue{21}{}{napój powstały po zalaniu wrzątkiem zmielonych ziaren kawy}
\Clue{22}{}{kompozytor angielski (1913-1976); koncerty, utwory orkiestrowe, kameralne, chóralne, opera 'Peter Grimes'}
\Clue{23}{}{Ornithopus - rodzaj rośliny z rodziny bobowatych, kwitnie na różowo, znajduje zastosowanie jako składnik pasz}
\Clue{24}{}{pisarz tworzący facecje}\end{PuzzleClues}

\begin{PuzzleClues}{\textbf{Pionowe}\\}\Clue{1}{}{Nanotyrannus - rodzaj dinozaura z rodziny tyranozaurów; żył w późnej kredzie na terenach Ameryki Północnej}
\Clue{2}{}{Pieridae - rodzina motyli licząca ok. 1000 gatunków; w Polsce występuje 17; jej przedstawiciele są spotykani na całym świecie; są to motyle o różnorodnych rozmiarach: od niewielkich (2,5 cm) jak np. rodzaj Nothalis do dużych jak azjatycki rodzaj Hebomoia}
\Clue{3}{}{nauczka, przykre zdarzenie, które daje do myślenia, często: kara za coś}
\Clue{5}{}{umiejętność wyrażania myśli znakami graficznymi, pisanie}
\Clue{6}{}{metoda badania za pomocą prądu}
\Clue{7}{}{gęba, zwykle gdy okrągła, uśmiechnięta}
\Clue{9}{}{zdarzenie, uroczystość, dzień inny niż zwykle; świętowanie}
\Clue{10}{}{(1896-1963), poeta francuski rumuńskiego pochodzenia, inicjator i przywódca dadaizmu}
\Clue{12}{}{ufoludek, istota z innej planety}
\Clue{13}{}{wirusowe zapalenie przewodu pokarmowego, choroba zakaźna}
\Clue{17}{}{osiągnąć granicę, kres}\end{PuzzleClues}\newpage\section*{Krzyżówka 86}

\noindent\begin{Puzzle}{24}{19}|*	|[1][S]\drarr	|s	|ą	|d	|*	|[2][S]\darr	|*	|*	|[3][S]\drarr	|a	|m	|o	|n	|i	|t	|*	|[4][S]\drarr	|p	|a	|i	|ź	|a	|*	|*	|.
|*	|n	|*	|*	|*	|*	|n	|*	|*	|p	|*	|*	|[5][S]\drarr	|m	|i	|k	|r	|o	|g	|r	|a	|f	|i	|a	|*	|.
|*	|i	|[6][S]\rarr	|x	|i	|x	|i	|a	|*	|r	|*	|*	|w	|*	|*	|[7][S]\rarr	|a	|b	|l	|a	|c	|j	|a	|*	|[8][S]\darr	|.
|*	|e	|*	|*	|[9][S]\darr	|*	|e	|[10][S]\rarr	|b	|e	|z	|w	|y	|s	|i	|ł	|k	|o	|w	|o	|ś	|ć	|*	|*	|b	|.
|*	|d	|*	|[11][S]\drarr	|o	|r	|k	|a	|*	|l	|[12][S]\darr	|*	|d	|*	|*	|*	|[13][S]\rarr	|w	|a	|t	|t	|*	|*	|[14][S]\darr	|i	|.
|[15][S]\drarr	|o	|f	|e	|r	|t	|a	|[][,]{ }	|k	|u	|p	|n	|a	|*	|*	|*	|*	|i	|[16][S]\darr	|*	|*	|[17][S]\darr	|*	|m	|e	|.
|p	|p	|[18][S]\darr	|w	|z	|*	|p	|*	|*	|d	|o	|*	|t	|*	|*	|[19][S]\rarr	|s	|ą	|c	|z	|e	|k	|*	|a	|g	|.
|ó	|r	|t	|a	|e	|*	|e	|*	|*	|i	|d	|[20][S]\rarr	|k	|e	|t	|*	|*	|z	|h	|*	|*	|o	|*	|s	|u	|.
|ł	|z	|k	|n	|ł	|[21][S]\darr	|k	|*	|*	|u	|a	|*	|i	|*	|*	|*	|*	|e	|o	|*	|*	|r	|*	|k	|s	|.
|c	|ę	|i	|i	|*	|d	|*	|*	|*	|m	|ż	|*	|[][,]{ }	|*	|*	|[22][S]\rarr	|s	|k	|r	|z	|y	|d	|ł	|o	|*	|.
|i	|d	|b	|e	|*	|u	|*	|*	|*	|*	|*	|[23][S]\rarr	|r	|ó	|j	|k	|a	|*	|r	|*	|[24][S]\darr	|*	|*	|t	|*	|.
|ą	|*	|u	|l	|[25][S]\darr	|s	|*	|*	|*	|[26][S]\rarr	|b	|r	|z	|m	|i	|e	|n	|i	|e	|*	|t	|*	|*	|k	|*	|.
|g	|[27][S]\drarr	|l	|i	|c	|z	|n	|i	|k	|[][,]{ }	|c	|i	|e	|p	|ł	|a	|*	|*	|r	|*	|o	|*	|[28][S]\darr	|a	|*	|.
|ł	|s	|i	|a	|h	|a	|*	|*	|[29][S]\rarr	|p	|s	|y	|c	|h	|o	|s	|o	|m	|a	|t	|y	|k	|a	|*	|*	|.
|o	|m	|*	|*	|a	|*	|*	|[30][S]\rarr	|s	|k	|o	|r	|z	|o	|n	|e	|r	|a	|*	|*	|o	|*	|n	|*	|*	|.
|ś	|a	|[31][S]\rarr	|i	|b	|i	|s	|[][,]{ }	|b	|r	|ą	|z	|o	|w	|a	|w	|y	|*	|*	|*	|t	|*	|h	|*	|*	|.
|ć	|r	|*	|*	|l	|[32][S]\rarr	|t	|ę	|s	|k	|l	|i	|w	|o	|ś	|ć	|*	|*	|*	|*	|a	|*	|u	|*	|*	|.
|*	|k	|*	|*	|i	|*	|*	|[33][S]\rarr	|k	|i	|s	|i	|e	|l	|e	|w	|s	|k	|i	|*	|*	|*	|i	|*	|*	|.
|*	|*	|*	|*	|s	|*	|*	|[34][S]\rarr	|z	|o	|n	|n	|*	|*	|*	|*	|*	|*	|*	|*	|*	|*	|*	|*	|*	|.
|*	|*	|*	|*	|*	|*	|*	|*	|*	|*	|*	|*	|*	|*	|*	|*	|*	|*	|*	|*	|*	|*	|*	|*	|*	|.\end{Puzzle}

\newpage

\begin{PuzzleClues}{\textbf{Poziome}\\}\Clue{1}{}{zdanie, opinia o czymś, ocena czegoś, pogląd, przekonanie}
\Clue{3}{}{kopalny głowonóg o spiralnej muszli}
\Clue{4}{}{krótka, metalowa tarcza, używana przez konnych rycerzy}
\Clue{5}{}{opis bardzo drobnych obiektów obserwowanych za pomocą mikroskopu}
\Clue{6}{}{miasto w Chinach na Półwyspie Szantuńskim}
\Clue{7}{}{zabieg kardiologiczny, który wykonywany jest w celu trwałego wyleczenie rodzaju zaburzenia rytmu serca nazywanego częstoskurczem}
\Clue{10}{}{cecha czegoś, co nie wymaga wysiłku, nie wiąże się z wysiłkiem}
\Clue{11}{}{miecznik, Orcinus orca - gatunek walenia z rodziny delfinowatych, największy przedstawiciel delfinowatych, jedyny przedstawiciel rodzaju Orcinus; zamieszkuje wszystkie oceany i większe morza (preferuje zwłaszcza zimne wody, ale występuje i w ciepłych)}
\Clue{13}{}{szkocki konstruktor i wynalazca (1736-1819); zbudował parowy silnik z regulatorem prędkości obrotowej}
\Clue{15}{}{znaczna dysproporcja między popytem a podażą papierów wartościowych w obrocie (z przewagą popytu)}
\Clue{19}{}{przedmiot laboratoryjny wykonany z bibułki, służący do odsączania ciała stałego}
\Clue{20}{}{jednostka pływająca o napędzie żaglowym, posiadająca tylko jeden maszt (grot), oraz tylko jeden żagiel (również grot) - dowolnego typu ożaglowania skośnego}
\Clue{22}{}{w terminologii sportowej określenie bocznej części boiska}
\Clue{23}{}{gromadny lot godowy różnych owadów}
\Clue{26}{}{bycie słyszanym}
\Clue{27}{}{służy do pomiaru ilości ciepła używanego np. na cele grzewcze}
\Clue{29}{}{całościowe ujmowanie problemów człowieka chorego}
\Clue{30}{}{Scorzonera hispanica - gatunek rośliny warzywnej z rodziny astrowatych}
\Clue{31}{}{Bostrychia bocagei - gatunek ptaka z rodziny ibisowatych (Threskiornithidae)}
\Clue{32}{}{uczucie wywołane brakiem czegoś lub kogoś ważnego dla danej osoby}
\Clue{33}{}{Stefan (1911-1991);  kompozytor, krytyk muzyczny, pisarz; utwory orkiestrowe, kameralne, fortepianowe}
\Clue{34}{}{Włodzimierz, ur. w 1905r. astronom; badania w dziedzinie astronomii pozagalaktycznej, popularyzator astronomii}\end{PuzzleClues}

\begin{PuzzleClues}{\textbf{Pionowe}\\}\Clue{1}{}{półprodukt do wyrobu przędzy, włókna o luźnej strukturze lekko  skręcone w wąski pasek}
\Clue{2}{}{rodzaj kubka z ustnikiem zapobiegającym wychlapywaniu się płynów}
\Clue{3}{}{instrumentalna forma muzyczna, która jest wstępem do większego dzieła muzycznego; na przestrzeni historii preludia bywały bardzo krótkie, ale też rozbudowane (np. preludia w fugach, suitach, sonatach, aktach w operach)}
\Clue{4}{}{konieczność zrobienia czegoś, wymóg konkretnego zachowania, który wynika z kwestii obyczajowch, moralnych}
\Clue{5}{}{wydatki budżetowe przeznaczone na zakup przedmiotów materialnych, żywności oraz usług}
\Clue{8}{}{ptak z rzędu mew-siewek, chroniony, podobny do brodźców; Europa, Ameryka}
\Clue{9}{}{miasto obwodowe w Federacji Rosyjskiej, nad Oką}
\Clue{11}{}{ewangelia - część mszy, podczas której czyta się Ewangelię}
\Clue{12}{}{ilość dóbr oferowana na rynku przez producentów przy określonej cenie, przy założeniu niezmienności innych elementów charakteryzujących sytuację na rynku}
\Clue{14}{}{ktoś lub coś, co ma przynosić szczęście, być dobrym duchem czegoś, często też: być symbolem, ocieplać wyzerunek imprezy}
\Clue{15}{}{w analizie matematycznej własność funkcji określonych w przestrzeniach metryczych o wartościach rzeczywistych}
\Clue{16}{}{część Zatoki Panamskiej}
\Clue{17}{}{sztruks - rodzaj prążkowanej tkaniny na odzież roboczą}
\Clue{18}{}{miasto w środkowej Gruzji; wydobycie węgla kamiennego}
\Clue{21}{}{rdzeń liny, często stalowy}
\Clue{24}{}{marka samochodu; japoński koncern motoryzacyjny Toyota Motor Corporation}
\Clue{25}{}{białe wino francuskie, produkowane w rejonie miasta o tej samej nazwie}
\Clue{27}{}{młody chłopak, smarkacz; słowo negatywne, używane często jako wyzwisko}
\Clue{28}{}{ANHUEJ}\end{PuzzleClues}\newpage\section*{Krzyżówka 87}

\noindent\begin{Puzzle}{24}{30}|*	|*	|*	|*	|*	|*	|*	|*	|*	|*	|*	|*	|*	|*	|*	|*	|*	|*	|[1][S]\drarr	|b	|a	|r	|r	|y	|*	|.
|*	|*	|*	|*	|*	|*	|*	|[2][S]\drarr	|z	|a	|s	|t	|a	|ł	|o	|ś	|ć	|*	|k	|*	|*	|*	|*	|*	|*	|.
|*	|*	|*	|*	|*	|[3][S]\rarr	|p	|a	|p	|u	|g	|a	|[][,]{ }	|m	|a	|s	|k	|a	|r	|e	|ń	|s	|k	|a	|*	|.
|*	|*	|*	|[4][S]\rarr	|g	|i	|p	|u	|r	|a	|*	|*	|*	|[5][S]\rarr	|l	|u	|ź	|n	|o	|ś	|ć	|*	|[6][S]\darr	|*	|*	|.
|*	|*	|[7][S]\rarr	|a	|n	|g	|s	|t	|r	|o	|m	|*	|*	|*	|*	|[8][S]\darr	|*	|*	|w	|[9][S]\darr	|*	|*	|p	|[10][S]\darr	|*	|.
|[11][S]\drarr	|n	|i	|e	|z	|a	|w	|o	|d	|o	|w	|i	|e	|c	|*	|k	|*	|*	|i	|d	|*	|*	|r	|c	|*	|.
|s	|*	|[12][S]\drarr	|p	|a	|d	|y	|s	|z	|a	|c	|h	|*	|*	|*	|u	|*	|*	|e	|ł	|*	|*	|e	|i	|*	|.
|t	|[13][S]\rarr	|p	|r	|z	|e	|s	|t	|r	|z	|e	|ń	|[][,]{ }	|b	|a	|r	|w	|*	|[][,]{ }	|a	|[14][S]\darr	|*	|z	|ą	|*	|.
|y	|[15][S]\darr	|i	|*	|*	|[16][S]\darr	|[17][S]\drarr	|o	|d	|m	|a	|[][,]{ }	|o	|p	|ł	|u	|c	|n	|o	|w	|a	|*	|y	|g	|*	|.
|l	|n	|ż	|[18][S]\rarr	|h	|a	|p	|p	|y	|[][,]{ }	|e	|n	|d	|*	|*	|r	|*	|*	|c	|i	|n	|*	|d	|n	|*	|.
|*	|i	|m	|[19][S]\darr	|*	|b	|o	|o	|*	|[20][S]\rarr	|j	|a	|s	|t	|r	|u	|n	|*	|z	|g	|k	|*	|e	|i	|[21][S]\darr	|.
|[22][S]\drarr	|t	|o	|r	|*	|s	|n	|w	|*	|[23][S]\drarr	|r	|u	|m	|u	|n	|*	|*	|*	|y	|a	|a	|*	|n	|k	|o	|.
|r	|r	|w	|o	|*	|u	|i	|i	|[24][S]\rarr	|s	|z	|k	|o	|l	|n	|o	|ś	|ć	|*	|d	|r	|*	|t	|[][,]{ }	|r	|.
|o	|y	|ó	|ż	|*	|r	|a	|c	|[25][S]\rarr	|z	|i	|m	|o	|w	|i	|s	|k	|o	|*	|*	|k	|*	|o	|a	|o	|.
|z	|l	|ł	|e	|*	|d	|t	|z	|*	|y	|*	|*	|*	|*	|*	|*	|*	|*	|*	|*	|a	|[26][S]\darr	|s	|r	|y	|.
|d	|*	|*	|k	|*	|*	|o	|*	|[27][S]\rarr	|f	|l	|o	|r	|e	|n	|t	|y	|n	|k	|a	|*	|b	|t	|t	|a	|.
|z	|*	|*	|*	|*	|[28][S]\rarr	|w	|y	|p	|r	|y	|s	|k	|[][,]{ }	|k	|o	|n	|t	|a	|k	|t	|o	|w	|y	|*	|.
|i	|*	|*	|[29][S]\rarr	|p	|r	|a	|w	|o	|[][,]{ }	|p	|r	|a	|c	|y	|*	|*	|*	|*	|*	|*	|m	|o	|l	|*	|.
|e	|[30][S]\rarr	|o	|p	|a	|l	|*	|*	|[31][S]\rarr	|p	|r	|z	|e	|p	|a	|d	|*	|[32][S]\drarr	|s	|e	|r	|b	|*	|e	|*	|.
|l	|*	|*	|*	|[33][S]\rarr	|r	|e	|g	|u	|ł	|a	|[][,]{ }	|t	|i	|n	|b	|e	|r	|g	|e	|n	|a	|*	|r	|*	|.
|c	|[34][S]\rarr	|ś	|w	|i	|e	|r	|a	|d	|o	|w	|i	|a	|n	|i	|n	|*	|a	|*	|*	|*	|*	|*	|y	|*	|.
|z	|[35][S]\rarr	|d	|r	|u	|g	|a	|[][,]{ }	|s	|t	|r	|o	|n	|a	|[][,]{ }	|m	|e	|d	|a	|l	|u	|*	|[36][S]\darr	|j	|*	|.
|o	|*	|*	|*	|[37][S]\drarr	|w	|o	|w	|*	|o	|[38][S]\darr	|[39][S]\rarr	|g	|w	|a	|r	|d	|i	|a	|n	|*	|*	|n	|s	|*	|.
|ś	|*	|*	|[40][S]\rarr	|k	|n	|y	|p	|*	|w	|k	|[41][S]\rarr	|o	|s	|n	|o	|w	|a	|*	|*	|*	|[42][S]\darr	|u	|k	|*	|.
|ć	|*	|*	|*	|u	|*	|*	|*	|*	|y	|o	|[43][S]\rarr	|s	|t	|i	|l	|o	|n	|*	|*	|*	|g	|c	|i	|*	|.
|*	|[44][S]\drarr	|k	|u	|b	|e	|l	|i	|k	|*	|n	|[45][S]\rarr	|k	|u	|r	|t	|a	|*	|*	|*	|*	|i	|z	|*	|*	|.
|[46][S]\drarr	|b	|o	|j	|a	|ź	|ń	|*	|*	|[47][S]\rarr	|w	|i	|e	|r	|t	|o	|w	|n	|i	|a	|*	|g	|a	|*	|*	|.
|d	|u	|*	|*	|n	|*	|*	|*	|[48][S]\rarr	|r	|ó	|ż	|a	|[][,]{ }	|b	|a	|r	|y	|t	|o	|w	|a	|*	|*	|*	|.
|z	|s	|*	|*	|k	|[49][S]\rarr	|p	|o	|d	|e	|j	|ź	|r	|z	|o	|n	|*	|*	|[50][S]\rarr	|l	|i	|n	|i	|a	|*	|.
|d	|*	|[51][S]\rarr	|k	|a	|w	|l	|a	|t	|a	|*	|*	|*	|*	|*	|*	|*	|[52][S]\rarr	|r	|z	|u	|t	|*	|*	|*	|.
|*	|*	|*	|*	|*	|*	|*	|*	|*	|*	|*	|*	|*	|*	|*	|*	|*	|*	|*	|*	|*	|*	|*	|*	|*	|.\end{Puzzle}

\newpage

\begin{PuzzleClues}{\textbf{Poziome}\\}\Clue{1}{}{architekt angielski (1795-1860) - przedstawiciel eklektyzmu}
\Clue{2}{}{cecha oznaczająca statyczność negatywnie ocenianą, brak zmian, niechęć do zmian}
\Clue{3}{}{Mascarinus mascarinus - wymarły gatunek ptaka z rodziny papugowatych (Psittacidae), z podrodziny papug wschodnich (Psittaculinae)}
\Clue{4}{}{jedwabna tkanina o wypukłym wzorze}
\Clue{5}{}{nieścisłość, niepełność, niejasność}
\Clue{7}{}{geofizyk szwedzki (1888-1981), badał aktywność Słońca}
\Clue{11}{}{człowiek robiący coś niezawodowo, bez umiejętności w danej dziedzinie}
\Clue{12}{}{osoba nosząca prestiżowy tytuł monarchy perskiego, tureckiego, szacha Iranu lub króla Afganistanu}
\Clue{13}{}{widma fal elektromagnetycznych z zakresu od 380 do 780 nm (tj. światło widzialne), których matematyczne modele są przedstawiane w trójwymiarowej przestrzeni barw}
\Clue{17}{}{wtargnięcie powietrza lub innych gazów do jamy opłucnej spowodowane najczęściej uszkodzeniem miąższu płucnego lub przedziurawieniem ściany klatki piersiowej}
\Clue{18}{}{pomyślne zakończenie}
\Clue{20}{}{bylina o białych kwiatach zebranych w koszyczki, pospolita na łąkach}
\Clue{22}{}{droga, po której porusza się obiekt, zwłaszcza pojazd}
\Clue{23}{}{lokomotywa spalinowa Electroputere 060DA; nazwa oddaje fakt pochodzenia lokomotywy z zakładów elektrycznych z Rumunii}
\Clue{24}{}{to, że coś jest szkolne - związane ze szkołą, właściwe szkole, charakterystyczne dla szkoły}
\Clue{25}{}{miejsce, w którym spędza się zimę}
\Clue{27}{}{cienkie kruche ciastko z dużą ilością płatków migdałowych, z dodatkiem innych bakalii, aromatów, chrupiących składników}
\Clue{28}{}{choroba skóry objawiająca się tworzeniem się na jej powierzchni pęcherzyków, wyraźnym zaczerwienieniem oraz wysuszeniem i łuszczeniem po kontakcie z alergenem, głównie z substancją chemiczną}
\Clue{29}{}{gałąź prawa obejmująca ogół regulacji w zakresie stosunku pracy konkretnego pracownika i pracodawcy jako stron stosunku pracy oraz regulacji dotyczących organizacji pracodawców i pracowników, układów i sporów zbiorowych, a także partycypacji pracowniczej i dialogu w zbiorowych stosunkach pracy}
\Clue{30}{}{półszlachetny kamień ozdobny; mineraloid zaliczany do krzemianów}
\Clue{31}{}{w prawie: kara polegająca na utracie na rzecz państwa mienia należącego do osoby skazanej wyrokiem sądowym lub posiadanego bezprawnie}
\Clue{32}{}{mieszkaniec Serbii, człowiek pochodzenia serbskiego}
\Clue{33}{}{reguła mówiąca o tym, że do osiągnięcia określonej liczby celów konieczne jest posiadanie co najmniej tyle samo niezależnych instrumentów}
\Clue{34}{}{mieszkaniec Świeradowa Zdroju}
\Clue{35}{}{inny punkt widzenia}
\Clue{37}{}{gra komputerowa z gatunku MMORPG wyprodukowana przez amerykańską firmę Blizzard Entertainment; jej akcja toczy się cztery lata po wydarzeniach przedstawionych w grze Warcraft III: The Frozen Throne, w świecie stworzonym w 1994 roku na potrzeby Warcraft: Orcs \& Humans}
\Clue{39}{}{przełożony klasztoru w niektórych zakonach}
\Clue{40}{}{szewski nóż}
\Clue{41}{}{podstawowy szkielet opony dźwigający obciążenie}
\Clue{43}{}{cienkie, elastyczne i wytrzymałe sztuczne włókno tekstylne; jedna z nazw handlowych produktów z polikaprolaktamu}
\Clue{44}{}{Jan (1880-1940); czeski skrzypek i kompozytor; poematy symfoniczne, koncerty i miniatury skrzypcowe}
\Clue{45}{}{duża, obszerna kurtka}
\Clue{46}{}{uczucie strachu}
\Clue{47}{}{urządzenie do wiercenia}
\Clue{48}{}{skupienie tabliczkowych lub płytkowych kryształów barytu przypominające wyglądem kształt róży}
\Clue{49}{}{Botrychium - rodzaj roślin z rodziny nasięźrzałowatych (Ophioglossaceae), stanowiącej takson monotypowy w obrębie rzędu nasięźrzałowców (Ophioglossales); gatunkiem typowym jest Botrychium lunaria}
\Clue{50}{}{sylwetka, figura człowieka; często: zgrabna sylwetka, szczupła figura}
\Clue{51}{}{zupa jarzynowa z wieprzowiną lub maltańskimi kiełbaskami, pochodząca z Malty}
\Clue{52}{}{część ugrupowania bojowego w armii, która wykonuje określone zadania}\end{PuzzleClues}

\begin{PuzzleClues}{\textbf{Pionowe}\\}\Clue{1}{}{o spojrzeniu; patrzenie w sposób bezrozumny, apatyczny, łagodny}
\Clue{2}{}{osoba, która podróżuje autostopem}
\Clue{6}{}{prezydent i jego żona}
\Clue{8}{}{AGA}
\Clue{9}{}{ptak z rodziny bocianów, dziób na końcu lekko zakrzywiony; płd. Afryka}
\Clue{10}{}{pojazd mechaniczny przeznaczony do ciągnięcia dział w artylerii}
\Clue{11}{}{sposób wykonywania określonych ruchów w danej dyscyplinie sportowej}
\Clue{12}{}{przeżuwacz z rodziny krętorogich, spotykany w północnej części Kanady i na Grenlandii}
\Clue{14}{}{mieszkanka Ankary}
\Clue{15}{}{organiczny związek chemiczny, pochodna cyjanowodoru o wzorze ogólnym R-C?N}
\Clue{16}{}{coś, co jest dziwne, nieprawdopodobne, nielogiczne i niezgodne z jakimkolwiek wyobrażeniem o rzeczywistości}
\Clue{17}{}{miasto w południowo-wschodniej Polsce, w województwie lubelskim, w powiecie opolskim, położone w Kotlinie Chodelskiej; siedziba miejsko-wiejskiej gminy Poniatowa}
\Clue{19}{}{instrument muzyczny lub sygnałowy, dość prymitywny, rodzaj trąbki}
\Clue{21}{}{miasto w Peru w Andach na wysokości ok. 3700 m; wydobycie rud i hutnictwo metali nieżelaznych}
\Clue{22}{}{przydatność określonego przyrządu optycznego do obserwacji obiektów o określonej odległości kątowej, im większa jest rozdzielczość, tym bliższe sobie punkty są obserwowane jako odrębne, a nie jako pojedyncza plama}
\Clue{23}{}{rodzaj szyfru polegający na zapisaniu kolejnych liter tekstu jawnego na zmianę w ustalonej ilości rzędach}
\Clue{26}{}{pocisk z mechanizmem zapłonowym napełniony materiałem wybuchowym, zapalającym, dymnym, jądrowym, itp}
\Clue{32}{}{rad - w geometrii jednostka miary łukowej kąta płaskiego, a ponadto niemianowana jednostka uzupełniająca układu SI, zdefiniowana za pomocą równości długości l łuku okręgu o środku w wierzchołku kąta ? i jego promienia r}
\Clue{36}{}{cedzidło; zbiornik z dnem sitowym służący do filtrowania niewielkich ilości cieczy}
\Clue{37}{}{mieszkanka Kuby, kobieta pochodzenia kubańskiego}
\Clue{38}{}{oddział żołnierzy do ochrony transportu, strzeżenia aresztantów i jeńców}
\Clue{42}{}{w mitologii greckiej olbrzym o wężowych splotach zamiast nóg i uskrzydlonym torsie; syn Gai i Uranosa}
\Clue{44}{}{minibus, mały autobus o długości mniejszej niż 7 metrów}
\Clue{46}{}{kod ISO 4217 dinara algierskiego}\end{PuzzleClues}\newpage\section*{Krzyżówka 88}

\noindent\begin{Puzzle}{25}{25}|*	|*	|*	|*	|[1][S]\drarr	|z	|a	|w	|o	|d	|z	|e	|n	|i	|e	|*	|*	|[2][S]\darr	|*	|*	|[3][S]\drarr	|p	|e	|l	|a	|*	|.
|*	|*	|*	|*	|ś	|*	|*	|*	|*	|*	|*	|[4][S]\drarr	|t	|e	|z	|a	|*	|d	|[5][S]\drarr	|a	|k	|r	|y	|l	|*	|*	|.
|*	|*	|*	|[6][S]\darr	|n	|*	|*	|*	|*	|*	|*	|s	|*	|*	|*	|*	|*	|o	|h	|*	|o	|[7][S]\darr	|[8][S]\darr	|*	|[9][S]\darr	|*	|.
|*	|*	|[10][S]\rarr	|m	|i	|n	|ó	|g	|[][,]{ }	|g	|r	|e	|c	|k	|i	|*	|*	|j	|o	|*	|s	|e	|k	|*	|w	|*	|.
|*	|*	|*	|ą	|a	|*	|*	|*	|[11][S]\drarr	|p	|e	|l	|y	|k	|o	|z	|a	|u	|r	|*	|t	|d	|a	|[12][S]\darr	|o	|*	|.
|*	|*	|*	|t	|d	|*	|*	|*	|b	|*	|[13][S]\darr	|s	|*	|*	|*	|*	|*	|t	|m	|[14][S]\darr	|r	|d	|z	|a	|d	|*	|.
|*	|*	|*	|w	|a	|*	|*	|*	|i	|*	|d	|y	|[15][S]\rarr	|s	|ą	|d	|[][,]{ }	|r	|o	|d	|z	|i	|n	|n	|y	|*	|.
|*	|*	|*	|a	|n	|*	|[16][S]\darr	|*	|e	|*	|w	|n	|*	|*	|*	|*	|*	|e	|n	|o	|e	|n	|o	|d	|[][,]{ }	|*	|.
|*	|*	|*	|*	|i	|*	|p	|*	|g	|*	|u	|*	|*	|*	|*	|*	|*	|k	|[][,]{ }	|g	|w	|g	|d	|o	|i	|*	|.
|*	|*	|*	|*	|ó	|*	|i	|*	|u	|*	|n	|*	|*	|[17][S]\drarr	|d	|ą	|b	|*	|t	|l	|a	|t	|z	|r	|n	|*	|.
|*	|*	|*	|[18][S]\drarr	|w	|i	|e	|ś	|n	|i	|a	|k	|*	|p	|*	|*	|*	|*	|k	|i	|[][,]{ }	|o	|i	|c	|g	|*	|.
|*	|*	|*	|p	|k	|*	|g	|*	|*	|*	|s	|*	|*	|ł	|[19][S]\darr	|*	|*	|[20][S]\darr	|a	|n	|n	|n	|e	|z	|l	|*	|.
|*	|*	|*	|r	|a	|*	|u	|*	|*	|[21][S]\darr	|t	|*	|*	|y	|f	|*	|*	|ł	|n	|g	|i	|*	|j	|y	|a	|*	|.
|[22][S]\rarr	|a	|o	|a	|*	|[23][S]\rarr	|s	|ł	|u	|g	|a	|*	|*	|w	|o	|*	|*	|o	|k	|*	|b	|[24][S]\darr	|s	|k	|c	|*	|.
|*	|*	|*	|w	|*	|*	|e	|*	|*	|ę	|*	|*	|*	|a	|s	|*	|*	|d	|o	|*	|y	|m	|t	|*	|j	|*	|.
|*	|*	|[25][S]\rarr	|o	|t	|o	|k	|*	|*	|d	|*	|*	|*	|c	|s	|*	|*	|z	|w	|*	|d	|i	|w	|[26][S]\darr	|a	|*	|.
|*	|*	|*	|[][,]{ }	|*	|*	|*	|*	|*	|ż	|*	|*	|*	|z	|a	|*	|*	|i	|y	|*	|a	|e	|o	|k	|l	|*	|.
|*	|*	|*	|h	|*	|*	|*	|[27][S]\rarr	|o	|b	|e	|d	|i	|e	|n	|c	|j	|a	|*	|*	|l	|j	|*	|o	|n	|*	|.
|*	|*	|*	|u	|*	|[28][S]\rarr	|k	|a	|n	|a	|ł	|*	|*	|k	|o	|*	|*	|n	|*	|*	|m	|s	|*	|m	|e	|*	|.
|*	|*	|[29][S]\rarr	|b	|o	|n	|i	|t	|o	|*	|*	|*	|*	|*	|*	|[30][S]\rarr	|p	|i	|s	|t	|a	|c	|j	|a	|*	|*	|.
|*	|*	|*	|b	|[31][S]\rarr	|d	|e	|l	|f	|i	|n	|[][,]{ }	|n	|a	|d	|o	|b	|n	|y	|*	|c	|e	|*	|r	|*	|*	|.
|*	|*	|*	|l	|*	|*	|*	|[32][S]\rarr	|e	|n	|t	|o	|d	|e	|r	|m	|a	|*	|*	|*	|k	|*	|*	|y	|*	|*	|.
|*	|*	|*	|e	|*	|*	|*	|[33][S]\rarr	|j	|ę	|z	|y	|c	|z	|e	|k	|[][,]{ }	|u	|[][,]{ }	|w	|a	|g	|i	|*	|*	|*	|.
|*	|*	|*	|[][S]'	|*	|*	|*	|*	|*	|*	|*	|*	|*	|*	|*	|*	|*	|*	|*	|*	|*	|*	|*	|*	|*	|*	|.
|*	|[34][S]\rarr	|ż	|a	|b	|y	|[][,]{ }	|a	|u	|s	|t	|r	|a	|l	|i	|j	|s	|k	|i	|e	|*	|*	|*	|*	|*	|*	|.
|*	|*	|*	|*	|*	|*	|*	|*	|*	|*	|*	|*	|*	|*	|*	|*	|*	|*	|*	|*	|*	|*	|*	|*	|*	|*	|.\end{Puzzle}

\newpage

\begin{PuzzleClues}{\textbf{Poziome}\\}\Clue{1}{}{powolny, fałszywy śpiew z długimi dźwiękami}
\Clue{3}{}{miękkie, cienkie włókno lub tkanina jedwabna; cienkie nici jedwabne}
\Clue{4}{}{podstawa metody dialektycznej u Hegla}
\Clue{5}{}{obraz namalowany farbami akrylowymi}
\Clue{10}{}{minog grecki, Eudontomyzon hellenicus - gatunek europejskiego, słodkowodnego, pasożytniczego bezżuchwowca z rodziny minogowatych (Petromyzontidae); minóg grecki występuje wyłącznie w Grecji, w zlewiskach Morza Egejskiego; stanowi on gatunek krytycznie zagrożony wyginięciem}
\Clue{11}{}{nazwa przedstawiciela grupy wymarłych gadów ssakokształtnych}
\Clue{15}{}{wydział rodzinny i nieletnich w sądzie rejonowym}
\Clue{17}{}{pospolite drzewo liściaste o cennym drewnie}
\Clue{18}{}{ktoś, kto robi wiochę, jest wieśniacki w swoim wyglądzie, zachowaniu, przynosi wstyd}
\Clue{22}{}{kod ISO 4217 waluty kwanza}
\Clue{23}{}{osoba poddana drugiej, zobowiązana do wykonywania jej rozkazów}
\Clue{25}{}{rodzaj myśliwskiej smyczy}
\Clue{27}{}{W Kościele rzymskokatolickim posłuszeństwo, do jakiego zobowiązani są wierni wobec papieża i biskupów, będących w jedności z papieżem, w sprawach wiary i moralności (duchowa władza Kościoła nad wiernymi)}
\Clue{28}{}{trudna sprawa, sytuacja}
\Clue{29}{}{ryba z rodziny makrelowatych, drapieżna; ceniona w przetwórstwie}
\Clue{30}{}{drobny orzech, owoc (pestkowiec) pistacji właściwej}
\Clue{31}{}{Stenella clymene - gatunek walenia z rodziny delfinowatych; zamieszkuje wody subtropikalne i tropikalne Atlantyku, włącznie z Zatoką Meksykańską}
\Clue{32}{}{u jamochłonów: warstwa, która wyściela wnętrze ciała}
\Clue{33}{}{w wagach starego typu (nieelektronicznych) - drobny element, wypustka, która wskazuje na przechylenie się wagi na jedną lub drugą szalę }
\Clue{34}{}{żółwinkowate, Myobatrachidae - rodzina płazów bezogonowych}\end{PuzzleClues}

\begin{PuzzleClues}{\textbf{Pionowe}\\}\Clue{1}{}{zawartość śniadaniówki, pojemnika służącego do przenoszenia i przechowywania posiłków}
\Clue{2}{}{osoba opieszała, która odkłada zrobienie wszystkiegodo jutra}
\Clue{3}{}{Festuca pseudodalmatica - gatunek trawy z rodziny wiechlinowatych}
\Clue{4}{}{maszyna elektryczna indukcyjna małej mocy służąca do przekazywania lub odtwarzania na odległość wychyleń kątowych na drodze elektrycznej}
\Clue{5}{}{hormon wytwarzany nie w gruczołach wydzielania wewnętrznego, ale w tkankach}
\Clue{6}{}{Sepioidea - mięczak z rodziny głowonogów}
\Clue{7}{}{angielski astronom i fizyk (1882-1944), twórca modeli kosmologicznych oraz podstaw wewnętrznej budowy gwiazd}
\Clue{8}{}{moralizowanie, pouczanie innych}
\Clue{9}{}{wody, które znajdują się wewnątrz lodowca}
\Clue{11}{}{rodzaj zawiasów u drzwi i wrót w starych budowlach}
\Clue{12}{}{mieszkaniec Andory - państwa, człowiek pochodzenia andorskiego}
\Clue{13}{}{godzina - południe, samo południe}
\Clue{14}{}{waleń z zębowców o długości do 9 m; tzw. wal butelkonosy}
\Clue{16}{}{ciasto w kropki (zwykle kropki zapewnia domieszka maku w cieście, ale można tak też nazwać ciasto z kropkami z np. czekolady lub drażetek)}
\Clue{17}{}{zdrobniale: pływak - szczelny zbiornik, wypełniony powietrzem lub gazem, służący do utrzymania samolotu na powierzchni wody}
\Clue{18}{}{podstawowe prawo kosmologii obserwacyjnej, wiążące odległości galaktyk r z ich tzw. prędkościami ucieczki v (których miarą jest przesunięcie ku czerwieni z)}
\Clue{19}{}{miasto we Włoszech (Piemont), ważny węzeł kolejowy}
\Clue{20}{}{mieszkaniec Łodzi}
\Clue{21}{}{staropolskie; muzyka, śpiew}
\Clue{24}{}{pozycja wśród innych, w grupie osób, rzeczy lub zjawisk; ranga, osiągnięta lokata}
\Clue{26}{}{Culicidae, komary - występująca na całym świecie rodzina owadów (nadrodzina Culicoidea) z rzędu muchówek}\end{PuzzleClues}\newpage\section*{Krzyżówka 89}

\noindent\begin{Puzzle}{18}{30}|*	|*	|*	|[1][S]\drarr	|p	|o	|d	|w	|ó	|j	|*	|[2][S]\darr	|*	|*	|[3][S]\drarr	|f	|*	|*	|[4][S]\darr	|.
|*	|*	|[5][S]\darr	|p	|[6][S]\drarr	|c	|z	|t	|e	|r	|o	|l	|e	|c	|i	|e	|*	|[7][S]\darr	|b	|.
|*	|[8][S]\rarr	|p	|r	|z	|e	|w	|o	|d	|z	|i	|e	|ń	|*	|z	|[9][S]\darr	|*	|r	|i	|.
|*	|[10][S]\darr	|e	|a	|ł	|*	|[11][S]\rarr	|u	|z	|i	|o	|m	|*	|*	|o	|o	|*	|o	|e	|.
|*	|s	|t	|o	|o	|*	|*	|*	|*	|*	|*	|u	|*	|*	|c	|k	|[12][S]\darr	|ś	|g	|.
|*	|z	|r	|j	|t	|*	|*	|*	|*	|*	|*	|r	|[13][S]\drarr	|g	|h	|o	|u	|l	|*	|.
|*	|c	|e	|c	|o	|[14][S]\darr	|[15][S]\drarr	|k	|o	|l	|o	|k	|o	|l	|o	|*	|m	|i	|*	|.
|*	|z	|l	|z	|*	|w	|d	|[16][S]\darr	|*	|*	|[17][S]\darr	|i	|g	|[18][S]\darr	|r	|[19][S]\darr	|o	|n	|*	|.
|*	|y	|[][,]{ }	|y	|*	|a	|i	|k	|*	|[20][S]\darr	|p	|*	|n	|o	|a	|c	|c	|a	|*	|.
|*	|p	|w	|z	|[21][S]\rarr	|p	|a	|w	|i	|l	|o	|n	|i	|k	|*	|h	|n	|[][,]{ }	|*	|.
|*	|i	|ę	|n	|*	|i	|p	|i	|[22][S]\darr	|y	|d	|[23][S]\darr	|w	|r	|*	|o	|i	|p	|*	|.
|*	|o	|d	|a	|*	|t	|e	|n	|w	|n	|m	|k	|o	|ą	|*	|i	|e	|u	|*	|.
|*	|r	|r	|*	|[24][S]\darr	|i	|d	|t	|i	|c	|i	|o	|*	|g	|*	|n	|n	|s	|[25][S]\darr	|.
|*	|n	|o	|*	|b	|*	|e	|a	|ę	|h	|o	|n	|*	|ł	|*	|a	|i	|t	|p	|.
|*	|i	|w	|*	|u	|*	|z	|*	|z	|*	|t	|w	|*	|o	|*	|[][,]{ }	|e	|y	|a	|.
|*	|s	|n	|[26][S]\drarr	|s	|z	|a	|t	|a	|n	|*	|e	|*	|ś	|*	|k	|*	|n	|r	|.
|[27][S]\drarr	|t	|y	|r	|o	|l	|*	|*	|r	|*	|*	|r	|*	|ć	|*	|a	|*	|n	|a	|.
|z	|a	|*	|y	|l	|*	|*	|*	|*	|*	|*	|s	|*	|*	|*	|n	|*	|a	|m	|.
|ł	|*	|[28][S]\rarr	|n	|a	|j	|a	|*	|*	|[29][S]\rarr	|z	|a	|w	|i	|j	|a	|s	|*	|e	|.
|o	|*	|[30][S]\darr	|i	|*	|*	|*	|[31][S]\rarr	|k	|o	|r	|t	|l	|a	|n	|d	|*	|*	|t	|.
|t	|*	|p	|e	|*	|[32][S]\rarr	|g	|m	|i	|n	|n	|o	|ś	|ć	|*	|y	|[33][S]\darr	|*	|r	|.
|o	|*	|r	|n	|*	|*	|*	|*	|*	|[34][S]\drarr	|m	|r	|o	|k	|*	|j	|a	|*	|y	|.
|g	|*	|o	|k	|*	|[35][S]\rarr	|a	|m	|y	|l	|o	|i	|d	|*	|*	|s	|d	|*	|z	|.
|o	|[36][S]\drarr	|f	|a	|ł	|d	|ó	|w	|k	|a	|[][,]{ }	|u	|n	|i	|t	|k	|a	|*	|a	|.
|n	|b	|i	|*	|[37][S]\rarr	|d	|y	|b	|u	|k	|*	|m	|*	|*	|*	|a	|r	|*	|c	|.
|e	|r	|t	|*	|*	|[38][S]\rarr	|w	|y	|r	|o	|k	|*	|*	|*	|*	|*	|[][,]{ }	|*	|j	|.
|k	|z	|*	|*	|[39][S]\rarr	|k	|r	|z	|a	|n	|o	|w	|s	|k	|i	|*	|i	|*	|a	|.
|*	|u	|*	|*	|*	|*	|*	|[40][S]\rarr	|p	|i	|ę	|c	|i	|o	|r	|n	|i	|k	|*	|.
|[41][S]\rarr	|c	|y	|g	|a	|r	|n	|i	|c	|z	|k	|a	|*	|*	|*	|*	|*	|*	|*	|.
|*	|h	|*	|*	|*	|[42][S]\rarr	|e	|k	|u	|m	|e	|n	|a	|*	|*	|*	|*	|*	|*	|.
|*	|*	|*	|[43][S]\rarr	|s	|p	|l	|i	|t	|*	|*	|*	|*	|*	|*	|*	|*	|*	|*	|.\end{Puzzle}

\newpage

\begin{PuzzleClues}{\textbf{Poziome}\\}\Clue{1}{}{denny skorupiak równonogi pospolity w strefie arktycznej}
\Clue{3}{}{dźwięk muzyczny, którego częstotliwość w oktawie razkreślnej wynosi 349,6 Hz.}
\Clue{6}{}{czwarta rocznica}
\Clue{8}{}{sworzeń; gruby gwóźdź łączący dyszel i śnice wozu}
\Clue{11}{}{metalowa elektroda lub zespół elektrod umieszczona w wilgotnej warstwie gruntu, zapewniający połączenie przedmiotów uziemianych i gruntu (ziemi) z możliwie małą rezystancją}
\Clue{13}{}{w fantastyce: nieumarła istota żywiąca się ludzkimi zwłokami, ożywiona po śmierci przez czarną magię albo przez zło, którego dokonała za życia}
\Clue{15}{}{andyjski mały drapieżnik z rodziny kotów}
\Clue{21}{}{zdrobniale o pawilonie, bocznym skrzydle budynku}
\Clue{26}{}{urwis, nicpoń}
\Clue{27}{}{kraj związkowy w zach. Australii, stolica Innsbruck, powierzchnia 12,6 tyś. km2, główny region turystyczny Austrii}
\Clue{28}{}{KOBRA, okularnik: duży jadowity wąż z rodziny zdradnicowatych}
\Clue{29}{}{figura stylistyczna mająca charakter ozdobnika}
\Clue{31}{}{Malus domestica 'Cortland' - odmiana uprawna jabłoni domowej}
\Clue{32}{}{grupa zamieszkująca gminę}
\Clue{34}{}{przenośnie: coś, co sprawia, że czegoś nie nie dostrzega lub nie pamięta dobrze; pomroka}
\Clue{35}{}{kwasochłonne, nierozpuszczalne białko fibrylarne, powstające w organizmie w wyniku długiego przebiegu wyniszczających chorób}
\Clue{36}{}{Eilema palliatella - gatunek motyla nocnego z rodziny niedźwiedziówkowatych, lata od lipca do połowy sierpnia, występuje na wrzosowiskach i ugorach}
\Clue{37}{}{według żydowskich wierzeń zły duch lub dusza zmarłego wcielające się w osoby żyjące}
\Clue{38}{}{orzeczenie sądu, rozstrzygające merytorycznie o kwestii będącej przedmiotem postępowania sądowego}
\Clue{39}{}{kompozytor i akordeonista (1951-1990); utwory orkiestrowe, kameralne, akordeonowe; 'Alkagran'}
\Clue{40}{}{tzw. kurze ziele; bylina z rodziny różowatych o owłosionych liściach i żółtych kwiatach}
\Clue{41}{}{rodzaj lufki, wydłużony ustnik do papierosa}
\Clue{42}{}{obszary na kuli ziemskiej stale zamieszkane i wykorzystywane gospodarczo przez człowieka}
\Clue{43}{}{w ekonomii - obniżenie wartości nominalnej akcji przy jednoczesnym utrzymaniu dotychczasowego kapitału akcyjnego spółki}\end{PuzzleClues}

\begin{PuzzleClues}{\textbf{Pionowe}\\}\Clue{1}{}{pierwotna ojczyzna, najdawniejsza siedziba jakiegoś ludu}
\Clue{2}{}{lemurkowate, Cheirogaleidae - rodzina małych małpiatek z podrzędu Strepsirrhini o nocnym trybie życia, najmniejsze ssaki naczelne; występują wyłącznie na Madagaskarze}
\Clue{3}{}{linia na wykresie termodynamicznym przedstawiająca przemianę izochoryczną}
\Clue{4}{}{pełne determinacji i energii dążenie do celu, starania, walka, realizowanie planu}
\Clue{5}{}{Pterodroma externa - gatunek ptaka z rodziny burzykowatych (Procellariidae)}
\Clue{6}{}{ktoś bardzo drogi, bliski}
\Clue{7}{}{roślina występująca w ciepłym, suchym klimacie, na terenach pustynnych}
\Clue{9}{}{wzrok w chwili patrzenia, wyrażający emocje; występuje w liczbie mnogiej}
\Clue{10}{}{sportowiec grający w piłkę ręczną}
\Clue{12}{}{to, co umacnia, wynik podjęcia określonych robót zmierzających do zapewnienia stateczności gruntu, budowli, utrwalenia brzegu rzeki czy morza oraz zabezpieczenia przed erozją, w szczególności erozją wodną}
\Clue{13}{}{źródło stałego prądu elektrycznego}
\Clue{14}{}{północnoamerykański. podgatunek jelenia, łowny; lasy strefy umiarkowanej}
\Clue{15}{}{zdolność do przenikania krwinek poza nieuszkodzone ściany naczyń krwionośnych}
\Clue{16}{}{najcieńsza struna skrzypiec}
\Clue{17}{}{osoba lub instytucja w terminologii prawniczej lub ekonomicznej}
\Clue{18}{}{cecha figury lub części ciała: pełność, większa ilość tkanki tłuszczowej, niż uznaje się za przeciętną, zaokrąglony kształt}
\Clue{19}{}{Tsuga canadensis - gatunek drzewa iglastego z rodziny sosnowatych; pochodzi z lasów Ameryki Północnej}
\Clue{20}{}{David Keith Lynch - amerykański reżyser, producent i scenarzysta filmowy oraz muzyk}
\Clue{22}{}{podstawowy element nośny konstrukcji dachu}
\Clue{23}{}{rodzaj zajęć dydaktycznych}
\Clue{24}{}{kompas}
\Clue{25}{}{w matematyce: opis krzywej za pomocą równać parametrycznych}
\Clue{26}{}{podłużne naczynie kuchenne używane do pieczenia}
\Clue{27}{}{Chrysuronia oenone - gatunek ptaka z rzędu jerzykowych (Apodiformes), z rodziny kolibrów (Trochilidae), z podrodziny kolibrów (Trochilinae)}
\Clue{30}{}{zysk - nadwyżka wpływów nad wydatkami; dodatni wynik finansowy przedsiębiorstwa, danej inwestycji lub zaciągniętej pożyczki}
\Clue{33}{}{szósty miesiąc kalendarza żydowskiego w roku przestępnym}
\Clue{34}{}{zwięzłość, zwartość, lakoniczność, lapidarność, krótkość - zwięzłość i precyzja w wyrażaniu myśli, lapidarny i oszczędny sposób wypowiadania się, według tradycji właściwy spartanom, mieszkańcom starożytnej Lakonii}
\Clue{36}{}{duży brzuch, widoczny przy nadwadze}\end{PuzzleClues}\newpage\section*{Krzyżówka 90}

\noindent\begin{Puzzle}{18}{33}|*	|*	|*	|*	|*	|*	|*	|*	|[1][S]\darr	|*	|*	|*	|*	|*	|*	|*	|*	|*	|*	|.
|*	|*	|*	|*	|*	|*	|*	|*	|l	|*	|*	|*	|*	|*	|[2][S]\darr	|*	|*	|*	|*	|.
|*	|*	|*	|*	|*	|*	|*	|*	|e	|[3][S]\rarr	|i	|m	|i	|t	|a	|t	|o	|r	|*	|.
|*	|*	|*	|*	|*	|[4][S]\darr	|[5][S]\darr	|[6][S]\rarr	|m	|a	|n	|i	|t	|o	|b	|a	|*	|*	|[7][S]\darr	|.
|*	|*	|*	|*	|*	|r	|n	|*	|u	|*	|*	|*	|*	|*	|i	|*	|*	|*	|r	|.
|*	|*	|*	|*	|*	|a	|a	|[8][S]\rarr	|r	|i	|g	|o	|r	|o	|s	|u	|m	|*	|o	|.
|*	|*	|*	|*	|*	|j	|k	|*	|k	|*	|*	|*	|*	|*	|y	|*	|*	|*	|z	|.
|*	|*	|[9][S]\darr	|*	|*	|a	|ł	|*	|a	|*	|*	|*	|*	|*	|ń	|*	|*	|*	|w	|.
|*	|*	|p	|*	|*	|[][,]{ }	|u	|*	|[][,]{ }	|*	|*	|*	|*	|[10][S]\darr	|c	|*	|*	|*	|ó	|.
|*	|*	|r	|*	|*	|g	|c	|*	|s	|*	|[11][S]\drarr	|k	|o	|s	|z	|*	|*	|*	|j	|.
|*	|*	|z	|*	|*	|w	|i	|*	|a	|*	|k	|*	|*	|u	|y	|*	|*	|*	|[][,]{ }	|.
|*	|*	|e	|[12][S]\darr	|*	|i	|e	|*	|m	|*	|w	|*	|*	|m	|k	|*	|[13][S]\darr	|*	|r	|.
|*	|*	|k	|m	|*	|a	|[][,]{ }	|*	|o	|[14][S]\rarr	|a	|m	|i	|a	|*	|*	|s	|*	|o	|.
|*	|*	|a	|e	|*	|ź	|l	|*	|t	|*	|s	|*	|*	|k	|*	|*	|k	|*	|d	|.
|*	|*	|ź	|c	|*	|d	|ę	|*	|n	|*	|[][,]{ }	|*	|*	|*	|*	|*	|r	|*	|o	|.
|*	|*	|n	|h	|*	|z	|d	|*	|a	|*	|k	|*	|*	|*	|*	|*	|z	|*	|w	|.
|*	|*	|i	|a	|*	|i	|ź	|*	|*	|[15][S]\rarr	|a	|p	|p	|e	|l	|*	|y	|[16][S]\darr	|y	|.
|*	|*	|k	|n	|*	|s	|w	|[17][S]\rarr	|k	|a	|r	|a	|w	|a	|n	|i	|n	|g	|*	|.
|*	|*	|[][,]{ }	|i	|[18][S]\drarr	|t	|i	|s	|c	|h	|b	|e	|i	|n	|*	|*	|k	|e	|*	|.
|*	|*	|b	|k	|s	|a	|o	|[19][S]\darr	|[20][S]\drarr	|p	|o	|n	|y	|*	|*	|*	|a	|n	|*	|.
|*	|*	|u	|a	|z	|*	|w	|z	|a	|*	|l	|*	|*	|*	|*	|*	|[][,]{ }	|[][,]{ }	|*	|.
|*	|*	|c	|[][,]{ }	|a	|*	|e	|a	|c	|*	|o	|*	|*	|*	|*	|*	|k	|s	|*	|.
|*	|*	|h	|f	|b	|*	|*	|k	|a	|*	|w	|[21][S]\darr	|*	|*	|*	|*	|o	|p	|*	|.
|*	|*	|h	|a	|l	|*	|*	|l	|p	|[22][S]\darr	|y	|l	|*	|*	|*	|*	|n	|r	|*	|.
|[23][S]\rarr	|k	|o	|l	|o	|k	|w	|i	|u	|m	|*	|i	|*	|*	|*	|*	|t	|z	|*	|.
|*	|*	|l	|o	|g	|*	|[24][S]\rarr	|k	|l	|i	|n	|g	|o	|n	|*	|*	|a	|ę	|*	|.
|*	|*	|z	|w	|r	|*	|*	|o	|c	|e	|*	|o	|*	|*	|*	|*	|k	|ż	|*	|.
|*	|*	|a	|a	|z	|*	|*	|w	|o	|ś	|[25][S]\rarr	|w	|i	|b	|r	|a	|t	|o	|*	|.
|*	|*	|*	|*	|b	|*	|*	|i	|*	|c	|*	|i	|*	|*	|*	|*	|o	|n	|*	|.
|*	|*	|*	|*	|i	|*	|*	|a	|*	|i	|*	|e	|*	|*	|*	|*	|w	|y	|*	|.
|*	|*	|*	|*	|e	|*	|*	|n	|*	|n	|[26][S]\rarr	|c	|h	|o	|i	|n	|a	|*	|*	|.
|*	|*	|*	|*	|t	|*	|*	|k	|*	|k	|*	|*	|*	|*	|*	|*	|*	|*	|*	|.
|*	|*	|*	|*	|*	|*	|*	|a	|*	|a	|*	|*	|*	|*	|*	|*	|*	|*	|*	|.
|*	|*	|*	|*	|*	|*	|*	|*	|*	|*	|*	|*	|*	|*	|*	|*	|*	|*	|*	|.\end{Puzzle}

\newpage

\begin{PuzzleClues}{\textbf{Poziome}\\}\Clue{3}{}{naśladowca, człowiek, który imituje kogoś lub coś}
\Clue{6}{}{prowincja w środkowej Kanadzie, pow. 650 tyś. km , stolica Winnipeg}
\Clue{8}{}{egzamin przedmiotowy dla kandydata do stopnia doktora}
\Clue{11}{}{(balonu) pomieszczenie dla załogi i wyposażenie balonu}
\Clue{14}{}{mękławka, miękławka, Amia calva - jedyny współcześnie żyjący przedstawiciel ryby z rodziny amiowatych (Amiidae); żyje w silnie zarośniętych rzekach w Ameryce Północnej, w dorzeczach Wielkich Jezior (z wyjątkiem Jeziora Górnego) i Missisipi}
\Clue{15}{}{holenderski. malarz i witrażysta ur. w 1921, współzałożyciel grupy Cobra}
\Clue{17}{}{swobodne podróżowanie bez napiętego grafiku podróży, noclegi na polach kempingowych}
\Clue{18}{}{Johann Heinrich (1722-89) malarz niemiecki; portrety dworskie, kompozycje mitologiczne i historyczne}
\Clue{20}{}{mały, łagodny i wytrzymały koń pochodzący z Wysp Brytyjskich}
\Clue{23}{}{praca pisemna, fizycznie istniejący egzemplarz sprawdzianu studenta wyższej uczelni}
\Clue{24}{}{przedstawiciel fikcyjnej rasy Klingonów w serialuStar Trek}
\Clue{25}{}{ozdobnik muzyczny, charakteryzujący się szybkimi, niewielkimi zmianami natężenia lub wysokości dźwięku}
\Clue{26}{}{gałąź, gałęzie drzewa iglastego}\end{PuzzleClues}

\begin{PuzzleClues}{\textbf{Pionowe}\\}\Clue{1}{}{Newtonia archboldi - gatunek ptaka z rodziny wang (Vangidae)}
\Clue{2}{}{mieszkaniec Abisynii, człowiek pochodzenia abisyńskiego (etiopskiego)}
\Clue{4}{}{Raja asterias - gatunek ryby chrzęstnoszkieletowej z rodziny rajowatych (Rajidae); raja gwiaździsta żyje w Morzu Śródziemnym oraz w Adriatyku, gdzie stanowi najliczniejszy gatunek z rodzaju Raja; sporadycznie występuje w rejonie Azorów oraz w rejonie zachodniego wybrzeża Półwyspu Iberyjskiego}
\Clue{5}{}{zabieg, polegający na wprowadzeniu igły punkcyjnej do przestrzeni podpajęczynkowej w odcinku lędźwiowym kręgosłupa, aby zdiagnozować pacjenta przez pobranie płynu mózgowo-rdzeniowego, podać mu leki lub wykonać znieczulenie}
\Clue{7}{}{droga rozwoju rodowego, pochodzenie i zmiany ewolucyjne grupy organizmów, zwykle gatunków}
\Clue{9}{}{przekaźnik gazowo-wydmuchowy wykorzystywany jako zabezpieczenie w transformatorach energetycznych}
\Clue{10}{}{krzew lub drzewo strefy umiarkowanej, w Polsce sadzony w parkach}
\Clue{11}{}{organiczny związek chemiczny, najprostszy związek z grupy fenoli}
\Clue{12}{}{pierwotne określenie mechaniki kwantowej, przed sformułowaniem teorii korpuskularno-falowej}
\Clue{13}{}{miejsce służące do przekazywania tajnych informacji}
\Clue{16}{}{jeden z dwóch lub więcej genów, które leżą na jednym chromosomie i dziedziczą się wspólnie}
\Clue{18}{}{orka karłowata, orka fałszywa, Pseudorca crassidens - gatunek walenia z rodziny delfinowatych, jedyny przedstawiciel rodzaju Pseudorca; żyje w wodach tropikalnych, subtropikalnych i niektórych umiarkowanych}
\Clue{19}{}{mieszkanka Zaklikowa, kobieta pochodząca z Zaklikowa}
\Clue{20}{}{miasto i port w Meksyku (Guerrero) nad Oceanem Spokojnym, światowej sławy kąpielisko i ośrodek turystyczno-wypoczynkowy}
\Clue{21}{}{zawodnik występujący wraz z drużyną w określonej lidze, posiadający doświadczenie i umiejętności, pozwalające na zaklasyfikowanie go do tej ligi}
\Clue{22}{}{zdrobniale o małej miejscowości}\end{PuzzleClues}\newpage\section*{Krzyżówka 91}

\noindent\begin{Puzzle}{20}{28}|*	|*	|*	|*	|*	|*	|[1][S]\drarr	|r	|z	|e	|c	|z	|n	|i	|c	|z	|k	|a	|*	|*	|*	|.
|*	|*	|[2][S]\darr	|*	|[3][S]\rarr	|z	|b	|i	|o	|r	|ó	|w	|k	|a	|*	|*	|*	|*	|*	|[4][S]\darr	|[5][S]\darr	|.
|*	|*	|g	|[6][S]\darr	|*	|[7][S]\darr	|e	|*	|*	|*	|*	|[8][S]\darr	|*	|[9][S]\darr	|[10][S]\darr	|*	|[11][S]\darr	|*	|*	|s	|s	|.
|*	|*	|a	|n	|[12][S]\darr	|c	|r	|*	|*	|*	|[13][S]\darr	|s	|[14][S]\darr	|h	|t	|*	|a	|*	|*	|i	|i	|.
|*	|[15][S]\rarr	|l	|e	|p	|a	|n	|t	|o	|*	|p	|p	|r	|u	|y	|*	|r	|*	|[16][S]\darr	|u	|l	|.
|*	|*	|w	|t	|o	|n	|a	|*	|[17][S]\rarr	|k	|a	|r	|e	|l	|k	|a	|*	|*	|d	|r	|n	|.
|*	|*	|a	|t	|l	|t	|r	|*	|*	|[18][S]\darr	|ź	|a	|s	|k	|*	|[19][S]\darr	|*	|*	|a	|e	|i	|.
|*	|*	|y	|o	|e	|u	|d	|*	|[20][S]\darr	|ł	|*	|w	|z	|*	|*	|m	|*	|[21][S]\darr	|t	|k	|k	|.
|*	|*	|*	|*	|*	|s	|y	|[22][S]\darr	|a	|a	|[23][S]\darr	|d	|t	|[24][S]\drarr	|ż	|e	|b	|r	|o	|*	|[][,]{ }	|.
|*	|[25][S]\darr	|*	|[26][S]\darr	|*	|[][,]{ }	|n	|s	|r	|d	|n	|z	|ó	|g	|*	|r	|*	|z	|w	|*	|p	|.
|*	|i	|*	|e	|*	|f	|*	|z	|c	|o	|a	|i	|w	|ł	|*	|y	|[27][S]\darr	|e	|n	|*	|u	|.
|*	|d	|*	|k	|*	|i	|[28][S]\darr	|m	|h	|w	|m	|a	|k	|u	|*	|s	|k	|ź	|i	|[29][S]\darr	|l	|.
|*	|e	|*	|w	|*	|r	|b	|i	|i	|a	|i	|n	|a	|p	|*	|t	|o	|b	|k	|p	|s	|.
|*	|m	|*	|i	|[30][S]\drarr	|m	|a	|r	|t	|r	|e	|*	|*	|t	|*	|e	|z	|a	|*	|o	|a	|.
|*	|[][,]{ }	|*	|n	|r	|u	|w	|a	|e	|k	|s	|[31][S]\darr	|*	|a	|*	|m	|a	|[][,]{ }	|*	|l	|c	|.
|*	|p	|*	|o	|ó	|s	|o	|*	|k	|a	|t	|b	|*	|k	|[32][S]\darr	|[][,]{ }	|[][,]{ }	|k	|*	|s	|y	|.
|*	|e	|*	|k	|ż	|*	|l	|*	|t	|*	|n	|j	|*	|[][,]{ }	|n	|b	|s	|r	|*	|k	|j	|.
|*	|r	|[33][S]\darr	|c	|a	|*	|i	|*	|o	|*	|i	|e	|[34][S]\darr	|z	|i	|o	|a	|a	|*	|o	|n	|.
|*	|[][,]{ }	|s	|j	|n	|*	|c	|[35][S]\darr	|n	|[36][S]\darr	|c	|l	|o	|w	|e	|c	|a	|s	|*	|ś	|y	|.
|*	|i	|t	|u	|i	|[37][S]\rarr	|a	|h	|i	|s	|t	|o	|r	|y	|c	|z	|n	|o	|ś	|ć	|*	|.
|[38][S]\rarr	|d	|o	|m	|e	|k	|*	|e	|k	|t	|w	|v	|z	|c	|z	|n	|e	|w	|*	|*	|*	|.
|*	|e	|p	|*	|c	|*	|*	|g	|a	|r	|o	|a	|e	|z	|y	|y	|ń	|a	|[39][S]\darr	|[40][S]\darr	|*	|.
|*	|m	|k	|*	|*	|*	|[41][S]\rarr	|e	|*	|o	|*	|r	|c	|a	|n	|*	|s	|*	|m	|s	|*	|.
|*	|*	|a	|[42][S]\rarr	|m	|u	|l	|m	|e	|j	|n	|*	|h	|j	|n	|[43][S]\rarr	|k	|r	|u	|k	|*	|.
|*	|*	|*	|[44][S]\rarr	|c	|e	|r	|o	|w	|n	|i	|a	|*	|n	|o	|*	|a	|*	|s	|i	|*	|.
|*	|[45][S]\rarr	|f	|i	|o	|r	|e	|n	|t	|i	|n	|a	|*	|y	|ś	|*	|*	|*	|z	|e	|*	|.
|[46][S]\rarr	|ź	|r	|ó	|d	|ł	|o	|*	|*	|c	|*	|*	|*	|*	|ć	|*	|*	|*	|k	|n	|*	|.
|*	|*	|*	|*	|*	|*	|[47][S]\rarr	|t	|r	|a	|w	|e	|r	|s	|*	|*	|*	|*	|a	|*	|*	|.
|[48][S]\rarr	|s	|t	|o	|s	|u	|n	|e	|k	|*	|*	|*	|*	|*	|*	|*	|*	|*	|*	|*	|*	|.\end{Puzzle}

\newpage

\begin{PuzzleClues}{\textbf{Poziome}\\}\Clue{1}{}{sympatyczka, obrończyni jakichś poglądów, jakiejś opcji}
\Clue{3}{}{wydarzenie, czynność, zajęcie, akcja, działanie, które wykonuje się w grupie; bardzo często mówi się tak na grupowe polowanie albo grupową wycieczkę}
\Clue{15}{}{miasto w Grecji, miejsce zwycięstwa Ligi Świętej nad f1otą turecką Alego Pasy}
\Clue{17}{}{mieszkanka Karelii, kobieta pochodzenia karelskiego}
\Clue{24}{}{część skrzydła niektórych samolotów lub szybowców}
\Clue{30}{}{jezioro w Kanadzie połączone z Wielkim Jeziorem Niewolniczym}
\Clue{37}{}{cecha tego, co istnieje poza historią; którego istnienie nie ma wymiaru historycznego}
\Clue{38}{}{rodzaj ciasta przygotowywanego na zimno z herbatników i masy (głównie serowej, np. z serka homogenizowanego) oraz dodatków (polewy, cukierków do ozdoby itp.), który przypomina kształetm domek - jest zazwyczaj trójkątny w przekroju, przywodzi na myśl swoisty daszek; ciasto popularne szczególnie jako coś słodkiego dla dzieci}
\Clue{41}{}{jednostka liczności fotonów}
\Clue{42}{}{miasto i port w Birmie nad Morzem Andamańskim, ośrodek administracyjny obwodu narodowościowego Mon}
\Clue{43}{}{największy ptak z wróblowatych; w Polsce rzadki, chroniony}
\Clue{44}{}{zakład, w którym ceruje się, naprawia zniszczoną odzież}
\Clue{45}{}{włoski klub piłkarski założony 26 sierpnia 1926 roku jako AC Fiorentina}
\Clue{46}{}{przyczyna, powód do czegoś}
\Clue{47}{}{to, co łączy równoległe elementy maszyny}
\Clue{48}{}{zależność wynikająca z porównania dwóch wartości}\end{PuzzleClues}

\begin{PuzzleClues}{\textbf{Pionowe}\\}\Clue{1}{}{członek zakonu będącego odłamem franciszkanów nazwany tak od kościoła św. Bernardyna w Krakowie}
\Clue{2}{}{zatoka Oceanu Atlantyckiego u zach. wybrzeży Irlandii, główny port Galway}
\Clue{4}{}{młody chłopiec (negatywnie nacechowane określenie)}
\Clue{5}{}{odmiana silników odrzutowych niewyposażona w zespół sprężarki}
\Clue{6}{}{cena bez podatku VAT}
\Clue{7}{}{linia melodyczna, do której dokomponowuje się kontrapunkt}
\Clue{8}{}{w procesie dydaktycznym: ćwiczenie, które ma na celu sprawdzenie wiedzy i umiejętności uczniów}
\Clue{9}{}{okręt nie nadający się do żeglugi, służący jako magazyn lub mieszkanie}
\Clue{10}{}{gęste zwykle czerwone lub różowe płótno, z którego szyje się wsypy}
\Clue{11}{}{jednostka powierzchni używana między innymi w rolnictwie, geodezji i leśnictwie, oznaczana symbolem a; 1 a = 1 dam2 = 100 m2 = 0,01 ha}
\Clue{12}{}{obszar, teren, fragment powierzchni przeznaczony na coś lub pokryty czymś, wydzielony na podstawie jakiegoś faktu, który właśnie tam zaistniał}
\Clue{13}{}{duży motyl dzienny, żółty z czarnym odcieniem, gąsienice żerują na baldaszkowatych}
\Clue{14}{}{część gruntu pozostała po podziale}
\Clue{16}{}{mechanizm pokazujący datę, np. w zegarku lub aparacie fotograficznym}
\Clue{18}{}{urządzenie służące do wprowadzenia energii elektrycznej do  akumulatora elektrycznego}
\Clue{19}{}{tkanka twórcza roślinna tworząca osiowy cylinder wewnątrz organów i biorąca udział we wtórnym przyroście łodygi i korzeni na grubość}
\Clue{20}{}{kompozcja, układ, zestawienie elementów budowli}
\Clue{21}{}{rzeźba terenu, której ukształtowanie związane jest z działaniem wody w wyniku procesów krasowienia}
\Clue{22}{}{KICZ; obraz bez żadnej wartości artystycznej}
\Clue{23}{}{biuro namiestnika}
\Clue{24}{}{głuptak, głuptak biały, Morus bassanus - gatunek ptaka z rodziny głuptaków (Sulidae); kolonie lęgowe znajdują się na wybrzeżach Wysp Brytyjskich, Norwegii, Islandii, Nowej Fundlandii, RPA, Australii, Tasmanii i Nowej Zelandii}
\Clue{25}{}{błąd logiczny polegający na użyciu w definicji równościowej wyrazu definiowanego w definiensie}
\Clue{26}{}{równonoc, - zrównanie dnia z nocą}
\Clue{27}{}{rasa kozy domowej, jedna z najbardziej znanych i cenionych ras mlecznych; wyhodowana w Szwajcarii w dolinie rzeki Saane, ale jej chów rozpowszechnił się w Europie, północnej Afryce, Stanach Zjednoczonych, Korei oraz Indonezji}
\Clue{28}{}{samica bawoła}
\Clue{29}{}{polska litera diakrytyzowana}
\Clue{30}{}{modlitwa maryjna duchowo jednocząca chrześcijanina z osobą Matki Jezusa i kontemplująca tajemnice zbawienia}
\Clue{31}{}{miasto w Chorwacji; przemysł spożywczy, włókienniczy, metalowy}
\Clue{32}{}{cecha czegoś, co nie świadczy usług, jest zamknięte}
\Clue{33}{}{bezpiecznik elektryczny}
\Clue{34}{}{w botanice: rodzaj suchych owoców zamkniętych (niepękających), jednonasiennych (rzadko dwunasiennych), odpadających od rośliny matecznej w całości}
\Clue{35}{}{grupa ludzi, np. instytucja, organizacja, firma, grupa społeczna, zawodowa, która ma zdecydowaną przewagę nad innymi lub wyłączność w jakiejś dziedzinie}
\Clue{36}{}{część ramy w harfie, na której znajdują się kołki przytrzymujące struny}
\Clue{39}{}{sztuczna przynęta wędkarska}
\Clue{40}{}{miasto i port w Norwegii na rzek Skien, ośrodek administracyjny okręgu Telemark}\end{PuzzleClues}\newpage\section*{Krzyżówka 92}

\noindent\begin{Puzzle}{25}{23}|*	|*	|*	|*	|*	|[1][S]\darr	|*	|*	|*	|*	|[2][S]\drarr	|w	|e	|l	|i	|n	|g	|t	|o	|n	|i	|a	|*	|*	|*	|[3][S]\darr	|.
|*	|*	|*	|*	|*	|t	|*	|*	|*	|[4][S]\rarr	|p	|o	|s	|a	|g	|*	|*	|[5][S]\darr	|*	|*	|*	|[6][S]\darr	|[7][S]\darr	|*	|*	|w	|.
|*	|*	|*	|*	|*	|y	|*	|*	|*	|*	|u	|*	|[8][S]\darr	|*	|*	|*	|*	|g	|[9][S]\darr	|*	|*	|d	|w	|*	|[10][S]\darr	|s	|.
|*	|[11][S]\rarr	|a	|b	|s	|c	|e	|s	|*	|*	|n	|*	|f	|*	|*	|*	|*	|i	|z	|*	|*	|o	|y	|*	|i	|p	|.
|*	|[12][S]\rarr	|b	|r	|ą	|z	|o	|w	|n	|i	|k	|*	|e	|*	|*	|*	|*	|a	|b	|*	|*	|k	|c	|*	|n	|ó	|.
|*	|*	|*	|*	|[13][S]\rarr	|k	|l	|a	|j	|s	|t	|e	|r	|*	|*	|[14][S]\drarr	|t	|e	|r	|m	|*	|t	|i	|*	|s	|ł	|.
|*	|*	|*	|*	|*	|a	|*	|*	|*	|*	|[][,]{ }	|*	|m	|*	|*	|t	|*	|v	|o	|*	|*	|o	|s	|*	|t	|w	|.
|*	|*	|*	|*	|*	|*	|*	|*	|*	|*	|g	|*	|e	|*	|*	|r	|*	|e	|j	|*	|*	|r	|k	|*	|y	|ł	|.
|*	|*	|*	|*	|*	|*	|*	|*	|[15][S]\rarr	|b	|r	|o	|n	|c	|h	|o	|g	|r	|a	|f	|i	|a	|*	|*	|n	|a	|.
|*	|*	|*	|*	|*	|*	|*	|*	|[16][S]\rarr	|k	|a	|f	|t	|a	|n	|i	|k	|*	|[][,]{ }	|*	|*	|t	|*	|*	|k	|d	|.
|*	|*	|*	|*	|*	|[17][S]\rarr	|ś	|c	|i	|a	|n	|a	|*	|*	|*	|c	|*	|*	|t	|*	|*	|*	|[18][S]\darr	|*	|t	|c	|.
|*	|[19][S]\drarr	|z	|a	|t	|r	|a	|c	|e	|n	|i	|e	|*	|*	|*	|k	|*	|*	|u	|*	|*	|[20][S]\darr	|s	|*	|o	|a	|.
|*	|a	|[21][S]\rarr	|ż	|ó	|ł	|t	|o	|l	|i	|c	|z	|k	|a	|*	|*	|*	|[22][S]\drarr	|r	|y	|b	|o	|ł	|ó	|w	|*	|.
|*	|s	|*	|[23][S]\rarr	|w	|s	|p	|ó	|ł	|r	|z	|ę	|d	|n	|a	|*	|*	|u	|n	|*	|*	|s	|u	|*	|n	|*	|.
|*	|z	|*	|[24][S]\drarr	|c	|z	|a	|t	|o	|w	|n	|i	|k	|*	|*	|*	|*	|p	|i	|*	|*	|a	|p	|[25][S]\darr	|o	|*	|.
|*	|*	|*	|m	|*	|*	|*	|*	|*	|*	|y	|*	|*	|*	|*	|[26][S]\darr	|*	|ł	|e	|*	|*	|k	|i	|s	|ś	|*	|.
|*	|[27][S]\rarr	|k	|a	|s	|o	|w	|n	|i	|k	|*	|*	|*	|*	|*	|p	|*	|a	|j	|*	|*	|a	|s	|y	|ć	|*	|.
|*	|*	|*	|r	|*	|[28][S]\rarr	|u	|s	|z	|a	|t	|e	|k	|[][,]{ }	|b	|r	|ą	|z	|o	|w	|y	|*	|k	|n	|*	|*	|.
|*	|*	|[29][S]\rarr	|s	|a	|m	|o	|c	|h	|o	|d	|z	|i	|a	|r	|z	|*	|*	|w	|*	|*	|*	|o	|t	|*	|*	|.
|*	|[30][S]\rarr	|d	|z	|i	|u	|p	|l	|a	|*	|[31][S]\rarr	|s	|i	|e	|d	|e	|m	|n	|a	|s	|t	|y	|*	|e	|*	|*	|.
|*	|*	|[32][S]\rarr	|a	|b	|a	|k	|a	|ń	|c	|z	|y	|k	|*	|*	|r	|*	|*	|*	|*	|*	|*	|*	|t	|*	|*	|.
|*	|[33][S]\rarr	|i	|n	|w	|e	|r	|s	|j	|a	|*	|[34][S]\rarr	|c	|z	|w	|ó	|r	|k	|a	|*	|*	|*	|*	|y	|*	|*	|.
|[35][S]\rarr	|m	|e	|d	|r	|e	|s	|a	|*	|[36][S]\rarr	|p	|s	|i	|z	|ą	|b	|[][,]{ }	|k	|a	|u	|k	|a	|s	|k	|i	|*	|.
|*	|*	|*	|*	|*	|*	|*	|*	|*	|*	|*	|*	|*	|*	|*	|*	|*	|*	|*	|*	|*	|*	|*	|*	|*	|*	|.\end{Puzzle}

\newpage

\begin{PuzzleClues}{\textbf{Poziome}\\}\Clue{2}{}{mamutowiec, sekwojadendron, Sequoiadendron, Wellingtonia - długowieczne drzewo iglaste; rodzaj z rodziny cyprysowatych, obejmujący jeden gatunek}
\Clue{4}{}{majątek wnoszony przez żonę przy zawieraniu małżeństwa}
\Clue{11}{}{ropień}
\Clue{12}{}{rzemieślnik zajmujący się produkcją przedmiotów z brązu i pokrywaniem obiektów brązem}
\Clue{13}{}{klej uzyskiwany ze skrobi roślinnej zmieszanej z wodą}
\Clue{14}{}{formuła nazwowa}
\Clue{15}{}{rentgen płuc i oskrzeli, przy którym stosuje się środek cieniujący i znieczulający}
\Clue{16}{}{rodzaj niemowlęcej koszulki}
\Clue{17}{}{przen. jednolita powierzchnia także: grupa ludzi lub obiektów, która stoi w zwartym szyku i (najczęściej) odgradza coś od czegoś}
\Clue{19}{}{stan, w którym traci się poczucie rzeczywistości na skutek bardzo mocnego zainteresowania się kimś lub czymś}
\Clue{21}{}{Ognorhynchus icterotis - gatunek ptaka z rodziny papugowatych (Psittacidae), z podrodziny papug neotropikalnych (Arinae)}
\Clue{22}{}{rybołów zwyczajny, Pandion haliaetus - gatunek dużego, wędrownego ptaka drapieżnego z rodziny rybołowów (Pandionidae)}
\Clue{23}{}{liczba wskazująca położenie punktu względem osi układu współrzędnych}
\Clue{24}{}{użytkownik czatu}
\Clue{27}{}{znak muzyczny w notacji muzycznej, który anuluje działanie innego znaku muzycznego np. krzyżyka}
\Clue{28}{}{Colibri delphinae - gatunek ptaka z rzędu jerzykowych (Apodiformes), z rodziny kolibrów (Trochilidae), z podrodziny kolibrów (Trochilinae)}
\Clue{29}{}{osoba, która zawodowo zajmuje się naprawą samochodów}
\Clue{30}{}{pusta komora wewnątrz pnia lub gałęzi drzewa, powstała w wyniku uszkodzenia, rozkładu drzewa, a także poprzez wykucie przez dzięciołowate}
\Clue{31}{}{siedemnasty dzień (najczęściej bieżącego lub przyszłego) miesiąca}
\Clue{32}{}{mieszkaniec Abakanu}
\Clue{33}{}{rodzaj przekształcenia geometrycznego}
\Clue{34}{}{w niektórych konkurencjach sportowych: czteroosobowa osada}
\Clue{35}{}{teologiczna, wyższa szkoła muzułmańska}
\Clue{36}{}{Erythronium caucasicum - gatunek roślin z rodziny liliowatych}\end{PuzzleClues}

\begin{PuzzleClues}{\textbf{Pionowe}\\}\Clue{1}{}{określenie osoby bardzo wysokiej i szczupłej}
\Clue{2}{}{punkt określający przebieg granicy nieruchomości}
\Clue{3}{}{osoba, która rządzi danym państwem razem z kimś innym (lub kilkoma innymi osobami), równym mu formalnie rangą i pozycją}
\Clue{5}{}{amerykański fizyk teoretyk ur. w 1929 r., badacz zjawisk zachodzących w półprzewodnikach i nadprzewodnikach, laureat nagrody Nobla}
\Clue{6}{}{stopień doktora w jakiejś dziedzinie nauki}
\Clue{7}{}{przymus, nacisk}
\Clue{8}{}{dawna nazwa enzymu}
\Clue{9}{}{typowa zbroja płytowa z hakiem do zawieszenia ciężkiej kopii na wysokości prawej piersi}
\Clue{10}{}{cecha tego, co jest instynktowne, podyktowane instynktem}
\Clue{14}{}{miasto w Rosji, w obwodzie czelabińskim, nad Ujem, przy ujściu Uwielki}
\Clue{18}{}{zgrubiale: słup - (zazwyczaj) pionowy element konstrukcyjny budowli w kształcie walca lub graniastosłupa}
\Clue{19}{}{pisarz żydowski (1880-1957), trylogia „Przed potopem”}
\Clue{20}{}{zatoka u wybrzeży Honsiu - Wewnętrzne Morze Japońskie}
\Clue{22}{}{górskie trawiaste zbocze, stok}
\Clue{24}{}{kupiec zajmujący się handlem dziełami sztuki}
\Clue{25}{}{sztuczny materiał; tkanina lub dzianina wykonana z włókien syntetycznych}
\Clue{26}{}{przerabianie surowca na produkt lub półprodukt, np. przerób surówki na stal}\end{PuzzleClues}\newpage\section*{Krzyżówka 93}

\noindent\begin{Puzzle}{24}{33}|*	|*	|*	|*	|*	|*	|*	|*	|*	|*	|*	|*	|[1][S]\darr	|*	|*	|*	|*	|*	|*	|*	|*	|*	|*	|*	|*	|.
|*	|*	|*	|*	|*	|*	|*	|*	|*	|*	|*	|*	|c	|*	|*	|*	|*	|*	|*	|*	|*	|*	|[2][S]\darr	|*	|*	|.
|*	|[3][S]\darr	|*	|*	|*	|*	|*	|*	|*	|*	|*	|*	|z	|*	|*	|*	|*	|*	|*	|*	|*	|*	|p	|*	|*	|.
|*	|o	|*	|*	|*	|*	|*	|[4][S]\rarr	|o	|d	|n	|i	|e	|s	|i	|e	|n	|i	|e	|*	|*	|*	|i	|[5][S]\darr	|*	|.
|*	|d	|*	|*	|*	|*	|*	|*	|[6][S]\rarr	|a	|w	|e	|r	|r	|o	|i	|z	|m	|*	|*	|*	|*	|e	|z	|*	|.
|*	|p	|*	|*	|*	|*	|*	|*	|[7][S]\darr	|*	|*	|*	|n	|*	|*	|*	|*	|*	|*	|*	|*	|*	|r	|a	|*	|.
|*	|o	|*	|*	|*	|*	|*	|*	|w	|*	|*	|*	|i	|*	|*	|*	|*	|*	|*	|*	|*	|*	|ś	|m	|*	|.
|*	|w	|*	|*	|*	|[8][S]\rarr	|n	|a	|r	|a	|m	|i	|e	|n	|n	|i	|k	|*	|*	|*	|*	|*	|c	|a	|*	|.
|*	|i	|*	|*	|*	|*	|*	|*	|ó	|*	|*	|*	|j	|*	|*	|*	|*	|*	|[9][S]\darr	|*	|*	|*	|i	|c	|*	|.
|*	|e	|*	|*	|*	|*	|*	|*	|b	|*	|*	|*	|e	|*	|*	|*	|*	|*	|w	|*	|*	|*	|e	|h	|*	|.
|*	|d	|*	|*	|*	|*	|*	|*	|l	|*	|*	|[10][S]\darr	|w	|*	|*	|*	|*	|*	|i	|*	|*	|*	|ń	|[][,]{ }	|*	|.
|*	|z	|*	|*	|*	|*	|*	|*	|o	|[11][S]\darr	|[12][S]\rarr	|p	|i	|e	|n	|i	|ą	|ż	|e	|k	|*	|*	|[][,]{ }	|s	|[13][S]\darr	|.
|*	|i	|*	|*	|*	|*	|[14][S]\rarr	|k	|w	|a	|d	|r	|a	|n	|t	|*	|*	|*	|l	|*	|*	|*	|w	|a	|s	|.
|*	|a	|*	|*	|*	|*	|*	|*	|a	|b	|*	|ą	|n	|*	|*	|*	|*	|*	|k	|*	|*	|*	|i	|m	|z	|.
|[15][S]\rarr	|l	|i	|n	|g	|w	|i	|s	|t	|a	|*	|t	|i	|*	|*	|*	|*	|*	|i	|*	|*	|*	|e	|o	|y	|.
|*	|n	|*	|*	|*	|*	|*	|*	|e	|ż	|*	|n	|n	|*	|*	|*	|*	|*	|[][,]{ }	|*	|*	|*	|l	|b	|n	|.
|*	|o	|*	|*	|*	|*	|*	|*	|*	|u	|*	|i	|*	|*	|*	|*	|*	|*	|w	|*	|[16][S]\darr	|*	|o	|ó	|k	|.
|*	|ś	|*	|*	|*	|*	|*	|*	|*	|r	|*	|k	|*	|*	|*	|*	|*	|*	|y	|*	|ś	|*	|m	|j	|a	|.
|*	|ć	|*	|*	|*	|*	|*	|*	|*	|*	|*	|[][,]{ }	|*	|*	|*	|*	|*	|*	|b	|*	|w	|*	|i	|c	|r	|.
|*	|[][,]{ }	|*	|*	|*	|*	|*	|*	|*	|*	|*	|b	|*	|*	|*	|*	|*	|*	|u	|*	|i	|*	|a	|z	|z	|.
|*	|d	|*	|*	|[17][S]\darr	|[18][S]\rarr	|c	|h	|o	|n	|d	|r	|y	|t	|[][,]{ }	|z	|w	|y	|c	|z	|a	|j	|n	|y	|*	|.
|*	|y	|*	|*	|w	|*	|*	|*	|*	|*	|*	|o	|*	|*	|*	|*	|*	|*	|h	|*	|d	|*	|ó	|*	|*	|.
|*	|s	|*	|*	|i	|*	|*	|*	|*	|*	|*	|d	|*	|*	|*	|*	|*	|*	|*	|*	|o	|*	|w	|*	|*	|.
|*	|c	|[19][S]\rarr	|m	|e	|t	|y	|l	|d	|o	|p	|a	|*	|*	|*	|*	|*	|*	|*	|*	|m	|*	|*	|*	|*	|.
|*	|y	|*	|*	|c	|*	|*	|*	|*	|*	|*	|w	|*	|*	|*	|*	|*	|*	|*	|*	|o	|*	|*	|*	|*	|.
|*	|p	|*	|*	|u	|*	|*	|*	|*	|*	|*	|k	|*	|*	|*	|*	|*	|*	|*	|*	|ś	|*	|*	|*	|*	|.
|*	|l	|*	|*	|j	|*	|*	|*	|[20][S]\rarr	|w	|r	|o	|c	|ł	|a	|w	|s	|k	|o	|ś	|ć	|*	|*	|*	|*	|.
|*	|i	|*	|*	|ą	|*	|*	|*	|*	|*	|*	|w	|*	|*	|*	|*	|*	|*	|*	|*	|*	|*	|*	|*	|*	|.
|*	|n	|*	|*	|c	|*	|*	|*	|*	|*	|*	|a	|*	|*	|*	|*	|*	|*	|*	|*	|*	|*	|*	|*	|*	|.
|*	|a	|*	|*	|y	|*	|*	|*	|*	|*	|*	|t	|*	|*	|*	|*	|*	|*	|*	|*	|*	|*	|*	|*	|*	|.
|*	|r	|*	|*	|*	|*	|*	|*	|*	|*	|*	|y	|*	|*	|*	|*	|*	|*	|*	|*	|*	|*	|*	|*	|*	|.
|*	|n	|*	|*	|*	|*	|*	|*	|*	|*	|*	|*	|*	|*	|*	|*	|*	|*	|*	|*	|*	|*	|*	|*	|*	|.
|*	|a	|*	|*	|*	|*	|*	|*	|*	|*	|*	|*	|*	|*	|*	|*	|*	|*	|*	|*	|*	|*	|*	|*	|*	|.
|*	|*	|*	|*	|*	|*	|*	|*	|*	|*	|*	|*	|*	|*	|*	|*	|*	|*	|*	|*	|*	|*	|*	|*	|*	|.\end{Puzzle}

\newpage

\begin{PuzzleClues}{\textbf{Poziome}\\}\Clue{4}{}{coś (realnego lub abstrakcyjnego), do czego można się odnieść}
\Clue{6}{}{doktryna filozoficzna popularna pod koniec XIII wieku, oparta na interpretacji myśli Arystotelesa dokonanych przez Awerroesa}
\Clue{8}{}{element odzieży (zwłaszcza mundurów), który stanowi pasek materiału wszyty z jednej strony między górny koniec rękawa kurtki mundurowej a samą kurtkę, a drugim końcem przypinany, najczęściej guzikiem, w pobliżu kołnierza, tak że nakrywa ramię wzdłuż obojczyka osoby ubranej w tę kurtkę}
\Clue{12}{}{drobny pieniądz}
\Clue{14}{}{ćwiartka koła; część płaszczyzny, która jest ograniczona osiami układu współrzędnych}
\Clue{15}{}{osoba specjalizująca się w nauce o języku}
\Clue{18}{}{meteoryt kamienny należący do chondrytów, stanowi ok. 80\% wszystkich znajdowanych meteorytów}
\Clue{19}{}{lek hipotensyjny, który wskutek podobieństwa do dihydroksyfenyloalaniny (DOPA) blokuje powstawanie dopaminy, która jest prekursorem noradrenaliny}
\Clue{20}{}{zespół cech typowych dla Wrocławia, wrocławian lub czegoś wrocławskiego}\end{PuzzleClues}

\begin{PuzzleClues}{\textbf{Pionowe}\\}\Clue{1}{}{mieszkaniec Czerniejewa, człowiek pochodzący z Czerniejewa}
\Clue{2}{}{pierścień określony na zbiorze wielomianów jednej lub więcej zmiennych o współczynnikach z ustalonego pierścienia}
\Clue{3}{}{odpowiedzialność oparta na zasadzie winy za postępowanie uchybiające obowiązkom pracownika wobec pracodawcy lub sprzeczne z zasadami etyki i godności zawodowej}
\Clue{5}{}{akt popełnienia samobójstwa, zamach na własne życie}
\Clue{7}{}{rząd kosmopolitycznych ptaków, gniazdowniki}
\Clue{9}{}{eksplozja, podczas której wyłonił się Wszechświat}
\Clue{10}{}{Bryum mamillatum - gatunek mchu z rodziny prątnikowatych}
\Clue{11}{}{osłona lampy służąca do jej zasłonięcia przed bezpośrednim widzeniem}
\Clue{13}{}{właściciel szynku}
\Clue{16}{}{stan psychiczny, w którym człowiek zdaje sobie sprawę ze zjawisk wewnętrznych, jak np. procesy myślowe}
\Clue{17}{}{ten, kto uczestniczy w wiecu}\end{PuzzleClues}\newpage\section*{Krzyżówka 94}

\noindent\begin{Puzzle}{24}{29}|*	|*	|*	|*	|*	|*	|*	|*	|*	|*	|*	|*	|*	|*	|*	|[1][S]\drarr	|z	|ą	|b	|*	|*	|[2][S]\darr	|[3][S]\darr	|*	|[4][S]\darr	|.
|*	|*	|*	|[5][S]\darr	|*	|*	|*	|[6][S]\rarr	|o	|s	|k	|r	|o	|b	|y	|w	|a	|n	|i	|e	|*	|g	|s	|*	|t	|.
|*	|*	|*	|d	|*	|*	|[7][S]\darr	|*	|*	|*	|[8][S]\rarr	|m	|u	|r	|z	|i	|m	|*	|*	|*	|*	|ł	|a	|*	|a	|.
|*	|[9][S]\rarr	|p	|e	|d	|o	|f	|i	|l	|i	|a	|*	|[10][S]\rarr	|k	|r	|e	|d	|a	|*	|*	|*	|ó	|r	|*	|n	|.
|*	|*	|*	|i	|[11][S]\rarr	|z	|a	|p	|o	|b	|i	|e	|g	|l	|i	|w	|o	|ś	|ć	|*	|*	|w	|d	|*	|t	|.
|*	|*	|*	|k	|*	|*	|s	|[12][S]\rarr	|d	|o	|b	|r	|a	|[][,]{ }	|w	|i	|a	|r	|a	|*	|*	|k	|e	|*	|a	|.
|*	|[13][S]\rarr	|o	|s	|m	|o	|z	|a	|*	|*	|[14][S]\darr	|*	|*	|*	|[15][S]\darr	|ó	|[16][S]\rarr	|t	|k	|e	|m	|a	|l	|i	|*	|.
|*	|*	|*	|a	|[17][S]\drarr	|ż	|y	|d	|o	|w	|s	|k	|i	|[][,]{ }	|p	|r	|o	|c	|e	|n	|t	|*	|a	|*	|*	|.
|*	|[18][S]\darr	|*	|*	|i	|*	|n	|*	|[19][S]\drarr	|s	|o	|n	|g	|*	|o	|k	|[20][S]\darr	|*	|*	|[21][S]\darr	|[22][S]\darr	|*	|*	|[23][S]\darr	|*	|.
|*	|s	|*	|*	|b	|*	|a	|*	|r	|*	|n	|*	|*	|[24][S]\drarr	|p	|a	|s	|i	|b	|r	|z	|u	|c	|h	|*	|.
|*	|t	|*	|*	|e	|*	|*	|*	|u	|*	|d	|*	|*	|d	|c	|*	|a	|*	|*	|e	|n	|*	|*	|a	|*	|.
|*	|a	|*	|[25][S]\darr	|r	|[26][S]\rarr	|t	|o	|n	|g	|a	|*	|*	|o	|o	|*	|l	|*	|*	|c	|o	|*	|[27][S]\darr	|r	|*	|.
|[28][S]\drarr	|p	|e	|k	|i	|ń	|c	|z	|y	|k	|*	|[29][S]\darr	|*	|c	|r	|*	|o	|*	|*	|e	|s	|[30][S]\darr	|m	|m	|*	|.
|t	|e	|[31][S]\darr	|r	|a	|*	|[32][S]\darr	|*	|*	|*	|*	|f	|*	|h	|n	|*	|n	|[33][S]\darr	|*	|p	|*	|ż	|o	|o	|*	|.
|r	|l	|u	|ę	|*	|[34][S]\darr	|f	|*	|*	|*	|*	|o	|*	|ó	|*	|*	|*	|t	|*	|c	|*	|ó	|c	|n	|*	|.
|a	|*	|c	|t	|*	|f	|o	|*	|*	|*	|*	|l	|*	|d	|*	|*	|*	|e	|*	|j	|*	|ł	|h	|i	|*	|.
|n	|[35][S]\drarr	|z	|a	|g	|a	|r	|n	|i	|ę	|c	|i	|e	|[][,]{ }	|m	|i	|e	|n	|i	|a	|*	|w	|w	|k	|*	|.
|k	|k	|u	|r	|*	|l	|m	|*	|*	|*	|*	|a	|*	|g	|*	|[36][S]\rarr	|j	|o	|n	|*	|[37][S]\darr	|[][,]{ }	|i	|a	|*	|.
|w	|ą	|c	|z	|*	|a	|a	|*	|*	|*	|[38][S]\rarr	|r	|e	|w	|i	|r	|*	|r	|[39][S]\rarr	|a	|l	|b	|a	|*	|*	|.
|i	|t	|i	|[][,]{ }	|*	|*	|*	|*	|[40][S]\drarr	|f	|u	|z	|j	|a	|*	|*	|*	|k	|*	|*	|e	|r	|n	|[41][S]\darr	|*	|.
|l	|[][,]{ }	|o	|m	|[42][S]\rarr	|p	|i	|r	|o	|t	|*	|*	|[43][S]\rarr	|r	|a	|k	|*	|a	|*	|*	|j	|u	|o	|p	|*	|.
|i	|k	|w	|n	|[44][S]\rarr	|n	|i	|e	|d	|o	|ś	|p	|i	|a	|ł	|e	|k	|*	|*	|*	|*	|n	|w	|r	|*	|.
|z	|u	|o	|i	|*	|*	|*	|*	|p	|*	|*	|*	|[45][S]\rarr	|n	|i	|e	|l	|s	|e	|n	|*	|a	|a	|ę	|*	|.
|a	|r	|ś	|e	|[46][S]\rarr	|w	|y	|d	|r	|z	|y	|k	|*	|t	|*	|*	|*	|*	|*	|*	|*	|t	|t	|t	|*	|.
|t	|s	|ć	|j	|*	|*	|[47][S]\drarr	|w	|a	|r	|z	|y	|w	|o	|[][,]{ }	|k	|a	|p	|u	|s	|t	|n	|e	|*	|*	|.
|o	|o	|*	|s	|*	|*	|s	|*	|w	|*	|*	|*	|*	|w	|*	|[48][S]\rarr	|ł	|o	|w	|c	|z	|y	|*	|*	|*	|.
|r	|w	|*	|z	|[49][S]\rarr	|b	|o	|g	|a	|t	|y	|n	|i	|a	|n	|k	|a	|*	|*	|*	|*	|*	|*	|*	|*	|.
|*	|y	|*	|y	|*	|*	|l	|*	|*	|[50][S]\rarr	|r	|ó	|ż	|n	|o	|w	|i	|c	|i	|o	|w	|c	|e	|*	|*	|.
|*	|*	|*	|*	|*	|*	|*	|*	|*	|[51][S]\rarr	|p	|i	|r	|y	|d	|o	|k	|s	|a	|m	|i	|n	|a	|*	|*	|.
|*	|*	|[52][S]\rarr	|o	|b	|s	|e	|r	|w	|a	|c	|j	|a	|*	|*	|*	|*	|*	|*	|*	|*	|*	|*	|*	|*	|.\end{Puzzle}

\newpage

\begin{PuzzleClues}{\textbf{Poziome}\\}\Clue{1}{}{złożony, twardy twór anatomiczny występujący u większości kręgowców (z wyjątkiem współczesnych ptaków, żółwi, ropuch i niektórych ssaków), stanowiący element układu trawienia i służący do rozdrabniania pożywienia}
\Clue{6}{}{usuwanie zewnętrznej warstwy czegoś za pomocą ostrego przedmiotu}
\Clue{8}{}{jedna z gwiazd w gwiazdozbiorze Psa Wielkiego}
\Clue{9}{}{stan, w którym jedynym lub preferowanym sposobem osiągania satysfakcji seksualnej jest kontakt z dziećmi w okresie przedpokwitaniowym lub wczesnej fazie pokwitania}
\Clue{10}{}{materiał przeznaczony do pisania po tablicy szkolnej, zbudowany z wapieni, pozyskiwany ze złóż skalistej kredy}
\Clue{11}{}{cecha człowieka: to, że ktoś jest przewidujący, przygotowuje się na różne okoliczności}
\Clue{12}{}{błędne, ale usprawiedliwione przekonanie jakiejś osoby, że przysługuje jej określone prawo podmiotowe}
\Clue{13}{}{dyfuzja rozpuszczalnika przez błonę półprzepuszczalną rozdzielającą dwa roztwory o różnym stężeniu}
\Clue{16}{}{gruziński sos}
\Clue{17}{}{procent, oprocentowanie, odsetki, które są duże, szybko, intensywnie rosną}
\Clue{19}{}{w muzyce jazzowej: temat muzyczny zapożyczony z muzyki rozrywkowej w przeciwieństwie do tematu oryginalnego}
\Clue{24}{}{osoba, która bardzo dużo je, często niezdrowo i szybko, grubas, ktoś, kto się spasł}
\Clue{26}{}{lekki, dwukołowy pojazd jednokonny na dwie lub cztery osoby używany w Indiach}
\Clue{28}{}{karłowaty pies pokojowy o obfitej sierści}
\Clue{35}{}{przestępstwo dotyczące osiągania korzyści majątkowych w wyniku kradzieży, przywłaszczenia, oszustwa lub wyłudzenia}
\Clue{36}{}{przedstawiciel plemion protogreckich, które wraz z Achajami i Eolami około 2 tysiąclecia p.n.e. zasiedliły środkową i południową Grecję}
\Clue{38}{}{obszar lasu znajdujący się pod opieką leśniczego}
\Clue{39}{}{długa, biała szata liturgiczna, noszona przez wszystkich duchownych i usługujących wszystkich stopni w czasie liturgii obrządku łacińskiego oraz przez duchownych wielu kościołów protestanckich}
\Clue{40}{}{połączenie co najmniej dwóch przedsiębiorstw}
\Clue{42}{}{miasto w Jugosławii (Serbia) nad Niszawą wyrób dywanów}
\Clue{43}{}{zodiakalny gwiazdozbiór, również znak Zodiaku}
\Clue{44}{}{JASTRZĘBIEC}
\Clue{45}{}{kompozytor duński (1865-1931); symfonie, utwory kameralne, opery}
\Clue{46}{}{ptak wodny}
\Clue{47}{}{roślina pochodząca z grupy warzyw wydzielonej na podstawie ich klasyfikacji biologicznej - przynależności do rodziny kapustowatych; do tej grupy należą: brokuł, kalafior, jarmuż, kalarepa oraz kilka rodzajów kapust}
\Clue{48}{}{we wczesnośredniowiecznej Polsce urzędnik królewski zajmujący się organizacją łowów monarchy i administrujący łowiskami}
\Clue{49}{}{mieszkanka Bogatyni}
\Clue{50}{}{Xanthophyceae - klasa protistów roślinopodobnych zaliczanych do glonów, blisko spokrewniona z brunatnicami; przedstawiciele tej klasy posiadają jedną lub dwie wici, przez co zaliczni są również do nienaturalnej grupy wiciowców}
\Clue{51}{}{pochodna pirydoksyny, w której grupa hydroksymetylowa z pozycji 4 pierścienia pirydynowego przekształcona jest w grupę aminometylową}
\Clue{52}{}{poddawanie kogoś lub czegoś badaniom diagnostycznym, czynności diagnostyczne mające na celu wykrycie schorzenia}\end{PuzzleClues}

\begin{PuzzleClues}{\textbf{Pionowe}\\}\Clue{1}{}{eurazjatycki gryzoń nadrzewny, żywi się owocami i nasionami}
\Clue{2}{}{łeb tłokowy korbowodu}
\Clue{3}{}{zwyczajowa nazwa ryb morskich z rodzaju Engraulis}
\Clue{4}{}{miasto w płn. Egipcie w delcie Nilu; węzeł komunikacyjny}
\Clue{5}{}{funkcja językowa polegająca na wskazywaniu elementów sytuacji, o której jest mowa}
\Clue{7}{}{umocnienie, obecnie najczęściej stosowane do regulacji brzgów rzek; powiązane ze sobą cienkie gałęzie wiklinowe lub innych drzew czy krzewów}
\Clue{14}{}{(potencjałowa) elektroda pomocnicza w lampie gazowanej}
\Clue{15}{}{prażone ziarno kukurydzy}
\Clue{17}{}{dawne określenie Hiszpanii}
\Clue{18}{}{długość włókna}
\Clue{19}{}{karelskie i fińskie ludowe pieśni epickie}
\Clue{20}{}{sklep z rzeczami, które nie są produktami pierwszej potrzeby}
\Clue{21}{}{oficjalne, uroczyste przyjęcie}
\Clue{22}{}{zjawisko znoszenia z kursu jednostki pływającej wskutek oddziaływania na nią prądu}
\Clue{23}{}{instr. składający się ze szklanych kloszów wydających dźwięk pod dotknięciem palców, popularny w XVII/ XIX w}
\Clue{24}{}{ustalone minimum finansowe, które zapewnia państwo swoim obywatelom}
\Clue{25}{}{wyniosłość na kości udowej, stanowiąca przyczep mięśnie biodrowo-lędźwiowego}
\Clue{27}{}{próchniczkowate, Aulacomniaceae - rodzina mchów należąca do rzędu prątnikowców}
\Clue{28}{}{środek przeciwlękowy, często o działaniu uspokajającym}
\Clue{29}{}{człowiek, który wierzy w teorie spiskowe, fascynuje się spiskami; nazwa pochodzi od folii aluminiowej, z której zrobione są czapeczki, w których chodzą niektórzy pasjonaci teorii spiskowych (folia aluminiowa ma odbijać promieniowanie elektromagnetyczne i zapobiegać próbom zawładnięcia umysłem człowieka, podejmowanym przez Rząd Światowy)}
\Clue{30}{}{Manouria emys - gatunek gada z rodziny żółwi lądowych, obejmujący dwa podgatunki, występujący w południowej Azji od Assamu i Birmy przez Półwysep Malajski aż po wyspy Sumatra i Borneo}
\Clue{31}{}{skłonność do ulegania uczuciom}
\Clue{32}{}{przejaw, postać, rodzaj czegoś}
\Clue{33}{}{ludowy instrument muzyczny, piszaczałka podobna do oboju, odmiana szałamai}
\Clue{34}{}{pasmo włosów, które się skręca}
\Clue{35}{}{kąt zawarty między dziobową linią symetrii statku a linią łączącą dany obiekt z obserwatorem}
\Clue{37}{}{nazwa waluty w Rumunii i Mołdawii}
\Clue{40}{}{urzędowe formalności związane z wyjazdem osób lub transportem towarów}
\Clue{41}{}{podłużny element konstrukcji}
\Clue{47}{}{w muzyce: nazwa solmizacyjna dźwięku g}\end{PuzzleClues}\newpage\section*{Krzyżówka 95}

\noindent\begin{Puzzle}{21}{28}|*	|*	|*	|*	|*	|*	|*	|*	|*	|*	|*	|*	|[1][S]\drarr	|s	|e	|f	|*	|[2][S]\darr	|[3][S]\darr	|*	|[4][S]\darr	|*	|.
|*	|[5][S]\rarr	|d	|o	|b	|r	|y	|[][,]{ }	|w	|u	|j	|a	|s	|z	|e	|k	|*	|b	|s	|[6][S]\darr	|l	|*	|.
|*	|*	|[7][S]\darr	|*	|[8][S]\drarr	|b	|a	|r	|k	|a	|s	|*	|y	|[9][S]\rarr	|r	|ó	|ż	|y	|c	|k	|i	|*	|.
|[10][S]\rarr	|p	|r	|ó	|b	|a	|*	|*	|[11][S]\drarr	|d	|u	|b	|l	|e	|t	|*	|*	|l	|*	|r	|c	|*	|.
|*	|*	|u	|*	|i	|*	|*	|[12][S]\rarr	|d	|y	|k	|t	|a	|n	|d	|o	|*	|i	|*	|w	|z	|*	|.
|*	|*	|s	|*	|a	|*	|*	|*	|z	|*	|*	|*	|b	|*	|*	|[13][S]\darr	|*	|c	|*	|a	|i	|*	|.
|*	|*	|i	|*	|ł	|*	|*	|*	|i	|[14][S]\darr	|[15][S]\drarr	|b	|i	|n	|a	|r	|k	|a	|*	|w	|*	|[16][S]\darr	|.
|*	|[17][S]\darr	|e	|[18][S]\darr	|y	|[19][S]\drarr	|m	|i	|e	|s	|z	|a	|c	|z	|*	|ó	|*	|[][,]{ }	|*	|n	|*	|d	|.
|*	|o	|c	|i	|[][,]{ }	|s	|[20][S]\darr	|[21][S]\darr	|n	|e	|ą	|[22][S]\darr	|z	|*	|*	|w	|*	|m	|*	|i	|*	|r	|.
|*	|p	|*	|n	|w	|y	|r	|o	|n	|t	|b	|h	|n	|*	|*	|*	|*	|i	|[23][S]\darr	|k	|[24][S]\darr	|a	|.
|*	|c	|*	|e	|ę	|g	|o	|p	|i	|k	|*	|o	|o	|*	|[25][S]\darr	|[26][S]\darr	|*	|o	|c	|[][,]{ }	|o	|c	|.
|*	|j	|*	|d	|g	|n	|m	|c	|k	|a	|*	|n	|ś	|*	|k	|z	|[27][S]\darr	|t	|o	|w	|b	|e	|.
|*	|a	|*	|i	|i	|a	|e	|j	|a	|*	|*	|i	|ć	|*	|a	|a	|p	|ł	|m	|r	|m	|n	|.
|*	|[][,]{ }	|[28][S]\drarr	|t	|e	|t	|r	|a	|r	|c	|h	|a	|*	|*	|m	|s	|l	|o	|p	|o	|i	|a	|.
|*	|w	|e	|a	|l	|u	|*	|[][,]{ }	|k	|[29][S]\rarr	|f	|r	|e	|g	|a	|t	|o	|w	|a	|t	|e	|*	|.
|*	|a	|f	|*	|*	|r	|[30][S]\rarr	|p	|a	|s	|*	|a	|[31][S]\darr	|[32][S]\darr	|i	|a	|t	|a	|c	|y	|r	|*	|.
|[33][S]\rarr	|l	|e	|g	|n	|a	|g	|o	|*	|*	|*	|*	|g	|w	|s	|ł	|e	|*	|t	|c	|z	|*	|.
|*	|u	|m	|*	|[34][S]\darr	|*	|*	|l	|[35][S]\darr	|[36][S]\darr	|[37][S]\darr	|*	|e	|a	|h	|o	|r	|*	|*	|z	|ł	|*	|.
|*	|t	|e	|*	|n	|*	|[38][S]\darr	|i	|m	|i	|t	|[39][S]\darr	|z	|t	|i	|ś	|[][,]{ }	|*	|*	|o	|o	|*	|.
|*	|o	|r	|[40][S]\drarr	|p	|o	|s	|t	|e	|r	|u	|n	|e	|k	|*	|ć	|b	|*	|*	|l	|ś	|*	|.
|*	|w	|y	|r	|r	|*	|ł	|y	|d	|l	|b	|e	|l	|o	|*	|*	|ę	|*	|*	|i	|ć	|*	|.
|*	|a	|d	|ó	|*	|*	|o	|c	|a	|a	|a	|r	|l	|w	|*	|*	|b	|*	|*	|s	|*	|*	|.
|*	|*	|y	|g	|*	|*	|d	|z	|n	|n	|*	|w	|e	|e	|*	|*	|n	|[41][S]\darr	|*	|t	|*	|*	|.
|*	|*	|*	|*	|*	|*	|y	|n	|*	|d	|*	|*	|*	|*	|[42][S]\rarr	|t	|o	|p	|*	|n	|*	|*	|.
|*	|*	|[43][S]\rarr	|u	|l	|i	|c	|a	|*	|z	|*	|[44][S]\rarr	|c	|a	|r	|o	|w	|a	|*	|y	|*	|*	|.
|*	|*	|[45][S]\rarr	|j	|a	|z	|z	|*	|[46][S]\rarr	|k	|o	|r	|d	|*	|*	|*	|y	|l	|*	|*	|*	|*	|.
|[47][S]\rarr	|k	|l	|e	|r	|k	|*	|[48][S]\rarr	|l	|i	|p	|p	|i	|*	|*	|*	|*	|i	|*	|*	|*	|*	|.
|*	|*	|[49][S]\rarr	|n	|e	|w	|t	|o	|n	|*	|*	|*	|[50][S]\rarr	|c	|u	|z	|c	|o	|*	|*	|*	|*	|.
|*	|*	|*	|*	|*	|*	|*	|*	|*	|*	|*	|*	|*	|*	|*	|*	|*	|*	|*	|*	|*	|*	|.\end{Puzzle}

\newpage

\begin{PuzzleClues}{\textbf{Poziome}\\}\Clue{1}{}{miejscowość w Austrii, nad jeziorem Mondsee}
\Clue{5}{}{osoba wyrozumiała, pomagająca innym, wspomagająca ich}
\Clue{8}{}{statek portowy z napędem mechanicznym do przewozu ludzi lub ładunków, np. poczty ze statków na redzie; obecnie wykorzystywany najczęściej do organizowania wycieczek turystycznych po porcie}
\Clue{9}{}{florecista, drużynowy wicemistrz olimpijski z Tokio}
\Clue{10}{}{znak na wyrobach jubilerskich, medalierskich i monetach, który informuje, z jakiego stopu wykonano przedmiot}
\Clue{11}{}{drugi egzemplarz jakiegoś przedmiotu, np. identyczny znaczek}
\Clue{12}{}{praca pisemna, fizycznie istniejący egzemplarz, zapis czyichś zmagań ze sprawdzianem, ćwiczeniem, polegającym na zapisaniu ze słuchu dyktowanego tekstu}
\Clue{15}{}{opcje binarna, jeden z rodzajów opcji: pochodnych instrumentów finansowych}
\Clue{19}{}{maszyna analityczno-licząca, segregująca perforowane karty według rodzaju zapisanych na nich danych}
\Clue{28}{}{tytuł nadawany w starożytności zarządcy określonych terenów Azji Mniejszej, stanowiących jedną czwartą terytorium Imperium Rzymskiego}
\Clue{29}{}{fregaty, Fregatidae - rodzina ptaków z rzędu głuptakowych (Suliformes)}
\Clue{30}{}{długi, wąski kawałek skóry, tkaniny lub innego materiału, część garderoby, podtrzymująca w pasie spodnie lub spódnicę lub noszona dla ozdoby}
\Clue{33}{}{miasto we Włoszech (Wenecja Euganejska) nad Adygą; ośrodek handlowy regionu rolniczego}
\Clue{40}{}{siedziba żandarmerii wojskowej}
\Clue{42}{}{szczyt sławy, bycia modnym, apogeum sukcesu}
\Clue{43}{}{ludzie, społeczeństwo, z którym należy się liczyć}
\Clue{44}{}{kobieta sprawująca władzę carską}
\Clue{45}{}{improwizowana muzyka instrumentalna powstała w XIX w w USA - DŻEZ}
\Clue{46}{}{ur. w 1930 r., dyrygent; dyrektor Filharmonii Narodowej w Warszawie}
\Clue{47}{}{urzędnik świecki lub kościelny}
\Clue{48}{}{malarz, rysownik (1906-62) metaforyczne kompozycje na tematy polityczne 'Kamienie krzyczą'}
\Clue{49}{}{ISAAC; angielski fizyk, matematyk i teolog (1642-1727); odkrył prawo powszechnego ciążenia, rozszczepił pryzmatem światło, współtwórca rachunku różniczkowego i całkowego, zajmował się historią Kościoła}
\Clue{50}{}{miasto w Peru w Andach na wysokości 3326 m, ośrodek adm. departamentu Cuzco}\end{PuzzleClues}

\begin{PuzzleClues}{\textbf{Pionowe}\\}\Clue{1}{}{podzielność na sylaby}
\Clue{2}{}{Artemisia scoparia - gatunek roślin z rodziny astrowatych}
\Clue{3}{}{w chemii: symbol skandu}
\Clue{4}{}{niskie drzewo pochodzące z Chin uprawiane ze względu na smaczne, czerwone owoce - śliwka chińska}
\Clue{6}{}{Achillea tanacetifolia - gatunek rośliny z rodziny astrowatych}
\Clue{7}{}{wieś w Polsce położona w województwie łódzkim, w powiecie bełchatowskim, w gminie Rusiec}
\Clue{8}{}{elektrownia wodna, która zamienia energię potencjalną wody na energię elektryczną}
\Clue{11}{}{zawód dziennikarza}
\Clue{13}{}{podłużne wgłębienie w ziemi, naturalne lub wykopane przez człowieka}
\Clue{14}{}{liczba 100, numer 100}
\Clue{15}{}{złożony, twardy twór anatomiczny występujący u większości kręgowców (z wyjątkiem współczesnych ptaków, żółwi, ropuch i niektórych ssaków), stanowiący element układu trawienia i służący do rozdrabniania pożywienia}
\Clue{16}{}{SMOKOWIEC drzewo lub krzew strefy podzwrotnikowej, wydziela czerwoną żywicę smocze drzewo}
\Clue{17}{}{umowa zawarta pomiędzy nabywcą (kupującym) a wystawcą (sprzedającym), według której nabywca ma prawo a nie obowiązek do kupna albo sprzedaży określonej ilości waluty CUR1 za walutę CUR2 według określonego kursu bazowego w uzgodnionym dniu lub jakimś odcinku czasu}
\Clue{18}{}{teksty dzieł literackich niewydanych za życia autora, które pozostają w rękopisach}
\Clue{19}{}{własnoręczny podpis artysty na jego dziele}
\Clue{20}{}{geograf (1871-1954); twórca nowoczesnej kartografii polskiej}
\Clue{21}{}{określenie kilku partii, niezwiązanych ze sobą organizacyjnie, które łączą wspólne główne poglądy polityczne (np. prawica, lewica, centrum)}
\Clue{22}{}{stolica i główny port morski Wysp Salomona; międzynarodowy port lotniczy}
\Clue{23}{}{płyta CD}
\Clue{24}{}{uczucie obrzydzenia}
\Clue{25}{}{miasto w Japonii (płn. Honsiu) port nad Oceanem Spokojnym; w pobliżu eksploatacja rud żelaza}
\Clue{26}{}{cecha wynikająca z fizycznego bezruchu, np. zastałość nóg, kości}
\Clue{27}{}{ploter rolkowy - rodzaj plotera; pióro porusza się w nim jedynie powyżej osi bębna w linii prostej, a bęben zapewnia ruch papieru}
\Clue{28}{}{dane dotyczące przebiegu przyszłego zjawiska astronomicznego}
\Clue{31}{}{(1830-99), poeta belgijski tworzący w języku flamandzkim, działacz oświatowy, badacz folkloru}
\Clue{32}{}{Ulvophyceae - klasa wielokomórkowych zielenic, słonowodnych glonów o dużym zróżnicowaniu budowy}
\Clue{34}{}{skrótowiec odnaturalnego planowania rodziny}
\Clue{35}{}{miasto w Indonezji, ośrodek administracyjny prowincji Sumatra Płn}
\Clue{36}{}{język z grupy goidelskiej (q-celtyckiej) języków celtyckich uznany za drugi obok angielskiego język urzędowy niepodległej Irlandii}
\Clue{37}{}{obcisła sukienka zwykle bez ramiączek}
\Clue{38}{}{cecha kogoś lub czegoś, kto/co jest słodki, słodko wygląda}
\Clue{39}{}{energia, emocja, którą ktoś lub coś przejawia}
\Clue{40}{}{rożek, który wyrasta na ciele niektórych bezkręgowców}
\Clue{41}{}{fiat z modelu Palio}\end{PuzzleClues}\newpage\section*{Krzyżówka 96}

\noindent\begin{Puzzle}{21}{23}|*	|*	|*	|[1][S]\drarr	|t	|r	|a	|n	|s	|g	|r	|e	|s	|j	|a	|[][,]{ }	|m	|o	|r	|z	|a	|*	|.
|[2][S]\rarr	|b	|u	|h	|a	|j	|*	|[3][S]\drarr	|p	|o	|r	|t	|u	|g	|a	|l	|s	|k	|o	|ś	|ć	|*	|.
|*	|[4][S]\rarr	|h	|e	|t	|m	|a	|n	|[][,]{ }	|s	|a	|j	|d	|a	|c	|z	|n	|y	|*	|*	|*	|*	|.
|*	|[5][S]\rarr	|p	|ł	|o	|n	|n	|i	|k	|[][,]{ }	|j	|a	|ł	|o	|w	|c	|o	|w	|a	|t	|y	|*	|.
|*	|[6][S]\drarr	|i	|m	|p	|r	|o	|w	|i	|z	|a	|c	|j	|a	|*	|*	|*	|*	|*	|*	|*	|*	|.
|*	|s	|*	|*	|[7][S]\rarr	|n	|i	|e	|p	|r	|z	|e	|j	|e	|z	|d	|n	|o	|ś	|ć	|*	|*	|.
|[8][S]\drarr	|k	|a	|p	|i	|t	|a	|l	|i	|z	|m	|[][,]{ }	|p	|a	|ń	|s	|t	|w	|o	|w	|y	|*	|.
|f	|o	|[9][S]\rarr	|s	|z	|u	|w	|a	|k	|s	|*	|*	|*	|[10][S]\drarr	|g	|ę	|ś	|*	|[11][S]\darr	|[12][S]\darr	|[13][S]\darr	|*	|.
|a	|k	|*	|*	|*	|*	|*	|c	|*	|[14][S]\darr	|[15][S]\darr	|*	|*	|t	|[16][S]\drarr	|n	|a	|*	|d	|c	|e	|*	|.
|r	|*	|[17][S]\rarr	|k	|o	|l	|e	|j	|*	|w	|ż	|*	|*	|a	|l	|[18][S]\darr	|*	|[19][S]\darr	|e	|u	|m	|*	|.
|t	|[20][S]\rarr	|b	|u	|r	|g	|r	|a	|b	|i	|a	|[][,]{ }	|k	|r	|a	|k	|o	|w	|s	|k	|i	|*	|.
|u	|[21][S]\rarr	|v	|e	|l	|d	|e	|*	|*	|ć	|b	|*	|*	|a	|m	|i	|*	|e	|e	|i	|s	|[22][S]\darr	|.
|s	|[23][S]\drarr	|j	|i	|r	|a	|s	|e	|k	|*	|n	|*	|*	|n	|p	|b	|*	|r	|n	|e	|j	|k	|.
|z	|o	|*	|[24][S]\drarr	|z	|d	|e	|c	|h	|l	|i	|n	|a	|*	|a	|i	|[25][S]\darr	|y	|i	|r	|a	|u	|.
|e	|l	|*	|m	|*	|[26][S]\darr	|[27][S]\darr	|*	|*	|*	|c	|*	|[28][S]\darr	|*	|[][,]{ }	|c	|t	|z	|o	|[][,]{ }	|*	|d	|.
|k	|b	|*	|a	|*	|m	|s	|[29][S]\rarr	|p	|r	|z	|y	|m	|u	|s	|*	|e	|m	|w	|m	|*	|ł	|.
|*	|r	|*	|c	|*	|e	|z	|*	|[30][S]\rarr	|z	|a	|b	|o	|r	|o	|w	|o	|*	|a	|l	|[31][S]\darr	|y	|.
|*	|o	|*	|h	|*	|i	|n	|*	|*	|[32][S]\darr	|n	|*	|r	|[33][S]\darr	|d	|[34][S]\drarr	|k	|o	|n	|e	|w	|*	|.
|*	|t	|*	|a	|*	|k	|a	|*	|*	|w	|i	|*	|s	|r	|o	|r	|r	|*	|i	|k	|ó	|*	|.
|*	|*	|*	|u	|*	|l	|p	|*	|*	|y	|n	|*	|*	|y	|w	|a	|a	|*	|e	|o	|z	|*	|.
|[35][S]\rarr	|s	|y	|l	|w	|e	|s	|t	|e	|r	|*	|*	|*	|z	|a	|d	|c	|*	|*	|w	|e	|*	|.
|[36][S]\rarr	|w	|a	|t	|a	|*	|*	|[37][S]\rarr	|p	|a	|d	|u	|n	|a	|*	|c	|j	|*	|*	|y	|k	|*	|.
|*	|*	|*	|*	|*	|*	|*	|[38][S]\rarr	|z	|j	|a	|w	|a	|*	|*	|a	|a	|*	|*	|*	|*	|*	|.
|*	|[39][S]\rarr	|w	|e	|r	|a	|n	|d	|a	|*	|*	|*	|*	|*	|*	|*	|*	|*	|*	|*	|*	|*	|.\end{Puzzle}

\newpage

\begin{PuzzleClues}{\textbf{Poziome}\\}\Clue{1}{}{stopniowe zalewanie powierzchni lądu przez morze}
\Clue{2}{}{stadnik, samiec rozpłodowy bydła domowego}
\Clue{3}{}{zespół cech czegoś lub kogoś takiego jak w Portugalii, także: stereotypowe cechy uznawane za właściwe Portugalczykom}
\Clue{4}{}{wojskowy piastujący godność hetmana sahajdacznego}
\Clue{5}{}{Polytrichum juniperinum - gatunek mchu z rodziny płonnikowatych}
\Clue{6}{}{spontaniczne komponowanie utworu muzycznego; utwór improwizowany}
\Clue{7}{}{cecha czegoś (np. ulicy, terenu), co jest nieprzejezdne - przez co nie można przejechać samochodem, rowerem itp}
\Clue{8}{}{koncepcja ekonomiczna, oznaczająca  gospodarkę, w której kapitalistyczne stosunki produkcji, oparte na własności prywatnej, znajdują się pod ścisłą kontrolą instytucji państwowych}
\Clue{9}{}{szwarc - czarna pasta do butów}
\Clue{10}{}{idiotka, głupia, naiwna kobieta}
\Clue{16}{}{w chemii: symbol sodu}
\Clue{17}{}{przebieg życia, wydarzenia rozłożone w czasie}
\Clue{20}{}{urzędnik koronny Korony Królestwa Polskiego I Rzeczypospolitej, mianowany przez króla, odpowiadał tylko przed sądem sejmowym}
\Clue{21}{}{belgijski architekt i malarz (1863-1957), teatr w Kolonii}
\Clue{23}{}{(1851-1930), pisarz czeski, opowiadania, powieści, dramaty; „Jan Ziżka”, „Jan Hus”, „Bractwo”}
\Clue{24}{}{człowiek niechciany, niepożądany, oceniany negatywnie; wyzwisko dla człowieka}
\Clue{29}{}{niezbędność dla dalszego działania, konieczność}
\Clue{30}{}{dawne miasto, obecnie dzielnica Leszna (województwo wielkopolskie)}
\Clue{34}{}{metalowe naczynie do noszenia i przechowywania płynów}
\Clue{35}{}{wigilia Nowego Roku, 31 grudnia w najbardziej rozpowszechnionym kalendarzu gregoriańskim (zwłaszcza: noc z 31 grudnia na 1 stycznia), kiedy to świętuje się koniec starego roku i początek nowego, stanowiąca okres hucznych zabaw i bali, toastów, sztucznych ogni, petard}
\Clue{36}{}{zbite i nieregularnie poplątane cienkie włókna bawełniane lub wiskozowe; produkt, który służy głównie do celów higienicznych (np. opatrunkowych) i kosmetycznych}
\Clue{37}{}{lina biegnąca od szczytu stęgi do burty statku}
\Clue{38}{}{myśl, refleksja, zwykle negatywna, która nie daje spokoju, powraca}
\Clue{39}{}{dobudowane do budynku, zadaszone, otwarte lub oszklone pomieszczenie, pełniące funkcje wypoczynkowe}\end{PuzzleClues}

\begin{PuzzleClues}{\textbf{Pionowe}\\}\Clue{1}{}{bojowa ochrona głowy, chroniąca czaszkę przed urazami, sporządzona z odpornego materiału}
\Clue{3}{}{likwidacja różnic}
\Clue{6}{}{słabo umięśniona część kończyny dolnej ptaków, której szkielet tworzy silnie wydłużona kość skokowa}
\Clue{8}{}{mały fartuch}
\Clue{10}{}{zaostrzona belka wystająca z dziobu okrętu służąca do taranowania, zatapiania innych okrętów}
\Clue{11}{}{zdobienie czegoś deseniem}
\Clue{12}{}{disacharyd , na który składa się galaktoza i glikoza}
\Clue{13}{}{wprowadzanie, wydzielanie zanieczyszczeń do atmosfery}
\Clue{14}{}{rozłóg - boczny pęd rośliny, rosnący po ziemi lub pod ziemią, wytwarzający korzenie, będący organem rozmnażania wegetatywnego}
\Clue{15}{}{mieszkaniec Żabnicy, wsi w województwie śląskim}
\Clue{16}{}{lampa wyładowcza, w której środowiskiem wyładowczym są pary sodu}
\Clue{18}{}{człowiek śledzący rozgrywki sportowe i dopingujący wybraną drużynę}
\Clue{19}{}{cecha opisu, prawda, wierność faktom}
\Clue{22}{}{włosy człowieka, najczęściej rozczochrane}
\Clue{23}{}{substancja otrzymywana z tłuszczu kaszalota zaliczana do wosków}
\Clue{24}{}{francuski kompozytor i poeta (1300-1377); reprezentant ars nova}
\Clue{25}{}{ustrój, w którym władzę sprawują kapłani}
\Clue{26}{}{wynalazca szkocki (1719-1811 ); ulepszył wiatrak, zbudował mechaniczną młocarnię}
\Clue{27}{}{wódka, kieliszek wódki}
\Clue{28}{}{KOŃ MORSKI; ssak z rzędu płetwonogich; żywi się mięczakami i skorupiakami chroniony}
\Clue{31}{}{w maszynach do pisania podzespołów się poziomo po prowadnicy, wyposażony w wałek i inne urządzenia}
\Clue{32}{}{ciepłe kraje - odlatują do nich ptaki na zimę}
\Clue{33}{}{ilość papieru drukarskiego zawierająca 500 arkuszy}
\Clue{34}{}{osoba, której działalność polega na udzielaniu porad; od doradcy różni się instytucjonalizacją zawodu i związanym z nim prestiżem}\end{PuzzleClues}\newpage\section*{Krzyżówka 97}

\noindent\begin{Puzzle}{23}{28}|*	|*	|*	|*	|*	|*	|*	|*	|*	|*	|*	|*	|*	|*	|*	|*	|*	|*	|*	|*	|*	|[1][S]\darr	|*	|*	|.
|*	|*	|*	|*	|*	|*	|*	|*	|*	|*	|[2][S]\darr	|*	|*	|*	|*	|*	|[3][S]\darr	|*	|*	|[4][S]\darr	|[5][S]\darr	|m	|*	|*	|.
|*	|[6][S]\darr	|[7][S]\darr	|[8][S]\darr	|[9][S]\darr	|[10][S]\rarr	|l	|i	|c	|z	|b	|a	|*	|[11][S]\darr	|*	|[12][S]\darr	|p	|[13][S]\rarr	|g	|a	|l	|o	|n	|*	|.
|*	|c	|o	|e	|n	|*	|[14][S]\rarr	|s	|p	|l	|i	|t	|*	|g	|*	|w	|i	|[15][S]\darr	|*	|n	|i	|s	|*	|[16][S]\darr	|.
|[17][S]\drarr	|a	|m	|p	|e	|r	|*	|*	|[18][S]\rarr	|b	|e	|r	|*	|r	|*	|o	|s	|k	|*	|t	|s	|k	|[19][S]\darr	|o	|.
|n	|*	|s	|i	|u	|*	|[20][S]\drarr	|w	|a	|j	|d	|a	|*	|z	|*	|d	|t	|o	|*	|y	|z	|a	|k	|d	|.
|a	|*	|k	|c	|c	|[21][S]\drarr	|p	|o	|d	|w	|a	|ł	|*	|y	|[22][S]\darr	|o	|o	|s	|[23][S]\darr	|l	|k	|l	|u	|s	|.
|w	|*	|*	|y	|h	|g	|o	|*	|*	|*	|k	|*	|*	|b	|c	|w	|l	|z	|k	|i	|a	|i	|s	|t	|.
|ó	|*	|[24][S]\rarr	|k	|a	|ł	|m	|u	|k	|*	|*	|[25][S]\darr	|*	|o	|z	|s	|e	|a	|o	|t	|*	|k	|a	|ę	|.
|z	|*	|[26][S]\darr	|l	|t	|u	|a	|*	|[27][S]\rarr	|l	|e	|p	|*	|w	|y	|k	|t	|t	|z	|e	|*	|*	|d	|p	|.
|[][,]{ }	|*	|d	|*	|e	|p	|k	|[28][S]\rarr	|c	|z	|e	|r	|w	|o	|n	|a	|[][,]{ }	|k	|a	|r	|t	|k	|a	|*	|.
|o	|*	|e	|*	|l	|t	|*	|*	|*	|*	|*	|z	|*	|*	|n	|z	|a	|a	|k	|a	|*	|[29][S]\darr	|s	|*	|.
|r	|*	|t	|[30][S]\darr	|*	|a	|[31][S]\rarr	|m	|o	|s	|t	|e	|k	|*	|i	|*	|u	|*	|*	|t	|[32][S]\darr	|p	|i	|*	|.
|g	|*	|e	|r	|[33][S]\rarr	|k	|o	|p	|a	|*	|*	|z	|*	|*	|k	|[34][S]\drarr	|t	|e	|n	|u	|t	|o	|*	|*	|.
|a	|*	|r	|o	|*	|[][,]{ }	|*	|[35][S]\rarr	|o	|s	|a	|d	|a	|*	|[][,]{ }	|n	|o	|*	|*	|r	|r	|z	|*	|*	|.
|n	|*	|m	|z	|[36][S]\drarr	|p	|o	|l	|i	|m	|o	|r	|f	|*	|e	|a	|m	|*	|*	|a	|a	|y	|*	|*	|.
|i	|*	|i	|w	|p	|e	|[37][S]\drarr	|a	|l	|d	|r	|o	|w	|a	|n	|d	|a	|*	|*	|*	|n	|t	|*	|*	|.
|c	|*	|n	|a	|o	|r	|m	|[38][S]\darr	|[39][S]\darr	|*	|[40][S]\darr	|w	|*	|*	|d	|k	|t	|*	|*	|*	|s	|y	|[41][S]\darr	|*	|.
|z	|*	|i	|ż	|a	|u	|y	|s	|p	|*	|i	|i	|*	|*	|o	|r	|y	|*	|*	|*	|p	|w	|m	|*	|.
|n	|*	|z	|n	|*	|w	|c	|p	|o	|*	|r	|e	|*	|*	|g	|w	|c	|*	|*	|[42][S]\darr	|o	|n	|e	|*	|.
|y	|[43][S]\drarr	|m	|o	|h	|i	|k	|a	|n	|k	|a	|*	|*	|*	|e	|i	|z	|*	|*	|l	|z	|o	|n	|*	|.
|*	|r	|*	|ś	|*	|a	|a	|l	|s	|*	|k	|[44][S]\darr	|*	|*	|n	|s	|n	|*	|*	|e	|y	|ś	|o	|*	|.
|*	|o	|*	|ć	|*	|ń	|*	|o	|*	|*	|i	|k	|*	|*	|i	|t	|y	|*	|*	|n	|c	|ć	|n	|*	|.
|*	|y	|*	|*	|[45][S]\rarr	|s	|o	|n	|k	|o	|j	|a	|*	|*	|c	|o	|*	|*	|*	|i	|j	|*	|i	|*	|.
|*	|*	|*	|*	|*	|k	|*	|y	|*	|*	|c	|p	|*	|*	|z	|ś	|*	|[46][S]\drarr	|g	|w	|a	|ł	|t	|*	|.
|*	|[47][S]\rarr	|p	|e	|n	|i	|s	|*	|*	|*	|z	|s	|*	|*	|n	|ć	|*	|ż	|*	|k	|*	|*	|y	|*	|.
|[48][S]\rarr	|d	|a	|n	|a	|*	|*	|*	|*	|*	|y	|l	|*	|*	|y	|*	|*	|a	|*	|a	|*	|*	|z	|*	|.
|*	|[49][S]\rarr	|a	|f	|e	|r	|z	|y	|s	|t	|k	|a	|*	|*	|*	|*	|*	|r	|*	|*	|*	|*	|m	|*	|.
|*	|[50][S]\rarr	|g	|r	|e	|e	|n	|o	|c	|k	|*	|*	|*	|*	|*	|*	|*	|*	|*	|*	|*	|*	|*	|*	|.\end{Puzzle}

\newpage

\begin{PuzzleClues}{\textbf{Poziome}\\}\Clue{10}{}{kategoria gramatyczna właściwa dla rzeczowników, czasowników, przymiotników i zaimków (poza zaimkiem liczebnym)}
\Clue{13}{}{pasamon będący tkaną lub plecioną taśmą wykonaną całkowicie lub z dodatkiem metalowych nitek}
\Clue{14}{}{w ekonomii - obniżenie wartości nominalnej akcji przy jednoczesnym utrzymaniu dotychczasowego kapitału akcyjnego spółki}
\Clue{17}{}{podstawowa jednostka prądu elektrycznego w układzie SI}
\Clue{18}{}{WŁOŚNICA, CZUMIZA, GOMI jednoroczna roślina zbożowa uprawiana na paszę na Dalekim Wschodzie}
\Clue{20}{}{Andrzej Wajda - polski reżyser filmowy i teatralny, w latach 1989-1991 senator I kadencji}
\Clue{21}{}{reklama na dole strony na szerokość strony}
\Clue{24}{}{prostak, zacofaniec}
\Clue{27}{}{przenośnie: coś kuszącego i zdradliwego}
\Clue{28}{}{kara w rozmaitych dyscyplinach sportowych}
\Clue{31}{}{część oprawki okularów, łącząca szkła}
\Clue{33}{}{liczba 60, 60 sztuk czegoś}
\Clue{34}{}{określenie wykonawcze; wytrzymując (wartości rytmiczne dźwięków)}
\Clue{35}{}{rowek w tylnej części drzewca strzały dostosowany do grubości cięciwy w łuku}
\Clue{36}{}{odmiana krystalograficzna jakiejś substancji chemicznej}
\Clue{37}{}{bezkorzeniowa, drobna roślina wieloletnia z rosiczkowatych, żywi się drobnymi organizmami wodnymi}
\Clue{43}{}{przedstawicielka plemienia Mohikanów, Indian Ameryki Północnej}
\Clue{45}{}{smaczny owoc (wielopestkowiec) flaszowca purpurowego}
\Clue{46}{}{stan chaosu, pełny zamieszania i pośpiechu}
\Clue{47}{}{narząd kopulacyjny wielu zwierząt, zwłaszcza - samców ssaków, gadów oraz ptaków}
\Clue{48}{}{w buddyzmie: jedna z cnót; dar, ofiara, jałmużna, która może mieć wymiar materialny, duchowy lub może polegać na zapewnieniu człowiekowi, duchowi lub zwierzęciu poczucia bezpieczeństwa, ukojenia}
\Clue{49}{}{kryminalistka zamieszana w jakąś aferę}
\Clue{50}{}{miasto w Szkocji, port nad zatoką Firth of Clyde}\end{PuzzleClues}

\begin{PuzzleClues}{\textbf{Pionowe}\\}\Clue{1}{}{mały śledź marynowany lub solony z dodatkiem korzeni}
\Clue{2}{}{człowiek nieszczęśliwy, skrzywdzony, zasługujący na współczucie}
\Clue{3}{}{krótka broń automatyczna o cechach zarówno pistoletu maszynowego (możliwość prowadzenia ognia seriami) jak i pistoletu samopowtarzalnego (wygląd i wymiary)}
\Clue{4}{}{twórczość piśmiennicza, w której zostają odrzucone obowiązujące dotychczas zasady, schematy i prawa w literaturze}
\Clue{5}{}{łow. samica lisa}
\Clue{6}{}{w chemii: symbol wapnia}
\Clue{7}{}{miasto obwodowe w azjatyckiej części Federacji Rosyjskiej, duży port nad Irtyszem}
\Clue{8}{}{orbita kolista, po której - w teorii geocentrycznej - poruszać się miały ciała niebieskie, środek tego koła przemieszczał się ruchem jednostajnym wokół Ziemi po wielkim kole orbity zwanej deferentem}
\Clue{9}{}{jezioro w Szwajcarii, u podnóża Jury, powierzchnia 218 km 2, głębokość do 153 m, - połączone z jeziorem Biel}
\Clue{11}{}{osada w Polsce położona w województwie warmińsko-mazurskim, w powiecie kętrzyńskim, w gminie Reszel}
\Clue{12}{}{urządzenie służące do pomiarów stanów wody}
\Clue{15}{}{gryzoń nadrzewny z liściastych lasów Europy i Azji w Polsce chroniona}
\Clue{16}{}{w spawalnictwie: najmniejsza odległość miedzy krawędziami części łączonych spoiną}
\Clue{17}{}{nawóz wyprodukowany z substancji organicznej lub z mieszanin substancji organicznych}
\Clue{19}{}{zatoka Morza Egejskiego w wschodnich wybrzeży Turcji}
\Clue{20}{}{bułgarski wyznawca islamu zamieszkujący głównie na terenie Rodop}
\Clue{21}{}{Sula variegata - gatunek ptaka z rodziny głuptaków (Sulidae); występuje wzdłuż wybrzeży Peru i Chile, gniazduje na wyspach}
\Clue{22}{}{czynnik geologiczny, którego siła tkwi we wnętrzu Ziemi; objawia się na powierzchni Ziemi np. jako trzęsienia ziemi}
\Clue{23}{}{śmiałek, wariat, człowiek nierozsądny}
\Clue{25}{}{toast za czyjeś zdrowie}
\Clue{26}{}{koncepcja filozoficzna, według której wszystkie zdarzenia w ramach przyjętego paradygmatu są połączone związkiem przyczynowo-skutkowym, a zatem każde zdarzenie jest zdeterminowane przez swoje przyczyny}
\Clue{29}{}{cecha człowieka dobrze (pozytywnie) nastawionego do innych i do świata}
\Clue{30}{}{cecha zachowania, działania: to, że coś jest przejawem refleksyjności, że robione jest po namyśle i bez pośpiechu}
\Clue{32}{}{w muzyce - przeniesienie fragmentu lub całego utworu z jednej tonacji do innej}
\Clue{34}{}{pierwotna choroba mieloproliferacyjna, przebiegająca ze zwiększeniem liczby erytrocytów, granulocytów i płytek krwi, z przewagą erytropoezy; nazwa ogólna, niespecjalistyczna, gdyż właściwie nadkrwistość to tylko jeden z objawów towarzyszących tej chorobie}
\Clue{36}{}{brazylijskie miasto w pobliżu Sao Paulo}
\Clue{37}{}{mała, okrągła czapeczka bez daszka ściśle przylegająca do głowy}
\Clue{38}{}{w grze drużynowej typu piłka nożna, hokej itp.: niezgodne z regułami zagranie w ataku, uniemożliwiające przeciwnej drużynie obronę pola gry}
\Clue{39}{}{śpiewaczka francuska (1904-1976); sopran koloraturowy}
\Clue{40}{}{mieszkaniec Iraku, człowiek pochodzenia irackiego}
\Clue{41}{}{chrześcijańskie wyznanie protestanckie o charakterze ewangelikalnym założone przez Menno Simmonsa; jeden z nurtów anabaptyzmu}
\Clue{42}{}{śruba umożliwiająca wykonywanie powolnych przesunięć lub obrotów części składowych przyrządów geodezyjnych}
\Clue{43}{}{malarz indyjski (1887-1972) malarstwo inspirowane sztuk ludowa}
\Clue{44}{}{mosiężna lub miedziana spłonka, która ma masę palną wybuchającą po uderzeniu kurka}
\Clue{46}{}{bardzo wysoka temperatura powietrza}\end{PuzzleClues}\newpage\section*{Krzyżówka 98}

\noindent\begin{Puzzle}{24}{25}|*	|*	|*	|*	|*	|*	|*	|[1][S]\drarr	|m	|a	|r	|k	|i	|e	|t	|a	|n	|k	|a	|*	|*	|*	|*	|*	|*	|.
|*	|[2][S]\rarr	|p	|a	|r	|o	|w	|a	|r	|*	|[3][S]\drarr	|s	|k	|o	|c	|z	|e	|k	|*	|*	|*	|*	|[4][S]\darr	|*	|[5][S]\darr	|.
|*	|[6][S]\drarr	|b	|u	|r	|*	|[7][S]\darr	|k	|*	|[8][S]\rarr	|c	|s	|c	|*	|[9][S]\darr	|*	|*	|*	|*	|[10][S]\darr	|[11][S]\darr	|*	|t	|[12][S]\darr	|r	|.
|*	|p	|*	|*	|[13][S]\rarr	|k	|w	|i	|e	|t	|y	|z	|m	|*	|s	|*	|[14][S]\darr	|*	|*	|k	|b	|*	|r	|p	|u	|.
|*	|r	|*	|*	|*	|[15][S]\darr	|i	|t	|*	|[16][S]\rarr	|g	|r	|u	|n	|t	|*	|p	|*	|[17][S]\darr	|o	|r	|[18][S]\darr	|ó	|r	|b	|.
|*	|z	|[19][S]\rarr	|s	|u	|s	|z	|a	|r	|k	|a	|*	|[20][S]\darr	|[21][S]\darr	|u	|[22][S]\drarr	|a	|m	|b	|r	|o	|z	|j	|a	|*	|.
|[23][S]\drarr	|y	|e	|r	|*	|ł	|j	|*	|*	|*	|r	|*	|k	|k	|p	|b	|w	|*	|a	|a	|ń	|a	|k	|k	|*	|.
|h	|b	|[24][S]\darr	|[25][S]\darr	|[26][S]\darr	|o	|e	|*	|*	|*	|o	|*	|o	|u	|o	|h	|i	|[27][S]\darr	|b	|l	|[][,]{ }	|m	|ą	|o	|*	|.
|u	|y	|w	|p	|b	|w	|r	|*	|*	|*	|*	|*	|m	|r	|r	|*	|l	|ś	|o	|k	|p	|u	|t	|l	|*	|.
|n	|w	|s	|r	|a	|a	|*	|*	|[28][S]\rarr	|r	|z	|e	|p	|a	|*	|*	|o	|w	|l	|a	|a	|s	|*	|c	|*	|.
|*	|a	|t	|o	|r	|c	|*	|[29][S]\drarr	|r	|z	|ę	|s	|a	|*	|*	|*	|n	|i	|*	|*	|n	|t	|[30][S]\darr	|z	|[31][S]\darr	|.
|[32][S]\drarr	|j	|ę	|z	|y	|k	|[][,]{ }	|s	|k	|r	|y	|p	|t	|o	|w	|y	|*	|ą	|*	|*	|c	|r	|w	|a	|h	|.
|ś	|ą	|p	|a	|k	|i	|*	|b	|[33][S]\rarr	|s	|t	|r	|y	|j	|*	|[34][S]\darr	|*	|t	|*	|*	|e	|o	|o	|t	|a	|.
|c	|c	|*	|u	|a	|*	|*	|*	|*	|*	|*	|[35][S]\rarr	|b	|r	|z	|m	|i	|k	|*	|[36][S]\darr	|r	|w	|s	|k	|r	|.
|i	|a	|*	|r	|d	|*	|*	|*	|*	|*	|*	|*	|i	|[37][S]\drarr	|s	|o	|w	|a	|*	|t	|n	|a	|k	|a	|m	|.
|ó	|*	|*	|o	|a	|*	|*	|*	|*	|*	|*	|*	|l	|t	|*	|t	|*	|r	|*	|o	|a	|n	|o	|*	|o	|.
|ł	|*	|*	|p	|*	|*	|[38][S]\rarr	|l	|a	|c	|h	|i	|n	|e	|*	|y	|*	|s	|*	|n	|*	|i	|w	|*	|n	|.
|k	|*	|*	|o	|[39][S]\rarr	|s	|y	|m	|u	|l	|a	|t	|o	|r	|[][,]{ }	|l	|o	|t	|u	|*	|*	|e	|a	|*	|i	|.
|o	|*	|*	|d	|[40][S]\drarr	|r	|e	|g	|e	|n	|t	|*	|ś	|*	|*	|e	|[41][S]\rarr	|w	|r	|a	|k	|*	|t	|*	|j	|.
|w	|*	|*	|*	|w	|*	|*	|*	|*	|*	|*	|*	|ć	|*	|[42][S]\rarr	|k	|i	|o	|t	|o	|*	|*	|o	|*	|n	|.
|a	|*	|[43][S]\rarr	|c	|e	|n	|t	|y	|m	|e	|t	|r	|*	|*	|*	|[][,]{ }	|*	|*	|*	|*	|*	|*	|ś	|*	|o	|.
|n	|*	|*	|*	|n	|*	|[44][S]\rarr	|h	|o	|m	|o	|z	|y	|g	|o	|t	|y	|c	|z	|n	|o	|ś	|ć	|*	|ś	|.
|i	|[45][S]\rarr	|r	|o	|z	|r	|a	|c	|h	|u	|n	|e	|k	|*	|*	|ę	|*	|*	|*	|*	|*	|*	|*	|*	|ć	|.
|e	|[46][S]\rarr	|d	|ż	|e	|z	|*	|*	|*	|[47][S]\rarr	|k	|r	|ę	|g	|[][,]{ }	|p	|o	|ł	|o	|w	|i	|c	|z	|y	|*	|.
|*	|[48][S]\rarr	|d	|y	|l	|*	|[49][S]\rarr	|k	|ę	|d	|z	|i	|e	|r	|z	|y	|n	|[][S]-	|k	|o	|ź	|l	|e	|*	|*	|.
|[50][S]\rarr	|m	|ó	|r	|*	|*	|*	|*	|[51][S]\rarr	|z	|o	|d	|i	|a	|k	|*	|*	|*	|*	|*	|*	|*	|*	|*	|*	|.\end{Puzzle}

\newpage

\begin{PuzzleClues}{\textbf{Poziome}\\}\Clue{1}{}{dawniej: handlarka towarzysząca wojsku}
\Clue{2}{}{urządzenie do gotowania na parze}
\Clue{3}{}{figura szachowa kształtem przypominająca konia}
\Clue{6}{}{Afrykaner, potomek kolonistów holenderskich osiadły w Afryce Południowej}
\Clue{8}{}{funkcja będąca odwrotnością sinusa, tzn. csc(x)=1/sin(x); funkcja ta w punktach k? ma asymptoty pionowe, k to liczba całkowita}
\Clue{13}{}{doktryna religijna i etyczna}
\Clue{16}{}{zasadniczy element czegoś, podstawa}
\Clue{19}{}{stojak, na którym umieszcza się najczęściej mokre, wilgotne rzeczy po to, żeby wyschły}
\Clue{22}{}{Ambrosia - rodzaj roślin należący do rodziny astrowatych}
\Clue{23}{}{kod ISO 4217 riala jemeńskiego}
\Clue{28}{}{warzywo, jadalna bulwa rośliny o tej samej nazwie}
\Clue{29}{}{Lemna - rodzaj roślin wodnych z rodziny obrazkowatych}
\Clue{32}{}{język programowania służący do kontrolowania danej aplikacji}
\Clue{33}{}{miasto na Ukrainie nad Stryjem}
\Clue{35}{}{trzmielec, Psithyrus - podrodzaj dużych, krępych pszczół, zaliczanych do rodzaju trzmieli (Bombus), pasożytujących na społecznościach innych gatunków swojego rodzaju}
\Clue{37}{}{inna nazwa czernidłaka kołpakowatego}
\Clue{38}{}{miasto w Kanadzie (Quebec) nad rzeką Świętego Wawrzyńca, ośrodek przemysłowy}
\Clue{39}{}{program komputerowy symulujący zachowanie statku powietrznego w locie oraz ewentualnie w innych fazach działania}
\Clue{40}{}{krzyżówka międzygatunkowa winorośli; odmiana zarejestrowana w Niemczech 15 października 1996}
\Clue{41}{}{statek, który został uszkodzony w taki sposób, że nie może już być używany lub zatopiony}
\Clue{42}{}{miasto w Japonii na wyspie Honsiu; rezydencja cesarska}
\Clue{43}{}{wąska taśma, która ma centymetrową podziałkę i półtora metra długości}
\Clue{44}{}{to, że coś lub ktoś jest homozygotą, jest organizmem mającym identyczne allele danego genu w chromosomach}
\Clue{45}{}{obliczenie wydatków, zysków i strat}
\Clue{46}{}{jazz}
\Clue{47}{}{kręg o patologicznym kształcie klina zwróconego w niewłaściwą stronę, co powoduje odpowiednie skrzywienie kręgosłupa (np. skoliozę)}
\Clue{48}{}{gruba deska, także element betonowy lub gipsowy jako element ścienny stropowy}
\Clue{49}{}{miasto w województwie opolskim oraz siedziba powiatu kędzierzyńsko-kozielskiego, położone na Nizinie Śląskiej, na Śląsku Opolskim}
\Clue{50}{}{epidemia (w historii najczęściej była to epidemia dżumy lub czarnej ospy) zwykle nieznanej współczesnym strasznej choroby}
\Clue{51}{}{Zwierzyniec Niebieski; pas na sferze niebieskiej, w obrębie którego przebiegają obserwowane drogi Słońca, Księżyca, planet}\end{PuzzleClues}

\begin{PuzzleClues}{\textbf{Pionowe}\\}\Clue{1}{}{miasto w Japonii w płn. części wyspy Honsiu, ważny port nad Morzem Japońskim}
\Clue{3}{}{ułożone rurkowato, spreparowane i sklejone zwinięte liście tytoniu używane do palenia}
\Clue{4}{}{przyrząd używany w wielu odmianach bilardu do ustawiania bil na stole bilardowym na początku meczu; jest to rama wykonana najczęściej z drewna lub plastiku, która w większości odmian bilarda ma kształt trójkąta}
\Clue{5}{}{kod ISO 4217 rubla rosyjskiego}
\Clue{6}{}{ta, która przybywa}
\Clue{7}{}{zasłona na oczy w hełmie}
\Clue{9}{}{objaw chorobowy, w którym pacjent (mimo zachowania świadomości) nie reaguje na bodźce zewnętrzne, nie porusza się, nie mówi, wzrok ma utkwiony w jednym punkcie}
\Clue{10}{}{KRASNOROST glon mórz tropikalnych, silnie zwapniałe plechy uczestniczą w powstawaniu złóż wapiennych}
\Clue{11}{}{ogół broni, w skład której wchodzą opancerzone wozy bojowe z uzbrojeniem}
\Clue{12}{}{Zaglossus bruijnii - gatunek jajorodnego ssaka z rodziny kolczatkowatych; zamieszkuje Nową Gwineę}
\Clue{14}{}{wolno stojąca budowla wchodząca w skład zespołu architektonicznego (szpitala, wystawy)}
\Clue{15}{}{przedmiot szkolny lub uczony w ramach kursu, na którym opanowuje się podstawy języka słowackiego}
\Clue{17}{}{gol, który był łatwy do obronienia}
\Clue{18}{}{wpisanie na listę załogi statku}
\Clue{20}{}{techniczne dopasowanie, możliwość współpracowania urządzeń lub programów lub ich części}
\Clue{21}{}{mięso kury}
\Clue{22}{}{w chemii: symbol pierwiastka bohr}
\Clue{23}{}{członek ludu koczowniczego (prawdopodobnie przodkowie Węgrów), który ok. roku 370 najechał Europę i wywołując Wielką wędrówkę ludów, przyczynił się do upadku Cesarstwa Rzymskiego}
\Clue{24}{}{jęz. pierwsza, wstępna faza głoski}
\Clue{25}{}{przedstawiciel prozauropodów}
\Clue{26}{}{doraźna fortyfikacja przegradzająca w całości przejazd danym szlakiem komunikacyjnym, zbudowana w miejscu uniemożliwiającym jego bezpośrednie obejście, zbudowana z materiałów dostępnych na miejscu lub zgromadzonym specjalnie w tym celu}
\Clue{27}{}{rodzaj rzeźbiarstwa ludowego, którego wytworami są figury świętych}
\Clue{29}{}{jednostka luminancji układu CGS; 1 sb = 10E+4 nt}
\Clue{30}{}{cecha czegoś, co przypomina figurę woskową, jest jak z wosku, pozbawione życia, nieruchome, sztywne}
\Clue{31}{}{to, że coś jest pozytywnie oceniane, jako całość jest uważane jest za przyjemne, zwłaszcza: takie, w którym elementy pasują do siebie}
\Clue{32}{}{zabieg stosowany w ogrodnictwie, polegający na przykrywaniu gleby w celu zmniejszenia parowania wody, niedopuszczenia do rozwoju chwastów, poprawy sprawności roli oraz zapobieżenia erozji wodnej i wietrznej; materiałami stosowanymi do ściółkowania mogą być np. słoma, trociny, kora, kompost, liście, drobne kamyki, agrowłóknina lub czarna folia}
\Clue{34}{}{Schizanthus retusus - gatunek rośliny z rodziny psiankowatych}
\Clue{36}{}{w językoznawstwie: sposób akcentowania danej samogłoski (gł. rozpatrywany w językach tonalnych - tam jest czynnikiem znaczącym, tworzy opozycje), który polega na zmianie wysokości brzmienia sylaby, intonacja}
\Clue{37}{}{smar składający się ze smoły, kalafonii i tłuszczu, używany przede wszystkim jako smar ochronny}
\Clue{40}{}{Hanni-, alpejka z Lichtensteinu, mistrzyni i wicemistrzyni olimpijska z Lake Placid w slalomie, slalomie gigancie i biegu zjazdowym}\end{PuzzleClues}\newpage\section*{Krzyżówka 99}

\noindent\begin{Puzzle}{25}{21}|*	|*	|*	|[1][S]\drarr	|z	|y	|s	|*	|[2][S]\darr	|[3][S]\darr	|*	|*	|*	|*	|[4][S]\drarr	|m	|a	|k	|o	|w	|i	|a	|n	|k	|a	|*	|.
|*	|*	|*	|o	|[5][S]\darr	|*	|[6][S]\darr	|*	|a	|f	|*	|[7][S]\rarr	|s	|m	|u	|s	|z	|k	|a	|*	|[8][S]\darr	|*	|*	|[9][S]\darr	|[10][S]\darr	|*	|.
|*	|*	|*	|ś	|k	|*	|r	|*	|l	|a	|*	|[11][S]\rarr	|m	|a	|g	|l	|o	|w	|n	|i	|c	|a	|*	|s	|m	|*	|.
|[12][S]\rarr	|p	|u	|l	|i	|*	|o	|*	|u	|t	|*	|*	|[13][S]\rarr	|w	|i	|g	|o	|ń	|*	|[14][S]\darr	|h	|*	|[15][S]\darr	|z	|o	|*	|.
|*	|*	|*	|i	|e	|[16][S]\darr	|d	|*	|m	|a	|*	|[17][S]\rarr	|l	|i	|n	|u	|x	|*	|*	|g	|o	|*	|k	|a	|t	|*	|.
|*	|*	|*	|c	|s	|i	|z	|*	|i	|l	|[18][S]\rarr	|b	|i	|l	|e	|t	|[][,]{ }	|n	|o	|r	|m	|a	|l	|n	|y	|*	|.
|*	|*	|[19][S]\drarr	|z	|a	|g	|a	|d	|n	|i	|e	|n	|i	|e	|*	|*	|*	|*	|*	|d	|i	|*	|u	|t	|l	|*	|.
|*	|*	|d	|k	|*	|n	|j	|*	|i	|s	|*	|*	|*	|*	|*	|*	|*	|*	|*	|*	|c	|[20][S]\darr	|c	|u	|e	|[21][S]\darr	|.
|*	|[22][S]\drarr	|ż	|o	|n	|a	|[][,]{ }	|p	|u	|t	|y	|f	|a	|r	|a	|*	|*	|*	|*	|*	|z	|o	|z	|n	|k	|p	|.
|*	|p	|d	|w	|*	|m	|m	|*	|m	|a	|*	|*	|*	|*	|*	|[23][S]\rarr	|ś	|w	|i	|d	|e	|r	|*	|g	|*	|s	|.
|*	|o	|ż	|a	|*	|*	|ę	|*	|*	|*	|*	|*	|*	|*	|*	|*	|*	|*	|*	|[24][S]\drarr	|k	|o	|t	|*	|*	|a	|.
|*	|l	|y	|t	|[25][S]\drarr	|w	|s	|z	|y	|s	|t	|k	|o	|w	|i	|d	|z	|ą	|c	|a	|*	|n	|*	|*	|*	|l	|.
|*	|i	|s	|e	|k	|[26][S]\drarr	|k	|w	|a	|s	|z	|a	|r	|n	|i	|a	|*	|*	|*	|k	|*	|g	|*	|*	|*	|m	|.
|*	|u	|t	|*	|r	|p	|i	|*	|*	|*	|*	|*	|*	|[27][S]\darr	|*	|[28][S]\drarr	|o	|p	|a	|s	|ł	|o	|ś	|ć	|*	|i	|.
|*	|r	|o	|*	|o	|ł	|*	|*	|*	|*	|*	|[29][S]\rarr	|d	|r	|o	|b	|n	|i	|c	|a	|*	|*	|*	|*	|[30][S]\darr	|s	|.
|*	|e	|ś	|[31][S]\rarr	|k	|a	|f	|k	|a	|*	|*	|*	|*	|u	|*	|r	|[32][S]\rarr	|h	|u	|m	|o	|r	|*	|*	|n	|t	|.
|*	|t	|ć	|*	|o	|t	|[33][S]\rarr	|z	|i	|ę	|c	|i	|a	|s	|z	|e	|k	|*	|*	|i	|*	|*	|*	|*	|a	|a	|.
|*	|a	|*	|*	|d	|*	|*	|*	|*	|*	|*	|*	|*	|k	|[34][S]\rarr	|s	|c	|e	|p	|t	|y	|c	|y	|z	|m	|*	|.
|*	|n	|*	|*	|y	|*	|[35][S]\rarr	|k	|a	|p	|e	|l	|m	|i	|s	|t	|r	|z	|*	|n	|*	|*	|*	|*	|i	|*	|.
|*	|*	|*	|[36][S]\rarr	|l	|a	|b	|r	|a	|d	|o	|r	|*	|*	|*	|*	|[37][S]\rarr	|t	|w	|i	|e	|r	|d	|z	|a	|*	|.
|*	|[38][S]\rarr	|w	|i	|e	|l	|o	|z	|a	|d	|a	|n	|i	|o	|w	|o	|ś	|ć	|*	|k	|*	|*	|*	|*	|r	|*	|.
|*	|*	|*	|*	|*	|*	|*	|*	|[39][S]\rarr	|b	|u	|l	|i	|o	|n	|ó	|w	|k	|a	|*	|*	|*	|*	|*	|*	|*	|.\end{Puzzle}

\newpage

\begin{PuzzleClues}{\textbf{Poziome}\\}\Clue{1}{}{ptak drapieżny z grupy orłów, rdzawo-brązowy, zamieszkuje wysokie góry Eurazji, Ameryki Płn.; w Polsce bardzo rzadki, chroniony}
\Clue{4}{}{mieszkanka Makowa Podhalańskiego}
\Clue{7}{}{skóra kilkudniowego (1-5) jagnięcia, zwykle rasy karakuł}
\Clue{11}{}{maszyna, służąca do maglowania, czyli prasowania przy użyciu systemu walców}
\Clue{12}{}{węgierska rasa niewielkich psów z grupy owczarków}
\Clue{13}{}{tkanina wełniana z wełny wigonia lub z domieszką wigonii (rzadkość, gdyż przędza wigoniowa używana jest głównie na dzianiny)}
\Clue{17}{}{każdy system z rodziny uniksopodobnych systemów operacyjnych opartych na jądrze Linux}
\Clue{18}{}{pełnowartościowa opłata za usługę transportową}
\Clue{19}{}{problem, kwestia wymagająca rozstrzygnięcia, rozwiązania, przemyślenia}
\Clue{22}{}{kobieta kokietująca niedostępnych mężczyzn, próbująca za wszelką cenę ich skusić, sprawić, aby jej ulegli}
\Clue{23}{}{narzędzie wiercące służące do zwiercania skał stosowane w wiertnictwie; narzędzie to połączone jest z przewodem wiertniczym i zapuszczane do otworu wiertniczego}
\Clue{24}{}{kot domowy - udomowiony gatunek małego, mięsożernego ssaka z rzędu drapieżnych z rodziny kotowatych, przez ludzi ceniony jako zwierzę domowe oraz z powodu jego zdolności do chwytania szkodników}
\Clue{25}{}{ta, który dostrzega każdy fakt, każdy element rzeczywistości, również niematerialnej, zdaje sobie sprawę ze wszystkiego, co zaistniało bądź zaistnieje}
\Clue{26}{}{pomieszczenie, w którym znajdują się urządzenia do kwaszenia kapusty, ogórków, grzybów i in}
\Clue{28}{}{otyłość, obfita tusza}
\Clue{29}{}{pieniądze w monetach, drobniaki}
\Clue{31}{}{(1883-1924), pisarz austriacki; „Wyrok”, „Obżartuch”, „Zamek”, „Proces”, „Ameryka”, „List do Felicji”}
\Clue{32}{}{zdolność dostrzegania tego, co śmieszne; bystrość umysłu, która pomaga w zabawny, błyskotliwy sposób mówić i się zachowywać}
\Clue{33}{}{pogardliwie lub ironicznie o zięciu - mężu córki}
\Clue{34}{}{cecha człowieka, który żywi wątpliwości, zachowuje wobec czegoś dystans}
\Clue{35}{}{DYRYGENT}
\Clue{36}{}{minerał z gromady krzemianów, zaliczany do grupy plagioklazów}
\Clue{37}{}{FORTECA; miejsce umocnione fortyfikacjami stałymi}
\Clue{38}{}{cecha systemu operacyjnego umożliwiająca mu równoczesne wykonywanie więcej niż jednego procesu; zwykle za poprawną realizację wielozadaniowości odpowiedzialne jest jądro systemu operacyjnego}
\Clue{39}{}{zawartość bulionówki, miseczki lub większej filiżanki, która służy do podawania klarownych zup}\end{PuzzleClues}

\begin{PuzzleClues}{\textbf{Pionowe}\\}\Clue{1}{}{Asellidae - rodzina pancerzowców z rzędu równonogów}
\Clue{2}{}{nazwa glinu używana w technice}
\Clue{3}{}{wyznawca, zwolennik fatalizmu (prądu filozoficznego)}
\Clue{4}{}{miasto we Francji na wsch. od Lyonu, przemysł metalowy}
\Clue{5}{}{woreczek, kieszeń na pieniądze}
\Clue{6}{}{jeden z rodzajów gramatycznych (w polszczyźnie w liczbie pojedynczej) określający części mowy łączące się z zaimkiem ten}
\Clue{8}{}{chomik}
\Clue{9}{}{SHANDONG}
\Clue{10}{}{styl pływacki uważany za najtrudniejszy z czterech podstawowych stylów; styl pływacki charakteryzujący się jednoczesnym ruchem nad wodą obu rąk oraz odbijaniem się obiema nogami}
\Clue{14}{}{kod ISO 4217 drachmy - waluty Grecji przed wprowadzeniem euro}
\Clue{15}{}{symbol graficzny notacji muzycznej, znak wyznaczający bezwzględną wysokość dźwięku na linii, na której się znajduje}
\Clue{16}{}{JAMS, POCHRZYN}
\Clue{19}{}{to, że coś jest dżdżyste}
\Clue{20}{}{cziru, antylopa tybetańska, Pantholops hodgsonii - antylopa z rodziny krętorogich, jedyny przedstawiciel rodzaju Pantholops; jest endemitem Wyżyny Tybetańskiej}
\Clue{21}{}{DIAK; śpiewak kościelny, kierunek chóru kościelnego}
\Clue{22}{}{PUR, PU - polimer powstający w wyniku addycyjnej polimeryzacji  wielofunkcyjnych izocyjanianów do amin i alkoholi; w jego głównym łańcuchu występuje ugrupowanie uretanowe}
\Clue{24}{}{Clubiona sp. - gatunek pająka z rodziny aksamitnikowatych}
\Clue{25}{}{Crocodilia - rząd gadów reprezentowany przez trzy rodziny: krokodylowatych, aligatorowatych oraz gawiali}
\Clue{26}{}{kawał czegoś, co ma znaczną powierzchnię i jest stosunkowo płaskie, np. materiału, papieru itp}
\Clue{27}{}{przedmiot szkolny lub uczony w ramach kursu, na którym opanowuje się podstawy języka rosyjskiego}
\Clue{28}{}{miasto we Francji (Bretania), port wojenny, rybacki i handlowy nad Oceanem Atlantyckim}
\Clue{30}{}{rezultat czynności namierzania; informacja o położeniu czegoś względem namierzającego}\end{PuzzleClues}\newpage\section*{Krzyżówka 100}

\noindent\begin{Puzzle}{19}{32}|*	|*	|*	|*	|*	|*	|*	|*	|*	|*	|*	|*	|*	|*	|*	|[1][S]\drarr	|s	|o	|k	|*	|.
|*	|*	|[2][S]\drarr	|t	|o	|n	|*	|[3][S]\drarr	|b	|e	|r	|r	|y	|*	|[4][S]\darr	|r	|*	|*	|*	|[5][S]\darr	|.
|*	|*	|w	|*	|*	|[6][S]\darr	|[7][S]\rarr	|p	|i	|o	|n	|*	|*	|*	|f	|e	|*	|[8][S]\darr	|*	|o	|.
|*	|*	|r	|*	|*	|h	|*	|o	|*	|[9][S]\rarr	|c	|s	|e	|r	|e	|s	|*	|p	|[10][S]\darr	|v	|.
|*	|*	|z	|[11][S]\drarr	|g	|a	|r	|d	|z	|i	|e	|l	|*	|[12][S]\darr	|m	|z	|*	|o	|b	|e	|.
|*	|[13][S]\rarr	|a	|d	|a	|m	|ó	|w	|*	|*	|*	|*	|*	|i	|i	|k	|*	|k	|i	|r	|.
|*	|[14][S]\rarr	|s	|o	|ł	|a	|*	|o	|*	|*	|[15][S]\darr	|*	|*	|n	|n	|o	|*	|o	|a	|m	|.
|*	|*	|k	|b	|*	|*	|*	|j	|*	|[16][S]\darr	|p	|*	|*	|f	|i	|*	|*	|l	|ł	|y	|.
|*	|*	|l	|r	|[17][S]\rarr	|w	|i	|o	|s	|ł	|o	|*	|*	|i	|s	|*	|[18][S]\darr	|e	|a	|e	|.
|*	|*	|i	|o	|*	|*	|*	|w	|*	|a	|l	|*	|*	|r	|t	|*	|m	|n	|[][,]{ }	|r	|.
|*	|*	|w	|[][,]{ }	|*	|*	|[19][S]\darr	|a	|*	|d	|s	|[20][S]\darr	|[21][S]\rarr	|m	|a	|q	|u	|i	|s	|*	|.
|*	|*	|o	|m	|*	|*	|g	|t	|*	|*	|k	|h	|[22][S]\darr	|e	|*	|*	|n	|e	|t	|*	|.
|*	|*	|ś	|a	|*	|*	|a	|e	|*	|*	|o	|i	|f	|r	|*	|*	|i	|[][,]{ }	|o	|*	|.
|*	|*	|ć	|t	|*	|[23][S]\darr	|s	|*	|[24][S]\darr	|*	|j	|s	|e	|i	|*	|*	|c	|l	|p	|*	|.
|*	|*	|*	|e	|*	|l	|k	|*	|p	|*	|ę	|p	|l	|a	|*	|*	|y	|i	|a	|*	|.
|*	|*	|[25][S]\darr	|r	|*	|e	|o	|*	|r	|*	|z	|a	|i	|*	|*	|*	|p	|t	|*	|*	|.
|*	|*	|w	|i	|*	|l	|ń	|[26][S]\darr	|z	|*	|y	|n	|b	|*	|*	|[27][S]\darr	|i	|e	|*	|*	|.
|*	|*	|y	|a	|*	|k	|c	|r	|e	|[28][S]\darr	|c	|i	|i	|[29][S]\darr	|*	|n	|u	|r	|*	|*	|.
|*	|*	|k	|l	|*	|o	|z	|y	|t	|d	|z	|s	|e	|d	|*	|i	|m	|a	|*	|*	|.
|*	|*	|u	|n	|[30][S]\rarr	|w	|y	|k	|w	|i	|n	|t	|n	|o	|ś	|ć	|*	|c	|*	|*	|.
|*	|[31][S]\drarr	|s	|e	|r	|e	|k	|*	|ó	|n	|o	|a	|*	|r	|*	|[][,]{ }	|[32][S]\darr	|k	|*	|*	|.
|*	|e	|z	|*	|*	|*	|*	|*	|r	|g	|ś	|*	|*	|a	|*	|w	|p	|i	|*	|*	|.
|*	|m	|*	|[33][S]\drarr	|t	|r	|i	|o	|*	|l	|ć	|*	|*	|d	|[34][S]\darr	|i	|a	|e	|*	|*	|.
|*	|f	|[35][S]\drarr	|k	|o	|m	|y	|s	|z	|e	|*	|*	|*	|c	|s	|o	|k	|*	|*	|*	|.
|*	|i	|r	|a	|*	|[36][S]\drarr	|s	|m	|s	|*	|*	|*	|[37][S]\darr	|a	|p	|d	|l	|[38][S]\darr	|*	|*	|.
|[39][S]\rarr	|t	|o	|r	|*	|f	|*	|[40][S]\rarr	|s	|z	|e	|w	|c	|*	|a	|ą	|a	|k	|*	|*	|.
|*	|e	|s	|o	|[41][S]\rarr	|u	|k	|l	|ę	|k	|*	|*	|v	|*	|c	|c	|k	|o	|*	|*	|.
|*	|u	|t	|l	|[42][S]\rarr	|r	|e	|t	|r	|o	|g	|r	|e	|s	|j	|a	|*	|s	|*	|*	|.
|[43][S]\rarr	|t	|r	|i	|s	|a	|c	|h	|a	|r	|y	|d	|*	|*	|a	|*	|*	|t	|*	|*	|.
|*	|a	|y	|n	|[44][S]\rarr	|ż	|y	|c	|i	|e	|*	|*	|*	|*	|*	|*	|*	|u	|*	|*	|.
|*	|*	|*	|k	|[45][S]\rarr	|e	|f	|a	|*	|*	|[46][S]\rarr	|s	|z	|y	|m	|b	|a	|r	|k	|*	|.
|[47][S]\rarr	|k	|w	|a	|d	|r	|a	|t	|[][,]{ }	|m	|a	|g	|i	|c	|z	|n	|y	|*	|*	|*	|.
|*	|*	|*	|*	|*	|*	|*	|*	|*	|*	|*	|*	|*	|*	|*	|*	|*	|*	|*	|*	|.\end{Puzzle}

\newpage

\begin{PuzzleClues}{\textbf{Poziome}\\}\Clue{1}{}{porcja soku, tyle, ile mieści się w jakimś naczyniu lub opakowaniu}
\Clue{2}{}{w muzyce: miara odległości pomiędzy dźwiękami w skali}
\Clue{3}{}{prowincja historyczna we Francji, obecnie departament Cher i Indre}
\Clue{7}{}{najsłabsza i zarazem najmniejsza bierka w szachach}
\Clue{9}{}{ur. 1915r, pisarz węgierski; „Zimne dni”}
\Clue{11}{}{w technice: część gaźnika, w której rozpylane jest paliwo}
\Clue{13}{}{wieś w Polsce położona w województwie lubelskim, w powiecie łukowskim, siedziba gminy Adamów}
\Clue{14}{}{rzeka w południowej Polsce}
\Clue{17}{}{większa łyżka będąca elementem zastawy stołowej (sztućców), służąca do jedzenia np. zup}
\Clue{21}{}{francuscy partyzanci z okresu II wojny światowej}
\Clue{30}{}{niecodzienność, wyjątkowość, elegencja}
\Clue{31}{}{dekolt o trójątnym kształcie}
\Clue{33}{}{układ trzech walców w walcarce}
\Clue{35}{}{zarośla, zwłaszcza w miejscach podmokłych}
\Clue{36}{}{wiadomość tekstowa, przesyłana za pomocą telefonów komórkowych}
\Clue{39}{}{droga ułożona z szyn umocowanych na podłożu za pomocą podkładów}
\Clue{40}{}{błąd edytorski polegający na pozostawieniu pierwszego wersu nowego akapitu w ostatniej linii strony}
\Clue{41}{}{pozycja klęcząca}
\Clue{42}{}{nawracanie do czegoś (np. do jakiegoś stanu) z przeszłości}
\Clue{43}{}{trójcukier, węglowodan powstały na skutek połączenia trzech monosacharydów wiązaniem glikozydowym}
\Clue{44}{}{sfera jakiejś działalności; funkcjonowanie tej sfery lub uczestnictwo w niej}
\Clue{45}{}{biblijna jednostka objętości materiałów sypkich}
\Clue{46}{}{wieś w Polsce położona w województwie małopolskim, w powiecie gorlickim, w gminie Gorlice}
\Clue{47}{}{krzyżówka o kwadratowym diagramie, w której wiersze i kolumny o tym samym numerze są identyczne}\end{PuzzleClues}

\begin{PuzzleClues}{\textbf{Pionowe}\\}\Clue{1}{}{dżudoka, wicemistrz świata z 1981 r}
\Clue{2}{}{cecha zwierzęcia, które jest wrzaskliwe, hałaśliwe}
\Clue{3}{}{Chaetilidae - rodzina skorupiaków z rzędu równonogów}
\Clue{4}{}{mężczyzna podejmujący działalność na rzecz równouprawnienia kobiet, dążący do przyznania kobietom równych praw}
\Clue{5}{}{astronauta amerykański na pokładzie Columbii w 1982 r}
\Clue{6}{}{miasto w zachodniej Syrii, ośrodek administracyjny Muhafazy hama; XI w. meczet z minaretem}
\Clue{8}{}{populacja twórców i odbiorców będących w podobnym wieku; pojęcie z zakresu literaturoznastwa}
\Clue{10}{}{o powierzchni pokrytej śniegiem, na której widoczne są ślady zwierząt}
\Clue{11}{}{produkt materialny, który ma zaspokoić potrzeby człowieka - posiadacza}
\Clue{12}{}{izba chorych w klasztorze}
\Clue{15}{}{to, że ktoś jest polskojęzyczny, o byciu polskojęzycznym}
\Clue{16}{}{stan, gdy nie ma bałaganu}
\Clue{18}{}{ograniczone prawa obywatelskie w średniowieczu}
\Clue{19}{}{popularna nazwa gończego gaskońskiego - psa z grupy psów gończych (sekcji psów gończych), klasyfikowanego też do posokowców}
\Clue{20}{}{specjalista z zakresu hispanistyki}
\Clue{22}{}{francuski historiograf i architekt (1619-95), teoretyk sztuki}
\Clue{23}{}{kozodoje, Caprimulgiformes - rząd ptaków z podgromady Neornithes}
\Clue{24}{}{produkt uzyskany z przetworzenia jakiegoś surowca}
\Clue{25}{}{występ na murze obronnym}
\Clue{26}{}{intensywny płacz}
\Clue{27}{}{nowosyntetyzowana nić DNA, która powstaje w sposób ciągły w procesie replikacji DNA}
\Clue{28}{}{zatoka Oceanu Atlantyckiego w płd-zach. wybrzeżu Irlandii}
\Clue{29}{}{człowiek, który doradza}
\Clue{31}{}{czynszownik na prawie emfiteuzy}
\Clue{32}{}{gruby, pospolity materiał z bawełny, lnu, konopii, juty itp. (z pakuł)}
\Clue{33}{}{Aix sponsa - gatunek średniego ptaka wodnego z rodziny kaczkowatych (Anatidae); zamieszkuje Amerykę Północną między 30°N a 50°N}
\Clue{34}{}{rodzaj justunku}
\Clue{35}{}{ażurowy pokład na śródokręciu, na którym ustawione są łodzie ratunkowe}
\Clue{36}{}{żołnierz zajmujący się dostawą paszy dla koni}
\Clue{37}{}{kod ISO 4217 escudo zielonoprzylądkowego}
\Clue{38}{}{ręczne narzędzie do sadzenia leśnych sadzonek}\end{PuzzleClues}\newpage\section*{Krzyżówka 101}

\noindent\begin{Puzzle}{24}{21}|*	|*	|*	|*	|*	|[1][S]\darr	|*	|*	|*	|*	|*	|*	|*	|*	|*	|*	|[2][S]\darr	|*	|*	|[3][S]\darr	|*	|[4][S]\darr	|*	|*	|[5][S]\darr	|.
|*	|*	|*	|*	|*	|u	|*	|[6][S]\rarr	|t	|u	|r	|c	|z	|y	|n	|e	|k	|*	|*	|g	|[7][S]\rarr	|o	|z	|*	|h	|.
|*	|*	|*	|*	|[8][S]\rarr	|r	|e	|k	|o	|n	|s	|t	|r	|u	|k	|t	|o	|r	|*	|l	|*	|w	|*	|[9][S]\darr	|u	|.
|*	|*	|[10][S]\rarr	|z	|a	|g	|a	|d	|n	|i	|e	|n	|i	|e	|*	|[11][S]\darr	|z	|*	|*	|i	|*	|e	|*	|n	|g	|.
|*	|*	|*	|[12][S]\rarr	|b	|e	|s	|t	|i	|a	|*	|*	|*	|*	|[13][S]\darr	|h	|a	|*	|*	|p	|*	|ł	|[14][S]\darr	|i	|g	|.
|*	|*	|[15][S]\rarr	|l	|a	|n	|g	|e	|*	|*	|*	|*	|*	|*	|p	|a	|[][,]{ }	|*	|*	|t	|*	|n	|s	|k	|i	|.
|*	|*	|*	|*	|[16][S]\rarr	|s	|i	|a	|t	|k	|a	|*	|*	|*	|i	|n	|b	|*	|*	|y	|*	|i	|t	|o	|n	|.
|*	|*	|*	|*	|*	|*	|[17][S]\darr	|*	|*	|*	|*	|*	|*	|*	|l	|g	|e	|*	|*	|k	|*	|c	|o	|l	|s	|.
|*	|[18][S]\rarr	|e	|k	|t	|o	|p	|l	|a	|z	|m	|a	|*	|*	|o	|a	|z	|[19][S]\rarr	|k	|a	|b	|a	|ł	|a	|*	|.
|[20][S]\rarr	|p	|a	|p	|i	|e	|r	|[][,]{ }	|m	|i	|l	|i	|m	|e	|t	|r	|o	|w	|y	|*	|[21][S]\darr	|*	|ó	|i	|*	|.
|*	|*	|[22][S]\rarr	|m	|e	|c	|z	|[][,]{ }	|d	|e	|r	|b	|o	|w	|y	|*	|a	|*	|*	|*	|m	|*	|w	|t	|*	|.
|[23][S]\drarr	|p	|a	|r	|k	|i	|e	|t	|*	|*	|[24][S]\rarr	|a	|w	|g	|*	|*	|r	|*	|*	|*	|o	|*	|k	|a	|*	|.
|a	|[25][S]\rarr	|m	|o	|l	|o	|s	|y	|*	|*	|*	|*	|*	|*	|[26][S]\rarr	|b	|o	|t	|n	|i	|c	|k	|a	|*	|*	|.
|f	|*	|[27][S]\rarr	|k	|r	|ó	|t	|k	|o	|t	|e	|r	|m	|i	|n	|o	|w	|o	|ś	|ć	|*	|[28][S]\darr	|*	|[29][S]\darr	|*	|.
|e	|*	|[30][S]\rarr	|p	|i	|e	|r	|ś	|c	|i	|e	|ń	|[][,]{ }	|r	|o	|t	|a	|t	|o	|r	|ó	|w	|*	|d	|*	|.
|r	|[31][S]\rarr	|b	|r	|o	|d	|z	|k	|i	|*	|[32][S]\rarr	|s	|i	|t	|w	|a	|*	|*	|*	|*	|*	|ł	|[33][S]\darr	|u	|*	|.
|z	|*	|*	|[34][S]\rarr	|w	|i	|e	|l	|k	|i	|[][,]{ }	|a	|t	|r	|a	|k	|t	|o	|r	|*	|*	|o	|w	|m	|*	|.
|y	|[35][S]\rarr	|r	|e	|k	|a	|l	|k	|u	|l	|a	|c	|j	|a	|*	|*	|[36][S]\rarr	|z	|a	|w	|i	|s	|a	|k	|*	|.
|s	|[37][S]\rarr	|p	|r	|o	|w	|i	|z	|j	|a	|[][,]{ }	|m	|a	|k	|l	|e	|r	|s	|k	|a	|*	|y	|z	|a	|*	|.
|t	|[38][S]\rarr	|g	|a	|t	|u	|n	|e	|k	|[][,]{ }	|e	|n	|d	|e	|m	|i	|c	|z	|n	|y	|*	|*	|k	|*	|*	|.
|a	|*	|*	|*	|[39][S]\rarr	|m	|a	|s	|k	|u	|l	|i	|n	|i	|z	|m	|*	|*	|*	|*	|*	|*	|a	|*	|*	|.
|*	|*	|*	|*	|*	|*	|*	|*	|*	|*	|*	|*	|*	|*	|*	|*	|*	|*	|*	|*	|*	|*	|*	|*	|*	|.\end{Puzzle}

\newpage

\begin{PuzzleClues}{\textbf{Poziome}\\}\Clue{6}{}{zdrobniale: turczyn - koń pochodzący z Turcji, zazwyczaj reprezentant typu koni wschodnich}
\Clue{7}{}{uncja; jednostka masy równa 31,1034768 grama}
\Clue{8}{}{preparat kosmetyczny używany do odbudowania struktury zniszczonych części ciała np. włosów, paznokci, skóry}
\Clue{10}{}{problem, kwestia wymagająca rozstrzygnięcia, rozwiązania, przemyślenia}
\Clue{12}{}{drapieżnik; dzikie zwierzę, takie, którego można się bać}
\Clue{15}{}{filozof niemiecki (1828-75); przedstawiciel neokantyzmu}
\Clue{16}{}{plan czegoś, rozpisany wykaz tego, co się dzieje w określonym czasie}
\Clue{18}{}{w parapsychologii: rodzaj energii emitowanej przez żyjące organizmy}
\Clue{19}{}{duchowa mistyczno-filozoficzna szkoła judaizmu}
\Clue{20}{}{papier z nadrukiem w kratkę złożoną z milimterowych kwadratów}
\Clue{22}{}{mecz rozgrywany przez dwie lokalne drużyny}
\Clue{23}{}{konstrukcja z drewna, którą montuje się na odwrotnej stronie obrazu malowanego na desce, by zapobiec jej wypaczaniu}
\Clue{24}{}{kod ISO 4217 florina arubiańskiego}
\Clue{25}{}{molosowate, Molossida - rodzina nietoperzy charakteryzujących się ogonem wystającym z błony lotnej i przeciwstawnym pierwszym palcem stopy}
\Clue{26}{}{zatoka Morza Bałtyckiego, między Szwecją a Finlandią, pow. 111 tyś. km2}
\Clue{27}{}{cecha czegoś, co trwało przez krótki czas, np. zdarzenie, kurs}
\Clue{30}{}{grupa czterech mięśni, pokrywających głowę kości ramiennej i odgrywających rolę w stabilizacji barku}
\Clue{31}{}{francuski malarz, grafik i dekorator (1703-70) sceny pasterskie, mitologiczne, rodzajowe portrety, dekoracje teatralne}
\Clue{32}{}{grupa ludzi i środowisk powiązanych ze sobą wspólnymi interesami, popierających i broniących się wzajemnie, których celem jest trwanie na różnych poziomach i wymiarach władzy}
\Clue{34}{}{olbrzymie skupisko gromad galaktyk, rozciągające się od gwiazdozbioru Centaura i Hydry do gwiazdozbioru Pawia, z centrum w gwiazdozbiorze Węgielnicy}
\Clue{35}{}{ponowne przeliczenie, obliczenie wysokości, wielkości, stawki czegoś}
\Clue{36}{}{FERTAK}
\Clue{37}{}{prowizja z tytułu obrotu papierami wartościowymi}
\Clue{38}{}{gatunek unikatowy dla danego miejsca albo regionu, występujący na ograniczonym obszarze, nigdzie indziej niewystępujący naturalnie}
\Clue{39}{}{naśladowanie mężczyzn przez kobiety w strojach lub w zachowaniu, upodobnianie się wyglądem zewnętrznym przez kobiety do mężczyzn}\end{PuzzleClues}

\begin{PuzzleClues}{\textbf{Pionowe}\\}\Clue{1}{}{upomnienie, pisemne ponaglenie}
\Clue{2}{}{kozioł bezoarowy, Capra aegagrus - ssak krętorogi, protoplasta kozy domowej; występuje na pustynnych obszarach górskich południowo-zachodniej Azji, w Iranie, Afganistanie, Iraku aż po sam Kaukaz}
\Clue{3}{}{sztuka rzeźbienia szlachetnych lub półszlachetnych kamieni}
\Clue{4}{}{LUCERNIANKA - gat. biedronki, żeruje na lucernie}
\Clue{5}{}{ur. w 1901 r. chirurg amerykański, pionier chemoterapii chorób nowotworowych, odkrywca metody leczenia raka gruczołu krokowego, laureat Nobla}
\Clue{9}{}{osoba należąca do sekty nikolaitów}
\Clue{11}{}{pomieszczenie dla samolotów, szybowców, balonów}
\Clue{13}{}{malarz niemiecki (1826-86) kompozycje historyczne}
\Clue{14}{}{zakład gastronomiczny pełniący rolę stołówki}
\Clue{17}{}{uszkodzenie czegoś powstałe w wyniku przestrzelenia pociskiem}
\Clue{21}{}{zdolność zakładu do produkcji swoich wytworów}
\Clue{23}{}{przestępca zamieszany w jakąś aferę}
\Clue{28}{}{owłosienie głowy człowieka}
\Clue{29}{}{DUMA}
\Clue{33}{}{mała waza, małe naczynie na zupę}\end{PuzzleClues}\newpage\section*{Krzyżówka 102}

\noindent\begin{Puzzle}{22}{31}|*	|*	|*	|[1][S]\drarr	|b	|o	|r	|o	|w	|i	|n	|a	|*	|*	|*	|*	|*	|*	|*	|*	|*	|*	|*	|.
|*	|*	|*	|k	|*	|*	|[2][S]\darr	|*	|*	|*	|*	|[3][S]\darr	|*	|*	|*	|[4][S]\drarr	|w	|r	|o	|n	|a	|*	|*	|.
|*	|[5][S]\drarr	|k	|u	|ł	|a	|n	|[][,]{ }	|t	|u	|r	|k	|m	|e	|ń	|s	|k	|i	|*	|*	|[6][S]\darr	|*	|*	|.
|*	|i	|[7][S]\drarr	|r	|e	|w	|o	|l	|u	|c	|j	|a	|*	|[8][S]\rarr	|z	|a	|m	|e	|k	|*	|p	|*	|*	|.
|*	|n	|s	|k	|*	|*	|ż	|*	|*	|*	|*	|w	|[9][S]\rarr	|k	|e	|d	|e	|n	|c	|j	|a	|*	|*	|.
|*	|t	|t	|u	|*	|*	|y	|*	|*	|*	|[10][S]\darr	|ę	|*	|*	|[11][S]\darr	|y	|[12][S]\darr	|[13][S]\darr	|*	|*	|n	|[14][S]\darr	|*	|.
|*	|e	|e	|m	|*	|[15][S]\darr	|c	|*	|*	|*	|k	|c	|*	|[16][S]\darr	|b	|s	|m	|a	|*	|*	|e	|p	|*	|.
|*	|r	|r	|a	|*	|f	|e	|[17][S]\darr	|*	|*	|i	|z	|[18][S]\rarr	|f	|o	|t	|e	|l	|*	|[19][S]\darr	|l	|r	|*	|.
|*	|f	|o	|*	|[20][S]\darr	|a	|*	|j	|*	|*	|e	|y	|[21][S]\darr	|a	|j	|y	|l	|a	|*	|d	|[][,]{ }	|z	|*	|.
|*	|e	|w	|*	|f	|ł	|*	|e	|*	|*	|l	|n	|b	|l	|e	|c	|i	|b	|*	|z	|s	|y	|*	|.
|*	|r	|a	|*	|u	|d	|*	|d	|*	|*	|i	|*	|i	|u	|r	|z	|l	|a	|*	|i	|t	|g	|*	|.
|*	|o	|n	|*	|n	|z	|*	|n	|*	|*	|s	|*	|o	|n	|*	|n	|i	|s	|*	|e	|e	|o	|*	|.
|*	|m	|i	|*	|k	|i	|*	|o	|*	|*	|z	|[22][S]\darr	|p	|*	|[23][S]\darr	|o	|t	|t	|[24][S]\darr	|w	|r	|t	|*	|.
|*	|e	|e	|*	|c	|s	|*	|d	|*	|*	|n	|a	|i	|[25][S]\darr	|c	|ś	|*	|e	|a	|i	|o	|o	|*	|.
|*	|t	|[][,]{ }	|*	|j	|t	|*	|n	|*	|*	|i	|f	|e	|g	|h	|ć	|*	|r	|s	|ę	|w	|w	|*	|.
|*	|r	|s	|*	|o	|k	|*	|i	|[26][S]\drarr	|b	|a	|r	|r	|e	|l	|*	|*	|*	|p	|t	|n	|a	|*	|.
|*	|[][,]{ }	|t	|[27][S]\drarr	|n	|a	|ł	|ó	|g	|*	|k	|y	|w	|n	|o	|*	|[28][S]\drarr	|r	|o	|n	|i	|n	|*	|.
|*	|l	|o	|l	|a	|*	|[29][S]\darr	|w	|ó	|[30][S]\darr	|*	|k	|i	|i	|r	|*	|i	|*	|ł	|a	|c	|i	|*	|.
|*	|a	|c	|i	|l	|*	|k	|k	|w	|g	|*	|a	|a	|u	|e	|*	|n	|*	|e	|s	|z	|e	|*	|.
|*	|s	|h	|n	|i	|*	|o	|a	|n	|o	|*	|n	|s	|s	|k	|[31][S]\darr	|w	|*	|c	|t	|y	|*	|*	|.
|*	|e	|a	|i	|z	|*	|n	|*	|o	|g	|*	|k	|t	|z	|[][,]{ }	|a	|o	|*	|z	|k	|*	|*	|*	|.
|*	|r	|s	|a	|m	|*	|t	|*	|*	|l	|*	|a	|e	|*	|m	|k	|l	|*	|n	|a	|*	|*	|*	|.
|*	|o	|t	|[][,]{ }	|*	|[32][S]\drarr	|r	|d	|z	|e	|ń	|*	|k	|*	|i	|r	|u	|*	|o	|*	|*	|*	|*	|.
|[33][S]\drarr	|w	|y	|b	|i	|j	|a	|c	|z	|*	|*	|*	|*	|*	|e	|e	|c	|*	|ś	|*	|*	|*	|*	|.
|o	|y	|c	|r	|[34][S]\drarr	|o	|p	|e	|r	|a	|t	|*	|*	|*	|d	|c	|j	|*	|ć	|*	|*	|*	|*	|.
|s	|*	|z	|z	|s	|r	|u	|*	|[35][S]\rarr	|m	|i	|r	|i	|s	|z	|j	|a	|*	|*	|*	|*	|*	|*	|.
|t	|*	|n	|e	|u	|d	|n	|[36][S]\rarr	|g	|e	|o	|f	|a	|g	|i	|a	|*	|*	|*	|*	|*	|*	|*	|.
|r	|*	|e	|g	|c	|a	|k	|*	|*	|*	|*	|*	|*	|*	|[][S](	|*	|*	|*	|*	|*	|*	|*	|*	|.
|o	|*	|*	|o	|h	|n	|t	|*	|*	|*	|[37][S]\rarr	|n	|a	|w	|i	|e	|w	|n	|i	|k	|*	|*	|*	|.
|g	|*	|*	|w	|a	|i	|*	|*	|*	|*	|*	|[38][S]\rarr	|w	|h	|i	|t	|e	|*	|*	|*	|*	|*	|*	|.
|i	|[39][S]\rarr	|k	|a	|r	|a	|w	|a	|s	|z	|*	|*	|*	|*	|[][S])	|*	|*	|*	|*	|*	|*	|*	|*	|.
|*	|*	|*	|*	|*	|*	|*	|*	|*	|*	|*	|*	|*	|*	|*	|*	|*	|*	|*	|*	|*	|*	|*	|.\end{Puzzle}

\newpage

\begin{PuzzleClues}{\textbf{Poziome}\\}\Clue{1}{}{gleba kredowa występująca m.in. na Lubelszczyźnie, rędzina namyta; niekiedy (zwłaszcza w regionie Małopolski) utożsamia się rędziny i borowiny}
\Clue{4}{}{potoczna nazwa wrony siwej, pospolity szary ptak z czarnymi skrzydłami}
\Clue{5}{}{Equus hemionus kulan -  podgatunek osła azjatyckiego z rodziny koniowatych; zamieszkuje Turkmenistan, Afganistan i Iran}
\Clue{7}{}{gwałtowna zmiana władzy, ustroju politycznego, organizacji społeczeństwa, której towarzyszą walka zbrojna i przemoc; wstrząs dziejowy spowodowany zmianą (potrzebą zmiany - gdyż często jest to wyraz oddolnego buntu) społeczno-polityczną, zwykle związany ze znaczną zmianą kulturową i ekonomiczną}
\Clue{8}{}{ruchoma część odtylcowej broni palnej}
\Clue{9}{}{następstwo dźwięków nadające frazie muzycznej charakter zakończenia}
\Clue{18}{}{rodzaj siedziska z oparciem na plecy wraz z podłokietnikami}
\Clue{26}{}{BARYŁKA; beczułka, amerykańska jednostka objętości równa 1,6 hl}
\Clue{27}{}{zwyczaj, przyzwyczajenie, od którego trudno się uwolnić, zrezygnować z niego}
\Clue{28}{}{w feudalnej Japonii: chłop pańszczyźniany}
\Clue{32}{}{zasadnicza część reaktora, w której następują przemiany jądrowe, będące źródłem energii w formie promieniowania oraz ciepła}
\Clue{33}{}{w grach zespołowych w piłkę oraz w pétanque: zawodnik wybijający piłkę (kulę do gry) przeciwnikowi}
\Clue{34}{}{pisemne opracowanie danego zagadnienia technicznego}
\Clue{35}{}{Mirischia - rodzaj dinozaura z grupy celurozaurów; żył w okresie wczesnej kredy na terenach Ameryki Południowej}
\Clue{36}{}{jedno z zaburzeń łaknienia lub stan chorobowy - spożywanie ziemi}
\Clue{37}{}{przewód rurowy wystający ponad górny pokład statku służący do naturalnej wentylacji pomieszczeń}
\Clue{38}{}{pisarz australijski ur. 1912, „Wóz ognisty”, „Węzeł”, „Oko cyklonu”}
\Clue{39}{}{naramiennik zbroi}\end{PuzzleClues}

\begin{PuzzleClues}{\textbf{Pionowe}\\}\Clue{1}{}{przyprawa; ekstrakt kłączy ostryżu}
\Clue{2}{}{narzędzie używane do cięcia materiałów przy udziale niewielkiej siły rąk}
\Clue{3}{}{osiedle położone w północno-zachodniej i zachodniej części warszawskiej dzielnicy Rembertów}
\Clue{4}{}{to, że ktoś jest sadystyczny - czerpie radość z krzywdy innych}
\Clue{5}{}{przyrząd pomiarowy oparty na zjawisku interferencji fal, w którym źródłem światła jest laser}
\Clue{6}{}{urządzenie elektryczne umożliwiające kontrolę innych urządzeń elektrycznych, realizujących pewne procesy, np. technologiczne lub produkcyjne}
\Clue{7}{}{gałąź teorii sterowania, która zajmuje się zagadnieniami występowania niepewności w układach regulacji}
\Clue{10}{}{żywiczlin, cyprzyk, Callitris - rodzaj z rodziny cyprysowatych (Cupressaceae); obejmuje 16 gatunków występujących w Australii oraz na Nowej Kaledonii}
\Clue{11}{}{ślizg sportowy}
\Clue{12}{}{minerał o barwie bladożółtej lub bezbarwny}
\Clue{13}{}{niemiecki malarz, grafik i architekt (1480-1538) reprezentant tzw. szkoły naddunajskiej; obrazy religijne i alegoryczne, pejzaże}
\Clue{14}{}{zrobienie jakiejś potrawy, dania}
\Clue{15}{}{LALEMACJA - jednoroczna roślina oleista pochodzenia azjatyckiego uprawiana dla nasion zawierających około 40\% tłuszczu}
\Clue{16}{}{miasto w środkowej Szwecji w regionie Bergslagen, ośrodek administracyjny okręgu Kopparberg}
\Clue{17}{}{wydawnictwo okazyjne, aperiodyczne, niezaliczane do prasy}
\Clue{19}{}{impreza z okazji czyichś dziewiętnastych urodzin; nie powiemy tak o ilości lat - dopiero wielokrotności 10 lat i ułamki stulecia mają takie podwójne znaczenie}
\Clue{20}{}{kierunek w socjologii zakładający, że istotne jest badanie funkcji, jakie zjawiska społeczne i kulturowe mają dla całego społeczeństwa; przeciwieństwo ewolucjonizmu}
\Clue{21}{}{pierwiastek śladowy, mikroelement - pierwiastek chemiczny występujący w niewielkich ilościach w organizmach żywych, konieczny im życia i rozwoju}
\Clue{22}{}{południowoamerykański ptak z papug}
\Clue{23}{}{nieorganiczny związek chemiczny, sól kwasu solnego i miedzi na +2 stopniu utlenienia}
\Clue{24}{}{cecha jakichś działań, postępowania, które nie jest nastawione na nawiązywanie relacji międzyludzkich, jest przejawem niechęci do przebywania wśród ludzi, interakcji z nimi}
\Clue{25}{}{osoba posiadająca wybitnie ponadprzeciętne zdolności intelektualne}
\Clue{26}{}{kał, odchód, ekskrement}
\Clue{27}{}{linia zetknięcia się powierzchni wody w zbiorniku lub cieku wodnym z powierzchnią lądu}
\Clue{28}{}{proces składowy gastrulacji występujący np. u zarodków płazów i ryb, polegający na zawinięciu się (wpukleniu) blastodermy (ściany blastuli) pod spód, do blastocelu, i przemieszczaniu się komórek po jego powierzchni wewnętrznej}
\Clue{29}{}{element przeciwstawny czemuś, uzupełniający coś na zasadzie kontrastu}
\Clue{30}{}{przyrząd optyczny, przypominający wyglądem okulary, którego zadaniem jest ochrona oczu i części twarzy przed niekorzystnymi czynnikami atmosferycznymi, zwłaszcza opadami i wiatrem, wykorzystywany przede wszystkim w narciarstwie oraz w pracy z bardzo silnym światłem, opiłkami i iskrami}
\Clue{31}{}{rozrost tkanek}
\Clue{32}{}{kraj na Bliskim Wschodzie położony w Azji Południowo-Zachodniej}
\Clue{33}{}{metalowe kabłąki z kolcem lub zębatą gwiazdą przypinane do butów, ułatwiają prowadzenie konia przez jeźdźca}
\Clue{34}{}{zwyczajowa nazwa jezior polihumusowych lub dystroficznych występujących na Suwalszczyźnie}\end{PuzzleClues}\newpage\section*{Krzyżówka 103}

\noindent\begin{Puzzle}{24}{33}|*	|*	|*	|*	|[1][S]\drarr	|j	|e	|d	|y	|n	|y	|[][,]{ }	|p	|i	|e	|r	|ś	|c	|i	|e	|ń	|*	|*	|*	|*	|.
|*	|[2][S]\rarr	|f	|r	|a	|n	|c	|z	|y	|z	|a	|[][,]{ }	|k	|o	|n	|w	|e	|r	|s	|y	|j	|n	|a	|*	|*	|.
|*	|[3][S]\darr	|[4][S]\darr	|[5][S]\rarr	|r	|y	|n	|e	|k	|[][,]{ }	|k	|o	|n	|t	|e	|s	|t	|o	|w	|a	|l	|n	|y	|*	|*	|.
|[6][S]\drarr	|j	|a	|c	|k	|s	|o	|n	|*	|*	|[7][S]\darr	|*	|*	|*	|[8][S]\drarr	|f	|a	|j	|z	|a	|b	|a	|d	|*	|*	|.
|k	|u	|m	|[9][S]\darr	|u	|*	|*	|[10][S]\darr	|*	|[11][S]\darr	|b	|*	|[12][S]\rarr	|o	|f	|l	|a	|g	|*	|*	|[13][S]\darr	|[14][S]\darr	|*	|*	|*	|.
|r	|t	|f	|h	|s	|*	|[15][S]\drarr	|w	|y	|d	|a	|l	|i	|n	|a	|*	|*	|[16][S]\drarr	|p	|u	|z	|o	|*	|*	|*	|.
|ó	|e	|i	|a	|z	|*	|w	|i	|*	|z	|n	|*	|*	|*	|s	|*	|*	|k	|*	|*	|a	|ś	|*	|*	|*	|.
|t	|r	|t	|n	|[][,]{ }	|*	|i	|z	|*	|w	|i	|*	|*	|[17][S]\darr	|o	|[18][S]\darr	|*	|l	|*	|*	|b	|w	|*	|*	|*	|.
|k	|b	|e	|d	|i	|*	|k	|y	|*	|o	|e	|[19][S]\darr	|*	|c	|l	|s	|[20][S]\darr	|a	|*	|[21][S]\darr	|i	|i	|*	|*	|*	|.
|o	|o	|a	|e	|n	|*	|t	|t	|[22][S]\darr	|n	|c	|r	|[23][S]\drarr	|h	|a	|n	|s	|u	|e	|z	|j	|a	|*	|*	|*	|.
|s	|g	|t	|l	|t	|*	|[][,]{ }	|k	|k	|e	|z	|o	|a	|m	|[][,]{ }	|o	|k	|s	|*	|ł	|a	|d	|*	|*	|*	|.
|z	|*	|r	|[][,]{ }	|r	|[24][S]\darr	|i	|a	|a	|k	|k	|z	|n	|a	|s	|r	|o	|t	|[25][S]\darr	|y	|k	|c	|*	|*	|*	|.
|[][,]{ }	|*	|*	|ł	|o	|p	|[][,]{ }	|*	|m	|*	|a	|t	|a	|r	|z	|r	|t	|r	|ż	|[][,]{ }	|a	|z	|*	|*	|*	|.
|n	|*	|*	|a	|l	|r	|o	|*	|y	|*	|*	|w	|l	|a	|a	|i	|n	|o	|ó	|c	|*	|e	|*	|*	|*	|.
|a	|[26][S]\darr	|*	|ń	|i	|o	|p	|*	|c	|*	|*	|ó	|i	|*	|b	|*	|i	|f	|ł	|z	|[27][S]\darr	|n	|*	|*	|*	|.
|m	|s	|*	|c	|g	|f	|i	|[28][S]\rarr	|z	|g	|o	|r	|z	|e	|l	|*	|k	|o	|t	|ł	|c	|i	|*	|*	|*	|.
|u	|k	|*	|u	|a	|i	|e	|*	|e	|[29][S]\darr	|*	|[][,]{ }	|a	|[30][S]\darr	|a	|*	|*	|b	|a	|o	|i	|e	|*	|*	|*	|.
|r	|a	|*	|s	|t	|l	|r	|[31][S]\darr	|k	|l	|[32][S]\darr	|i	|[][,]{ }	|d	|s	|*	|*	|i	|c	|w	|a	|[][,]{ }	|*	|*	|*	|.
|n	|l	|*	|z	|o	|[][,]{ }	|u	|t	|*	|o	|e	|z	|w	|y	|t	|*	|[33][S]\darr	|a	|z	|i	|ł	|w	|*	|*	|*	|.
|i	|n	|*	|k	|r	|l	|n	|e	|*	|k	|k	|o	|s	|j	|a	|[34][S]\darr	|d	|*	|e	|e	|o	|o	|*	|*	|*	|.
|k	|i	|[35][S]\drarr	|o	|s	|i	|e	|m	|n	|a	|s	|t	|k	|a	|*	|c	|e	|*	|k	|k	|[][,]{ }	|l	|*	|*	|*	|.
|o	|k	|k	|w	|k	|p	|k	|a	|[36][S]\darr	|t	|p	|o	|a	|m	|*	|i	|s	|*	|[][,]{ }	|*	|a	|i	|*	|*	|*	|.
|w	|[][,]{ }	|r	|y	|i	|i	|*	|t	|d	|o	|o	|n	|ź	|e	|*	|ę	|z	|*	|c	|*	|c	|*	|*	|*	|*	|.
|y	|p	|ó	|*	|*	|d	|*	|*	|r	|r	|z	|i	|n	|n	|*	|ż	|c	|*	|e	|*	|e	|[37][S]\darr	|*	|*	|*	|.
|*	|o	|l	|[38][S]\drarr	|g	|o	|ł	|d	|a	|*	|y	|c	|i	|t	|*	|k	|z	|[39][S]\darr	|j	|*	|t	|b	|*	|*	|*	|.
|*	|k	|i	|w	|*	|w	|*	|*	|b	|*	|c	|z	|k	|*	|*	|a	|o	|s	|l	|*	|o	|r	|[40][S]\darr	|*	|*	|.
|*	|r	|k	|y	|*	|y	|*	|*	|i	|*	|j	|n	|o	|*	|*	|[][,]{ }	|w	|z	|o	|*	|n	|y	|s	|*	|*	|.
|*	|e	|*	|s	|*	|*	|*	|*	|n	|*	|a	|y	|w	|*	|*	|g	|n	|a	|ń	|*	|o	|g	|e	|[41][S]\darr	|*	|.
|[42][S]\drarr	|w	|a	|p	|i	|e	|n	|n	|i	|k	|*	|*	|a	|*	|*	|ł	|i	|t	|s	|[43][S]\drarr	|w	|a	|m	|s	|*	|.
|r	|n	|*	|a	|[44][S]\rarr	|l	|i	|g	|a	|w	|k	|a	|*	|*	|*	|o	|c	|a	|k	|c	|e	|d	|p	|i	|*	|.
|ó	|y	|*	|*	|*	|[45][S]\rarr	|z	|a	|k	|w	|a	|t	|e	|r	|o	|w	|a	|n	|i	|e	|*	|i	|e	|e	|*	|.
|g	|*	|[46][S]\rarr	|p	|o	|n	|c	|z	|*	|[47][S]\rarr	|t	|e	|o	|r	|i	|a	|*	|*	|*	|l	|*	|e	|r	|ć	|*	|.
|*	|*	|*	|*	|*	|*	|*	|*	|[48][S]\rarr	|f	|r	|a	|n	|i	|a	|*	|*	|*	|*	|*	|*	|r	|*	|*	|*	|.
|*	|*	|*	|*	|*	|*	|*	|*	|*	|*	|*	|*	|*	|*	|*	|*	|*	|*	|*	|*	|*	|*	|*	|*	|*	|.\end{Puzzle}

\newpage

\begin{PuzzleClues}{\textbf{Poziome}\\}\Clue{1}{}{fikcyjny artefakt ze stworzonej przez J. R. R. Tolkiena mitologii Śródziemia, najpotężniejszy z Pierścieni Władzy, zniszczony w ogniu Góry Przeznaczenia}
\Clue{2}{}{franczyza, w której franczyzobiorca rozszerza działalność, włączając we franczyzę podmioty działające w jego środowisku i jego branży i robiąc z nich swoich franczyzobiorców}
\Clue{5}{}{rynek oligopolistyczny, na którym sprzedawcy zachowują się podobnie jak w warunkach konkurencyjnych}
\Clue{6}{}{miasto w USA (Michigan); przemysł samochodowy, maszynowy, elektrotechniczny, chemiczny}
\Clue{8}{}{miasto w płn.-wsch. części Afganistanu ośrodek administracyjny prowincji Badachszan}
\Clue{12}{}{niemiecki obóz jeniecki, w którym trzymani byli oficerowie wzięci do niewoli w czasie działań wojennych lub okupacyjnych}
\Clue{15}{}{substancja będąca zbędnym produktem przemiany materii i wydalana przez żywy organizm poza jego obręb}
\Clue{16}{}{pisarz amerykański, autor; „Ojca chrzestnego”}
\Clue{23}{}{Hanssuesia - rodzaj dinozaura z rodziny pachycefalozaurów, żyjący w okresie późnej kredy na terenach Ameryki Północnej; długość ciała 3 m, wysokość 1,2 m, ciężar 60 kg}
\Clue{28}{}{rodzaj nekrozy w organizmie roślinnym, prowadzący do zamierania całych organów}
\Clue{35}{}{grupa obiektów, która składa się z 18 elementów}
\Clue{38}{}{tania wódka, drugorzędnej jakości}
\Clue{42}{}{piec, w którym przetwarza się skały wapienne na wapno palone}
\Clue{43}{}{krótki męski kaftan usztywniony, podszyty watą, wykonany z sukna, atłasu lub aksamitu noszony w XVI-XVII w}
\Clue{44}{}{ludowy instrument muzyczny o długiej trąbie - trombita}
\Clue{45}{}{czas zajmowania przez kogoś jakiegoś pomieszczenia}
\Clue{46}{}{syrop (zwykle alkoholowy), który jest używany do nasączania ciast, do nadawania im specyficznego aromatu}
\Clue{47}{}{koncepcja oparta na poznaniu}
\Clue{48}{}{popoluarna nazwa elektrycznej pralki wirnikowa używanej w Polsce przed upowszechnieniem się w latach 80. XX w. bębnowych pralek automatycznych; pochodzi od nazwy pralki Frania, która była produkowana pod znakiem firmowym SHL w Zakładach Wyrobów Metalowych w Kielcach}\end{PuzzleClues}

\begin{PuzzleClues}{\textbf{Pionowe}\\}\Clue{1}{}{jednostka stanowiąca mierniki produkcji przy oprawie książki}
\Clue{3}{}{miasto w Niemczech (Brandenburgia) na płd. od Poczdamu; ważny węzeł kolejowy}
\Clue{4}{}{półkolista widownia wznosząca się ku górze, np. w sali koncertowej}
\Clue{6}{}{Brachythecium geheebii - gatunek mchu z rodziny krótkoszowatych; mech objęty w Polsce ścisłą ochroną gatunkową}
\Clue{7}{}{część żarówki}
\Clue{8}{}{Canavalia gladiata - gatunek fasoli o długich strąkach uprawianiej w Azji i Afryce}
\Clue{9}{}{forma handlu, w której obrót odbywa się przy udziale nadmiernej ilości pośredników}
\Clue{10}{}{członkini żeńskiego zakonu kontemplacyjnego o regule augustiańskiej założonego we Francji w 1610 r}
\Clue{11}{}{karta koloru karo}
\Clue{13}{}{osoba niefrasobliwa, lekkomyślna, wszczynająca zażarte bójki}
\Clue{14}{}{pojęcie z zakresu prawa cywilnego oznaczające przejaw woli ludzkiej zmierzający do wywołania skutku prawnego w postaci powstania, zmiany, ustania stosunku prawnego}
\Clue{15}{}{utrzymanie, takie, które zaspakaja potrzeby}
\Clue{16}{}{przenośnie: ciasnota umysłowa, strach przed myśleniem ambitnym i przyszłościowym; pewna szkodliwa perspektywa w widzeniu świata}
\Clue{17}{}{stado jeleni w gwarze łowieckiej}
\Clue{18}{}{(1179-1241), islandzki uczony i pisarz, najwybitniejszy przedstawiciel literatury staroskandynawskiej}
\Clue{19}{}{roztwór, który w kontakcie z innym roztworem przez błonę półprzepuszczalną pozostaje z nim w osmotycznej równowadze dynamicznej}
\Clue{20}{}{pastuch}
\Clue{21}{}{ktoś, kto jest zły, czyni zło, ma wiele negatywnych cech}
\Clue{22}{}{zdrobniale o minerale}
\Clue{23}{}{rodzaj badania ilościowego oparty na wskaźnikach, przedstawiających relacje określonych wielkości finansowych, ważnych z punktu widzenia ich wzajemnych związków}
\Clue{24}{}{badanie krwi oznaczające stężenie cholesterolu, jego frakcji LDL i HDL oraz trójglicerydów}
\Clue{25}{}{Etroplus suratensis - ryba z rodziny pielęgnicowatych}
\Clue{26}{}{Buckladiella affinis - gatunek mchu z rodziny strzechwowatych; gatunek borealno-górski, w Polsce występuje głównie w Sudetach}
\Clue{27}{}{rodzaj ciała ketonowego, związek chemiczny, który powstaje w organizmie wskutek utleniania się kwasów tłuszczowych}
\Clue{29}{}{zasadźca; ktoś, kto lokuje miasto lub wieś}
\Clue{30}{}{minerał zbudowany z węgla o strukturze diamentu}
\Clue{31}{}{sprawa do załatwienia, jakieś czynności i wydarzenia powiązane ze sobą, istotne dla kogoś}
\Clue{32}{}{ilość światła padającego na film (lub na sensor elektroniczny w przypadku aparatu cyfrowego) konieczna dla prawidłowego zrobienia zdjęcia fotograficznego}
\Clue{33}{}{część instalacji prysznicowej, umieszczany u góry kabiny, rozpraszający strumień wody na wzór kropel deszczu}
\Clue{34}{}{zmęczenie, zwłaszcza niechęć do wstawania z łóżka, również wywołane kacem}
\Clue{35}{}{wyprawione futerko królików}
\Clue{36}{}{wóz o bokach przypominających drabiny, używany przez rolników}
\Clue{37}{}{dawny stopień wojskowy, pośredni między pułkownikiem a generałem}
\Clue{38}{}{struktura anatomiczna w ludzkim mózgu zaliczana do kresomózgowia}
\Clue{39}{}{w niektórych religiach uosobienie zła; diabeł}
\Clue{40}{}{architekt niemiecki (1803-79), gmach opery w Dreźnie}
\Clue{41}{}{wytwór zwierzęcy, konstrukcja, którą tkają niektóre stawonogi z nici przędnych, głównie w celach łownych}
\Clue{42}{}{w piłce nożnej: rzut z rogu boiska}
\Clue{43}{}{efekt, który ktoś zakłada i chce osiągnąć, planowane zwieńczenie podejmowanych działań}\end{PuzzleClues}\newpage\section*{Krzyżówka 104}

\noindent\begin{Puzzle}{25}{24}|*	|*	|*	|*	|*	|*	|*	|*	|[1][S]\drarr	|s	|k	|o	|r	|p	|i	|o	|n	|*	|*	|[2][S]\drarr	|s	|m	|i	|t	|h	|*	|.
|*	|*	|*	|*	|*	|[3][S]\rarr	|ł	|y	|ż	|a	|*	|[4][S]\rarr	|k	|r	|w	|a	|w	|i	|e	|n	|i	|e	|c	|*	|*	|*	|.
|*	|*	|*	|[5][S]\rarr	|l	|a	|u	|d	|a	|*	|[6][S]\darr	|*	|*	|[7][S]\rarr	|w	|e	|l	|o	|c	|i	|z	|a	|u	|r	|*	|*	|.
|*	|*	|*	|*	|*	|*	|*	|*	|g	|*	|t	|[8][S]\rarr	|s	|i	|t	|[][,]{ }	|d	|r	|o	|b	|n	|y	|*	|*	|[9][S]\darr	|[10][S]\darr	|.
|*	|*	|*	|*	|*	|[11][S]\rarr	|ć	|w	|i	|e	|r	|ć	|w	|i	|a	|t	|r	|*	|*	|y	|*	|*	|*	|*	|p	|m	|.
|*	|*	|*	|*	|[12][S]\drarr	|z	|g	|i	|e	|r	|z	|a	|n	|k	|a	|*	|*	|*	|*	|b	|*	|*	|*	|*	|s	|a	|.
|*	|*	|*	|*	|n	|*	|*	|*	|w	|*	|y	|[13][S]\rarr	|m	|e	|z	|o	|z	|a	|u	|r	|y	|*	|*	|*	|i	|k	|.
|[14][S]\rarr	|u	|p	|ł	|a	|z	|e	|k	|*	|*	|n	|*	|*	|[15][S]\rarr	|w	|ą	|s	|o	|n	|o	|g	|i	|*	|[16][S]\darr	|z	|s	|.
|*	|*	|*	|*	|g	|[17][S]\drarr	|p	|ę	|t	|l	|a	|*	|*	|*	|*	|*	|*	|*	|*	|d	|*	|*	|*	|s	|ą	|y	|.
|*	|*	|*	|*	|a	|h	|*	|[18][S]\drarr	|w	|ą	|s	|k	|o	|p	|y	|s	|k	|o	|w	|a	|t	|e	|*	|i	|b	|m	|.
|*	|*	|[19][S]\darr	|*	|n	|o	|*	|o	|*	|*	|t	|*	|*	|*	|*	|*	|*	|*	|*	|w	|*	|*	|*	|ł	|[][,]{ }	|a	|.
|*	|*	|o	|*	|i	|n	|*	|p	|*	|*	|k	|*	|[20][S]\darr	|[21][S]\drarr	|u	|r	|d	|z	|i	|k	|*	|*	|[22][S]\darr	|y	|t	|l	|.
|*	|*	|r	|*	|a	|w	|*	|i	|[23][S]\darr	|*	|a	|*	|d	|w	|*	|*	|[24][S]\darr	|[25][S]\darr	|[26][S]\darr	|o	|[27][S]\darr	|*	|ż	|[][,]{ }	|a	|i	|.
|*	|*	|i	|[28][S]\rarr	|c	|e	|r	|e	|s	|*	|*	|*	|e	|o	|*	|*	|t	|k	|e	|w	|p	|*	|ó	|p	|y	|z	|.
|*	|*	|o	|*	|z	|d	|*	|ń	|e	|*	|*	|*	|r	|d	|*	|*	|r	|u	|p	|i	|o	|[29][S]\darr	|ł	|o	|l	|m	|.
|*	|*	|n	|*	|*	|*	|*	|s	|k	|[30][S]\rarr	|o	|s	|k	|a	|r	|ż	|y	|c	|i	|e	|l	|s	|t	|w	|o	|*	|.
|*	|*	|i	|[31][S]\darr	|*	|*	|[32][S]\rarr	|k	|r	|o	|p	|k	|a	|*	|*	|*	|l	|h	|k	|c	|e	|z	|o	|i	|r	|*	|.
|*	|*	|s	|q	|*	|*	|*	|i	|e	|*	|*	|*	|*	|*	|*	|*	|m	|n	|u	|[][,]{ }	|w	|a	|d	|e	|a	|*	|.
|*	|[33][S]\drarr	|t	|u	|n	|e	|l	|*	|t	|*	|*	|*	|*	|*	|*	|*	|a	|i	|r	|c	|k	|m	|z	|t	|*	|*	|.
|*	|e	|a	|e	|*	|*	|[34][S]\rarr	|p	|a	|w	|i	|l	|o	|n	|i	|k	|*	|a	|e	|z	|a	|p	|i	|r	|*	|*	|.
|*	|r	|*	|n	|*	|*	|*	|[35][S]\rarr	|r	|a	|k	|i	|e	|t	|a	|*	|*	|*	|i	|y	|*	|i	|ó	|z	|*	|*	|.
|*	|e	|*	|d	|*	|*	|*	|[36][S]\rarr	|z	|i	|e	|m	|n	|i	|a	|k	|i	|*	|z	|s	|*	|t	|b	|n	|*	|*	|.
|*	|m	|*	|i	|[37][S]\rarr	|a	|k	|t	|y	|w	|n	|o	|ś	|ć	|*	|*	|*	|*	|m	|t	|*	|e	|*	|e	|*	|*	|.
|*	|*	|*	|*	|[38][S]\rarr	|l	|u	|p	|k	|a	|*	|*	|*	|*	|*	|*	|*	|*	|*	|y	|*	|r	|*	|*	|*	|*	|.
|*	|[39][S]\rarr	|l	|e	|ś	|n	|i	|k	|*	|[40][S]\rarr	|n	|i	|k	|l	|o	|w	|i	|e	|c	|*	|*	|*	|*	|*	|*	|*	|.\end{Puzzle}

\newpage

\begin{PuzzleClues}{\textbf{Poziome}\\}\Clue{1}{}{człowiek spod znaku Skorpiona}
\Clue{2}{}{jednorazowe pozowanie malarzowi, rzeźbiarzowi}
\Clue{3}{}{narta do chodzenia po śniegu}
\Clue{4}{}{człowiek chory na hemofilię}
\Clue{5}{}{włoska religijna pieśń dziękczynna, popularna w średniowieczu}
\Clue{7}{}{Velocisaurus - rodzaj małego teropoda z rodziny noazaurów; żył w okresie późnej kredy na terenach Ameryki Południowej}
\Clue{8}{}{Juncus bulbosus - gatunek roślin zielnych należący do rodziny sitowatych}
\Clue{11}{}{wiatr wiejący z kierunków pomiędzy bokiem a dziobem jednostki żeglującej}
\Clue{12}{}{mieszkanka Zgierza}
\Clue{13}{}{rząd Mesosauria, rodzina Mesosauridae - grupa prymitywnych gadów z podgromady Anapsida, które powróciły do życia w wodzie; ich szczątki znajdowano w osadach z wczesnego permu na półkuli południowej na obszarze superkontynentu Gondwany - w południowej Brazylii, południowej i zachodniej Afryce i na Antarktydzie}
\Clue{14}{}{upłaz zdrobniale}
\Clue{15}{}{wicionogi: podgromada skorupiaków, drobne, osiadłe, głównie obojnaki}
\Clue{17}{}{coś, czego kształt przypomina koło lub jest konceptualizowane jako kolisty obiekt (określenie szczególnie często stosowane, by opisać trajektorię ruchu)}
\Clue{18}{}{żaby wąskopyskie, Microhylidae - rodzina płazów bezogonowych, prowadzących lądowy tryb życia, występujących głównie w rejonach tropikalnych}
\Clue{21}{}{bylina wysokich gór Europy, hodowana w ogródkach skalnych}
\Clue{28}{}{największa planetoida odkryta przez Piazziego}
\Clue{30}{}{zarzucanie czegoś, formułowanie oskarżeń}
\Clue{32}{}{znak muzyczny przedłużający dźwięk o połowę wartości nuty}
\Clue{33}{}{rodzaj kolczyka, który wymaga dość szerokiej dziurki w ciele (zwykle w uchu) - wkłada się go do takiej dziurki, dzięki czemu skóra nie jest wiotka i uzyskuje się efekt interesująco wyglądającej, odpowiednio zaprezentowanej dziury na przestrzał w człowieku}
\Clue{34}{}{zdrobniale o pawilonie, wolno stojącym budynku o lekkiej konstrukcji}
\Clue{35}{}{urządzenie pirotechniczne stosowane do sygnalizacji i oświetlenia}
\Clue{36}{}{potrawa z ziemniaków, zazwyczaj rodzaj garnirunku}
\Clue{37}{}{cecha człowieka, który coś robi, jest czynny, przejawia zapał, inicjatywę, dużo robi, a jego działanie odznacza się intensywnością i jest zauważalne}
\Clue{38}{}{pole wyszukiwania na stronie internetowej}
\Clue{39}{}{leśniczy - pracownik Służby Leśnej w nadleśnictwie, sprawujący opiekę nad powierzonym mu leśnictwem}
\Clue{40}{}{pierwiastek z dziesiątej grupy układu okresowego pierwiastków}\end{PuzzleClues}

\begin{PuzzleClues}{\textbf{Pionowe}\\}\Clue{1}{}{Aeshna - rodzaj stosunkowo dużych ważek z rodziny żagnicowatych; opisano ponad 100 gatunków występujących w obydwu Amerykach, Eurazji i Afryce}
\Clue{2}{}{brodawkowiec czysty, Pseudoscleropodium purum - gatunek mchu z rodziny krótkoszowatych; roślina objęta w Polsce częściową ochroną}
\Clue{6}{}{dodatkowa pensja wypłacana pracownikom etatowym sfery budżetowej}
\Clue{9}{}{Erythronium taylorii - gatunek roślin z rodziny liliowatych}
\Clue{10}{}{stawianie maksymalnych wymagań i żądań}
\Clue{12}{}{sutener, alfons, człowiek czerpiący korzyści majątkowe z uprawiania prostytucji przez inną osobę}
\Clue{16}{}{część wojska, której bronią są różnego rodzaju statki powietrzne}
\Clue{17}{}{dawny żołnierz węgierski}
\Clue{18}{}{muzykolog i kompozytor (1870-1942); opery i pieśni, poematy symfoniczne; 'Zygmunt August'}
\Clue{19}{}{zakonnik będący członkiem zakonnego zgromadzenia orionistów}
\Clue{20}{}{gruby koc służący jako okrycie dla konia}
\Clue{21}{}{przen. wypowiedź lub cała komunikacja, która nie wnosi nic nowego, mowa trawa}
\Clue{22}{}{nowicjusz, osoba, która jeszcze nie ma rozeznanej tematyki, nie zna się na danej rzeczy}
\Clue{23}{}{rodzaj mebla, przy którym się pisze, siedząc, służący również do przechowywania korespondencji i podręcznych drobiazgów}
\Clue{24}{}{zestaw do gry w trylmę}
\Clue{25}{}{urządzenie będące źródłem ciepła potrzebnego do przyrządzania jedzenia, współczesna zazwyczaj składa się z palników (gazowych lub elektrycznych) i piekarnika, dawna zmiast palników miała tzw.blachę}
\Clue{26}{}{wygodny tryb życia, charakteryzujący się upodobaniem przyjemności i unikaniem przykrości}
\Clue{27}{}{lekka, postna zupa lub zupa, która nie ma swoistej nazwy i receptury}
\Clue{29}{}{żartobliwie: szampan}
\Clue{31}{}{elfy (wszystkich szczepów) w literackim legendarium wykreowanym przez J. R. R. Tolkiena}
\Clue{33}{}{klasztor zakonników żyjących w odosobnieniu, prowadzących pustelnicze życie}\end{PuzzleClues}\newpage\section*{Krzyżówka 105}

\noindent\begin{Puzzle}{23}{26}|*	|*	|[1][S]\drarr	|z	|a	|b	|y	|t	|e	|k	|[][,]{ }	|n	|i	|e	|r	|u	|c	|h	|o	|m	|y	|*	|*	|*	|.
|[2][S]\rarr	|a	|b	|l	|e	|n	|a	|*	|*	|*	|*	|[3][S]\darr	|*	|*	|*	|*	|*	|*	|*	|*	|*	|[4][S]\darr	|*	|*	|.
|*	|[5][S]\darr	|a	|*	|[6][S]\rarr	|e	|l	|e	|u	|s	|i	|s	|*	|[7][S]\rarr	|t	|a	|n	|k	|[][,]{ }	|t	|o	|p	|*	|[8][S]\darr	|.
|[9][S]\rarr	|d	|r	|e	|v	|e	|r	|*	|[10][S]\rarr	|k	|a	|p	|s	|a	|*	|*	|[11][S]\darr	|*	|*	|[12][S]\darr	|*	|o	|*	|g	|.
|*	|u	|t	|[13][S]\drarr	|t	|u	|r	|b	|o	|o	|d	|r	|z	|u	|t	|o	|w	|i	|e	|c	|*	|k	|[14][S]\darr	|o	|.
|[15][S]\drarr	|c	|h	|w	|y	|t	|a	|k	|*	|*	|*	|a	|*	|*	|[16][S]\darr	|*	|a	|[17][S]\darr	|*	|a	|*	|r	|r	|c	|.
|m	|i	|*	|i	|[18][S]\rarr	|ż	|e	|l	|a	|z	|o	|w	|i	|e	|c	|*	|r	|s	|*	|r	|*	|y	|a	|k	|.
|o	|ć	|[19][S]\darr	|e	|*	|*	|[20][S]\rarr	|g	|i	|r	|l	|a	|n	|d	|a	|*	|g	|u	|*	|p	|*	|w	|u	|i	|.
|n	|*	|c	|r	|*	|*	|*	|*	|*	|*	|*	|*	|[21][S]\darr	|*	|s	|*	|a	|z	|[22][S]\drarr	|i	|m	|a	|m	|*	|.
|o	|*	|e	|s	|[23][S]\rarr	|w	|i	|e	|r	|z	|c	|h	|o	|ł	|e	|k	|*	|u	|k	|n	|[24][S]\darr	|*	|a	|[25][S]\darr	|.
|g	|*	|l	|z	|[26][S]\drarr	|c	|i	|e	|n	|k	|i	|[][,]{ }	|b	|o	|l	|e	|k	|*	|w	|e	|b	|*	|*	|w	|.
|a	|*	|e	|*	|s	|*	|*	|*	|[27][S]\rarr	|a	|n	|g	|e	|o	|l	|o	|g	|i	|a	|*	|e	|*	|[28][S]\darr	|e	|.
|m	|*	|s	|[29][S]\rarr	|k	|o	|n	|d	|o	|t	|i	|e	|r	|*	|a	|*	|*	|*	|s	|*	|n	|*	|b	|r	|.
|i	|[30][S]\drarr	|t	|r	|a	|k	|t	|o	|r	|e	|k	|*	|t	|[31][S]\darr	|*	|*	|*	|*	|*	|*	|z	|*	|i	|a	|.
|c	|p	|a	|[32][S]\rarr	|l	|w	|ó	|w	|*	|[33][S]\darr	|[34][S]\rarr	|l	|a	|k	|e	|w	|o	|o	|d	|*	|o	|*	|e	|n	|.
|z	|a	|*	|*	|a	|*	|*	|*	|*	|w	|[35][S]\drarr	|s	|s	|a	|k	|*	|*	|*	|*	|*	|p	|*	|g	|d	|.
|n	|r	|*	|*	|*	|[36][S]\rarr	|d	|z	|i	|u	|r	|a	|*	|w	|*	|*	|*	|*	|*	|*	|i	|[37][S]\darr	|[][,]{ }	|a	|.
|o	|e	|*	|[38][S]\drarr	|j	|a	|s	|z	|c	|z	|u	|r	|k	|a	|[][,]{ }	|l	|i	|l	|f	|o	|r	|d	|a	|*	|.
|ś	|n	|*	|r	|[39][S]\rarr	|m	|a	|g	|n	|e	|s	|i	|k	|*	|*	|*	|*	|*	|*	|*	|e	|y	|l	|*	|.
|ć	|e	|*	|a	|[40][S]\rarr	|s	|t	|r	|a	|t	|y	|f	|i	|k	|a	|c	|j	|a	|*	|*	|n	|s	|p	|*	|.
|*	|t	|[41][S]\drarr	|d	|y	|s	|f	|u	|n	|k	|c	|j	|o	|n	|a	|l	|n	|o	|ś	|ć	|*	|z	|e	|*	|.
|*	|y	|p	|ż	|*	|*	|*	|*	|*	|a	|y	|*	|*	|*	|*	|*	|*	|*	|*	|*	|*	|e	|j	|*	|.
|*	|k	|u	|a	|*	|[42][S]\rarr	|b	|o	|k	|*	|s	|*	|*	|*	|*	|*	|*	|*	|*	|*	|*	|l	|s	|*	|.
|*	|a	|a	|*	|*	|*	|[43][S]\rarr	|k	|r	|a	|t	|k	|a	|[][,]{ }	|v	|i	|c	|h	|y	|*	|*	|*	|k	|*	|.
|*	|*	|z	|*	|*	|*	|[44][S]\rarr	|s	|s	|a	|k	|[][,]{ }	|w	|y	|m	|a	|r	|ł	|y	|*	|*	|*	|i	|*	|.
|*	|*	|*	|*	|*	|*	|*	|[45][S]\rarr	|b	|i	|a	|ł	|y	|[][,]{ }	|t	|y	|d	|z	|i	|e	|ń	|*	|*	|*	|.
|*	|*	|*	|[46][S]\rarr	|k	|o	|r	|d	|e	|l	|*	|*	|*	|*	|*	|*	|*	|*	|*	|*	|*	|*	|*	|*	|.\end{Puzzle}

\newpage

\begin{PuzzleClues}{\textbf{Poziome}\\}\Clue{1}{}{nieruchomość lub część nieruchomości, która jest świadectwem istotnych zdarzeń albo rozwoju gospodarki, tradycji i kultury}
\Clue{2}{}{Ablennes hians - ryba morska z rodziny belonowatych, jedyny przedstawiciel rodzaju Ablennes}
\Clue{6}{}{stowarzyszenie o zabarwieniu religijno-filozoficznym (nawiązujące w nazwie do greckiego miasta Eleusis, słynnego z misteriów kultu Demeter i Persefony), założone przez Wincentego Lutosławskiego oraz Joachima Sołtysa w 1902 roku}
\Clue{7}{}{damski top o kroju bokserki}
\Clue{9}{}{rasa psów z grupy psów gończych i posokowców, zaklasyfikowana do małych psów gończych}
\Clue{10}{}{kabza - sakiewka, woreczek na pieniądze}
\Clue{13}{}{samolot z silnikiem turboodrzutowym}
\Clue{15}{}{element maszyn, który zwykle jest zawieszony na linie nośnej i służy do chwytania i przenoszenia różnych substancji i materiałów}
\Clue{18}{}{pierwiastek należący do ósmej grupy układu okresowego pierwiastków}
\Clue{20}{}{wieniec, poroże jelenia w gwarze łowieckiej}
\Clue{22}{}{przełożony meczetu, kierownik wspólnych modlitw w meczecie}
\Clue{23}{}{najwyższy, najdalszy punkt obiektu}
\Clue{26}{}{człowiek słaby w tym, co robi, nieodnoszący żadnych sukcesów w danej dziedzinie}
\Clue{27}{}{w teologii chrześcijańskiej część dogmatyki zajmująca się bytami duchowymi jakimi są anioły}
\Clue{29}{}{żołnierz oddziałów najemnych}
\Clue{30}{}{mechaniczny pojazd czterokołowy, najczęściej mający zastosowanie jako ciągnik rolniczy}
\Clue{32}{}{miasto na Ukrainie, w zaborze austriackim stolica Galicji}
\Clue{34}{}{miasto w USA (Ohio), w zespole miejskim Cleveland; przemysł maszynowy, chemiczny, elektrotechniczny}
\Clue{35}{}{urządzenie na przenośniku pneumatycznym służące do wytwarzania mieszanki nosiwo - powietrze}
\Clue{36}{}{przenośnie: pewien brak, luka w czymś, co nie jest przedmiotem materialnym}
\Clue{38}{}{Podarcis lilfordi - gatunek gada z rodziny jaszczurkowatych, występujący na Balearach}
\Clue{39}{}{gadżet, który składa się z ozdoby i przytwierdzonego do niej magnesu, służący do przymocowywania (także do przymocowywania czegoś) do metalowych powierzchni}
\Clue{40}{}{powstawanie poziomów i warstw czegoś, podział na warstwy}
\Clue{41}{}{to, że coś nie spełnia swojej funkcji, kuleje, jest niesprawne}
\Clue{42}{}{odcinek łączący dwa leżące obok siebie wierzchołki wielokąta}
\Clue{43}{}{prosta kratka, którą tworzą przecinające się pasy koloru i bieli}
\Clue{44}{}{gatunek lub podgatunek ssaka, który wyginął wskutek procesów naturalnych lub spowodowanych przez człowieka}
\Clue{45}{}{w tradycji katolickiej okres oktawy komunijnej; osoba po komunii każdego dnia uczestniczy w uroczystej liturgii}
\Clue{46}{}{sznurek z papieru izolacyjnego służący do owijania żył kabli}\end{PuzzleClues}

\begin{PuzzleClues}{\textbf{Pionowe}\\}\Clue{1}{}{niemiecki socjolog i filozof (1858-1922); zwolennik neoheglizmu}
\Clue{3}{}{coś do załatwienia, do zrobienia; interes, który ktoś potrzebuje - chce lub musi - załatwić}
\Clue{4}{}{odejmowana lub odchylana część obudowy maszyny}
\Clue{5}{}{(1871-1943), poeta serbski, liryka miłosna i patriotyczna, aforyzmy, eseje}
\Clue{8}{}{wymarły język wschodniogermański, który był używany przez germańskie plemię Gotów}
\Clue{11}{}{część korony kwiatu}
\Clue{12}{}{włoski podróżnik, franciszkanin (1182-1252); poseł papieski do chana mongolskiego, pozostawił cenny opis podróży}
\Clue{13}{}{utwór poetycki}
\Clue{14}{}{miasto i port Finlandii nad Zatoką Botnicką}
\Clue{15}{}{cecha człowieka, który uznaje pozostawanie w relacji seksualnej tylko z jednym partnerem w tym samym czasie}
\Clue{16}{}{włoski pianista, kompozytor i dyrygent (1883-1947); propagator muzyki współczesnej}
\Clue{17}{}{delfin gangesowy, delfin indyjski, susuka, Platanista gangetica - gatunek walenia z rodziny delfinów słodkowodnych żyjący w wodach Gangesu i Indusu}
\Clue{19}{}{instrument muzyczny z grupy idiofonów klawiszowych, wynaleziony pod koniec XIX w., przypominający kształtem fisharmonię; dźwięki wytwarzane są poprzez uderzania uruchamianych przy pomocy klawiszy młoteczków w połączone z rezonatorem skrzynkowym metalowe płytki, które wydają dźwięk zbliżony do dzwonków}
\Clue{21}{}{polski taniec ludowy, o żywym tempie i skocznej melodii w rytmie nieparzystym; popularny na wsi w wielu regionach Polski, szczególnie na Mazowszu i Radomszczyźnie}
\Clue{22}{}{orzeźwiający napój (zazwyczaj bezalkoholowy) uzyskiwany w wyniku fermentacji, kojarzony z kuchniami Europy Wschodniej}
\Clue{24}{}{organiczny związek chemiczny, wielopierścieniowy węglowodór aromatyczny o pięciu skondensowanych pierścieniach benzenowych, obecny w smole pogazowej, dymie tytoniowym}
\Clue{25}{}{drewniane lub murowane pomieszczenie otwarte, zwykle o charakterze dobudówki}
\Clue{26}{}{zakres, szereg dźwięków ułożonych według stałego schematu}
\Clue{28}{}{bieg , który odbywa się w terenie górskim i którego trasa prowadzi w większości pod górę}
\Clue{30}{}{utwory piśmiennicze kształtujące i propagujące wzory postępowania związane z odgrywaniem określonych ról społecznych}
\Clue{31}{}{produkt spożywczy służący do sporządzenia kawy-napoju; najczęściej są to spreparowane nasiona kawowca}
\Clue{33}{}{ciastko (porcja ciasta sprzedawana w cukierni) z bitą śmietaną, zwykle na spodzie kakaowym/czekoladowym, czasem z warstewką dżemu i zastygniętą polewą czekoladową}
\Clue{35}{}{studentka, która studiuje na filologii rosyjskiej}
\Clue{37}{}{drąg umocowany do przedniej części wozu, umożliwiający kierowanie nim}
\Clue{38}{}{historyczny tytuł lokalnego władcy w Indiach, który sprawował władzę sądowniczą i był zwierzchnikiem wojsk}
\Clue{41}{}{P - jednostka lepkości dynamicznej w układzie jednostek miar CGS, nazwana na cześć francuskiego fizyka i lekarza Jeana L. M. Poiseuille'a. 1 P = 1 dyn·s/cm2 = 1 g·/(cm·s)}\end{PuzzleClues}\newpage\section*{Krzyżówka 106}

\noindent\begin{Puzzle}{22}{28}|*	|*	|*	|*	|*	|*	|*	|*	|*	|[1][S]\drarr	|s	|a	|h	|a	|j	|d	|a	|c	|z	|n	|y	|*	|*	|.
|*	|*	|[2][S]\rarr	|o	|p	|o	|z	|y	|c	|j	|o	|n	|i	|s	|t	|a	|*	|[3][S]\drarr	|s	|u	|r	|*	|*	|.
|*	|*	|*	|[4][S]\darr	|*	|*	|*	|[5][S]\rarr	|w	|a	|l	|o	|s	|z	|e	|k	|*	|n	|*	|*	|*	|*	|*	|.
|*	|*	|*	|c	|[6][S]\rarr	|c	|z	|o	|s	|n	|e	|k	|[][,]{ }	|p	|u	|r	|p	|u	|r	|o	|w	|y	|*	|.
|*	|[7][S]\drarr	|b	|i	|s	|k	|u	|p	|i	|c	|e	|[][,]{ }	|p	|o	|d	|g	|ó	|r	|n	|e	|*	|*	|[8][S]\darr	|.
|*	|s	|*	|ę	|*	|*	|[9][S]\drarr	|r	|o	|z	|p	|r	|a	|w	|a	|*	|[10][S]\darr	|z	|*	|*	|[11][S]\darr	|[12][S]\darr	|l	|.
|*	|i	|*	|g	|*	|[13][S]\darr	|c	|[14][S]\drarr	|m	|a	|d	|a	|c	|h	|*	|*	|r	|e	|[15][S]\darr	|*	|t	|d	|i	|.
|*	|m	|[16][S]\darr	|i	|[17][S]\drarr	|w	|z	|b	|u	|r	|z	|e	|n	|i	|e	|*	|a	|c	|c	|[18][S]\darr	|o	|o	|s	|.
|*	|l	|s	|*	|k	|i	|a	|r	|[19][S]\darr	|*	|*	|[20][S]\darr	|[21][S]\darr	|*	|*	|*	|c	|[][,]{ }	|e	|g	|r	|l	|t	|.
|*	|a	|ę	|*	|o	|k	|s	|o	|k	|[22][S]\darr	|*	|z	|z	|[23][S]\darr	|[24][S]\darr	|[25][S]\darr	|h	|c	|b	|i	|f	|i	|e	|.
|*	|*	|d	|*	|r	|i	|*	|d	|u	|b	|*	|a	|l	|t	|i	|c	|u	|z	|u	|b	|o	|n	|r	|.
|*	|[26][S]\darr	|z	|[27][S]\drarr	|o	|p	|ł	|a	|t	|a	|[][,]{ }	|t	|e	|r	|m	|i	|n	|a	|l	|o	|w	|a	|*	|.
|*	|s	|i	|m	|ł	|e	|[28][S]\darr	|w	|y	|s	|*	|y	|w	|a	|p	|a	|e	|r	|i	|n	|i	|[][,]{ }	|*	|.
|*	|p	|a	|i	|a	|d	|s	|c	|k	|k	|[29][S]\darr	|ł	|*	|s	|l	|ł	|k	|n	|c	|[][,]{ }	|e	|z	|*	|.
|*	|i	|[][,]{ }	|ś	|z	|i	|k	|z	|u	|*	|a	|*	|*	|z	|a	|o	|[][,]{ }	|o	|a	|c	|c	|a	|*	|.
|*	|e	|k	|*	|y	|a	|a	|y	|l	|*	|s	|*	|[30][S]\darr	|k	|n	|[][,]{ }	|o	|s	|[][,]{ }	|z	|[][,]{ }	|w	|*	|.
|*	|l	|a	|*	|*	|*	|n	|c	|a	|[31][S]\darr	|t	|[32][S]\darr	|s	|a	|t	|l	|p	|k	|t	|a	|g	|i	|*	|.
|[33][S]\drarr	|b	|l	|o	|k	|a	|d	|a	|*	|t	|r	|w	|h	|[][,]{ }	|a	|o	|e	|r	|r	|r	|i	|e	|*	|.
|j	|e	|o	|[34][S]\darr	|*	|*	|a	|[][,]{ }	|[35][S]\darr	|e	|o	|i	|o	|w	|c	|t	|r	|z	|ó	|n	|r	|s	|*	|.
|o	|r	|s	|p	|*	|*	|l	|k	|g	|l	|c	|o	|l	|ł	|j	|n	|a	|y	|j	|o	|g	|z	|*	|.
|r	|g	|z	|a	|*	|*	|*	|o	|a	|i	|h	|ś	|e	|o	|a	|e	|t	|d	|l	|r	|e	|o	|*	|.
|*	|*	|*	|s	|*	|*	|*	|n	|l	|g	|e	|l	|s	|s	|[][,]{ }	|*	|o	|ł	|i	|ę	|n	|n	|*	|.
|*	|*	|*	|i	|*	|*	|[36][S]\rarr	|i	|l	|a	|m	|a	|*	|k	|j	|*	|r	|y	|s	|k	|s	|a	|*	|.
|[37][S]\drarr	|q	|u	|e	|n	|d	|i	|*	|e	|*	|i	|k	|*	|a	|o	|*	|o	|*	|t	|i	|o	|*	|*	|.
|m	|*	|*	|c	|*	|[38][S]\rarr	|b	|a	|n	|k	|a	|*	|*	|*	|n	|*	|w	|*	|n	|*	|h	|*	|*	|.
|ł	|[39][S]\rarr	|s	|z	|e	|r	|e	|g	|*	|*	|*	|*	|*	|*	|ó	|*	|y	|*	|a	|*	|n	|*	|*	|.
|o	|*	|[40][S]\rarr	|k	|s	|e	|n	|o	|f	|i	|l	|*	|*	|*	|w	|*	|*	|*	|*	|*	|a	|*	|*	|.
|t	|[41][S]\rarr	|g	|a	|b	|i	|n	|e	|c	|i	|k	|*	|*	|*	|*	|*	|*	|[42][S]\rarr	|s	|b	|*	|*	|*	|.
|*	|*	|*	|*	|*	|*	|*	|*	|*	|*	|*	|*	|*	|*	|*	|*	|*	|*	|*	|*	|*	|*	|*	|.\end{Puzzle}

\newpage

\begin{PuzzleClues}{\textbf{Poziome}\\}\Clue{1}{}{u Kozaków: godność odpowiadająca stanowisku hetmana}
\Clue{2}{}{członek partii politycznej opozycyjnej w stosunku do rządu}
\Clue{3}{}{kod ISO 4217 rubla radzieckiego}
\Clue{5}{}{muzyka, do której tańczy się waloszka}
\Clue{6}{}{Allium atropurpureum - gatunek rośliny z rodziny amarylkowatych}
\Clue{7}{}{wieś w Polsce położona w województwie dolnośląskim, w powiecie wrocławskim, w gminie Kobierzyce}
\Clue{9}{}{porządek zasadniczych czynności podejmowanych na sesji sądowej w celu rozpoznania sprawy, to jest orzeczenia o prawach i obowiązkach zainteresowanych}
\Clue{14}{}{(1823-64), pisarz węgierski, poematy, dramaty, komedie, opowiadania, liryki; „Tragedia człowieka”}
\Clue{17}{}{stan silnych emocji}
\Clue{27}{}{w lotnictwie - opłata za nawigację terminalową}
\Clue{33}{}{utrudnienie przejścia lub dostępu do czegoś}
\Clue{36}{}{jadalny, smaczny owoc (wielopestkowiec) flaszowca różnolistnego}
\Clue{37}{}{elfy (wszystkich szczepów) w literackim legendarium wykreowanym przez J. R. R. Tolkiena}
\Clue{38}{}{sosna Banksa, Pinus banksiana - gatunek drzewa iglastego z rodziny sosnowatych}
\Clue{39}{}{układ, grupa jednostek uporządkowanych według jakiejś zasady tworząca system}
\Clue{40}{}{osoba, która preferuje kontakty seksualne z nieznajomymi, bez chęci nawiązania kontaktu emocjonalnego z nimi}
\Clue{41}{}{zdrobniale lub żartobliwie o gabinecie jakiejś ważnej osobistości, np. prezesa, doktora, profesora}
\Clue{42}{}{w chemii: symbol antymonu}\end{PuzzleClues}

\begin{PuzzleClues}{\textbf{Pionowe}\\}\Clue{1}{}{żołnierz regularnej wyborowej piechoty tureckiej, utworzonej początkowo z młodych brańców chrześcijańskich}
\Clue{3}{}{Pelecanoides urinatrix urinatrix - nominatywny podgatunek perkoza czarnoskrzydłego (Pelecanoides urinatrix); występuje na obszarze wysp niedaleko południowych wybrzeży Australii, Tasmanii i Nowej Zelandii}
\Clue{4}{}{mocne bicie, zwykle chłosta}
\Clue{7}{}{miasto w Indiach, stolica stanu Himacal Prades; znany ośrodek wypoczynkowy}
\Clue{8}{}{chirurg angielski (1827-1912); twórca metody antyseptycznej w chirurgii, wynalazł katgut}
\Clue{9}{}{nieprzerwany ciąg następujących po sobie wydarzeń}
\Clue{10}{}{dział matematyki, który obejmuje metody rozwiązywania niektórych typów równań różniczkowych}
\Clue{11}{}{Sphagnum girgensohnii - gatunek mchu należący do rodziny torfowcowatych}
\Clue{12}{}{dolina boczna oddzielona od doliny głównej stromym progiem}
\Clue{13}{}{wielojęzyczna encyklopedia internetowa działająca w oparciu o zasadę otwartej treści}
\Clue{14}{}{choroba wirusowa koni wywoływana przez różne typy wirusa EcPV (Equus Caballus Papilloma Virus), czyli wirusa brodawczaka końskiego, należącego do grupy papillomawirusów}
\Clue{15}{}{Scilla kladnii - gatunek rośliny z rodziny szparagowatych}
\Clue{16}{}{określenie sędziego, którego decyzje nie podobają się kibicom}
\Clue{17}{}{Climacteridae - rodzina ptaków z rzędu wróblowych (Passeriformes), obejmująca kilka gatunków małych ptaków, występujących wyłącznie w Australii i na Nowej Gwinei}
\Clue{18}{}{gibon ungko, Hylobates agilis - ssak z rodziny gibonowatych, zamieszkujący Sumatrę i Półwysep Malajski}
\Clue{19}{}{OSKÓREK}
\Clue{20}{}{miejsce na tyle, za kimś lub za czymś, np. zatył wojsk, zatył stodoły}
\Clue{21}{}{zlewozmywak - funkcjonalny odpowiednik umywalki (ujęcie wody) zamontowany w kuchni, służący głównie do mycia naczyń oraz produktów spożywczych i usuwania zbędnych płynów, pozostałych z gotowania potraw}
\Clue{22}{}{mieszkaniec Baskonii - regionu i krainy historycznej w Hiszpanii, człowiek pochodzenia baskijskiego}
\Clue{23}{}{Lissotriton italicus - gatunek płaza ogoniastego z rodziny salamandrowatych, występujący w środkowych i południowych Włoszech}
\Clue{24}{}{domieszkowanie materiałów polegające na rozpędzeniu jonów w polu elektrycznym i zderzeniu z domieszkowanym materiałem}
\Clue{25}{}{substancja mająca postać gazową}
\Clue{26}{}{Steven Spielberg - amerykański reżyser, scenarzysta i producent filmowy}
\Clue{27}{}{pieczątka w paszporcie, którą otrzymuje się za jakieś przewinienie i która może w przyszłości sprawić, że jej posiadaczowi trudniej będzie ponownie przekroczyć granicę; nazwa nawiązuje do kształtu pieczątki, którą przybijali niemieccy pogranicznicy}
\Clue{28}{}{afera, głośna, poruszająca sprawa}
\Clue{29}{}{zajmuje się badaniem składu chemicznego ciał niebieskich}
\Clue{30}{}{amerykański dziennikarz i wydawca (1819-90); uzyskał patent na pierwsza użytkową maszynę do pisania}
\Clue{31}{}{żeglarz, pierwszy Polak, który na jachcie 'Opty' w latach 1967-69 opłynął ziemię}
\Clue{32}{}{słodkowodny pluskwiak różnoskrzydły, roślinożerny lub drapieżny}
\Clue{33}{}{prasłowiańska półsamogłoska tworząca sylabę}
\Clue{34}{}{zdrobniale o pasiece}
\Clue{35}{}{Kallela; fiński malarz i grafik (1865-1931 ); cykl malowideł opartych na eposie 'Kalewala'}
\Clue{37}{}{głowomłot pospolity, Sphyrna zygaena - gatunek morskiej ryby żarłaczokształtnej z rodziny młotowatych (Sphyrnidae), której przednia część głowy przypomina młot}\end{PuzzleClues}\newpage\section*{Krzyżówka 107}

\noindent\begin{Puzzle}{22}{32}|*	|*	|*	|[1][S]\drarr	|p	|s	|z	|c	|z	|o	|ł	|a	|[][,]{ }	|i	|b	|e	|r	|y	|j	|s	|k	|a	|*	|.
|*	|*	|*	|ł	|*	|[2][S]\darr	|*	|*	|*	|*	|*	|*	|*	|*	|*	|*	|[3][S]\darr	|*	|*	|*	|[4][S]\darr	|*	|[5][S]\darr	|.
|[6][S]\drarr	|g	|r	|o	|ź	|b	|a	|*	|[7][S]\drarr	|s	|k	|a	|r	|t	|a	|b	|e	|l	|l	|u	|s	|*	|w	|.
|g	|*	|*	|n	|*	|r	|*	|*	|c	|*	|[8][S]\rarr	|b	|a	|b	|i	|n	|i	|e	|c	|*	|t	|[9][S]\darr	|y	|.
|ó	|[10][S]\drarr	|r	|o	|w	|e	|r	|*	|o	|*	|*	|*	|*	|*	|*	|*	|n	|*	|[11][S]\darr	|[12][S]\darr	|r	|d	|b	|.
|r	|t	|*	|*	|[13][S]\darr	|a	|[14][S]\drarr	|e	|l	|e	|k	|t	|r	|o	|l	|i	|t	|*	|w	|k	|e	|r	|r	|.
|a	|r	|*	|*	|r	|k	|g	|*	|u	|*	|*	|[15][S]\darr	|[16][S]\darr	|[17][S]\darr	|[18][S]\darr	|[19][S]\darr	|o	|*	|o	|u	|f	|o	|z	|.
|*	|a	|[20][S]\darr	|*	|a	|b	|ł	|*	|m	|*	|*	|c	|s	|w	|k	|o	|p	|*	|l	|r	|a	|m	|e	|.
|*	|n	|w	|[21][S]\darr	|m	|e	|a	|*	|b	|*	|*	|y	|z	|ę	|a	|t	|f	|*	|b	|t	|[][,]{ }	|o	|ż	|.
|*	|s	|i	|a	|i	|a	|d	|[22][S]\darr	|i	|*	|[23][S]\darr	|b	|y	|z	|s	|t	|*	|*	|ó	|y	|b	|s	|e	|.
|[24][S]\drarr	|g	|e	|n	|e	|t	|y	|k	|a	|[][,]{ }	|m	|o	|l	|e	|k	|u	|l	|a	|r	|n	|a	|*	|[][,]{ }	|.
|e	|r	|w	|i	|n	|*	|s	|r	|*	|*	|e	|r	|i	|ł	|*	|*	|*	|*	|z	|a	|t	|*	|ś	|.
|k	|e	|i	|o	|i	|*	|z	|ą	|[25][S]\darr	|[26][S]\darr	|n	|i	|n	|*	|*	|*	|[27][S]\drarr	|c	|*	|[][,]{ }	|i	|*	|r	|.
|r	|s	|ó	|ł	|c	|[28][S]\darr	|[][,]{ }	|g	|s	|s	|a	|u	|g	|*	|[29][S]\drarr	|g	|r	|u	|s	|z	|a	|*	|o	|.
|a	|j	|r	|e	|a	|m	|k	|ł	|t	|i	|c	|m	|*	|[30][S]\rarr	|p	|i	|e	|ń	|*	|e	|l	|*	|d	|.
|n	|a	|k	|k	|[][,]{ }	|u	|r	|o	|o	|l	|h	|*	|[31][S]\rarr	|m	|o	|d	|e	|l	|*	|r	|n	|[32][S]\darr	|k	|.
|o	|[][,]{ }	|a	|[][,]{ }	|g	|l	|u	|l	|c	|o	|a	|*	|*	|*	|l	|*	|d	|*	|[33][S]\darr	|o	|a	|g	|o	|.
|p	|l	|[][,]{ }	|c	|r	|t	|c	|i	|z	|s	|*	|*	|*	|*	|e	|*	|u	|*	|n	|*	|*	|ę	|w	|.
|l	|o	|r	|h	|z	|i	|h	|s	|e	|[][,]{ }	|*	|*	|*	|*	|r	|*	|k	|[34][S]\darr	|i	|*	|*	|s	|e	|.
|a	|d	|a	|a	|y	|l	|y	|t	|k	|b	|[35][S]\darr	|[36][S]\darr	|*	|*	|o	|[37][S]\rarr	|a	|p	|e	|t	|y	|t	|*	|.
|n	|o	|f	|r	|w	|a	|*	|*	|[][,]{ }	|u	|ż	|o	|[38][S]\darr	|[39][S]\darr	|w	|*	|t	|r	|b	|*	|*	|o	|*	|.
|*	|w	|i	|l	|i	|t	|*	|*	|ł	|d	|e	|p	|d	|k	|n	|[40][S]\rarr	|o	|y	|o	|*	|*	|ś	|*	|.
|*	|c	|o	|i	|a	|e	|[41][S]\darr	|*	|u	|o	|l	|i	|w	|o	|i	|[42][S]\darr	|r	|m	|*	|[43][S]\darr	|*	|ć	|*	|.
|*	|a	|w	|e	|s	|r	|b	|[44][S]\drarr	|k	|w	|a	|s	|ó	|w	|k	|a	|*	|k	|[45][S]\darr	|d	|*	|*	|*	|.
|*	|*	|a	|g	|t	|a	|r	|p	|o	|l	|z	|t	|j	|a	|*	|r	|[46][S]\darr	|a	|r	|o	|[47][S]\darr	|[48][S]\darr	|[49][S]\darr	|.
|*	|*	|*	|o	|a	|l	|a	|a	|w	|a	|o	|o	|k	|r	|[50][S]\darr	|n	|a	|*	|z	|b	|a	|t	|t	|.
|*	|*	|[51][S]\darr	|*	|*	|i	|h	|t	|s	|n	|b	|r	|a	|i	|k	|o	|n	|*	|ą	|i	|n	|a	|t	|.
|*	|*	|t	|*	|*	|z	|m	|y	|k	|y	|e	|c	|*	|*	|a	|t	|i	|*	|d	|t	|t	|u	|b	|.
|*	|[52][S]\rarr	|a	|d	|a	|m	|s	|k	|i	|*	|t	|h	|*	|[53][S]\rarr	|m	|a	|l	|i	|n	|n	|i	|k	|*	|.
|*	|*	|r	|*	|*	|*	|*	|*	|*	|*	|o	|o	|*	|*	|r	|*	|a	|*	|o	|o	|b	|a	|*	|.
|*	|*	|*	|*	|*	|*	|*	|*	|*	|*	|n	|z	|*	|*	|a	|*	|n	|*	|ś	|ś	|e	|*	|*	|.
|*	|*	|*	|*	|*	|[54][S]\rarr	|s	|ł	|o	|ń	|*	|a	|*	|*	|t	|*	|a	|*	|ć	|ć	|s	|*	|*	|.
|*	|*	|[55][S]\rarr	|l	|i	|s	|i	|c	|z	|k	|a	|*	|*	|*	|*	|*	|*	|*	|*	|*	|*	|*	|*	|.\end{Puzzle}

\newpage

\begin{PuzzleClues}{\textbf{Poziome}\\}\Clue{1}{}{Apis mellifera iberica - podgatunek pszczoły miodnej pochodzący z Półwyspu Iberyjskiego}
\Clue{6}{}{zastraszenie, grożenie}
\Clue{7}{}{członek szlachty niższej, skartabellatu}
\Clue{8}{}{żartobliwie o zbiorowisku kobiet}
\Clue{10}{}{pojazd zwykle jednośladowy napędzany siłą mięśni}
\Clue{14}{}{substancja, która w stanie rozpuszczonym rozpada się na jony lub która pozostając w stanie ciekłym, składa się z jonów i w każdej z wymienionych postaci jest zdolna do przewodzenia prądu elektrycznego}
\Clue{24}{}{dział biologii zajmujący się genetyką na poziomie biologii molekularnej}
\Clue{27}{}{symbol stałej fizycznej prędkości światła w próżni}
\Clue{29}{}{Pyrus - rodzaj w większości niewielkich drzew z rodziny różowatych (Rosaceae), uprawianych ze względu na ich słodkie owoce o kulistym zwężającym się ku górze kształcie}
\Clue{30}{}{pień mózgu, struktura anatomiczna ośrodkowego układu nerwowego obejmująca wszystkie twory leżące na podstawie czaszki}
\Clue{31}{}{wzór określający sposób wykonania czegoś}
\Clue{37}{}{pożądanie; pociąg do kogoś}
\Clue{40}{}{miasto w płn.-zach Nigerii i ośrodek handlowy}
\Clue{44}{}{stal odporna na działanie kwasów o mniejszej mocy od kwasu siarkowego}
\Clue{52}{}{pięściarz, wicemistrz olimpijski z Rzymu w kategorii piórkowej, mistrz Europy z 1959}
\Clue{53}{}{miód pitny na malinach, w którym co najmniej 30\% objętości wody (względem oryginalnego przepisu na miód niesmakowy) zastąpiono sokiem z malin}
\Clue{54}{}{niezgrabnie się poruszający grubas}
\Clue{55}{}{zdrobniale: lisica - samica lisa}\end{PuzzleClues}

\begin{PuzzleClues}{\textbf{Pionowe}\\}\Clue{1}{}{wnętrze brzucha kobiety, tam, gdzie rozwija się lub może się rozwijać dziecko}
\Clue{2}{}{gatunek elektronicznej muzyki tanecznej}
\Clue{3}{}{potrawa jednogarnkowa, popularny posiłek przygotowany w jednym naczyniu, zastępujący cały obiad}
\Clue{4}{}{podwodna strefa zbiorników wodnych obejmująca zbocza lądu do górnej granicy strefy abysalnej}
\Clue{5}{}{część polskiego wybrzeża w obrębie Pomorza Zachodniego, rozciągająca się między Trójmiastem a zachodnią granicą Polski}
\Clue{6}{}{sterta, wielka ilość, stos}
\Clue{7}{}{miasto w USA, stolica stanu Karolina Płd}
\Clue{9}{}{krótki bieg - starożytna konkurencja olimpijska}
\Clue{10}{}{przesuwanie się czoła lodowca ku przodowi}
\Clue{11}{}{miasto w centralnej Polsce, położone w województwie łódzkim, w powiecie piotrkowskim, siedziba gminy Wolbórz}
\Clue{12}{}{strefa pomiędzy warstwą czynną a wieczną zmarzliną, w której stale utrzymuje się temperatura bliska zeru}
\Clue{13}{}{Chara filiformis - słodkowodny gatunek ramienicy}
\Clue{14}{}{Leocarpus fragilis - gatunek śluzowca}
\Clue{15}{}{ozdobna puszka na komunikanty}
\Clue{16}{}{stosunkowo rozpowszechniona nazwa waluty funkcjonującej w wielu krajach}
\Clue{17}{}{zgrupowanie w jednym miejscu i pionie urządzeń sanitarnych lub przemysłowych przyłączonych do jednego zespołu przewodów}
\Clue{18}{}{HEŁM}
\Clue{19}{}{rodzaj małego hinduskiego oboju}
\Clue{20}{}{Epixerus ebii - gatunek gryzonia z rodziny wiewiórkowatych; występuje w gęstych lasach w pobliżu bagien w Ghanie i Sierra Leone}
\Clue{21}{}{kobieta stylizowana lub stylizująca się na postać z serialuAniołki Charliego}
\Clue{22}{}{Rhizomnium - rodzaj mchów należący do rzędu prątnikowców}
\Clue{23}{}{zawartość menachy, dużej menażki}
\Clue{24}{}{statek poruszający się nisko ponad powierzchnią wody; unoszony siłą aerodynamiczną działającą na jego kadłub}
\Clue{25}{}{miasto w województwie lubelskim, w powiecie łukowskim, nad Świdrem}
\Clue{26}{}{służy przechowywaniu materiałów sypkich na budowie, może być stały albo przenośny}
\Clue{27}{}{rehabilitant, który pomaga przywrócić sprawność fizyczną}
\Clue{28}{}{w ekonomii: system międzynarodowy, w którym rachunki państw są rozliczanie w oparciu o umowy wielostronne, poprzez kompensatę bilansów płatniczych}
\Clue{29}{}{osoba zajmująca się polerowaniem czegoś}
\Clue{32}{}{wielkość fizyczna określana przez iloraz masy do objętości}
\Clue{33}{}{pozorne sklepienie otaczające obserwatora - firmament}
\Clue{34}{}{tytoń do żucia, występujący w postaci laseczek do krojenia}
\Clue{35}{}{ŻELBET}
\Clue{36}{}{choroba pasożytnicza, wywoływana przez przywrę z gatunków Opisthorchis viverrini i Opisthorchis felineus}
\Clue{38}{}{dwie osoby, para}
\Clue{39}{}{Dasyuroides byrnei, Dasycercus byrnei - torbacz z rodziny niełazowatych, jedyny przedstawiciel rodzaju Dasyuroides; występuje w południowo-zachodnim Queensland i północno-wschodniej Australii Południowej, zamieszkując suche obszary ubogie w roślinność}
\Clue{41}{}{kompozytor niemiecki (1833-1897); nawiązywał do muzyki Beethovena; symfonie, koncerty, utwory fortepianowe, kameralne}
\Clue{42}{}{Bixa - rodzaj drzewa egzotycznego z rodziny arnotowatych}
\Clue{43}{}{cecha czegoś, co jest jest bardzo wyraziste, mocne, trafiające w sedno}
\Clue{44}{}{mały podłużny pręt, często drewniany (choć nie zawsze), który najczęściej służy jako element konstrukcyjny czegoś}
\Clue{45}{}{cecha człowieka, który jest praworządny}
\Clue{46}{}{włókno akrylowe, które nazywane bywa sztuczną wełną, bo jest dość podobne z wyglądu i pod względem właściwości do wełny; syntetyczne włókno wełnopodobne z poliakrylonitrylu}
\Clue{47}{}{znane kąpielisko morskie we Francji nad Morzem Śródziemnym}
\Clue{48}{}{krzyż lub krzyżyk w kształcie greckiej litery tau, szczególnie popularny wśród franciszkanów; tauka jest pozbawiona górnej belki pionowej}
\Clue{49}{}{terier typu bull, pies domowy będący krzyżówką buldoga i teriera}
\Clue{50}{}{kolega, druh, towarzysz}
\Clue{51}{}{TING; łowny ssak z rodziny krętorogich, pokrojem zbliżony do kozy}\end{PuzzleClues}\newpage\section*{Krzyżówka 108}

\noindent\begin{Puzzle}{21}{29}|*	|*	|*	|*	|*	|*	|[1][S]\drarr	|z	|n	|a	|k	|[][,]{ }	|g	|r	|a	|n	|i	|c	|z	|n	|y	|*	|.
|*	|*	|*	|*	|*	|[2][S]\rarr	|l	|a	|r	|p	|o	|w	|i	|c	|z	|*	|*	|*	|*	|[3][S]\darr	|*	|*	|.
|*	|*	|[4][S]\rarr	|g	|i	|g	|a	|n	|t	|*	|*	|*	|*	|*	|[5][S]\drarr	|s	|u	|w	|*	|m	|*	|*	|.
|*	|[6][S]\rarr	|a	|g	|l	|o	|m	|e	|r	|a	|c	|j	|a	|*	|t	|[7][S]\rarr	|m	|a	|h	|o	|n	|*	|.
|*	|[8][S]\darr	|*	|*	|*	|*	|p	|*	|[9][S]\drarr	|b	|i	|e	|r	|n	|o	|ś	|ć	|*	|*	|d	|*	|*	|.
|*	|ć	|[10][S]\darr	|*	|*	|[11][S]\darr	|k	|*	|k	|[12][S]\darr	|[13][S]\rarr	|b	|i	|o	|c	|y	|d	|*	|*	|r	|*	|*	|.
|[14][S]\drarr	|w	|c	|i	|n	|k	|a	|*	|o	|m	|[15][S]\darr	|[16][S]\darr	|*	|[17][S]\darr	|j	|*	|*	|*	|*	|z	|*	|*	|.
|m	|i	|r	|*	|[18][S]\darr	|a	|[][,]{ }	|[19][S]\drarr	|m	|a	|r	|k	|i	|z	|a	|*	|*	|*	|*	|e	|*	|*	|.
|a	|k	|c	|[20][S]\darr	|r	|r	|w	|h	|i	|s	|a	|r	|*	|n	|*	|*	|*	|*	|[21][S]\darr	|w	|*	|*	|.
|n	|*	|*	|w	|u	|l	|i	|a	|n	|k	|t	|o	|*	|a	|[22][S]\darr	|[23][S]\darr	|*	|*	|a	|[][,]{ }	|[24][S]\darr	|[25][S]\darr	|.
|g	|[26][S]\drarr	|w	|ę	|g	|i	|e	|l	|*	|a	|e	|l	|[27][S]\darr	|k	|c	|o	|*	|*	|n	|e	|d	|p	|.
|a	|m	|*	|z	|b	|s	|c	|l	|*	|*	|l	|o	|w	|*	|y	|d	|*	|*	|g	|u	|ł	|o	|.
|b	|ó	|*	|e	|y	|t	|z	|*	|*	|*	|*	|w	|y	|[28][S]\darr	|f	|w	|*	|*	|l	|r	|u	|l	|.
|a	|z	|*	|ł	|*	|a	|y	|*	|*	|*	|*	|*	|s	|p	|r	|a	|*	|*	|o	|o	|g	|o	|.
|[][,]{ }	|g	|*	|[][,]{ }	|*	|*	|s	|[29][S]\darr	|[30][S]\drarr	|k	|r	|y	|p	|t	|o	|p	|s	|*	|a	|p	|[][,]{ }	|n	|.
|z	|[][,]{ }	|*	|k	|*	|[31][S]\darr	|t	|f	|ł	|*	|[32][S]\darr	|*	|a	|e	|w	|n	|[33][S]\darr	|[34][S]\darr	|r	|e	|w	|i	|.
|w	|e	|*	|o	|[35][S]\drarr	|b	|a	|l	|a	|s	|t	|*	|*	|r	|y	|i	|ś	|k	|a	|j	|d	|s	|.
|y	|m	|*	|m	|p	|e	|*	|a	|j	|*	|o	|*	|[36][S]\darr	|o	|[][,]{ }	|e	|w	|o	|b	|s	|z	|t	|.
|c	|o	|*	|u	|i	|t	|*	|s	|d	|[37][S]\darr	|k	|[38][S]\darr	|s	|z	|a	|n	|i	|ń	|[][,]{ }	|k	|i	|a	|.
|z	|c	|*	|n	|n	|o	|[39][S]\drarr	|z	|a	|w	|i	|k	|ł	|a	|n	|i	|e	|[][,]{ }	|s	|i	|ę	|*	|.
|a	|j	|*	|i	|g	|n	|h	|a	|c	|i	|j	|r	|u	|u	|i	|e	|t	|p	|a	|[][,]{ }	|c	|*	|.
|j	|o	|*	|k	|w	|[][,]{ }	|i	|*	|t	|ą	|c	|u	|ż	|r	|o	|*	|l	|a	|r	|ż	|z	|*	|.
|n	|n	|*	|a	|i	|g	|e	|*	|w	|z	|z	|s	|b	|o	|ł	|*	|i	|r	|d	|ó	|n	|*	|.
|a	|a	|*	|c	|n	|i	|r	|*	|o	|a	|y	|z	|a	|m	|*	|*	|k	|o	|y	|ł	|o	|*	|.
|*	|l	|*	|y	|[][,]{ }	|p	|o	|*	|*	|n	|k	|y	|*	|o	|*	|*	|o	|w	|ń	|t	|ś	|*	|.
|*	|n	|*	|j	|m	|s	|f	|*	|*	|i	|*	|n	|*	|r	|*	|*	|w	|y	|s	|y	|c	|*	|.
|*	|y	|*	|n	|a	|o	|a	|*	|*	|e	|*	|*	|*	|f	|*	|*	|a	|*	|k	|*	|i	|*	|.
|*	|*	|*	|y	|ł	|w	|n	|*	|*	|*	|*	|*	|*	|y	|*	|*	|t	|*	|i	|*	|*	|*	|.
|*	|*	|*	|*	|y	|y	|t	|*	|*	|*	|*	|*	|*	|*	|*	|*	|e	|*	|*	|*	|*	|*	|.
|*	|*	|*	|*	|*	|*	|*	|*	|*	|*	|*	|*	|*	|*	|*	|*	|*	|*	|*	|*	|*	|*	|.\end{Puzzle}

\newpage

\begin{PuzzleClues}{\textbf{Poziome}\\}\Clue{1}{}{znak wyznaczający granice czegoś, np. państwa, posiadłości, w terenie}
\Clue{2}{}{gracz, uczestnik larpu}
\Clue{4}{}{bardzo wielki przedmiot}
\Clue{5}{}{część cyklu pracy silnika tłokowego}
\Clue{6}{}{teren, na którym zaludnienie lub działalność gospodarcza są wystarczająco skoncentrowane, aby ścieki komunalne były zbierane i przekazywane do oczyszczalni ścieków komunalnych}
\Clue{7}{}{główne miasto hiszpańskiej wyspy. Minorka, ośrodek handlowy i turystyczny}
\Clue{9}{}{obojętność, brak wigoru, inicjatywy, zaangażowania}
\Clue{13}{}{związek syntetyczny lub pochodzenia naturalnego do zwalczania szkodliwych organizmów w rolnictwie, leśnictwie i przechowalnictwie}
\Clue{14}{}{podłączenie ze złączem siodłowym, za pomocą którego przyłącza się najczęściej (choć nie tylko) system ściekowy danej nieruchomości do zewnętrznej sieci kanalizacyjnej}
\Clue{19}{}{żona markiza}
\Clue{26}{}{pierwiastek chemiczny o liczbie atomowej 6, niemetal z bloku p układu okresowego}
\Clue{30}{}{Kryptops - rodzaj prymitywnego teropoda z rodziny abelizaurów; żył w okresie wczesnej kredy na terenach dzisiejszego Nigru}
\Clue{35}{}{w kolejnictwie: podkład ze żwiru, umieszczony pod torami (na podtorzu)}
\Clue{39}{}{znalezienie się w złożonej, nieprzyjemnej sytuacji, z której trudno wybrnąć}\end{PuzzleClues}

\begin{PuzzleClues}{\textbf{Pionowe}\\}\Clue{1}{}{nigdy nie gaszona lampka stawiana przed tabernakulum}
\Clue{3}{}{Larix decidua Mill. f. sulphurea Fig. - forma modrzewia europejskiego}
\Clue{5}{}{Tozzia - rodzaj pasożytniczych roślin z rodziny zarazowatych}
\Clue{8}{}{osoba chytra i przebiegła}
\Clue{9}{}{piec kuchenny}
\Clue{10}{}{kod ISO 4217 colona}
\Clue{11}{}{przedstawiciel konserwatywnego nurtu politycznego powstałego w Hiszpanii w latach 30. XIX w., zmierzającego do odtworzenia mocarstwowej Hiszpanii poprzez reformy konserwatywne, prokatolickie, decentralizacyjne, antyabsolutystyczne, antydemokratyczne i antyliberalne w duchu legitymistycznym}
\Clue{12}{}{odlew twarzy, najczęściej zrobiony z gipsu}
\Clue{14}{}{mangaba rudoczelna, Cercocebus torquatus - gatunek średniej wielkości małpy z rodziny makakowatych, występującej w lasach tropikalnych na zachodzie Afryki: od zachodniej Nigerii, przez Gwineę Równikową i Gabon do południowego Kamerunu}
\Clue{15}{}{miodożer; ssak z rodziny łasicowatych, wszystkożerny, lubi zwłaszcza miód}
\Clue{16}{}{poeta niemiecki ur,1915r, liryka wyrażająca grozę wojny}
\Clue{17}{}{ślad pozostawiony przez coś}
\Clue{18}{}{miasto w Anglii nad rzeką Avon; ośrodek handlowy, znana szkoła dla chłopców}
\Clue{19}{}{HOL}
\Clue{20}{}{miejsce, w którym zbiega się kilka szlaków transportowych}
\Clue{21}{}{rasa konia lekkiego, półkrwi, pochodząca z Sardynii (Włochy)}
\Clue{22}{}{system informatyczny gromadzący dane z biometrycznych systemów identyfikacji}
\Clue{23}{}{zmniejszenie się ilości wapnia w tkankach organizmu (np. w kościach, zębach)}
\Clue{24}{}{zobowiązanie moralne wobec kogoś za jakąś ważną przysługę, którą nieodpłatnie wyświadczył na rzeczdłużnika}
\Clue{25}{}{nauczyciel uczący języka polskiego}
\Clue{26}{}{część psychiki człowieka odpowiadająca za reakcje emocjonalne}
\Clue{27}{}{kawałek lądu ze wszystkich stron otoczony wodą}
\Clue{28}{}{Pterosauromorpha - klad archozauromorfów, prawdopodobnie archozaurów z grupy Ornithodira}
\Clue{29}{}{zawartość flaszy, smukłej butelki}
\Clue{30}{}{podły czyn, nieuczciwość, zachowanie charakterystyczne dla łajdaka}
\Clue{31}{}{beton sporządzany z gipsu, kruszywa i wody}
\Clue{32}{}{mieszkaniec Tokio}
\Clue{33}{}{Myctophidae - rodzina morskich, planktonożernych ryb świetlikokształtnych (Myctophiformes), najliczniejsza i najbardziej zróżnicowana rodzina ryb występujących w otwartych wodach oceanicznych, a jednocześnie najliczniejsza wśród ryb głębinowych}
\Clue{34}{}{pozaukładowa jednostka mocy używana w krajach anglosaskich}
\Clue{35}{}{Eudyptula minor - gatunek średniego, nielotnego ptaka wodnego z rodziny pingwinów (Spheniscidae), występujący u wybrzeży Nowej Zelandii i Australii; jest jedynym przedstawicielem rodzaju Eudyptula}
\Clue{36}{}{osoby wykonujące pracę służacego/służącej}
\Clue{37}{}{główny element konstrukcji}
\Clue{38}{}{wieś w Polsce położona w województwie kujawsko-pomorskim, w powiecie bydgoskim, w gminie Sicienko}
\Clue{39}{}{w starożytnej Grecji: kapłan, który wprowadza innych w tajniki misterium}\end{PuzzleClues}\newpage\section*{Krzyżówka 109}

\noindent\begin{Puzzle}{22}{33}|*	|*	|*	|[1][S]\darr	|*	|*	|*	|*	|*	|*	|*	|*	|*	|*	|*	|*	|*	|*	|*	|*	|*	|*	|*	|.
|*	|*	|*	|i	|*	|*	|*	|[2][S]\darr	|*	|[3][S]\drarr	|n	|i	|e	|ż	|y	|c	|i	|o	|w	|o	|ś	|ć	|*	|.
|*	|*	|[4][S]\rarr	|m	|a	|m	|b	|o	|*	|d	|*	|*	|*	|[5][S]\drarr	|h	|e	|d	|a	|j	|a	|t	|*	|*	|.
|*	|[6][S]\darr	|*	|a	|*	|*	|[7][S]\drarr	|b	|e	|z	|k	|i	|e	|r	|u	|n	|k	|o	|w	|o	|ś	|ć	|*	|.
|*	|s	|*	|g	|*	|*	|f	|i	|*	|i	|*	|*	|*	|ę	|*	|[8][S]\darr	|*	|*	|*	|*	|*	|*	|*	|.
|*	|t	|*	|i	|*	|*	|r	|e	|*	|w	|[9][S]\drarr	|r	|y	|b	|a	|k	|ó	|w	|k	|a	|*	|[10][S]\darr	|*	|.
|*	|o	|*	|n	|*	|*	|a	|g	|[11][S]\drarr	|o	|n	|o	|m	|a	|t	|o	|m	|a	|n	|c	|j	|a	|*	|.
|*	|p	|*	|a	|*	|*	|z	|[][,]{ }	|w	|l	|e	|[12][S]\drarr	|s	|k	|a	|r	|b	|n	|y	|*	|*	|r	|*	|.
|*	|a	|*	|c	|*	|*	|i	|s	|i	|i	|g	|c	|*	|*	|*	|a	|*	|*	|*	|*	|*	|u	|*	|.
|*	|[][,]{ }	|*	|j	|*	|*	|e	|y	|e	|c	|a	|z	|*	|*	|*	|[][,]{ }	|*	|*	|*	|*	|*	|m	|*	|.
|*	|f	|*	|a	|*	|*	|r	|n	|r	|z	|t	|e	|*	|*	|*	|w	|*	|*	|*	|*	|*	|u	|*	|.
|*	|u	|*	|*	|*	|*	|*	|o	|t	|e	|y	|r	|[13][S]\darr	|*	|*	|i	|*	|[14][S]\darr	|*	|*	|*	|ń	|*	|.
|*	|n	|[15][S]\darr	|[16][S]\darr	|*	|*	|[17][S]\darr	|d	|ł	|k	|w	|w	|k	|*	|[18][S]\darr	|n	|[19][S]\darr	|k	|*	|*	|*	|s	|*	|.
|*	|d	|s	|p	|*	|*	|m	|y	|o	|[][,]{ }	|*	|o	|r	|*	|w	|t	|p	|o	|*	|*	|*	|k	|*	|.
|*	|a	|z	|a	|*	|[20][S]\darr	|e	|c	|[][,]{ }	|z	|[21][S]\darr	|n	|e	|*	|i	|e	|o	|ń	|*	|[22][S]\darr	|*	|i	|*	|.
|*	|m	|k	|p	|*	|s	|t	|z	|p	|ł	|j	|y	|a	|*	|s	|r	|d	|[][,]{ }	|*	|c	|*	|*	|*	|.
|*	|e	|o	|i	|*	|z	|o	|n	|i	|o	|o	|[][,]{ }	|c	|*	|i	|a	|a	|m	|*	|i	|*	|*	|*	|.
|*	|n	|ł	|e	|*	|c	|d	|y	|ó	|t	|u	|k	|j	|*	|e	|*	|t	|e	|*	|e	|*	|*	|*	|.
|[23][S]\drarr	|t	|a	|r	|c	|z	|a	|*	|r	|y	|a	|a	|a	|*	|l	|*	|e	|k	|*	|m	|*	|*	|*	|.
|e	|o	|*	|z	|*	|y	|[][,]{ }	|*	|k	|*	|l	|r	|[][,]{ }	|*	|c	|*	|k	|l	|*	|n	|*	|*	|*	|.
|p	|w	|*	|y	|[24][S]\drarr	|p	|a	|g	|o	|n	|*	|z	|k	|*	|z	|*	|[][,]{ }	|e	|[25][S]\darr	|i	|*	|*	|*	|.
|i	|a	|*	|s	|a	|k	|g	|*	|w	|[26][S]\rarr	|p	|e	|a	|r	|y	|*	|d	|m	|r	|a	|*	|*	|*	|.
|s	|*	|*	|k	|n	|a	|l	|*	|e	|*	|*	|ł	|r	|[27][S]\darr	|[][,]{ }	|*	|e	|b	|a	|c	|*	|*	|*	|.
|t	|*	|*	|o	|c	|*	|o	|*	|*	|*	|*	|*	|d	|r	|h	|*	|g	|u	|j	|t	|*	|*	|*	|.
|e	|*	|*	|*	|y	|*	|m	|*	|*	|*	|*	|*	|y	|u	|u	|*	|r	|r	|s	|w	|*	|*	|*	|.
|m	|*	|*	|*	|m	|*	|e	|*	|*	|*	|*	|*	|n	|m	|m	|*	|e	|s	|k	|o	|*	|*	|*	|.
|o	|*	|[28][S]\drarr	|m	|o	|d	|r	|z	|e	|w	|[][,]{ }	|j	|a	|p	|o	|ń	|s	|k	|i	|*	|*	|*	|*	|.
|l	|[29][S]\drarr	|f	|e	|n	|t	|a	|n	|y	|l	|*	|*	|l	|o	|r	|*	|y	|i	|[][,]{ }	|*	|*	|*	|*	|.
|o	|k	|u	|*	|*	|*	|c	|*	|*	|*	|*	|*	|s	|l	|*	|*	|w	|*	|p	|*	|*	|*	|*	|.
|g	|i	|x	|*	|*	|*	|y	|*	|*	|*	|*	|*	|k	|o	|*	|*	|n	|*	|t	|*	|*	|*	|*	|.
|i	|j	|*	|*	|*	|*	|j	|*	|*	|*	|*	|*	|a	|g	|*	|*	|y	|*	|a	|*	|*	|*	|*	|.
|z	|*	|*	|*	|*	|[30][S]\rarr	|n	|o	|r	|n	|i	|k	|*	|*	|*	|*	|*	|*	|k	|*	|*	|*	|*	|.
|m	|[31][S]\rarr	|h	|u	|l	|m	|a	|n	|[][,]{ }	|z	|w	|y	|c	|z	|a	|j	|n	|y	|*	|*	|*	|*	|*	|.
|*	|*	|*	|*	|*	|*	|*	|*	|*	|*	|*	|*	|*	|*	|*	|*	|*	|*	|*	|*	|*	|*	|*	|.\end{Puzzle}

\newpage

\begin{PuzzleClues}{\textbf{Poziome}\\}\Clue{3}{}{postawa wyrażająca się w braku przystosowania do życia lub niemożności zastosowania w życiu}
\Clue{4}{}{taniec towarzyski pochodzący z Kuby, w metrum parzystym}
\Clue{5}{}{pisarz perski (1903-51), obyczajowe nowele, sztuki teatralne, eseje, studia folklorystyczne}
\Clue{7}{}{cecha czegoś bezkierunkowego, czegoś co nie ma sprecyzowanego kierunku, nie jest nakierowane na konkretny cel ani precyzyjnie określone, np. bezkierunkowy niepokój - czyli taki, który odczuwamy, chociaż nie umiemy dokładnie wyjaśnić co jest jego przyczyną ani czego konkretnie się obawiamy}
\Clue{9}{}{kompleks budynków: gospodarskich, mieszkalnych i administracyjnych oraz urządzeń pomocniczych w gospodarstwie rybackim}
\Clue{11}{}{wróżenie z imion}
\Clue{12}{}{urząd centralny niesenatorski Wielkiego Księstwa Litewskiego I RP, którego pełnienie polegało na strzeżeniu archiwum oraz skarbca Księstwa}
\Clue{23}{}{rodzaj naszywki z numerem szkoły, tarcza szkolna}
\Clue{24}{}{patka na ramieniu ubrania wojskowego lub cywilnego}
\Clue{26}{}{amerykański badacz polarny (1856-1920); odkrył północny kraniec Grenlandii, pierwszy w 1909 r. dotarł do bieguna północnego}
\Clue{28}{}{Larix kaempferi - gatunek drzewa należącego do rodziny sosnowatych}
\Clue{29}{}{organiczny związek chemiczny, pochodna piperydyny; syntetyczny środek przeciwbólowy i anestezjologiczny}
\Clue{30}{}{POLNIK - drobny gryzoń żerujący częściowo pod ziemią groźny szkodnik upraw}
\Clue{31}{}{langur hulman, hanuman, smukluch, Semnopithecus entellus - gatunek małpy wąskonosej z rodziny makakowatych; zamieszkuje Indie i kraje sąsiadujące}\end{PuzzleClues}

\begin{PuzzleClues}{\textbf{Pionowe}\\}\Clue{1}{}{wytwór wyobraźni, zmyślona historia}
\Clue{2}{}{uśredniony czas, po którym ciało niebieskie pojawi się ponownie w tym samym punkcie w stosunku do dwóch innych obiektów (liniowych węzłów)}
\Clue{3}{}{Melipotes gymnops - gatunek ptaka z rodziny miodojadów (Meliphagidae) występujący na Nowej Gwinei}
\Clue{5}{}{młot kamieniarski o dwu naprzeciwległych ostrzach}
\Clue{6}{}{rodzaj fundamentu umieszczanego pod pionową konstrukcją w celu przeniesienia ciężaru na grunt z tej konstrukcji}
\Clue{7}{}{bokser amerykański, mistrz olimpijski w kategorii ciężkiej z Tokio, zawodowy mistrz świata wszechwag w latach 1970-73}
\Clue{8}{}{cynamon magellański - rzadko spotykana w Polsce kamforowa przyprawa; kora zacierpu Wintera}
\Clue{9}{}{obraz fotograficzny, na którym miejsca ciemne są jasne i na odwrót}
\Clue{10}{}{język, a w zasadzie grupa dialektów i etnolektów, używana przez mieszkańców północnej Grecji, Macedonii, Bułgarii Albanii, spotykany także na terenie Rumunii, Serbii, Czarnogóry oraz Bośni i Hercegowiny}
\Clue{11}{}{płaskie; wiertło o kształcie ostrza zbliżonym do łopatki}
\Clue{12}{}{karzeł o typie widmowym K-M}
\Clue{13}{}{nadanie arcybiskupowi godności kardynalskiej}
\Clue{14}{}{koń gorącokrwisty pochodzący z Niemiec, używany jako koń wierzchowy}
\Clue{15}{}{czas spędzony w szkole na zajęciach}
\Clue{16}{}{papier}
\Clue{17}{}{metoda analizy zbiorów wielu elementów, która opiera się na tworzeniu skupień przez łączenie mniejszych zbiorowisk}
\Clue{18}{}{zły humor}
\Clue{19}{}{podatek, którego stawka zależy od wynagrodzenia osoby}
\Clue{20}{}{kawałek drewna, szczapka}
\Clue{21}{}{socjolekt robotniczych warstw społecznych w prowincji Québec}
\Clue{22}{}{zacofanie, ciemnota (ogólne niedouczenie albo w jakiejś konkretnej dziedzinie)}
\Clue{23}{}{nazwa każdego kierunku w filozofii, który podkreśla udział rozumu w procesie poznawania}
\Clue{24}{}{agregat, model, utrapienie, nicpoń}
\Clue{25}{}{Strelitzia reginae - gatunek dużej, ozdobnej rośliny z rodziny strelicjowatych}
\Clue{27}{}{wróżbita, który odczytuje ludzką osobowość i przepowiada przyszłość na podstawie kształtu pośladków}
\Clue{28}{}{austriacki kompozytor (1660-1741); opery, utwory religijne}
\Clue{29}{}{kawałek drewna o cylindrycznym kształcie, o różnej długości i grubości, posiadający dwa lub więcej końców (rozwidlenia), który został ułamany lub obcięty z drzewa, krzewu, trzciny lub trawy (bambus), ewentualnie wystrugany z drewna}\end{PuzzleClues}\newpage\section*{Krzyżówka 110}

\noindent\begin{Puzzle}{20}{24}|*	|*	|*	|*	|*	|*	|*	|*	|*	|*	|*	|*	|*	|*	|[1][S]\drarr	|f	|u	|t	|r	|o	|*	|.
|*	|[2][S]\darr	|*	|*	|*	|*	|*	|*	|[3][S]\rarr	|k	|o	|l	|u	|m	|b	|a	|r	|i	|u	|m	|*	|.
|[4][S]\drarr	|c	|o	|u	|s	|i	|n	|*	|*	|[5][S]\drarr	|g	|o	|w	|a	|r	|d	|h	|a	|n	|*	|[6][S]\darr	|.
|s	|z	|*	|[7][S]\rarr	|c	|z	|u	|w	|a	|s	|z	|k	|a	|*	|u	|*	|*	|*	|*	|*	|b	|.
|z	|a	|*	|[8][S]\darr	|*	|[9][S]\darr	|*	|*	|*	|y	|*	|[10][S]\drarr	|r	|a	|k	|i	|e	|t	|a	|*	|i	|.
|e	|r	|[11][S]\darr	|j	|[12][S]\darr	|e	|[13][S]\drarr	|p	|o	|l	|i	|g	|e	|n	|*	|*	|*	|*	|*	|[14][S]\darr	|e	|.
|*	|o	|d	|a	|i	|f	|c	|*	|[15][S]\darr	|w	|[16][S]\darr	|y	|*	|*	|*	|[17][S]\darr	|*	|[18][S]\darr	|*	|s	|r	|.
|[19][S]\drarr	|d	|i	|s	|n	|e	|y	|*	|o	|e	|m	|u	|[20][S]\drarr	|d	|*	|s	|[21][S]\darr	|w	|[22][S]\darr	|o	|n	|.
|p	|z	|p	|z	|f	|k	|g	|[23][S]\drarr	|s	|t	|i	|l	|b	|*	|*	|t	|d	|o	|k	|l	|o	|.
|ó	|i	|l	|c	|o	|t	|a	|r	|ł	|a	|k	|a	|o	|[24][S]\darr	|[25][S]\drarr	|r	|a	|d	|a	|*	|ś	|.
|ł	|e	|o	|z	|m	|o	|r	|e	|o	|*	|r	|*	|m	|z	|o	|z	|m	|o	|t	|*	|ć	|.
|b	|j	|i	|u	|a	|r	|n	|i	|n	|*	|o	|[26][S]\drarr	|b	|a	|t	|y	|s	|t	|a	|t	|*	|.
|a	|*	|d	|r	|t	|*	|i	|d	|k	|[27][S]\darr	|m	|c	|r	|ć	|u	|k	|k	|r	|l	|*	|*	|.
|j	|*	|*	|*	|*	|*	|c	|*	|a	|m	|e	|i	|a	|m	|l	|a	|i	|y	|o	|[28][S]\darr	|*	|.
|t	|[29][S]\drarr	|n	|u	|d	|y	|z	|m	|*	|e	|r	|o	|m	|a	|i	|w	|*	|s	|ń	|j	|*	|.
|*	|s	|*	|*	|*	|[30][S]\drarr	|k	|a	|n	|d	|*	|s	|r	|[][,]{ }	|n	|k	|*	|k	|c	|a	|*	|.
|[31][S]\drarr	|k	|r	|y	|s	|t	|a	|l	|o	|i	|d	|*	|e	|d	|a	|a	|*	|*	|z	|ź	|*	|.
|k	|r	|*	|[32][S]\darr	|[33][S]\darr	|y	|*	|*	|*	|a	|*	|*	|j	|o	|*	|*	|*	|*	|y	|ń	|*	|.
|a	|z	|*	|r	|s	|g	|*	|*	|*	|*	|[34][S]\rarr	|l	|a	|j	|k	|o	|n	|i	|k	|*	|*	|.
|p	|y	|[35][S]\drarr	|u	|t	|r	|a	|k	|w	|i	|z	|m	|*	|r	|*	|*	|*	|*	|*	|*	|*	|.
|i	|d	|l	|d	|o	|y	|*	|[36][S]\rarr	|c	|i	|o	|t	|c	|z	|y	|s	|k	|o	|*	|*	|*	|.
|s	|ł	|ó	|n	|p	|s	|*	|[37][S]\rarr	|p	|o	|d	|m	|i	|a	|n	|k	|a	|*	|*	|*	|*	|.
|t	|o	|d	|i	|a	|i	|*	|*	|*	|*	|*	|[38][S]\rarr	|b	|ł	|a	|g	|a	|n	|i	|e	|*	|.
|a	|*	|*	|k	|*	|ę	|*	|[39][S]\rarr	|k	|o	|p	|c	|i	|a	|ł	|k	|a	|*	|*	|*	|*	|.
|*	|*	|*	|*	|*	|*	|*	|[40][S]\rarr	|u	|n	|i	|o	|n	|*	|*	|*	|*	|*	|*	|*	|*	|.\end{Puzzle}

\newpage

\begin{PuzzleClues}{\textbf{Poziome}\\}\Clue{1}{}{skóra niektórych zwierząt (pokryta włosami), która jest po wyprawieniu używana do wyrobu różnych przedmiotów, głównie odzieży, galanterii i elementów wystroju wnętrz}
\Clue{3}{}{zbiorowy grobowiec w formie budowli lub sali z niszami na urny z prochami}
\Clue{4}{}{rodzina francuski malarzy i grafików (XV-XVI w.) przedstawiciele manieryzmu}
\Clue{5}{}{malowidła jednobarwne naśladujące płaskorzeźby, wykonane w szarym kolorze w różnych odcieniach}
\Clue{7}{}{mieszkanka Czuwaszji, kobieta pochodzenia czuwaszkiego}
\Clue{10}{}{sprzęt do gry w tenisa i ping-ponga}
\Clue{13}{}{gen, który razem z innymi genami, należącymi do innych par alleli, warunkuje tę samą cechę}
\Clue{19}{}{film z wytwórni Disneya}
\Clue{20}{}{nazwa literowa drugiego dźwięku w gamie, także od niej bierze oznaczenie tonacja, której toniką jest d}
\Clue{23}{}{jednostka luminacji, nielegalna}
\Clue{25}{}{posiedzenie rady, zebranie ludzi, którzy czymś rządzą, są za coś odpowiedzialni}
\Clue{26}{}{stalowa kabina w postaci pionowego cylindra do badań głębin morskich; bez własnego napędu}
\Clue{29}{}{praktykowanie wspólnej nagości w miejscach specjalnie wydzielonych albo odosobnionych mające na celu zbliżenie się do życia bliższego naturze}
\Clue{30}{}{pokarm dla pszczelich matek sporządzany z cukru pudru lub syropu i z miodu}
\Clue{31}{}{produkt przemiany materii w komórce roślinnej}
\Clue{34}{}{bohater tradycyjnego pochodu krakowskiego}
\Clue{35}{}{system szkolny łączący nauczanie klasyczne z nauczaniem realnym, obowiązywał w gimnazjach pruskich w latach 1801-1814}
\Clue{36}{}{określenie ciotki używane przeważnie jako żartobliwe, choć może też być określeniem wyrażającym współczucie lub politowanie}
\Clue{37}{}{zamiana czegoś na coś, często wykonywana niejawnie}
\Clue{38}{}{żarliwe, usilne proszenie kogoś o coś}
\Clue{39}{}{cegła okopcona sadzami w czasie wypalania}
\Clue{40}{}{miasto w Salwadorze nad Oceanem Spokojnym, jeden z głównych portów handlowych kraju}\end{PuzzleClues}

\begin{PuzzleClues}{\textbf{Pionowe}\\}\Clue{1}{}{nawierzchnia ulicy, placu lub chodnika utwardzona za pomocą układania na niej warstwy przylegających do siebie ściśle kamieni, kostek kamiennych, betonowych lub ceramicznych}
\Clue{2}{}{człowiek, który zna się na czarach, posługuje się czarami; w mitach, okultyzmie, legendach i literaturze fantasy, osoba praktykująca magię}
\Clue{4}{}{rodzaj cytry o 25 strunach}
\Clue{5}{}{budowa ciała, sylwetka człowieka}
\Clue{6}{}{obojętność, brak wigoru, inicjatywy, zaangażowania}
\Clue{8}{}{gatunek gada żyjącego w erze mezozoicznej}
\Clue{9}{}{w cybernetyce: organ sterowniczy, za pomocą którego układ oddziałuje na otoczenie}
\Clue{10}{}{miasto w płd.-wsch. Węgrzech w pobliżu granicy z Rumunią komitat Bekej}
\Clue{11}{}{komórka zawierająca po dwa chromosomy danego typu}
\Clue{12}{}{rodzaj urządzenia, najczęściej komputera, który ma służyć prezentowaniu lub wyszukiwaniu informacji}
\Clue{13}{}{rodzaj lufki, wydłużony ustnik do papierosa}
\Clue{14}{}{sylabowa nazwa dźwięku „g”}
\Clue{15}{}{mała osłona}
\Clue{16}{}{mały blastomer powstający na skutek nierówności segmentacji jaja}
\Clue{17}{}{instrument medyczny z pojemnikiem z podziałką i tłokiem, służący do wykonywania zastrzyków lub do pobierania płynów ustrojowych (np. krwi)}
\Clue{18}{}{ironiczne określenie jakiegoś bajeru, fajerwerku, czegoś zbędnego, ale fajnego}
\Clue{19}{}{połowa bajtu, 4 bity}
\Clue{20}{}{piąta w kolejności (patrząc od pokładu) reja, znajduje się na maszcie nad bramreją}
\Clue{21}{}{damska toaleta}
\Clue{22}{}{mieszkaniec Katalonii, człowiek, który pochodzi z Katalonii}
\Clue{23}{}{pisarz angielski (1818-83), przygodowe powieści dla młodzieży; „Łowcy skalpów”, „Jeździec bez głowy”}
\Clue{24}{}{cataracta matura - zaćma nabyta, w której występuje całkowite zamglenie soczewki}
\Clue{25}{}{warstwa betonu}
\Clue{26}{}{blok kamienny w kształcie graniastosłupa, zwykle taki, którego przynajmniej jeden bok jest ozdobny}
\Clue{27}{}{starożytne państwo na terenach zachodniej części obecnego Iranu, utworzone przez indo-irańskie plemiona Medów}
\Clue{28}{}{archetyp symbolizujący doskonałość człowieka}
\Clue{29}{}{w lotnictwie: jeden z płatów nośnych samolotu lub szybowca}
\Clue{30}{}{mały, młody tygrys}
\Clue{31}{}{malarz, będący członkiem grupy kapistów}
\Clue{32}{}{górnik; osoba pracująca w górnictwie, w kopalni przy urabianiu kopaliny}
\Clue{33}{}{najmniejsza jednostka miary wierszowej wywodząca się z antycznej wersyfikacji iloczasowej}
\Clue{35}{}{woda w stanie stałym}\end{PuzzleClues}\newpage\section*{Krzyżówka 111}

\noindent\begin{Puzzle}{21}{25}|*	|*	|*	|*	|*	|*	|*	|*	|*	|*	|*	|*	|*	|*	|*	|[1][S]\drarr	|m	|o	|n	|g	|u	|*	|.
|*	|*	|*	|*	|*	|*	|*	|*	|*	|[2][S]\drarr	|l	|u	|g	|o	|*	|ł	|[3][S]\drarr	|m	|a	|h	|e	|*	|.
|*	|*	|*	|*	|[4][S]\darr	|*	|*	|[5][S]\darr	|*	|m	|[6][S]\darr	|*	|*	|*	|[7][S]\drarr	|o	|b	|ł	|o	|*	|[8][S]\darr	|*	|.
|*	|*	|*	|*	|t	|*	|*	|m	|[9][S]\rarr	|a	|p	|e	|l	|*	|h	|s	|a	|*	|[10][S]\darr	|*	|s	|*	|.
|*	|*	|[11][S]\drarr	|m	|a	|g	|*	|a	|[12][S]\darr	|j	|e	|*	|[13][S]\rarr	|s	|e	|k	|r	|e	|c	|j	|a	|*	|.
|*	|*	|n	|[14][S]\darr	|j	|*	|*	|r	|c	|k	|d	|*	|*	|*	|l	|o	|i	|*	|o	|*	|m	|[15][S]\darr	|.
|*	|*	|a	|b	|n	|*	|*	|a	|h	|o	|o	|*	|*	|*	|i	|t	|b	|*	|a	|*	|o	|k	|.
|*	|*	|u	|i	|i	|*	|*	|s	|ł	|p	|f	|*	|*	|*	|n	|*	|a	|*	|c	|*	|w	|n	|.
|*	|*	|k	|o	|a	|*	|[16][S]\darr	|k	|o	|*	|i	|*	|*	|*	|g	|*	|l	|*	|h	|*	|o	|y	|.
|*	|*	|i	|l	|k	|[17][S]\rarr	|k	|a	|n	|c	|l	|e	|r	|z	|*	|*	|*	|*	|*	|*	|l	|p	|.
|*	|*	|[][,]{ }	|o	|*	|*	|ą	|*	|n	|*	|*	|*	|[18][S]\drarr	|p	|e	|n	|s	|ó	|w	|k	|a	|*	|.
|*	|*	|p	|g	|*	|*	|t	|[19][S]\darr	|y	|*	|*	|[20][S]\rarr	|k	|a	|n	|t	|o	|r	|a	|t	|*	|[21][S]\darr	|.
|*	|*	|o	|i	|[22][S]\darr	|[23][S]\darr	|[][,]{ }	|h	|[][,]{ }	|[24][S]\rarr	|s	|m	|o	|ł	|o	|w	|i	|e	|c	|*	|*	|b	|.
|*	|[25][S]\rarr	|m	|a	|s	|z	|k	|a	|r	|a	|*	|*	|l	|*	|*	|[26][S]\darr	|[27][S]\darr	|[28][S]\darr	|[29][S]\darr	|[30][S]\darr	|[31][S]\darr	|a	|.
|*	|*	|o	|[][,]{ }	|ł	|i	|u	|r	|y	|*	|*	|[32][S]\darr	|o	|*	|[33][S]\darr	|b	|s	|h	|a	|t	|e	|k	|.
|*	|*	|c	|e	|o	|e	|r	|m	|n	|*	|[34][S]\darr	|h	|n	|[35][S]\drarr	|n	|i	|k	|o	|p	|o	|l	|*	|.
|*	|*	|n	|w	|m	|l	|s	|a	|e	|[36][S]\darr	|d	|i	|i	|f	|o	|z	|i	|r	|r	|l	|i	|*	|.
|*	|*	|i	|o	|i	|o	|o	|t	|k	|w	|e	|n	|z	|i	|k	|o	|p	|n	|o	|n	|t	|*	|.
|*	|*	|c	|l	|a	|n	|w	|t	|*	|o	|c	|d	|a	|l	|s	|n	|[][,]{ }	|e	|b	|a	|a	|*	|.
|*	|*	|z	|u	|k	|e	|y	|a	|[37][S]\drarr	|s	|h	|u	|t	|e	|*	|*	|b	|b	|a	|*	|r	|*	|.
|*	|*	|e	|c	|*	|[][,]{ }	|*	|n	|n	|k	|a	|s	|o	|m	|*	|*	|*	|e	|c	|*	|y	|*	|.
|*	|*	|*	|y	|*	|l	|*	|*	|e	|*	|*	|k	|r	|o	|*	|*	|*	|r	|j	|*	|s	|*	|.
|[38][S]\rarr	|p	|e	|j	|z	|a	|ż	|*	|*	|*	|*	|a	|*	|n	|*	|*	|*	|*	|a	|*	|t	|*	|.
|[39][S]\rarr	|p	|u	|n	|k	|t	|[][,]{ }	|r	|o	|s	|y	|*	|*	|*	|*	|*	|*	|*	|*	|*	|a	|*	|.
|*	|[40][S]\rarr	|h	|a	|r	|a	|s	|i	|e	|w	|i	|c	|z	|*	|*	|*	|*	|*	|*	|*	|*	|*	|.
|*	|*	|*	|*	|*	|*	|*	|*	|*	|*	|*	|*	|*	|*	|*	|*	|*	|*	|*	|*	|*	|*	|.\end{Puzzle}

\newpage

\begin{PuzzleClues}{\textbf{Poziome}\\}\Clue{1}{}{miasto w Zambii nad rzeką Kabompo}
\Clue{2}{}{miasto w Hiszpanii (Galicja) nad rzeką Mino; ośrodek administracyjny prowincji Lugo}
\Clue{3}{}{miasto w płd. Indiach nad Morzem Arabskim}
\Clue{7}{}{zaokrąglenie kadłuba statku w miejscu, w którym dno przechodzi w burtę}
\Clue{9}{}{zorganizowane zebranie się grupy osób o określonej porze, w określonym miejscu, najczęściej poprzedzające coś}
\Clue{11}{}{kapłan-astrolog zoroastryzmu, któremu przypisywano umiejętność przepowiadania przyszłość z układu gwiazd, uzdrawiania i leczenia ludzi}
\Clue{13}{}{skupienie minerałów, zwykle o kształcie owalnym lub nieregularnym, wypełniające wolną przestrzeń lub szczelinę skalną; powstaje przez wytrącanie się kryształów z krążących w obrębie skały roztworów, pochodzących najczęściej z wyługowania tej skały lub z zewnątrz}
\Clue{17}{}{urzędnik państwowy w Polsce od XII wieku kierujący kancelarią monarchy oraz odpowiadający za politykę zagraniczną}
\Clue{18}{}{1 moneta o wartości jednego pensa}
\Clue{20}{}{stanowisko kantora w kościele protestanckim}
\Clue{24}{}{smoła zmieszana z piaskiem lub żwirem, używana do budowy nawierzchni dróg w okresie międzywojennym}
\Clue{25}{}{ktoś bardzo brzydki, wzbudzający odrazę}
\Clue{35}{}{miasto w Bułgarii (okręg Łowecz), port nad Dunajem}
\Clue{37}{}{(1899-1960), angielski pisarz, lotnik, powieści ukazujące wpływ wojny na losy ludzi; „Ostatni brzeg”, „Requiem dla dziewczyny”}
\Clue{38}{}{pejzaż, wydarzenia i fakty, składające się na jakieś zjawisko}
\Clue{39}{}{punkt w przestrzeni fazowej o określonych wartościach ciśnienia i temperatury, w którym zachodzi równowaga fazy ciekłej i gazowej, co - w przypadku jednoczesnego oddawania ciepła przez układ - prowadzi do natychmiastowego skraplania się gazu}
\Clue{40}{}{ur. w 1932 r., pianista, laureat Konkursu im. F. Chopina w 1955 r}\end{PuzzleClues}

\begin{PuzzleClues}{\textbf{Pionowe}\\}\Clue{1}{}{głośny, głuchy dźwięk powtarzający się kilka razy}
\Clue{2}{}{stolica Adygei, autonomicznej republiki wchodzącej w skład Federacji Rosyjskiej, leżąca nad Biełą}
\Clue{3}{}{niedźwiedź z lasów Ameryki Płd}
\Clue{4}{}{emisja gazów metabolicznych przez odbyt odbywająca się bezdźwięcznie}
\Clue{5}{}{rodzaj owocu, gorzkiej wiśni używanej w cukiernictwie i będącej np. materiałem na likier maraschino}
\Clue{6}{}{gatunek żyjący (rozwijający się) w glebie}
\Clue{7}{}{stanowisko budowy małych statków wodnych, szybowców itp}
\Clue{8}{}{nieposłuszeństwo zwierzchnikom lub brak względu na innych członków jakiegoś zespołu}
\Clue{10}{}{doradca, pomagający klientowi odkryć jego wiedzę i ukryte umiejętności}
\Clue{11}{}{nauki wspomagające pracę naukowców, niektóre z nich mogą być odrębnymi i samodzielnymi gałęziami nauki}
\Clue{12}{}{rynek, który cechuje wysoki popyt}
\Clue{14}{}{dział biologii zajmujący się badaniem pochodzenia gatunków od wspólnych przodków, ich zmianami i różnicowaniem się w czasie}
\Clue{15}{}{szewski nóż}
\Clue{16}{}{kąt zawarty między dziobową linią symetrii statku a linią łączącą dany obiekt z obserwatorem}
\Clue{18}{}{państwo lub naród, prowadzące politykę kolonizatorską}
\Clue{19}{}{silny, północno-wschodni wiatr wiejący w porze suchej znad Sahary na wybrzeże Zatoki Gwinejskiej, a także na zachodnie wybrzeże Afryki Północnej}
\Clue{21}{}{górny pokład dziobowej części statku lub jego nadbudówki dziobowej}
\Clue{22}{}{ul słomiany}
\Clue{23}{}{lata młodzieńcze}
\Clue{26}{}{żubr amerykański, Bison bison - duży ssak łożyskowy z rodziny krętorogich, rzędu parzystokopytnych, największy obecnie ssak Ameryki Północnej; zamieszkuje prerie oraz rzadkie, prześwietlone lasy Ameryki Północnej}
\Clue{27}{}{rodzaj ćwiczenia biegowego - bieg z wysokim unoszeniem kolan i kopnięciami w przód}
\Clue{28}{}{strzelczyni niemiecka, dwukrotna medalistka olimpijska z Atlanty, rywalka R. Mauer}
\Clue{29}{}{aprobata, pozwolenie dla jakichś działań, często wyrażone w sposób oficjalny}
\Clue{30}{}{komitat w południowych Węgrzech}
\Clue{31}{}{taki, który uważa, że ludzie różnią się pod względem wartości i nie należy wszystkich traktować na równi}
\Clue{32}{}{mieszkanka Indii, kobieta pochodzenia hinduskiego}
\Clue{33}{}{jednostka natężenia oświetlenia}
\Clue{34}{}{duża, gruba deska - kawał drewna}
\Clue{35}{}{australijski ptak z rodziny miodojadów}
\Clue{36}{}{wydzielina gruczołów odwłokowych robotnic pszczoły młodej}
\Clue{37}{}{w chemii: symbol neonu}\end{PuzzleClues}\newpage\section*{Krzyżówka 112}

\noindent\begin{Puzzle}{25}{28}|*	|*	|*	|*	|*	|*	|*	|*	|*	|*	|*	|*	|*	|*	|*	|*	|[1][S]\drarr	|f	|l	|ą	|d	|e	|r	|k	|a	|*	|.
|*	|*	|*	|*	|*	|*	|*	|*	|*	|*	|*	|*	|*	|[2][S]\rarr	|o	|b	|s	|t	|r	|u	|k	|c	|j	|a	|*	|*	|.
|*	|*	|[3][S]\drarr	|g	|e	|o	|m	|e	|t	|r	|i	|a	|[][,]{ }	|e	|l	|i	|p	|t	|y	|c	|z	|n	|a	|*	|*	|*	|.
|*	|*	|e	|*	|*	|*	|[4][S]\rarr	|k	|u	|l	|i	|k	|[][,]{ }	|m	|n	|i	|e	|j	|s	|z	|y	|*	|*	|*	|[5][S]\darr	|*	|.
|*	|*	|g	|*	|*	|*	|[6][S]\drarr	|a	|m	|e	|r	|y	|k	|a	|ń	|s	|k	|o	|ś	|ć	|*	|*	|*	|[7][S]\darr	|b	|*	|.
|*	|*	|z	|*	|*	|*	|s	|*	|*	|*	|*	|*	|*	|[8][S]\darr	|*	|*	|u	|*	|*	|*	|*	|*	|*	|p	|ą	|*	|.
|*	|*	|a	|*	|*	|*	|z	|*	|*	|*	|*	|*	|*	|z	|*	|*	|l	|*	|*	|*	|*	|*	|[9][S]\darr	|i	|b	|*	|.
|*	|*	|m	|*	|*	|[10][S]\rarr	|k	|o	|n	|f	|i	|r	|m	|a	|c	|j	|a	|*	|*	|*	|*	|[11][S]\darr	|d	|l	|e	|*	|.
|*	|*	|i	|*	|*	|[12][S]\rarr	|a	|m	|b	|i	|t	|n	|o	|ś	|ć	|*	|n	|*	|*	|*	|*	|m	|a	|n	|l	|*	|.
|*	|*	|n	|*	|*	|*	|r	|*	|*	|*	|*	|*	|*	|w	|*	|*	|t	|*	|*	|*	|*	|a	|l	|i	|[][,]{ }	|*	|.
|*	|*	|[][,]{ }	|*	|*	|*	|ł	|*	|*	|*	|*	|*	|*	|i	|*	|*	|*	|*	|*	|*	|*	|r	|m	|k	|s	|*	|.
|*	|*	|p	|*	|*	|*	|a	|[13][S]\darr	|*	|*	|[14][S]\darr	|[15][S]\rarr	|m	|a	|l	|i	|s	|z	|e	|w	|s	|k	|i	|*	|p	|*	|.
|*	|*	|o	|*	|*	|*	|t	|g	|*	|*	|k	|*	|*	|t	|*	|*	|*	|*	|*	|*	|*	|i	|e	|*	|e	|*	|.
|*	|*	|p	|*	|*	|*	|*	|r	|[16][S]\rarr	|m	|a	|n	|*	|y	|[17][S]\rarr	|g	|r	|o	|t	|t	|g	|e	|r	|*	|k	|*	|.
|*	|*	|r	|[18][S]\drarr	|u	|k	|ł	|a	|d	|a	|r	|k	|a	|*	|*	|*	|*	|*	|*	|*	|*	|t	|z	|*	|u	|*	|.
|*	|*	|a	|p	|*	|*	|*	|[][,]{ }	|*	|*	|o	|*	|*	|*	|*	|*	|*	|*	|*	|*	|*	|a	|[][,]{ }	|*	|l	|*	|.
|*	|*	|w	|e	|*	|*	|*	|c	|*	|*	|*	|[19][S]\rarr	|p	|ł	|o	|z	|a	|[][,]{ }	|o	|g	|o	|n	|o	|w	|a	|*	|.
|*	|*	|k	|s	|*	|[20][S]\rarr	|s	|y	|s	|t	|e	|m	|[][,]{ }	|a	|l	|a	|r	|m	|o	|w	|y	|*	|p	|*	|c	|*	|.
|*	|*	|o	|z	|[21][S]\rarr	|h	|o	|l	|[][,]{ }	|r	|e	|c	|e	|p	|c	|y	|j	|n	|y	|*	|*	|*	|t	|*	|y	|*	|.
|*	|*	|w	|t	|*	|*	|*	|i	|*	|*	|*	|*	|*	|*	|*	|*	|*	|*	|*	|*	|*	|*	|y	|*	|j	|*	|.
|*	|*	|y	|*	|*	|*	|*	|n	|*	|*	|*	|*	|*	|*	|*	|*	|*	|*	|*	|*	|*	|*	|c	|*	|n	|*	|.
|*	|*	|*	|*	|*	|*	|[22][S]\rarr	|d	|i	|y	|e	|r	|t	|i	|m	|e	|n	|t	|o	|*	|*	|*	|z	|*	|y	|*	|.
|*	|[23][S]\rarr	|k	|o	|o	|p	|e	|r	|a	|c	|j	|a	|*	|*	|*	|*	|*	|*	|*	|*	|*	|*	|n	|*	|*	|*	|.
|*	|*	|[24][S]\rarr	|g	|r	|u	|b	|y	|[][,]{ }	|z	|w	|i	|e	|r	|z	|*	|*	|*	|*	|*	|*	|*	|y	|*	|*	|*	|.
|*	|*	|*	|*	|*	|*	|*	|c	|*	|*	|*	|*	|*	|*	|*	|*	|*	|*	|*	|*	|*	|*	|*	|*	|*	|*	|.
|*	|*	|*	|[25][S]\rarr	|p	|u	|s	|z	|c	|z	|y	|k	|[][,]{ }	|j	|a	|s	|n	|y	|*	|*	|*	|*	|*	|*	|*	|*	|.
|*	|[26][S]\rarr	|i	|m	|p	|l	|a	|n	|t	|a	|c	|j	|a	|[][,]{ }	|j	|o	|n	|ó	|w	|*	|*	|*	|*	|*	|*	|*	|.
|*	|*	|*	|*	|*	|[27][S]\rarr	|j	|a	|j	|k	|o	|[][,]{ }	|p	|o	|[][,]{ }	|w	|i	|e	|d	|e	|ń	|s	|k	|u	|*	|*	|.
|[28][S]\rarr	|k	|o	|s	|t	|k	|a	|*	|*	|*	|*	|*	|*	|*	|*	|*	|*	|*	|*	|*	|*	|*	|*	|*	|*	|*	|.\end{Puzzle}

\newpage

\begin{PuzzleClues}{\textbf{Poziome}\\}\Clue{1}{}{STORNIA}
\Clue{2}{}{hamowanie przebiegu jakiejś sprawy}
\Clue{3}{}{jeden z rodzajów geometrii nieeuklidesowej, szczególny przypadek geometrii Riemanna dla stałej i dodatniej krzywizny}
\Clue{4}{}{Numenius phaeopus phaeopus - nominatywny podgatunek ptaka wyróżniony w obrębie gatunku kulik mniejszy (Numenius phaeopus); występuje na obszarze od Islandii, Wysp Owczych i północnej Szkocji przez Półwysep Skandynawski po południowo-zachodni Tajmyr}
\Clue{6}{}{zespół cech typowych dla czegoś z Ameryki Południowej lub dla mieszkańca tego kontynentu}
\Clue{10}{}{w logice - odmiana rozumowania}
\Clue{12}{}{to, że coś jest ambitne pod względem artystycznym; ma dużą wartość, przekazuje ważne treści, jest dopracowane formalnie}
\Clue{15}{}{kompozytor i pedagog (1873-1939); symfonie, balety, utwory kameralne, msze, pieśni}
\Clue{16}{}{wyspa należąca do archipelagu Wysp Brytyjskich, położona na Morzu Irlandzkim między Wielką Brytanią a Irlandią}
\Clue{17}{}{dzieła Grottgera}
\Clue{18}{}{ROZKŁADARKA}
\Clue{19}{}{pomocniczy element podwozia samolotu wspomagający operowanie na podłożach nieutwardzonych}
\Clue{20}{}{zespół urządzeń służących najczęściej zabezpieczeniu danego obiektu przed włamaniem (system alarmu włamania) lub pożarem (system przeciwpożarowy)}
\Clue{21}{}{duży korytarz znajdujący prowadzący od głównego wejścia holu, często połączony z recepcją}
\Clue{22}{}{instrumentalna forma muzyki popularnej w XVIII w}
\Clue{23}{}{typ relacji między konkurentami, w których występują jednocześnie strumienie kooperacji i konkurencji}
\Clue{24}{}{zwierzyna, na którą poluje się, strzelając kulami}
\Clue{25}{}{Strix occidentalis lucida - podgatunek ptaka drapieżnego wyróżniony w obrębie gatunku puszczyk plamisty (Strix occidentalis)}
\Clue{26}{}{domieszkowanie materiałów polegające na rozpędzeniu jonów w polu elektrycznym i zderzeniu z domieszkowanym materiałem}
\Clue{27}{}{jajko w koszulce ugotowane na miękko (w naczynku zanurzonym we wrzątku) z dodatkiem pieprzu i masła}
\Clue{28}{}{wystająca mała kość przy nadgarstku}\end{PuzzleClues}

\begin{PuzzleClues}{\textbf{Pionowe}\\}\Clue{1}{}{osoba przeprowadzająca nieuczciwe transakcje handlowe, których celem jest osiągnięcie określonego dochodu poprzez wykorzystanie przewidywanych zmian cenowych pomiędzy terminem zawarcia określonej umowy a jej terminem realizacji}
\Clue{3}{}{rodzaj egzaminu promocyjnego stosowanego w stosunku do uczniów lub studentów, którzy nie uzyskali w wymaganym terminie zaliczenia z danego przedmiotu lub przedmiotów (uzyskali ocenę niedostateczną)}
\Clue{5}{}{samonapędzający się proces niezrównoważonego wzrostu cen dóbr na rynku}
\Clue{6}{}{intensywnie, ciemnoczerwony strój, zazwyczaj postrzegany jako oznaka jakiejś godności, pozycji społecznej}
\Clue{7}{}{narzędzie do obróbki ręcznej w postaci pręta stalowego osadzonego w uchwycie}
\Clue{8}{}{w wierzeniach religijnych świat (lub światy) istniejący poza przestrzenią dostępną bezpośredniej percepcji wyznaczony najczęściej poprzez konkretny model kosmosu dla różnych ludów lub obszarów świata}
\Clue{9}{}{dalmierz, który wskazuje odległość obserwowanego obietku przez pomiar kąta paralaksy osi dwóch obiektywów}
\Clue{11}{}{handlarz wędrowny ciągnący za wojskiem, sprzedający żołnierzom żywność, napoje i drobne przedmioty codziennego użytku}
\Clue{13}{}{gra, w której wygrana zależy od wytypowanego wcześniej wyniku, a końcowy wynik jest ustalany za pomocą urządzenia obrotowego}
\Clue{14}{}{jeden z 4 kolorów w kartach, oznaczony czerwonym rombem (¦)}
\Clue{18}{}{wsch. części Budapesztu na lewym brzegu Dunaju}\end{PuzzleClues}\newpage\section*{Krzyżówka 113}

\noindent\begin{Puzzle}{22}{24}|*	|*	|*	|*	|*	|*	|*	|*	|*	|*	|*	|*	|[1][S]\drarr	|d	|r	|u	|k	|*	|*	|[2][S]\darr	|[3][S]\darr	|*	|*	|.
|*	|*	|*	|*	|[4][S]\darr	|*	|[5][S]\rarr	|b	|r	|o	|ń	|[][,]{ }	|s	|t	|r	|z	|e	|l	|e	|c	|k	|a	|*	|.
|*	|*	|*	|[6][S]\darr	|s	|*	|*	|[7][S]\drarr	|k	|r	|u	|c	|z	|ę	|*	|[8][S]\darr	|*	|*	|[9][S]\darr	|h	|m	|*	|*	|.
|*	|*	|*	|p	|y	|*	|*	|b	|*	|*	|*	|*	|m	|*	|*	|l	|[10][S]\darr	|*	|a	|o	|i	|[11][S]\darr	|[12][S]\darr	|.
|*	|*	|[13][S]\darr	|ó	|r	|*	|[14][S]\rarr	|l	|i	|p	|a	|*	|a	|*	|[15][S]\darr	|a	|p	|*	|e	|r	|n	|c	|o	|.
|*	|[16][S]\rarr	|g	|ł	|o	|ś	|n	|o	|ś	|ć	|*	|*	|r	|*	|i	|m	|r	|[17][S]\darr	|r	|o	|*	|e	|c	|.
|*	|*	|o	|*	|p	|[18][S]\darr	|*	|g	|*	|*	|*	|*	|t	|*	|n	|i	|o	|f	|o	|b	|*	|c	|i	|.
|*	|*	|d	|*	|e	|j	|*	|o	|[19][S]\drarr	|t	|o	|r	|u	|*	|w	|n	|s	|i	|z	|a	|*	|h	|e	|.
|*	|*	|o	|[20][S]\darr	|k	|a	|*	|n	|c	|*	|*	|*	|z	|*	|e	|a	|t	|l	|o	|[][,]{ }	|*	|o	|k	|.
|*	|*	|w	|p	|*	|s	|*	|o	|h	|*	|*	|*	|*	|*	|n	|t	|a	|t	|l	|f	|[21][S]\darr	|w	|a	|.
|*	|*	|n	|o	|*	|k	|*	|t	|r	|*	|[22][S]\rarr	|o	|k	|o	|t	|*	|[][,]{ }	|r	|[][,]{ }	|o	|p	|n	|c	|.
|*	|*	|i	|l	|[23][S]\darr	|ó	|[24][S]\drarr	|k	|u	|l	|c	|z	|y	|b	|a	|*	|p	|[][,]{ }	|b	|r	|o	|i	|z	|.
|*	|*	|k	|a	|m	|ł	|s	|a	|s	|[25][S]\drarr	|s	|u	|p	|e	|r	|k	|o	|b	|i	|e	|t	|a	|*	|.
|*	|*	|*	|n	|i	|c	|z	|*	|t	|r	|*	|*	|*	|*	|z	|*	|t	|a	|o	|s	|e	|*	|*	|.
|*	|[26][S]\rarr	|k	|i	|s	|z	|k	|a	|*	|o	|*	|*	|*	|*	|[][,]{ }	|*	|ę	|r	|l	|t	|n	|*	|*	|.
|*	|*	|*	|e	|o	|ę	|a	|*	|[27][S]\rarr	|k	|m	|[][S]2	|*	|*	|m	|*	|g	|w	|o	|i	|c	|*	|*	|.
|*	|*	|*	|*	|*	|*	|p	|*	|[28][S]\drarr	|p	|s	|i	|a	|r	|a	|*	|o	|n	|g	|e	|j	|*	|*	|.
|*	|*	|*	|*	|*	|[29][S]\rarr	|l	|i	|s	|o	|w	|o	|*	|*	|r	|*	|w	|y	|i	|r	|a	|*	|*	|.
|[30][S]\rarr	|o	|k	|ę	|c	|i	|e	|*	|r	|l	|[31][S]\rarr	|g	|e	|s	|t	|i	|a	|*	|c	|a	|ł	|*	|*	|.
|[32][S]\rarr	|m	|u	|l	|l	|e	|r	|*	|o	|*	|*	|*	|[33][S]\rarr	|i	|w	|o	|*	|*	|z	|*	|[][,]{ }	|*	|*	|.
|*	|*	|*	|*	|*	|*	|z	|[34][S]\rarr	|k	|r	|a	|j	|c	|z	|y	|*	|*	|*	|n	|*	|z	|*	|*	|.
|[35][S]\rarr	|l	|i	|l	|i	|a	|*	|[36][S]\rarr	|a	|p	|o	|l	|l	|o	|*	|*	|*	|*	|y	|*	|e	|*	|*	|.
|*	|[37][S]\rarr	|m	|e	|d	|i	|a	|s	|t	|i	|n	|o	|s	|k	|o	|p	|i	|a	|*	|*	|t	|*	|*	|.
|[38][S]\rarr	|j	|ę	|z	|y	|k	|[][,]{ }	|s	|a	|m	|o	|j	|e	|d	|z	|k	|i	|*	|*	|*	|a	|*	|*	|.
|[39][S]\rarr	|n	|o	|ś	|n	|o	|ś	|ć	|*	|*	|*	|*	|[40][S]\rarr	|t	|e	|l	|a	|m	|o	|n	|*	|*	|*	|.\end{Puzzle}

\newpage

\begin{PuzzleClues}{\textbf{Poziome}\\}\Clue{1}{}{tkanina wzorzysta, pokryta deseniem odbitym przy użyciu szablonów i specjalnych farb}
\Clue{5}{}{broń palna kalibru poniżej 20 mm o stosunkowo niewielkim zasięgu}
\Clue{7}{}{pisklę kruka}
\Clue{14}{}{towar lub usługa, które są niesolidne, niezgodne z oczekiwaniami; badziewie, fuszerka, popierdółka}
\Clue{16}{}{to, że coś jest głośne; np. głośność dyskoteki}
\Clue{19}{}{drewniany lub kamienny łuk nad bramą świątyni sintoistycznej}
\Clue{22}{}{okocenie się}
\Clue{24}{}{Strychnos - rodzaj roślin należący do rzędu goryczkowców; wyróżnia się ok.100 gatunków, pochodzi z Azji i Australii}
\Clue{25}{}{osobnik o cechach fenotypowych żeńskich, często z wyjątkowo dobrze zaznaczonymi trzeciorzędowymi cechami płciowymi, z wrodzoną trisomią chromosomu X}
\Clue{26}{}{jelito - fragment przewodu pokarmowego, w którym zachodzi proces wchłaniania substancji powstałych w wyniku rozkładu pokarmów przez enzymy}
\Clue{27}{}{jednostka wielokrotna jednostki pola powierzchni - metra kwadratowego}
\Clue{28}{}{zespół roślinny zbudowany z roślinności murawowej z domieszką trawy kępowej, porastający ubogie, kwaśne i suche gleby}
\Clue{29}{}{osada w północno-zachodniej Polsce, położona w województwie zachodniopomorskim, w powiecie gryfickim, w gminie Płoty}
\Clue{30}{}{warszawskie osiedle leżące na terenie Dzielnicy Włochy}
\Clue{31}{}{zakres uprawnień, komptencji w zakresie jakiegoś działania}
\Clue{32}{}{ur. 1928, pisarz niemiecki, powieści, opowiadania, słuchowiska radiowe}
\Clue{33}{}{miasto w płd.-zach. Nigerii, ośrodek handlu i rzemiosła}
\Clue{34}{}{w Polsce wczesnośredniowiecznej urzędnik nadworny zobowiązany do krajania potraw podawanych panującemu}
\Clue{35}{}{cebulkowa bylina o dużych, wonnych kwiatach, w Polsce chroniona, uprawiana na kwiat cięty i na rabatach}
\Clue{36}{}{piękny, przystojny, młody, wysportowany i silny młodzieniec}
\Clue{37}{}{badanie pozwalające na zobrazowanie zawartości śródpiersia}
\Clue{38}{}{język z rodziny języków samojedzkich}
\Clue{39}{}{dopuszczalny maksymalny ciężar jaki można załadować na statek wodny}
\Clue{40}{}{architektoniczna podpora}\end{PuzzleClues}

\begin{PuzzleClues}{\textbf{Pionowe}\\}\Clue{1}{}{kram, stragan, buda kupiecka, rodzaj prowizorycznego sklepiku}
\Clue{2}{}{choroba, która cechuje się rozrostem kości w miejscach przyczepów ścięgien, rozcięgien i torebek stawowych}
\Clue{3}{}{Cuminum - rodzaj roślin należący do rodziny selerowatych}
\Clue{4}{}{roztwór cukru zawierający ekstrakty produktu, z którego uzyskuje się syrop}
\Clue{6}{}{jedna z dwóch równych części jakiejś całości; jedna druga}
\Clue{7}{}{notka, tekst, wypowiedź, umieszczana na blogu internetowym}
\Clue{8}{}{rodzaj kompozytu, tworzywa powstającego z połączenia dwóch materiałów o różnych właściwościach mechanicznych, fizycznych i technologicznych, w których składnik wzmacniający (tzw. zbrojenie) jest układany w postaci warstw, między którymi znajduje się wypełnienie, pełniące rolę lepiszcza}
\Clue{9}{}{cząstki pochodzenia naturalnego, takie jak bakterie, wirusy, pyłki roślin unoszące się w powietrzu atmosferycznym}
\Clue{10}{}{zbiór punktów między dwoma okręgami,mających równe potęgi względem ich obu}
\Clue{11}{}{część kopalni, w której zbierają się górnicy przed zjazdem pod powierzchnię i po powrocie na nią w celu omówienia spraw służbowych, czasem odbywają się tam również uroczystości górnicze}
\Clue{12}{}{element wyposażenia kuchennego, na którym układa się do obcieknięcia lub wysuszenia warzywa, owoce albo mokre naczynia}
\Clue{13}{}{uczestnik godów - wesela}
\Clue{15}{}{trwałe składniki majątku - budynki, urządzenia, maszyny}
\Clue{17}{}{płaska nasadka na obiektyw zmieniająca wygląd fotografowanego obrazu, nakładana najczęściej na przód obiektywu, a jeśli jest to niemożliwe, to na jego tył}
\Clue{18}{}{pisklę jaskółki}
\Clue{19}{}{suche części drzewa wykorzystywane do rozpalania ognia lub innych celów, np. wyrobu mioteł}
\Clue{20}{}{plemię słowiańskie, zamieszkujące Pojezierze Wielkopolskie, które miało doprowadzić do powstania pierwszego państwa na ziemiach dzisiejszej Polski}
\Clue{21}{}{Potencjał elektrokinetyczny, potencjał dzeta, potencjał zeta, potencjał ? - potencjał występujący przy powierzchni ciała stałego lub innych cząstek rozproszonych (emulsje), kontaktującej się z roztworem elektrolitu, określany na granicy poślizgu}
\Clue{23}{}{tradycyjna i bardzo pożywna gęsta pasta japońska, produkowana ze sfermentowanej soi, najczęściej z dodatkiem ryżu lub jęczmienia, soli oraz drożdży}
\Clue{24}{}{kawałek materiału z naszytymi wizerunkami Chrystusa i Matki Bożej, noszony przez osoby świeckie na znak duchowej i cielesnej czystości oraz duchowego związku z Matką Boską; noszenie szkaplerza i stosowanie się do pewnych reguł ma odganiać od noszącego złe moce i zapewnić mu miejsce w niebie}
\Clue{25}{}{gatunek polskiego owczego sera produkowanego na wzór rokforu}
\Clue{28}{}{klacz o maści pstrej}\end{PuzzleClues}\newpage\section*{Krzyżówka 114}

\noindent\begin{Puzzle}{17}{24}|*	|*	|*	|[1][S]\drarr	|c	|h	|i	|n	|o	|o	|k	|*	|[2][S]\drarr	|g	|o	|d	|y	|*	|.
|*	|*	|[3][S]\rarr	|s	|z	|e	|*	|*	|[4][S]\darr	|*	|*	|[5][S]\darr	|w	|*	|[6][S]\darr	|[7][S]\darr	|*	|*	|.
|*	|*	|[8][S]\darr	|p	|[9][S]\darr	|*	|*	|*	|s	|*	|[10][S]\darr	|c	|i	|*	|r	|r	|[11][S]\darr	|[12][S]\darr	|.
|*	|*	|z	|ó	|h	|*	|*	|*	|w	|*	|w	|y	|e	|*	|ó	|y	|k	|s	|.
|*	|[13][S]\darr	|a	|j	|o	|*	|*	|*	|e	|*	|i	|p	|ń	|*	|w	|p	|o	|a	|.
|*	|i	|s	|n	|m	|[14][S]\rarr	|f	|r	|e	|u	|d	|*	|c	|*	|n	|s	|n	|l	|.
|*	|n	|t	|i	|e	|*	|[15][S]\darr	|[16][S]\rarr	|t	|e	|m	|p	|e	|r	|a	|*	|i	|e	|.
|*	|t	|r	|c	|s	|*	|j	|*	|*	|*	|o	|*	|*	|*	|n	|*	|c	|p	|.
|[17][S]\rarr	|r	|z	|e	|p	|k	|a	|*	|*	|[18][S]\darr	|[][,]{ }	|[19][S]\drarr	|z	|n	|i	|c	|z	|*	|.
|*	|o	|y	|*	|u	|[20][S]\drarr	|s	|z	|u	|b	|r	|a	|w	|i	|e	|c	|*	|[21][S]\darr	|.
|*	|s	|k	|*	|n	|r	|t	|[22][S]\darr	|*	|o	|e	|d	|[23][S]\darr	|*	|[][,]{ }	|*	|*	|p	|.
|*	|p	|*	|*	|*	|a	|r	|j	|*	|d	|n	|j	|b	|*	|n	|*	|[24][S]\darr	|l	|.
|*	|e	|*	|*	|[25][S]\darr	|f	|o	|e	|*	|o	|t	|u	|u	|[26][S]\darr	|i	|*	|a	|i	|.
|*	|k	|*	|*	|a	|i	|w	|l	|*	|t	|g	|w	|k	|m	|e	|[27][S]\darr	|r	|o	|.
|*	|c	|*	|[28][S]\drarr	|m	|n	|i	|c	|h	|*	|e	|a	|r	|ł	|w	|h	|t	|z	|.
|*	|j	|*	|s	|m	|e	|e	|z	|[29][S]\darr	|*	|n	|n	|a	|y	|y	|e	|e	|a	|.
|*	|o	|[30][S]\drarr	|t	|o	|r	|*	|*	|r	|[31][S]\darr	|o	|t	|n	|n	|m	|r	|m	|u	|.
|*	|n	|j	|o	|z	|*	|*	|*	|a	|d	|w	|*	|i	|a	|i	|a	|i	|r	|.
|*	|i	|e	|p	|a	|*	|*	|[32][S]\rarr	|k	|a	|s	|t	|o	|r	|e	|k	|s	|*	|.
|*	|z	|d	|a	|u	|*	|*	|*	|*	|n	|k	|*	|n	|z	|r	|l	|a	|[33][S]\darr	|.
|*	|m	|y	|*	|r	|*	|*	|*	|*	|i	|i	|*	|*	|ó	|n	|i	|*	|l	|.
|*	|*	|n	|*	|*	|*	|*	|*	|*	|e	|e	|*	|*	|w	|e	|t	|*	|t	|.
|*	|*	|k	|[34][S]\rarr	|m	|e	|n	|z	|e	|l	|*	|*	|*	|n	|*	|*	|*	|l	|.
|[35][S]\rarr	|c	|a	|ł	|y	|[][,]{ }	|t	|o	|n	|*	|[36][S]\rarr	|d	|r	|a	|p	|e	|r	|*	|.
|*	|*	|*	|*	|*	|*	|[37][S]\rarr	|k	|l	|ą	|t	|w	|a	|*	|*	|*	|*	|*	|.\end{Puzzle}

\newpage

\begin{PuzzleClues}{\textbf{Poziome}\\}\Clue{1}{}{ciepły, suchy i porywisty wiatr typu fenowego, wiejący ze wschodnich stoków Gór Skalistych na równiny}
\Clue{2}{}{rykowisko u jeleni}
\Clue{3}{}{rodzaj cytry o 25 strunach}
\Clue{14}{}{austriacki lekarz neurolog i psychiatra, twórca psychoanalizy}
\Clue{16}{}{odmiana malarstwa niegeometrycznego, polega na tworzeniu ze swobodnych palm barwnych uzyskiwanych np. przez rozlewanie bądź rozpryskiwanie farb}
\Clue{17}{}{część odnóża u stawonogów}
\Clue{19}{}{szklane, metalowe bądź plastikowe naczynie wypełnione stearyną, w której zatopiony jest knot; zapalane na grobach w celu uczczenia pamięci  zmarłych}
\Clue{20}{}{człowiek nikczemny, łajdak, łotr, niegodziwiec}
\Clue{28}{}{zakonnik z katolickiego lub prawosławnego zakonu kontemplacyjnego}
\Clue{30}{}{droga, po której porusza się obiekt, zwłaszcza pojazd}
\Clue{32}{}{REKS}
\Clue{34}{}{niemiecki malarzy i grafik (1815-1905) reprezentant realizmu, kompozycje historyczne z czasów Fryderyka II}
\Clue{35}{}{jednostka odległości między dwoma dźwiękami, składa się z dwóch półtonów}
\Clue{36}{}{Henzy, ur.w1837r. amerykański astronom, przyrodnik; autor prac ze spektroskopii astronomicznej}
\Clue{37}{}{zaklęcie, którym czyni się zło}\end{PuzzleClues}

\begin{PuzzleClues}{\textbf{Pionowe}\\}\Clue{1}{}{Melittinae - podrodzina owadów należąca do pszczół samotnic w podrzędzie trzonkówek}
\Clue{2}{}{barwna poświata wokół tarczy Słońca lub Księżyca}
\Clue{4}{}{odmiana jazzu wykonywana bez improwizacji}
\Clue{5}{}{kod ISO 4217 funta cypryjskiego}
\Clue{6}{}{równanie, w którym niewiadoma znajduje się pod pierwiastkiem}
\Clue{7}{}{płótno o specyficznym splocie}
\Clue{8}{}{przenośnie: przypływ czegoś, nowa porcja czegoś, której pojawienie się przynosi pozytywne skutki}
\Clue{9}{}{gruba, miękka, lekko spilśniona tkanina wełniana o splocie płóciennym, z przędzy z pęczkami lub supełkami, używana na kostiumy, płaszcze itp}
\Clue{10}{}{rozkład promieniowania rentgenowskiego w zależności od zmiany długośli lub częstotliwości fal}
\Clue{11}{}{KUNTZE}
\Clue{12}{}{napój przyrządzany ze sproszkowanych bulw storczyka męskiego, znany w kuchni tureckiej i w kuchni bałkańskiej}
\Clue{13}{}{technika zbierania danych, rozwijana przez XIX-wiecznych psychologów, m.in. Wundta, kładąca nacisk na bezpośrednie i subiektywne badanie świadomości - introspekcję}
\Clue{15}{}{miasto w województwie wielkopolskim, w powiecie złotowskim, siedziba gminy miejsko-wiejskiej Jastrowie}
\Clue{18}{}{aparat telegraficzny o zapisie drukowym umożliwiający jednoczesne przekazywanie kilku informacji}
\Clue{19}{}{substancja pomocnicza w szczepionce}
\Clue{20}{}{młyn rozbijający; urządzenie do mielenia masy celulozowej}
\Clue{21}{}{Pliosaurus - rodzaj dużego drapieżnego plezjozaura, należącego do rodziny Pliosauridae, która wzięła od niego swą nazwę; żył od końca środkowej po połowę późnej jury (kelowej-kimeryd, około 165-151 milionów lat temu)}
\Clue{22}{}{marka samochodów ciężarowych i autobusów; polskie przedsiębiorstwo motoryzacyjne z siedzibą w Jelczu-Laskowicach}
\Clue{23}{}{motyw dekoracyjny w kształcie czaszki byka ozdobionej wstęgami girlandami}
\Clue{24}{}{miasto w zach. Kubie; przemysł spożywczy}
\Clue{25}{}{Ammosaurus - rodzaj zauropodomorfa z rodziny anchizaurów; żył w okresie wczesnej jury na terenach Ameryki Północnej}
\Clue{26}{}{córka młynarza}
\Clue{27}{}{twórczość Heraklita, zbiór jego myśli i poglądów}
\Clue{28}{}{(zęba) część zęba koła zębatego zawarta między powierzchnią podziałową i powierzchnią podstaw}
\Clue{29}{}{jeden ze znaków zodiaku}
\Clue{30}{}{coś lub ktoś oznaczone jedynką, noszące taki numer}
\Clue{31}{}{łac. DAMA DAMA; ssak z rodziny jeleniowatych, brązowy w białe cętki, łowny}
\Clue{33}{}{kod ISO 4217 lita}\end{PuzzleClues}\newpage\section*{Krzyżówka 115}

\noindent\begin{Puzzle}{20}{25}|*	|*	|*	|*	|*	|*	|*	|*	|*	|*	|*	|*	|*	|[1][S]\darr	|*	|*	|*	|*	|[2][S]\drarr	|f	|*	|.
|*	|*	|[3][S]\darr	|*	|*	|[4][S]\darr	|*	|*	|*	|[5][S]\darr	|[6][S]\darr	|*	|*	|i	|*	|[7][S]\darr	|*	|*	|s	|[8][S]\darr	|*	|.
|*	|*	|k	|*	|*	|s	|*	|[9][S]\darr	|*	|p	|w	|*	|[10][S]\darr	|w	|*	|s	|*	|*	|c	|h	|*	|.
|[11][S]\rarr	|t	|r	|a	|c	|z	|*	|w	|*	|o	|a	|*	|ż	|a	|*	|p	|*	|[12][S]\darr	|y	|e	|*	|.
|*	|*	|a	|*	|*	|u	|[13][S]\darr	|y	|*	|r	|n	|*	|e	|s	|*	|ó	|*	|e	|n	|ł	|*	|.
|*	|*	|i	|*	|*	|m	|p	|t	|[14][S]\darr	|t	|i	|*	|l	|i	|*	|j	|*	|g	|k	|m	|*	|.
|*	|*	|n	|*	|[15][S]\drarr	|s	|a	|r	|d	|e	|l	|k	|a	|*	|*	|n	|*	|o	|[][,]{ }	|[][,]{ }	|*	|.
|*	|*	|a	|[16][S]\darr	|z	|k	|s	|z	|z	|r	|i	|[17][S]\darr	|z	|*	|[18][S]\darr	|i	|*	|c	|n	|k	|[19][S]\darr	|.
|*	|*	|*	|s	|r	|*	|a	|e	|i	|*	|a	|n	|i	|*	|a	|k	|*	|e	|i	|o	|z	|.
|*	|*	|[20][S]\darr	|ł	|*	|*	|ż	|s	|a	|*	|*	|a	|c	|*	|l	|[][,]{ }	|*	|n	|e	|r	|a	|.
|*	|[21][S]\drarr	|w	|o	|d	|a	|*	|z	|d	|*	|*	|f	|a	|*	|m	|ł	|[22][S]\darr	|t	|b	|y	|j	|.
|*	|p	|i	|w	|[23][S]\darr	|*	|*	|c	|e	|*	|[24][S]\darr	|a	|*	|*	|a	|ą	|h	|r	|i	|n	|ą	|.
|*	|a	|z	|a	|c	|*	|*	|z	|k	|*	|g	|z	|[25][S]\darr	|*	|[][,]{ }	|c	|a	|y	|e	|c	|c	|.
|[26][S]\rarr	|r	|y	|c	|h	|l	|i	|k	|*	|*	|n	|o	|k	|[27][S]\darr	|m	|z	|j	|c	|s	|k	|e	|.
|*	|t	|t	|k	|ó	|[28][S]\rarr	|ł	|a	|n	|[][,]{ }	|f	|l	|a	|m	|a	|n	|d	|z	|k	|i	|*	|.
|*	|i	|ó	|i	|r	|*	|*	|[][,]{ }	|*	|*	|*	|i	|t	|u	|t	|y	|u	|n	|o	|*	|*	|.
|*	|a	|w	|*	|*	|[29][S]\rarr	|s	|t	|a	|n	|*	|n	|o	|n	|e	|*	|k	|o	|o	|*	|[30][S]\darr	|.
|*	|[][,]{ }	|k	|*	|*	|[31][S]\drarr	|s	|a	|l	|a	|m	|a	|n	|d	|r	|a	|*	|ś	|g	|[32][S]\darr	|k	|.
|*	|w	|a	|*	|[33][S]\drarr	|t	|a	|r	|t	|a	|k	|*	|*	|u	|*	|*	|*	|ć	|o	|h	|o	|.
|*	|ł	|*	|*	|r	|r	|*	|c	|*	|[34][S]\rarr	|m	|ę	|d	|r	|z	|e	|c	|*	|n	|i	|n	|.
|*	|o	|*	|*	|ó	|a	|[35][S]\rarr	|z	|r	|a	|z	|*	|*	|*	|*	|*	|*	|[36][S]\darr	|o	|b	|n	|.
|*	|s	|*	|*	|w	|p	|*	|ó	|*	|*	|*	|*	|*	|*	|*	|*	|*	|s	|w	|e	|i	|.
|*	|k	|*	|*	|*	|*	|[37][S]\rarr	|w	|y	|k	|a	|r	|c	|z	|a	|k	|*	|b	|y	|r	|c	|.
|*	|a	|[38][S]\rarr	|g	|r	|u	|d	|k	|a	|[][,]{ }	|c	|h	|ł	|o	|n	|n	|a	|*	|*	|n	|a	|.
|*	|*	|*	|[39][S]\rarr	|h	|u	|b	|a	|[][,]{ }	|l	|a	|k	|i	|e	|r	|o	|w	|a	|n	|a	|*	|.
|[40][S]\rarr	|k	|o	|ż	|u	|c	|h	|*	|*	|*	|*	|*	|*	|*	|*	|*	|*	|*	|*	|*	|*	|.\end{Puzzle}

\newpage

\begin{PuzzleClues}{\textbf{Poziome}\\}\Clue{2}{}{częsty symbol franka (waluty)}
\Clue{11}{}{ptak z rzędu blaszkodziobych o zmiennym upierzeniu i wąskim dziobie, rybożerny, doskonale nurkuje, chroniony}
\Clue{15}{}{Clupeonella engrauliformis - ryba śledziokształtna z rodziny śledziowatych (Clupeidae)}
\Clue{21}{}{woda pitna}
\Clue{26}{}{wczesna odmiana owsa}
\Clue{28}{}{dawna jednostka powierzchni odpowiadająca od 16,7 do 17,5 ha}
\Clue{29}{}{talia, wcięcie w pasie, kibić}
\Clue{31}{}{płaz ogoniasty o wydłużonym ciele, dość dużej, spłaszczonej głowie i zwykle jaskrawym ubarwieniu}
\Clue{33}{}{zakład, w którym dokonuje się przerobu drewna okrągłego na tarcicę w procesie technologicznym tarcia (inaczej: przecierania), czyli rozpiłowywania za pomocą urządzenia zwanego trakiem}
\Clue{34}{}{osoba mądra, posiadająca otwarty umysł i dużą wiedzę - humanistyczną, filozoficzną (nie praktyczną, techniczną)}
\Clue{35}{}{środkowy odcinek pędu z kilkoma pączkami szczepiony na podkładce w celu otrzymania szczepu}
\Clue{37}{}{chrząszcz z rodziny kózkowatych}
\Clue{38}{}{kuliste skupisko tkanki limfoidalnej (tkanki łącznej włóknistej i tkanki łącznej siateczkowatej) o wielkości 0,5-1,0 mm}
\Clue{39}{}{Ganoderma lucidum - grzyb z rodziny lakierowanych o lśniącej powierzchni}
\Clue{40}{}{powłoka roślinna tworząca się zwykle na zbiornikach wodnych lub terenach podmokłych}\end{PuzzleClues}

\begin{PuzzleClues}{\textbf{Pionowe}\\}\Clue{1}{}{jadalna sardynka pacyficzna ze śledziwatych}
\Clue{2}{}{Trachylepis quinquetaeniata - gatunek gada z rodziny scynkowatych, występujący we wschodniej i północnej Afryce}
\Clue{3}{}{jeden z regionów historyczno-etnograficznych tworzących dzisiejszą Słowenię}
\Clue{4}{}{miasto na Ukrainie w obwodzie tarnopolskim, siedziba rejonu szumskiego, do 1945 w Polsce, w województwie wołyńskim, w powiecie krzemienieckim, siedziba gminy Szumsk}
\Clue{5}{}{George ur. w 1920 r., chemik brytyjski, współtwórca chemii stanów wzbudzonych, laureat nagrody Nobla}
\Clue{6}{}{Vanilla - rodzaj roślin jednoliściennych z rodziny storczykowatych (Orchidaceae)}
\Clue{7}{}{spójnik występujący w zdaniach prostych lub złożonych, gdy człony zdania są równorzędne, np. i, oraz, tudzież}
\Clue{8}{}{rodzaj dopasowanego hełmu zakrywającego całą głowę}
\Clue{9}{}{Astata boops - gatunek owada z rodziny grzebaczowatych}
\Clue{10}{}{siderosis - pylica, która powstaje wskutek wdychania przez długi okres czasu tlenków żelaza}
\Clue{12}{}{cecha człowieka, który skupia swoje myśli i działania na sobie, wywyższa się, przejawia egocentryzm}
\Clue{13}{}{przejście na poziomie chodnika, łączące budynki lub ulice, przykryte dachem, często szklanym. Mieszczą się w nim wejścia do przyległych sklepów}
\Clue{14}{}{staruszek}
\Clue{15}{}{w chemii: symbol cyrkonu}
\Clue{16}{}{przedmiot szkolny lub uczony w ramach kursu, na którym opanowuje się podstawy języka słowackiego}
\Clue{17}{}{organiczny związek chemiczny zbudowany z reszt imidazoliny i naftalenu połączonych mostkiem metylenowym; lek o działaniu sympatykomimetycznym}
\Clue{18}{}{podniosłe określenie szkoły wyższej, najczęściej uniwersytetu}
\Clue{19}{}{zającowate, Leporidae - rodzina zajęczaków obejmująca około 60 gatunków; zamieszkują wszystkie kontynenty poza Antarktydą}
\Clue{20}{}{to, przez co ktoś jest w jakiś sposób postrzegany przez otoczenie (ze względu na swoje cechy charakteru lub fizyczne)}
\Clue{21}{}{otwarcie szachowe, które charakteryzuje się posunięciami: 1. e4 e5, 2. Sf3 Sc6, 3.Gc4 Gc5}
\Clue{22}{}{melodia do tańczenia hajduka}
\Clue{23}{}{struna bądź pewna liczba jednobrzmiących strun, najczęściej para, rzadziej trójka}
\Clue{24}{}{kod ISO 4217 franka gwinejskiego}
\Clue{25}{}{człowiek przypominający charakterem Katona - bardzo surowy, odznaczający się bezkompromisowością, wymagający od innych i od siebie, wierny prawu}
\Clue{27}{}{uniform; przepisowy, ustalony z góry ubiór żołnierza}
\Clue{30}{}{KAWALERIA, JAZDA}
\Clue{31}{}{pomost ze statku na ląd lub schodki pomiędzy pokładami}
\Clue{32}{}{zimowe leże wojska}
\Clue{33}{}{polowa fortyfikacja ziemna, obronna lub oblężnicza, w postaci wykopu o głębokości chroniącej przed ostrzałem nieprzyjaciela na wprost, redukująca przy okazji skutki ostrzału od góry i bombardowań, usytuowana na froncie walk, służąca głównie do prowadzenia ognia z broni osobistej i zespołowej oraz obserwacji przedpola}
\Clue{36}{}{w chemii: symbol antymonu}\end{PuzzleClues}\newpage\section*{Krzyżówka 116}

\noindent\begin{Puzzle}{20}{30}|*	|*	|*	|*	|*	|*	|[1][S]\drarr	|z	|a	|b	|i	|e	|g	|*	|[2][S]\drarr	|s	|z	|l	|i	|f	|*	|.
|*	|*	|*	|[3][S]\rarr	|k	|o	|p	|r	|o	|w	|i	|n	|a	|*	|m	|*	|*	|*	|*	|*	|*	|.
|*	|*	|*	|*	|*	|*	|s	|[4][S]\rarr	|z	|a	|k	|o	|n	|n	|i	|c	|z	|e	|k	|*	|*	|.
|*	|*	|[5][S]\rarr	|m	|o	|c	|z	|a	|r	|*	|[6][S]\rarr	|a	|s	|e	|k	|u	|r	|a	|n	|t	|*	|.
|*	|*	|*	|*	|*	|*	|c	|*	|[7][S]\rarr	|t	|e	|l	|l	|u	|r	|i	|u	|m	|*	|*	|*	|.
|*	|*	|*	|[8][S]\rarr	|j	|e	|z	|i	|o	|r	|o	|[][,]{ }	|f	|i	|o	|r	|d	|o	|w	|e	|*	|.
|*	|*	|*	|*	|[9][S]\rarr	|j	|o	|r	|d	|a	|ń	|c	|z	|y	|k	|*	|[10][S]\darr	|[11][S]\darr	|*	|[12][S]\darr	|*	|.
|*	|*	|[13][S]\rarr	|c	|i	|o	|ł	|k	|o	|w	|s	|k	|l	|*	|o	|*	|b	|n	|*	|t	|*	|.
|*	|*	|[14][S]\rarr	|p	|e	|t	|a	|r	|k	|a	|*	|*	|*	|*	|m	|*	|u	|a	|*	|a	|*	|.
|*	|[15][S]\rarr	|c	|i	|ą	|g	|[][,]{ }	|d	|a	|l	|s	|z	|y	|*	|ó	|*	|r	|r	|*	|o	|*	|.
|*	|*	|[16][S]\drarr	|s	|m	|u	|k	|l	|i	|k	|*	|[17][S]\rarr	|t	|u	|r	|c	|z	|y	|n	|*	|*	|.
|*	|*	|t	|[18][S]\rarr	|e	|l	|a	|s	|t	|y	|k	|*	|*	|[19][S]\darr	|k	|*	|y	|*	|*	|[20][S]\darr	|*	|.
|*	|*	|e	|*	|*	|*	|r	|[21][S]\darr	|*	|*	|[22][S]\darr	|*	|*	|m	|a	|[23][S]\darr	|k	|[24][S]\darr	|*	|m	|*	|.
|*	|*	|o	|*	|*	|*	|ł	|p	|[25][S]\drarr	|g	|o	|b	|b	|i	|*	|p	|[][,]{ }	|r	|[26][S]\darr	|x	|*	|.
|*	|*	|r	|*	|[27][S]\darr	|*	|o	|ł	|b	|[28][S]\darr	|a	|*	|[29][S]\darr	|n	|[30][S]\darr	|ł	|k	|e	|w	|n	|*	|.
|*	|*	|i	|[31][S]\darr	|a	|[32][S]\drarr	|w	|y	|r	|a	|z	|[][,]{ }	|m	|o	|d	|a	|l	|n	|y	|*	|*	|.
|*	|*	|a	|b	|p	|p	|a	|w	|z	|l	|a	|[33][S]\darr	|a	|t	|e	|w	|i	|*	|r	|*	|*	|.
|*	|[34][S]\darr	|[][,]{ }	|i	|p	|e	|t	|a	|o	|e	|*	|a	|l	|o	|l	|k	|n	|*	|z	|*	|*	|.
|*	|h	|d	|a	|e	|s	|a	|n	|z	|p	|[35][S]\darr	|p	|m	|r	|f	|a	|o	|*	|e	|*	|*	|.
|*	|a	|o	|ł	|l	|*	|*	|k	|a	|a	|w	|r	|*	|p	|i	|*	|s	|*	|c	|*	|*	|.
|*	|g	|w	|y	|l	|*	|*	|a	|[][,]{ }	|[][,]{ }	|ą	|e	|*	|e	|n	|*	|t	|[36][S]\darr	|z	|*	|*	|.
|*	|i	|o	|[][,]{ }	|a	|*	|*	|*	|g	|k	|s	|t	|*	|d	|*	|*	|e	|a	|e	|*	|*	|.
|*	|o	|d	|s	|t	|[37][S]\rarr	|m	|a	|r	|a	|k	|u	|j	|a	|*	|*	|r	|n	|n	|*	|*	|.
|[38][S]\drarr	|g	|u	|p	|i	|k	|*	|*	|y	|l	|o	|r	|*	|*	|*	|*	|n	|d	|i	|*	|*	|.
|e	|r	|*	|o	|v	|*	|*	|*	|ż	|a	|ś	|a	|*	|*	|*	|*	|y	|e	|e	|*	|*	|.
|l	|a	|[39][S]\rarr	|r	|u	|n	|w	|a	|y	|*	|ć	|*	|[40][S]\rarr	|m	|p	|a	|*	|r	|*	|*	|*	|.
|e	|f	|*	|t	|m	|*	|*	|*	|ń	|*	|*	|*	|*	|*	|[41][S]\rarr	|m	|o	|s	|t	|*	|*	|.
|a	|*	|*	|*	|*	|*	|*	|[42][S]\rarr	|s	|k	|ł	|a	|d	|n	|o	|ś	|ć	|*	|*	|*	|*	|.
|t	|*	|*	|[43][S]\rarr	|p	|r	|z	|y	|k	|r	|y	|c	|i	|e	|*	|*	|*	|*	|*	|*	|*	|.
|a	|*	|*	|*	|*	|*	|[44][S]\rarr	|ż	|a	|r	|n	|o	|w	|i	|e	|c	|*	|*	|*	|*	|*	|.
|*	|[45][S]\rarr	|s	|o	|f	|c	|i	|k	|*	|*	|*	|*	|*	|*	|*	|*	|*	|*	|*	|*	|*	|.\end{Puzzle}

\newpage

\begin{PuzzleClues}{\textbf{Poziome}\\}\Clue{1}{}{każde działanie, które jest podejmowane dla osiągnięcia określonego celu}
\Clue{2}{}{zgład}
\Clue{3}{}{drobna moneta miedziana}
\Clue{4}{}{hodowlana rasa gołębia lotnego}
\Clue{5}{}{moczary, podmokły teren}
\Clue{6}{}{ktoś asekurancki w zachowaniu, ostrożny}
\Clue{7}{}{model układu Słońce-Ziemia-Księżyc, służący do demonstrowania zjawisk pór dnia, roku, faz Księżyca, itp}
\Clue{8}{}{jezioro powstałe w głębokiej polodowcowej dolinie, zagrodzonej przez morenę czołową lodowca}
\Clue{9}{}{mieszkaniec Jordanii, człowiek pochodzenia jordańskiego}
\Clue{13}{}{rosyjski uczony i wynalazca (1857-1935), syn polskiego zesłańca, twórca podstaw astronautyki}
\Clue{14}{}{narciarz słoweński, czołowy skoczek świata, zwycięzca Turnieju Czterech Skoczni w l997 r}
\Clue{15}{}{następna część, kontynuowanie}
\Clue{16}{}{Halictus - rodzaj pszczoły z rodziny smuklikowatych (Halictidae) i podrodziny smuklikowatych właściwych (Halictinae); zakłada gniazda w ziemi, a także zapyla rośliny uprawne}
\Clue{17}{}{tytoń fajkowy z Turcji, uznawany za produkt dobrej jakości}
\Clue{18}{}{strój z elastycznej tkaniny sztucznej}
\Clue{25}{}{śpiewak włoski (1913-1984); baryton, solista La Scali}
\Clue{32}{}{modulant - wyraz określający stosunek mówiącego do wyrażanej treści}
\Clue{37}{}{smaczny owoc (jagoda) męczennicy jadalnej}
\Clue{38}{}{PAWIE OCZKO; ryba słodkowodna z żyworódek hodowana też w akwariach}
\Clue{39}{}{betonowy pas startowy na lotnisku}
\Clue{40}{}{jednostka ciśnienia, która jest równa milionowi paskali}
\Clue{41}{}{zespół elementów nośnych i mechanizmów napędowych samochodu; doprowadza napęd z wału do kół}
\Clue{42}{}{sprawność, sensowność, skuteczność, dobra organizacja}
\Clue{43}{}{coś, czym się przykrywa}
\Clue{44}{}{krzew z motylkowatych o złocistych kwiatach, w Polsce nad Bałtykiem}
\Clue{45}{}{odzież softshell}\end{PuzzleClues}

\begin{PuzzleClues}{\textbf{Pionowe}\\}\Clue{1}{}{pszczoła wschodnioazjatycka, Apis florea - gatunek z rodzaju Apis (pszczoła), występujący w Azji Południowej, w strefie klimatu tropikalnego}
\Clue{2}{}{obszar gęsto zaludniony w promieniu od kilkuset metrów do kilku kilometrów, w którym dominacje sygnał radiowy emitowany przez jedną stację przekaźnikową}
\Clue{10}{}{Puffinus pacificus - gatunek ptaka z rodziny burzykowatych (Procellariidae)}
\Clue{11}{}{prycza - prymitywne łóżko zbudowane najczęściej z żerdzi lub desek}
\Clue{12}{}{w taoizmie: zasada leżąca u podstawy wszechświata, rządząca przyrodą wraz ze społeczeństwem jako jej częścią}
\Clue{16}{}{dział logiki matematycznej zajmujący się analizą pojęcia dowodu oraz możliwych sposobów używania go w rozważaniach matematycznych}
\Clue{19}{}{podwodne, samowyzwalające urządzenie z ładunkiem wybuchowym}
\Clue{20}{}{kod ISO 4217 peso meksykańskiego}
\Clue{21}{}{spływ towarów statkami; wyraz używany w etnografii, w środowisku marynarskim oraz jako regionalizm}
\Clue{22}{}{teren o bardzo bujnej roślinności na obszarze pustyń i półpustyń}
\Clue{23}{}{niewielka boja - urządzenie pływające zakotwiczone za pomocą martwej kotwicy lub swobodnie dryfujące}
\Clue{24}{}{RENIFER; ssak z rodziny jeleniowatych o rozgałęzionym porożu, udomowiony, ogólnoużytkowy}
\Clue{25}{}{olbrzymia brzoza rosnąca do 1875 roku we wsi Gryżyna w pobliżu Kościana, znana z podań ludowych}
\Clue{26}{}{trudność, konieczność odmówienia sobie czegoś}
\Clue{27}{}{nazwa mogąca się odnosić do dowolnego egzemplarza desygnatów danej klasy przedmiotów}
\Clue{28}{}{gatunek ryby z rodziny ostrobokowatych (Carangidae)}
\Clue{29}{}{formacja geologiczna z oddziału o tej samej nazwie}
\Clue{30}{}{motyw drapieżnego ssaka morskiego; rodzaj ornamentu, ozdoby, elementu w formie delfina}
\Clue{31}{}{tenis, sport dżentelmenów; kiedyś uprawiany głównie przez ludzi białych}
\Clue{32}{}{znak notacji chorałowej}
\Clue{33}{}{substancja służąca do uszlachetniania, wykończenia materiału, nadająca mu nowych lub lepszych cech i właściwości podnoszących jego właściwości użytkowe}
\Clue{34}{}{przenośnie o autorze wyidealizowanej biografii}
\Clue{35}{}{cecha ubrania, które leży blisko ciała}
\Clue{36}{}{astronauta amerykański, pierwszy lot wokół Księżyca na Apollo 8}
\Clue{38}{}{filozof, przedstawiciel szkoły eleatów; wierzył w jednorodność materii i odrzucał istnienie czasu i ruchu}\end{PuzzleClues}\newpage\section*{Krzyżówka 117}

\noindent\begin{Puzzle}{25}{26}|*	|*	|*	|*	|*	|*	|*	|*	|*	|*	|*	|*	|*	|*	|*	|*	|*	|*	|*	|*	|*	|*	|*	|*	|[1][S]\darr	|*	|.
|*	|*	|*	|*	|*	|*	|*	|*	|*	|*	|*	|*	|*	|*	|*	|*	|*	|*	|*	|*	|*	|*	|[2][S]\darr	|*	|f	|*	|.
|*	|*	|*	|*	|*	|*	|*	|*	|*	|*	|*	|*	|*	|*	|[3][S]\rarr	|a	|g	|e	|r	|a	|t	|u	|m	|*	|r	|*	|.
|*	|*	|*	|*	|*	|[4][S]\darr	|*	|*	|*	|*	|*	|*	|[5][S]\rarr	|p	|r	|z	|y	|m	|i	|o	|t	|n	|o	|*	|e	|*	|.
|*	|*	|*	|[6][S]\darr	|*	|d	|[7][S]\darr	|*	|*	|*	|*	|*	|*	|*	|*	|*	|*	|*	|[8][S]\rarr	|s	|y	|t	|n	|i	|k	|*	|.
|[9][S]\rarr	|h	|i	|p	|h	|o	|p	|o	|w	|i	|e	|c	|*	|*	|*	|*	|*	|*	|*	|*	|*	|*	|t	|*	|w	|*	|.
|*	|*	|*	|r	|*	|j	|i	|*	|*	|*	|[10][S]\rarr	|t	|e	|r	|m	|i	|n	|o	|l	|o	|g	|i	|a	|*	|e	|*	|.
|*	|*	|*	|z	|*	|ś	|ł	|*	|*	|*	|*	|*	|*	|*	|*	|*	|[11][S]\darr	|*	|*	|*	|*	|*	|ż	|*	|n	|*	|.
|*	|*	|*	|e	|*	|c	|k	|*	|*	|[12][S]\drarr	|d	|i	|a	|b	|e	|l	|s	|k	|o	|ś	|ć	|*	|[][,]{ }	|*	|c	|*	|.
|*	|*	|*	|s	|*	|i	|a	|*	|[13][S]\drarr	|r	|[][S]2	|[][S]0	|*	|[14][S]\drarr	|k	|l	|i	|e	|n	|t	|*	|[15][S]\darr	|h	|[16][S]\darr	|j	|*	|.
|*	|*	|*	|z	|*	|e	|r	|*	|r	|o	|*	|*	|*	|s	|*	|*	|ł	|*	|*	|*	|*	|n	|o	|n	|a	|*	|.
|*	|*	|*	|c	|*	|*	|z	|*	|ó	|z	|[17][S]\rarr	|c	|z	|a	|r	|n	|a	|[][,]{ }	|d	|z	|i	|u	|r	|a	|*	|*	|.
|*	|*	|*	|z	|*	|*	|y	|*	|w	|t	|[18][S]\rarr	|s	|z	|l	|a	|m	|*	|*	|*	|*	|*	|n	|y	|m	|*	|*	|.
|*	|*	|*	|e	|*	|*	|k	|*	|n	|w	|*	|[19][S]\rarr	|z	|a	|c	|h	|ó	|d	|*	|*	|*	|e	|z	|u	|*	|*	|.
|*	|*	|*	|p	|*	|*	|i	|*	|a	|ó	|*	|[20][S]\rarr	|z	|m	|r	|o	|k	|*	|*	|*	|*	|z	|o	|r	|*	|*	|.
|*	|*	|*	|[][,]{ }	|[21][S]\darr	|[22][S]\darr	|*	|*	|n	|r	|*	|*	|*	|a	|[23][S]\darr	|*	|*	|*	|[24][S]\darr	|[25][S]\darr	|*	|*	|n	|n	|*	|*	|.
|*	|*	|*	|k	|a	|s	|[26][S]\darr	|*	|i	|[][,]{ }	|*	|*	|*	|n	|s	|[27][S]\darr	|*	|*	|p	|h	|*	|*	|t	|i	|*	|*	|.
|*	|[28][S]\rarr	|p	|o	|r	|t	|a	|m	|e	|n	|t	|o	|*	|d	|t	|o	|[29][S]\rarr	|w	|i	|e	|s	|z	|a	|k	|*	|*	|.
|*	|*	|[30][S]\drarr	|s	|t	|r	|u	|ś	|[][,]{ }	|a	|u	|s	|t	|r	|a	|l	|i	|j	|s	|k	|i	|*	|l	|[][,]{ }	|*	|*	|.
|*	|*	|k	|t	|[][,]{ }	|y	|t	|*	|l	|s	|*	|*	|*	|a	|r	|b	|*	|*	|a	|e	|*	|*	|n	|g	|*	|*	|.
|*	|*	|r	|n	|d	|j	|o	|*	|i	|y	|*	|*	|*	|*	|s	|r	|*	|*	|r	|l	|*	|*	|y	|ó	|*	|*	|.
|*	|*	|e	|y	|e	|e	|s	|*	|n	|c	|*	|*	|*	|*	|z	|z	|*	|*	|e	|e	|*	|*	|*	|r	|*	|*	|.
|*	|*	|p	|*	|c	|ń	|o	|*	|i	|o	|*	|*	|*	|*	|y	|y	|*	|*	|k	|*	|*	|*	|*	|s	|*	|*	|.
|*	|*	|e	|*	|o	|s	|m	|*	|o	|n	|*	|*	|*	|*	|z	|m	|*	|*	|*	|*	|*	|*	|*	|k	|*	|*	|.
|*	|*	|l	|*	|*	|k	|*	|*	|w	|y	|*	|*	|*	|*	|n	|*	|*	|*	|*	|*	|*	|*	|*	|i	|*	|*	|.
|*	|*	|*	|*	|*	|i	|*	|*	|e	|*	|[31][S]\rarr	|e	|x	|t	|a	|z	|y	|*	|*	|*	|*	|*	|*	|*	|*	|*	|.
|[32][S]\rarr	|s	|i	|m	|o	|*	|*	|*	|*	|*	|*	|*	|*	|*	|*	|*	|*	|*	|*	|*	|*	|*	|*	|*	|*	|*	|.\end{Puzzle}

\newpage

\begin{PuzzleClues}{\textbf{Poziome}\\}\Clue{3}{}{ŻENISZEK ameryk. roślina zielna lub krzew ze złożonych}
\Clue{5}{}{roślina zielna z rodziny złożonych, w Polsce pospolity chwast polny i roślina ruderalna}
\Clue{8}{}{saturator - aparat do saturacji, tj. sycenia cieczy gazem}
\Clue{9}{}{twórca i wykonawca muzyki hiphopowej}
\Clue{10}{}{słownictwo i terminy używane w danej dziedzinie wiedzy, branży}
\Clue{12}{}{to, że ktoś lub coś ma diabelski wygląd}
\Clue{13}{}{gruba i okrągła bateria o napięciu 1,5 V}
\Clue{14}{}{usługobiorca, nabywca}
\Clue{17}{}{obiekt fizyczny czasoprzestrzeni, który wszystko pochłania w wyniku silnego pola grawitacyjnego}
\Clue{18}{}{każdy osad lub brud, który zbiera się na dnie zbiornika wodnego lub pozostaje po wyschnięciu cieczy}
\Clue{19}{}{zachodnia część jakiegoś obszaru}
\Clue{20}{}{część doby, rozpoczynająca się z końcem dnia}
\Clue{28}{}{płynne przechodzenie z jednego dźwięku w dźwięk innej wysokości}
\Clue{29}{}{wysoka i chuda osoba}
\Clue{30}{}{inna nazwa kazuara}
\Clue{31}{}{twardy narkotyk rozpowszechniany zwykle w postaci pigułek}
\Clue{32}{}{jezioro w Finlandii, przepływa przez nie rzeka Simo}\end{PuzzleClues}

\begin{PuzzleClues}{\textbf{Pionowe}\\}\Clue{1}{}{najczęściej o grupie ludzi: określona liczebność, obecność w pewnej liczbie}
\Clue{2}{}{sposób montażu teleskopu, pozwalający obracać urządzeniem wzdłuż osi azymutu i wysokości}
\Clue{4}{}{o listach, przesyłkach: zostać dostarczonym}
\Clue{6}{}{zabieg chirurgiczny polegający na uzupełnieniu ubytków kości za pomocą granulatu}
\Clue{7}{}{gra w piłkarzyki - przedmiot}
\Clue{11}{}{pewien parametr, natężenie jakiegoś zjawiska}
\Clue{12}{}{roztwór, który w określonych warunkach termodynamicznych (ciśnienie, temperatura) nie zmienia swego stężenia w kontakcie z substancją rozpuszczoną}
\Clue{13}{}{równanie algebraiczne stopnia pierwszego}
\Clue{14}{}{płaz ogoniasty, w skórze gruczoły jadowe, Eurazja, Afryka, Ameryka Płd}
\Clue{15}{}{portugalski astronom ur. w 1492r. - kartograf}
\Clue{16}{}{Homalothecium philippeanum - gatunek mchu z rodziny krótkoszowatych}
\Clue{21}{}{styl w architekturze, malarstwie, grafice oraz w architekturze wnętrz, rozpowszechniony w latach 1919-1939, charakteryzujący się klasycyzującym zgeometryzowaniem, dążeniem do syntetycznego ujmowania form, poszukiwaniem piękna w funkcji przedmiotu użytkowego i jasności przekazu w grafice czy malarstwie, żywą i jasną kolorystyką}
\Clue{22}{}{Karol (1887-1932) architekt, grafik i publicysta profesor ASP w Krakowie}
\Clue{23}{}{członkowie Związku Harcerstwa Polskiego powyżej 21. roku życia, którzy złożyli przyrzeczenie harcerskie, ale nie pełnią funkcji instruktorskiej w harcerstwie}
\Clue{24}{}{pilot (1913-1942), uczestnik walk o Wielką Brytanię}
\Clue{25}{}{śląska potrawa sporządzana na zimno z wymoczonego i pokrojonego śledzia, jajka, ogórka kiszonego i cebuli}
\Clue{26}{}{każdy z chromosomów kariotypu za wyjątkiem chromosomów płci}
\Clue{27}{}{człowiek, który jest bardzo wysoki}
\Clue{30}{}{po śląsku: pączek}\end{PuzzleClues}\newpage\section*{Krzyżówka 118}

\noindent\begin{Puzzle}{22}{28}|*	|*	|*	|*	|*	|*	|*	|*	|[1][S]\darr	|*	|*	|*	|[2][S]\drarr	|o	|s	|i	|k	|a	|*	|[3][S]\drarr	|n	|a	|*	|.
|*	|*	|*	|*	|[4][S]\darr	|[5][S]\darr	|[6][S]\darr	|[7][S]\darr	|w	|[8][S]\drarr	|r	|y	|b	|o	|s	|o	|m	|*	|[9][S]\darr	|k	|[10][S]\darr	|*	|*	|.
|*	|*	|*	|*	|w	|n	|k	|k	|y	|t	|[11][S]\drarr	|k	|a	|l	|o	|n	|g	|*	|j	|i	|l	|*	|*	|.
|*	|*	|*	|[12][S]\darr	|c	|a	|u	|r	|d	|l	|a	|*	|k	|[13][S]\darr	|[14][S]\drarr	|b	|z	|d	|u	|r	|a	|*	|[15][S]\darr	|.
|*	|*	|[16][S]\drarr	|f	|i	|l	|l	|e	|r	|*	|u	|[17][S]\darr	|t	|l	|d	|*	|[18][S]\darr	|[19][S]\darr	|j	|y	|m	|*	|p	|.
|*	|*	|c	|a	|ą	|e	|t	|i	|a	|*	|*	|m	|r	|i	|*	|*	|a	|h	|u	|s	|p	|*	|ł	|.
|*	|*	|z	|ł	|g	|ś	|u	|s	|*	|[20][S]\drarr	|f	|a	|i	|r	|[][,]{ }	|p	|l	|a	|y	|*	|a	|*	|ó	|.
|*	|*	|a	|d	|a	|n	|r	|l	|*	|s	|*	|k	|a	|y	|[21][S]\darr	|*	|u	|d	|*	|*	|[][,]{ }	|*	|d	|.
|*	|*	|s	|o	|c	|i	|a	|e	|[22][S]\drarr	|t	|k	|a	|n	|k	|a	|[][,]{ }	|m	|i	|ę	|k	|k	|a	|*	|.
|*	|*	|[][,]{ }	|w	|z	|k	|*	|r	|k	|r	|[23][S]\darr	|k	|*	|a	|b	|*	|i	|s	|*	|*	|s	|*	|[24][S]\darr	|.
|*	|[25][S]\darr	|z	|n	|*	|*	|*	|*	|i	|u	|f	|[][,]{ }	|*	|*	|a	|[26][S]\darr	|n	|*	|*	|*	|e	|[27][S]\darr	|e	|.
|*	|f	|i	|i	|*	|*	|*	|*	|n	|ś	|l	|m	|*	|*	|j	|p	|o	|*	|*	|[28][S]\darr	|n	|ś	|k	|.
|*	|i	|m	|k	|[29][S]\rarr	|t	|i	|c	|o	|*	|o	|u	|*	|*	|a	|ó	|g	|*	|[30][S]\darr	|b	|o	|w	|s	|.
|*	|t	|o	|[][,]{ }	|*	|*	|*	|*	|t	|*	|t	|r	|*	|[31][S]\darr	|*	|ł	|r	|[32][S]\darr	|c	|e	|n	|[][S].	|f	|.
|*	|n	|w	|t	|[33][S]\drarr	|p	|r	|z	|e	|k	|a	|z	|*	|r	|[34][S]\darr	|c	|a	|b	|h	|t	|o	|[][,]{ }	|o	|.
|*	|e	|y	|r	|k	|*	|[35][S]\drarr	|r	|a	|j	|*	|y	|*	|a	|h	|z	|f	|a	|y	|o	|w	|a	|l	|.
|*	|s	|*	|z	|o	|*	|p	|*	|t	|*	|*	|n	|*	|m	|y	|ł	|i	|j	|b	|n	|a	|u	|i	|.
|*	|s	|*	|y	|r	|[36][S]\drarr	|o	|b	|r	|u	|s	|*	|*	|i	|d	|o	|a	|d	|i	|[][,]{ }	|*	|g	|a	|.
|*	|[][,]{ }	|*	|r	|b	|c	|l	|*	|*	|*	|*	|*	|*	|e	|r	|w	|*	|y	|ń	|g	|*	|u	|c	|.
|*	|k	|[37][S]\drarr	|z	|a	|s	|i	|a	|d	|k	|a	|*	|*	|n	|o	|i	|*	|k	|s	|i	|*	|s	|j	|.
|*	|l	|f	|ę	|*	|o	|s	|*	|*	|*	|*	|*	|*	|i	|m	|e	|*	|o	|k	|p	|[38][S]\darr	|t	|a	|.
|*	|u	|a	|d	|[39][S]\rarr	|k	|o	|ń	|[][,]{ }	|g	|o	|r	|ą	|c	|o	|k	|r	|w	|i	|s	|t	|y	|*	|.
|[40][S]\drarr	|b	|r	|o	|k	|*	|l	|*	|*	|*	|*	|*	|*	|a	|r	|*	|[41][S]\darr	|*	|*	|o	|r	|n	|*	|.
|s	|*	|o	|w	|[42][S]\rarr	|s	|o	|c	|z	|e	|w	|k	|a	|*	|f	|[43][S]\rarr	|p	|i	|g	|w	|a	|*	|*	|.
|i	|*	|s	|y	|*	|*	|k	|[44][S]\rarr	|m	|a	|g	|n	|e	|t	|o	|s	|t	|a	|t	|y	|k	|a	|*	|.
|c	|*	|*	|*	|[45][S]\rarr	|p	|a	|l	|n	|i	|k	|*	|*	|*	|n	|*	|a	|*	|*	|*	|t	|*	|*	|.
|z	|[46][S]\rarr	|p	|o	|d	|s	|t	|a	|w	|e	|k	|*	|*	|*	|*	|*	|k	|*	|*	|*	|*	|*	|*	|.
|*	|[47][S]\rarr	|j	|o	|d	|ł	|a	|[][,]{ }	|d	|e	|l	|a	|v	|e	|y	|a	|*	|*	|*	|*	|*	|*	|*	|.
|*	|[48][S]\rarr	|a	|m	|b	|a	|*	|*	|*	|*	|*	|*	|*	|*	|*	|*	|*	|*	|*	|*	|*	|*	|*	|.\end{Puzzle}

\newpage

\begin{PuzzleClues}{\textbf{Poziome}\\}\Clue{2}{}{topola osika, topola drżąca, osina, osiczyna - gatunek drzewa należący do rodziny wierzbowatych (Salicaceae); najpospolitszy gatunek topoli w Polsce}
\Clue{3}{}{w chemii: symbol sodu}
\Clue{8}{}{kompleks białek z kwasami nukleinowymi służący do produkcji białek w procesie translacji}
\Clue{11}{}{pies latający, Pteropus - rodzaj nietoperzy z rodziny rudawkowatych (Pteropodidae); występuje w strefie tropikalnej i subtropikalnej Australii, Afryki, Azji i Oceanii}
\Clue{14}{}{wypowiedź, coś, co nie ma sensu (może być kłamstwem, ale nie musi), niedorzeczność, głupstwo, brednia}
\Clue{16}{}{w ogrodnictwie: wcześnie owocujące drzewo owocowe sadzone między drzewami stałymi}
\Clue{20}{}{uczciwa rywalizacja prowadzona zgodnie z ustalonymi regułami}
\Clue{22}{}{tkanka, która łączy, wspiera lub otacza narządy organizmu, i nie jest tkanką kostną}
\Clue{29}{}{daewoo z modelu Tico}
\Clue{33}{}{przekazanie/przekazywanie czegoś, np. przekaz bodźców}
\Clue{35}{}{piękne miejsce, okoliczności}
\Clue{36}{}{tkanina lub dzianina używana do przykrycia stołu, zwykle na czas posiłków, pełniąca funkcję dekoracyjną i ochronną}
\Clue{37}{}{w żargonie łowieckim - rodzaj zasadzki, w której myśliwy poluje na zwierzynę z ukrycia}
\Clue{39}{}{koń o dużej sprawności fizycznej, który ma cechy konia sportowego, użytkowany najczęściej w sportach konnych i na wyścigach; jest energiczny, przystosowane do pracy w szybkim ruchu, ma żywy temperament i lekką budowę, co jest jest charakterystyczną cechą}
\Clue{40}{}{śrut strzelecki, nazwa drobnego śrutu używana kiedyś wśród myśliwych}
\Clue{42}{}{część oka, przezroczysty, elastyczny narząd ogniskujący promienie świetlne tak, by tworzyły obraz odwrócony i pomniejszony na siatkówce; znajduje się w przedniej części oka, między tęczówką a ciałem szklistym, jest umocowana na więzadełkach i mięśniach}
\Clue{43}{}{CYDONIA azjatycki krzew lub drzewko z rodziny różowatych, uprawiana dla aromatycznych owoców}
\Clue{44}{}{nauka o magnetyzmie,  która zajmuje się badaniem zjawisk magnetycznych oraz oddziaływań pól magnetycznych i prądów elektrycznych niezależnych od czasu}
\Clue{45}{}{urządzenie techniczne przeznaczone do wytwarzania wysokich temperatur poprzez spalanie gazu, względnie innego paliwa; wykorzystywany do spawania, cięcia, hartowania czy np. w paleniskach kotłów i pieców}
\Clue{46}{}{PONTICELLO}
\Clue{47}{}{Abies delavayi - gatunek z rodziny sosnowatych}
\Clue{48}{}{trudne położenie, sytuacja bez wyjścia}\end{PuzzleClues}

\begin{PuzzleClues}{\textbf{Pionowe}\\}\Clue{1}{}{wydra europejska, Lutra lutra - gatunek niewielkiego drapieżnego ssaka z rodziny łasicowatych (Mustelidae), jedyny żyjący w Polsce w stanie naturalnym przedstawiciel rodzaju Lutra}
\Clue{2}{}{wielbłąd dwugarbny: dziko żyje na Gobi}
\Clue{3}{}{zbroja, której korpus składająca się z dwóch części: ochraniającego pierś napierśnika i osłaniającego plecy naplecznika, połączonych za pomocą rzemieni w pasie i na ramionach}
\Clue{4}{}{mięsień służący do wciągania jakiejś części ciała, np. korpusu do muszli (u muszlowców)}
\Clue{5}{}{rodzaj ciasta z mleka lub wody, jajka i mąki smażony i podawany z serem lub konfiturami}
\Clue{6}{}{zbiór działalności człowieka związanych z rozwojem duchowym, najczęściej ze sztuką}
\Clue{7}{}{austriacki skrzypek wirtuoz i kompozytor (1875-1962); znane miniatury skrzypcowe}
\Clue{8}{}{w chemii: symbol talu}
\Clue{9}{}{miasto w Argentynie, w Andach, stolica prowincji Jujuy}
\Clue{10}{}{rodzaj lampy wyładowczej wykorzystywanej jako źródło światła w elektronicznych lampach błyskowych i stroboskopach; nazywany niekiedy błędnie żarnikiem lub palnikiem}
\Clue{11}{}{w chemii: symbol złota}
\Clue{12}{}{fałdownik szeleszczący, Rhytidiadelphus triquetrus - gatunek mchu z rodziny gajnikowatych; mech objęty w Polsce częściową ochroną}
\Clue{13}{}{rodzaj literacki obejmujacy utwory, których tematem są przede wszystkim przeżycia wewnętrzne, emocje i przekonania jednostki}
\Clue{14}{}{litera alfabetu używana w numeracji porządkowej}
\Clue{15}{}{w embriologii: zarodek ssaków od momentu, kiedy można rozpoznać cechy morfologiczne dla danego gatunku}
\Clue{16}{}{czas, który zostaje przesunięty na zimę o jedną godzinę w stosunku do czasu strefowego lub urzędowego}
\Clue{17}{}{makak czarny, małpo-pies, Macaca maura - małpa wąskonosa z rodziny makakowatych; gatunek endemiczny, występujący jedynie na wyspie Celebes w Indonezji}
\Clue{18}{}{obraz uzyskany przy użyciu techniki aluminografii}
\Clue{19}{}{czyn lub wypowiedź Mahometa bądź jego towarzyszy związane z życiem gminy muzułmańskiej}
\Clue{20}{}{wybitny lekarz epoki odrodzenia (1510-68); lekarz Zygmunta Augusta}
\Clue{21}{}{ABA - szeroki, luźny płaszcz bez rękawów noszony przez Beduinów, z wełny lub sierści koziej lub Wielbłądziej, haftowany złotem lub srebrem}
\Clue{22}{}{sala z miejscami dla widzów filmowych, budynek lub część budynku z takimi salami}
\Clue{23}{}{ogół okrętów wojennych danego państwa}
\Clue{24}{}{wypadanie zębów}
\Clue{25}{}{klub wyposażony w sprzęt do uprawiania fitnessu}
\Clue{26}{}{istota będąca fizycznie człowiekiem tylko w połowie}
\Clue{27}{}{twórczość Augustyna z Hippony, zbiór jego myśli i poglądów}
\Clue{28}{}{beton sporządzany z gipsu, kruszywa i wody}
\Clue{30}{}{muzykolog (1880-1952); badacz XVI i XVII w. muzyki polskiej}
\Clue{31}{}{chara, Chara - rodzaj glonów z rzędu ramienicowców}
\Clue{32}{}{pilot radziecki, wspólnie z Czkałowem i Bielakowem dokonał w 1937 r. przelotu nad biegunem północnym}
\Clue{33}{}{pilny uczeń, który dużo się uczy i ma dobre wyniki w nauce; kujon}
\Clue{34}{}{silny środek odurzający, mający krótsze działanie przeciwbólowe od morfiny; łatwo wywołuje euforię, ma właściwości uzależniające}
\Clue{35}{}{produkt bankowy łączący cechy polisy i lokaty, jest to w rzeczywistości ubezpieczenie, po wygaśnięciu którego klient otrzymuję wpłaconą kwotę plus dodatkowe oprocentowanie; umowę klient podpisać może w oddziale bankowym, ale pieniądze przekazuje towarzystwu ubezpieczeniowemu}
\Clue{36}{}{malarz węgierski (1865-1961 ) wpływ postimpresjonizmu}
\Clue{37}{}{wysepka z pierwszą latarnią morską; zaliczana do 7 cudów świata}
\Clue{38}{}{ciąg pomieszczeń, które zostały usytuowane wzdłuż jednej osi}
\Clue{40}{}{społeczność zorganizowana na sposób wojskowy, której początki sięgają XV w. i związane są z południowo-wschodnimi kresami Wielkiego Księstwa Litewskiego}
\Clue{41}{}{każdy przedstawiciel kręgowców stałocieplnych z grupy owodniowców}\end{PuzzleClues}\newpage\section*{Krzyżówka 119}

\noindent\begin{Puzzle}{24}{31}|*	|*	|*	|*	|*	|*	|*	|*	|[1][S]\drarr	|d	|e	|s	|k	|a	|[][,]{ }	|ś	|n	|i	|e	|ż	|n	|a	|*	|*	|*	|.
|*	|*	|[2][S]\rarr	|f	|l	|a	|s	|z	|o	|w	|i	|e	|c	|[][,]{ }	|g	|ł	|a	|d	|k	|i	|*	|*	|[3][S]\darr	|*	|*	|.
|[4][S]\rarr	|s	|a	|d	|ż	|d	|ż	|a	|d	|a	|*	|*	|[5][S]\drarr	|a	|k	|a	|n	|t	|*	|[6][S]\drarr	|k	|r	|w	|*	|[7][S]\darr	|.
|[8][S]\rarr	|k	|o	|n	|i	|a	|k	|ó	|w	|k	|a	|*	|d	|*	|*	|*	|*	|*	|*	|w	|*	|*	|i	|*	|o	|.
|[9][S]\rarr	|ź	|r	|e	|b	|i	|ę	|c	|i	|n	|a	|*	|i	|[10][S]\rarr	|d	|u	|c	|a	|l	|i	|a	|*	|t	|*	|r	|.
|*	|*	|*	|*	|[11][S]\drarr	|ś	|n	|i	|e	|g	|u	|ł	|a	|*	|[12][S]\rarr	|s	|t	|u	|h	|r	|*	|*	|t	|*	|g	|.
|*	|[13][S]\rarr	|k	|a	|w	|k	|a	|*	|r	|*	|*	|[14][S]\rarr	|b	|a	|ł	|t	|*	|*	|*	|[][,]{ }	|*	|[15][S]\darr	|e	|*	|a	|.
|*	|[16][S]\rarr	|p	|ł	|y	|t	|a	|*	|t	|*	|*	|*	|e	|[17][S]\rarr	|m	|e	|b	|l	|o	|w	|ó	|z	|*	|*	|n	|.
|[18][S]\rarr	|h	|i	|n	|d	|i	|j	|a	|*	|[19][S]\darr	|*	|*	|ł	|*	|[20][S]\darr	|*	|*	|[21][S]\darr	|*	|o	|[22][S]\darr	|b	|*	|[23][S]\darr	|i	|.
|[24][S]\rarr	|s	|p	|l	|a	|s	|h	|*	|[25][S]\darr	|d	|*	|*	|[][,]{ }	|*	|p	|*	|*	|a	|*	|d	|c	|i	|*	|w	|s	|.
|*	|*	|[26][S]\darr	|*	|w	|*	|*	|*	|o	|z	|*	|*	|m	|*	|r	|*	|*	|k	|[27][S]\darr	|n	|h	|ó	|*	|i	|t	|.
|[28][S]\rarr	|o	|g	|o	|n	|e	|k	|*	|r	|i	|*	|*	|o	|*	|a	|*	|*	|a	|l	|y	|o	|r	|*	|e	|k	|.
|*	|*	|a	|*	|i	|*	|*	|*	|g	|ó	|[29][S]\drarr	|f	|r	|y	|k	|a	|n	|d	|o	|*	|r	|k	|[30][S]\darr	|ż	|a	|.
|*	|*	|l	|*	|c	|*	|[31][S]\drarr	|ż	|a	|b	|a	|[][,]{ }	|s	|z	|t	|y	|l	|e	|t	|o	|w	|a	|t	|a	|*	|.
|*	|*	|o	|*	|t	|*	|ł	|*	|n	|k	|t	|[32][S]\darr	|k	|*	|y	|*	|*	|m	|o	|*	|a	|*	|c	|[][,]{ }	|*	|.
|*	|*	|n	|*	|w	|*	|u	|*	|i	|o	|a	|t	|i	|*	|k	|*	|*	|i	|s	|[33][S]\darr	|t	|*	|h	|c	|*	|.
|*	|*	|[][,]{ }	|*	|o	|*	|k	|*	|s	|w	|k	|a	|*	|*	|a	|*	|*	|k	|*	|o	|*	|*	|ó	|i	|*	|.
|*	|*	|a	|*	|[][,]{ }	|*	|[][,]{ }	|*	|t	|i	|*	|k	|*	|*	|*	|*	|*	|*	|*	|b	|*	|*	|r	|ś	|*	|.
|*	|*	|m	|*	|p	|*	|z	|[34][S]\rarr	|r	|e	|m	|i	|n	|i	|s	|c	|e	|n	|c	|j	|a	|*	|z	|n	|*	|.
|*	|*	|e	|*	|o	|*	|ę	|*	|u	|c	|*	|n	|*	|[35][S]\rarr	|s	|z	|t	|u	|k	|a	|s	|*	|l	|i	|[36][S]\darr	|.
|*	|*	|r	|*	|d	|*	|b	|*	|m	|[][,]{ }	|*	|*	|*	|*	|[37][S]\rarr	|k	|r	|o	|c	|z	|e	|*	|i	|e	|k	|.
|*	|*	|y	|*	|z	|*	|o	|[38][S]\darr	|*	|b	|[39][S]\rarr	|k	|o	|r	|d	|e	|g	|a	|r	|d	|a	|*	|w	|ń	|i	|.
|*	|*	|k	|*	|i	|*	|w	|m	|[40][S]\rarr	|r	|o	|t	|a	|*	|[41][S]\rarr	|w	|i	|l	|k	|*	|*	|[42][S]\darr	|o	|*	|e	|.
|*	|*	|a	|*	|e	|*	|y	|u	|[43][S]\rarr	|u	|i	|s	|t	|i	|t	|i	|*	|*	|*	|*	|*	|k	|ś	|*	|r	|.
|*	|*	|ń	|*	|m	|*	|*	|z	|[44][S]\rarr	|z	|e	|t	|t	|a	|f	|l	|o	|p	|s	|*	|*	|o	|ć	|*	|a	|.
|*	|*	|s	|*	|n	|*	|*	|y	|[45][S]\rarr	|d	|r	|e	|p	|a	|n	|o	|z	|a	|u	|r	|*	|r	|*	|*	|t	|.
|*	|*	|k	|*	|e	|*	|[46][S]\rarr	|c	|i	|o	|t	|e	|c	|z	|n	|y	|[][,]{ }	|d	|z	|i	|a	|d	|e	|k	|*	|.
|*	|*	|i	|*	|*	|*	|*	|z	|[47][S]\rarr	|w	|i	|k	|a	|r	|i	|u	|s	|z	|*	|*	|*	|*	|*	|*	|*	|.
|*	|*	|*	|*	|*	|*	|[48][S]\rarr	|k	|w	|a	|c	|z	|*	|*	|*	|*	|*	|*	|*	|*	|*	|*	|*	|*	|*	|.
|*	|*	|*	|*	|[49][S]\rarr	|k	|r	|a	|ś	|n	|i	|k	|[][,]{ }	|p	|u	|r	|p	|u	|r	|a	|c	|z	|e	|k	|*	|.
|*	|*	|[50][S]\rarr	|s	|e	|k	|s	|*	|*	|y	|*	|*	|*	|*	|*	|*	|*	|*	|*	|*	|*	|*	|*	|*	|*	|.
|[51][S]\rarr	|p	|o	|k	|o	|n	|a	|n	|y	|*	|*	|*	|*	|*	|*	|*	|*	|*	|*	|*	|*	|*	|*	|*	|*	|.\end{Puzzle}

\newpage

\begin{PuzzleClues}{\textbf{Poziome}\\}\Clue{1}{}{płat nieprzytwierdzonego do podłoża śniegu, który przesuwa się po oblodzonym podłożu; może stanowić sama lawinę śnieżną lub powodować poruszanie się innych kawałków śniegu, tworzących lawinę}
\Clue{2}{}{Annona glabra - roślina z rodziny flaszowcowatych}
\Clue{4}{}{muzułmański dywanik modlitewny}
\Clue{5}{}{motyw zdobniczy w formie stylizowanego liścia rośliny śródziemnomorskiej - akantu}
\Clue{6}{}{kod ISO 4217 wona południowokoreańskiego}
\Clue{8}{}{zawartość koniakówki, kieliszka do koniaku}
\Clue{9}{}{mięso ze źrebiąt, najczęściej koni}
\Clue{10}{}{w średniowieczu, obszary gospodarcze prawa do których przysługiwały jedynie panującemu}
\Clue{11}{}{biało-brązowy ptak z rzędu wróblowatych, do Polski zalatuje zimą; chroniona}
\Clue{12}{}{Jerzy Oskar Stuhr - polski aktor filmowy i teatralny, profesor sztuki oraz reżyser}
\Clue{13}{}{dawny instrument muzyczny}
\Clue{14}{}{członek grupy indoeuropejskich plemion nadbałtyckich}
\Clue{16}{}{(akumulatorowa) elektroda ogniwa akumulatorowego składająca się z masy i konstrukcji nośnej}
\Clue{17}{}{wóz meblowy - pojazd (samochód ciężarowy lub półciężarówka) służący do transportu mebli}
\Clue{18}{}{miasto w środkowym Iraku w dolinie Eufratu; w pobliżu duża zapora wodna}
\Clue{24}{}{rodzaj talerza perkusyjnego wykorzystywanego do krótkich (szybko wybrzmiewających) akcentów}
\Clue{28}{}{znak diakrytyczny służący do oznaczania głosek długich lub głosek nosowych (w zależności od języka)}
\Clue{29}{}{każda z części, na które można podzielić udziec, biorąc pod uwagę układ mięśni}
\Clue{31}{}{Odorrana hosii - gatunek płaza bezogonowego z rodziny żabowatych, występujący na górzystych terenach wyspy Okinawa w Japonii, posiadający szczególną broń: kciuki przednich kończyn przekształcone w długie, ostre i twarde skostniałe wyrostki, pokryte cienką skórą i skierowane do osi ciała, które gdy żaba zostanie uchwycona przez małe zwierzę lub wzięta do ręki przez człowieka, wbijają się w ciało napastnika, powodując krwawiące rany}
\Clue{34}{}{wspomnienie czegoś, refleksja nad czymś, zwykle pojawiające się po jakimś czasie}
\Clue{35}{}{niemiecki samolot bombowy z okresu II wojny światowej}
\Clue{37}{}{region pomiędzy obszarem genitalnym a odbytem u obu płci; uważany za jedną z intymnych części ciała}
\Clue{39}{}{budynek przeznaczony na wartownię dla straży wojskowej}
\Clue{40}{}{formuła przysięgi}
\Clue{41}{}{inna nazwa tocznia - przewlekłej choroby autoimmunologicznej}
\Clue{43}{}{matołka}
\Clue{44}{}{jednostka mocy obliczeniowej komputerów wynosząca10\textasciicircum21 flopsów}
\Clue{45}{}{Drepanosaurus - nazwa rodzajowa nadrzewnego diapsyda żyjącego w późnym triasie na terenach obecnej Europy i Ameryki Północnej; jego skamieniałości odkryto w północnych Włoszech oraz w Nowym Meksyku}
\Clue{46}{}{wujek matki lub ojca}
\Clue{47}{}{WIKARY}
\Clue{48}{}{rodzaj długiego pędzla; zwitek szmat lub pakuł, umocowany na długim kiju}
\Clue{49}{}{Zygaena purpuralis - motyl średniej wielkości z rodziny kraśnikowatych; czas lotu przypada na maj-sierpień, w zależności od miejsca}
\Clue{50}{}{zbliżenie dwóch osobników w celu prokreacyjnym lub dla osiągnięcia przyjemności}
\Clue{51}{}{uczestnik rywalizacji, który przegrał}\end{PuzzleClues}

\begin{PuzzleClues}{\textbf{Pionowe}\\}\Clue{1}{}{praca górnicza lub geologiczna polegająca na tworzeniu otworu wiertniczego}
\Clue{3}{}{w malarstwie i grafice; stopniowe natężenie barwy odpowiadające rozłożeniu światła na powierzchni np. obrazu czy przedstawionego na nim przedmiotu}
\Clue{5}{}{Lophius piscatorius - ryba morska z rodziny żabnicowatych (Lophiidae)}
\Clue{6}{}{szybki ruch rotacyjny cząsteczek płynu}
\Clue{7}{}{kobieta (muzyk, instrumentalistka), która umie grać na organach, zazwyczaj zna się też na budowie i konserwacji sprzętu}
\Clue{11}{}{np. druk, ulotka}
\Clue{15}{}{zbieranie pieniędzy lub rzeczy w określonym celu}
\Clue{19}{}{Eurhynchium striatum - gatunek mchu z rodziny krótkoszowatych; tworzy mocne, lśniące darnie wysokości 10-15 cm, łodygi drzewkowate, liście łodygowe sercowato-lancetowate z krótkim szczytem, fałdowane, o piłkowanym brzegu, puszka podłużna, brązowa, wieczko z długim, zgiętym dziobkiem, rośnie w lasach, na próchnicznej ziemi}
\Clue{20}{}{praca, zwłaszcza lekarza lub prawnika}
\Clue{21}{}{student}
\Clue{22}{}{mieszkaniec Chorwacji, człowiek pochodzenia chorwackiego}
\Clue{23}{}{budynek w formie wieży, na którego szczycie znajduje się zbiornik wody, służący do zapewnienia stabilnego ciśnienia w wodociągu}
\Clue{25}{}{lira korbowa, rodzaj katarynki}
\Clue{26}{}{galon amerykański dla płynów = 231 cali sześciennych = 3,785411784 litra}
\Clue{27}{}{kwiat lotosu - specyficzny rodzaj siadu skrzyżnego, w którym lewa stopa znajduje się na prawym udzie oraz prawa stopa na lewym udzie; pozycja stosowana przy medytacji, znana już w starożytnych Indiach}
\Clue{29}{}{szybka i nieprzewidywana napaść, najczęściej z użyciem siły}
\Clue{30}{}{cecha człowieka: to, że ktoś ma skłonność do tchórzostwa, brakuje mu odwagi i kieruje się strachem o samego siebie}
\Clue{31}{}{jeden z dwóch łuków - części uzębienia każdego człowieka}
\Clue{32}{}{Budorcas taxicolor - ssak z rodziny krętorogich, jedyny przedstawiciel rodzaju Budorca; zamieszkuje wschodnie Himalaje i Wyżynę Tybetańską, od 2500 do 4250 m n.p.m}
\Clue{33}{}{boczna, okrężna droga, często tymczasowa, dla ominięcia przeszkody}
\Clue{36}{}{urządzenie wykorzystujące siłę pociągową zwierząt (koni lub wołów) do napędu stacjonarnych maszyn rolniczych takich jak sieczkarnia, wialnia czy młocarnia, bądź do wydobywania wody}
\Clue{38}{}{nauczycielka muzyki}
\Clue{42}{}{rodzaj tkaniny sztucznej (z włókien szklanych, innych chemicznych, nawet drutów) używanej do wzmacniania opon}\end{PuzzleClues}\newpage\section*{Krzyżówka 120}

\noindent\begin{Puzzle}{23}{26}|*	|*	|*	|[1][S]\darr	|*	|*	|*	|*	|[2][S]\darr	|*	|*	|*	|*	|*	|[3][S]\darr	|*	|[4][S]\darr	|*	|*	|*	|*	|*	|*	|[5][S]\darr	|.
|*	|[6][S]\rarr	|w	|a	|p	|n	|i	|a	|k	|*	|[7][S]\darr	|*	|*	|[8][S]\darr	|z	|[9][S]\rarr	|d	|a	|n	|t	|e	|*	|*	|b	|.
|*	|[10][S]\rarr	|t	|r	|z	|y	|k	|r	|o	|t	|k	|a	|*	|w	|g	|*	|z	|*	|[11][S]\drarr	|d	|y	|n	|*	|a	|.
|*	|*	|*	|n	|*	|*	|*	|*	|r	|[12][S]\darr	|r	|*	|*	|y	|o	|[13][S]\darr	|i	|[14][S]\drarr	|p	|o	|l	|e	|*	|z	|.
|*	|*	|*	|i	|[15][S]\darr	|*	|*	|*	|a	|d	|ó	|*	|*	|g	|d	|f	|k	|p	|a	|*	|*	|*	|[16][S]\darr	|a	|.
|*	|*	|[17][S]\darr	|k	|r	|*	|*	|[18][S]\darr	|i	|o	|l	|[19][S]\darr	|*	|a	|n	|a	|i	|y	|r	|[20][S]\darr	|*	|*	|d	|[][,]{ }	|.
|*	|*	|r	|a	|ó	|[21][S]\darr	|*	|ś	|s	|g	|e	|j	|*	|s	|o	|r	|e	|ł	|a	|t	|*	|*	|r	|w	|.
|*	|*	|o	|*	|w	|a	|*	|w	|*	|m	|w	|a	|[22][S]\darr	|z	|ś	|a	|[][,]{ }	|e	|b	|r	|*	|*	|o	|i	|.
|*	|*	|t	|*	|[][,]{ }	|n	|*	|i	|[23][S]\darr	|a	|i	|r	|g	|a	|ć	|o	|m	|k	|o	|a	|[24][S]\darr	|*	|b	|e	|.
|*	|*	|t	|*	|o	|g	|*	|a	|a	|t	|ę	|z	|w	|c	|[][,]{ }	|n	|i	|[][,]{ }	|l	|n	|z	|*	|n	|d	|.
|*	|*	|w	|*	|c	|i	|[25][S]\drarr	|t	|r	|y	|t	|e	|i	|z	|m	|*	|ę	|k	|a	|s	|b	|*	|o	|z	|.
|*	|*	|e	|[26][S]\darr	|e	|e	|m	|ł	|y	|z	|a	|n	|a	|*	|o	|*	|s	|w	|[][,]{ }	|i	|r	|*	|ś	|y	|.
|*	|*	|i	|s	|a	|l	|a	|ó	|s	|m	|*	|i	|z	|[27][S]\drarr	|t	|r	|o	|i	|s	|t	|o	|ś	|ć	|*	|.
|*	|*	|l	|y	|n	|c	|j	|w	|t	|*	|[28][S]\darr	|e	|d	|l	|y	|*	|*	|a	|z	|*	|j	|[29][S]\darr	|*	|*	|.
|*	|*	|e	|c	|i	|z	|s	|k	|o	|*	|ś	|*	|o	|i	|w	|[30][S]\darr	|*	|t	|e	|[31][S]\darr	|a	|k	|*	|*	|.
|*	|[32][S]\rarr	|r	|e	|c	|y	|t	|a	|t	|y	|w	|*	|s	|c	|a	|b	|[33][S]\darr	|o	|ś	|o	|[][,]{ }	|o	|*	|[34][S]\darr	|.
|*	|*	|k	|n	|z	|k	|e	|[][,]{ }	|e	|*	|i	|*	|z	|a	|c	|o	|g	|w	|c	|m	|n	|n	|*	|t	|.
|*	|*	|a	|i	|n	|*	|r	|b	|l	|*	|e	|*	|e	|*	|j	|r	|l	|y	|i	|d	|i	|a	|*	|r	|.
|*	|*	|*	|e	|y	|*	|*	|a	|e	|*	|c	|*	|k	|*	|i	|s	|i	|*	|e	|l	|e	|j	|[35][S]\darr	|u	|.
|*	|*	|*	|*	|*	|*	|*	|b	|s	|*	|e	|*	|*	|*	|*	|o	|n	|*	|n	|e	|m	|ą	|h	|d	|.
|*	|[36][S]\rarr	|t	|y	|s	|i	|ą	|c	|*	|*	|*	|[37][S]\rarr	|w	|ł	|o	|s	|i	|e	|n	|n	|i	|c	|a	|*	|.
|*	|*	|*	|*	|[38][S]\rarr	|p	|r	|z	|e	|b	|r	|a	|n	|i	|e	|*	|a	|*	|a	|i	|e	|y	|k	|*	|.
|[39][S]\drarr	|ś	|m	|i	|e	|t	|k	|a	|[][,]{ }	|ć	|w	|i	|k	|l	|a	|n	|k	|a	|*	|e	|c	|*	|o	|*	|.
|l	|*	|*	|*	|[40][S]\rarr	|a	|k	|r	|e	|d	|y	|t	|a	|c	|j	|a	|*	|*	|*	|*	|k	|*	|n	|*	|.
|b	|*	|*	|*	|[41][S]\rarr	|p	|u	|k	|n	|i	|ę	|c	|i	|e	|*	|[42][S]\rarr	|w	|i	|l	|i	|a	|*	|o	|*	|.
|*	|[43][S]\rarr	|b	|l	|i	|n	|d	|a	|ż	|*	|*	|[44][S]\rarr	|k	|a	|p	|a	|n	|i	|n	|a	|*	|*	|s	|*	|.
|[45][S]\rarr	|c	|e	|d	|u	|ł	|a	|*	|*	|*	|*	|*	|*	|*	|*	|*	|*	|*	|*	|*	|*	|*	|*	|*	|.\end{Puzzle}

\newpage

\begin{PuzzleClues}{\textbf{Poziome}\\}\Clue{6}{}{lekceważąco lub pogardliwie o starszej osobie, rodzicu}
\Clue{9}{}{najwybitniejszy poeta włoski (1265-1321); „Boska komedia”, uważana na jedno z najwybitniejszych osiągnięć poetyckich}
\Clue{10}{}{tradeskancja - bylina z komelionowatych ze zwrotnikowej Ameryki}
\Clue{11}{}{jednostka siły, pochodna w układzie miar CGS; jest to siła nadająca ciału o masie 1 grama przyspieszenie równe 1 cm/s2}
\Clue{14}{}{wycinek struktury społecznej, wyodrębniany dla badań i obserwacji socjologicznych}
\Clue{25}{}{rel}
\Clue{27}{}{cecha czegoś, co występuje w trzech postaciach, rolach, funkcjach, niezależnych od siebie i czasem będących ze sobą w sprzeczności}
\Clue{32}{}{część utworu instrumentalnego, w której muzyka jest melodyczna  i rytmiczna (taka, aby można było w jej rytmie wykonywać śpiewną deklamację)}
\Clue{36}{}{gra karciana dla maksymalnie czterech osób, w której w każdej z kolejek zbiera się punkty za pary kart (meldunki) i przebijanie wartości kart przeciwnika}
\Clue{37}{}{WŁOSIANKA; szorstka tkanina z końskiego włosia}
\Clue{38}{}{kostium - strój ułatwiający odegranie jakieś roli, noszony w szczególnych sytuacjach, wymagających specjalnego przebrania, np. na scenie czy podczas balu przebierańców}
\Clue{39}{}{Pegomyja hyosciami, Phorbia hyoscyami - gatunek muchówki z rodziny śmietkowatych}
\Clue{40}{}{zgoda, którą otrzymuje dziennikarz na relacjonowanie jakiegoś wydarzenia}
\Clue{41}{}{pojedyncze uderzenie, stuknięcie, najczęściej w drzwi w celu oznajmienia przybycia}
\Clue{42}{}{przeddzień jakiegoś wydarzenia, zwykle ważnego}
\Clue{43}{}{opancerzenie okrętu wojennego}
\Clue{44}{}{to, co się daje w niewielkiej ilości co pewien czas, zwykle nieduża suma pieniędzy}
\Clue{45}{}{sporządzany przez obsługę pociągu raport dotyczący sprzedanych danego dnia biletów}\end{PuzzleClues}

\begin{PuzzleClues}{\textbf{Pionowe}\\}\Clue{1}{}{kupalnik górski; bylina z rodziny złożonych; liście w rozecie, chroniona}
\Clue{2}{}{(1748-1833), filolog i pisarz grecki, lekarz}
\Clue{3}{}{zgodność, która zachodzi pomiędzy uczestnikami danej konfiguracji zachowań, gdy żaden z nich nie będzie zyskiwał na odstąpieniu od tej konfiguracji, pod warunkiem, że pozostali uczestnicy też nie będą od niej odchodzili}
\Clue{4}{}{nadmiernie przerośnięta tkanka ziarninowa}
\Clue{5}{}{szczegółowy, rozległy zbiór powiązanych logicznie danych dotyczących danej dziedziny (obszaru tematycznego); przechowywany w pamięci komputera wraz z regułami logicznymi}
\Clue{7}{}{w publikacjach historycznych (zwłaszcza popularnonaukowych) określenie polskich magnatów pochodzenia ruskiego, litewskiego i polskiego posiadających rozległe majątki (latyfundia) na kresach wschodnich dawnej Rzeczypospolitej, mających nieograniczoną władzę, własne wojska, często prowadzących własną politykę zagraniczną, a także nierzadko skoligaconych z rodami panującymi}
\Clue{8}{}{program komputerowy, którego zadaniem jest zwiększenie czasu bezawaryjnej pracy monitora poprzez wyłączanie go lub wypełnianie nieustannie, często losowo, zmieniającymi się wzorami lub obrazami}
\Clue{11}{}{krzywa płaska opisana równanie, w którym y równe jest iloczynowi a i x do potęgi trzeciej, gdy a jest większe od zera}
\Clue{12}{}{stanowisko bezkrytycznego przyjmowania danych twierdzeń jako prawdy, bez weryfikacji i podawania w wątpliwość, na zasadzie bezkrytycznej wiary}
\Clue{13}{}{tytuł władcy w starożytnym Egipcie}
\Clue{14}{}{pyłek - męskie cząstki rozrodcze roślin nasiennych}
\Clue{15}{}{silnie wydłużone obniżenie dna oceanu o głębokości ponad 6000 m}
\Clue{16}{}{cecha czegoś, co jest drobne, mało istotne}
\Clue{17}{}{suka rasy rottweiler}
\Clue{18}{}{Hoplodrina ambigua - gatunek motyla z rodziny sówkowatych, rozpowszechniony, lata w lipcu}
\Clue{19}{}{delikatne świecenie}
\Clue{20}{}{ford z modelu Transit}
\Clue{21}{}{rasowy koń, wierzchowiec, koń rasy angielskiej}
\Clue{22}{}{przedstawiciel słodkowodnych kolonijnych glonów z gromady zielenic (Chlorophyta) z rodzaju Pediastrum}
\Clue{23}{}{filozof, jeden z trzech (obok Sokratesa i Platona) najsławniejszych filozofów starożytnej Grecji}
\Clue{24}{}{typ późnośredniowiecznej zbroi, której cechą charakterystyczną były ostre krawędzie oraz ozdobne, tłoczone pancerze}
\Clue{25}{}{rzemieślnik, który prowadzi samodzielnie warsztat i ma uczniów, osoba doświadczona w swoim fachu}
\Clue{26}{}{zaspokajanie głodu}
\Clue{27}{}{przewód nawojowy składający się z cienkich emaliowanych drucików, silnie skręconych}
\Clue{28}{}{łow. oczy większości dużych zwierząt}
\Clue{29}{}{ktoś, kto kona, z kogo uchodzi życie, kto jest w agonii}
\Clue{30}{}{malarz węgierski (1821-83) czynny w Wiedniu i Budapeszcie, portrety, sceny rodzajowe}
\Clue{31}{}{to, że ktoś zemdlał}
\Clue{33}{}{naczynie wykonane z gliny}
\Clue{34}{}{duży wysiłek, praca, których nakład jest potrzebny do zrobienia czegoś}
\Clue{35}{}{Callorhinchus callorhynchus - gatunek morskiej ryby chrzęstnoszkieletowej z rodziny hakonosowatych (Callorhinchidae); ryba ta jest poławiana dla smacznego mięsa}
\Clue{39}{}{pozaukładowa jednostka masy wywodząca się od rzymskiej libry, wynosząca na przestrzeni wieków w różnych państwach około 0,4-0,5 kilograma}\end{PuzzleClues}\newpage\section*{Krzyżówka 121}

\noindent\begin{Puzzle}{22}{28}|*	|*	|[1][S]\darr	|*	|[2][S]\drarr	|d	|z	|i	|d	|z	|i	|a	|[][S]-	|p	|i	|e	|r	|n	|i	|k	|*	|*	|*	|.
|*	|*	|m	|*	|s	|*	|[3][S]\drarr	|ś	|w	|i	|ę	|t	|e	|[][,]{ }	|k	|o	|l	|e	|g	|i	|u	|m	|*	|.
|[4][S]\drarr	|p	|e	|n	|t	|i	|m	|e	|n	|t	|o	|*	|*	|[5][S]\drarr	|t	|u	|p	|a	|j	|e	|*	|[6][S]\darr	|*	|.
|p	|*	|z	|[7][S]\darr	|a	|[8][S]\darr	|i	|[9][S]\rarr	|p	|o	|t	|a	|ń	|c	|ó	|w	|k	|a	|*	|[10][S]\darr	|*	|m	|*	|.
|u	|*	|a	|m	|r	|o	|n	|[11][S]\darr	|*	|*	|*	|*	|[12][S]\rarr	|h	|e	|ł	|m	|i	|a	|t	|k	|a	|*	|.
|z	|[13][S]\darr	|n	|a	|u	|b	|i	|r	|[14][S]\rarr	|f	|i	|l	|t	|r	|a	|c	|j	|a	|*	|a	|*	|s	|*	|.
|o	|k	|i	|r	|c	|c	|s	|e	|[15][S]\darr	|*	|*	|*	|[16][S]\darr	|z	|*	|*	|*	|*	|[17][S]\darr	|r	|*	|z	|*	|.
|n	|o	|n	|i	|h	|i	|t	|c	|p	|[18][S]\darr	|*	|[19][S]\darr	|p	|e	|[20][S]\darr	|[21][S]\darr	|*	|[22][S]\darr	|t	|k	|*	|y	|*	|.
|*	|l	|*	|o	|*	|ą	|r	|y	|i	|p	|*	|m	|a	|s	|k	|d	|[23][S]\darr	|ś	|r	|a	|*	|n	|*	|.
|[24][S]\drarr	|d	|e	|l	|e	|g	|a	|c	|j	|a	|[][,]{ }	|u	|s	|t	|a	|w	|o	|w	|a	|*	|*	|a	|*	|.
|k	|i	|*	|o	|*	|a	|n	|l	|a	|w	|[25][S]\darr	|s	|t	|*	|n	|u	|p	|i	|n	|*	|*	|[][,]{ }	|*	|.
|n	|n	|*	|g	|*	|c	|t	|i	|w	|ł	|a	|z	|i	|*	|g	|k	|ł	|e	|s	|*	|*	|ż	|*	|.
|y	|g	|[26][S]\darr	|i	|*	|z	|[][,]{ }	|n	|k	|o	|r	|t	|s	|[27][S]\darr	|u	|o	|a	|t	|c	|*	|*	|n	|*	|.
|p	|*	|m	|a	|*	|*	|ś	|g	|a	|w	|s	|r	|z	|b	|r	|n	|t	|l	|e	|*	|[28][S]\darr	|i	|*	|.
|*	|*	|o	|*	|*	|*	|w	|*	|*	|o	|z	|a	|*	|r	|[][,]{ }	|k	|a	|i	|n	|*	|n	|w	|*	|.
|[29][S]\rarr	|l	|a	|t	|a	|w	|i	|c	|a	|*	|y	|*	|*	|y	|c	|a	|[][,]{ }	|s	|d	|*	|o	|n	|*	|.
|*	|*	|[][,]{ }	|*	|[30][S]\rarr	|g	|a	|l	|t	|o	|n	|*	|[31][S]\darr	|k	|z	|*	|p	|t	|e	|*	|ś	|a	|*	|.
|*	|*	|m	|*	|*	|*	|t	|*	|*	|*	|*	|*	|c	|a	|e	|*	|r	|o	|n	|*	|n	|*	|*	|.
|[32][S]\drarr	|g	|u	|z	|[][,]{ }	|z	|ł	|o	|ś	|l	|i	|w	|y	|*	|r	|*	|o	|ś	|t	|*	|i	|*	|*	|.
|e	|[33][S]\rarr	|s	|t	|u	|k	|a	|w	|k	|a	|*	|*	|k	|*	|w	|*	|d	|ć	|a	|*	|k	|[34][S]\darr	|*	|.
|*	|[35][S]\rarr	|k	|u	|r	|a	|*	|*	|*	|*	|*	|*	|l	|*	|o	|[36][S]\darr	|u	|*	|l	|*	|[][,]{ }	|a	|*	|.
|[37][S]\rarr	|j	|u	|b	|i	|l	|a	|c	|j	|a	|*	|*	|o	|*	|n	|t	|k	|*	|i	|*	|f	|m	|*	|.
|*	|*	|l	|*	|[38][S]\rarr	|l	|e	|w	|i	|a	|t	|a	|n	|*	|y	|u	|t	|[39][S]\drarr	|s	|ł	|u	|p	|*	|.
|[40][S]\rarr	|h	|a	|g	|i	|o	|g	|r	|a	|f	|i	|a	|*	|*	|*	|n	|o	|m	|t	|*	|n	|e	|*	|.
|[41][S]\rarr	|g	|r	|u	|p	|a	|[][,]{ }	|l	|i	|e	|g	|o	|*	|*	|*	|d	|w	|k	|a	|*	|k	|r	|*	|.
|*	|*	|n	|*	|*	|*	|*	|[42][S]\rarr	|s	|u	|b	|l	|i	|t	|o	|r	|a	|l	|*	|*	|c	|*	|*	|.
|[43][S]\rarr	|d	|y	|s	|k	|o	|p	|a	|t	|i	|a	|*	|*	|*	|*	|a	|*	|i	|*	|*	|j	|*	|*	|.
|*	|*	|*	|*	|*	|*	|*	|*	|*	|*	|*	|*	|*	|*	|*	|*	|*	|k	|*	|*	|i	|*	|*	|.
|*	|*	|*	|*	|*	|*	|*	|*	|*	|*	|*	|*	|*	|*	|*	|*	|*	|*	|*	|*	|*	|*	|*	|.\end{Puzzle}

\newpage

\begin{PuzzleClues}{\textbf{Poziome}\\}\Clue{2}{}{o dorosłej kobiecie starającej się wyglądać i zachowywać jak nastolatka}
\Clue{3}{}{ogół kardynałów Kościoła katolickiego, którego zadaniem jest wspomaganie papieża w kierowaniu Kościołem; kolegium zbiera się na konsystorzach zwoływanych przez papieża, po wystąpieniu wakatu Stolicy Apostolskiej zbiera się na konklawe w celu wybrania nowego papieża}
\Clue{4}{}{zmiana wprowadzona przez malarza w czasie pracy nad obrazem, której ślady są widoczne na ostatecznej wersji}
\Clue{5}{}{wiewióreczniki, ryjówki nadrzewne, Scandentia - rząd ssaków łożyskowych z nadrzędu euarchontów, blisko spokrewnionych z skóroskrzydłymi i naczelnymi; występują w wilgotnych lasach równikowych Azji Południowo-Wschodniej, m.in. na Archipelagu Malajskim, Filipinach i na Półwyspie Indyjskim}
\Clue{9}{}{spotkanie towarzyskie z tańcami}
\Clue{12}{}{hełmiatka zwyczajna, kaczka hełmiasta, Netta rufina - gatunek średniego wodnego ptaka wędrownego z rodziny kaczkowatych (Anatidae); zamieszkuje południową część krainy palearktycznej od Półwyspu Iberyjskiego po Azję Środkową}
\Clue{14}{}{proces oczyszczania czegoś (samoistny lub powodowany przez człowieka)}
\Clue{24}{}{przeniesienie obowiązków z jednej jednostki organizacyjnej do innej na mocy ustawy}
\Clue{29}{}{bezwstydnica, puszczalska}
\Clue{30}{}{angielski przewodnik i antropolog (1822-1911 ); twórca zasad eugeniki, stworzył podstawy daktyloskopii}
\Clue{32}{}{nowotwór utworzony z komórek o niskim zróżnicowaniu (niedojrzałych), o budowie znacznie odbiegającej od obrazu prawidłowych tkanek, charakteryzujący się atypią i szybkim wzrostem}
\Clue{33}{}{urządzenie wykonane z kawałka bambusowej rurki}
\Clue{35}{}{ptak domowy pochodzący od bankiwy, udomowiony ok. 2500 p.n.e}
\Clue{37}{}{długa wokaliza śpiewana na 1 sylabie}
\Clue{38}{}{potwór morski opisany w Biblii jako ogromna ryba przeznaczona na ucztę dla wszystkich sprawiedliwych Żydów po przyjściu Mesjasza}
\Clue{39}{}{bal, belka wolno stojąca lub stanowiąca element większej konstrukcji}
\Clue{40}{}{dział biografistyki, który obejmuje żywoty świętych, legendy z nimi związane oraz opisy cudów}
\Clue{41}{}{grupa, która jest jednocześnie gładką rozmaitością (topologiczną) i na którą można patrzeć jako na zbiór}
\Clue{42}{}{strefa dna zbiornika wodnego granicząca z litoralem, poniżej granicy występowania roślinności; to strefa tuż poniżej litoralu, często w miejscu gdzie zaczyna się bardziej (niż w litoralu) gwałtowny spadek dna}
\Clue{43}{}{szerokie pojęcie obejmujące schorzenia krążka międzykręgowego}\end{PuzzleClues}

\begin{PuzzleClues}{\textbf{Pionowe}\\}\Clue{1}{}{ANTRESOLA, PÓŁPIĘTRO}
\Clue{2}{}{grubiańsko i z niechęcią: starszy, odpychający mężczyzna}
\Clue{3}{}{ministrant odpowiedzialny za noszenie świec oraz przygotowanie ich do mszy i nabożeństw}
\Clue{4}{}{dęty, blaszany instrument muzyczny o niskiej skali, mający ruchomy suwak do regulacji wysokości wydobywanego dźwięku}
\Clue{5}{}{wydarzenie towarzysko-religijne, podczas którego ktoś jest chrzczony, przyjmowany do chrześcijańskiej wspólnoty wiernych i zostaje mu nadane imię}
\Clue{6}{}{maszyna rolnicza wykorzystywana przy żniwach}
\Clue{7}{}{dział teologii dogmatycznej, nauka o Maryi z Nazaretu, Matce Jezusa Chrystusa}
\Clue{8}{}{linka z wielokrążkiem do utrzymywania bomu}
\Clue{10}{}{owoc rośliny o tej samej nazwie (śliwy tarniny), mały pestkowiec o cierpkogorzkawym smaku}
\Clue{11}{}{recykling, recyrkulacja, recyklizacja - wykorzystanie przetworzonych odpadów w przemyśle}
\Clue{13}{}{miasto i port w Danii na Płw. Jutlandzkim, nad zatoką Kolding Fjord}
\Clue{15}{}{przenośnie: pasożyt, wampir, ktoś, kto na kimś żeruje, przysysa się do kogoś, nie chce się odczepić}
\Clue{16}{}{konstrukcja z drewnianych listewek przytwierdzona do odwrotnej strony obrazu malowanego na desce w celu zabezpieczenia jej przed spęczeniem się i pękaniem}
\Clue{17}{}{dziewiętnastowieczny amerykański romantyk, wyznawca autonomii moralnej i równości (zgodnie z transcendetalizmem - nurtem amerykańskiego romantyzmu)}
\Clue{18}{}{miasto w Federacji Rosyjskiej nad Oką}
\Clue{19}{}{ćwiczenia wojskowe mające na celu sprawne wykonywanie rozkazów przez żołnierzy}
\Clue{20}{}{kangur rudy, Macropus rufus - torbacz z rodziny kangurowatych, największy z torbaczy; występuje pospolicie w całej Australii z wyjątkiem części południowego i wschodniego wybrzeża oraz północnych lasów równikowych}
\Clue{21}{}{dwukonna dorożka}
\Clue{22}{}{to, że coś jaśnieje na jakimś tle, świeci lub mieni się}
\Clue{23}{}{opłata, której podlegają przedsiębiorcy wprowadzający produkt w opakowaniach, jeśli nie zapewnili określonych poziomów odzysku i recyklingu odpadów opakowaniowych}
\Clue{24}{}{krzywy, krótki nóż szewski}
\Clue{25}{}{jednostka używana dawniej w Rosji, 71 cm}
\Clue{26}{}{Dinornis torosus - gatunek wymarłego ptaka nielota z rodziny moa (Dinornithidae)}
\Clue{27}{}{oceniany pozytywnie (z podziwem) samochód osobowy}
\Clue{28}{}{domknięcie zbioru argumentów funkcji, dla których ma ona wartość różną od zera}
\Clue{31}{}{w potocznym rozumieniu: bardzo silny wiatr}
\Clue{32}{}{jednostka liczności fotonów}
\Clue{34}{}{podstawowa jednostka prądu elektrycznego w układzie SI}
\Clue{36}{}{bezleśne zbiorowisko roślinności w zimnym klimacie strefy arktycznej i subarktycznej}
\Clue{39}{}{gatunek motyli z rodziny omacnicowatych. Przypomina wyglądem mola mieszkającego w szafach, szkodnik magazynów zbożowych i spożywczych oraz szafek kuchennych. Osiąga 2,5 cm rozpiętości skrzydeł. Ma białawe skrzydła z ciemnym deseniem, tylna para skrzydeł jest szarawa}\end{PuzzleClues}\newpage\section*{Krzyżówka 122}

\noindent\begin{Puzzle}{18}{23}|*	|[1][S]\drarr	|m	|ł	|o	|d	|z	|i	|e	|j	|o	|w	|s	|k	|i	|*	|[2][S]\darr	|*	|*	|.
|*	|s	|*	|[3][S]\darr	|*	|*	|[4][S]\darr	|*	|*	|*	|*	|[5][S]\darr	|[6][S]\darr	|[7][S]\darr	|*	|[8][S]\darr	|p	|*	|*	|.
|*	|p	|[9][S]\darr	|w	|*	|[10][S]\darr	|n	|*	|*	|*	|*	|t	|m	|k	|*	|e	|ł	|*	|*	|.
|[11][S]\drarr	|i	|r	|o	|n	|s	|i	|d	|e	|s	|*	|a	|e	|o	|[12][S]\darr	|t	|y	|*	|*	|.
|g	|k	|e	|j	|*	|e	|e	|*	|*	|*	|*	|w	|d	|r	|k	|o	|n	|*	|*	|.
|l	|e	|s	|e	|*	|k	|p	|*	|*	|*	|[13][S]\drarr	|l	|a	|m	|u	|s	|*	|*	|*	|.
|e	|r	|t	|w	|*	|s	|r	|*	|*	|*	|c	|i	|n	|o	|r	|u	|[14][S]\darr	|*	|[15][S]\darr	|.
|b	|*	|a	|ó	|*	|a	|a	|*	|*	|*	|h	|n	|*	|r	|u	|k	|p	|*	|m	|.
|a	|*	|u	|d	|*	|p	|w	|[16][S]\rarr	|u	|n	|i	|a	|*	|a	|s	|s	|r	|*	|i	|.
|[][,]{ }	|*	|r	|z	|[17][S]\darr	|i	|o	|*	|[18][S]\darr	|*	|ń	|*	|[19][S]\darr	|n	|z	|y	|z	|*	|e	|.
|a	|*	|a	|t	|b	|l	|ś	|[20][S]\drarr	|g	|ą	|s	|k	|a	|*	|*	|m	|e	|*	|d	|.
|u	|*	|c	|w	|ł	|*	|ć	|c	|e	|*	|z	|*	|x	|*	|[21][S]\darr	|i	|w	|*	|n	|.
|t	|[22][S]\drarr	|j	|o	|y	|*	|*	|o	|o	|*	|c	|[23][S]\darr	|a	|*	|k	|d	|i	|*	|i	|.
|o	|k	|a	|*	|s	|[24][S]\darr	|*	|s	|d	|*	|z	|z	|r	|*	|o	|*	|e	|*	|c	|.
|g	|r	|*	|[25][S]\darr	|z	|k	|*	|b	|e	|*	|y	|r	|*	|*	|n	|*	|l	|*	|z	|.
|e	|i	|*	|r	|c	|o	|*	|u	|t	|*	|z	|z	|*	|*	|c	|*	|e	|*	|k	|.
|n	|s	|*	|e	|z	|n	|[26][S]\rarr	|c	|a	|r	|n	|e	|g	|i	|e	|*	|b	|[27][S]\darr	|a	|.
|i	|*	|*	|j	|y	|s	|[28][S]\darr	|*	|*	|[29][S]\rarr	|a	|s	|p	|i	|r	|a	|n	|t	|*	|.
|c	|[30][S]\drarr	|j	|a	|k	|u	|b	|o	|w	|o	|*	|z	|*	|*	|t	|*	|o	|a	|*	|.
|z	|p	|*	|*	|*	|m	|y	|*	|*	|[31][S]\rarr	|d	|o	|m	|y	|*	|*	|ś	|i	|*	|.
|n	|r	|[32][S]\drarr	|n	|i	|e	|d	|o	|ł	|ę	|ż	|n	|o	|ś	|ć	|*	|ć	|a	|*	|.
|a	|ę	|x	|*	|*	|n	|l	|*	|*	|[33][S]\rarr	|t	|y	|r	|o	|l	|*	|*	|n	|*	|.
|*	|t	|u	|*	|*	|t	|ę	|[34][S]\rarr	|i	|s	|e	|*	|[35][S]\rarr	|l	|i	|c	|o	|*	|*	|.
|*	|*	|*	|*	|*	|*	|*	|*	|*	|*	|*	|*	|*	|*	|*	|*	|*	|*	|*	|.\end{Puzzle}

\newpage

\begin{PuzzleClues}{\textbf{Poziome}\\}\Clue{1}{}{kompozytor, dyrygent i krytyk muzyczny (1909-1985); dyrygent orkiestry symfonicznej w Opolu, Poznaniu}
\Clue{11}{}{angielskie oddziały jazdy w służbie parlamentu, dowodzone przez Cromwella podczas angielskiej rewolucji 1640-60}
\Clue{13}{}{rodzaj spichlerza; budynek, w którym przechowywano zboże, zbroje, uprząż, dokumenty: potocznie rupieciarnia, graciarnia}
\Clue{16}{}{skrótowo o Unii Europejskiej}
\Clue{20}{}{Tricholoma - rodzaj grzybów należący do rodziny gąskowatych}
\Clue{22}{}{astronom amerykański (1882-1973); wyznaczył prędkość obiegu Słońca wokół środka Galaktyki, odkrył gwiazdy zmienne}
\Clue{26}{}{jezioro w zachodniej Australii}
\Clue{29}{}{kandydat, osoba ubiegająca się o coś, kandydująca do czegoś}
\Clue{30}{}{wieś w Polsce położona w województwie kujawsko-pomorskim, w powiecie bydgoskim, w gminie Nowa Wieś Wielka}
\Clue{31}{}{część nieba lub gwiazdozbioru służąca do określania położenia Słońca, Księżyca i planet}
\Clue{32}{}{cecha czegoś, co jest marnej jakości, słabe, co nie spełnia wymagań, oczekiwań}
\Clue{33}{}{kraj związkowy w zachodniej Austrii}
\Clue{34}{}{miasto w Japonii na wyspie Honsiu; hodowla perłopławów, przemysł włókienniczy i spożywczy}
\Clue{35}{}{w garbarstwie: zewnętrzna powierzchnia skóry, o charakterystycznym dla każdego gatunku zwierząt rysunku}\end{PuzzleClues}

\begin{PuzzleClues}{\textbf{Pionowe}\\}\Clue{1}{}{ten, kto w studiu telewizyjnym lub radiowym zapowiada program, odczytuje coś wcześniej przygotowanego}
\Clue{2}{}{ciecz, ciało ciekłe; substancja, która charakteryzuje się wielką łatwością zmieniania wzajemnego położenia poszczególnych elementów nawet dla niewielkich sił - w przeciwieństwie do ciał stałych, które przy niewielkich siłach wykazują proporcjonalność odkształcenia do naprężeń}
\Clue{3}{}{jednostka podziału administracyjnego wyższego stopnia, od 1990 r. jednostka zasadniczego podziału terytorialnego dla administracji rządowej, od 1999 r. także jednostka samorządu terytorialnego}
\Clue{4}{}{to, co jest złe, nieprawe}
\Clue{5}{}{TAWULEC - azjatycki krzew z rodziny różowatych}
\Clue{6}{}{wieś we Francji koło Wersalu, posiadłość Emila Zoli}
\Clue{7}{}{ptak wodny z rzędu wiosłonogich, rybożerny, doskonale nurkujący; w Polsce rzadki, chroniony}
\Clue{8}{}{lek działający przeciwdrgawkowo; hamuje impulsy elektryczne w mózgu, będące przyczyną napadów padaczki}
\Clue{9}{}{powrót do władzy jakiejś dynastii, przywrócenie ustroju}
\Clue{10}{}{atrakcyjność fizyczna, energia seksualna}
\Clue{11}{}{gleba utworzona bez udziału materiałów i czynników zewnętrznych (np. wód gruntowych)}
\Clue{12}{}{jednostka zdawkowa w Turcji; 1/100 liry tureckiej}
\Clue{13}{}{coś zupełnie niezrozumiałego (coś - zwykle jakiś komunikat - tak mało zrozumiałego, jakby było po chińsku)}
\Clue{14}{}{forma zwracania się do wyższych duchownych, używana w wyrażeniu: Wasza lub Jego Przewielebność}
\Clue{15}{}{zdrobniale o miednicy, misce służącej do prania, mycia się, zmywania}
\Clue{17}{}{kosmetyk kolorowy służący do podkreślenia ust i nadawania im delikatnego połysku i barwy, używany zamiast szminki lub razem z nią}
\Clue{18}{}{człowiek, który naukowo lub zawodowo zajmuje się geodezją}
\Clue{19}{}{zatoka Oceanu Atlantyckiego o północnych wybrzeży Islandii}
\Clue{20}{}{(1866-1918), poeta rumuński, liryka miłosna, poematy}
\Clue{21}{}{instrumentalny utwór przeznaczony do wykonania przez solistę i orkiestrę symfoniczną}
\Clue{22}{}{sztylet malajski o wężykowatej klindze i bogato zdobionej rękojeści}
\Clue{23}{}{ten, który jest należy do jakiegoś zrzeszenia - związku, ugrupowania}
\Clue{24}{}{klient w restauracji lub w innym lokalu; osoba, która coś je}
\Clue{25}{}{poziome drzewce omasztowania}
\Clue{27}{}{(TAI'AN); miasto w Chinach (Shandong) na płd. od Jinan}
\Clue{28}{}{krowa, byk, wół lub cielę}
\Clue{30}{}{dawna jednostka długości i powierzchni: 3,5 ÷ 5 m oraz 0,0019 ÷ 2,08 ha}
\Clue{32}{}{jednostka zdawkowa w Wietnamie; 1/100 donga, 1/10 hao}\end{PuzzleClues}\newpage\section*{Krzyżówka 123}

\noindent\begin{Puzzle}{23}{23}|*	|*	|*	|*	|*	|*	|*	|*	|*	|*	|*	|*	|*	|*	|*	|*	|*	|*	|*	|*	|[1][S]\darr	|*	|*	|*	|.
|*	|*	|*	|*	|*	|*	|[2][S]\darr	|*	|*	|*	|*	|[3][S]\drarr	|o	|l	|e	|j	|*	|[4][S]\darr	|*	|[5][S]\darr	|n	|[6][S]\darr	|[7][S]\darr	|*	|.
|*	|[8][S]\rarr	|w	|i	|e	|c	|z	|ó	|r	|*	|*	|j	|*	|*	|[9][S]\rarr	|f	|e	|m	|i	|n	|i	|z	|m	|*	|.
|*	|[10][S]\rarr	|c	|i	|ę	|ż	|a	|r	|*	|*	|*	|ę	|[11][S]\rarr	|m	|y	|j	|k	|a	|*	|i	|e	|a	|a	|*	|.
|*	|*	|[12][S]\rarr	|c	|h	|i	|m	|b	|o	|r	|a	|z	|o	|*	|*	|*	|*	|n	|*	|e	|o	|s	|n	|*	|.
|*	|*	|[13][S]\rarr	|m	|u	|m	|i	|a	|*	|[14][S]\darr	|[15][S]\darr	|y	|*	|*	|*	|*	|[16][S]\darr	|y	|*	|n	|s	|i	|n	|*	|.
|*	|*	|*	|*	|*	|[17][S]\darr	|e	|*	|[18][S]\darr	|s	|p	|k	|*	|[19][S]\darr	|*	|*	|s	|a	|*	|e	|t	|ł	|i	|*	|.
|*	|*	|*	|*	|*	|s	|n	|*	|p	|m	|a	|[][,]{ }	|*	|w	|[20][S]\rarr	|p	|p	|s	|*	|u	|r	|e	|c	|*	|.
|*	|*	|[21][S]\darr	|*	|[22][S]\rarr	|a	|n	|g	|i	|o	|t	|e	|n	|s	|y	|n	|a	|*	|*	|t	|o	|k	|a	|*	|.
|*	|*	|t	|*	|*	|m	|i	|*	|e	|ł	|e	|z	|*	|p	|*	|*	|h	|*	|*	|r	|ś	|[][,]{ }	|[][,]{ }	|*	|.
|*	|*	|o	|*	|*	|o	|k	|*	|z	|o	|n	|o	|[23][S]\darr	|ó	|*	|*	|i	|*	|*	|a	|ć	|w	|o	|*	|.
|*	|*	|r	|*	|*	|u	|*	|*	|a	|w	|t	|p	|i	|ł	|*	|*	|s	|*	|*	|l	|*	|y	|d	|*	|.
|*	|[24][S]\darr	|f	|*	|[25][S]\darr	|p	|*	|*	|*	|i	|*	|o	|m	|m	|*	|*	|*	|*	|*	|n	|*	|c	|s	|*	|.
|*	|w	|o	|[26][S]\drarr	|d	|r	|u	|g	|i	|e	|*	|w	|a	|a	|*	|*	|*	|*	|*	|o	|*	|h	|t	|*	|.
|*	|ó	|w	|s	|e	|o	|[27][S]\darr	|[28][S]\darr	|*	|c	|*	|y	|g	|ł	|*	|*	|*	|*	|*	|ś	|[29][S]\darr	|o	|a	|*	|.
|*	|j	|i	|z	|p	|w	|o	|p	|[30][S]\darr	|*	|*	|*	|o	|ż	|*	|*	|*	|*	|*	|ć	|z	|w	|j	|*	|.
|[31][S]\drarr	|t	|e	|t	|r	|a	|p	|l	|e	|g	|i	|k	|*	|o	|*	|*	|*	|*	|*	|*	|l	|a	|ą	|*	|.
|w	|o	|c	|a	|e	|d	|o	|i	|k	|[32][S]\rarr	|g	|r	|o	|n	|o	|w	|o	|*	|*	|*	|e	|w	|c	|*	|.
|y	|w	|*	|j	|s	|z	|z	|k	|s	|*	|*	|*	|*	|e	|[33][S]\rarr	|ł	|a	|s	|k	|a	|w	|c	|a	|*	|.
|l	|i	|*	|m	|j	|e	|y	|*	|p	|*	|*	|[34][S]\rarr	|a	|k	|s	|o	|l	|o	|t	|l	|*	|z	|*	|*	|.
|e	|c	|*	|e	|a	|n	|c	|*	|e	|*	|[35][S]\rarr	|p	|t	|*	|*	|[36][S]\rarr	|l	|u	|b	|c	|z	|y	|k	|*	|.
|w	|z	|*	|s	|*	|i	|j	|*	|r	|*	|*	|*	|[37][S]\rarr	|h	|o	|l	|o	|w	|n	|i	|k	|*	|*	|*	|.
|*	|*	|*	|*	|*	|e	|a	|[38][S]\rarr	|t	|o	|r	|y	|z	|m	|*	|*	|*	|*	|*	|*	|*	|*	|*	|*	|.
|[39][S]\rarr	|g	|o	|d	|y	|*	|*	|*	|*	|*	|*	|*	|*	|*	|*	|*	|*	|*	|*	|*	|*	|*	|*	|*	|.\end{Puzzle}

\newpage

\begin{PuzzleClues}{\textbf{Poziome}\\}\Clue{3}{}{technika malarska, która polega na malowaniu farbami olejnymi}
\Clue{8}{}{część doby rozpoczynająca się z końcem dnia}
\Clue{9}{}{ruch społeczny, którego naczelną ideą jest równouprawnienie kobiet}
\Clue{10}{}{obiekt, który dużo waży, coś ciężkiego}
\Clue{11}{}{mechaniczne urządzenie do mycia np. surowców i opakowań}
\Clue{12}{}{prowincja w Ekwadorze, w Andach, stolica Riobambo}
\Clue{13}{}{wyschnięte szczątki owoców ziarnkowych i pestkowych; składają się z nie do końca rozłożonych fragmentów tkanki miękiszowej przerośniętych grzybnią pasożyta}
\Clue{20}{}{skrót od Polska Partia Socjalistyczna}
\Clue{22}{}{hormon peptydowy, którego zadaniem jest kontrola stężenia jonów sodowych i potasowych w organizmie}
\Clue{26}{}{drugie danie, potrawa podawana (zazwyczaj) po zupie}
\Clue{31}{}{osoba, która uległa paraliżowi czterokończynowemu}
\Clue{32}{}{Gronowo - wieś}
\Clue{33}{}{ktoś, kto jest łaskawy, czyni dobro, pomaga komuś, ratuje kogoś, często bezinteresownie}
\Clue{34}{}{odmiana płazów ogoniastych ambystom - długości ok. 20 cm}
\Clue{35}{}{w chemii: symbol platyny}
\Clue{36}{}{znana przyprawa (afrodyzjak); natka lubczyku}
\Clue{37}{}{statek lub okręt pomocniczy konstrukcyjnie przewidziany do prac holowniczych}
\Clue{38}{}{konserwatyzm brytyjski; poglądy przedstawicieli stronnictwa torysów}
\Clue{39}{}{rykowisko u jeleni}\end{PuzzleClues}

\begin{PuzzleClues}{\textbf{Pionowe}\\}\Clue{1}{}{cecha obrazu lub fotografii widzianego niewyraźnie, sprawiającego wrażenie rozmazanego}
\Clue{2}{}{coś, co zastępuje coś innego}
\Clue{3}{}{formowanie wypowiedzi tak, aby jej treści zostały zasugerowane w sposób utajony, pośredni, zawoalowany; charakterystyczne jest posługiwanie się m.in. metaforą, peryfrazą, synekdochą, parabolą}
\Clue{4}{}{jezioro w Turcji}
\Clue{5}{}{postawa nieneutralna, zaangażowanie w coś}
\Clue{6}{}{zasiłek przysługujący rodzicowi będącemu na urlopie wychowawczym}
\Clue{7}{}{Puccinellia distans - roślina z rodziny wiechlinowatych}
\Clue{14}{}{smoła w stanie płynnym}
\Clue{15}{}{świadectwo ukończenia szkoły}
\Clue{16}{}{żołnierz konnych oddziałów arabskich we francuskich wojskach kolonialnych}
\Clue{17}{}{uprowadzenie, porwanie, które jest oszustwem, nie zaszło naprawdę, ale było planem, spiskiem osoby, która podaje się za uprowadzoną; upozorowane uprowadzenie}
\Clue{18}{}{jednostka ciśnienia w układzie MKS; nielegalna}
\Clue{19}{}{osoba, z którą pozostaje się w związku małżeńskim, mąż lub żona}
\Clue{21}{}{Sphagnum - rodzaj mchów, należący do klasy torfowców (Sphagnopsida); należy do niego ok. 350 gatunków}
\Clue{23}{}{ostatecznie ukształtowana dorosła postać owada, z reguły zdolna do rozrodu}
\Clue{24}{}{ur. w 1920 r. grafik, drzeworyty o tematyce fantastycznej kompozycje abstrakcyjne}
\Clue{25}{}{obszar lądu położony poniżej poziomu morza}
\Clue{26}{}{człowiek, żyjący na marginesie społeczeństwa, menel}
\Clue{27}{}{w szermierce; zabezpieczenie ręki gardą przed końcem szpady przeciwnika}
\Clue{28}{}{zbiór rzeczy, najczęściej papieru, banknotów, czasopism itp., fizycznie ze sobą powiązanych}
\Clue{29}{}{nazwa czynności, polegającej na zbieraniu cieczy w jedno miejsce}
\Clue{30}{}{biegły, rzeczoznawca, człowiek, od którego można spodziewać się profesjonalnej opinii na jakiś temat, który z powodu swoich kompetencji może być powołany przez sąd jako świadek}
\Clue{31}{}{nagłe uzewnętrznienie często niespodziewanych uczuć, zwykle serdecznych}\end{PuzzleClues}\newpage\section*{Krzyżówka 124}

\noindent\begin{Puzzle}{25}{24}|*	|[1][S]\darr	|*	|*	|*	|*	|*	|*	|*	|*	|*	|*	|*	|[2][S]\darr	|*	|*	|*	|*	|*	|*	|*	|*	|*	|*	|*	|*	|.
|*	|t	|*	|*	|*	|*	|[3][S]\drarr	|r	|a	|m	|a	|p	|i	|t	|e	|k	|*	|*	|*	|*	|*	|*	|*	|*	|*	|*	|.
|*	|o	|*	|*	|*	|*	|u	|*	|*	|*	|[4][S]\rarr	|o	|b	|r	|ó	|ż	|k	|a	|*	|*	|*	|*	|*	|*	|*	|*	|.
|*	|r	|*	|*	|*	|*	|k	|*	|*	|*	|*	|*	|*	|ó	|*	|*	|*	|*	|*	|*	|*	|*	|*	|*	|*	|*	|.
|*	|f	|*	|*	|[5][S]\darr	|*	|ł	|*	|*	|*	|*	|*	|*	|j	|*	|*	|*	|*	|*	|*	|*	|*	|*	|*	|*	|*	|.
|*	|o	|*	|*	|p	|*	|a	|*	|*	|*	|*	|*	|*	|l	|[6][S]\darr	|*	|*	|*	|*	|*	|*	|*	|*	|*	|*	|*	|.
|*	|w	|*	|*	|o	|*	|d	|*	|*	|*	|*	|*	|*	|i	|a	|*	|*	|*	|*	|*	|*	|*	|*	|*	|*	|*	|.
|*	|i	|*	|*	|s	|*	|[][,]{ }	|*	|*	|*	|*	|[7][S]\rarr	|e	|s	|k	|o	|r	|t	|o	|w	|i	|e	|c	|*	|*	|*	|.
|*	|e	|*	|*	|t	|*	|p	|*	|*	|*	|*	|*	|*	|t	|c	|*	|*	|*	|*	|*	|*	|*	|*	|*	|*	|*	|.
|*	|c	|[8][S]\rarr	|o	|r	|k	|i	|s	|z	|*	|*	|*	|*	|[][,]{ }	|j	|*	|*	|*	|*	|*	|*	|*	|*	|*	|*	|*	|.
|*	|[][,]{ }	|*	|*	|z	|*	|a	|[9][S]\rarr	|k	|s	|i	|ą	|ż	|k	|a	|[][,]{ }	|k	|u	|c	|h	|a	|r	|s	|k	|a	|*	|.
|*	|b	|[10][S]\drarr	|p	|a	|ń	|s	|z	|c	|z	|y	|z	|n	|a	|*	|*	|*	|*	|*	|*	|*	|*	|*	|*	|*	|*	|.
|*	|a	|a	|*	|ł	|*	|e	|*	|*	|*	|*	|*	|*	|m	|*	|*	|*	|*	|*	|*	|*	|*	|*	|*	|*	|*	|.
|*	|ł	|ż	|*	|k	|*	|c	|*	|*	|*	|*	|*	|*	|c	|*	|*	|*	|*	|*	|*	|*	|*	|*	|*	|*	|*	|.
|[11][S]\rarr	|t	|u	|ł	|a	|*	|k	|*	|*	|*	|*	|*	|*	|z	|*	|*	|[12][S]\darr	|*	|*	|*	|*	|*	|*	|*	|*	|*	|.
|*	|y	|r	|*	|*	|*	|i	|*	|*	|*	|*	|[13][S]\drarr	|z	|a	|w	|i	|e	|w	|*	|*	|*	|*	|*	|*	|*	|*	|.
|*	|c	|*	|*	|*	|*	|e	|*	|*	|*	|*	|o	|*	|c	|*	|*	|n	|*	|*	|*	|*	|*	|*	|*	|*	|*	|.
|*	|k	|*	|*	|*	|*	|g	|*	|*	|*	|*	|d	|*	|k	|*	|*	|a	|*	|*	|*	|*	|*	|*	|*	|*	|*	|.
|*	|i	|*	|*	|*	|*	|o	|*	|[14][S]\darr	|*	|*	|c	|*	|i	|*	|*	|m	|*	|*	|*	|*	|*	|*	|*	|*	|*	|.
|*	|*	|*	|*	|*	|*	|*	|*	|o	|*	|*	|i	|*	|*	|*	|*	|i	|*	|*	|*	|*	|*	|*	|*	|*	|*	|.
|*	|*	|*	|*	|*	|*	|*	|*	|m	|*	|*	|n	|*	|*	|*	|*	|n	|*	|*	|*	|*	|*	|*	|*	|*	|*	|.
|*	|[15][S]\rarr	|z	|w	|o	|l	|n	|i	|e	|n	|i	|e	|[][,]{ }	|l	|e	|k	|a	|r	|s	|k	|i	|e	|*	|*	|*	|*	|.
|*	|*	|*	|*	|*	|*	|*	|*	|g	|*	|*	|k	|*	|*	|*	|*	|*	|*	|*	|*	|*	|*	|*	|*	|*	|*	|.
|*	|*	|*	|*	|*	|*	|*	|*	|a	|*	|*	|*	|*	|*	|*	|*	|*	|*	|*	|*	|*	|*	|*	|*	|*	|*	|.
|*	|*	|*	|*	|*	|*	|*	|*	|*	|*	|*	|*	|*	|*	|*	|*	|*	|*	|*	|*	|*	|*	|*	|*	|*	|*	|.\end{Puzzle}

\newpage

\begin{PuzzleClues}{\textbf{Poziome}\\}\Clue{3}{}{siwapitek, Ramapithecus, obecnie Sivapithecus - rodzaj kopalnych małp człekokształtnych żyjących w epoce miocenu (ok. 12,5-8,5 mln lat temu) na terenie północnych Indii i wschodniej Afryki}
\Clue{4}{}{zdrobniale: obroża - krótki naszyjnik, często w formie paska okalającego szyję}
\Clue{7}{}{DOZOROWIEC}
\Clue{8}{}{Triticum spelta - gatunek zboża należący do rodziny wiechlinowatych}
\Clue{9}{}{książka z przepisami kulinarnymi, która może oprócz nich zawierać ogólne porady i wskazówki dotyczące gotowania}
\Clue{10}{}{uciążliwy obowiązek}
\Clue{11}{}{miasto obwodowe w europejskiej części Federacji Rosyjskiej w Podmoskiewskim Zagłębiu Węglowym; samowary}
\Clue{13}{}{pojedynczy podmuch wiatru, zazwyczaj zimnego}
\Clue{15}{}{zaświadczenie urzędowe wystawiane przez lekarza w celu usprawiedliwienia nieobecności pracownika w pracy z powodu niezdolności do pracy wskutek choroby lub konieczności zapewnienia opieki choremu członkowi rodziny}\end{PuzzleClues}

\begin{PuzzleClues}{\textbf{Pionowe}\\}\Clue{1}{}{Sphagnum balticum - gatunek mchu z rodziny torfowcowatych}
\Clue{2}{}{Trillium camschatcense - gatunek rośliny zielnej z rodziny melantkowatych}
\Clue{3}{}{układ wirników w śmigłowcu, charakteryzujący się umieszczeniem wirników w układzie tandemowym, poziomo i wzdłużnie, zamontowany jeden przy drugim, najczęściej blisko końców kadłuba. Każdy z wirników porusza się w przeciwnym kierunku}
\Clue{5}{}{Pedetes capensis - gryzoń należący do rodziny postrzałek (Pedetidae); występuje na piaszczystych, trawiastych równinach wschodniej i południowej Afryki}
\Clue{6}{}{chronologiczny ciąg wydarzeń przedstawionych w filmie lub utworze literackim}
\Clue{10}{}{ozdobny układ otworów stosowany w architekturze}
\Clue{12}{}{amina, w której grupa aminowa (NR/3; R = grupa alkilowa lub atom wodoru) jest przyłączona bezpośrednio do atomu węgla połączonego podwójnym wiązaniem z innym atomem węgla}
\Clue{13}{}{teren wyznaczony do działań jednostki wojskowej}
\Clue{14}{}{jasna mgławica gazowa na pograniczu gwiazdozbiorów Strzelca, Tarczy i Węża}\end{PuzzleClues}\newpage\section*{Krzyżówka 125}

\noindent\begin{Puzzle}{22}{31}|*	|*	|*	|*	|[1][S]\drarr	|t	|r	|o	|f	|o	|b	|l	|a	|s	|t	|*	|[2][S]\drarr	|o	|s	|j	|a	|n	|*	|.
|*	|[3][S]\rarr	|a	|n	|t	|y	|p	|o	|d	|y	|*	|*	|*	|*	|*	|*	|g	|*	|*	|*	|[4][S]\darr	|*	|*	|.
|[5][S]\drarr	|p	|i	|e	|r	|w	|i	|o	|s	|n	|e	|k	|*	|*	|[6][S]\rarr	|p	|a	|l	|l	|i	|u	|m	|*	|.
|k	|[7][S]\darr	|[8][S]\drarr	|w	|o	|j	|s	|k	|o	|[][,]{ }	|f	|e	|d	|e	|r	|a	|l	|n	|e	|*	|r	|[9][S]\darr	|*	|.
|o	|p	|w	|[10][S]\darr	|p	|*	|*	|*	|*	|[11][S]\drarr	|a	|n	|t	|r	|e	|s	|o	|l	|a	|*	|o	|w	|*	|.
|m	|r	|i	|s	|i	|[12][S]\rarr	|r	|z	|e	|p	|a	|*	|*	|*	|[13][S]\darr	|*	|n	|*	|*	|*	|d	|s	|*	|.
|a	|z	|e	|t	|k	|*	|[14][S]\rarr	|k	|u	|r	|t	|a	|c	|z	|e	|k	|*	|*	|*	|*	|y	|z	|*	|.
|*	|e	|l	|u	|*	|[15][S]\rarr	|m	|o	|c	|z	|a	|r	|n	|i	|k	|[][,]{ }	|b	|ł	|o	|t	|n	|y	|*	|.
|[16][S]\drarr	|w	|o	|d	|a	|[][,]{ }	|h	|i	|p	|e	|r	|o	|s	|m	|o	|t	|y	|c	|z	|n	|a	|*	|*	|.
|p	|ó	|k	|i	|*	|*	|*	|*	|*	|j	|*	|[17][S]\darr	|*	|*	|l	|[18][S]\darr	|[19][S]\darr	|*	|*	|*	|m	|*	|*	|.
|o	|d	|s	|o	|*	|*	|*	|*	|*	|e	|[20][S]\darr	|e	|*	|*	|o	|r	|d	|[21][S]\darr	|*	|*	|i	|*	|[22][S]\darr	|.
|z	|[][,]{ }	|z	|[][,]{ }	|*	|*	|*	|*	|*	|m	|m	|k	|*	|*	|g	|a	|o	|ł	|*	|*	|k	|*	|k	|.
|y	|o	|t	|n	|*	|*	|*	|[23][S]\darr	|*	|c	|a	|w	|*	|*	|i	|c	|b	|u	|*	|[24][S]\darr	|a	|*	|o	|.
|c	|d	|a	|a	|*	|[25][S]\darr	|[26][S]\drarr	|s	|p	|a	|c	|a	|l	|*	|a	|j	|r	|p	|*	|p	|*	|[27][S]\darr	|z	|.
|j	|g	|ł	|g	|*	|s	|c	|z	|*	|*	|a	|d	|[28][S]\darr	|*	|*	|o	|y	|i	|[29][S]\darr	|s	|*	|k	|a	|.
|a	|a	|t	|r	|[30][S]\darr	|z	|i	|e	|*	|*	|*	|o	|d	|[31][S]\darr	|*	|n	|[][,]{ }	|n	|t	|t	|*	|r	|[][,]{ }	|.
|[][,]{ }	|ł	|n	|a	|n	|c	|e	|o	|[32][S]\drarr	|k	|a	|r	|o	|n	|g	|a	|z	|a	|u	|r	|*	|ę	|p	|.
|b	|ę	|o	|ń	|o	|z	|k	|l	|o	|*	|*	|c	|p	|i	|*	|l	|n	|*	|c	|o	|[33][S]\darr	|p	|i	|.
|r	|z	|ś	|*	|r	|e	|a	|*	|g	|*	|*	|z	|ł	|e	|*	|n	|a	|*	|u	|k	|b	|n	|e	|.
|a	|i	|ć	|*	|m	|n	|w	|[34][S]\darr	|r	|[35][S]\darr	|*	|y	|y	|d	|*	|a	|j	|[36][S]\darr	|x	|ó	|e	|i	|r	|.
|m	|e	|*	|[37][S]\darr	|a	|i	|s	|l	|a	|p	|[38][S]\darr	|k	|w	|ź	|*	|[][,]{ }	|o	|s	|i	|w	|z	|k	|w	|.
|k	|n	|[39][S]\drarr	|f	|l	|a	|k	|o	|n	|i	|k	|*	|*	|w	|[40][S]\darr	|i	|m	|p	|*	|k	|b	|[][,]{ }	|o	|.
|o	|i	|ż	|a	|i	|c	|o	|k	|i	|k	|l	|*	|*	|i	|k	|g	|y	|i	|[41][S]\darr	|a	|o	|p	|t	|.
|w	|o	|a	|l	|z	|t	|ś	|s	|c	|a	|e	|[42][S]\drarr	|l	|e	|o	|n	|*	|e	|e	|[][,]{ }	|l	|u	|n	|.
|a	|w	|r	|k	|a	|w	|ć	|a	|z	|*	|j	|m	|*	|d	|n	|o	|[43][S]\darr	|l	|m	|n	|e	|s	|a	|.
|*	|y	|n	|o	|c	|o	|*	|p	|e	|[44][S]\drarr	|n	|i	|e	|z	|g	|r	|a	|b	|i	|a	|s	|z	|*	|.
|*	|*	|o	|n	|j	|*	|*	|i	|n	|ł	|o	|e	|*	|i	|r	|a	|d	|e	|s	|d	|n	|y	|*	|.
|*	|*	|w	|k	|a	|*	|*	|n	|i	|o	|t	|c	|*	|c	|e	|n	|e	|r	|j	|o	|o	|s	|*	|.
|*	|*	|i	|a	|*	|*	|*	|a	|e	|ś	|*	|h	|*	|a	|s	|c	|r	|g	|a	|b	|ś	|t	|*	|.
|*	|*	|e	|*	|*	|*	|*	|*	|*	|*	|*	|*	|*	|*	|*	|j	|*	|*	|*	|n	|ć	|y	|*	|.
|*	|*	|c	|*	|*	|[45][S]\rarr	|c	|y	|p	|r	|o	|h	|e	|p	|t	|a	|d	|y	|n	|a	|*	|*	|*	|.
|*	|*	|*	|*	|*	|*	|*	|*	|*	|*	|*	|*	|*	|*	|*	|*	|*	|*	|*	|*	|*	|*	|*	|.\end{Puzzle}

\newpage

\begin{PuzzleClues}{\textbf{Poziome}\\}\Clue{1}{}{zewnętrzna warstwa komórek błony płodowej u ssaków}
\Clue{2}{}{legendarny bard celtycki stworzony przez Mc Persona}
\Clue{3}{}{poglądy, opinie, stanowiska, założenia, które są wobec siebie przeciwne, całkiem różne, skrajne}
\Clue{5}{}{PRYMULA - bylina górskich obszarów strefy umiarkowanej o barwnych. lejkowatych bądź dzwonkowatych kwiatach}
\Clue{6}{}{wierzchnie okrycie męskie w formie prostokątnego spinanego na ramieniu płata tkaniny noszone w starożytnym Rzymie}
\Clue{8}{}{rodzaj wojska w krajach federalnych; w krajach tych istnieć mogą dwa rodzaje wojsk: federalne (rządowe) i wojska poszczególnych stanów lub prowinicji}
\Clue{11}{}{pomieszczenie mieszkalne lub biurowe, jedno- lub kilkupokojowe, wydzielone z przestrzeni parteru, piętra budynku mieszkalnego lub hali przemysłowej}
\Clue{12}{}{warzywo, jadalna bulwa rośliny o tej samej nazwie}
\Clue{14}{}{leśny ptak o kolorowym upierzeniu i krótkim ogonie; zamieszkuje Australię i Afrykę}
\Clue{15}{}{Hygrohypnum luridum - gatunek mchu z rodziny krzywoszyjowatych}
\Clue{16}{}{woda mineralna i lecznicza, krzepnięcie której ma miejsce w temperaturze poniżej -0,55 °C, ciśnienie osmotyczne zaś jest wyższe od 7,7*103 hPa}
\Clue{26}{}{słoweński malarz i grafik zm. 1907 r. 'Abstrakcyjne, kompozycje, mozaiki, dekoracje wnętrz, rzeźba}
\Clue{32}{}{Karongasaurus - rodzaj zauropoda z grupy tytanozaurów; żył w okresie wczesnej kredy na terenach południowej Afryki}
\Clue{39}{}{zawartość flakonika, ozdobnej, małej butelki służącej do przechowywania drogocennych cieczy, np. perfum}
\Clue{42}{}{kraina historyczna w Hiszpanii, obecnie w regionie Kastylia, powierzchnia 38,4 tyś. km2, główne miasto Salamanca}
\Clue{44}{}{niezdara, która nie panuje nad swoim ciałem, np. potrąca rzeczy, nadeptuje na coś, ma drewniane ręce}
\Clue{45}{}{lek przeciwhistaminowy I generacji, charakteryzujący się dodatkowo silnym działaniem serotoninolitycznym}\end{PuzzleClues}

\begin{PuzzleClues}{\textbf{Pionowe}\\}\Clue{1}{}{wielki upał z dużą wilgotnością powietrza, charakterystyczny dla strefy międzyzwrotnikowej}
\Clue{2}{}{anglosaska jednostka objętości cieczy i ciał sypkich około 4,54 l}
\Clue{4}{}{część urologii, której przedmiotem jest badanie czynności, działania pęcherza moczowego i cewki moczowej}
\Clue{5}{}{śpiączka, głębokie zaburzenie świadomości i przytomności, związane z rozpadem cyklu czuwania i wzbudzenia, objawiające się brakiem reakcji nawet na silne bodźce oraz uogólnionym bezruchem}
\Clue{7}{}{przewód wodociągowy doprowadzający wodę do punktu czerpalnego}
\Clue{8}{}{cecha czegoś, co występuje w różnych formach}
\Clue{9}{}{rząd drobnych bezskrzydłowych owadów, żywią się krwią ssaków}
\Clue{10}{}{obiekt przeznaczony do rejestracji nagrań dźwiękowych, obejmujący zazwyczaj reżyserkę i pomieszczenia do miksowania i masteringu}
\Clue{11}{}{osoba fizyczna lub prawna, która przejmuje zadłużenie dłużnika od wierzyciela i występuje przed nim w imieniu tego dłużnika}
\Clue{13}{}{ruch społeczny, działający na rzecz ekologicznego podejścia do życia}
\Clue{16}{}{pozycja piłkarza wobec bramki umożliwiająca strzelenie gola}
\Clue{17}{}{mieszkaniec Ekwadoru, człowiek pochodzenia ekwadorskiego}
\Clue{18}{}{ignorowanie pewnych spraw lub działań wówczas, gdy koszt uzyskania informacji istotnej dla podjęcia decyzji jest wyższy, niż oczekiwana wartość korzyści, przez co nieracjonalne jest jego poniesienie; z dziedziny ekonomii}
\Clue{19}{}{ktoś, kogo dobrze się zna, z kim pozostaje się w bliskich relacjach}
\Clue{20}{}{placek z nie kwaszonego ciasta bez soli pieczony przez Żydów na święta Pesach}
\Clue{21}{}{mała, często mizernie wyglądająca łódka}
\Clue{22}{}{kozioł europejski, Capra prisca - gatunek ssaka parzystokopytnego z rodziny krętorogich, opisany w 1915 roku przez Adametza w oparciu o szczątki pochodzące z Galicji i uznany za przodka współczesnych kóz domowych; prawdopodobnie jest blisko spokrewniony z kozą bezoarową}
\Clue{23}{}{w judaizmie miejsce pobytu zmarłych, pozbawionych wszelkiej radości istnienia i życia}
\Clue{24}{}{Utetheisa pulchella - gatunek motyla z rodziny niedźwiedziówkowatych, o pięknym ubarwieniu z rysunkiem w formie prostokątnych czerwonych i czarnych plamek; lata od końca maja do sierpnia oraz we wrześniu i w październiku}
\Clue{25}{}{cecha zachowania typowego dla kogoś niepoważnego, kto postępuje jak człowiek niedorosły}
\Clue{26}{}{cecha człowieka, który jest ciekawski, nadmiernie ciekawy, chce wszystko wiedzieć, wtyka nos w cudze sprawy}
\Clue{27}{}{Batis margaritae - gatunek ptaka z rodziny krępaczków (Platysteiridae)}
\Clue{28}{}{docieranie czegoś w jakieś miejsce}
\Clue{29}{}{delfin amazoński, Sotalia fluviatilis - gatunek walenia z rodziny delfinowatych; jedyny przedstawiciel Delphinidae, który pływa w wodzie słodkiej}
\Clue{30}{}{proces osiągania równowagi, doprowadzenia do stabilnego lub normalnego stanu}
\Clue{31}{}{samica niedźwiedzia}
\Clue{32}{}{stan rzeczy, którego nie da się lub nie chce się przekroczyć, np. ograniczenie prędkości}
\Clue{33}{}{to, że coś jest emocjonalnie bezbolesne - nie sprawia bólu, nie wywołuje negatywnych odczuć}
\Clue{34}{}{pochodna dibenzodiazepiny, stosowana jako lek przeciwpsychotyczny}
\Clue{35}{}{broń drzewcowa o długości około 5 m}
\Clue{36}{}{Steven Spielberg - amerykański reżyser, scenarzysta i producent filmowy}
\Clue{37}{}{jednorazowa probówka wirówkowa wykonana z tworzywa sztucznego; używa się jej głównie w laboratoriach biochemicznych}
\Clue{38}{}{drogi przedmiot wykonany z użyciem metali i kamieni szlachetnych i półszlachetnych}
\Clue{39}{}{krzew z motylkowatych o złocistych kwiatach, w Polsce nad Bałtykiem}
\Clue{40}{}{zjazd przedstawicieli (np. naukowców, polityków)}
\Clue{41}{}{zjawisko polegające na wysyłaniu energii w postaci promieniowania elektromagnetycznego, akustycznego lub cząsteczek}
\Clue{42}{}{przestarzałe określenie torby, sakwy}
\Clue{43}{}{francuski pionier lotnictwa (1841-1925); zbudował pierwszy latający samolot z silnikiem parowym}
\Clue{44}{}{Alces alces - największy współcześnie żyjący gatunek ssaka kopytnego z rodziny jeleniowatych, wyróżniający się charakterystycznym porożem i wyjątkowo długimi kończynami}\end{PuzzleClues}\newpage\section*{Krzyżówka 126}

\noindent\begin{Puzzle}{22}{24}|*	|*	|*	|*	|*	|*	|[1][S]\drarr	|l	|o	|g	|o	|p	|a	|t	|o	|l	|o	|g	|i	|a	|*	|*	|*	|.
|*	|*	|*	|*	|[2][S]\darr	|[3][S]\drarr	|b	|r	|u	|t	|a	|l	|n	|o	|ś	|ć	|*	|*	|*	|*	|*	|*	|*	|.
|*	|*	|*	|[4][S]\rarr	|d	|ż	|o	|k	|e	|j	|k	|a	|*	|[5][S]\darr	|[6][S]\darr	|*	|*	|*	|*	|*	|[7][S]\darr	|*	|*	|.
|*	|*	|*	|*	|e	|ó	|j	|*	|*	|*	|[8][S]\drarr	|o	|w	|o	|c	|n	|i	|c	|a	|*	|ł	|*	|*	|.
|*	|*	|*	|*	|k	|ł	|e	|[9][S]\drarr	|b	|*	|ż	|*	|[10][S]\drarr	|l	|e	|w	|i	|s	|*	|*	|u	|[11][S]\darr	|[12][S]\darr	|.
|*	|*	|[13][S]\darr	|[14][S]\darr	|i	|w	|[][,]{ }	|t	|[15][S]\darr	|*	|a	|*	|k	|b	|l	|[16][S]\drarr	|p	|*	|*	|*	|p	|j	|k	|.
|*	|*	|b	|b	|e	|[][,]{ }	|h	|r	|p	|*	|r	|*	|o	|r	|*	|f	|*	|*	|*	|*	|e	|u	|i	|.
|*	|*	|r	|i	|l	|e	|o	|a	|e	|*	|t	|*	|m	|z	|*	|o	|*	|*	|[17][S]\darr	|*	|k	|r	|r	|.
|*	|*	|a	|z	|*	|g	|m	|n	|i	|*	|*	|*	|e	|y	|*	|u	|*	|*	|o	|*	|[][,]{ }	|a	|g	|.
|*	|[18][S]\rarr	|d	|o	|m	|i	|e	|s	|z	|k	|a	|*	|n	|m	|*	|q	|*	|*	|k	|[19][S]\darr	|m	|[][,]{ }	|i	|.
|*	|*	|l	|n	|[20][S]\darr	|p	|r	|p	|o	|*	|*	|[21][S]\darr	|t	|*	|*	|u	|*	|[22][S]\darr	|a	|h	|i	|b	|z	|.
|*	|*	|e	|i	|h	|s	|y	|o	|l	|*	|*	|m	|a	|*	|*	|e	|*	|g	|p	|i	|k	|r	|k	|.
|*	|*	|y	|ę	|i	|k	|c	|n	|d	|[23][S]\rarr	|z	|a	|r	|z	|u	|t	|k	|a	|*	|p	|o	|u	|a	|.
|*	|*	|*	|*	|p	|i	|k	|d	|*	|*	|*	|u	|z	|*	|*	|*	|[24][S]\drarr	|b	|r	|o	|w	|n	|*	|.
|[25][S]\rarr	|a	|n	|s	|a	|*	|i	|e	|[26][S]\rarr	|p	|l	|a	|y	|e	|r	|*	|j	|i	|*	|d	|y	|a	|*	|.
|*	|[27][S]\drarr	|f	|u	|r	|i	|e	|r	|y	|z	|m	|*	|k	|[28][S]\darr	|*	|*	|u	|n	|*	|r	|*	|t	|*	|.
|*	|r	|[29][S]\drarr	|n	|i	|n	|*	|*	|*	|*	|*	|*	|*	|d	|*	|*	|t	|e	|*	|o	|*	|n	|*	|.
|[30][S]\drarr	|z	|b	|r	|o	|j	|a	|[][,]{ }	|p	|ł	|y	|t	|k	|o	|w	|a	|*	|t	|[31][S]\darr	|m	|*	|a	|*	|.
|j	|ą	|a	|*	|n	|*	|[32][S]\rarr	|ł	|o	|w	|c	|a	|[][,]{ }	|g	|ł	|ó	|w	|*	|n	|*	|*	|*	|*	|.
|o	|d	|b	|*	|*	|*	|*	|*	|*	|*	|*	|[33][S]\rarr	|b	|l	|o	|k	|a	|d	|a	|*	|*	|*	|*	|.
|l	|*	|a	|*	|*	|*	|*	|[34][S]\rarr	|b	|a	|t	|a	|l	|i	|o	|n	|*	|*	|c	|*	|*	|*	|*	|.
|*	|*	|*	|*	|*	|[35][S]\rarr	|p	|i	|o	|c	|y	|j	|a	|n	|i	|n	|a	|*	|h	|*	|*	|*	|*	|.
|*	|*	|*	|*	|*	|*	|*	|*	|*	|*	|[36][S]\rarr	|p	|o	|g	|r	|o	|m	|*	|ó	|*	|*	|*	|*	|.
|*	|*	|*	|*	|*	|*	|*	|*	|*	|*	|*	|*	|*	|*	|*	|[37][S]\rarr	|b	|h	|d	|*	|*	|*	|*	|.
|*	|*	|*	|*	|*	|*	|*	|*	|*	|*	|*	|*	|*	|*	|*	|*	|*	|*	|*	|*	|*	|*	|*	|.\end{Puzzle}

\newpage

\begin{PuzzleClues}{\textbf{Poziome}\\}\Clue{1}{}{dział patologii badający zaburzenia mowy}
\Clue{3}{}{cecha człowieka: to, że ktoś jest agresywny, dopuszcza się przemocy}
\Clue{4}{}{półkolista czapeczka z daszkiem noszona najczęściej do konnej jazdy}
\Clue{8}{}{drobna błonkówka z rodziny pilarzy, szkodnik śliw}
\Clue{9}{}{w chemii: symbol boru}
\Clue{10}{}{Sinclair (1885-1951), pisarz amerykański, powieści społeczno-obyczajowe; „Babbit”, „Ulica główna” - Nobel 1930}
\Clue{16}{}{w chemii: symbol fosforu}
\Clue{18}{}{odcień}
\Clue{23}{}{narzutka, peleryna}
\Clue{24}{}{astronom amerykański (1866-1938) opracował analityczną teorię ruchu Księżyca}
\Clue{25}{}{nieprzyjazne uczucie wobec kogoś, uprzedzenie, pretensja}
\Clue{26}{}{program, aplikacja służąca do odtwarzania multimediów}
\Clue{27}{}{idea socjalizmu utopijnego opartego na tzw. falansterach, w których podział pracy miał być zgodny z tzw. namiętnościami, dzięki czemu praca miała przynosić radość}
\Clue{29}{}{(1903-77)   pisarka   amerykańska   pochodzenia   hiszpańskiego,   powieści psychoanalityczne}
\Clue{30}{}{zbroja wykonana z płyt metalowych}
\Clue{32}{}{headhunter - rodzaj rekrutera, pracownik dużych korporacji lub agencji pośrednictwa pracy, którego zadaniem jest identyfikacja szczególnie cenionych na rynku specjalistów i nawiązanie z nimi kontaktu w celu skłonienia ich do zmiany pracodawcy - na rzecz własnego pracodawcy lub klienta}
\Clue{33}{}{niezdolność kontynuowania lub podjęcia jakiegoś działania, procesu}
\Clue{34}{}{nowożytna jednostka organizacyjna wojska}
\Clue{35}{}{niebiesko-zielony barwnik wytwarzany przez pałeczkę ropy błękitnej, antybiotyk o właściwościach cytotoksycznych}
\Clue{36}{}{klęska, całkowite niepowodzenie}
\Clue{37}{}{kod ISO 4217 dinara Bahrajnu}\end{PuzzleClues}

\begin{PuzzleClues}{\textbf{Pionowe}\\}\Clue{1}{}{długotrwała, zacięta walka}
\Clue{2}{}{pokrywka, przykrywka, zamknięcie czegoś}
\Clue{3}{}{Testudo kleinmanni - gatunek gada z rodziny żółwi lądowych, podrzędu żółwi skrytoszyjnych, najmniejszy żółw lądowy, występujący u wybrzeży Morza Śródziemnego od zachodniej Libii do Izraela, krytycznie zagrożony wyginięciem}
\Clue{5}{}{istota z folkloru, baśniowa, mitologiczna (spotykana w licznych mitologiach świata), bardzo silna i duża postać przypominająca człowieka}
\Clue{6}{}{przeznaczenie czegoś, coś, czemu ma służyć dany przedmiot lub dana osoba}
\Clue{7}{}{skała metamorficzna, w skład której wchodzą przede wszystkim kwarc, skalenie i miki (łyszczyki)}
\Clue{8}{}{koncept, pewien atrakcyjny, dowcipny pomysł, który jest ukryty w czymś, co jest kreacją człowieka}
\Clue{9}{}{bezprzewodowe urządzenie komunikacyjne, które automatycznie odbiera, moduluje, wzmacnia i odpowiada na sygnał przychodzący w czasie rzeczywistym}
\Clue{10}{}{najczęściej z pobłażaniem: komentarz - tekst z wyjaśnieniem, omówieniem czegoś}
\Clue{11}{}{w chronostratygrafii drugi oddział systemu jurajskiego, wyższy od dolnej jury, a niższy od jury górnej}
\Clue{12}{}{mieszkanka Kirgistanu (Kirgizji), kobieta pochodzenia kirgiskiego}
\Clue{13}{}{astronom angielski (1693-1762), odkrył aberrację światła i nutację osi Ziemi}
\Clue{14}{}{młode bizona}
\Clue{15}{}{narciarka niemiecka, mistrzyni olimpijska Lake Placid w biegu na 10 km i w sztafecie 4x5 km}
\Clue{16}{}{hiszpański malarz i grafik (1834-74) sceny historyczne i rodzajowe}
\Clue{17}{}{wystająca część czegoś, która jest dla czegoś innego osłoną (np. okap hełmu)}
\Clue{19}{}{stadion, plac przeznaczony do wyścigów koni i zaprzęgów konnych}
\Clue{20}{}{kopalny koń wielkości kuca o lekkiej budowie i trójpalczastych kończynach}
\Clue{21}{}{miasto w Brazylii, stan Sao Paulo; rafinacja ropy naftowej}
\Clue{22}{}{członkowie rządu wraz z Radą Ministrów}
\Clue{24}{}{przedstawiciel starożytnego ludu germańskiego z obszaru Półwyspu Jutlandzkiego}
\Clue{27}{}{szacunkowe określenie liczby przybliżające jej wartość częścią całkowitą jej logarytmu dziesiętnego}
\Clue{28}{}{waleń z zębowców o długości do 9 m; tzw. wal butelkonosy}
\Clue{29}{}{przekupka}
\Clue{30}{}{żaglowiec 2-masztowy z ożaglowaniem skośnym; także ożaglowanie takiego statku}
\Clue{31}{}{zbrojne wtargnięcie na terytorium innego państwa, najazd}\end{PuzzleClues}\newpage\section*{Krzyżówka 127}

\noindent\begin{Puzzle}{21}{29}|*	|*	|*	|*	|[1][S]\drarr	|k	|o	|m	|p	|o	|z	|y	|c	|j	|a	|*	|*	|*	|*	|*	|*	|*	|.
|[2][S]\drarr	|o	|p	|e	|r	|a	|t	|*	|*	|[3][S]\darr	|*	|[4][S]\drarr	|z	|a	|p	|l	|o	|t	|*	|[5][S]\darr	|*	|*	|.
|c	|*	|*	|*	|u	|[6][S]\darr	|*	|*	|*	|o	|[7][S]\drarr	|p	|ó	|ł	|p	|r	|o	|f	|i	|l	|*	|*	|.
|z	|*	|[8][S]\darr	|*	|t	|g	|*	|*	|*	|d	|k	|r	|[9][S]\darr	|[10][S]\darr	|*	|[11][S]\drarr	|s	|y	|f	|o	|n	|*	|.
|o	|*	|k	|[12][S]\darr	|y	|ł	|*	|*	|*	|d	|a	|o	|l	|t	|*	|k	|*	|[13][S]\darr	|[14][S]\darr	|g	|*	|*	|.
|ł	|[15][S]\drarr	|r	|u	|n	|o	|*	|*	|*	|z	|l	|m	|o	|y	|[16][S]\darr	|w	|[17][S]\darr	|u	|c	|o	|*	|*	|.
|o	|s	|u	|l	|a	|s	|*	|*	|*	|i	|a	|i	|t	|l	|l	|i	|a	|s	|h	|s	|*	|*	|.
|[][,]{ }	|a	|g	|i	|*	|*	|*	|[18][S]\rarr	|l	|a	|n	|e	|[][,]{ }	|c	|i	|a	|s	|t	|o	|*	|*	|*	|.
|l	|m	|e	|c	|*	|*	|*	|*	|*	|ł	|d	|n	|ś	|z	|t	|t	|t	|a	|d	|*	|*	|*	|.
|o	|i	|r	|a	|[19][S]\drarr	|v	|e	|g	|a	|*	|r	|n	|l	|a	|o	|[][,]{ }	|r	|w	|z	|*	|*	|*	|.
|d	|k	|r	|*	|p	|[20][S]\darr	|*	|*	|*	|*	|a	|o	|i	|k	|g	|p	|o	|a	|ą	|*	|*	|*	|.
|o	|a	|a	|*	|i	|g	|[21][S]\darr	|*	|*	|[22][S]\darr	|*	|ś	|z	|*	|r	|u	|n	|[][,]{ }	|c	|*	|*	|*	|.
|w	|*	|n	|*	|r	|a	|g	|*	|*	|a	|*	|ć	|g	|*	|a	|ł	|o	|e	|y	|*	|*	|*	|.
|c	|[23][S]\drarr	|d	|w	|o	|j	|a	|c	|z	|k	|i	|*	|o	|*	|f	|a	|m	|p	|[][,]{ }	|*	|*	|*	|.
|a	|p	|*	|[24][S]\darr	|m	|*	|r	|*	|*	|c	|[25][S]\darr	|*	|w	|*	|i	|p	|i	|i	|k	|*	|*	|*	|.
|*	|r	|[26][S]\drarr	|f	|a	|r	|b	|a	|*	|e	|a	|*	|y	|[27][S]\darr	|a	|k	|a	|z	|o	|*	|*	|*	|.
|*	|z	|b	|r	|n	|*	|u	|*	|*	|n	|r	|*	|*	|h	|*	|o	|[][,]{ }	|o	|ś	|*	|*	|*	|.
|*	|e	|i	|a	|c	|*	|s	|*	|*	|t	|a	|*	|*	|e	|[28][S]\darr	|w	|s	|d	|c	|*	|*	|*	|.
|*	|c	|a	|m	|j	|*	|i	|*	|*	|[][,]{ }	|b	|*	|*	|r	|w	|y	|f	|y	|i	|*	|*	|*	|.
|*	|h	|ł	|u	|a	|*	|a	|[29][S]\drarr	|k	|s	|i	|ą	|ż	|k	|a	|*	|e	|c	|o	|*	|*	|*	|.
|*	|l	|y	|g	|*	|*	|r	|c	|*	|ł	|k	|*	|*	|u	|l	|*	|r	|z	|t	|[30][S]\darr	|*	|*	|.
|*	|e	|[][,]{ }	|a	|*	|*	|z	|*	|*	|a	|a	|*	|*	|l	|a	|*	|y	|n	|r	|k	|*	|*	|.
|*	|w	|w	|*	|*	|*	|*	|*	|*	|b	|*	|*	|*	|e	|n	|*	|c	|a	|u	|a	|*	|*	|.
|*	|i	|ę	|[31][S]\rarr	|k	|u	|l	|c	|z	|y	|k	|*	|*	|s	|s	|*	|z	|*	|p	|t	|*	|*	|.
|*	|a	|g	|[32][S]\rarr	|j	|e	|l	|c	|z	|*	|*	|*	|*	|*	|j	|*	|n	|*	|*	|e	|*	|*	|.
|[33][S]\drarr	|n	|i	|e	|r	|ó	|b	|k	|a	|*	|*	|*	|[34][S]\rarr	|s	|e	|r	|a	|j	|*	|r	|*	|*	|.
|r	|k	|e	|*	|*	|[35][S]\rarr	|a	|u	|t	|o	|r	|a	|m	|e	|n	|t	|*	|*	|*	|i	|*	|*	|.
|u	|a	|l	|*	|*	|[36][S]\rarr	|s	|u	|p	|e	|r	|m	|a	|r	|k	|e	|t	|*	|*	|n	|*	|*	|.
|m	|*	|*	|*	|*	|*	|*	|*	|*	|*	|*	|*	|[37][S]\rarr	|z	|a	|p	|r	|z	|ą	|g	|*	|*	|.
|*	|*	|[38][S]\rarr	|s	|p	|o	|c	|z	|n	|i	|e	|n	|i	|e	|*	|*	|*	|*	|*	|*	|*	|*	|.\end{Puzzle}

\newpage

\begin{PuzzleClues}{\textbf{Poziome}\\}\Clue{1}{}{proces tworzenia utworu muzycznego i nauka o tym procesie, dział muzykologii}
\Clue{2}{}{przedstawienie w formie opisowej stanu zagospodarowania lasu, sporządzane na okres 10 lat}
\Clue{4}{}{efekt zaplatania, np. włosów}
\Clue{7}{}{przedstawienie (ujęcie) twarzy w lekkim zwrocie w prawo lub w lewo}
\Clue{11}{}{odcinek kanału służący do pokonywania przeszkód terenowych, takich jak koryta rzeczne, niecki czy kolidujące z trasą kanału podziemne obiekty, poprowadzony pod tego typu przeszkodami}
\Clue{15}{}{okrywa z włókna, która jest charakterystyczna dla niektórych tkanin}
\Clue{18}{}{półpłynna masa na kluski lane}
\Clue{19}{}{poeta hiszpański (1503-36), wybitny humanista - sonety, kancony, elegie}
\Clue{23}{}{dzieci urodzone z ciąży bliźniaczej heterozygotycznej}
\Clue{26}{}{substancja barwiąca}
\Clue{29}{}{zapiski, dokument w postaci uzupełnianej książki}
\Clue{31}{}{ptak z  rodziny łuszczaków (Fringillidae)}
\Clue{32}{}{marka samochodów ciężarowych i autobusów; polskie przedsiębiorstwo motoryzacyjne z siedzibą w Jelczu-Laskowicach}
\Clue{33}{}{Anergates - rodzaj mrówki z podrodziny wścieklic, zawierający tylko jeden gatunek}
\Clue{34}{}{zamknięta część domu, przeznaczona wyłącznie dla kobiet z rodziny męża i żony}
\Clue{35}{}{zaciąg w dawnym wojsku polskim}
\Clue{36}{}{sklep o bardzo dużej powierzchni sprzedający szeroki asortyment towarów codziennego użytku, takich jak żywność, ubrania, kosmetyki, środki czyszczące itp}
\Clue{37}{}{zwierzęta zaprzężone do pojazdu, ciągnące pojazd}
\Clue{38}{}{odpoczynek}\end{PuzzleClues}

\begin{PuzzleClues}{\textbf{Pionowe}\\}\Clue{1}{}{organiczny związek pchodzenia naturalnego, należący do grupy flawonoidów}
\Clue{2}{}{najniżej położona krawędź lodowca, zawsze prostopadła do toru ruchu i znajdująca się w obszarze ablacji}
\Clue{3}{}{budynek przedstawicielstwa urzędu, kraju}
\Clue{4}{}{jasność, świetlistość, słoneczność}
\Clue{5}{}{wieloznaczne pojęcie filozoficzne, oznaczające m.in. rozum, słowo, myśl, prawo}
\Clue{6}{}{głos ludzki postrzegany jako instrument muzyczny, skala dźwiękowa, barwa i ustawienie charakterystyczne dla różnych typów śpiewu}
\Clue{7}{}{gatunek skowronka, pięknie śpiewa; Europa, Azja}
\Clue{8}{}{złota południowoafrykańska moneta bulionowa}
\Clue{9}{}{rodzaj lotu, w którym poruszający się obiekt nie jest w jego trakcie napędzany}
\Clue{10}{}{skórna postać nosacizny}
\Clue{11}{}{kwiat przystosowany do chwytania owadów i zatrzymywania ich w swoim wnętrzu tak długo, by dokonały zapylenia przyniesionym pyłkiem i zabrały dalej zebrany pyłek}
\Clue{12}{}{ludzie znajdujący na ulicy, mieszkańcy ulicy}
\Clue{13}{}{ustawa, która ma określony czas obowiązywania}
\Clue{14}{}{osoba bardzo chuda}
\Clue{15}{}{materiał elektroizolacyjny w postaci folii mikowej}
\Clue{16}{}{technika graficzna zaliczana do druku płaskiego, gdzie rysunek przeznaczony do powielania wykonuje się na kamieniu litograficznym}
\Clue{17}{}{dział astronomii zajmujący się matematycznym opisem widomych położeń ciał niebieskich i ich zmian, układami współrzędnych niebieskich itp}
\Clue{19}{}{wróżenie z ognia}
\Clue{20}{}{niewielki las lub grupa dziko rosnących drzew}
\Clue{21}{}{człowiek, który jeździ Volkswagenem Typu 1 (Volkswagenem Beetle zwanym potocznie garbusem), posiadacz, kierowca albo miłośnik tego modelu samochodu}
\Clue{22}{}{znak diakrytyczny używany w językach: greckim, katalońskim, wietnamskim, norweskim, portugalskim, francuskim, walijskim, włoskim i innych, oznaczający akcent samogłoski krótkiej o intonacji opadającej}
\Clue{23}{}{mieszkanka Przechlewa, kobieta pochodząca z Przechlewa - wsi w województwie pomorskim}
\Clue{24}{}{wnęka w ścianie; najczęściej wstawia się w nią okna lub drzwi}
\Clue{25}{}{porcja kawy, napoju, zaparzonego z arabiki (popularnego gatunku kawy)}
\Clue{26}{}{elektrownia wodna, która zamienia energię potencjalną wody na energię elektryczną}
\Clue{27}{}{chrząszcz z rodziny żukowatych: żyje w podzwrotnikowej Ameryce Płd}
\Clue{28}{}{wąska, cienka, bardzo delikatna koronka klockowa}
\Clue{29}{}{symbol chemiczny węgla}
\Clue{30}{}{usługa, która polega na dostarczaniu posiłków}
\Clue{33}{}{napój alkoholowy otrzymywany przez destylację melasy z trzciny cukrowej}\end{PuzzleClues}\newpage\section*{Krzyżówka 128}

\noindent\begin{Puzzle}{20}{33}|*	|*	|*	|[1][S]\darr	|*	|*	|*	|*	|*	|*	|*	|*	|*	|*	|*	|*	|*	|*	|*	|*	|*	|.
|*	|*	|*	|b	|[2][S]\drarr	|k	|a	|r	|m	|a	|n	|*	|*	|[3][S]\drarr	|e	|[][S]1	|[][S]2	|[][S]0	|*	|[4][S]\darr	|*	|.
|*	|*	|*	|a	|t	|[5][S]\rarr	|w	|n	|ę	|t	|r	|o	|s	|t	|w	|o	|*	|[6][S]\darr	|*	|k	|*	|.
|*	|[7][S]\darr	|[8][S]\drarr	|h	|e	|r	|b	|a	|t	|a	|*	|*	|*	|e	|*	|[9][S]\darr	|[10][S]\darr	|k	|*	|w	|*	|.
|*	|s	|s	|a	|r	|*	|*	|[11][S]\rarr	|w	|i	|o	|s	|ł	|o	|*	|ś	|b	|r	|*	|e	|*	|.
|*	|y	|e	|m	|a	|*	|*	|[12][S]\rarr	|o	|b	|r	|ę	|c	|z	|*	|m	|a	|a	|*	|f	|*	|.
|*	|s	|j	|y	|p	|*	|*	|*	|*	|*	|*	|[13][S]\drarr	|k	|o	|m	|i	|c	|j	|e	|*	|*	|.
|*	|t	|m	|*	|i	|*	|*	|*	|*	|*	|*	|c	|*	|f	|*	|g	|c	|a	|[14][S]\darr	|*	|*	|.
|*	|e	|i	|*	|a	|*	|*	|*	|*	|[15][S]\rarr	|l	|e	|w	|*	|*	|ł	|a	|n	|s	|*	|*	|.
|*	|m	|k	|*	|[][,]{ }	|*	|*	|[16][S]\rarr	|g	|w	|e	|b	|r	|*	|*	|o	|r	|*	|p	|*	|*	|.
|*	|[][,]{ }	|[][,]{ }	|*	|p	|*	|*	|*	|[17][S]\rarr	|g	|r	|u	|n	|t	|*	|ś	|t	|[18][S]\darr	|i	|*	|[19][S]\darr	|.
|*	|o	|w	|[20][S]\drarr	|r	|a	|c	|j	|o	|n	|a	|l	|n	|o	|ś	|ć	|*	|n	|r	|*	|s	|.
|[21][S]\drarr	|p	|o	|l	|e	|[][,]{ }	|b	|r	|o	|d	|m	|a	|n	|n	|a	|*	|*	|i	|y	|*	|t	|.
|o	|e	|j	|i	|n	|*	|*	|[22][S]\darr	|*	|*	|*	|[][,]{ }	|*	|*	|*	|*	|[23][S]\darr	|e	|t	|*	|o	|.
|g	|r	|e	|n	|a	|*	|[24][S]\drarr	|m	|e	|t	|a	|p	|l	|a	|n	|*	|u	|t	|u	|*	|j	|.
|o	|a	|w	|i	|t	|[25][S]\drarr	|k	|u	|l	|t	|u	|r	|a	|*	|*	|*	|r	|y	|s	|*	|o	|.
|n	|c	|ó	|e	|a	|t	|l	|s	|*	|*	|*	|z	|[26][S]\darr	|*	|*	|*	|a	|k	|[][,]{ }	|*	|w	|.
|ó	|y	|d	|[][,]{ }	|l	|r	|a	|e	|*	|*	|*	|y	|m	|*	|[27][S]\darr	|[28][S]\darr	|z	|a	|b	|*	|s	|.
|w	|j	|z	|o	|n	|ó	|s	|e	|*	|[29][S]\darr	|[30][S]\drarr	|b	|e	|i	|r	|a	|*	|l	|e	|*	|k	|.
|k	|n	|t	|c	|a	|j	|a	|u	|*	|i	|r	|y	|s	|*	|a	|t	|[31][S]\darr	|n	|z	|[32][S]\darr	|i	|.
|a	|y	|w	|e	|*	|l	|[][,]{ }	|w	|[33][S]\drarr	|m	|y	|s	|z	|[][,]{ }	|d	|o	|m	|o	|w	|a	|*	|.
|[][,]{ }	|*	|a	|a	|*	|i	|ś	|*	|p	|i	|t	|z	|e	|*	|a	|n	|a	|ś	|o	|u	|*	|.
|b	|*	|*	|n	|*	|s	|r	|*	|i	|g	|m	|o	|k	|*	|*	|a	|m	|ć	|d	|t	|*	|.
|i	|*	|*	|i	|*	|t	|e	|*	|e	|r	|*	|w	|*	|*	|*	|l	|u	|[][,]{ }	|n	|o	|*	|.
|a	|*	|*	|c	|*	|[][,]{ }	|d	|*	|r	|a	|*	|a	|*	|*	|*	|n	|t	|c	|y	|e	|*	|.
|ł	|*	|*	|z	|*	|w	|n	|*	|w	|c	|*	|*	|*	|*	|[34][S]\darr	|o	|a	|i	|*	|k	|*	|.
|o	|*	|*	|n	|[35][S]\darr	|c	|i	|*	|o	|j	|*	|*	|*	|*	|t	|ś	|k	|e	|*	|s	|*	|.
|l	|*	|*	|e	|c	|z	|a	|*	|b	|a	|*	|*	|*	|*	|a	|ć	|i	|l	|*	|p	|*	|.
|i	|*	|[36][S]\darr	|*	|h	|e	|*	|*	|y	|*	|[37][S]\rarr	|t	|r	|y	|k	|*	|*	|e	|*	|r	|*	|.
|c	|*	|s	|*	|a	|s	|*	|*	|t	|*	|*	|[38][S]\rarr	|b	|i	|e	|g	|u	|s	|*	|e	|*	|.
|a	|*	|y	|*	|r	|n	|*	|*	|*	|*	|*	|*	|*	|*	|r	|*	|*	|n	|*	|s	|*	|.
|*	|[39][S]\rarr	|f	|e	|t	|y	|s	|z	|y	|z	|m	|*	|*	|*	|*	|*	|*	|a	|*	|j	|*	|.
|*	|*	|*	|*	|*	|*	|*	|*	|*	|*	|*	|*	|*	|*	|*	|*	|*	|*	|*	|a	|*	|.
|[40][S]\rarr	|s	|y	|s	|t	|e	|m	|[][,]{ }	|s	|z	|w	|a	|j	|c	|a	|r	|s	|k	|i	|*	|*	|.\end{Puzzle}

\newpage

\begin{PuzzleClues}{\textbf{Poziome}\\}\Clue{2}{}{aerodynamik amerykański (1881-1963); specjalista w zakresie przepływów z wielkimi prędkościami w zastosowaniu do lotnictwa i techniki rakietowej}
\Clue{3}{}{skrót, alternatywna nazwa handlowa kwasu karminowego}
\Clue{5}{}{wada wrodzona u mężczyzn i samców ssaków, polegająca na niewłaściwym umieszczeniu jednego lub obu jąder w jamie brzusznej lub kanale pachwinowym zamiast w mosznie}
\Clue{8}{}{porcja herbaty, zazwyczaj naczynie ze świeżo zaparzonym naparem, rzadziej w Polsce: butelka czy puszka z kupnym zimnym napojem}
\Clue{11}{}{ręczny pędnik łodzi wykonany z drewna, tworzyw sztucznych, lekkich stopów metali lub kombinacji tych materiałów}
\Clue{12}{}{okrągły, niebędący okuciem element konstrukcyjny, na którym oparta została budowa jakiegoś przedmiotu lub przyrządu}
\Clue{13}{}{zgromadzenie ludowe w starożytnym Rzymie}
\Clue{15}{}{gwiazdozbiór zodiakalny; znak Zodiaku}
\Clue{16}{}{irański wyznawca mazdaizmu}
\Clue{17}{}{belgijski malarz i grafik (1825-70) sceny z życia ludu, obrazy historyczne}
\Clue{20}{}{sensowność, zgodność z zasadami logicznego myślenia}
\Clue{21}{}{jeden z wyróżnionych  na podstawie badań cytoarchitektonicznych obszarów kory mózgu u człowieka i naczelnych}
\Clue{24}{}{metoda dyskusji i jej zapisu polegająca na wspólnym tworzeniu przez uczestników plakatu stanowiącego graficzne przedstawienie debaty}
\Clue{25}{}{zbiór działalności człowieka związanych z rozwojem duchowym, najczęściej ze sztuką}
\Clue{30}{}{miasto; ważny port w Mozambiku nad Kanałem Mozambickim, połączenie kolejowe z Angolą}
\Clue{33}{}{Mus musculus - gatunek gryzonia z rodziny myszowatych, prawdopodobnie pochodzącego od myszy zamieszkującej stepy i tereny półpustynne od północnej Afryki, poprzez południowo-wschodnią część Europy aż po Wyspy Japońskie; obecnie znajduje się ją wszędzie tam, gdzie żyje człowiek (nawet na stacjach antarktycznych)}
\Clue{37}{}{baran - dorosły samiec owcy}
\Clue{38}{}{ptak z rzędu mew-siewek, chroniony, podobny do brodźców; Europa, Ameryka}
\Clue{39}{}{kult fetyszów w wierzeniach pierwotnych}
\Clue{40}{}{sposób rozgrywania zawodów sportowych w dyscyplinach, w których rywalizacja polega na bezpośrednich pojedynkach pomiędzy uczestnikami}\end{PuzzleClues}

\begin{PuzzleClues}{\textbf{Pionowe}\\}\Clue{1}{}{państwo wyspiarskie na Oceanie Atlantyckim}
\Clue{2}{}{terapia, której działania mają pomóc w wyleczeniu wad i schorzeń dziecka przed urodzeniem}
\Clue{3}{}{okultysta, przedstawiciel teozofii, członek ruchu religijnego opartego na koncepcji, która mówi o tym, że istnieje możliwość bezpośredniego kontaktu ze światem nadprzyrodzonym dzięki wiedzy zdobytej za pomocą praktyk mistycznych}
\Clue{4}{}{dawne nakrycie głowy zakonnic, wdów i starszych kobiet}
\Clue{6}{}{człowiek pochodzący z tej samej okolicy}
\Clue{7}{}{oprogramowanie zarządzające systemem komputerowym, tworzące środowisko do uruchamiania i kontroli zadań użytkownika}
\Clue{8}{}{organ stanowiący i kontrolny samorządu województwa, który tworzą radni, wybierani w wyborach bezpośrednich}
\Clue{9}{}{to, że coś jest wysokie i cienkie, lub że ktoś jest wysoki i chudy; szczupłość, wysmukłość}
\Clue{10}{}{miejscowość w płd.-wsch. Francji, znana z wyrobów ze szkła kryształowego}
\Clue{13}{}{rodzaj cebulki, wykorzystywany przy rozmnażaniu wegetatywnym roślin cebulowych, inaczej: cebulka przybyszowa}
\Clue{14}{}{alkohol etylowy o niewielkiej zawartości wody (0,2\%)}
\Clue{18}{}{naturalne prawo do decydowania o swoim ciele, przysługujące każdemu człowiekowi; termin prawniczy}
\Clue{19}{}{kompozytor, pianista i pedagog (1870-1946); od 1906 r. w USA; koncerty, utwory symfoniczne, kameralne. pieśni}
\Clue{20}{}{przedsiębiorstwo zajmujące się działalnością żeglugową}
\Clue{21}{}{Turacoena manadensis - gatunek ptaka z rodziny gołębiowatych (Columbidae)}
\Clue{22}{}{zawodowy kolarz belgijski, członek grupy Mapei GB}
\Clue{23}{}{reakcja psychiczna, negatywne wyobrażenie o czymś lub o kimś, uprzedzenie do czegoś lub kogoś, przeczulenie, które jest wynikiem przykrych doświadczeń}
\Clue{24}{}{klasa społeczna, do której należą przede wszystkim osoby o wyższym wykształceniu oraz o wyższych kompetencjach kulturowych niż klasa robotnicza (niższa), ale nieposiadające znacznego majątku ani arystokratycznego pochodzenia}
\Clue{25}{}{Trillium kurabayashii - gatunek rośliny zielnej z rodziny melantkowatych}
\Clue{26}{}{krótkie, gęste włoski na liściach, łodygach lub owocach roślin}
\Clue{27}{}{lokalne zgromadzenie robotnicze, chłopskie i żołnierskie, na początku dwudziestego wieku, w okresie ostatnich lat dynastii Romanowów}
\Clue{28}{}{brak zależności poszczególnych dźwięków i akordów od dźwięku centralnego np. toniki}
\Clue{29}{}{przyjazd ludności napływowej do jakiegoś państwa}
\Clue{30}{}{ugrupowanie artystyczne działające w latach 1922-32 w Warszawie, skupiało malarzy, grafików i rzeźbiarzy}
\Clue{31}{}{Aepyornithiformes - rząd ptaków paleognatycznych, wymarłych w czasach historycznych, prawdopodobnie pomiędzy X a XVI w.; były endemitami Madagaskaru}
\Clue{32}{}{wyrażenie samego siebie, fakt okazywania włąsnych uczuć, emocjii, przemyśleń, charakteru}
\Clue{33}{}{w koncepcji mistyczno-idealistycznej: sfera przebywania dusz istniejących przed przyjściem tych istot na świat}
\Clue{34}{}{pistolet na zszywki}
\Clue{35}{}{jeden z najstarszych typów psa, o charakterystycznej aerodynamicznej budowie ciała, umożliwiająca osiąganie dużej prędkości w pogoni za zwierzyną, podczas której psy te kierują się wzrokiem, a nie węchem; pod względem użytkowym zaliczane są do psów myśliwskich - gończych, ale organizacje kynologiczne klasyfikują je w odrębnej grupie}
\Clue{36}{}{pryszcz}\end{PuzzleClues}\newpage\section*{Krzyżówka 129}

\noindent\begin{Puzzle}{22}{32}|*	|*	|[1][S]\darr	|[2][S]\drarr	|t	|ę	|p	|o	|s	|z	|[][,]{ }	|n	|a	|d	|b	|r	|z	|e	|ż	|n	|y	|*	|*	|.
|*	|[3][S]\darr	|k	|a	|[4][S]\darr	|[5][S]\darr	|[6][S]\drarr	|p	|o	|t	|a	|l	|a	|*	|*	|[7][S]\drarr	|p	|a	|ł	|k	|a	|*	|*	|.
|[8][S]\rarr	|c	|o	|n	|f	|i	|t	|*	|[9][S]\darr	|[10][S]\rarr	|n	|a	|b	|i	|o	|d	|r	|n	|i	|k	|*	|*	|*	|.
|*	|z	|e	|g	|r	|n	|y	|*	|l	|*	|[11][S]\rarr	|d	|u	|m	|b	|a	|d	|z	|e	|*	|*	|*	|*	|.
|*	|e	|l	|i	|o	|w	|f	|*	|a	|*	|*	|*	|*	|[12][S]\rarr	|t	|i	|c	|h	|o	|n	|o	|w	|*	|.
|*	|r	|*	|e	|n	|a	|u	|[13][S]\drarr	|w	|a	|n	|d	|a	|l	|*	|m	|*	|*	|*	|*	|*	|*	|*	|.
|*	|n	|*	|l	|t	|l	|s	|z	|a	|*	|[14][S]\drarr	|n	|e	|b	|u	|l	|i	|z	|a	|c	|j	|a	|*	|.
|*	|i	|*	|s	|i	|i	|[][,]{ }	|a	|t	|*	|p	|[15][S]\rarr	|l	|i	|t	|e	|w	|s	|k	|o	|ś	|ć	|*	|.
|*	|e	|*	|k	|s	|d	|p	|p	|e	|[16][S]\rarr	|s	|o	|l	|b	|e	|r	|g	|*	|*	|*	|*	|*	|*	|.
|*	|j	|*	|o	|p	|a	|l	|ł	|r	|*	|z	|*	|*	|*	|[17][S]\darr	|*	|*	|*	|*	|*	|*	|*	|[18][S]\darr	|.
|*	|e	|*	|ś	|i	|[][,]{ }	|a	|o	|z	|*	|c	|[19][S]\drarr	|m	|a	|c	|z	|a	|n	|k	|a	|*	|*	|b	|.
|*	|w	|*	|ć	|s	|w	|m	|d	|*	|[20][S]\darr	|z	|a	|[21][S]\rarr	|s	|z	|y	|m	|u	|r	|a	|*	|*	|ł	|.
|*	|o	|*	|*	|*	|o	|i	|n	|[22][S]\drarr	|k	|o	|l	|a	|g	|e	|n	|o	|z	|a	|*	|[23][S]\darr	|*	|a	|.
|*	|*	|*	|*	|*	|j	|s	|i	|ś	|a	|ł	|a	|*	|[24][S]\darr	|k	|*	|*	|*	|*	|[25][S]\darr	|o	|*	|h	|.
|*	|*	|*	|*	|[26][S]\drarr	|s	|t	|e	|l	|m	|a	|s	|z	|k	|a	|*	|*	|[27][S]\rarr	|t	|o	|p	|*	|o	|.
|*	|*	|*	|*	|ż	|k	|y	|n	|i	|a	|[][,]{ }	|k	|*	|o	|d	|[28][S]\darr	|*	|[29][S]\darr	|*	|l	|e	|*	|s	|.
|*	|*	|*	|[30][S]\darr	|y	|o	|*	|i	|w	|*	|s	|a	|*	|m	|e	|g	|*	|w	|[31][S]\darr	|e	|r	|*	|t	|.
|*	|*	|*	|p	|z	|w	|*	|e	|ó	|*	|k	|c	|*	|ó	|ł	|r	|*	|c	|e	|j	|a	|[32][S]\darr	|k	|.
|*	|*	|*	|r	|n	|y	|*	|*	|w	|[33][S]\darr	|a	|e	|*	|r	|k	|a	|[34][S]\darr	|i	|u	|[][,]{ }	|c	|z	|a	|.
|*	|*	|*	|e	|o	|*	|*	|*	|k	|ś	|l	|f	|[35][S]\rarr	|k	|o	|n	|w	|e	|r	|s	|j	|a	|*	|.
|*	|[36][S]\darr	|*	|m	|ś	|*	|*	|*	|a	|w	|i	|a	|*	|a	|*	|i	|i	|l	|o	|i	|a	|k	|*	|.
|*	|f	|*	|i	|ć	|*	|*	|*	|*	|i	|s	|l	|[37][S]\darr	|*	|*	|c	|e	|e	|p	|l	|[][,]{ }	|r	|*	|.
|*	|e	|*	|a	|*	|*	|*	|*	|*	|ę	|t	|*	|o	|*	|*	|a	|l	|n	|o	|n	|p	|z	|*	|.
|*	|t	|*	|[][,]{ }	|[38][S]\rarr	|c	|e	|f	|o	|t	|a	|k	|s	|y	|m	|*	|o	|i	|s	|i	|l	|e	|*	|.
|*	|y	|[39][S]\rarr	|g	|a	|l	|a	|r	|d	|a	|*	|*	|t	|*	|*	|*	|m	|e	|ł	|k	|a	|w	|*	|.
|[40][S]\rarr	|s	|w	|ó	|j	|*	|*	|*	|*	|[][,]{ }	|*	|*	|r	|*	|*	|*	|i	|*	|a	|o	|s	|s	|*	|.
|[41][S]\rarr	|z	|m	|r	|o	|c	|z	|n	|i	|k	|*	|*	|o	|*	|*	|*	|a	|*	|n	|w	|t	|k	|*	|.
|*	|y	|[42][S]\rarr	|s	|o	|p	|e	|l	|*	|s	|*	|*	|ż	|*	|*	|*	|n	|*	|k	|y	|y	|i	|*	|.
|*	|z	|*	|k	|*	|[43][S]\rarr	|o	|b	|l	|i	|c	|z	|n	|o	|ś	|ć	|*	|*	|a	|*	|c	|*	|*	|.
|*	|m	|*	|a	|*	|*	|*	|*	|*	|ę	|[44][S]\rarr	|k	|o	|m	|a	|n	|d	|o	|*	|*	|z	|*	|*	|.
|*	|*	|*	|*	|*	|*	|*	|*	|*	|g	|*	|*	|ś	|*	|[45][S]\rarr	|t	|h	|o	|m	|o	|n	|*	|*	|.
|*	|*	|[46][S]\rarr	|m	|ę	|ż	|a	|t	|k	|a	|*	|*	|ć	|[47][S]\rarr	|p	|i	|w	|o	|n	|i	|a	|*	|*	|.
|[48][S]\rarr	|p	|o	|d	|j	|ę	|z	|y	|k	|*	|*	|*	|*	|*	|*	|*	|*	|*	|*	|*	|*	|*	|*	|.\end{Puzzle}

\newpage

\begin{PuzzleClues}{\textbf{Poziome}\\}\Clue{2}{}{Leptodictyum riparium - gatunek mchu z rodziny krzywoszyjowatych}
\Clue{6}{}{siedziba Dalajlamy w Tybecie}
\Clue{7}{}{mięso z nóżki drobiowej, samo udko z kostką}
\Clue{8}{}{francuska potrawa, powstająca przez powolne gotowanie w tłuszczu; tradycyjnie z mięsa, ale można w ten sposób przygotowywać warzywa czy żółtka jaj}
\Clue{10}{}{opaska ze skóry, tkaniny lub innego materiału noszona na biodrach u ludów pierwotnych, żyjących w gorącym klimacie}
\Clue{11}{}{lekkoatletka radz., rekordzistka świata z 1954 r}
\Clue{12}{}{biatlonista radziecki, mistrz olimpijski z Grenoble, Sapporo, Innsbrucku i Lake Placid}
\Clue{13}{}{osoba, która niszczy przedmioty, demoluje otoczenie (zazwyczaj bezmyślnie, z samej chęci niszczenia)}
\Clue{14}{}{rodzaj zabiegu medycznego, który polega na podawaniu pacjentowi płynnych leków w postaci rozpylonej mgiełki (aerozolu) metodą inhalacji poprzez usta, rurkę intubacyjną lub przez otwór tracheotomijny - bezpośrednio do układu oddechowego}
\Clue{15}{}{zespół cech czegoś lub kogoś takiego jak na Litwie, także: stereotypowe cechy uznawane za właściwe Litwinom}
\Clue{16}{}{dwuboista norweski, dwukrotny mistrz olimpijski z Grenoble, Sapporo}
\Clue{19}{}{potrawa białoruska, rodzaj mięsnego gulaszu, który je się, maczając w nim pieczywo, kawałki placków ziemniaczanych lub blinów}
\Clue{21}{}{pięściarz, w okresie międzywojennym dwukrotny wicemistrz Europy w kategorii półciężkiej}
\Clue{22}{}{choroba, w której zmiana patologiczna pierwotnie występuje w tkance łącznej; termin wychodzący z użycia w medycynie}
\Clue{26}{}{kobieta-stelmach}
\Clue{27}{}{szczyt sławy, bycia modnym, apogeum sukcesu}
\Clue{35}{}{zamiana informacji zapisanych w jednym formacie na inny}
\Clue{38}{}{półsyntetyczny antybiotyk, będący cefalosporyną III generacji o działaniu bakteriobójczym do stosowania pozajelitowego}
\Clue{39}{}{FORKASZTEL}
\Clue{40}{}{człowiek, który jest rozpoznawany jako należący do tej samej społeczności lub inaczej definiowanej grupy, co inni}
\Clue{41}{}{WILCZOMLECZEK; motyl nocny z rodziny zawisaków, gąsienice żerują na wilczomleczu}
\Clue{42}{}{naciek zwisający z krawędzi dachu, gałęzi lub innego podobnego miejsca, utworzony z zamarzniętej wody; powstaje najczęściej wówczas, kiedy temperatura powietrza wynosi nieco poniżej 0°C, a promieniowanie słoneczne powoduje ogrzewanie się śniegu lub lodu leżącego na połaci dachu bądź na gałęziach drzewa i w rezultacie - jego topnienie i ściekanie, podczas którego część wody ponownie zamarza}
\Clue{43}{}{czyjaś sylwetka, zwłaszcza kogoś ważnego; używane w sformułowaniuprzed oblicznością, np. stawić się przed czyjąś oblicznością}
\Clue{44}{}{elitarna jednostka wojskowa przeznaczona do prowadzenia działań specjalnych}
\Clue{45}{}{architekt rosyjski szwajcarskiego pochodzenia (1760-1813), przedstawiciel klasycyzmu}
\Clue{46}{}{kobieta zamężna}
\Clue{47}{}{PEONIA bylina ozdobna uprawiana na kwietnikach i na kwiat cięty}
\Clue{48}{}{jakaś odmiana języka}\end{PuzzleClues}

\begin{PuzzleClues}{\textbf{Pionowe}\\}\Clue{1}{}{ptak z rodziny kukułek, zamieszkujący ogrody i lasy płn. wsch. Azji i Wysp Sundajskich; samica podrzuca jaja do gniazd innych ptaków}
\Clue{2}{}{zespół cech czegoś lub kogoś takiego jak w Anglii, także: stereotypowe cechy uznawane za właściwe Anglikom}
\Clue{3}{}{miasto w województwie wielkopolskim, w powiecie gnieźnieńskim, siedziba gminy miejsko-wiejskiej Czerniejewo}
\Clue{4}{}{ozdobna karta tytułowa wykonana techniką miedziorytniczą występująca w starych drukach}
\Clue{5}{}{żołnierz niezawodowy Sił Zbrojnych Rzeczypospolitej Polskiej, który stał się inwalidą w czasie odbywania czynnej służby wojskowej w okresie pokoju lub w ciągu 3 lat od zwolnienia z tej służby}
\Clue{6}{}{osoba nielubiana, taka, której trudno się pozbyć}
\Clue{7}{}{niemiecki konstruktor i przemysłowiec (1834-1900); zbudował szybkoobrotowy silnik spalinowy a następnie motocykl i samochód}
\Clue{9}{}{budynek w klasztornym wirydażu, mieszczący studnię, służący do celów higienicznych, gospodarczych i obrzędowych}
\Clue{13}{}{połączenie się komórek rozrodczych (komórki męskiej i żeńskiej) w wyniku czego powstaje nowa komórka nazywana zygotą}
\Clue{14}{}{pszczoła himalajska, Apis dorsata laboriosa - największa z pszczół; pszczoła ta miejsc swego bytowania wyszukuje w niedostępnych zagłębieniach skalnych, na takich wysokościach, gdzie dzięki swoim rozmiarom i gęstemu owłosieniu może wytrzymać trudne warunki klimatyczne}
\Clue{17}{}{mała przekąska, którą dostaje się gratis w restauracji po zajęciu stolika, by umilała czas oczekiwania na zamówienie}
\Clue{18}{}{rzecz, sprawa mało ważna, błaha}
\Clue{19}{}{Alaskacephale - rodzaj roślinożernego dinozaura z rodziny pachycefalozaurów, żyjący w okresie późnej kredy na terenach Ameryki Północnej; długość ciała 4,5 m, wysokość 1,5 m, ciężar 430 kg}
\Clue{20}{}{gatunek antylopy o brązowoszarym ubarwieniu; występuje w rezerwatach południowej Afryki}
\Clue{22}{}{nalewka o smaku śliwek}
\Clue{23}{}{zabieg chirurgiczny przeprowadzany w celu korekty defektów ciała}
\Clue{24}{}{w systemach telekomunikacyjnej łączności bezprzewodowej to obszar dominacji sygnału radiowego emitowanego przez jedną stację przekaźnikową}
\Clue{25}{}{czynnik smarujący w silnikach spalinowych pozyskiwany w wyniku destylacji ropy naftowej}
\Clue{26}{}{cecha zbiornika wodnego decydująca o zamieszkiwaniu tego zbiornika przez różne organizmy żywe}
\Clue{28}{}{kres}
\Clue{29}{}{absorpcja jednej firmy przez drugą, co oznacza przejęcie zarówno jej aktywów, jak i pasywów}
\Clue{30}{}{podjazd, miejsce, gdzie rozgrywa się sprint kolarzy pod górę w wyścigu kolarskim}
\Clue{31}{}{posłanka wybrana do Parlamentu Europejskiego}
\Clue{32}{}{Stanisław, ojciec Witkacego (1851-1915) krytyk, pisarz i malarz; obrazy rodzajowe, pejzaże portrety}
\Clue{33}{}{pismo, które uznawane jest za źródło prawd religijnych}
\Clue{34}{}{wyrażenie algebraiczne złożone ze zmiennych i stałych, połączonych działaniami dodawania, odejmowania, mnożenia i podnoszenia do potęgi o stałym wykładniku naturalnym}
\Clue{36}{}{kult fetyszów w wierzeniach pierwotnych}
\Clue{37}{}{cecha człowieka rozważnie podejmującego działanie, nieskorego do szybkich czynów}\end{PuzzleClues}\newpage\section*{Krzyżówka 130}

\noindent\begin{Puzzle}{24}{25}|*	|*	|*	|*	|*	|[1][S]\darr	|*	|*	|*	|[2][S]\drarr	|p	|ę	|d	|*	|[3][S]\drarr	|ł	|a	|c	|i	|n	|n	|i	|k	|*	|*	|.
|*	|*	|[4][S]\darr	|*	|*	|n	|[5][S]\rarr	|c	|i	|o	|s	|[][,]{ }	|p	|r	|o	|s	|t	|y	|*	|*	|*	|*	|*	|*	|*	|.
|*	|*	|c	|[6][S]\darr	|*	|e	|*	|[7][S]\rarr	|u	|r	|i	|*	|*	|[8][S]\rarr	|g	|a	|l	|e	|c	|h	|i	|r	|u	|s	|*	|.
|*	|*	|h	|b	|*	|r	|*	|*	|[9][S]\rarr	|d	|o	|j	|ś	|c	|i	|e	|*	|*	|*	|*	|*	|*	|*	|*	|*	|.
|*	|[10][S]\darr	|w	|u	|[11][S]\drarr	|p	|e	|r	|s	|y	|j	|k	|a	|*	|e	|*	|*	|*	|*	|*	|*	|*	|*	|*	|*	|.
|[12][S]\drarr	|p	|o	|d	|l	|a	|s	|i	|a	|n	|k	|a	|*	|*	|r	|*	|[13][S]\darr	|*	|[14][S]\darr	|*	|[15][S]\darr	|[16][S]\darr	|*	|[17][S]\darr	|*	|.
|r	|i	|j	|a	|a	|*	|*	|[18][S]\rarr	|k	|a	|p	|ł	|o	|n	|*	|*	|f	|*	|r	|*	|s	|o	|[19][S]\darr	|k	|*	|.
|o	|k	|a	|*	|u	|*	|*	|*	|*	|c	|*	|*	|*	|[20][S]\darr	|[21][S]\drarr	|k	|i	|t	|a	|*	|z	|g	|e	|i	|*	|.
|ś	|e	|*	|*	|f	|[22][S]\rarr	|f	|r	|a	|j	|e	|r	|*	|a	|v	|*	|t	|*	|k	|*	|c	|r	|r	|l	|*	|.
|l	|*	|[23][S]\darr	|*	|e	|[24][S]\drarr	|t	|u	|j	|a	|*	|*	|*	|z	|a	|*	|o	|*	|[][,]{ }	|[25][S]\darr	|z	|o	|g	|l	|*	|.
|i	|[26][S]\rarr	|n	|e	|r	|w	|*	|*	|*	|[][,]{ }	|*	|*	|[27][S]\darr	|o	|c	|[28][S]\darr	|r	|[29][S]\darr	|n	|z	|ę	|d	|o	|e	|*	|.
|n	|*	|a	|*	|[][,]{ }	|i	|[30][S]\darr	|[31][S]\drarr	|a	|p	|a	|r	|a	|t	|*	|k	|e	|o	|e	|a	|k	|o	|d	|r	|*	|.
|a	|*	|s	|[32][S]\drarr	|b	|e	|z	|t	|o	|r	|b	|i	|k	|[][,]{ }	|b	|a	|m	|b	|u	|s	|o	|w	|y	|*	|*	|.
|[][,]{ }	|*	|t	|d	|i	|r	|a	|r	|*	|o	|*	|*	|t	|a	|*	|r	|e	|m	|r	|a	|c	|i	|c	|*	|*	|.
|ż	|*	|a	|e	|a	|z	|w	|ó	|*	|p	|[33][S]\rarr	|ł	|o	|m	|*	|u	|d	|u	|o	|d	|z	|z	|z	|*	|*	|.
|y	|*	|w	|w	|ł	|c	|ó	|j	|*	|o	|*	|*	|r	|o	|*	|z	|i	|r	|e	|n	|u	|n	|n	|*	|*	|.
|w	|*	|a	|a	|o	|h	|d	|l	|*	|r	|*	|*	|z	|n	|*	|e	|a	|o	|n	|i	|ł	|a	|o	|*	|*	|.
|i	|*	|n	|n	|p	|n	|*	|i	|*	|c	|*	|*	|y	|o	|*	|l	|c	|w	|d	|c	|k	|*	|ś	|*	|*	|.
|c	|*	|i	|a	|o	|i	|*	|ś	|*	|j	|*	|*	|n	|w	|*	|*	|j	|a	|o	|z	|o	|*	|ć	|*	|*	|.
|i	|*	|e	|g	|l	|c	|*	|ć	|*	|o	|*	|*	|a	|y	|*	|*	|a	|n	|k	|o	|w	|*	|*	|*	|*	|.
|e	|*	|*	|a	|o	|a	|*	|*	|*	|n	|*	|*	|*	|*	|*	|*	|*	|i	|r	|ś	|i	|*	|*	|*	|*	|.
|l	|*	|*	|r	|w	|*	|[34][S]\rarr	|c	|z	|a	|k	|m	|a	|*	|*	|*	|*	|e	|y	|ć	|e	|*	|*	|*	|*	|.
|s	|*	|*	|i	|y	|*	|[35][S]\rarr	|w	|i	|l	|k	|o	|s	|z	|*	|*	|*	|*	|n	|*	|c	|*	|*	|*	|*	|.
|k	|*	|*	|*	|*	|*	|*	|*	|*	|n	|*	|[36][S]\rarr	|m	|a	|f	|i	|k	|e	|n	|g	|*	|*	|*	|*	|*	|.
|a	|[37][S]\rarr	|p	|s	|z	|e	|n	|i	|c	|a	|[][,]{ }	|d	|u	|r	|u	|m	|*	|*	|y	|*	|*	|*	|*	|*	|*	|.
|*	|*	|[38][S]\rarr	|e	|o	|n	|i	|z	|m	|*	|*	|*	|*	|[39][S]\rarr	|d	|o	|l	|e	|*	|*	|*	|*	|*	|*	|*	|.\end{Puzzle}

\newpage

\begin{PuzzleClues}{\textbf{Poziome}\\}\Clue{2}{}{fizyczna wielkość wektorowa, której wartość jest równa iloczynowi masy i prędkości punktu materialnego}
\Clue{3}{}{nauczyciel, który uczy łaciny}
\Clue{5}{}{cios bokserski charakteryzujące się duża szybkością i skutecznością, stosowany w ataku na dystans i w półdystansie oraz jako kontrcios w obronie}
\Clue{7}{}{kanton w środkowej Szwajcarii w Alpach, stolica Altdore, powierzchnia 1,1 tyś. km2}
\Clue{8}{}{Galechirus scholtzi - rodzaj jaszczurkopodobnego terapsyda z grupy anomodontów; żył w okresie późnego permu na terenach obecnej południowej Afryki}
\Clue{9}{}{o środkach komunikacji: dojechać, dopłynąć, zostać doprowadzonym}
\Clue{11}{}{mieszkanka Persji, od 1935 roku przemianowanej na Iran}
\Clue{12}{}{mieszkanka Podlasia, kobieta pochodzenia podlaskiego}
\Clue{18}{}{deprecjonująco, obraźliwie o wykastrowanym mężczyźnie}
\Clue{21}{}{FORGA}
\Clue{22}{}{drobnostka, błahostka, coś niewymagającego trudu, nieistotnego}
\Clue{24}{}{żywotnik, Thuja - rodzaj roślin iglastych z rodziny cyprysowatych; obejmuje 5 gatunków drzew i krzewów, z których 3 występują na terenie Azji (Korea, Chiny, Japonia), a 2 w Ameryce Północnej}
\Clue{26}{}{energia, emocja, którą ktoś lub coś przejawia}
\Clue{31}{}{urządzenie, które spełnia określone zadanie w wyniku przebiegających w nim procesów fizycznych lub chemicznych}
\Clue{32}{}{Dromiciops gliroides - południowoamerykański torbacz z rodziny beztorbikowatych, jedyny współcześnie żyjący przedstawiciel rzędu beztorbików; występuje w gęstych, górskich lasach Chile i Argentyny, zwłaszcza wśród bambusów Chusquea}
\Clue{33}{}{drąg stalowy na końcu zaostrzony lub spłaszczony}
\Clue{34}{}{małpa wąskonosa z pawianów}
\Clue{35}{}{Alopecosa - rodzaj pająka z rodziny pogońcowatych}
\Clue{36}{}{miasto w R.P.A (Prowincja Przylądkowa), w pobliżu granicy z Bostwana, ważny węzeł kolejowy}
\Clue{37}{}{ziarno rośliny o tej samej nazwie; twarda odmiana pszenicy, z której wytwarza się mąkę nazywaną semoliną i kuskus}
\Clue{38}{}{zaburzenie psychiczne, które polega na upodobnianiu się do osoby płci przeciwnej poprzez ubiór i zachowanie w celu osiągnięcia satysfakcji emocjonalnej bądź seksualnej}
\Clue{39}{}{miasto we Francji (Burgundia) nad rzeką Doubs}\end{PuzzleClues}

\begin{PuzzleClues}{\textbf{Pionowe}\\}\Clue{1}{}{foka obrączkowana o długości około 1,8 m}
\Clue{2}{}{sposób przeliczania preferencji wyborców na liczbę mandatów partii lub na liczbę głosów kandydata na urząd państwowy w procesie wyłaniania reprezentanta}
\Clue{3}{}{dorosły koń (samiec) zdolny do rozrodu}
\Clue{4}{}{używane we wschodnich częściach Polski określenie choinki, iglaka}
\Clue{6}{}{duży samochód policyjny}
\Clue{10}{}{podróżnik amerykański (1779-1813); badacz Missisipi, Meksyku, Teksasu}
\Clue{11}{}{laufer poruszający się tylko po białych polach}
\Clue{12}{}{roślina będąca gospodarzem dla pasożyta}
\Clue{13}{}{technologia wykorzystująca rośliny wyższe w procesie oczyszczania środowiska (gleby, wód gruntowych i powierzchniowych, osadów ściekowych oraz powietrza)}
\Clue{14}{}{stosunkowo rzadki nowotwór, będący wynikiem zaburzenia układu hormonalnego, dający niespecyficzne objawy}
\Clue{15}{}{okaz stawonoga z podtypu szczękoczułkowców}
\Clue{16}{}{warzywo, jarzyna - jadalna część rośliny uprawianej w ogrodzie}
\Clue{17}{}{płatny zabójca, morderca}
\Clue{19}{}{cecha układu, która wynika ze zmierzania tego układu do stanu równowagi}
\Clue{20}{}{całkowita ilość azotu znajdującego się w wodzie w formie jonów amonowych, amoniaku oraz pozostałych trójujemnych odmian tego pierwiastka}
\Clue{21}{}{miasto na Węgrzech (komitat Peszt) port nad Dunajem}
\Clue{23}{}{naciskanie na kogoś, aby coś zrobił; domaganie się, żądanie, upieranie się przy czymś}
\Clue{24}{}{wierzchnia, zewnętrzna część czegoś}
\Clue{25}{}{cecha charakteru, pryncypialność}
\Clue{27}{}{lekceważąco o kiepskim aktorze}
\Clue{28}{}{popularna na siedemnastowiecznych dworach europejskich zabawa wzorowana na średniowiecznych turniejach rycerskich, w czasie której popisywano się sztuką jeździecką i fechtunkiem}
\Clue{29}{}{ciągła konstrukcja warowna lub umocnieniowa w postaci muru wykonanego z kamienia lub cegły}
\Clue{30}{}{rozczarowanie, nieziszczenie się, niespełnienie marzeń, planów}
\Clue{31}{}{dekoracja architektoniczna w kształcie trójdzielnego liścia}
\Clue{32}{}{indyjskie pismo alfabetyczno-sylabiczne}\end{PuzzleClues}\newpage\section*{Krzyżówka 131}

\noindent\begin{Puzzle}{25}{27}|*	|*	|*	|*	|*	|*	|*	|*	|*	|*	|*	|*	|*	|*	|*	|*	|*	|*	|*	|*	|*	|[1][S]\darr	|*	|*	|*	|*	|.
|*	|*	|*	|*	|*	|*	|*	|*	|*	|*	|*	|*	|*	|*	|*	|*	|*	|*	|*	|*	|*	|ł	|*	|*	|*	|*	|.
|*	|*	|*	|*	|*	|*	|*	|*	|*	|*	|*	|*	|*	|*	|*	|*	|*	|*	|*	|*	|*	|u	|*	|*	|*	|*	|.
|*	|*	|*	|*	|*	|*	|*	|*	|*	|*	|*	|*	|[2][S]\darr	|*	|*	|*	|*	|*	|*	|*	|[3][S]\darr	|s	|*	|*	|*	|*	|.
|*	|*	|*	|*	|*	|*	|*	|*	|*	|[4][S]\rarr	|k	|u	|m	|u	|l	|a	|c	|j	|a	|*	|k	|k	|*	|*	|*	|*	|.
|*	|*	|*	|*	|*	|*	|*	|*	|*	|*	|*	|*	|a	|[5][S]\drarr	|f	|l	|a	|d	|r	|o	|w	|a	|n	|i	|e	|*	|.
|*	|*	|*	|*	|*	|*	|*	|*	|*	|*	|*	|*	|j	|ż	|*	|*	|*	|*	|*	|*	|a	|*	|*	|*	|*	|*	|.
|*	|*	|*	|*	|*	|*	|*	|*	|*	|*	|*	|*	|ó	|ł	|*	|*	|*	|*	|*	|*	|s	|*	|*	|*	|*	|*	|.
|*	|*	|*	|*	|*	|*	|*	|*	|*	|*	|[6][S]\darr	|*	|w	|ó	|*	|*	|*	|*	|*	|*	|[][,]{ }	|*	|*	|*	|*	|*	|.
|*	|*	|*	|*	|*	|*	|*	|*	|*	|*	|k	|*	|k	|b	|*	|*	|*	|*	|*	|*	|p	|*	|*	|*	|*	|*	|.
|*	|*	|*	|*	|[7][S]\rarr	|k	|o	|l	|o	|n	|a	|d	|a	|*	|*	|*	|*	|*	|*	|*	|o	|*	|*	|*	|*	|*	|.
|*	|*	|*	|*	|*	|*	|*	|*	|*	|*	|s	|*	|*	|*	|[8][S]\darr	|*	|*	|*	|*	|*	|r	|*	|*	|*	|*	|*	|.
|*	|*	|*	|*	|*	|*	|*	|*	|*	|*	|z	|*	|*	|*	|b	|*	|*	|*	|*	|*	|o	|*	|*	|*	|*	|*	|.
|*	|*	|*	|*	|*	|*	|[9][S]\darr	|*	|*	|*	|t	|*	|*	|*	|e	|*	|*	|*	|*	|*	|s	|*	|*	|*	|*	|*	|.
|*	|*	|*	|*	|*	|*	|k	|[10][S]\rarr	|f	|r	|a	|n	|c	|u	|z	|*	|*	|*	|*	|*	|t	|*	|*	|*	|*	|*	|.
|*	|*	|*	|*	|*	|*	|i	|*	|*	|*	|n	|*	|*	|*	|r	|*	|*	|*	|*	|*	|o	|*	|*	|*	|*	|*	|.
|*	|[11][S]\rarr	|f	|i	|r	|e	|b	|a	|l	|l	|*	|*	|*	|*	|o	|*	|*	|*	|*	|*	|w	|*	|*	|*	|*	|*	|.
|*	|*	|*	|*	|*	|*	|i	|*	|*	|*	|*	|*	|*	|*	|l	|*	|*	|*	|*	|*	|y	|*	|*	|*	|*	|*	|.
|*	|*	|[12][S]\rarr	|m	|o	|s	|t	|[][,]{ }	|z	|w	|o	|d	|z	|o	|n	|y	|*	|*	|*	|*	|*	|*	|*	|*	|*	|*	|.
|*	|*	|*	|*	|*	|*	|k	|*	|*	|*	|*	|[13][S]\darr	|*	|*	|y	|*	|*	|*	|*	|*	|*	|*	|*	|*	|*	|*	|.
|*	|*	|*	|*	|[14][S]\rarr	|m	|a	|r	|k	|o	|w	|s	|k	|i	|*	|*	|*	|*	|*	|*	|*	|*	|*	|*	|*	|*	|.
|*	|*	|*	|*	|*	|*	|*	|*	|*	|*	|*	|c	|*	|*	|*	|*	|*	|*	|*	|*	|*	|*	|*	|*	|*	|*	|.
|*	|*	|*	|*	|*	|*	|*	|*	|*	|*	|*	|h	|*	|*	|*	|*	|*	|*	|*	|*	|*	|*	|*	|*	|*	|*	|.
|*	|*	|*	|*	|*	|*	|*	|*	|*	|*	|*	|a	|*	|*	|*	|*	|*	|*	|*	|*	|*	|*	|*	|*	|*	|*	|.
|*	|*	|*	|*	|*	|*	|*	|*	|*	|*	|*	|d	|*	|*	|*	|*	|*	|*	|*	|*	|*	|*	|*	|*	|*	|*	|.
|[15][S]\rarr	|r	|u	|b	|e	|l	|[][,]{ }	|b	|i	|a	|ł	|o	|r	|u	|s	|k	|i	|*	|*	|*	|*	|*	|*	|*	|*	|*	|.
|*	|*	|*	|*	|*	|*	|*	|*	|*	|*	|*	|w	|*	|*	|*	|*	|*	|*	|*	|*	|*	|*	|*	|*	|*	|*	|.
|*	|*	|*	|*	|*	|*	|*	|*	|*	|*	|*	|*	|*	|*	|*	|*	|*	|*	|*	|*	|*	|*	|*	|*	|*	|*	|.\end{Puzzle}

\newpage

\begin{PuzzleClues}{\textbf{Poziome}\\}\Clue{4}{}{sumowanie się działania leków lub trucizn w organizmie}
\Clue{5}{}{mazerowanie, słojowanie - malowanie na powierzchni drewna słojów mających imitować drogie gatunki drewna}
\Clue{7}{}{czeski słodki andrut z nadzieniem (zwykle śmietankowym, waniliowym lub orzechowym), który dawniej tradycyjnie był sprzedawany w miejscowościach uzdrowiskowych}
\Clue{10}{}{w gwarze młodzieżowej, uczniowskiej: język francuski - przedmiot, na którym poznaje się podstawy francuskiego}
\Clue{11}{}{typ jachtu regatowego}
\Clue{12}{}{most ruchomy, który można wielokrotnie podnosić i opuszczać}
\Clue{14}{}{kompozytor i dyrygent (1924-1986); twórca festiwalu 'Vratislavia Cantans'}
\Clue{15}{}{jednostka monetarna Republiki Białorusi równa 100 kopiejkom, wprowadzona do obiegu w roku 1992 po uzyskaniu przez Białoruś niepodległości}\end{PuzzleClues}

\begin{PuzzleClues}{\textbf{Pionowe}\\}\Clue{1}{}{złuszczająca się warstwa rogowa naskórka o charakterystycznym wyglądzie i fakturze}
\Clue{2}{}{wiosenna wycieczka za miasto, zwykle odbywająca się w maju}
\Clue{3}{}{wytwarzany przez porosty związek organiczny, metabolit wtórny, wytrącany na zewnętrznej stronie strzępek porostów w postaci kryształków o różnej barwie}
\Clue{5}{}{rodzaj zazwyczaj drewnianego (dziś coraz częściej plastikowego) koryta na paszę o charakterystycznym przekroju w kształcie litery U używanego do podawania jedzenia dla większych zwierząt gospodarskich, głównie koni i krów}
\Clue{6}{}{koń maści kasztanowej}
\Clue{8}{}{chłop, który posiadał tylko chałupę i niewielki ogródek, nie miał własnej ziemi uprawnej}
\Clue{9}{}{szeroki, kryty wóz na płozach lub kołach używany w rosyjskich gospodarstwach, w okresie caratu do przewożenia więźniów}
\Clue{13}{}{dział sztuk plastycznych; przedstawianie na płaszczyźnie bryłowatości ciał oraz ich położenia w przestrzeni}\end{PuzzleClues}\newpage\section*{Krzyżówka 132}

\noindent\begin{Puzzle}{19}{26}|*	|*	|*	|*	|*	|*	|*	|*	|*	|*	|*	|*	|*	|[1][S]\drarr	|r	|o	|m	|n	|y	|*	|.
|*	|*	|*	|*	|*	|[2][S]\darr	|*	|*	|*	|*	|*	|*	|[3][S]\darr	|k	|*	|[4][S]\darr	|[5][S]\darr	|*	|*	|*	|.
|*	|*	|*	|[6][S]\darr	|[7][S]\darr	|u	|*	|[8][S]\darr	|[9][S]\darr	|*	|*	|[10][S]\rarr	|p	|i	|l	|o	|s	|*	|*	|*	|.
|*	|*	|*	|r	|s	|k	|*	|m	|s	|[11][S]\rarr	|i	|d	|o	|l	|*	|l	|t	|[12][S]\darr	|*	|*	|.
|*	|*	|*	|i	|a	|ł	|*	|i	|t	|*	|*	|*	|n	|e	|*	|e	|r	|ś	|*	|*	|.
|*	|[13][S]\darr	|*	|e	|l	|u	|*	|m	|y	|*	|[14][S]\darr	|*	|s	|r	|*	|j	|z	|c	|[15][S]\darr	|*	|.
|*	|b	|[16][S]\drarr	|s	|o	|c	|j	|o	|l	|e	|k	|t	|*	|*	|*	|*	|e	|i	|n	|*	|.
|*	|r	|s	|z	|w	|i	|*	|z	|i	|[17][S]\darr	|i	|[18][S]\darr	|*	|*	|[19][S]\darr	|*	|c	|e	|i	|[20][S]\darr	|.
|*	|y	|i	|*	|a	|e	|*	|a	|s	|n	|p	|p	|*	|*	|p	|*	|h	|ż	|e	|m	|.
|*	|t	|e	|*	|*	|*	|*	|*	|k	|o	|a	|l	|*	|*	|a	|*	|w	|k	|s	|e	|.
|*	|y	|r	|*	|[21][S]\drarr	|h	|u	|m	|o	|r	|*	|e	|*	|*	|n	|*	|a	|a	|t	|z	|.
|*	|j	|o	|*	|e	|*	|*	|*	|*	|m	|*	|ś	|*	|*	|d	|[22][S]\darr	|[][,]{ }	|[][,]{ }	|e	|a	|.
|[23][S]\drarr	|s	|t	|ą	|g	|i	|e	|w	|k	|a	|*	|n	|*	|[24][S]\darr	|a	|o	|d	|h	|r	|n	|.
|p	|k	|a	|*	|e	|*	|*	|[25][S]\darr	|*	|[][,]{ }	|*	|i	|*	|z	|n	|b	|a	|a	|o	|i	|.
|ó	|o	|[][,]{ }	|*	|r	|[26][S]\drarr	|e	|s	|k	|a	|l	|a	|d	|a	|*	|r	|r	|m	|w	|n	|.
|ł	|ś	|n	|*	|*	|l	|*	|u	|[27][S]\rarr	|b	|u	|k	|o	|w	|i	|a	|n	|i	|n	|*	|.
|p	|ć	|a	|*	|*	|o	|[28][S]\darr	|w	|*	|s	|*	|*	|*	|a	|*	|z	|i	|l	|o	|*	|.
|r	|*	|t	|*	|*	|t	|d	|*	|*	|t	|*	|*	|[29][S]\darr	|d	|*	|a	|o	|t	|ś	|[30][S]\darr	|.
|z	|*	|u	|[31][S]\drarr	|n	|o	|r	|g	|a	|r	|d	|*	|z	|z	|*	|[][,]{ }	|w	|o	|ć	|s	|.
|e	|[32][S]\drarr	|r	|o	|z	|p	|a	|ł	|k	|a	|*	|*	|e	|k	|*	|b	|a	|n	|*	|z	|.
|s	|s	|a	|b	|*	|a	|n	|*	|*	|k	|*	|[33][S]\rarr	|w	|a	|c	|o	|*	|a	|*	|a	|.
|t	|o	|l	|w	|*	|ł	|i	|*	|[34][S]\rarr	|c	|u	|d	|*	|*	|*	|s	|*	|*	|[35][S]\darr	|k	|.
|r	|m	|n	|ó	|*	|a	|c	|*	|*	|y	|[36][S]\rarr	|m	|u	|z	|y	|k	|a	|*	|p	|ł	|.
|z	|m	|y	|d	|*	|n	|a	|*	|*	|j	|[37][S]\rarr	|z	|a	|k	|ł	|a	|d	|*	|o	|a	|.
|e	|e	|*	|*	|*	|k	|*	|[38][S]\rarr	|u	|n	|i	|s	|e	|k	|s	|*	|*	|*	|l	|k	|.
|ń	|n	|*	|*	|*	|a	|*	|[39][S]\rarr	|c	|a	|ł	|u	|s	|k	|o	|w	|a	|t	|e	|*	|.
|*	|*	|*	|*	|*	|*	|*	|*	|*	|*	|*	|*	|*	|*	|*	|*	|*	|*	|*	|*	|.\end{Puzzle}

\newpage

\begin{PuzzleClues}{\textbf{Poziome}\\}\Clue{1}{}{miasto na Ukrainie nad Sułą}
\Clue{10}{}{PYLOS; miasto w Grecji na Peloponezie nad Morzem Jońskim}
\Clue{11}{}{osoba uwielbiana, popularna, sławna}
\Clue{16}{}{odmiana języka wyróżniona ze względu na przynależność społeczną komunikujących}
\Clue{21}{}{zdolność dostrzegania tego, co śmieszne; bystrość umysłu, która pomaga w zabawny, błyskotliwy sposób mówić i się zachowywać}
\Clue{23}{}{mała stągiew}
\Clue{26}{}{zdobywanie ściany fortecznej za pomocą drabin}
\Clue{27}{}{człowiek, który mieszka w Bukowinie Tatrzańskiej albo pochodzi z tej miejscowości}
\Clue{31}{}{ur.  w 1932 r., kompozytor duński; utwory orkiestrowe, kameralne, opery, balety}
\Clue{32}{}{łatwopalny materiał służący do rozpalania ognia}
\Clue{33}{}{miasto w USA (Teksas) nad rzeką Brazos, ośrodek handlowy regionu rolniczego}
\Clue{34}{}{coś niezwykłego, rzadkiego, ktoś wyjątkowy}
\Clue{36}{}{wytwór aktywności artystycznej człowieka, przybierający postać przemyślanych szeregów dźwięków, śpiewanych lub granych na instrumentach; także: sama ta aktywność oraz zbiór utworów muzycznych}
\Clue{37}{}{instytucja o charakterze naukowym, społecznym, kulturalnym, leczniczym (np. Zakład Leczenia Zeza)}
\Clue{38}{}{idea zakładająca równość płci, uniwersalność}
\Clue{39}{}{Helostomatidae - rodzina słodkowadnych ryb okoniokształtnych (Perciformes)}\end{PuzzleClues}

\begin{PuzzleClues}{\textbf{Pionowe}\\}\Clue{1}{}{płatny morderca}
\Clue{2}{}{dotknięcie kogoś, zranienie w sposób niefizyczny, sprawienie przykrości}
\Clue{3}{}{śpiewaczka francuska (1904-1976); sopran koloraturowy}
\Clue{4}{}{tłusta substancja ciekła lub łatwo topniejąca substancja stała nierozpuszczalna w wodzie; substancja, która może być otrzymywana z nasion i owoców niektórych roślin lub tkanek zwierząt, może być też płynnym produktem destylacji ropy naftowej, węgla kamiennego lub smoły}
\Clue{5}{}{Grimmia pulvinata - gatunek mchu z rodziny strzechwowatych}
\Clue{6}{}{matematyk węgierski (1880-1956); jeden z twórców analizy funkcjonalnej i teorii przestrzeni topologicznych}
\Clue{7}{}{pracownik służby zdrowia, zajmujący się utrzymaniem czystości w placówkach ochrony zdrowia, w szczególności w salach szpitalnych}
\Clue{8}{}{CZUŁEK}
\Clue{9}{}{uchwyt ręcznych narzędzi gospodarskich, budowlanych, ogrodniczych, broni i sprzętu alpinistycznego}
\Clue{12}{}{ścieżka w grafie przebiegająca przez wszystkie jego wierzchołki dokładnie raz}
\Clue{13}{}{zespół cech właściwych Brytyjczykom, Wielkiej Brytanii lub czemuś brytyjskiemu}
\Clue{14}{}{drewniane lub metalowe oczko do przeprowadzania szotów żagli przednich i obciągania ich ku dołowi}
\Clue{15}{}{to, że coś jest niesterowne, nie można nad tym zapanować, kierować, zarządzać (np.niesterowność państwa, systemu)}
\Clue{16}{}{chłopiec, którego oboje rodzice zmarli; forma rzadsza}
\Clue{17}{}{norma, która wyznacza zachowania, postępowania w sytuacjach wielorazowych, powtarzalnych}
\Clue{18}{}{Mucor - rodzaj grzybów sprzężniaków z rzędu pleśniakowców (Mucorales) obejmujący około 50 gatunków}
\Clue{19}{}{POCHUTNIK drzewo lub pnący się krzew, niektóre gatunki uprawiane w doniczkach jako ozdobne}
\Clue{20}{}{ANTRESOLA, PÓŁPIĘTRO}
\Clue{21}{}{miasto w płn. Węgrzech, ośrodek administracyjny komitatu Heves, u podnóża Gór Bukowych}
\Clue{22}{}{coś, czego nie powinno się robić}
\Clue{23}{}{każda z dwóch części trójwymiarowej przestrzeni euklidesowej, na które dzieli tę przestrzeń płaszczyzna, wraz z tą płaszczyzną}
\Clue{24}{}{śpiewaczka (sopran) i pedagog (1899-1988); solistka La Scali, Covent Garden, Opery Poznańskiej}
\Clue{25}{}{część cyklu pracy silnika tłokowego}
\Clue{26}{}{Petaurus australis -  gatunek torbacza z rodziny lotopałankowatych, zamieszkujący lasy eukaliptusowe we wschodniej Australii, od północnego Queensland do Wiktorii}
\Clue{28}{}{DRANKA}
\Clue{29}{}{wezwanie do działania, na które oczekuje się natychmiastowej reakcji}
\Clue{30}{}{ciernisty, pospolity w Polsce krzew o czarnych leczniczych owocach}
\Clue{31}{}{zamknięta instalacja elektryczna}
\Clue{32}{}{jezioro w Szwecji (Gotlandia)}
\Clue{35}{}{w informatyce: pojedyncza zmienna, stanowiąca fragment struktury, unii, klasy, obiektu lub rekordu tabeli bazodanowej, związany jest z nią pewien typ}\end{PuzzleClues}\newpage\section*{Krzyżówka 133}

\noindent\begin{Puzzle}{20}{28}|*	|*	|*	|*	|*	|[1][S]\darr	|*	|*	|*	|*	|*	|*	|*	|*	|*	|*	|*	|*	|[2][S]\darr	|*	|*	|.
|*	|[3][S]\rarr	|e	|k	|s	|p	|a	|r	|t	|n	|e	|r	|*	|*	|*	|*	|*	|*	|k	|[4][S]\darr	|*	|.
|[5][S]\rarr	|g	|o	|e	|t	|e	|l	|*	|[6][S]\darr	|*	|*	|*	|[7][S]\darr	|[8][S]\rarr	|p	|e	|r	|i	|o	|d	|*	|.
|*	|[9][S]\darr	|*	|[10][S]\drarr	|t	|r	|a	|n	|s	|g	|r	|e	|s	|j	|a	|*	|*	|*	|d	|z	|*	|.
|*	|c	|[11][S]\drarr	|f	|i	|l	|m	|[][,]{ }	|o	|b	|y	|c	|z	|a	|j	|o	|w	|y	|*	|i	|*	|.
|*	|h	|d	|r	|[12][S]\darr	|i	|*	|*	|ł	|*	|[13][S]\drarr	|ł	|a	|ń	|c	|u	|s	|z	|e	|k	|*	|.
|*	|e	|u	|i	|k	|c	|*	|*	|t	|[14][S]\darr	|t	|*	|t	|*	|[15][S]\rarr	|s	|t	|ę	|p	|a	|*	|.
|*	|m	|s	|s	|a	|a	|[16][S]\darr	|*	|y	|w	|r	|*	|a	|*	|*	|*	|*	|*	|*	|[][,]{ }	|*	|.
|*	|i	|i	|c	|n	|[][,]{ }	|g	|*	|k	|ó	|ó	|[17][S]\darr	|n	|[18][S]\rarr	|l	|i	|n	|t	|e	|r	|*	|.
|*	|o	|k	|h	|a	|s	|e	|*	|*	|d	|j	|a	|i	|*	|*	|*	|*	|*	|*	|ó	|*	|.
|*	|t	|*	|*	|ł	|o	|o	|[19][S]\rarr	|j	|e	|l	|i	|s	|j	|e	|j	|e	|w	|*	|ż	|*	|.
|*	|e	|[20][S]\darr	|*	|[][,]{ }	|p	|m	|*	|*	|c	|u	|l	|t	|*	|[21][S]\rarr	|b	|u	|z	|i	|a	|*	|.
|*	|r	|p	|[22][S]\darr	|p	|l	|e	|*	|[23][S]\darr	|z	|f	|a	|k	|*	|[24][S]\drarr	|d	|ż	|i	|g	|*	|*	|.
|*	|a	|i	|k	|ó	|o	|t	|[25][S]\darr	|o	|k	|k	|n	|a	|[26][S]\rarr	|k	|l	|e	|i	|n	|*	|*	|.
|*	|p	|o	|u	|ł	|w	|r	|g	|ś	|a	|a	|t	|*	|[27][S]\darr	|o	|[28][S]\darr	|*	|*	|*	|*	|*	|.
|*	|e	|t	|s	|k	|a	|i	|ł	|l	|*	|*	|*	|*	|c	|w	|m	|*	|*	|*	|*	|*	|.
|*	|u	|r	|z	|u	|t	|a	|u	|a	|[29][S]\drarr	|g	|r	|z	|e	|b	|i	|e	|ń	|*	|*	|*	|.
|*	|t	|k	|y	|l	|a	|[][,]{ }	|p	|[][,]{ }	|m	|*	|[30][S]\drarr	|f	|l	|o	|r	|y	|d	|a	|*	|*	|.
|*	|y	|o	|t	|i	|*	|f	|k	|c	|i	|*	|k	|*	|t	|j	|*	|*	|*	|*	|*	|*	|.
|[31][S]\drarr	|k	|w	|a	|s	|[][,]{ }	|r	|o	|z	|o	|l	|o	|w	|y	|*	|[32][S]\darr	|*	|*	|*	|*	|[33][S]\darr	|.
|h	|*	|i	|*	|t	|*	|a	|w	|a	|d	|[34][S]\darr	|s	|[35][S]\rarr	|k	|o	|p	|i	|a	|*	|[36][S]\darr	|o	|.
|u	|*	|a	|[37][S]\darr	|y	|[38][S]\drarr	|k	|a	|p	|u	|s	|t	|a	|*	|*	|r	|*	|*	|*	|g	|b	|.
|m	|*	|n	|b	|*	|r	|t	|t	|k	|n	|a	|i	|*	|*	|*	|z	|*	|*	|*	|z	|r	|.
|a	|[39][S]\darr	|i	|u	|*	|o	|a	|o	|a	|k	|t	|u	|*	|*	|*	|ę	|*	|*	|*	|y	|o	|.
|n	|z	|n	|r	|*	|ż	|l	|ś	|*	|a	|y	|m	|[40][S]\rarr	|k	|a	|s	|e	|t	|a	|*	|n	|.
|*	|ł	|*	|n	|*	|e	|n	|ć	|*	|*	|r	|*	|*	|*	|*	|ł	|*	|*	|*	|*	|a	|.
|[41][S]\rarr	|o	|l	|e	|i	|c	|a	|*	|*	|*	|a	|*	|*	|[42][S]\rarr	|n	|o	|w	|o	|ś	|ć	|*	|.
|*	|ć	|*	|t	|*	|*	|*	|*	|*	|*	|*	|*	|*	|*	|*	|*	|*	|*	|*	|*	|*	|.
|*	|*	|*	|*	|*	|*	|*	|*	|*	|*	|*	|*	|*	|*	|*	|*	|*	|*	|*	|*	|*	|.\end{Puzzle}

\newpage

\begin{PuzzleClues}{\textbf{Poziome}\\}\Clue{3}{}{były partner - osoba, która była kiedyś towarzyszem czyjegoś życia, lecz teraz już nie jest}
\Clue{5}{}{geolog, badacz Tatr (1889-1972); współtwórca parków narodowych w Tatrach, Pieninach i na Babiej Górze}
\Clue{8}{}{powtarzanie się zjawiska w czasie; wyodrębniony okres, określona pora, kiedy coś się dzieje lub coś jest wykonywane, odcinek czasu}
\Clue{10}{}{zalewanie lądu przez morza w wyniku topnienia lodowców lub przez ruchy tektoniczne podwyższające poziom dna morskiego}
\Clue{11}{}{gatunek filmowy traktujący o sposobach postępowania, charakterystycznych zachowaniach społecznych, zwyczajach, nawykach, przyzwyczajeniach lub codzienności}
\Clue{13}{}{ścieg w szyciu, który jest dość mocny i ma walor ozdobny; ścieg, w którym tworzy się małe pętelki z nici i przeszywa się je tak, że powstaje wzór przypominający ogniwa łańcucha}
\Clue{15}{}{w garbarstwie: urządzenie do ugniatania skór}
\Clue{18}{}{krótkie, nieprzędne włókno, porastające nasiona bawełny}
\Clue{19}{}{kosmonauta radziecki na pokładzie Sojuza 5 w 1969 r}
\Clue{21}{}{miła lub drobna twarz; słowo pieszczotliwe}
\Clue{24}{}{utwór muzyczny do tańca dżig}
\Clue{26}{}{przyrodnik (1685-1759); założyciel ogrodu botanicznego w Gdańsku}
\Clue{29}{}{grzywa na karku kozicy w gwarze łowieckiej}
\Clue{30}{}{jeden ze stanów USA}
\Clue{31}{}{syntetyczny barwnik trójarylometanowy stosowany w analizie chemicznej}
\Clue{35}{}{odbitka z negatywu lub pozytywu, kopia fotograficzna}
\Clue{38}{}{znane warzywo (taka kapusta rośnie w ogródku i można ją kupić pod tą nazwą w zieleniaku)}
\Clue{40}{}{nośnik danych; płaski plastikowy przedmiot z taśmą magnetyczną nawiniętą na szpulki w środku, przeznaczony do zapisu, przechowywania i późniejszego odtwarzania sygnału audiowizualnego w magnetowidzie lub audiofonicznego w magnetofonie kasetowym}
\Clue{41}{}{chrząszcz wydzielający oleistą ciecz o odrażającym zapachu np. kantaryda}
\Clue{42}{}{coś nowego, coś, czego wcześniej nie było}\end{PuzzleClues}

\begin{PuzzleClues}{\textbf{Pionowe}\\}\Clue{1}{}{Numida meleagris mitratus - podgatunek ptaka wyróżniony w obrębie gatunku perlica zwyczajna (Numida meleagris)}
\Clue{2}{}{odtwarzalne schematy, wzorce postępowania}
\Clue{4}{}{owoc (z botanicznego punktu widzenia jest to owoc szupinkowy) dzikiej róży}
\Clue{6}{}{ur. w 1909 r. konstruktor lotniczy i automatyk}
\Clue{7}{}{wyznawczyni jednego z wielu różnych systemów religijnych, wierzeń oraz praktyk, które nawiązują do różnie rozumianego Szatana}
\Clue{9}{}{lek przeciwdrobnoustrojowy otrzymany na zasadzie całkowitej syntezy chemicznej, nieposiadający swojego odpowiednika w przyrodzie}
\Clue{10}{}{Ragnar (1895-1973); ekonomista norweski, laureat nagrody Nobla}
\Clue{11}{}{ozdoba na szyję, naszyjnik-obroża, przylegający do szyi}
\Clue{12}{}{jeden z trzech kostnych kanałów ucha wewnętrznego, znajdujących się w błędniku}
\Clue{13}{}{DRYLING, DRYLINGER}
\Clue{14}{}{zdrobniale: wódka - mocny alkohol}
\Clue{16}{}{dziedzina geometrii, której kluczowym pojęciem są fraktale, obiekty wykazujące własność samopodobieństwa}
\Clue{17}{}{motyw dekoracyjny w formie stylizowanych liści, także gałązek i kwiatów rośliny o tej nazwie}
\Clue{20}{}{mieszkaniec Piotrkowa Kujawskiego}
\Clue{22}{}{członek grupy etnicznej Kuszytów, zamieszkujących wschodnią Afrykę}
\Clue{23}{}{czapka, którą nosi się za karę, informująca otoczenie o tym, że ten, który ją nosi, został ukarany}
\Clue{24}{}{przenośnie, nieraz ironicznie: waleczny, pewny siebie mężczyzna}
\Clue{25}{}{cecha czegoś, co postrzegane jest jako głupkowate (w znaczeniu: wyrażające zmieszanie, zakłopotanie itp.; np. głupkowatość czyjegoś śmiechu, spojrzenia}
\Clue{27}{}{muzyka celtycka, folk Wysp Brytyjskich}
\Clue{28}{}{stan pokoju, dobra, przyjazna relacja, ład}
\Clue{29}{}{miododajna bylina z szorstkolistnych - Eurazja}
\Clue{30}{}{damskie plażowe ubranie wykonane z elastycznej tkaniny}
\Clue{31}{}{przedmiot humanistyczny, zwłaszcza na studiach technicznych}
\Clue{32}{}{element sklepienia oddzielony żebrami lub gurtami}
\Clue{33}{}{egzamin, jaki należy zdać, aby zakończyć studia I, II lub III stopnia}
\Clue{34}{}{coś, co ma za zadanie wyszydzenie czegoś, skrytykowanie}
\Clue{36}{}{rodzina dużych muchówek, których larwy rozwijają się w organizmie niektórych zwierząt parzysto kopytnych}
\Clue{37}{}{australijski lekarz immunolog (1899-1985); nagroda Nobla za badania nad nabytą tolerancją immunologiczną}
\Clue{38}{}{GAFEL}
\Clue{39}{}{cebulkowata roślina zielna z rodziny liliowatych}\end{PuzzleClues}\newpage\section*{Krzyżówka 134}

\noindent\begin{Puzzle}{25}{21}|*	|*	|*	|*	|*	|*	|*	|*	|*	|*	|*	|*	|*	|*	|*	|*	|[1][S]\darr	|*	|*	|*	|*	|*	|*	|*	|*	|*	|.
|*	|*	|*	|*	|*	|*	|*	|*	|*	|[2][S]\darr	|*	|*	|*	|*	|*	|*	|k	|*	|*	|*	|*	|*	|*	|*	|*	|*	|.
|*	|*	|*	|*	|*	|*	|*	|*	|*	|s	|*	|*	|*	|*	|*	|*	|o	|*	|*	|*	|*	|[3][S]\darr	|*	|[4][S]\darr	|*	|*	|.
|*	|*	|[5][S]\rarr	|o	|k	|o	|l	|i	|c	|z	|n	|i	|k	|*	|*	|*	|n	|*	|*	|*	|*	|e	|*	|f	|*	|*	|.
|*	|*	|*	|*	|*	|*	|*	|*	|*	|a	|*	|*	|*	|*	|*	|*	|k	|[6][S]\drarr	|k	|o	|m	|t	|u	|r	|*	|*	|.
|*	|*	|*	|*	|*	|*	|*	|*	|*	|m	|*	|*	|*	|*	|*	|*	|u	|w	|*	|*	|*	|k	|[7][S]\darr	|e	|*	|*	|.
|*	|[8][S]\rarr	|n	|a	|t	|r	|ę	|t	|n	|o	|ś	|ć	|*	|*	|*	|*	|r	|c	|*	|*	|*	|i	|k	|d	|*	|*	|.
|*	|*	|*	|*	|*	|*	|*	|*	|*	|t	|*	|*	|*	|*	|*	|*	|e	|i	|*	|*	|*	|n	|r	|r	|*	|*	|.
|*	|*	|*	|*	|*	|*	|*	|*	|*	|a	|*	|*	|*	|*	|*	|*	|n	|ą	|*	|*	|*	|*	|e	|o	|*	|*	|.
|*	|*	|*	|*	|*	|*	|*	|*	|*	|n	|*	|*	|*	|*	|*	|*	|c	|g	|*	|*	|*	|*	|p	|l	|*	|*	|.
|*	|*	|*	|*	|*	|*	|*	|*	|*	|i	|[9][S]\rarr	|c	|o	|o	|l	|[][,]{ }	|j	|a	|z	|z	|*	|*	|i	|o	|*	|*	|.
|*	|*	|*	|*	|*	|*	|[10][S]\darr	|*	|*	|n	|*	|[11][S]\rarr	|b	|u	|l	|w	|a	|r	|*	|*	|*	|*	|n	|g	|*	|*	|.
|*	|*	|*	|*	|*	|*	|i	|[12][S]\rarr	|n	|a	|j	|e	|m	|n	|i	|k	|*	|k	|[13][S]\rarr	|a	|l	|f	|a	|*	|*	|*	|.
|*	|*	|*	|*	|*	|*	|n	|*	|*	|*	|*	|*	|*	|*	|*	|[14][S]\rarr	|d	|a	|m	|r	|o	|t	|*	|*	|*	|*	|.
|*	|[15][S]\rarr	|p	|u	|n	|k	|t	|[][,]{ }	|o	|k	|o	|s	|t	|n	|o	|w	|y	|*	|*	|*	|*	|*	|*	|*	|*	|*	|.
|*	|*	|*	|[16][S]\rarr	|a	|g	|e	|n	|t	|[][,]{ }	|g	|o	|s	|p	|o	|d	|a	|r	|c	|z	|y	|*	|*	|*	|*	|*	|.
|[17][S]\rarr	|r	|o	|ś	|l	|i	|n	|a	|[][,]{ }	|j	|a	|w	|n	|o	|p	|ą	|c	|z	|k	|o	|w	|a	|*	|*	|*	|*	|.
|*	|*	|[18][S]\rarr	|m	|i	|o	|d	|o	|j	|a	|d	|e	|k	|[][,]{ }	|b	|i	|a	|ł	|o	|g	|a	|r	|d	|ł	|y	|*	|.
|*	|*	|*	|*	|[19][S]\rarr	|ż	|e	|g	|l	|a	|r	|z	|[][,]{ }	|j	|a	|c	|h	|t	|o	|w	|y	|*	|*	|*	|*	|*	|.
|*	|*	|*	|*	|*	|*	|n	|*	|*	|*	|*	|*	|*	|*	|*	|*	|*	|*	|*	|*	|*	|*	|*	|*	|*	|*	|.
|*	|[20][S]\rarr	|p	|o	|l	|i	|t	|y	|k	|a	|[][,]{ }	|d	|o	|s	|t	|o	|s	|o	|w	|a	|w	|c	|z	|a	|*	|*	|.
|*	|*	|*	|*	|*	|*	|*	|*	|*	|*	|*	|*	|*	|*	|*	|*	|*	|*	|*	|*	|*	|*	|*	|*	|*	|*	|.\end{Puzzle}

\newpage

\begin{PuzzleClues}{\textbf{Poziome}\\}\Clue{5}{}{część zdania, która pełni funkcję określającą czasownik, uzupełniajac go o dodatkowe elementy}
\Clue{6}{}{zwierzchnik domu zakonnego w niektórych zakonach rycerskich}
\Clue{8}{}{cecha kogoś, kto uporczywie coś robi, denerwuje, zajmuje}
\Clue{9}{}{styl w muzyce jazzowej, który rozwijał się na końcu lat 40. i w latach 50. XX wieku, charakteryzujący się wyciszonym i stonowanym brzmienim oraz  unowocześnieniem harmonii i sposobu improwizacji, zbliżającym go do awangardowej muzyki XX-wiecznej}
\Clue{11}{}{w dawnych fortyfikacjach o narysie bastionowym, podstawowy element umocnień wznoszony na załamaniach obwałowania twierdzy (na wysuniętych narożnikach)}
\Clue{12}{}{pracownik, którego można wynająć, zwykle do niegodnych, nieprzyjemnych zajęć}
\Clue{13}{}{Pyrus communis 'Alfa' - odmiana uprawna gruszy pospolitej}
\Clue{14}{}{(1841-95) pisarz i działacz narodowy na Górnym Śląsku}
\Clue{15}{}{miejsce na powierzchni kości, które jest bolesne bądź zniekształcone przez stan chorobowy lub długo utrzymujący się odruch trzewno-okostnowy}
\Clue{16}{}{każda, niezależnie od jej formy organizacyjnej, jednostka ekonomiczna prowadząca działalność gospodarczą}
\Clue{17}{}{forma życiowa rośliny; roślina o pędach wzniesionych, odnawiająca się z pąków znajdujących się co najmniej 0,5 m nad ziemią}
\Clue{18}{}{Lichmera notabilis - gatunek ptaka z rodziny miodojadów (Meliphagidae) obejmujący gatunki występujące na Małych Wyspach Sundajskich, Molukach, Nowej Gwinei, Vanuatu, Nowej Kaledonii i w Australii}
\Clue{19}{}{najniższy z trzech wymaganych prawem patentów żeglarskich w Polsce}
\Clue{20}{}{świadome działanie w kierunku przywracania zachwianej równowagi bilansu płatniczego}\end{PuzzleClues}

\begin{PuzzleClues}{\textbf{Pionowe}\\}\Clue{1}{}{przedsiębiorstwo (zwłaszcza handlowe lub produkcyjne, rzadziej usługowe), które działa w tej samej branży, co inne}
\Clue{2}{}{sytuacja, w której ktoś z kimś się szarpie}
\Clue{3}{}{pianista, laureat Konkursu im F. Chopina w 1927 r}
\Clue{4}{}{znawca twórczości i życia Aleksandra Fredry}
\Clue{6}{}{urządzenie służące do przenoszenia ładunków (opuszczania, podnoszenia itp.) za pomocą liny lub łańcucha nawijających się na bęben}
\Clue{7}{}{pomarszczona kolorowa bibułka, do tworzenia dekoracji}
\Clue{10}{}{pracownik odpowiedzialny za zaopatrzenie żywnościowe, szczególnie w instytucjach oświatowych}\end{PuzzleClues}\newpage\section*{Krzyżówka 135}

\noindent\begin{Puzzle}{20}{28}|*	|*	|*	|*	|*	|*	|*	|*	|[1][S]\drarr	|p	|o	|r	|ę	|c	|z	|*	|*	|*	|*	|*	|*	|.
|*	|*	|*	|*	|[2][S]\rarr	|a	|c	|h	|r	|o	|m	|a	|t	|y	|n	|a	|*	|[3][S]\darr	|*	|*	|*	|.
|*	|*	|*	|[4][S]\rarr	|l	|e	|p	|k	|i	|*	|*	|*	|*	|[5][S]\rarr	|a	|u	|s	|t	|e	|n	|*	|.
|*	|*	|*	|*	|*	|*	|*	|[6][S]\rarr	|b	|a	|t	|u	|t	|a	|*	|[7][S]\rarr	|f	|o	|k	|a	|*	|.
|*	|*	|*	|*	|*	|*	|*	|*	|a	|[8][S]\rarr	|n	|i	|e	|d	|o	|z	|ó	|r	|*	|[9][S]\darr	|*	|.
|*	|*	|[10][S]\darr	|*	|*	|*	|*	|[11][S]\drarr	|t	|a	|k	|a	|l	|o	|*	|[12][S]\darr	|*	|t	|*	|s	|*	|.
|*	|[13][S]\drarr	|t	|e	|c	|z	|k	|a	|*	|*	|[14][S]\rarr	|t	|o	|b	|o	|ł	|k	|i	|*	|z	|*	|.
|[15][S]\drarr	|b	|a	|j	|c	|h	|e	|n	|*	|*	|*	|*	|[16][S]\drarr	|k	|r	|ę	|t	|l	|i	|k	|*	|.
|p	|o	|k	|[17][S]\drarr	|d	|o	|d	|a	|j	|n	|a	|*	|i	|*	|*	|g	|*	|l	|*	|o	|*	|.
|o	|r	|t	|n	|*	|[18][S]\darr	|[19][S]\rarr	|s	|y	|s	|t	|e	|m	|i	|k	|*	|*	|a	|*	|ł	|*	|.
|g	|z	|y	|a	|*	|k	|[20][S]\rarr	|a	|n	|a	|l	|e	|p	|t	|y	|k	|*	|[][,]{ }	|[21][S]\darr	|a	|*	|.
|o	|e	|k	|t	|*	|o	|[22][S]\darr	|z	|*	|[23][S]\drarr	|w	|i	|e	|ż	|o	|w	|i	|e	|c	|*	|*	|.
|t	|ś	|*	|u	|*	|m	|n	|i	|[24][S]\darr	|ł	|*	|*	|r	|[25][S]\darr	|*	|*	|*	|s	|h	|*	|*	|.
|o	|l	|*	|r	|*	|p	|o	|z	|p	|u	|*	|*	|i	|k	|*	|*	|[26][S]\darr	|p	|o	|*	|*	|.
|w	|a	|*	|a	|*	|r	|w	|a	|r	|k	|*	|*	|u	|a	|[27][S]\darr	|*	|o	|a	|r	|*	|*	|.
|i	|d	|*	|l	|*	|e	|o	|u	|z	|*	|*	|*	|m	|m	|d	|*	|m	|n	|o	|*	|*	|.
|e	|[][,]{ }	|*	|i	|[28][S]\drarr	|s	|t	|r	|y	|c	|h	|*	|[][,]{ }	|e	|r	|*	|a	|o	|b	|*	|*	|.
|[][,]{ }	|z	|*	|z	|d	|*	|a	|*	|c	|*	|*	|*	|m	|r	|a	|*	|c	|l	|a	|*	|*	|.
|g	|w	|*	|a	|o	|[29][S]\darr	|r	|[30][S]\rarr	|z	|u	|s	|*	|a	|a	|p	|*	|n	|a	|[][,]{ }	|*	|*	|.
|a	|i	|*	|c	|b	|b	|ż	|*	|y	|[31][S]\drarr	|b	|e	|l	|l	|i	|n	|i	|*	|w	|*	|*	|.
|z	|s	|*	|j	|i	|l	|a	|*	|n	|k	|*	|*	|a	|n	|e	|[32][S]\darr	|c	|*	|o	|*	|*	|.
|o	|ł	|*	|a	|t	|e	|n	|*	|e	|u	|*	|*	|z	|o	|s	|s	|a	|*	|d	|[33][S]\darr	|*	|.
|w	|y	|*	|*	|k	|j	|i	|*	|k	|r	|*	|*	|a	|ś	|t	|z	|*	|*	|u	|o	|*	|.
|e	|*	|*	|*	|a	|t	|n	|*	|*	|s	|*	|*	|ń	|ć	|w	|l	|*	|*	|n	|b	|*	|.
|*	|*	|[34][S]\rarr	|c	|*	|r	|*	|*	|*	|*	|*	|*	|s	|*	|o	|i	|*	|*	|k	|o	|*	|.
|*	|[35][S]\rarr	|m	|e	|t	|a	|l	|[][,]{ }	|c	|i	|ę	|ż	|k	|i	|*	|f	|*	|*	|o	|ź	|*	|.
|*	|*	|[36][S]\rarr	|k	|u	|m	|a	|k	|[][,]{ }	|n	|i	|z	|i	|n	|n	|y	|*	|*	|w	|n	|*	|.
|[37][S]\rarr	|d	|r	|ó	|b	|*	|[38][S]\rarr	|d	|o	|ś	|p	|i	|e	|w	|*	|*	|[39][S]\rarr	|b	|a	|y	|*	|.
|*	|[40][S]\rarr	|m	|i	|ę	|s	|o	|ż	|e	|r	|c	|a	|*	|[41][S]\rarr	|s	|i	|e	|ć	|*	|*	|*	|.\end{Puzzle}

\newpage

\begin{PuzzleClues}{\textbf{Poziome}\\}\Clue{1}{}{przyrząd do ćwiczeń w gimnastyce sportowej}
\Clue{2}{}{substancja w jądrze komórkowym, która się nie zabarwia pod wpływem różnych czynników}
\Clue{4}{}{ukraiński pisarz i historyk literatury (1872-1941), prace krytyczne, przekłady, liryki, opowiadania, zginął w Oświęcimiu}
\Clue{5}{}{(1775-1817), pisarka angielska, powieści z życia ziemiaństwa; „Duma i uprzedzenie”}
\Clue{6}{}{kierownictwo muzyczne}
\Clue{7}{}{morski ssak z rzędu płetwonogich; zamieszkuje chłodne morza obu półkul}
\Clue{8}{}{brak dozoru, niedopilnowanie czegoś, kogoś, zaniedbanie dozoru}
\Clue{11}{}{narciarka fińska, mistrzyni świata i wicemistrzyni olimpijska w biegu na 5 i 10 km z Innsbrucku}
\Clue{13}{}{rodzaj prostokątnej płaskiej torby z niedużym uchwytem do trzymania jej w dłoni}
\Clue{14}{}{Thlaspi - rodzaj roślin należący do rodziny kapustowatych}
\Clue{15}{}{miasto w Chinach (Jilin); eksploatacja węgla kamiennego; przemysł papierniczy, maszynowy, skórzany}
\Clue{16}{}{część osprzętu kotwicznego łącząca łańcuch z kotwicą; zapobiega skręceniu łańcucha}
\Clue{17}{}{liczba, do której jest dodawana inna liczba}
\Clue{19}{}{zaplanowany, ustalony sposób postępowania, który jest realizowany świadomie, by otrzymać jakiś rezultat}
\Clue{20}{}{środek cucący - lek pobudzający ośrodkowy układ nerwowy (zwłaszcza ośrodek oddechowy i naczynioruchowy), stosowany w stanach przemęczenia, depresji}
\Clue{23}{}{wysoki budynek, drapacz chmur}
\Clue{28}{}{przestrzeń nad najwyższym stropem lub sklepieniem, pod pokryciem dachowym, jako najwyższa kondygnacja budynku}
\Clue{30}{}{składka do Zakładu Ubezpieczeń Społecznych}
\Clue{31}{}{kompozytor włoski (1801-1835); opery w  stylu belcanta; 'Lunatyczka', Norma', 'Purytanie'}
\Clue{34}{}{symbol kulomba - jednostki ładunku elektrycznego w układzie SI}
\Clue{35}{}{metal lub półmetal charakteryzujący się dużą gęstością, często także właściwościami toksycznymi}
\Clue{36}{}{Bombina bombina - gatunek płaza z rodziny kumakowatych, blisko spokrewniony z kumakiem górskim, występujący na obszarze obejmującym Europę Środkową i Wschodnią, aż po Ural}
\Clue{37}{}{mięso drobiu}
\Clue{38}{}{epilog}
\Clue{39}{}{architekt zm. w 1742 r. - kościół Wizytek w Warszawie}
\Clue{40}{}{gatunek zwierzęcia lub rośliny odżywiający się żywymi lub martwymi tkankami zwierzęcymi}
\Clue{41}{}{gwiazdozbiór leżący w obrębie nieba południowego}\end{PuzzleClues}

\begin{PuzzleClues}{\textbf{Pionowe}\\}\Clue{1}{}{otoczona murem twierdza w kraju muzułmańskim, mająca za zadanie strzec granic państwowych, posiadająca salę modlitw}
\Clue{3}{}{potrawa kuchni hiszpańskiej, omlet z dodatkiem ziemniaków i innych składników}
\Clue{9}{}{uczniowie i pracownicy pracownicy szkoły, społeczność skupiona wokół szkoły - instytucji zajmującej się kształceniem dzieci i młodzieży}
\Clue{10}{}{znawca sztuki wojennej, biegły w sztuce dowodzenia}
\Clue{11}{}{Anasazisaurus - rodzaj dinozaura żyjący w późnej kredzie (74-70 mln lat temu) na terenach Ameryki Północnej; długość czaszki 90 cm}
\Clue{12}{}{osada w Polsce położona w województwie wielkopolskim, w powiecie śremskim, w gminie Śrem}
\Clue{13}{}{knotnik zwisły, Pohlia nutans - gatunek mchu z rodziny prątnikowatych; mech do 5 cm wysokości, tworzy zielone lub żółtozielone darnie, łodygi płodne, z lancetowatymi liśćmi perychecjalnymi, puszka gruszkowata, z krótką szyją, zwisająca}
\Clue{15}{}{specjalnie powołany zespół pracowników zakładu gazowniczego, mający na celu podjęcie działań w nagłych przypadkach ulatniania się gazu ziemnego w miejscach do tego nieprzeznaczonych}
\Clue{16}{}{fikcyjne państwo na wyspie Malaz, stworzone przez kanadyjskiego pisarza Stevena Eriksona w cyklu powieści fantasyMalazańska księga poległych}
\Clue{17}{}{nadanie cudzoziemcowi obywatelstwa państwa, na którego terenie się osiedlił}
\Clue{18}{}{zimny lub gorący okład}
\Clue{21}{}{choroba wywoływana przez tasiemce, w przebiegu której pojawiają się torbiele w ogranach wewnętrznych}
\Clue{22}{}{mieszkaniec Nowego Targu}
\Clue{23}{}{jeden z najstarszych typów ręcznej broni miotającej}
\Clue{24}{}{niewielka praca, której zadaniem jest wyjaśnienie ogólniejszej kwestii, częściowe bądź wstępne przedstawienie pewnego zagadnienia}
\Clue{25}{}{niewielka liczba członków}
\Clue{26}{}{nazwa motyla nocnego z rodziny omacnicowatych, którego gąsienice wyrządzają szkody, żywiąc się liśćmi i łodygami roślin}
\Clue{27}{}{grabież - bezprawne zabieranie siłą cudzych, wartościowych przedmiotów}
\Clue{28}{}{ostateczne, końcowe uderzenie}
\Clue{29}{}{dekoracyjne obramowanie płaskorzeźby, kompozycji malarskiej lub graficznej w formie pasa z ornamentem roślinnym}
\Clue{31}{}{kierunek poruszania się statku}
\Clue{32}{}{patki z odznakami stopni wojskowych na ramieniu munduru}
\Clue{33}{}{dawny urzędnik centralny i wyższy rangą oficer, do którego obowiązków należało dowodzenie taborem i obozem, prowadzenie wywiadu}\end{PuzzleClues}\newpage\section*{Krzyżówka 136}

\noindent\begin{Puzzle}{21}{28}|*	|*	|*	|*	|*	|*	|*	|*	|*	|*	|*	|*	|*	|*	|*	|[1][S]\drarr	|d	|o	|v	|e	|r	|*	|.
|*	|*	|*	|*	|*	|*	|*	|*	|*	|*	|*	|[2][S]\rarr	|z	|i	|o	|b	|r	|o	|*	|*	|[3][S]\darr	|*	|.
|*	|*	|[4][S]\darr	|[5][S]\darr	|*	|[6][S]\rarr	|t	|r	|a	|p	|e	|r	|s	|t	|w	|o	|*	|*	|*	|*	|b	|[7][S]\darr	|.
|*	|[8][S]\darr	|h	|a	|*	|*	|*	|[9][S]\darr	|*	|*	|*	|*	|[10][S]\drarr	|w	|o	|l	|e	|*	|*	|*	|a	|l	|.
|*	|o	|e	|n	|*	|*	|*	|b	|*	|[11][S]\rarr	|w	|y	|j	|a	|d	|a	|c	|z	|*	|[12][S]\darr	|c	|o	|.
|*	|p	|n	|e	|*	|[13][S]\rarr	|p	|r	|z	|y	|d	|r	|o	|ż	|e	|*	|*	|*	|*	|k	|z	|g	|.
|*	|a	|r	|g	|[14][S]\drarr	|p	|i	|e	|n	|i	|e	|*	|l	|*	|*	|*	|*	|*	|*	|a	|e	|i	|.
|[15][S]\drarr	|d	|y	|d	|k	|a	|*	|t	|*	|[16][S]\rarr	|l	|u	|a	|n	|s	|h	|y	|a	|*	|r	|k	|k	|.
|d	|*	|*	|o	|o	|[17][S]\rarr	|k	|a	|s	|j	|a	|*	|*	|[18][S]\drarr	|b	|i	|e	|r	|k	|a	|*	|a	|.
|w	|[19][S]\darr	|*	|t	|ś	|*	|[20][S]\drarr	|n	|i	|e	|u	|ż	|y	|t	|e	|c	|z	|n	|o	|ś	|ć	|*	|.
|a	|d	|*	|a	|c	|*	|o	|i	|*	|[21][S]\rarr	|h	|y	|d	|r	|o	|p	|l	|a	|n	|*	|*	|*	|.
|[][,]{ }	|w	|*	|*	|i	|*	|p	|a	|*	|[22][S]\rarr	|l	|o	|d	|ó	|w	|k	|a	|*	|*	|[23][S]\darr	|[24][S]\darr	|*	|.
|b	|u	|*	|*	|ó	|*	|ł	|*	|[25][S]\darr	|*	|*	|*	|*	|j	|*	|*	|[26][S]\darr	|*	|*	|b	|m	|*	|.
|i	|d	|[27][S]\darr	|[28][S]\darr	|ł	|*	|o	|*	|w	|*	|[29][S]\darr	|*	|*	|k	|*	|*	|d	|[30][S]\darr	|*	|i	|ą	|*	|.
|e	|z	|l	|d	|[][,]{ }	|[31][S]\darr	|t	|*	|a	|*	|d	|*	|*	|o	|*	|*	|r	|a	|*	|r	|k	|*	|.
|g	|i	|i	|a	|w	|h	|*	|*	|k	|*	|ę	|*	|*	|l	|*	|[32][S]\darr	|y	|r	|*	|m	|a	|*	|.
|u	|e	|ś	|r	|s	|y	|[33][S]\drarr	|d	|a	|k	|t	|y	|l	|o	|[][S]-	|ś	|l	|i	|w	|a	|*	|*	|.
|n	|s	|c	|t	|c	|d	|w	|[34][S]\drarr	|s	|e	|k	|t	|o	|r	|*	|w	|*	|o	|[35][S]\darr	|ń	|*	|*	|.
|y	|t	|i	|f	|h	|r	|i	|s	|a	|*	|a	|*	|[36][S]\darr	|o	|*	|i	|*	|s	|h	|c	|[37][S]\darr	|*	|.
|*	|y	|o	|o	|o	|o	|d	|e	|*	|*	|*	|*	|b	|w	|*	|a	|*	|t	|a	|z	|b	|*	|.
|*	|[][,]{ }	|n	|r	|d	|f	|z	|l	|*	|*	|*	|*	|i	|i	|*	|t	|*	|o	|d	|y	|r	|*	|.
|[38][S]\drarr	|p	|o	|d	|n	|i	|e	|b	|i	|e	|n	|i	|e	|*	|*	|*	|*	|*	|ż	|k	|a	|*	|.
|b	|i	|s	|*	|i	|t	|w	|y	|[39][S]\rarr	|d	|y	|s	|c	|y	|p	|l	|i	|n	|a	|*	|t	|*	|.
|a	|ą	|*	|*	|*	|*	|i	|*	|*	|*	|*	|*	|z	|[40][S]\rarr	|l	|e	|n	|a	|r	|d	|*	|*	|.
|t	|t	|[41][S]\rarr	|m	|s	|z	|a	|k	|i	|*	|[42][S]\rarr	|p	|a	|l	|o	|t	|y	|n	|*	|*	|*	|*	|.
|a	|y	|*	|*	|*	|*	|k	|[43][S]\rarr	|k	|a	|b	|a	|n	|o	|s	|*	|*	|*	|*	|*	|*	|*	|.
|k	|*	|[44][S]\rarr	|w	|i	|r	|*	|*	|[45][S]\rarr	|s	|c	|h	|i	|n	|k	|e	|l	|*	|*	|*	|*	|*	|.
|*	|*	|*	|[46][S]\rarr	|m	|a	|s	|a	|[][,]{ }	|s	|o	|l	|n	|a	|*	|*	|*	|*	|*	|*	|*	|*	|.
|*	|[47][S]\rarr	|p	|a	|s	|k	|a	|r	|s	|t	|w	|o	|*	|*	|*	|*	|*	|*	|*	|*	|*	|*	|.\end{Puzzle}

\newpage

\begin{PuzzleClues}{\textbf{Poziome}\\}\Clue{1}{}{miasto w Anglii nad kanałem La Manche, główny brytyjski port pasażerski w komunikacji z Europą}
\Clue{2}{}{pilot szybowcowy, wicemistrz świata z 1976 r. w klasie otwartej}
\Clue{6}{}{rodzaj myślistwa, głównie w Ameryce Północnej}
\Clue{10}{}{powiększenie tarczycy}
\Clue{11}{}{człowiek, który zna się na czymś bardzo dobrze, jest bardzo doświadczony}
\Clue{13}{}{pobocze, pas przy drodze, często zarośnięty roślinnością polną}
\Clue{14}{}{śpiewanie, śpiew}
\Clue{15}{}{pierś w gwarze poznańskiej}
\Clue{16}{}{miasto w płn. Zambii na obszarze Pasa Miedzionośnego; ośrodek eksploatacji rud miedzi}
\Clue{17}{}{cynamon chiński - przyprawa uzyskiwana z kory cynamonowca wonnego, mniej cenna niż cynamon cejloński}
\Clue{18}{}{pionki do gry w szachy czy warcaby; figura zabrana}
\Clue{20}{}{cecha człowieka, który nie jest w niczym pomocny, nie posiada odpowiednich cech by się do czegoś przydać, również człowiek nieuczynny}
\Clue{21}{}{WODNOSAMOLOT, WODNOPŁAT; samolot o specjalnej budowie mogący startować i lądować na wodzie}
\Clue{22}{}{pomieszczenie do przechowywania żywności, w którym przy pomocy lodu jest utrzymywana niska temperatura}
\Clue{33}{}{jadalny, smaczny owoc hurmy kaukaskiej}
\Clue{34}{}{część widowni}
\Clue{38}{}{widoczna z dołu (z wnętrza pomieszczenia) powierzchnia sklepienia}
\Clue{39}{}{karność, rygor, podporządkowanie zasadom, przepisom, regułom}
\Clue{40}{}{fizyk niemiecki (1862-1947); badacz luminescencji i zjawiska fotoelektrycznego, laureat Nobla}
\Clue{41}{}{grupa roślin charakteryzujących się przemianą pokoleń z dominacją gametofitu; dawniej jednostka taksonomiczna w randze gromady, dziś: grupa poza taksonomią}
\Clue{42}{}{członek zgromadzenia zakonnego założonego w 1835 r., którego celem były misje apostolskie}
\Clue{43}{}{cienka, suszona, wędzona kiełbasa}
\Clue{44}{}{wir powietrzny, zawirowanie cząsteczek powietrza, kręcące się masy powietrza}
\Clue{45}{}{malarz rosyjski (1830-97) członek pieriedwiżników}
\Clue{46}{}{masa do modelowania, którą wykonuje się z mąki, wody i soli}
\Clue{47}{}{nieuczciwy handel, polegający na sprzedaży po zawyżonych cenach deficytowych towarów}\end{PuzzleClues}

\begin{PuzzleClues}{\textbf{Pionowe}\\}\Clue{1}{}{broń myśliwska składająca się z jednego lub paru rzemieni, używana do polowania przez Indian Ameryki Płd. BOLAS}
\Clue{3}{}{rodzaj zarostu, który porasta bok twarzy}
\Clue{4}{}{fizyk amerykański (1797-1878); odkrył zjawisko indukcji wzajemnej i samoindukcję, wprowadził mapy pogody}
\Clue{5}{}{ujęta zwięźle treść dzieła}
\Clue{7}{}{logiczne myślenie, poprawny sposób rozumowania}
\Clue{8}{}{zjawisko, zdarzenie, nieintencjonalne spadanie z nieba jakiejś substancji}
\Clue{9}{}{kraina historyczna w płn-zach. Francji, gł. miasto: Rennes}
\Clue{10}{}{mała łódź sportowa ze skośnym żaglem}
\Clue{12}{}{językoznawca, polonista i slawista (1924-77); były rektor UJ}
\Clue{14}{}{każdy kościół, który wywodzi się z tradycji staroobrzędowej}
\Clue{15}{}{kolosalne różnice, różne światy}
\Clue{18}{}{drużyna reprezentująca kraj, którego flaga ma trzy kolory}
\Clue{19}{}{dwudziesty piąty dzień (najczęściej bieżącego lub przyszłego) miesiąca}
\Clue{20}{}{to, co oplata}
\Clue{23}{}{mieszkaniec Birmy, człowiek pochodzenia birmańskiego}
\Clue{24}{}{produkt powstały w wyniku silnego rozdrobnienia części rośliny (zazwyczaj ziarna zboża)}
\Clue{25}{}{zatoka Morza Japońskiego u wybrzeży Honsiu}
\Clue{26}{}{metoda (np. nauczania), polegająca na rygorystycznym, wielokrotnym powtarzaniu czegoś i prowadząca do zachowania absolutnej dyscypliny (pomaga w tym system przydzielanych za nieposłuszeństwo kar)}
\Clue{27}{}{nazwa nietoperzy z kilku gaunków należących do rodziny liścionosów; występują w Ameryce Środkowej i Południowej}
\Clue{28}{}{miasto w Anglii, w regionie Londynu}
\Clue{29}{}{komora z elastycznego materiału, wypełniona gazem (najczęściej powietrzem) tłoczonym poprzez wentyl; element ogumienia pojazdów}
\Clue{30}{}{poeta włoskiego renesansu (1474-1533); „Orland szalony”}
\Clue{31}{}{roślina wodna przytwierdzona do dna lub swobodnie pływająca}
\Clue{32}{}{kultura w węższym znaczeniu - ogół wytworów i osiągnieć określonego społeczeństwa w danej epoce historycznej}
\Clue{33}{}{kibic Widzewa Łódź}
\Clue{34}{}{miasto w środkowej Anglii}
\Clue{35}{}{Czarny Kamień w świątyni Kaaba czczony przez pielgrzymujących do Mekki muzułmanów}
\Clue{36}{}{mieszkaniec Biecza}
\Clue{37}{}{inny człowiek (rzadziej: inne stworzenie), każdy bliźni}
\Clue{38}{}{jezioro w Bułgarii, w pobliżu Plovdiv}\end{PuzzleClues}\newpage\section*{Krzyżówka 137}

\noindent\begin{Puzzle}{25}{27}|*	|*	|*	|*	|*	|[1][S]\drarr	|o	|c	|h	|o	|t	|k	|a	|*	|[2][S]\drarr	|k	|o	|n	|o	|p	|n	|i	|c	|a	|*	|*	|.
|*	|*	|*	|*	|*	|p	|*	|*	|[3][S]\drarr	|c	|h	|o	|r	|ą	|ż	|y	|[][,]{ }	|s	|z	|t	|a	|b	|o	|w	|y	|*	|.
|*	|*	|*	|*	|[4][S]\darr	|y	|*	|*	|s	|*	|*	|*	|[5][S]\rarr	|g	|o	|s	|t	|y	|n	|i	|a	|n	|i	|n	|*	|*	|.
|*	|*	|*	|*	|k	|t	|*	|*	|u	|*	|*	|*	|[6][S]\rarr	|d	|ł	|u	|g	|o	|s	|z	|p	|o	|n	|*	|*	|*	|.
|*	|*	|*	|*	|a	|o	|*	|[7][S]\drarr	|p	|a	|s	|k	|o	|w	|n	|i	|k	|i	|*	|*	|*	|*	|*	|*	|*	|*	|.
|*	|*	|*	|*	|c	|n	|*	|t	|e	|*	|*	|*	|*	|*	|i	|*	|*	|*	|*	|*	|*	|*	|*	|*	|*	|*	|.
|*	|*	|*	|*	|z	|[][,]{ }	|*	|r	|r	|*	|*	|*	|[8][S]\rarr	|d	|e	|s	|t	|y	|l	|a	|t	|o	|r	|*	|*	|*	|.
|*	|*	|*	|*	|k	|k	|*	|y	|p	|*	|*	|[9][S]\rarr	|s	|t	|r	|z	|e	|l	|c	|z	|y	|n	|i	|*	|*	|*	|.
|*	|*	|*	|*	|a	|r	|[10][S]\rarr	|b	|o	|j	|k	|o	|w	|s	|z	|c	|z	|y	|z	|n	|a	|*	|*	|[11][S]\darr	|*	|*	|.
|*	|*	|*	|*	|[][,]{ }	|ó	|*	|*	|z	|*	|*	|*	|*	|*	|[][,]{ }	|[12][S]\darr	|[13][S]\darr	|*	|*	|*	|*	|*	|*	|o	|*	|*	|.
|*	|*	|*	|*	|r	|t	|*	|*	|y	|*	|*	|*	|*	|*	|l	|b	|k	|*	|*	|*	|*	|*	|*	|s	|[14][S]\darr	|*	|.
|*	|*	|*	|*	|d	|k	|*	|*	|t	|*	|*	|*	|*	|*	|i	|a	|r	|*	|*	|*	|*	|*	|[15][S]\darr	|t	|n	|*	|.
|*	|*	|*	|*	|z	|o	|*	|*	|y	|*	|*	|*	|*	|*	|n	|t	|z	|*	|*	|*	|*	|*	|l	|r	|a	|*	|.
|*	|*	|*	|*	|a	|o	|*	|*	|w	|*	|*	|*	|*	|*	|i	|e	|y	|*	|*	|[16][S]\darr	|*	|*	|a	|z	|s	|*	|.
|*	|*	|*	|[17][S]\darr	|w	|g	|*	|*	|*	|*	|*	|*	|*	|*	|o	|r	|ż	|*	|[18][S]\darr	|w	|*	|*	|t	|e	|i	|*	|.
|*	|*	|*	|d	|o	|o	|*	|*	|*	|[19][S]\darr	|*	|*	|*	|*	|w	|i	|ó	|*	|p	|e	|*	|*	|o	|s	|a	|*	|.
|*	|*	|*	|y	|g	|n	|*	|*	|*	|t	|*	|*	|*	|[20][S]\darr	|y	|a	|w	|*	|a	|g	|*	|*	|s	|z	|d	|*	|.
|*	|*	|[21][S]\drarr	|m	|ł	|o	|d	|z	|i	|e	|ż	|ó	|w	|a	|*	|[][,]{ }	|k	|*	|r	|a	|*	|*	|z	|y	|ó	|*	|.
|*	|*	|c	|n	|o	|w	|*	|*	|*	|o	|*	|[22][S]\rarr	|p	|r	|o	|b	|a	|n	|t	|*	|*	|*	|y	|n	|w	|*	|.
|*	|*	|h	|i	|w	|y	|*	|*	|*	|w	|*	|*	|*	|t	|*	|u	|*	|*	|*	|*	|*	|*	|ń	|[][,]{ }	|k	|*	|.
|*	|*	|r	|c	|a	|*	|*	|*	|*	|n	|*	|*	|*	|y	|*	|f	|*	|*	|*	|*	|*	|*	|s	|j	|a	|*	|.
|*	|*	|u	|a	|*	|[23][S]\rarr	|p	|y	|s	|i	|o	|*	|[24][S]\rarr	|s	|t	|o	|c	|h	|a	|s	|t	|y	|k	|a	|*	|*	|.
|*	|*	|p	|*	|*	|*	|*	|*	|*	|k	|*	|*	|*	|t	|*	|r	|*	|*	|*	|*	|*	|*	|i	|d	|*	|*	|.
|[25][S]\rarr	|s	|k	|u	|p	|i	|s	|k	|o	|*	|[26][S]\rarr	|h	|e	|k	|t	|o	|p	|a	|s	|k	|a	|l	|*	|a	|*	|*	|.
|*	|*	|o	|*	|[27][S]\rarr	|p	|r	|z	|y	|m	|u	|s	|[][,]{ }	|a	|d	|w	|o	|k	|a	|c	|k	|i	|*	|l	|*	|*	|.
|*	|[28][S]\rarr	|ś	|m	|i	|e	|c	|i	|u	|s	|z	|k	|a	|*	|*	|a	|*	|*	|*	|*	|*	|*	|*	|n	|*	|*	|.
|*	|*	|ć	|*	|[29][S]\rarr	|k	|n	|i	|a	|z	|i	|ó	|w	|n	|a	|*	|*	|*	|*	|*	|*	|*	|*	|y	|*	|*	|.
|*	|*	|*	|*	|*	|*	|*	|*	|*	|*	|*	|*	|*	|*	|*	|*	|*	|*	|*	|*	|*	|*	|*	|*	|*	|*	|.\end{Puzzle}

\newpage

\begin{PuzzleClues}{\textbf{Poziome}\\}\Clue{1}{}{nadwodna muchówka, larwy żyją w wodzie będąc pokarmem dla ryb}
\Clue{2}{}{Datisca - rodzaj bylin z rodziny konopnicowatych; obejmuje dwa gatunki}
\Clue{3}{}{żołnierz noszący tytuł chorążego sztabowego}
\Clue{5}{}{człowiek mieszkający w Gostyniu}
\Clue{6}{}{ptak z rzędu siewkowatych o długich: nogach, palcach i pazurach; tropikalne wody stojące}
\Clue{7}{}{Aegithinidae - monotypowa rodzina ptaków z rzędu wróblowych (Passeriformes)}
\Clue{8}{}{urządzenie, które służy do destylacji}
\Clue{9}{}{kobieta, która potrafi strzelać}
\Clue{10}{}{historyczno-etnograficzny obszar zamieszkany przez Bojków, na północnych oraz południowych stokach Karpat}
\Clue{21}{}{kobieta, która stara się wyglądać jak nastolatka, zachowuje się tak, chciałaby prowadzić podobny, luźny sposób bycia; może też przestawać z młodzieżą, pozować na równiachę}
\Clue{22}{}{nowicjusz, osoba przechodząca okres probacji w zakonie}
\Clue{23}{}{o twarzy, buzi; słowo pieszczotliwe}
\Clue{24}{}{dział matematyki; bada modele zjawisk losowych w czasie}
\Clue{25}{}{miejsce nagromadzenia ludzi lub innych organizmów żywych, rzeczy, zjawisk}
\Clue{26}{}{jednostka ciśnienia, równa 100 paskalom (Pa), najczęściej stosowana przy podawaniu ciśnienia atmosferycznego}
\Clue{27}{}{obowiązek zastąpienia strony przez fachowego pełnomocnika przy dokonywaniu określonej czynności procesowej albo w określonej fazie postępowania}
\Clue{28}{}{DZIERLATKA, POŚMIECIUSZKA}
\Clue{29}{}{córka kniazia}\end{PuzzleClues}

\begin{PuzzleClues}{\textbf{Pionowe}\\}\Clue{1}{}{pyton krwisty, Python curtus - gatunek gada z rodziny pytonów, występujący na Sumatrze, Borneo, niektórych wyspach Malezji i Indonezji}
\Clue{2}{}{żołnierz wystawiony do walki w pierwszej linii frontu}
\Clue{3}{}{bardzo pozytywny komentarz, wieńczący jakąś internetową transakcję; opinia, która jest wyjątkowo dobra}
\Clue{4}{}{głowienka, głowienka zwyczajna, Aythya ferina - gatunek ptaka z rodziny kaczkowatych (Anatidae); zamieszkuje środkowe szerokości geograficzne Eurazji - Wyspy Brytyjskie, Europę Środkową i Wschodnią i pas w Azji Środkowej po Mandżurię i północną Japonię, poza tym izolowana populacja występuje w Azji Mniejszej}
\Clue{7}{}{jeden z typów skali muzycznej}
\Clue{11}{}{Calathea allouia - gatunek rośliny z rodziny marantowatych}
\Clue{12}{}{bateria akumulatorowa połączona równolegle z prądnicami prądu stałego}
\Clue{13}{}{krzyżówka zwyczajna, Anas platyrhynchos platyrhynchos - nominatywny podgatunek ptaka, wyróżniony w obrębie gatunku krzyżówka (Anas platyrhynchos)}
\Clue{14}{}{spotkanie, zebranie, które jest nudne i zbyt długie}
\Clue{15}{}{kompozytor ukraiński (1895-1968); utwory symfoniczne kameralne opery kantaty}
\Clue{16}{}{najjaśniejsza gwiazda w gwiazdozbiorze Lutni}
\Clue{17}{}{roślina zielna z makowatych, lecznicza, w Polsce zwana polną rutką}
\Clue{18}{}{taśma, plecionka ze szpagatu używana w różnych rzemiosłach na giętkie połączenia pomiędzy sztywnymi elementami}
\Clue{19}{}{kształtownik o przekroju zbliżonym do litery „T”}
\Clue{20}{}{o kobiecie, która robi coś dziwnego, ma opinię ekscentryczki, jest skłonna do wygłupiania się}
\Clue{21}{}{cecha przedmiotu, który łatwo się łamie, łupie, kruszy na większe kawałki}\end{PuzzleClues}\newpage\section*{Krzyżówka 138}

\noindent\begin{Puzzle}{23}{32}|*	|[1][S]\darr	|*	|*	|*	|[2][S]\drarr	|t	|r	|o	|p	|i	|k	|*	|[3][S]\darr	|*	|[4][S]\drarr	|v	|a	|r	|k	|a	|u	|s	|*	|.
|*	|o	|*	|[5][S]\drarr	|e	|p	|o	|k	|a	|[][,]{ }	|g	|i	|e	|r	|k	|o	|w	|s	|k	|a	|*	|*	|[6][S]\darr	|*	|.
|*	|g	|[7][S]\darr	|r	|[8][S]\rarr	|o	|k	|r	|z	|o	|s	|*	|*	|o	|*	|s	|*	|[9][S]\drarr	|z	|e	|f	|i	|r	|*	|.
|*	|n	|j	|o	|*	|m	|[10][S]\darr	|[11][S]\rarr	|k	|a	|r	|k	|a	|s	|*	|k	|[12][S]\rarr	|p	|e	|r	|ć	|*	|e	|*	|.
|*	|i	|e	|m	|*	|i	|w	|[13][S]\drarr	|w	|o	|ł	|o	|s	|z	|k	|a	|*	|r	|*	|*	|*	|*	|a	|*	|.
|[14][S]\drarr	|w	|s	|p	|i	|n	|a	|c	|z	|k	|a	|[][,]{ }	|s	|p	|o	|r	|t	|o	|w	|a	|*	|*	|d	|*	|.
|b	|o	|i	|*	|[15][S]\darr	|i	|t	|z	|*	|*	|*	|*	|*	|u	|*	|ż	|*	|c	|*	|[16][S]\darr	|[17][S]\darr	|*	|i	|*	|.
|ł	|*	|o	|*	|k	|ę	|e	|o	|*	|*	|[18][S]\darr	|[19][S]\darr	|*	|n	|*	|e	|[20][S]\darr	|e	|*	|e	|k	|[21][S]\darr	|n	|*	|.
|ę	|[22][S]\darr	|t	|[23][S]\darr	|a	|c	|r	|s	|*	|*	|h	|k	|*	|k	|*	|n	|r	|s	|[24][S]\darr	|r	|o	|h	|g	|*	|.
|d	|t	|r	|k	|n	|i	|s	|n	|*	|[25][S]\drarr	|u	|r	|w	|a	|n	|i	|e	|[][,]{ }	|d	|u	|p	|y	|*	|*	|.
|n	|e	|o	|r	|i	|e	|z	|k	|*	|k	|m	|ę	|*	|[][,]{ }	|[26][S]\darr	|e	|m	|s	|e	|p	|a	|d	|*	|*	|.
|i	|r	|k	|o	|u	|*	|t	|o	|*	|o	|o	|g	|*	|z	|m	|*	|a	|p	|k	|c	|r	|r	|[27][S]\darr	|*	|.
|k	|m	|s	|p	|k	|*	|a	|w	|*	|ź	|r	|o	|*	|ą	|a	|[28][S]\darr	|k	|o	|o	|j	|a	|o	|s	|*	|.
|[][,]{ }	|o	|z	|l	|[][,]{ }	|*	|g	|a	|*	|l	|*	|w	|*	|b	|r	|s	|e	|ł	|n	|a	|*	|m	|k	|[29][S]\darr	|.
|k	|r	|t	|a	|a	|*	|*	|t	|*	|a	|*	|c	|[30][S]\darr	|k	|a	|z	|*	|e	|s	|*	|[31][S]\darr	|o	|r	|k	|.
|o	|e	|a	|[][,]{ }	|u	|*	|*	|e	|*	|k	|*	|e	|a	|o	|d	|a	|*	|c	|t	|*	|k	|r	|ę	|u	|.
|s	|g	|ł	|d	|s	|*	|*	|*	|*	|*	|*	|*	|p	|w	|o	|t	|*	|z	|r	|*	|a	|f	|t	|r	|.
|t	|u	|t	|e	|t	|*	|*	|*	|[32][S]\rarr	|f	|o	|n	|t	|a	|n	|a	|*	|n	|u	|*	|n	|o	|e	|o	|.
|n	|l	|n	|s	|r	|*	|*	|*	|*	|*	|*	|[33][S]\darr	|*	|n	|a	|n	|*	|y	|k	|[34][S]\darr	|o	|n	|k	|n	|.
|y	|a	|e	|z	|a	|[35][S]\rarr	|c	|o	|l	|u	|m	|b	|i	|a	|*	|*	|*	|*	|t	|f	|p	|*	|[][,]{ }	|i	|.
|*	|c	|*	|c	|l	|[36][S]\drarr	|g	|e	|o	|g	|r	|a	|f	|*	|*	|[37][S]\darr	|[38][S]\darr	|*	|y	|r	|a	|*	|m	|ó	|.
|[39][S]\drarr	|j	|e	|z	|i	|o	|r	|o	|[][,]{ }	|w	|y	|t	|o	|p	|i	|s	|k	|o	|w	|e	|*	|[40][S]\darr	|i	|w	|.
|g	|a	|*	|o	|j	|p	|*	|[41][S]\rarr	|w	|s	|c	|h	|ó	|d	|*	|c	|o	|*	|i	|i	|*	|b	|e	|k	|.
|r	|*	|*	|w	|s	|e	|*	|*	|*	|[42][S]\rarr	|l	|u	|d	|e	|k	|*	|m	|*	|z	|s	|[43][S]\darr	|a	|s	|a	|.
|ó	|[44][S]\darr	|*	|a	|k	|r	|*	|[45][S]\rarr	|t	|w	|a	|r	|d	|o	|ś	|c	|i	|o	|m	|i	|e	|r	|z	|*	|.
|d	|t	|*	|*	|i	|a	|*	|*	|[46][S]\drarr	|a	|d	|s	|o	|r	|b	|e	|n	|t	|*	|n	|l	|a	|a	|*	|.
|*	|a	|*	|*	|*	|*	|*	|*	|b	|*	|*	|t	|[47][S]\rarr	|n	|i	|m	|*	|*	|*	|g	|e	|n	|ń	|*	|.
|[48][S]\drarr	|n	|a	|d	|c	|z	|u	|ł	|o	|ś	|ć	|*	|[49][S]\rarr	|ł	|o	|d	|z	|i	|k	|*	|m	|e	|c	|*	|.
|r	|t	|[50][S]\rarr	|z	|m	|i	|e	|n	|n	|o	|k	|s	|z	|t	|a	|ł	|t	|n	|y	|*	|e	|k	|o	|*	|.
|a	|r	|*	|[51][S]\rarr	|m	|e	|j	|o	|z	|a	|*	|*	|[52][S]\rarr	|k	|ł	|o	|c	|h	|t	|u	|n	|*	|w	|*	|.
|k	|a	|*	|[53][S]\rarr	|b	|i	|a	|ł	|o	|r	|u	|s	|z	|c	|z	|y	|z	|n	|a	|*	|t	|*	|y	|*	|.
|*	|*	|[54][S]\rarr	|b	|u	|ł	|a	|t	|*	|*	|*	|*	|*	|*	|*	|*	|*	|*	|[55][S]\rarr	|ż	|y	|d	|*	|*	|.
|*	|*	|*	|*	|*	|*	|[56][S]\rarr	|t	|o	|t	|a	|l	|i	|t	|a	|r	|y	|s	|t	|a	|*	|*	|*	|*	|.\end{Puzzle}

\newpage

\begin{PuzzleClues}{\textbf{Poziome}\\}\Clue{2}{}{wielki upał z dużą wilgotnością powietrza, charakterystyczny dla strefy międzyzwrotnikowej}
\Clue{4}{}{miasto w płd.-wsch. Finlandii; przemysł drzewny, celulozowo-papierniczy}
\Clue{5}{}{okres sprawowania przez Edwarda Gierka władzy na stanowisku I sekretarza KC PZPR w latach 1970-1980}
\Clue{8}{}{kawałek drewna lub kamienia, który odpadł przy okrzesywaniu czegoś}
\Clue{9}{}{bawełniana cienka i gęsta tkanina o gładkiej powierzchni, która zwykle posiada jakiś efekt tkacki (fakturę, widoczny wzór, jeśli wykonana z użyciem kolorowych nici), stosowana na koszule męskie i bluzki damskie}
\Clue{11}{}{dawny pocisk artyleryjski składający się z worka umocowanego na metalowych żebrach i wypełnionego materiałem wybuchowym, stosowany dawniej w działach gładkolufowych}
\Clue{12}{}{stroma górska ścieżka}
\Clue{13}{}{ubiór wojskowy zwany też krakuską; długi do kolan, wcięty, mający na obu piersiach, wzorem ludów zakaukaskich, kieszonki na ładunki w postaci tulejek w liczbie 5-7 i więcej}
\Clue{14}{}{wspinaczka górska lub organizowana na ściankach wspinaczkowych, która jest elementem zawodów, towarzyszy jej rywalizacja sportowa}
\Clue{25}{}{nagromadzenie zajęć, obowiązków, które trzeba wykonać w krótkim czasie, prowadzące niekiedy do chaosu}
\Clue{32}{}{Domenico (1543-1607), włoski architekt przełomu renesansu i baroku}
\Clue{35}{}{pierwszy wahadłowiec amerykański, start pierwszego z nich w 1981 r}
\Clue{36}{}{nauczyciel uczący w szkole geografii}
\Clue{39}{}{jezioro polodowcowe powstałe w zagłębieniu utworzonym po wytopieniu się brył martwego lodu, klinów lodowych lub soczewek lodu gruntowego}
\Clue{41}{}{zespół państw, które wprowadziły bądź starały się wprowadzić zmiany życia społecznego i gospodarczego zgodne z ideologią komunizmu, na który składały się: dyktatura proletariatu, gospodarka planowa oraz idee sprawiedliwości społecznej; blok ten był wyróżniany od zakończenia II wojny światowej do początku lat 90. XX wieku}
\Clue{42}{}{niewielka grupa ludzi, mały tłum}
\Clue{45}{}{przyrząd mierzący twardość różnych materiałów}
\Clue{46}{}{substancja, będąca sorbentem w reakcji adsorpcji, czyli sorpcji powierzchniowej}
\Clue{47}{}{stara chińska gra planszowa dla dwóch osób z użyciem 15 do 60 pionków, nazywana potocznie grą w zabieranie kamieni}
\Clue{48}{}{wzmożona pobudliwość nerwów na bodźce zewnętrzne, stan przeczulenia jakiegoś zmysłu, np. wzroku, słuchu}
\Clue{49}{}{głowonóg z podgromady łodzikowców}
\Clue{50}{}{w fantastyce: stworzenie, postać, która może zmieniać formę, przeistaczać się np. z człowieka w zwierzę}
\Clue{51}{}{proces podziału redukcyjnego jądra komórkowego, z którego powstają 4 jądra o połowie chromosomów (po jednym z każdej pary) komórki wyjściowej}
\Clue{52}{}{cyraneczka bajkalska, cyranka syberyjska, Anas formosa - gatunek średniego, wędrownego ptaka wodnego z rodziny kaczkowatych (Anatidae); zamieszkuje środkową i wschodnią Syberię oraz wybrzeża Bajkału}
\Clue{53}{}{język białoruski}
\Clue{54}{}{rodzaj szabli orientalnej, wykonanej z bardzo twardej i sprężystej stali; oprócz materiału (stali zwanej bułatową lub damasceńską) broń ta wyróżniała się także krótką, krzywą głownią o szerokim piórze, rozszerzającym się ku końcowi}
\Clue{55}{}{osoba chciwa, skąpa}
\Clue{56}{}{zwolennik totalitaryzmu}\end{PuzzleClues}

\begin{PuzzleClues}{\textbf{Pionowe}\\}\Clue{1}{}{jedno z kółek tworzących łańcuch}
\Clue{2}{}{opuszczenie, nieporuszenie czegoś w tekście, niewymienienie}
\Clue{3}{}{Valerianella dentata - gatunek roszpunki, rośliny należącej do rodziny kozłkowatych}
\Clue{4}{}{czynność prawna; pismo procesowe wnoszone do sądu w postępowaniu karnym przez uprawniony podmiot, w którym domaga się on od sądu wydania orzeczenia o winie i karze lub środkach karnych wobec osoby wskazanej w tym piśmie}
\Clue{5}{}{kołnierz na wewnętrznej stronie kafla}
\Clue{6}{}{miasto w USA (Pensylwania) nad rzeką Schuylkill}
\Clue{7}{}{Acipenseriformes - rząd ryb promieniopłetwych (Actinopterygii) obejmujący około 30 współcześnie żyjących gatunków o wielu archaicznych cechach budowy, w tym typowych dla rekinów; ryby te mają duże znaczenie gospodarcze ze względu na smaczne i delikatne mięso oraz czarny kawior wyrabiany z ich ikry, jesiotrokształtne zamieszkują morskie, słodkie i słonawe wody półkuli północnej - Europy, Ameryki Północnej i północnej Azji}
\Clue{9}{}{seria zjawisk mających wpływ na osobowość jednostki bądź grupy społecznej}
\Clue{10}{}{lina łącząca nok bukszprytu z dziobem statku, usztywnia bukszpryt}
\Clue{13}{}{Allioideae Herbert -  podrodzina roślin zielnych zaliczana do amarylkowatych (Amaryllidaceae)}
\Clue{14}{}{część ucha wewnętrznego. Jest on wypełniony płynem, zwanym przychłonką. Wewnątrz błędnika kostnego mieści się błędnik błoniasty zawieszony na licznych pasemkach łącznotkankowych. Błędnik kostny składa się z: przewodu słuchowego wewnętrznego, przedsionka, trzech kanałów półkolistych i ślimaka.}
\Clue{15}{}{Elanus axillaris - gatunek ptaka drapieżnego z rodziny jastrzębiowatych (Accipitridae), z podrodziny kaniuków (Elaninae)}
\Clue{16}{}{przen. wybuch uczuć}
\Clue{17}{}{żuchwa, dolna szczęka}
\Clue{18}{}{zdolność dostrzegania tego, co śmieszne; bystrość umysłu, która pomaga w zabawny, błyskotliwy sposób mówić i się zachowywać}
\Clue{19}{}{podtyp strunowców, których szkieletem osiowym jest kręgosłup}
\Clue{20}{}{nowa wersja istniejącego wcześniej filmu, często ze zmianami}
\Clue{21}{}{silny środek odurzający, mający krótsze działanie przeciwbólowe od morfiny; łatwo wywołuje euforię, ma właściwości uzależniające}
\Clue{22}{}{proces utrzymywania optymalnej temperatury}
\Clue{23}{}{kropla wodna o średnicy większej niż 0,5mm, spadająca w atmosferze}
\Clue{24}{}{kierunek w architekturze rozpoczęty w drugiej połowie lat 80. XX wieku}
\Clue{25}{}{najstarszy i dość prymitywny typ wiatraka europejskiego}
\Clue{26}{}{piłkarz argentyński, uznawany za następcę Pelego, mistrz i wicemistrz świata z 1986 i 1990 r., zdyskwalifikowany za stosowanie narkotyków}
\Clue{27}{}{Funaria hybrida - gatunek mchu z rodziny skrętkowanych}
\Clue{28}{}{z podziwem o kimś, kto jest np. sprytny, szybki itp}
\Clue{29}{}{utarte określenie darmowego posiłku (ściślej mówiąc zupy - była to podobno grochówka), który był wydawany na ulicach Warszawy biednym mieszkańcom miasta m.in. osobiście przez ministra pracy i polityki społecznej Jacka Kuronia na początku lat 90. XX wieku}
\Clue{30}{}{astronauta amerykański biorący udział w wyprawie Atlantisa w 1991 r}
\Clue{31}{}{egipska lub etruska urna do przechowywania wnętrzności zmarłego}
\Clue{33}{}{miasto w Australii (Nowa Płd. Walia) nad rzeką Macquarie}
\Clue{34}{}{miasto w Niemczech (Bawaria) nad Izarą; ośrodek kulturalno-naukowy, fabryka spożywczy, drzewny jedwabniczy ciągników, odlewnia}
\Clue{36}{}{komiczne lub dramatyczne zdarzenie, które ma świadków}
\Clue{37}{}{w chemii: symbol skandu}
\Clue{38}{}{w geologii - naturlanie powstała w skale wnęka o pionowych ścianach}
\Clue{39}{}{siedziba księcia, warownia, w której mieszkali ludzie, toczyło się życie}
\Clue{40}{}{przen. człowiek o potulnym, zgodnym charakterze, cichy i grzeczny}
\Clue{43}{}{podstawowe, elementarne informacje z jakiejś dziedziny}
\Clue{44}{}{tantryczne pismo objawione}
\Clue{46}{}{człowiek wyniosły, hardy; potentat}
\Clue{48}{}{osoba spod znaku Raka}\end{PuzzleClues}\newpage\section*{Krzyżówka 139}

\noindent\begin{Puzzle}{21}{32}|*	|*	|*	|*	|*	|*	|*	|*	|*	|*	|*	|*	|*	|*	|[1][S]\darr	|[2][S]\darr	|[3][S]\darr	|*	|[4][S]\darr	|*	|*	|*	|.
|*	|*	|[5][S]\darr	|*	|[6][S]\darr	|*	|*	|*	|*	|[7][S]\rarr	|b	|r	|y	|l	|a	|n	|t	|*	|i	|*	|*	|*	|.
|[8][S]\drarr	|f	|a	|n	|a	|t	|y	|c	|z	|n	|o	|ś	|ć	|*	|s	|o	|r	|[9][S]\darr	|n	|*	|*	|*	|.
|k	|[10][S]\drarr	|p	|e	|r	|r	|a	|u	|l	|t	|*	|*	|*	|*	|z	|n	|ę	|n	|s	|*	|*	|[11][S]\darr	|.
|e	|c	|a	|*	|i	|*	|*	|[12][S]\darr	|*	|*	|*	|*	|*	|*	|*	|a	|d	|i	|t	|*	|*	|s	|.
|p	|h	|r	|*	|o	|[13][S]\rarr	|z	|b	|i	|o	|r	|ó	|w	|k	|a	|*	|o	|e	|r	|*	|*	|a	|.
|l	|e	|a	|[14][S]\darr	|s	|[15][S]\rarr	|d	|e	|r	|o	|g	|a	|c	|j	|a	|*	|w	|z	|u	|*	|*	|m	|.
|e	|e	|t	|r	|o	|[16][S]\rarr	|r	|e	|g	|i	|s	|t	|e	|r	|*	|*	|n	|l	|m	|*	|*	|o	|.
|r	|s	|[][,]{ }	|a	|*	|[17][S]\rarr	|s	|t	|e	|k	|o	|w	|c	|e	|*	|*	|i	|i	|e	|*	|*	|g	|.
|*	|e	|t	|s	|*	|[18][S]\rarr	|c	|h	|w	|y	|t	|n	|i	|k	|*	|[19][S]\darr	|k	|c	|n	|*	|*	|ł	|.
|[20][S]\rarr	|s	|e	|k	|u	|n	|d	|o	|g	|e	|n	|i	|t	|u	|r	|a	|*	|z	|t	|[21][S]\darr	|*	|o	|.
|*	|t	|l	|*	|*	|*	|*	|v	|[22][S]\darr	|[23][S]\rarr	|f	|u	|l	|l	|e	|r	|*	|o	|[][,]{ }	|k	|*	|s	|.
|*	|e	|e	|[24][S]\drarr	|k	|a	|p	|e	|t	|y	|n	|g	|o	|w	|i	|e	|*	|n	|d	|o	|*	|k	|.
|*	|a	|f	|o	|[25][S]\rarr	|g	|e	|n	|e	|r	|a	|l	|i	|c	|j	|a	|*	|o	|ę	|l	|*	|a	|.
|[26][S]\rarr	|k	|o	|r	|o	|n	|a	|*	|t	|*	|*	|*	|*	|*	|[27][S]\darr	|*	|*	|ś	|t	|e	|[28][S]\darr	|[][,]{ }	|.
|*	|*	|n	|l	|[29][S]\darr	|*	|[30][S]\darr	|*	|*	|*	|*	|[31][S]\rarr	|z	|o	|m	|o	|*	|ć	|y	|k	|s	|p	|.
|[32][S]\drarr	|n	|i	|e	|d	|ź	|w	|i	|e	|d	|ź	|[][,]{ }	|s	|z	|a	|r	|y	|*	|[][,]{ }	|t	|t	|r	|.
|k	|*	|c	|ń	|z	|*	|i	|[33][S]\rarr	|w	|ę	|g	|l	|ó	|w	|k	|a	|*	|*	|d	|o	|r	|z	|.
|w	|*	|z	|[][,]{ }	|i	|*	|e	|*	|[34][S]\drarr	|k	|o	|m	|o	|r	|a	|*	|[35][S]\darr	|*	|r	|r	|o	|e	|.
|a	|*	|n	|k	|e	|*	|l	|*	|p	|*	|*	|*	|[36][S]\darr	|*	|k	|*	|q	|*	|e	|[][,]{ }	|n	|d	|.
|s	|*	|y	|a	|r	|*	|o	|*	|i	|*	|*	|*	|t	|*	|[][,]{ }	|[37][S]\darr	|u	|*	|w	|s	|n	|n	|.
|[][,]{ }	|*	|*	|l	|z	|*	|e	|*	|ł	|*	|*	|*	|o	|*	|t	|f	|e	|*	|n	|ł	|i	|i	|.
|g	|*	|*	|i	|b	|*	|t	|*	|k	|*	|*	|*	|c	|*	|y	|o	|n	|[38][S]\darr	|i	|o	|c	|a	|.
|l	|*	|*	|f	|a	|*	|a	|*	|a	|*	|*	|*	|z	|[39][S]\rarr	|b	|r	|y	|g	|a	|n	|t	|*	|.
|u	|*	|*	|o	|*	|*	|t	|*	|[][,]{ }	|*	|*	|*	|y	|*	|e	|e	|a	|i	|n	|e	|w	|*	|.
|k	|*	|*	|r	|*	|*	|o	|*	|n	|*	|*	|*	|s	|*	|t	|m	|*	|g	|y	|c	|o	|*	|.
|o	|*	|*	|n	|*	|*	|w	|*	|o	|*	|*	|[40][S]\rarr	|k	|ł	|a	|k	|i	|*	|*	|z	|*	|*	|.
|n	|*	|*	|i	|*	|*	|o	|*	|ż	|*	|*	|*	|o	|*	|ń	|a	|*	|*	|*	|n	|[41][S]\darr	|*	|.
|o	|*	|*	|j	|*	|*	|ś	|*	|n	|*	|*	|*	|*	|*	|s	|*	|*	|*	|*	|y	|y	|*	|.
|w	|*	|*	|s	|*	|*	|ć	|*	|a	|[42][S]\rarr	|w	|y	|p	|u	|k	|l	|i	|n	|a	|*	|b	|*	|.
|y	|*	|*	|k	|*	|*	|*	|*	|*	|*	|*	|[43][S]\rarr	|s	|p	|i	|r	|y	|t	|u	|s	|*	|*	|.
|*	|*	|*	|i	|*	|*	|*	|*	|*	|[44][S]\rarr	|d	|u	|m	|p	|*	|*	|*	|*	|*	|*	|*	|*	|.
|*	|*	|*	|*	|*	|*	|*	|*	|*	|*	|*	|*	|*	|*	|*	|*	|*	|*	|*	|*	|*	|*	|.\end{Puzzle}

\newpage

\begin{PuzzleClues}{\textbf{Poziome}\\}\Clue{7}{}{oszlifowany diament, często jako część biżuterii}
\Clue{8}{}{to, że coś jest wykonywane z fanatyzmem, nadmierną gorliwością, bezrefleksyjnym zaangażowaniem}
\Clue{10}{}{architekt francuski (1613-88), przedstawiciel barokowego klasycznego Ludwika XIV}
\Clue{13}{}{zdjęcie grupowe}
\Clue{15}{}{uchylenie części normy prawnej i zastąpienie jej nową}
\Clue{16}{}{w drukarstwie - cecha estetyczna złamanego tekstu polegająca na takim rozmieszczeniu łamów w kolumnach, aby wiersze w sąsiednich łamach były położone na jednakowej wysokości}
\Clue{17}{}{Monotremata - rząd prymitywnych ssaków charakteryzujących się jajorodnością, występujące na obszarze krainy australijskiej, a ściślej na kontynencie australijskim, na Nowej Gwinei i Tasmanii; nazwa łacińska, Monotremata (jednootworowce), nawiązuje do faktu, że ich układ pokarmowy, wydalniczy i rozrodczy mają wspólne ujście w steku (cloaca)}
\Clue{18}{}{element w maszynach i urządzenia, który służy do chwytania i przenoszenia różnych materiałów}
\Clue{20}{}{państwo, w którym rządzi młodsza linia dynastyczna monarchów innego kraju}
\Clue{23}{}{konstruktor amerykański (1895-1983); ażurowe konstrukcje kopułowe}
\Clue{24}{}{francuski ród królewski panujący we Francji od 987 do 1848 roku, z krótką przerwą w okresie rewolucji francuskiej i I Cesarstwa}
\Clue{25}{}{wszyscy generałowie w armii danego kraju; także złożony z nich korpus}
\Clue{26}{}{u motyli: element samczych narządów genitalnych}
\Clue{31}{}{oddziały Milicji Obywatelskiej powołane w drugiej połowie 1956 roku (po Poznańskim Czerwcu) do zaprowadzania porządku w sytuacjach wyjątkowych, naruszających porządek publiczny, a także do udzielania pomocy ludności w czasie klęsk żywiołowych i ochrony imprez masowych}
\Clue{32}{}{Ursus arctos horribilis - podgatunek niedźwiedzia brunatnego należącego do rodziny niedźwiedziowatych}
\Clue{33}{}{model RC wykonany z włókna węglowego}
\Clue{34}{}{niewielkie pomieszczenie do przechowywania żywności}
\Clue{39}{}{rozbójnik, rabuś}
\Clue{40}{}{o splątanych włosach lub sierści}
\Clue{42}{}{powierzchnia wypukła, miejsce wypukłe}
\Clue{43}{}{2\% roztwór kwasu salicylowego w alkoholu etylowym i wodzie (2\% kwasu salicylowego, 30\% wody, 68\% etanolu 760g/l); spirytus używany do dezynfekcji}
\Clue{44}{}{dawna pieśń angielska lub irlandzka}\end{PuzzleClues}

\begin{PuzzleClues}{\textbf{Pionowe}\\}\Clue{1}{}{pisarz żydowski (1880-1957), trylogia „Przed potopem”}
\Clue{2}{}{część Oficjum kościelnego odprawiana dawniej o godzinie dziewiątej}
\Clue{3}{}{SKROFULARIA trująca bylina lub półkrzew półkuli północnej, w Polsce w lasach i nad brzegami wód}
\Clue{4}{}{instrument drewniany - instrument dęty, w którym wibratorem jest drewniany stroik bądź krawędź, o którą rozpraszany jest strumień powietrza; sam instrument może być wykonany z dowolnego materiału, np. drewna, metalu, lub tworzywa sztucznego}
\Clue{5}{}{telefon; urządzenie umożliwiające prowadzenie rozmów na odległość; dołączane do zakończenia łącza telefonicznego}
\Clue{6}{}{solowy fragment wokalny z akompaniamentem w operze, oratorium, kantacie}
\Clue{8}{}{niemiecki astronom i matematyk (1571-1630) propagator idei Kopernika, odkrył ruchy planet}
\Clue{9}{}{wielość, mnóstwo niedające się policzyć}
\Clue{10}{}{kanapka amerykańska, podłużna bułka pszenna z cienkimi plastrami wołowiny i serem}
\Clue{11}{}{samogłoska, której artykulacja wymaga ruchu języka w przód jamy ustnej - i, y, e, ę}
\Clue{12}{}{twórczość Beethovena, zbiór jego kompozycji i nut (rzadko mówi się tak o jednym utworze)}
\Clue{14}{}{germanista duński (1787-1832); reprezentant metody historyczno-porównawczej w językoznawstwie}
\Clue{19}{}{w starożytnym Rzymie plac dookoła świątyni lub gmachu publicznego}
\Clue{21}{}{urządzenie służące do konwersji energii promieniowania słonecznego na ciepło}
\Clue{22}{}{wietnamskie święta powitania Nowego Roku}
\Clue{24}{}{Myliobatis californica - gatunek morskiej ryby chrzęstnoszkieletowej z rodziny orleniowatych (Myliobatidae); orleń kalifornijski zamieszkuje skaliste dna wzdłuż brzegów wschodniego Oceanu Spokojnego, między wybrzeżem Oregonu a Zatoką Kalifornijską. }
\Clue{27}{}{Macaca thibetana - gatunek małpy wąskonosej z rodziny makakowatych; zamieszkuje okolice górskie środkowych i wschodnich Chin}
\Clue{28}{}{organizacja powstała i działająca w ramach jakiegoś programu, jej celem jest zdobycie i utrzymanie władzy}
\Clue{29}{}{SROKOSZ}
\Clue{30}{}{cecha kogoś, kto pracuje na więcej niż jeden etat}
\Clue{32}{}{organiczny związek chemiczny, pochodna glukozy powstała przez utlenienie jej grupy aldehydowej}
\Clue{34}{}{gra zespołowa, obecnie jest jedną z najpopularniejszych dyscyplin sportowych na świecie}
\Clue{35}{}{język elfów w literackim legendarium, wykreowanym przez J. R. R. Tolkiena}
\Clue{36}{}{część sieci rybackiej, drewniany krążek ułatwiający przesuwanie się jej po dnie}
\Clue{37}{}{zawartość foremki do ciasta}
\Clue{38}{}{wiosłowa łódź okrętowa}
\Clue{41}{}{jednostka informacji w systemie dwójkowym, oznaczająca 2\textasciicircum80 bajtów}\end{PuzzleClues}\newpage\section*{Krzyżówka 140}

\noindent\begin{Puzzle}{24}{31}|*	|*	|[1][S]\drarr	|h	|e	|c	|a	|*	|*	|[2][S]\drarr	|p	|r	|a	|w	|o	|z	|n	|a	|w	|c	|z	|y	|n	|i	|*	|.
|*	|*	|n	|*	|*	|[3][S]\darr	|[4][S]\rarr	|l	|e	|p	|i	|d	|o	|z	|a	|u	|r	|o	|m	|o	|r	|f	|y	|*	|[5][S]\darr	|.
|*	|*	|o	|[6][S]\darr	|[7][S]\rarr	|s	|p	|a	|d	|o	|c	|h	|r	|o	|n	|i	|a	|r	|k	|a	|*	|*	|[8][S]\darr	|*	|d	|.
|*	|*	|r	|b	|*	|*	|[9][S]\rarr	|j	|a	|k	|o	|ś	|ć	|[][,]{ }	|r	|y	|n	|k	|o	|w	|a	|*	|k	|*	|w	|.
|*	|*	|g	|i	|[10][S]\drarr	|t	|a	|t	|a	|r	|z	|y	|n	|*	|[11][S]\rarr	|k	|u	|m	|o	|s	|t	|w	|o	|*	|a	|.
|*	|*	|a	|o	|f	|*	|*	|*	|[12][S]\drarr	|z	|e	|s	|k	|r	|o	|b	|i	|n	|a	|*	|*	|*	|ń	|*	|[][,]{ }	|.
|*	|*	|r	|g	|*	|*	|*	|*	|f	|y	|*	|*	|*	|*	|*	|*	|*	|*	|*	|[13][S]\darr	|*	|*	|[][,]{ }	|*	|g	|.
|*	|[14][S]\drarr	|d	|e	|p	|e	|s	|z	|o	|w	|i	|e	|c	|*	|[15][S]\rarr	|a	|s	|y	|s	|t	|a	|*	|s	|*	|r	|.
|*	|k	|*	|n	|[16][S]\rarr	|g	|l	|i	|n	|k	|a	|*	|*	|[17][S]\drarr	|t	|o	|p	|*	|[18][S]\darr	|o	|*	|*	|z	|*	|z	|.
|*	|a	|*	|*	|*	|*	|[19][S]\darr	|[20][S]\drarr	|t	|a	|i	|f	|*	|n	|[21][S]\darr	|[22][S]\drarr	|s	|l	|u	|p	|*	|[23][S]\darr	|l	|*	|y	|.
|[24][S]\drarr	|b	|i	|s	|k	|u	|p	|k	|a	|*	|*	|[25][S]\darr	|[26][S]\darr	|o	|d	|l	|*	|*	|b	|i	|*	|s	|a	|*	|b	|.
|z	|a	|[27][S]\drarr	|a	|s	|t	|r	|o	|n	|*	|*	|t	|v	|c	|i	|i	|*	|*	|e	|*	|*	|t	|c	|*	|y	|.
|a	|n	|s	|*	|*	|*	|o	|c	|a	|*	|*	|ł	|a	|n	|a	|t	|*	|*	|r	|*	|*	|o	|h	|*	|[][,]{ }	|.
|p	|o	|e	|*	|[28][S]\darr	|*	|p	|i	|*	|[29][S]\rarr	|k	|o	|l	|i	|b	|e	|r	|*	|a	|[30][S]\darr	|[31][S]\darr	|p	|e	|*	|w	|.
|y	|s	|r	|*	|a	|*	|o	|a	|*	|[32][S]\darr	|*	|*	|p	|c	|e	|r	|[33][S]\darr	|*	|b	|b	|s	|a	|t	|*	|[][,]{ }	|.
|l	|*	|y	|*	|d	|[34][S]\darr	|r	|r	|*	|d	|*	|[35][S]\darr	|a	|z	|l	|a	|p	|*	|a	|u	|ł	|[][,]{ }	|n	|[36][S]\darr	|b	|.
|e	|*	|j	|*	|d	|b	|c	|z	|[37][S]\darr	|e	|[38][S]\drarr	|p	|r	|e	|s	|t	|i	|ż	|*	|r	|u	|l	|y	|h	|a	|.
|n	|*	|n	|*	|i	|i	|j	|*	|s	|d	|u	|i	|a	|k	|k	|u	|e	|*	|*	|m	|ż	|o	|[][,]{ }	|a	|r	|.
|i	|*	|y	|[39][S]\drarr	|s	|z	|a	|b	|e	|l	|s	|k	|i	|*	|o	|r	|s	|[40][S]\darr	|[41][S]\darr	|i	|b	|m	|p	|ń	|s	|.
|e	|*	|[][,]{ }	|k	|o	|o	|*	|*	|r	|a	|t	|o	|s	|*	|ś	|a	|[][,]{ }	|e	|b	|s	|a	|b	|ó	|b	|z	|.
|[][,]{ }	|*	|m	|u	|n	|n	|*	|*	|[][,]{ }	|j	|n	|l	|o	|*	|ć	|[][,]{ }	|g	|m	|r	|t	|[][,]{ }	|a	|ł	|i	|c	|.
|k	|*	|o	|d	|*	|*	|*	|*	|t	|n	|y	|o	|*	|*	|*	|d	|o	|i	|a	|r	|z	|r	|k	|c	|z	|.
|r	|*	|r	|o	|*	|*	|*	|*	|r	|*	|*	|*	|*	|*	|*	|w	|ń	|t	|m	|z	|d	|d	|r	|i	|u	|.
|z	|*	|d	|w	|[42][S]\drarr	|g	|n	|i	|a	|z	|d	|o	|[][,]{ }	|p	|r	|o	|c	|e	|s	|o	|r	|o	|w	|e	|*	|.
|y	|*	|e	|i	|m	|*	|*	|[43][S]\rarr	|p	|i	|e	|r	|ó	|g	|*	|r	|z	|r	|e	|s	|o	|w	|i	|l	|*	|.
|ż	|*	|r	|a	|i	|[44][S]\drarr	|c	|h	|i	|n	|o	|n	|*	|*	|*	|s	|y	|*	|l	|t	|w	|a	|*	|*	|*	|.
|o	|*	|c	|n	|e	|k	|*	|[45][S]\rarr	|s	|z	|l	|e	|j	|a	|*	|k	|*	|*	|*	|w	|i	|*	|*	|*	|*	|.
|w	|*	|a	|k	|c	|r	|*	|[46][S]\rarr	|t	|y	|c	|*	|*	|*	|*	|a	|*	|*	|*	|o	|a	|*	|*	|*	|*	|.
|e	|*	|*	|a	|h	|y	|*	|*	|ó	|[47][S]\rarr	|c	|o	|z	|i	|a	|*	|*	|*	|*	|*	|*	|*	|*	|*	|*	|.
|*	|*	|*	|*	|*	|z	|*	|[48][S]\rarr	|w	|i	|l	|k	|[][,]{ }	|p	|s	|z	|c	|z	|e	|l	|i	|*	|*	|*	|*	|.
|[49][S]\rarr	|i	|n	|t	|r	|a	|d	|a	|*	|*	|*	|*	|*	|*	|*	|*	|*	|*	|*	|*	|*	|*	|*	|*	|*	|.
|*	|*	|*	|*	|*	|*	|*	|*	|*	|*	|*	|*	|*	|*	|*	|*	|*	|*	|*	|*	|*	|*	|*	|*	|*	|.\end{Puzzle}

\newpage

\begin{PuzzleClues}{\textbf{Poziome}\\}\Clue{1}{}{ciekawe, nagłe wydarzenie, niezwykła historia, która może mieć interesujące dalsze konsekwencje, np. może być początkiem skandalu}
\Clue{2}{}{kobieta o wykształceniu prawniczym}
\Clue{4}{}{Lepidosauromorpha - infragromada gadów z podgromady Diapsida; znane są w zapisie kopalnym od permu}
\Clue{7}{}{sportsmenka zajmująca się spadochroniarstwem zawodowo lub hobbystycznie}
\Clue{9}{}{zbiór własności produktu dający mu przewagę na rynku; do własności handlowych produktu należą: ekskluzywność, estetyczność, prezentacja (marketing), koszt nabycia, satysfakcja klientów}
\Clue{10}{}{przedstawiciel grupy ludów tureckich z Europy wschodniej oraz północnej Azji}
\Clue{11}{}{kumowie: kum z kumą}
\Clue{12}{}{fragment czegoś, co zostało zeskrobane}
\Clue{14}{}{fan Depeche Mode}
\Clue{15}{}{w sporcie: podanie do zawodnika z tej samej drużyny, po którym zostają zdobyte punkty lub jest zdobyta bramka}
\Clue{16}{}{ilasta skała osadowa}
\Clue{17}{}{wolny, górny koniec pionowych drzewc w omasztowaniu żaglowca, czyli masztów}
\Clue{20}{}{miasto w zachodniej części Arabii Saudyjskiej, ośrodek wypoczynkowy}
\Clue{22}{}{jednomasztowy żaglowiec z dwoma żaglami; grotem i fokiem}
\Clue{24}{}{kobieta o święceniach biskupich, sprawująca funkcje biskupa w niektórych kościołach, np. metodystycznych, starokatolickich, niektórych anglikańskich, luterańskich, kalwińskich, a także w Kościele Adwentystów Dnia Siódmego}
\Clue{27}{}{pyrotron zaopatrzony w dodatkową warstwę elektronów}
\Clue{29}{}{najmniejszy ptak świata; poszczególne gatunki tego ptaka w taksonomii biologicznej klasyfikowane są w obrębie rzędu jerzykowych (Apodiformes), w rodzinie kolibrów (Trochilidae)}
\Clue{38}{}{pochlebna opinia na temat kogoś lub czegoś cieszącego się uznaniem}
\Clue{39}{}{kompozytor, organista i pedagog (1896-1979); profesor katowickiej PWSM; utwory orkiestrowe, kameralne, organowe, wokalne, koncerty}
\Clue{42}{}{rodzaj złącza znajdującego się na płycie głównej; pełniące rolę interfejsu pomiędzy procesorem a pozostałymi elementami systemu komputerowego, umożliwiając jego współpracę z systemem za pośrednictwem odpowiednich magistrali i układów znajdujących się na płycie głównej}
\Clue{43}{}{jeden mały lub średniej wielkości wyrób gastronomiczny, sakiewka z ciasta z jakimś nadzieniem, podawana zwykle w ilości kilku sztuk na porcję}
\Clue{44}{}{organiczny związek chemiczny będący cyklicznym, nienasyconym diketonem}
\Clue{45}{}{tania i prosta uprząż złożona z parcianych pasków}
\Clue{46}{}{historyk, docent Uniwersytetu Poznańskiego (1896-1927); czynny w akcji plebiscytowej na Śląsku}
\Clue{47}{}{rumuńska miejscowość z cennymi zabytkami sakralnymi, cerkwie XIV-XVIII w}
\Clue{48}{}{taszczyn pszczeli, Philanthus triangulum - owad zaliczany do rodziny grzebaczowatych z rzędu błonkówek; wilk pszczeli ma ubarwienie czarne w żółte trójkątne plamki i paski; zamieszkuje piaszczyste lasy, żywi się pszczołami miodnymi, którym odbiera zebrany nektar, paraliżując je przy użyciu jadu, a następnie zaciąga je do wygrzebanej w ziemi norki}
\Clue{49}{}{krótki utwór instrumentalny rozpoczynający niektóre suity, opery, balet}\end{PuzzleClues}

\begin{PuzzleClues}{\textbf{Pionowe}\\}\Clue{1}{}{ur.  w 1932 r., kompozytor duński; utwory orkiestrowe, kameralne, opery, balety}
\Clue{2}{}{choroba skóry; niejednolity zespół chorobowy, w którym wykwitem pierwotnym jest bąbel pokrzywkowy}
\Clue{3}{}{w chemii: symbol siarki}
\Clue{5}{}{bogactwo, urodzaj, więcej, niż ktoś się spodziewał}
\Clue{6}{}{najczęściej w liczbie mnogiej: substancja, która powstała jako element uboczny funkcjonowania organizmu; to, co wytwarzył żywy organizm}
\Clue{8}{}{populacja koni hodowanych w Polsce do jeździeckiego sportu wyczynowego, pochodząca od klaczy wielkopolskich, małopolskich i klaczy pochodzenia zagranicznego kojarzonych z ogierami wpisanymi do ksiąg stadnych zrzeszonych w Światowej Federacji Hodowli Koni Sportowych; jedne z najpopularniejszych koni w Polsce, użytkowane także w sporcie amatorskim i rekreacyjnym}
\Clue{10}{}{częsty symbol franka (waluty)}
\Clue{12}{}{Józef (1670-1741), budowniczy i architekt, pałace: Bielińskich, Paców, Zamoyskich}
\Clue{13}{}{gatunek antylopy}
\Clue{14}{}{przysmak dla psów i kotów w kształcie przypominającym cienką kiełbasę - kabanosa, zawierający oprócz mięsa (którego zazwyczaj jest niewiele) różne dodatki odżywcze i wypełniające, a także najczęściej substancje wabiące}
\Clue{17}{}{zawartość nocniczka, pojemnika, który służy do oddawania moczu i kału}
\Clue{18}{}{miasto we wsch. Brazylii (Minas Gerais); ośrodek handlowy regionu rolniczego}
\Clue{19}{}{równość dwóch stosunków postaci}
\Clue{20}{}{osoba kochająca koty, hobbysta}
\Clue{21}{}{cecha czegoś, co jest bardzo silne}
\Clue{22}{}{literatura o tematyce dworskiej, charakterystyczna dla literatury barokowej}
\Clue{23}{}{stopa procentowa określająca cenę, po której bank centralny udziela bankom komercyjnym pożyczek pod zastaw papierów wartościowych}
\Clue{24}{}{zapylenie kwiatu pyłkiem pochodzącym od innej rośliny tego samego gatunku}
\Clue{25}{}{coś mało ważnego lub ktoś mało ważny}
\Clue{26}{}{region w Chile u podnóża Andów}
\Clue{27}{}{określenie opisujące osobę, która dopuściła się trzech lub więcej morderstw w oddzielnych epizodach}
\Clue{28}{}{angielski poeta i prozaik (1672-1719), redaktor „Spectatora”}
\Clue{30}{}{urząd burmistrza}
\Clue{31}{}{zespół instytucji dostarczających usług opieki zdrowotnej}
\Clue{32}{}{ostateczny termin wykonania czegoś}
\Clue{33}{}{pies wykorzystywany głównie do tropienia i pogoni za zwierzyną, posługujący się przede wszystkim węchem}
\Clue{34}{}{północnoamerykański. przeżuwacz spokrewniony z żubrem; pod ochroną}
\Clue{35}{}{instrument muzyczny, rodzaj małego fletu o wysokiej skali, ostrym przenikliwym dźwięku - PIKULINA, PICCOLO}
\Clue{36}{}{osoba, która hańbi}
\Clue{37}{}{półtwardy ser podpuszczkowy, który ma kształt krążków}
\Clue{38}{}{egzamin ustny, przebiegający w formie rozmowy}
\Clue{39}{}{mieszkanka Kudowy Zdroju}
\Clue{40}{}{elektroda tranzystora emitująca nośniki mniejszościowe}
\Clue{41}{}{BRAMŻAGIEL; prostokątny lub trapezowy żagiel podnoszony na bramrei}
\Clue{42}{}{składające się ścianki aparatu lub obiektywu, najczęściej w starych aparatach fotograficznych}
\Clue{44}{}{rodzaj zwężki, cienka tarcza, dysk z otworem w środku wmontowywany w przewód w taki sposób, że oś otworu pokrywa się z osią przewodu, stosowana do dławienia przepływu w przewodzie}\end{PuzzleClues}\newpage\section*{Krzyżówka 141}

\noindent\begin{Puzzle}{19}{28}|*	|[1][S]\drarr	|w	|o	|d	|n	|o	|s	|a	|m	|o	|l	|o	|t	|*	|*	|*	|*	|*	|*	|.
|*	|d	|[2][S]\darr	|*	|*	|[3][S]\drarr	|w	|n	|i	|k	|l	|i	|w	|o	|ś	|ć	|*	|*	|*	|*	|.
|*	|i	|p	|[4][S]\darr	|*	|d	|[5][S]\drarr	|p	|i	|e	|r	|w	|i	|a	|s	|t	|e	|k	|*	|[6][S]\darr	|.
|*	|t	|a	|f	|[7][S]\drarr	|r	|e	|g	|e	|s	|t	|r	|*	|*	|*	|*	|*	|*	|*	|w	|.
|[8][S]\drarr	|h	|r	|e	|c	|z	|k	|a	|*	|*	|*	|*	|*	|*	|[9][S]\darr	|*	|*	|[10][S]\darr	|*	|ą	|.
|d	|e	|e	|l	|y	|e	|s	|[11][S]\rarr	|m	|i	|e	|s	|z	|e	|k	|*	|*	|l	|*	|ż	|.
|o	|r	|o	|d	|k	|w	|p	|*	|*	|[12][S]\rarr	|c	|y	|n	|k	|o	|w	|i	|e	|c	|*	|.
|c	|i	|*	|f	|l	|o	|o	|*	|*	|[13][S]\drarr	|s	|t	|e	|n	|m	|a	|r	|k	|*	|*	|.
|h	|n	|*	|e	|*	|[][,]{ }	|r	|*	|*	|w	|[14][S]\rarr	|p	|a	|p	|u	|a	|*	|[][,]{ }	|*	|*	|.
|ó	|g	|*	|b	|[15][S]\darr	|k	|t	|*	|*	|r	|*	|*	|*	|*	|n	|*	|*	|p	|*	|*	|.
|d	|*	|*	|e	|s	|o	|a	|[16][S]\drarr	|s	|z	|l	|a	|m	|*	|i	|*	|*	|r	|[17][S]\darr	|*	|.
|[][,]{ }	|*	|*	|l	|o	|s	|c	|s	|[18][S]\darr	|e	|*	|[19][S]\darr	|*	|[20][S]\darr	|k	|*	|*	|z	|a	|*	|.
|g	|*	|*	|*	|n	|m	|j	|z	|c	|ś	|*	|p	|[21][S]\darr	|e	|a	|[22][S]\darr	|*	|e	|n	|*	|.
|w	|*	|*	|[23][S]\rarr	|d	|i	|a	|k	|o	|n	|*	|i	|p	|p	|t	|k	|[24][S]\darr	|c	|g	|*	|.
|a	|*	|*	|*	|a	|c	|*	|o	|p	|i	|*	|l	|i	|i	|o	|e	|u	|i	|l	|*	|.
|r	|*	|*	|*	|*	|z	|*	|t	|y	|a	|*	|o	|e	|z	|r	|n	|n	|w	|e	|*	|.
|a	|[25][S]\rarr	|o	|p	|o	|n	|a	|*	|p	|*	|*	|t	|r	|o	|*	|d	|i	|z	|z	|*	|.
|n	|*	|[26][S]\drarr	|o	|f	|e	|r	|t	|a	|[][,]{ }	|h	|a	|n	|d	|l	|o	|w	|a	|*	|*	|.
|t	|[27][S]\rarr	|a	|s	|*	|*	|*	|*	|s	|*	|*	|ż	|a	|z	|*	|k	|e	|k	|[28][S]\darr	|*	|.
|o	|*	|b	|[29][S]\drarr	|c	|h	|e	|s	|t	|e	|r	|*	|c	|i	|[30][S]\darr	|a	|r	|r	|d	|*	|.
|w	|*	|r	|p	|[31][S]\rarr	|ł	|o	|p	|a	|t	|e	|c	|z	|k	|a	|*	|e	|z	|o	|*	|.
|a	|*	|o	|ó	|[32][S]\drarr	|s	|o	|n	|*	|*	|*	|*	|*	|*	|p	|*	|k	|e	|g	|*	|.
|n	|[33][S]\drarr	|g	|ł	|a	|d	|ź	|*	|*	|*	|*	|[34][S]\rarr	|m	|o	|s	|t	|*	|p	|r	|*	|.
|y	|ż	|a	|o	|z	|*	|*	|*	|[35][S]\rarr	|m	|a	|j	|o	|n	|e	|z	|*	|o	|y	|*	|.
|*	|o	|c	|k	|o	|*	|[36][S]\rarr	|s	|z	|p	|e	|c	|i	|e	|l	|*	|*	|w	|w	|*	|.
|*	|ł	|j	|r	|w	|*	|[37][S]\rarr	|m	|a	|k	|a	|b	|r	|a	|*	|[38][S]\rarr	|r	|y	|k	|*	|.
|*	|n	|a	|e	|*	|*	|[39][S]\rarr	|a	|k	|o	|m	|o	|d	|a	|c	|j	|a	|*	|a	|*	|.
|*	|a	|*	|s	|*	|*	|*	|*	|*	|*	|*	|*	|*	|*	|*	|*	|*	|*	|*	|*	|.
|*	|*	|*	|*	|*	|*	|*	|*	|*	|*	|*	|*	|*	|*	|*	|*	|*	|*	|*	|*	|.\end{Puzzle}

\newpage

\begin{PuzzleClues}{\textbf{Poziome}\\}\Clue{1}{}{HYDROPLAN; WODNOPŁAT}
\Clue{3}{}{cecha człowieka - to, że ktoś chce coś poznać, bada to głęboko}
\Clue{5}{}{metaforycznie o ważnym elemencie czegoś, składniku; np. pierwiastek uczucia}
\Clue{7}{}{rejestr - spis, wykaz czegoś}
\Clue{8}{}{gryka - roślina uprawna, zbożowa}
\Clue{11}{}{element konstrukcyjny w różnych urządzeniach, mający budowę zbliżoną do miecha}
\Clue{12}{}{pierwiastek należący do dwunastej grupy układu okresowego pierwiastków}
\Clue{13}{}{alpejczyk szwedzki, mistrz olimpijski w slalomie i slalomie gigancie z Lake Placid, trzykrotny mistrz świata}
\Clue{14}{}{południowo-wschodnia część Nowej Gwinei, od 1975 r. w składzie państwa Papua Nowa Gwinea}
\Clue{16}{}{osad na dnie zbiornika wodnego lub pozostały po wyschnięciu wody, będący mieszaniną pyłu i łu z dodatkiem substancji organicznych}
\Clue{23}{}{we wczesnym chrześcijaństwie pomocnik biskupa}
\Clue{25}{}{zewnętrzna część koła pojazdu o przekroju otwartym, nakładana na felgę lub obręcz i wypełniana powietrzem (lub innym gazem) pod ciśnieniem}
\Clue{26}{}{propozycja zawarcia transakcji handlowej}
\Clue{27}{}{figura karciana}
\Clue{29}{}{miasto w Anglii, port nad rzeką Dee}
\Clue{31}{}{zdrobniale o łopatce, małej łopacie - prostym narzędziu służącym do podnoszenia i przenoszenia materiałów sypkich (również o zabawce)}
\Clue{32}{}{liniowa jednostka głośności dźwięku - 1 son odpowiada głośności tonu o częstotliwości 1000 Hz i natężeniu 40 dB}
\Clue{33}{}{gładko obrobiona powierzchnia część maszyny lub urządzenia}
\Clue{34}{}{w przenośni: to, co służy porozumieniu między ludźmi}
\Clue{35}{}{porcja majonezu, zimnego, emulsyjnego sosu na bazie oliwy z dodatkiem surowego żółtka; określona ilość majonezu, zazwyczaj słoik lub plastikowa butelka}
\Clue{36}{}{roztocz z nadrodziny roślinożernych szpecieli; szkodnik drzew owocowych}
\Clue{37}{}{coś strasznego}
\Clue{38}{}{intensywny płacz}
\Clue{39}{}{zmiana czegoś polegająca na dostosowaniu się tego lub dostosowaniu (przez kogoś) tego czegoś do nowych warunków zewnętrznych}\end{PuzzleClues}

\begin{PuzzleClues}{\textbf{Pionowe}\\}\Clue{1}{}{proces polegający na dodaniu do odtwarzanego strumienia dźwiękowego niskopoziomowego szumu (neutralnego dla ludzkich uszu) w zamian za redukcję zakłóceń}
\Clue{2}{}{element garderoby damskiej, część stroju plażowego}
\Clue{3}{}{w wielu mitologiach: ogromne drzewo, które rośnie w środku świata, jest podporą niebios i gwarantem stabilności Wszechświata}
\Clue{4}{}{niemiecki, podoficerski stopień wojskowy odpowiednik polskiego sierżanta (wachmistrza)}
\Clue{5}{}{wyprowadzenie zwłok na miejsce, gdzie pozostają do pogrzebu}
\Clue{6}{}{giętka rura lub rurka, często wykonana z tworzywa sztucznego, przez którą przepływa gaz albo płyn}
\Clue{7}{}{szereg procesów, zdarzeń, które tworzą całość i się powtarzają}
\Clue{8}{}{ustalone minimum finansowe, które zapewnia państwo swoim obywatelom}
\Clue{9}{}{program komputerowy pozwalający na przesyłanie natychmiastowych komunikatów pomiędzy dwoma lub większą liczbą komputerów poprzez sieć komputerową, zazwyczaj Internet}
\Clue{10}{}{antykoagulant - lek spowalniający, utrudniający lub uniemożliwiający krzepnięcie krwi; stosowany w profilaktyce zakrzepic różnego pochodzenia oraz w leczeniu istniejących zakrzepów w żyłach o powolnym przepływie krwi}
\Clue{13}{}{Myricaria L. - rodzaj krzewów i krzewinek z rodziny tamaryszkowatych}
\Clue{15}{}{badanie opinii publicznej, rodzaj badań o charakterze socjologicznym, mających na celu określenie np. preferencji danej grupy ludności}
\Clue{16}{}{mieszkaniec Szkocji, człowiek pochodzenia szkockiego}
\Clue{17}{}{koń rasy angielskiej}
\Clue{18}{}{krótka historyjka, która jest rozpowszechniana w internecie za pomocą kopiowania i wklejania głównie w mediach społecznościowych}
\Clue{19}{}{umiejętność prowadzenia statków latających}
\Clue{20}{}{niewielki i niezbyt istotny fragment fabuły}
\Clue{21}{}{rodzaj buławy zakończonej piórami, używana jako oznaka władzy}
\Clue{22}{}{zawodnik uprawiający kendo}
\Clue{24}{}{żartobliwie o uniwersytecie - instytucji zajmującej się kształceniem}
\Clue{26}{}{całkowite zniesienie normy prawnej bez zastępowania jej nową}
\Clue{28}{}{dodatkowa tura podczas wyborów do władz}
\Clue{29}{}{czas potrzebny na przebieg połowy cyklu zjawiska fizycznego}
\Clue{30}{}{żagiel skośny dolny trójkątny podnoszony w płaszczyźnie symetrii statku między grotmasztem i bezanmasztem}
\Clue{32}{}{miasto w Federacji Rosyjskiej, port w pobliżu ujścia Donu do Morza Azowskiego,80 tys. mieszkańców (1986)}
\Clue{33}{}{owadożerny ptak z rzędu kraskowatych o barwnym upierzeniu, w Polsce chroniona}\end{PuzzleClues}\newpage\section*{Krzyżówka 142}

\noindent\begin{Puzzle}{16}{33}|*	|[1][S]\darr	|*	|*	|*	|*	|*	|*	|*	|*	|*	|*	|*	|*	|*	|*	|*	|.
|*	|d	|[2][S]\drarr	|p	|i	|r	|y	|m	|i	|d	|y	|n	|a	|*	|*	|*	|*	|.
|*	|e	|w	|*	|*	|*	|*	|*	|*	|*	|*	|[3][S]\drarr	|g	|i	|p	|s	|*	|.
|*	|p	|a	|*	|*	|*	|*	|*	|*	|*	|[4][S]\rarr	|t	|e	|m	|p	|o	|*	|.
|*	|r	|t	|*	|[5][S]\rarr	|s	|z	|c	|z	|e	|d	|r	|i	|n	|*	|*	|*	|.
|*	|e	|a	|[6][S]\rarr	|k	|u	|c	|[][,]{ }	|e	|x	|m	|o	|o	|r	|*	|*	|*	|.
|[7][S]\drarr	|c	|h	|r	|z	|e	|s	|t	|[][,]{ }	|b	|o	|j	|o	|w	|y	|*	|*	|.
|c	|h	|a	|*	|*	|*	|*	|[8][S]\drarr	|t	|u	|m	|a	|n	|*	|*	|*	|*	|.
|z	|a	|*	|*	|[9][S]\rarr	|p	|y	|ł	|*	|[10][S]\darr	|*	|k	|*	|*	|[11][S]\darr	|*	|*	|.
|t	|*	|*	|[12][S]\rarr	|c	|e	|b	|u	|l	|a	|k	|*	|*	|*	|s	|*	|*	|.
|e	|[13][S]\drarr	|l	|i	|*	|[14][S]\darr	|[15][S]\rarr	|k	|o	|d	|z	|i	|k	|*	|a	|*	|*	|.
|r	|p	|[16][S]\rarr	|m	|o	|r	|a	|*	|*	|ż	|*	|*	|*	|*	|m	|[17][S]\darr	|*	|.
|o	|i	|[18][S]\rarr	|b	|i	|e	|ł	|o	|w	|a	|*	|*	|*	|*	|o	|j	|*	|.
|t	|ę	|*	|*	|*	|t	|*	|*	|[19][S]\rarr	|n	|o	|g	|a	|*	|c	|o	|*	|.
|a	|c	|[20][S]\rarr	|g	|r	|o	|t	|*	|*	|t	|*	|*	|*	|*	|h	|j	|*	|.
|k	|i	|*	|[21][S]\rarr	|g	|r	|u	|c	|h	|a	|c	|z	|*	|*	|ó	|o	|*	|.
|t	|o	|*	|*	|*	|y	|*	|[22][S]\darr	|[23][S]\darr	|*	|[24][S]\darr	|[25][S]\darr	|[26][S]\darr	|*	|d	|b	|*	|.
|*	|b	|[27][S]\darr	|*	|*	|c	|[28][S]\darr	|p	|p	|[29][S]\darr	|k	|s	|d	|[30][S]\darr	|[][,]{ }	|a	|*	|.
|*	|ó	|ś	|[31][S]\rarr	|s	|z	|k	|a	|r	|ł	|u	|p	|n	|i	|e	|*	|*	|.
|*	|j	|w	|*	|*	|n	|o	|s	|e	|o	|r	|ł	|o	|g	|l	|*	|*	|.
|*	|[][,]{ }	|[][S].	|*	|*	|o	|l	|[][,]{ }	|r	|s	|o	|a	|*	|i	|e	|*	|*	|.
|*	|n	|[][,]{ }	|*	|[32][S]\darr	|ś	|c	|r	|a	|i	|p	|t	|*	|e	|k	|*	|*	|.
|*	|o	|a	|[33][S]\darr	|e	|ć	|z	|a	|f	|ę	|a	|a	|[34][S]\darr	|ł	|t	|*	|*	|.
|[35][S]\drarr	|w	|u	|h	|u	|*	|a	|t	|a	|*	|t	|[][,]{ }	|d	|e	|r	|*	|*	|.
|r	|o	|g	|i	|r	|*	|t	|u	|e	|*	|w	|b	|u	|c	|y	|*	|*	|.
|z	|c	|u	|s	|o	|*	|k	|n	|l	|*	|a	|a	|m	|z	|c	|*	|*	|.
|a	|z	|s	|t	|*	|*	|a	|k	|i	|*	|*	|l	|b	|k	|z	|*	|*	|.
|d	|e	|t	|o	|*	|*	|*	|o	|t	|*	|*	|o	|a	|a	|n	|*	|*	|.
|k	|s	|y	|r	|*	|[36][S]\darr	|*	|w	|y	|*	|*	|n	|d	|*	|y	|*	|*	|.
|o	|n	|n	|y	|*	|r	|*	|y	|z	|*	|*	|o	|z	|*	|*	|*	|*	|.
|ś	|y	|*	|k	|*	|y	|*	|*	|m	|*	|[37][S]\rarr	|w	|e	|k	|*	|*	|*	|.
|ć	|*	|*	|*	|*	|f	|*	|*	|*	|*	|*	|a	|*	|*	|*	|*	|*	|.
|*	|*	|[38][S]\rarr	|o	|s	|t	|r	|o	|k	|ó	|ł	|*	|*	|*	|*	|*	|*	|.
|*	|*	|*	|*	|*	|*	|*	|*	|*	|*	|*	|*	|*	|*	|*	|*	|*	|.\end{Puzzle}

\newpage

\begin{PuzzleClues}{\textbf{Poziome}\\}\Clue{2}{}{organiczny związek chemiczny z grupy heterocyklicznych związków aromatycznych, strukturalnie zbliżony do pirydyny}
\Clue{3}{}{odlew gipsowy, rzeźba odlana z gipsu}
\Clue{4}{}{rytm, według którego wykonuje się pewne ruchy lub czynności}
\Clue{5}{}{kompozytor radziecki ur. w 1932 r.; balety, utwory symfoniczne, kameralne, fortepianowe, opery; 'Martwe dusze'}
\Clue{6}{}{rasa konia domowego pochodząca z Wielkiej Brytanii, z rejonu Exmoor, w Somerset, dawniej bardzo liczna, szczególnie w hrabstwie Devon, gdzie do dziś prowadzi na wpół dziki żywot}
\Clue{7}{}{wykonywanie czegoś po raz pierwszy w określonej sytuacji}
\Clue{8}{}{chmura czegoś (raczej czegoś sypkiego, lekkiego niż gazu) wzbita w powietrze}
\Clue{9}{}{proch, bardzo drobne zanieczyszczenie}
\Clue{12}{}{bułka lub placek z ułożoną na wierzchu i zapieczoną cebulą}
\Clue{13}{}{chińska jednostka odległości, li odpowiada 500 metrom}
\Clue{15}{}{kod pocztowy; ciąg cyfr (rzadziej liter i cyfr) dodawany do adresu, mający ułatwiać sortowanie przesyłek}
\Clue{16}{}{jednostka iloczasu używana w fonologii, określającawagę (\textasciitilde czas trwania) sylaby, która decyduje o synchronizacji czasowej i akcentowaniu w niektórych językach}
\Clue{18}{}{florecistka radziecka, mistrzyni i wicemistrzyni olimpijska Meksyku, Monachium, Montrealu i Moskwy}
\Clue{19}{}{inna nazwa piłki nożnej - gry zespołowej, dziedziny sportu}
\Clue{20}{}{szpicak; rodzaj dłuta kamieniarskiego}
\Clue{21}{}{przedstawiciel rodziny gruchaczy pochodzących z tropikalnych lasów Ameryki Południowej}
\Clue{31}{}{Echinodermata - typ halobiontycznych, bezkręgowych zwierząt wtóroustych (Deuterostomia) o wtórnej symetrii pięciopromiennej; zwierzęta te charakteryzują się wapiennym szkieletem wewnętrznym oraz obecnością unikalnego wśród zwierząt układu ambulakralnego pełniącego funkcję lokomocyjną, dotykową, a częściowo wydalniczą i oddechową}
\Clue{35}{}{miasto w Chinach (Anhui) port nad Jangcy, ważny węzeł komunikacyjny}
\Clue{37}{}{słoik, który służy do przechowywania przetworów}
\Clue{38}{}{PALISADA, CZĘSTOKÓŁ}\end{PuzzleClues}

\begin{PuzzleClues}{\textbf{Pionowe}\\}\Clue{1}{}{chandra, gorszy nastrój, chwilowy kiepski humor, smutek, przygnębienie}
\Clue{2}{}{dawniej: oddział zbrojny}
\Clue{3}{}{ludowy taniec śląski}
\Clue{7}{}{pojazd z silnikiem czteroaktowym}
\Clue{8}{}{OSTROŁUK}
\Clue{10}{}{miejscowość w Indiach słynna z zespołu 30 świątyń buddyjskich wykutych w skale (II w p.n.e.-VIII w. n.e.)}
\Clue{11}{}{samochód napędzany silnikiem elektrycznym}
\Clue{13}{}{dyscyplina olimpijska łącząca różnorodne dyscypliny z elementami treningu przekrojowego}
\Clue{14}{}{to, że wypowiedź jest ukształtowana zgodnie z zasadami retoryki}
\Clue{17}{}{ciekły wosk, pozyskiwany z nasion krzewu simondsii kalifornijskiej, w wyniku ich wyciskania}
\Clue{22}{}{kamizelka ratunkowa o dużej wyporności, w jaskrawym kolorze, o sztywnym kołnierzu zapewniającym utrzymanie głowy ponad wodą nawet w przypadku utraty przytomności}
\Clue{23}{}{nurt w sztuce II połowy XIX wieku nawiązujący do twórczości malarzy wczesnego renesansu włoskiego}
\Clue{24}{}{śródpolowy ptak z rzędu kuraków, pożyteczny (tępi stonkę ziemniaczaną), w Polsce pospolita, łowna}
\Clue{25}{}{ostatnia rata do spłaty w okresie kredytowania metodą płatności balonowej, będąca równowartością zaciągniętego zadłużenia}
\Clue{26}{}{tandeta, kicz, lipa (nie tylko o wytworze materialnym, ale i o sytuacji)}
\Clue{27}{}{twórczość Augustyna z Hippony, zbiór jego myśli i poglądów}
\Clue{28}{}{brona kolczasta; brona składająca się z kilku sekcji doczepianych do belki zaczepowej}
\Clue{29}{}{młode łosia}
\Clue{30}{}{zdrobniale: igiełka - liść o podłużnej, wąskiej, sztywnej, ostrej blaszce}
\Clue{32}{}{moneta jednoeurowa}
\Clue{33}{}{naukowiec, który zajmuje się prowadzeniem badań nad przeszłością i przekazywaniem o tym wiedzy}
\Clue{34}{}{lekkoatletka radz., rekordzistka świata z 1954 r}
\Clue{35}{}{coś, co pojawia się lub występuje nieczęsto}
\Clue{36}{}{typ rowu tektonicznego o rozciągłości setek lub nawet tysięcy kilometrów, ograniczony równoległymi do siebie uskokami normalnymi}\end{PuzzleClues}\newpage\section*{Krzyżówka 143}

\noindent\begin{Puzzle}{19}{25}|*	|*	|*	|*	|[1][S]\drarr	|d	|e	|m	|*	|*	|[2][S]\darr	|*	|*	|[3][S]\drarr	|t	|r	|z	|o	|n	|*	|.
|*	|*	|*	|*	|b	|[4][S]\rarr	|p	|o	|r	|u	|s	|z	|e	|n	|i	|e	|*	|[5][S]\darr	|*	|*	|.
|*	|*	|*	|*	|e	|[6][S]\darr	|[7][S]\darr	|*	|[8][S]\rarr	|m	|u	|ł	|ł	|a	|*	|*	|*	|m	|*	|[9][S]\darr	|.
|*	|*	|*	|*	|r	|w	|b	|[10][S]\darr	|*	|[11][S]\drarr	|b	|a	|s	|s	|*	|*	|*	|e	|[12][S]\darr	|d	|.
|*	|*	|*	|*	|l	|a	|u	|b	|[13][S]\rarr	|p	|l	|a	|j	|t	|a	|*	|*	|s	|t	|s	|.
|*	|*	|*	|*	|i	|r	|r	|l	|*	|o	|i	|*	|[14][S]\drarr	|u	|k	|l	|e	|j	|a	|*	|.
|*	|*	|*	|*	|ń	|s	|g	|o	|*	|w	|t	|*	|f	|l	|*	|[15][S]\rarr	|p	|a	|n	|*	|.
|*	|*	|*	|[16][S]\drarr	|c	|z	|e	|c	|h	|ł	|o	|*	|*	|a	|*	|*	|*	|s	|i	|*	|.
|*	|*	|*	|o	|z	|a	|r	|h	|*	|o	|r	|[17][S]\darr	|*	|*	|[18][S]\darr	|[19][S]\rarr	|c	|z	|u	|*	|.
|*	|*	|*	|r	|y	|w	|*	|*	|[20][S]\rarr	|k	|a	|r	|t	|o	|n	|*	|*	|*	|s	|[21][S]\darr	|.
|*	|*	|*	|d	|k	|a	|*	|[22][S]\drarr	|p	|a	|l	|e	|c	|[][,]{ }	|b	|o	|ż	|y	|*	|l	|.
|*	|*	|*	|y	|*	|*	|*	|k	|[23][S]\darr	|[][,]{ }	|*	|t	|*	|*	|*	|[24][S]\darr	|*	|*	|*	|i	|.
|*	|*	|*	|n	|*	|*	|*	|a	|b	|k	|*	|r	|[25][S]\drarr	|g	|o	|ł	|ą	|b	|*	|s	|.
|*	|*	|[26][S]\rarr	|a	|s	|*	|*	|m	|a	|o	|*	|a	|t	|*	|*	|ą	|*	|*	|*	|i	|.
|*	|*	|*	|r	|[27][S]\rarr	|o	|m	|a	|t	|n	|i	|k	|o	|w	|a	|t	|e	|*	|*	|c	|.
|*	|*	|*	|i	|*	|[28][S]\darr	|[29][S]\darr	|i	|o	|w	|*	|c	|n	|*	|*	|k	|*	|*	|*	|z	|.
|*	|*	|*	|u	|[30][S]\darr	|c	|s	|s	|n	|e	|[31][S]\drarr	|j	|a	|g	|u	|a	|r	|*	|*	|k	|.
|*	|*	|*	|s	|p	|a	|a	|h	|i	|r	|k	|a	|ż	|*	|*	|*	|*	|*	|[32][S]\darr	|a	|.
|*	|*	|*	|z	|r	|n	|b	|i	|*	|s	|o	|*	|*	|[33][S]\rarr	|s	|z	|w	|y	|c	|*	|.
|*	|*	|*	|*	|y	|t	|a	|*	|*	|y	|z	|*	|*	|[34][S]\drarr	|h	|e	|v	|i	|z	|*	|.
|*	|*	|*	|[35][S]\rarr	|m	|o	|ł	|n	|i	|j	|a	|*	|*	|u	|[36][S]\rarr	|c	|y	|n	|a	|*	|.
|*	|*	|*	|*	|a	|*	|a	|*	|*	|n	|c	|*	|*	|g	|*	|[37][S]\rarr	|p	|a	|s	|*	|.
|*	|*	|*	|*	|*	|*	|*	|[38][S]\rarr	|l	|a	|t	|o	|p	|i	|s	|i	|e	|c	|*	|*	|.
|[39][S]\rarr	|l	|e	|s	|k	|i	|n	|e	|n	|*	|w	|*	|[40][S]\rarr	|e	|l	|u	|r	|a	|*	|*	|.
|[41][S]\rarr	|ł	|a	|ń	|c	|u	|c	|h	|[][,]{ }	|p	|o	|k	|a	|r	|m	|o	|w	|y	|*	|*	|.
|[42][S]\rarr	|k	|a	|t	|h	|a	|r	|s	|i	|s	|*	|*	|*	|*	|*	|*	|*	|*	|*	|*	|.\end{Puzzle}

\newpage

\begin{PuzzleClues}{\textbf{Poziome}\\}\Clue{1}{}{w biologii: populacja lokalna, populacja genetyczna, populacja geograficzna - grupa osobników jednego gatunku zasiedlająca jednolity obszar}
\Clue{3}{}{(pieca) spodnia część pieca, na której spoczywa wsad}
\Clue{4}{}{ruszenie czegoś, np. jakąś częścią ciała}
\Clue{8}{}{w szyizmie: nauczyciel, interpretator praw religijnych i doktryn islamu}
\Clue{11}{}{ryba z rzędu okoniokształtnych}
\Clue{13}{}{bankructwo, upadłość, ruina, brak pieniędzy}
\Clue{14}{}{srebrzysta ryba z rodziny karpiowatych}
\Clue{15}{}{forma grzecznościowa używana przy zwracaniu się do mężczyzny, z którym się nie jest na ty}
\Clue{16}{}{koszula kobieca noszona w XVI/XVIII w}
\Clue{19}{}{drewniany instrument perkusyjny; kołatka, korytkowa}
\Clue{20}{}{gatunek sztywnego, grubego papieru do rysowania}
\Clue{22}{}{znak, który upatruje się jako ostrzeżenie}
\Clue{25}{}{przesyłka wysłana przy pomocy gołębia pocztowego}
\Clue{26}{}{w muzyce: dźwięk A obniżony o półton}
\Clue{27}{}{Theridiidae - rodzina pająków z podrzędu Opisthothela, obejmująca ponad 2350 gatunków podzielonych na ok. 120 rodzajów; w Polsce występuje ponad 60 gatunków, głównie z rodzajów: Achaearanea, Dipoena, Steatoda, Theridion}
\Clue{31}{}{Panthera onca - gatunek ssaka z rodziny kotowatych, zaliczany do wielkich kotów}
\Clue{33}{}{bydło rasy szwajcarskiej}
\Clue{34}{}{słynne uzdrowisko w zach. Węgrzech (komitat Zala) w pobliżu jeziora Balaton; gorące źródła mineralne}
\Clue{35}{}{seria radzieckich satelitów łącznościowych}
\Clue{36}{}{szpan, coś, czym można zaimponować (przycynić)}
\Clue{37}{}{zagranie karciane polegające na chwilowej rezygnacji z dalszego udziału w jakiejś części rozgrywki}
\Clue{38}{}{kronika (zazwyczaj ruska)}
\Clue{39}{}{strzelec fiński, zdobywca 38 medali mistrzostw świata w latach 1930-37}
\Clue{40}{}{wieś w Indiach; jeden z najcenniejszych zabytków sztuki sakralnej z V-XI w}
\Clue{41}{}{szereg organizmów ustawionych w takiej kolejności, że każda poprzedzająca grupa (ogniwo) jest podstawą pożywienia następnej}
\Clue{42}{}{zdefiniowana przez Arystotelesa jedna z podstawowych cech tragedii, zwłaszcza antycznej, polagająca na wzbudzeniu u widza uczuć litości i trwogi, aby przez to następnie oczyścić jego umysł z tych doznań}\end{PuzzleClues}

\begin{PuzzleClues}{\textbf{Pionowe}\\}\Clue{1}{}{mieszkaniec Berlina}
\Clue{2}{}{strefa dna zbiornika wodnego granicząca z litoralem, poniżej granicy występowania roślinności; to strefa tuż poniżej litoralu, często w miejscu gdzie zaczyna się bardziej (niż w litoralu) gwałtowny spadek dna}
\Clue{3}{}{dżudoka, złoty medalista olimpijski z Atlanty, dwukrotny mistrz świata}
\Clue{5}{}{u Żydów; król, wybawiciel Izraela przepowiadany przez proroków; w chrześcijaństwie jeden z tytułów Jezusa}
\Clue{6}{}{stolica i największe miasto Polski}
\Clue{7}{}{poeta niemiecki (1747-94), twórca niemieckiej ballady literackiej; „Lenora”, „Przygody Munchaussena”}
\Clue{9}{}{w chemii: symbol pierwiastka darmsztadt}
\Clue{10}{}{biochemik amerykański ur. w 1912 r.; badania nad metabolizmem aminokwasów i lipidów, nagroda Nobla}
\Clue{11}{}{powłoka ochronna metalu, w skład której wchodzą cząsteczki ochranianego metalu}
\Clue{12}{}{Tanius - rodzaj dinozaura z rodziny hadrozaurów (Hadrosauridae); żył w okresie późnej kredy (89-65 mln lat temu) na terenach wschodniej Azji; długość ciała 10 m, ciężar 3 ton}
\Clue{14}{}{symbol jednostki pojemności elektrycznej w układzie SI}
\Clue{16}{}{w Kościele katolickim biskup zarządzający diecezją}
\Clue{17}{}{proces ściągania skrzepu, polegający na wzmocnieniu jego struktury i zwężeniu ścianek uszkodzonego naczynia krwionośnego}
\Clue{18}{}{w chemii: symbol niobu}
\Clue{21}{}{zdrobniale: lisica - samica lisa}
\Clue{22}{}{miasto w Japonii (płn. Honsiu) port nad Oceanem Spokojnym; w pobliżu eksploatacja rud żelaza}
\Clue{23}{}{malarz włoski (708-87) obrazy religijne, mityczne, portrety}
\Clue{24}{}{ważka z rodziny łątkowatych}
\Clue{25}{}{wyrażona w tonach masa, jaką urządzenie holownicze lub takie, które służy do przenoszenia ładunków, może podnieść lub pociągnąć}
\Clue{28}{}{śpiew: melodia, głos wykonujący melodię}
\Clue{29}{}{polski góral, honorowy przewodnik tatrzański, muzykant, myśliwy, gawędziarz i pieśniarz}
\Clue{30}{}{pierwszy dźwięk skali diatonicznej, podstawowy dźwięk trójdźwięku}
\Clue{31}{}{rodzaj osadnictwa wojskowego na zasiedlanych pogranicznych ziemiach Rzeczypospolitej i Rosji}
\Clue{32}{}{chwila, moment, pora}
\Clue{34}{}{kolor pomiędzy żółtym a złocistobrunatnym}\end{PuzzleClues}\newpage\section*{Krzyżówka 144}

\noindent\begin{Puzzle}{21}{29}|*	|[1][S]\drarr	|p	|r	|a	|c	|a	|[][,]{ }	|p	|i	|s	|e	|m	|n	|a	|*	|[2][S]\drarr	|d	|u	|m	|a	|*	|.
|*	|k	|*	|*	|[3][S]\drarr	|s	|t	|a	|d	|i	|u	|m	|[][,]{ }	|l	|a	|r	|w	|a	|l	|n	|e	|*	|.
|*	|o	|[4][S]\darr	|[5][S]\darr	|k	|*	|*	|*	|*	|[6][S]\rarr	|k	|s	|e	|n	|o	|b	|i	|o	|t	|y	|k	|*	|.
|*	|l	|p	|d	|u	|*	|*	|*	|*	|*	|*	|*	|[7][S]\drarr	|s	|i	|e	|w	|c	|a	|*	|[8][S]\darr	|[9][S]\darr	|.
|*	|o	|y	|ż	|o	|*	|*	|*	|*	|*	|[10][S]\rarr	|s	|p	|a	|r	|k	|i	|n	|g	|*	|ł	|ś	|.
|[11][S]\drarr	|r	|z	|e	|k	|o	|t	|k	|a	|[][,]{ }	|s	|a	|r	|d	|y	|ń	|s	|k	|a	|*	|o	|r	|.
|k	|a	|a	|r	|a	|[12][S]\darr	|[13][S]\drarr	|z	|w	|o	|l	|n	|i	|e	|n	|i	|e	|*	|*	|[14][S]\darr	|j	|o	|.
|o	|t	|t	|e	|*	|t	|m	|*	|*	|[15][S]\rarr	|z	|w	|o	|r	|n	|i	|k	|*	|*	|p	|ó	|d	|.
|i	|k	|o	|ń	|*	|u	|a	|*	|*	|*	|*	|*	|r	|[16][S]\drarr	|r	|u	|c	|h	|*	|r	|w	|e	|.
|m	|a	|ś	|*	|[17][S]\darr	|l	|s	|[18][S]\darr	|*	|*	|*	|*	|y	|g	|*	|*	|j	|*	|*	|o	|k	|k	|.
|e	|*	|ć	|*	|m	|e	|ł	|m	|[19][S]\rarr	|k	|w	|o	|t	|a	|[][,]{ }	|b	|a	|z	|o	|w	|a	|*	|.
|k	|*	|*	|[20][S]\darr	|ł	|j	|o	|e	|[21][S]\rarr	|h	|y	|c	|e	|l	|*	|*	|*	|*	|*	|i	|*	|*	|.
|[][,]{ }	|*	|*	|k	|o	|k	|*	|t	|[22][S]\rarr	|t	|a	|r	|t	|i	|n	|k	|a	|*	|*	|n	|[23][S]\darr	|*	|.
|b	|[24][S]\drarr	|l	|a	|d	|a	|c	|z	|n	|i	|c	|a	|*	|l	|*	|*	|*	|*	|*	|c	|p	|*	|.
|e	|h	|*	|p	|y	|*	|*	|*	|*	|*	|*	|*	|[25][S]\rarr	|e	|w	|o	|k	|a	|c	|j	|a	|*	|.
|z	|i	|*	|i	|[][,]{ }	|[26][S]\rarr	|o	|d	|m	|i	|e	|ń	|c	|o	|w	|a	|t	|e	|*	|u	|c	|*	|.
|ł	|a	|*	|t	|g	|*	|*	|*	|[27][S]\drarr	|g	|b	|p	|*	|*	|[28][S]\darr	|*	|*	|*	|*	|s	|i	|*	|.
|o	|l	|*	|a	|n	|[29][S]\rarr	|h	|a	|b	|e	|c	|k	|*	|*	|w	|*	|*	|*	|*	|z	|o	|*	|.
|d	|o	|*	|ł	|i	|*	|*	|[30][S]\darr	|e	|[31][S]\rarr	|z	|a	|ś	|w	|i	|a	|t	|y	|*	|k	|r	|[32][S]\darr	|.
|y	|f	|*	|[][,]{ }	|e	|[33][S]\drarr	|n	|a	|r	|z	|ą	|d	|[][,]{ }	|ł	|z	|o	|w	|y	|*	|a	|n	|s	|.
|g	|a	|*	|f	|w	|r	|*	|p	|n	|[34][S]\rarr	|f	|a	|k	|c	|j	|a	|*	|*	|[35][S]\darr	|*	|i	|e	|.
|o	|g	|*	|i	|n	|y	|*	|e	|e	|[36][S]\rarr	|m	|ł	|o	|k	|o	|s	|*	|*	|t	|*	|c	|c	|.
|w	|i	|*	|n	|y	|s	|*	|l	|ń	|[37][S]\rarr	|k	|a	|t	|a	|n	|g	|a	|*	|e	|*	|e	|e	|.
|y	|a	|*	|a	|*	|*	|*	|*	|c	|[38][S]\rarr	|n	|u	|r	|c	|e	|*	|*	|*	|l	|*	|*	|s	|.
|*	|*	|*	|n	|*	|[39][S]\rarr	|r	|e	|z	|o	|n	|a	|t	|o	|r	|*	|*	|*	|e	|*	|*	|j	|.
|[40][S]\rarr	|d	|y	|s	|k	|u	|r	|s	|y	|w	|n	|o	|ś	|ć	|*	|*	|*	|*	|f	|*	|*	|a	|.
|*	|*	|[41][S]\rarr	|o	|l	|e	|j	|e	|k	|[][,]{ }	|b	|e	|r	|g	|a	|m	|o	|t	|o	|w	|y	|*	|.
|[42][S]\rarr	|ż	|y	|w	|i	|c	|a	|*	|*	|[43][S]\rarr	|s	|z	|k	|a	|r	|ł	|a	|t	|n	|i	|k	|*	|.
|*	|[44][S]\rarr	|w	|y	|k	|l	|i	|n	|a	|[][,]{ }	|a	|l	|p	|e	|j	|s	|k	|a	|*	|*	|*	|*	|.
|*	|*	|*	|*	|*	|*	|*	|*	|*	|*	|*	|*	|*	|*	|*	|*	|*	|*	|*	|*	|*	|*	|.\end{Puzzle}

\newpage

\begin{PuzzleClues}{\textbf{Poziome}\\}\Clue{1}{}{praca szkolna w formie pisemnej}
\Clue{2}{}{wysokie mniemanie o sobie}
\Clue{3}{}{stadium rozwojowe organizmu, w którym organizm młodociany wzrasta w formie anatomicznie, fizjologicznie i ekologicznie różniącej się od dojrzałej}
\Clue{6}{}{związek chemiczny występujący w organizmie, który ani go nie produkuje, ani też w normalnych warunkach nie przyjmuje z pożywieniem}
\Clue{7}{}{ten, kto głosi jakieś poglądy}
\Clue{10}{}{rodzaj błyszczącej, jedwabnej tkaniny połączonej ze sztucznym włóknem}
\Clue{11}{}{Hyla sarda - gatunek płaza bezogonowego z rodziny rzekotkowatych, występujący na wyspach Morza Śródziemnego: Korsyce, Elbie, Sardynii i mniejszych: Cavallo, Maddalena, Caprera, San Pietro}
\Clue{13}{}{decyzja o zwolnieniu, wydaleniu kogoś}
\Clue{15}{}{malarz grecki z Heraklei (V-IV w. p.n.e.) twórca pierwszego obrazu idyllicznego 'Rodzina centaurów'}
\Clue{16}{}{zespół czynności, działań, akcji podejmowanych w jakimś celu (taki ruch to nigdy nie jest grupa ludzi, zawsze tylko czynności)}
\Clue{19}{}{wskaźnik, wedle którego wylicza się wysokość emerytury, obliczany na podstawie wynagrodzenia}
\Clue{21}{}{urwis, niezły numer, huncwot, hultaj, ziółko, gagatek}
\Clue{22}{}{cienki kawałek chleba lub bułki posmarowany masłem z dodatkiem wędliny, sera; kanapka}
\Clue{24}{}{kobieta rozpustna, niemoralna, grzesznica}
\Clue{25}{}{dawniej: odwołanie się do innej (wyższej) instytucji lub wezwanie do sądu innego niż właściwy dla danej sprawy czy pozwanego}
\Clue{26}{}{Proteidae - rodzina płazów z rzędu płazów ogoniastych, której przedstawiciele występują w południowo-wschodniej Europie (rodzaj Proteus) i we wschodniej Ameryce Północnej (rodzaj Necturus)}
\Clue{27}{}{kod ISO 4217 funta szterlinga}
\Clue{29}{}{pisarz austriacki ur. 1916r, powieści krytyczne wobec faszyzmu, sztuki sceniczne}
\Clue{31}{}{w wierzeniach religijnych świat (lub światy) istniejący poza przestrzenią dostępną bezpośredniej percepcji wyznaczony najczęściej poprzez konkretny model kosmosu dla różnych ludów lub obszarów świata}
\Clue{33}{}{część aparatu ochronnego oka, produkująca ciecz łzową}
\Clue{34}{}{w socjologii: wyróżniona kategoria w obrębie klasy społecznej.}
\Clue{36}{}{młody człowiek, chłopak; słowo używane z lekceważeniem}
\Clue{37}{}{SHABA}
\Clue{38}{}{Pelecanoididae - monotypowa rodzina ptaków z rzędu rurkonosych; występują na wodach morskich półkuli południowej}
\Clue{39}{}{układ rezonansowy, który bierze udział w wytwarzaniu dźwięków przez istoty żywe}
\Clue{40}{}{cecha tekstu, bądącego pośrednikiem w przekazie informacji,  interpretowanego poprzez jego użycie}
\Clue{41}{}{olejek eteryczny pozyskiwany ze skórek owoców pomarańczy bergamota}
\Clue{42}{}{substancja wytwarzana w niektórych roślinach, najczęściej w drzewach, szczególnie iglastych}
\Clue{43}{}{PURPUROWIEC}
\Clue{44}{}{Poa alpina - gatunek rośliny należący do rodziny wiechlinowatych}\end{PuzzleClues}

\begin{PuzzleClues}{\textbf{Pionowe}\\}\Clue{1}{}{biały kołnierzyk noszony przez duchownych, występuje w dwóch postaciach: krótkiej (białego listka wsuwanego do stójki w koszuli) i dłuższej (odrębnego białego kołnierzyka z doczepionym pektorałem)}
\Clue{2}{}{zabieg operacyjny wykonywany na żywym zwierzęciu w celach badawczych}
\Clue{3}{}{Setonix brachyurus - gatunek ssaka z rodziny kangurowatych, jedyny przedstawiciel rodzaju Setonix; występuje w południowo-zachodniej Australii oraz na wyspach Rottnest i Bald, gdzie jest liczniejszy niż na kontynencie}
\Clue{4}{}{cecha wyglądu człowieka, który ma okrągłą, pucołowatą buzię}
\Clue{5}{}{gazela mongolska, Procapra gutturosa - gatunek ssaka parzystokopytnego z rodziny krętorogich; zamieszkuje wschodnią Mongolię i przylegające do niej tereny Rosji oraz północnych Chin}
\Clue{7}{}{wielka waga, pierwszeństwo pomiędzy innymi sprawami, rzeczami}
\Clue{8}{}{część mięsa wołowego przerośniętego łojem; słabizna wołowa}
\Clue{9}{}{miejsce (mniej więcej) jednakowo oddalone od brzegów obiektu}
\Clue{11}{}{Diphyscium foliosum - mech z rodziny koimkowatych; w Polsce nie jest objęty ochroną, występuje m.in. na terenie Bieszczadzkiego Parku Narodowego, Pienińskiego Parku Narodowego, Ślężańskiego Parku Krajobrazowego. }
\Clue{12}{}{krótka rura, element niektórych urządzeń i maszyn}
\Clue{13}{}{nazwa niektórych tłuszczów roślinnych lub innych tłustych przetworów roślinnych (np. pulpy zawierającej naturalne tłuszcze)}
\Clue{14}{}{mieszkanka prowincji}
\Clue{16}{}{wystrzelony z Atlantisa celem zbadania planety Wenus}
\Clue{17}{}{młody, zbuntowany człowiek, który jasno wyraża swoje poglądy}
\Clue{18}{}{miasto we Francji (Lotaryngia) port nad Mozelą ważny ośrodek handlowy, ośrodek prowincji Moselle}
\Clue{20}{}{kapitał, który powstaje z połączenia kapitału bankowego  i przemysłowego}
\Clue{23}{}{drobne muchówki z rodziny pryszczarków, szkodniki roślin}
\Clue{24}{}{zjadanie szkła}
\Clue{27}{}{mieszkaniec Berna}
\Clue{28}{}{twórca, który ma wizję, zwykle taki, który tworzy coś ponadprzeciętnego lub nowatorskiego}
\Clue{30}{}{zorganizowane zebranie się grupy osób o określonej porze, w określonym miejscu, najczęściej poprzedzające coś}
\Clue{32}{}{Rottluff, niemiecki grafik, malarz i rzeźbiarz (1884-1976) reprezentant ekspresjonizmu; kompozycje figuralne i pejzaże}
\Clue{33}{}{stała cecha, wyróżnik}
\Clue{35}{}{numer abonenta, osoby lub instytucji, do której wykonywane jest połączenie}\end{PuzzleClues}\newpage\section*{Krzyżówka 145}

\noindent\begin{Puzzle}{23}{22}|*	|[1][S]\darr	|[2][S]\darr	|[3][S]\drarr	|o	|ś	|l	|i	|c	|z	|k	|a	|[][,]{ }	|w	|o	|d	|n	|a	|*	|*	|*	|*	|*	|*	|.
|*	|g	|k	|n	|*	|*	|*	|*	|*	|*	|*	|*	|*	|*	|*	|*	|*	|*	|*	|*	|*	|*	|*	|[4][S]\darr	|.
|*	|r	|o	|i	|*	|*	|*	|*	|*	|*	|*	|*	|*	|*	|*	|*	|*	|*	|*	|*	|*	|*	|*	|ż	|.
|[5][S]\drarr	|u	|r	|e	|i	|d	|*	|*	|*	|*	|*	|*	|*	|*	|*	|*	|*	|*	|*	|*	|[6][S]\darr	|*	|*	|ó	|.
|w	|d	|z	|o	|*	|*	|*	|*	|*	|*	|[7][S]\rarr	|j	|e	|m	|i	|o	|ł	|u	|s	|z	|k	|a	|*	|ł	|.
|a	|z	|e	|b	|*	|[8][S]\rarr	|m	|o	|w	|a	|[][,]{ }	|p	|o	|g	|r	|z	|e	|b	|o	|w	|a	|*	|*	|w	|.
|g	|i	|ń	|l	|*	|[9][S]\drarr	|t	|a	|u	|*	|[10][S]\drarr	|j	|ę	|z	|y	|k	|[][,]{ }	|h	|a	|u	|s	|a	|*	|[][,]{ }	|.
|o	|e	|[][,]{ }	|i	|*	|o	|[11][S]\rarr	|m	|o	|a	|k	|[][,]{ }	|w	|y	|ż	|y	|n	|n	|y	|*	|z	|*	|*	|o	|.
|n	|ń	|f	|g	|[12][S]\rarr	|s	|k	|o	|r	|z	|o	|n	|e	|r	|a	|*	|*	|*	|*	|*	|k	|[13][S]\darr	|*	|l	|.
|[][,]{ }	|*	|i	|a	|[14][S]\darr	|ł	|[15][S]\rarr	|t	|r	|y	|b	|u	|n	|a	|*	|*	|[16][S]\darr	|*	|*	|*	|i	|s	|*	|i	|.
|b	|*	|o	|t	|g	|o	|*	|[17][S]\rarr	|b	|r	|y	|g	|a	|d	|z	|i	|s	|t	|a	|*	|e	|y	|*	|w	|.
|r	|*	|ł	|o	|r	|n	|*	|*	|*	|*	|ł	|[18][S]\rarr	|s	|m	|u	|s	|z	|k	|a	|*	|t	|l	|*	|k	|.
|e	|[19][S]\rarr	|k	|r	|z	|a	|k	|ó	|w	|k	|a	|[][,]{ }	|b	|a	|g	|i	|e	|n	|n	|a	|*	|i	|*	|o	|.
|k	|*	|o	|y	|e	|*	|[20][S]\rarr	|b	|ą	|k	|*	|[21][S]\rarr	|ś	|m	|i	|e	|s	|z	|k	|a	|*	|k	|[22][S]\darr	|w	|.
|o	|*	|w	|j	|c	|[23][S]\rarr	|h	|e	|p	|t	|a	|s	|t	|e	|o	|r	|n	|i	|s	|*	|*	|a	|d	|y	|.
|w	|*	|y	|n	|z	|*	|*	|*	|[24][S]\drarr	|s	|f	|i	|g	|m	|o	|m	|a	|n	|o	|m	|e	|t	|r	|*	|.
|y	|*	|*	|o	|n	|*	|*	|*	|d	|*	|[25][S]\drarr	|g	|i	|e	|r	|a	|s	|i	|m	|o	|w	|*	|y	|*	|.
|*	|*	|*	|ś	|o	|*	|*	|*	|i	|*	|t	|[26][S]\rarr	|r	|a	|b	|a	|t	|y	|*	|*	|*	|*	|f	|*	|.
|*	|*	|*	|ć	|ś	|[27][S]\rarr	|b	|a	|l	|e	|r	|y	|n	|a	|*	|[28][S]\rarr	|k	|n	|e	|d	|l	|e	|*	|*	|.
|*	|*	|*	|*	|ć	|[29][S]\rarr	|t	|l	|e	|n	|e	|k	|[][,]{ }	|ż	|e	|l	|a	|z	|o	|w	|y	|*	|*	|*	|.
|*	|*	|*	|*	|*	|*	|*	|*	|r	|*	|ś	|*	|*	|*	|[30][S]\rarr	|n	|*	|*	|*	|*	|*	|*	|*	|*	|.
|*	|*	|*	|*	|*	|*	|*	|*	|*	|*	|ć	|*	|*	|*	|*	|*	|*	|*	|*	|*	|*	|*	|*	|*	|.
|*	|*	|*	|*	|*	|*	|*	|*	|*	|*	|*	|*	|*	|*	|*	|*	|*	|*	|*	|*	|*	|*	|*	|*	|.\end{Puzzle}

\newpage

\begin{PuzzleClues}{\textbf{Poziome}\\}\Clue{3}{}{Asellus aquaticus - gatunek skorupiaka z rzędu równonogów (Isopoda); ciało spłaszczone, zbudowane z prawie jednorodnych segmentów; posiada siedem par odnóży, długie, antenowate czułki, samiec długości około 13 mm, samica 8 mm}
\Clue{5}{}{związek organiczny, pochodna mocznika, w której atom wodoru mocznika został zastąpiony grupą organiczną - acylem}
\Clue{7}{}{chroniony ptak leśny z rzędu wróblowatych, na głowie czubek, upierzenie brązowe; Eurazja, Ameryka Płn}
\Clue{8}{}{uroczysta wypowiedź pożegnalna i chwalebna, wygłoszona na pogrzebie pod adresem osoby zmarłej}
\Clue{9}{}{litera alfabetu greckiego}
\Clue{10}{}{jeden z ważniejszych języków Afryki z podrodziny czadyjskiej języków afroazjatyckich, rodzimy dla ludu Hausa}
\Clue{11}{}{Megalapteryx didinus - gatunek wymarłego ptaka nielota z rodziny moaków (Emeidae)}
\Clue{12}{}{wężymord, czarny korzeń, dwuletnia roślina warzywna ze złożonych, jadalny korzeń spichrzowy o ciemnej skórce}
\Clue{15}{}{miejsce dla widzów na stadionie}
\Clue{17}{}{zwierzchnik nad brygadą pracowników}
\Clue{18}{}{futro ze skórek kilkudniowego jagnięcia}
\Clue{19}{}{Gerygone tenebrosa - gatunek ptaka z rodziny buszówkowatych (Acanthizidae)}
\Clue{20}{}{rodzaj owada z rzędu muchówek}
\Clue{21}{}{kobieta, dziewczyna, która bardzo często się śmieje, którą łatwo jest rozśmieszyć}
\Clue{23}{}{Heptasteornis - rodzaj dwunożnego, mięsożernego dinozaura, przypuszczalnie z rodziny alwarezaurów, a dokładniej z podrodziny mononykinów}
\Clue{24}{}{aparat do pośredniego pomiaru ciśnienia tętniczego krwi, składający się z manometru (rtęciowego, sprężynowego lub elektronicznego), pompki tłoczącej powietrze, mankietu z komorą powietrzną i zaworka do kontrolowanego wypuszczania powietrza z mankietu}
\Clue{25}{}{zbiór rzeźb antycznych lub starożytne muzeum sztuki}
\Clue{26}{}{barwne wyłogi przy mundurze wojskowym typowe dla kurtek ułańskich}
\Clue{27}{}{baletnica, tancerka na scenie baletowej}
\Clue{28}{}{potrawa mączna wywodząca się z niemieckiego obszaru kulturowego, popularna także w Polsce}
\Clue{29}{}{nieorganiczny związek chemiczny, składający się z żelaza i  tlenu}
\Clue{30}{}{w chemii: symbol azotu}\end{PuzzleClues}

\begin{PuzzleClues}{\textbf{Pionowe}\\}\Clue{1}{}{dwunasty miesiąc w roku (według kalendarza gregoriańskiego), ma 31 dni}
\Clue{2}{}{kłącze kosaćca}
\Clue{3}{}{fakt, że coś nie jest obligatoryjne, nie jest obowiązkowe}
\Clue{4}{}{Lepidochelys olivacea - najmniejszy gatunek gada z całej rodziny żółwi morskich; jest najliczniej występującym żółwiem na świecie i jednym z najbardziej wykorzystywanych przez ludzi (polowania, handel jajami)}
\Clue{5}{}{wagon, który jest zaopatrzony w hamulec; obsługiwany przez brekowego}
\Clue{6}{}{czapka męska z daszkiem z przodu i okrągłym denkiem lub z podwiniętym daszkiem z przodu i z tyłu}
\Clue{9}{}{lek lub suplement, który ma niwelować negatywne skutki działania np. antybiotyku}
\Clue{10}{}{KLACZ; samica konia}
\Clue{13}{}{sztuczny kamień z piasku i wapna używany w budownictwie}
\Clue{14}{}{przysługa, uprzejmość, zrobienie czegoś dla kogoś}
\Clue{16}{}{coś lub ktoś oznaczone szesnastką, noszące taki numer}
\Clue{22}{}{ZNOS; znoszenie statku z kursu pod wpływem fal i wiatru}
\Clue{24}{}{człowiek, który rozprowadza i sprzedaje narkotyki}
\Clue{25}{}{w biologii, medycynie - substancja gromadząca się w narządach}\end{PuzzleClues}\newpage\section*{Krzyżówka 146}

\noindent\begin{Puzzle}{25}{21}|*	|*	|*	|*	|*	|*	|*	|*	|*	|*	|[1][S]\drarr	|z	|a	|l	|a	|n	|i	|e	|[][,]{ }	|s	|i	|ę	|*	|*	|*	|*	|.
|*	|[2][S]\rarr	|t	|r	|y	|b	|[][,]{ }	|w	|s	|a	|d	|o	|w	|y	|*	|*	|[3][S]\darr	|[4][S]\drarr	|p	|o	|j	|a	|z	|d	|*	|*	|.
|*	|*	|*	|*	|*	|*	|*	|[5][S]\drarr	|d	|r	|u	|ż	|y	|n	|a	|[][,]{ }	|s	|p	|o	|r	|t	|o	|w	|a	|*	|*	|.
|*	|[6][S]\rarr	|l	|i	|z	|e	|s	|k	|a	|*	|c	|*	|*	|*	|*	|*	|z	|l	|*	|[7][S]\darr	|[8][S]\darr	|*	|[9][S]\darr	|[10][S]\darr	|[11][S]\darr	|*	|.
|*	|*	|*	|[12][S]\rarr	|w	|i	|e	|r	|z	|c	|h	|o	|ł	|*	|[13][S]\darr	|*	|p	|a	|*	|k	|c	|*	|m	|w	|p	|*	|.
|*	|*	|*	|*	|*	|[14][S]\drarr	|k	|y	|d	|*	|[][,]{ }	|*	|*	|*	|t	|*	|i	|t	|*	|o	|z	|*	|i	|o	|o	|*	|.
|*	|*	|*	|[15][S]\darr	|*	|s	|*	|t	|*	|*	|c	|*	|*	|[16][S]\darr	|ł	|*	|c	|a	|[17][S]\rarr	|r	|a	|t	|a	|j	|*	|*	|.
|*	|*	|[18][S]\darr	|k	|*	|k	|*	|y	|*	|[19][S]\darr	|z	|[20][S]\darr	|*	|z	|u	|*	|*	|*	|*	|*	|b	|[21][S]\darr	|ł	|n	|*	|*	|.
|*	|[22][S]\darr	|r	|i	|*	|r	|*	|k	|[23][S]\darr	|c	|a	|d	|[24][S]\darr	|r	|m	|[25][S]\rarr	|r	|z	|e	|z	|a	|k	|*	|a	|*	|*	|.
|*	|p	|u	|n	|*	|z	|*	|a	|w	|h	|s	|y	|t	|a	|i	|*	|*	|*	|*	|*	|n	|l	|[26][S]\darr	|[][,]{ }	|*	|*	|.
|*	|i	|r	|o	|*	|y	|[27][S]\drarr	|n	|i	|e	|u	|p	|r	|z	|e	|j	|m	|o	|ś	|ć	|*	|a	|m	|s	|*	|*	|.
|*	|e	|o	|[][,]{ }	|*	|d	|b	|t	|a	|v	|*	|t	|ą	|ó	|n	|*	|*	|*	|*	|*	|*	|r	|a	|t	|*	|*	|.
|*	|p	|w	|a	|*	|e	|a	|*	|d	|i	|*	|y	|b	|w	|n	|[28][S]\drarr	|k	|o	|p	|i	|a	|*	|c	|u	|*	|*	|.
|*	|r	|n	|k	|*	|ł	|k	|*	|r	|o	|[29][S]\rarr	|k	|o	|k	|o	|s	|y	|*	|*	|*	|[30][S]\rarr	|b	|e	|l	|a	|*	|.
|*	|z	|i	|c	|*	|k	|*	|*	|o	|t	|*	|*	|w	|a	|ś	|o	|*	|*	|*	|*	|*	|*	|d	|e	|*	|*	|.
|[31][S]\drarr	|n	|a	|j	|d	|o	|r	|f	|*	|*	|*	|*	|i	|*	|ć	|l	|*	|*	|*	|*	|*	|*	|o	|t	|*	|*	|.
|f	|i	|*	|i	|*	|*	|[32][S]\rarr	|h	|a	|m	|u	|l	|e	|c	|*	|a	|*	|*	|*	|*	|*	|*	|ń	|n	|*	|*	|.
|i	|k	|*	|*	|[33][S]\rarr	|d	|z	|i	|e	|s	|i	|ę	|c	|i	|o	|n	|o	|g	|i	|*	|*	|*	|c	|i	|*	|*	|.
|l	|*	|*	|*	|*	|*	|[34][S]\rarr	|m	|o	|t	|y	|l	|*	|*	|*	|k	|[35][S]\rarr	|p	|r	|z	|ę	|d	|z	|a	|*	|*	|.
|c	|[36][S]\drarr	|d	|ł	|u	|g	|o	|s	|z	|p	|o	|n	|[][,]{ }	|b	|i	|a	|ł	|o	|b	|r	|e	|w	|y	|*	|*	|*	|.
|*	|c	|*	|*	|*	|*	|*	|[37][S]\rarr	|n	|a	|d	|z	|i	|a	|ł	|*	|[38][S]\rarr	|p	|r	|ą	|ż	|e	|k	|*	|*	|*	|.
|*	|*	|*	|*	|*	|*	|*	|*	|*	|*	|*	|*	|*	|*	|*	|*	|*	|*	|*	|*	|*	|*	|*	|*	|*	|*	|.\end{Puzzle}

\newpage

\begin{PuzzleClues}{\textbf{Poziome}\\}\Clue{1}{}{spożycie nadmiaru alkoholu, upicie się}
\Clue{2}{}{wykonywanie serii zadań (programów) przez komputer}
\Clue{4}{}{urządzenie do poruszania się po lądzie, morzu, w powietrzu czy przestrzeni kosmicznej za pomocą własnego napędu}
\Clue{5}{}{zespół składający się z określonej liczby graczy w danej dyscyplinie sportowej}
\Clue{6}{}{lekkie damskie okrycie wierzchnie, zakładane w domu na piżamę, na nocną bieliznę; rodzaj lekkiego szlafroka}
\Clue{12}{}{wierzchołek - czubek czegoś, najwyżej położona część czegoś (drzewa, wzgórza)}
\Clue{14}{}{dramatopisarz angielski (1558-94), uważany za najwybitniejszego twórcę tragedii przed Szekspirem}
\Clue{17}{}{w Polsce średniow.: (wolny) chłop, uprawiający ziemię}
\Clue{25}{}{rodzaj specjalnego ostrza występującego w gospodarczych narzędziach mechanicznych (np. w sieczkarni lub pługu)}
\Clue{27}{}{cecha człowieka: to, że ktoś jest niemiły, zachowuje się niegrzecznie}
\Clue{28}{}{podobizna, najczęściej wydrukowany wizerunek czegoś}
\Clue{29}{}{znaczna suma pieniędzy}
\Clue{30}{}{miasto w płd. Pakistanie}
\Clue{31}{}{szachista polski zamieszkały w Argentynie, uczestnik czternastu olimpiad szachowych}
\Clue{32}{}{urządzenie służące do zatrzymywania lub zwalniania ruchu pojazdów i maszyn}
\Clue{33}{}{Decapoda - rząd skorupiaków wyższych liczący ponad 6000 gatunków, z nadrzędu raków właściwych (Eucarida), u których zwykle pierwsza para odnóży krocznych jest przekształcona w duże szczypce; obejmuje kraby, homary, krewetki, raki oraz kilka mniej znanych rodzin}
\Clue{34}{}{SPINAKER;}
\Clue{35}{}{nitka skręcona z włókien naturalnych lub chemicznych, otrzymywana w wyniku przędzenia}
\Clue{36}{}{Metopidius indicus - gatunek ptaka z rodziny długoszponów (Jacanidae) w rzędzie siewkowych (Charadriiformes)}
\Clue{37}{}{część gruntu otrzymana w wyniku podziału gruntu}
\Clue{38}{}{rodzaj ornamentu o wąskim, długim kształcie}\end{PuzzleClues}

\begin{PuzzleClues}{\textbf{Pionowe}\\}\Clue{1}{}{koncepcje dominujące w danej epoce, w danym okresie}
\Clue{3}{}{pies pokojowy lub stróżnych o długiej, gęstej sierści}
\Clue{4}{}{miasto w Argentynie, port morski, stolica prowincji Buenos Aires}
\Clue{5}{}{człowiek, który krytykuje innych, wytyka błędy, gani}
\Clue{7}{}{polska organizacja opozycyjna działająca od września 1976 do września 1977, sprzeciwiająca się polityce władz PRL, niosąca pomoc osobom represjonowanym w wyniku wydarzeń Czerwca 1976, przede wszystkim w Radomiu i Ursusie}
\Clue{8}{}{pasterz owiec lub wołów w Rumunii, Mołdawii, Węgrzech oraz części Ukrainy, a także u ludów Kaukazu i Azji Środkowej}
\Clue{9}{}{substancja, która jest drobno roztarta, bardzo pokruszona}
\Clue{10}{}{wydarzenie historyczne}
\Clue{11}{}{skrótowa nazwa dawnego przedmiotu realizowanego w szkołach ponadgimnazjalnych i obecnie nauczanego częściowo na studiach, dawniej także w VII i VIII klasach szkół podstawowych; zakres nauczania obejmował szeroko pojętą obronę cywilną, podstawy pierwszej pomocy, metody ochrony przed różnymi zagrożeniami i przygotowanie do postępowania w wypadku katastrof}
\Clue{13}{}{parametr opisujący zdolność danego łącza do realizacji transmisji}
\Clue{14}{}{w botanice: jeden z dwóch bocznych płatków w kwiatach niektórych roślin}
\Clue{15}{}{filmy sensacyjne, których głównym zadaniem jest dostarczanie rozrywki widzom poprzez pokazywanie pościgów samochodowych, strzelanin, bijatyk i innych scen kaskaderskich o dużym ładunku napięcia i emocji}
\Clue{16}{}{część półtuszy wołowej lub wieprzowej z okolic zadu, która wyśmienicie nadaje się na zrazy}
\Clue{18}{}{dział fabryki, który zajmuje sie produkcją rur}
\Clue{19}{}{SZEWIOT}
\Clue{20}{}{niewielki ołtarzyk przenośny wykonany najczęściej z kości słoniowej, zamykany jak książka, dekorowany w środku scenami z Pisma Świętego, w gotyku ozdabiany dodatkowo małymi rzeżbionymi pinaklami czy szczytami (np. maswerkowymi)}
\Clue{21}{}{staranne ułożenie osprzętu pokładowego; porządek na statku}
\Clue{22}{}{Cantharellus - rodzaj grzybów z rodziny pieprznikowatych; cechą charakterystyczną owocników jest łagodne połączenie kapelusza i trzonu - nie ma ostrej granicy między nimi}
\Clue{23}{}{naczynie w kształcie walca lub odwróconego, ściętego stożka, najczęściej blaszane albo z tworzywa sztucznego, zaopatrzone w ruchomy, kabłąkowaty uchwyt}
\Clue{24}{}{przedstawiciel jednego z rzędów ssaków łożyskowych;  obecnie reprezentują go jedynie słonie}
\Clue{26}{}{mieszkaniec Macedonii - państwa, człowiek pochodzenia macedońskiego}
\Clue{27}{}{górny pokład dziobowej części statku lub jego nadbudówki dziobowej}
\Clue{28}{}{roślina zielna lub krzew suchych obszarów strefy umiarkowanej, jest karmą zwierząt}
\Clue{31}{}{wyrób włókienniczy z nieuporządkowanych włókien wełny lub sierści z domieszką wiskozy lub bawełny poddawanych spilśnieniu}
\Clue{36}{}{nazwa literowa pierwszego dźwięku w gamie, także od niej bierze oznaczenie tonacja, której toniką jest c}\end{PuzzleClues}\newpage\section*{Krzyżówka 147}

\noindent\begin{Puzzle}{23}{23}|*	|*	|*	|*	|*	|*	|*	|*	|*	|*	|*	|*	|*	|*	|*	|*	|*	|*	|[1][S]\drarr	|s	|i	|e	|ć	|*	|.
|*	|*	|*	|*	|*	|*	|*	|*	|*	|*	|*	|*	|*	|*	|*	|*	|*	|*	|h	|*	|*	|[2][S]\darr	|*	|*	|.
|*	|*	|[3][S]\drarr	|g	|n	|i	|a	|z	|d	|o	|*	|[4][S]\drarr	|n	|a	|u	|p	|l	|i	|u	|s	|*	|p	|*	|*	|.
|*	|*	|p	|[5][S]\darr	|[6][S]\darr	|*	|*	|*	|*	|*	|*	|b	|*	|[7][S]\rarr	|s	|z	|a	|b	|l	|a	|*	|o	|*	|*	|.
|*	|*	|o	|n	|d	|[8][S]\darr	|*	|*	|*	|*	|*	|u	|*	|[9][S]\rarr	|f	|a	|s	|o	|l	|a	|*	|l	|*	|[10][S]\darr	|.
|*	|*	|l	|i	|w	|k	|[11][S]\rarr	|r	|o	|l	|l	|s	|[][S]-	|r	|o	|y	|c	|e	|*	|*	|*	|s	|*	|r	|.
|*	|*	|i	|e	|u	|r	|[12][S]\drarr	|w	|g	|r	|y	|z	|o	|ń	|*	|*	|[13][S]\rarr	|s	|e	|c	|i	|k	|*	|e	|.
|*	|*	|c	|m	|k	|o	|k	|[14][S]\rarr	|l	|i	|c	|ó	|w	|k	|a	|*	|*	|*	|*	|*	|*	|o	|*	|g	|.
|*	|*	|y	|r	|ó	|t	|o	|[15][S]\rarr	|p	|ł	|y	|w	|i	|k	|*	|*	|*	|*	|*	|*	|*	|ś	|[16][S]\darr	|u	|.
|*	|[17][S]\rarr	|j	|a	|ł	|o	|w	|i	|e	|c	|[][,]{ }	|k	|o	|l	|c	|z	|a	|s	|t	|y	|*	|ć	|b	|ł	|.
|*	|*	|n	|w	|k	|c	|a	|*	|[18][S]\darr	|[19][S]\drarr	|n	|a	|z	|a	|r	|e	|j	|c	|z	|y	|k	|*	|a	|a	|.
|*	|*	|o	|i	|a	|h	|l	|*	|o	|a	|*	|[][,]{ }	|*	|*	|*	|*	|[20][S]\drarr	|m	|o	|l	|n	|a	|r	|*	|.
|*	|*	|ś	|e	|*	|w	|*	|*	|f	|l	|*	|m	|[21][S]\darr	|*	|*	|*	|j	|*	|[22][S]\darr	|*	|*	|*	|i	|*	|.
|*	|[23][S]\darr	|ć	|c	|*	|i	|*	|*	|i	|t	|*	|y	|p	|*	|[24][S]\darr	|[25][S]\darr	|o	|*	|k	|[26][S]\darr	|*	|*	|o	|*	|.
|*	|m	|*	|*	|*	|l	|*	|[27][S]\darr	|a	|*	|*	|s	|l	|*	|t	|s	|l	|*	|a	|h	|*	|*	|n	|*	|.
|*	|u	|*	|*	|*	|a	|[28][S]\darr	|s	|k	|[29][S]\rarr	|p	|i	|e	|p	|r	|z	|*	|[30][S]\darr	|l	|a	|*	|*	|*	|*	|.
|*	|s	|*	|*	|*	|*	|f	|t	|o	|*	|*	|a	|w	|*	|ą	|t	|*	|c	|e	|k	|*	|*	|[31][S]\darr	|*	|.
|*	|z	|*	|*	|[32][S]\drarr	|p	|r	|o	|d	|i	|ż	|*	|k	|[33][S]\rarr	|b	|o	|l	|u	|s	|*	|*	|[34][S]\darr	|s	|*	|.
|[35][S]\rarr	|n	|i	|e	|s	|t	|a	|ł	|o	|ś	|ć	|*	|a	|*	|k	|k	|*	|g	|o	|*	|*	|p	|e	|*	|.
|*	|i	|*	|*	|m	|*	|m	|ó	|n	|*	|*	|*	|*	|*	|a	|f	|*	|*	|n	|*	|*	|ó	|s	|*	|.
|[36][S]\rarr	|k	|o	|ł	|o	|*	|u	|w	|t	|*	|*	|*	|*	|*	|*	|i	|*	|*	|y	|*	|*	|l	|j	|*	|.
|*	|*	|*	|*	|k	|*	|g	|k	|y	|*	|*	|*	|*	|*	|*	|s	|*	|*	|*	|*	|*	|k	|a	|*	|.
|*	|*	|*	|*	|*	|*	|a	|a	|*	|*	|*	|*	|*	|[37][S]\rarr	|s	|z	|t	|u	|t	|o	|w	|o	|*	|*	|.
|[38][S]\rarr	|e	|r	|p	|e	|g	|*	|*	|*	|*	|*	|*	|*	|*	|*	|*	|*	|*	|*	|*	|*	|*	|*	|*	|.\end{Puzzle}

\newpage

\begin{PuzzleClues}{\textbf{Poziome}\\}\Clue{1}{}{przenośnie: pułapka, zasadzka}
\Clue{3}{}{miejsce skupienia czegoś, siedlisko}
\Clue{4}{}{PŁYWIK; larwa wielu szkodników: ma jedno oko i dwie pary odnóży}
\Clue{7}{}{broń szermiercza}
\Clue{9}{}{roślina z motylkowatych hodowana dla jadalnych nasion oraz strąków}
\Clue{11}{}{angielska marka samochodu, produkowanego przez przedsiębiorstwo Rolls-Royce Motor Cars (obecnie własność BMW)}
\Clue{12}{}{chrząszcz z podrodziny kornikowatych}
\Clue{13}{}{zestaw elementów związanych z wykonywaniem jakiejś czynności lub też z określonymi okolicznościami}
\Clue{14}{}{łania, przewodniczka stada w gwarze łowieckiej}
\Clue{15}{}{pierwsze stadium larwalne większości niższych skorupiaków; ma jedno oko i trzy pary odnóży, które odpowiadają pierwszej i drugiej parze czułków oraz żuwaczkom u postaci dorosłej}
\Clue{17}{}{Juniperus oxycedrus - gatunek z rodziny cyprysowatych}
\Clue{19}{}{członek wspólnoty pierwszych chrześcijan (w Dziejach Apostolskich), później, w IV wieku, także wyznawca jednej z sekt chrześcijańskich wywodzących się z Jerozolimy, zachowującej prawo mojżeszowe i żydowskie tradycje obrzędowe, jak na przykład obrzezanie}
\Clue{20}{}{(1878-1852), pisarz węgierski; „Chłopcy z Placu Broni”}
\Clue{29}{}{rodzaj ostrej przyprawy, najczęściej otrzymywanej z owoców rośliny o tej samej nazwie, czasem jednakże będący mieszanką innych pikantnych przypraw (poza pieprzem głównie papryk)}
\Clue{32}{}{metalowe naczynie przystosowane do pieczenia ciast i potraw za pomocą prądu elektrycznego}
\Clue{33}{}{tabletka bardzo dużych rozmiarów, stosowana przede wszystkim w wterynarii}
\Clue{35}{}{cecha osobowości przejawiająca się w braku stałości, chwiejności i zmienności poglądów, upodobań itp.}
\Clue{36}{}{związek, zrzeszenie, stowarzyszenie}
\Clue{37}{}{duża wieś położona w województwie pomorskim, w powiecie nowodworskim, w gminie Sztutowo na obszarze Żuław Wiślanych przy drodze wojewódzkiej nr 501}
\Clue{38}{}{ręczny granatnik przeciwpancery RPG-7}\end{PuzzleClues}

\begin{PuzzleClues}{\textbf{Pionowe}\\}\Clue{1}{}{miasto w Anglii nad estuarium Humber, ważny port handlowy i rybacki}
\Clue{2}{}{zespół cech czegoś lub kogoś takiego jak w Polsce, także: stereotypowe cechy uznawane za właściwe Polakom}
\Clue{3}{}{cecha czegoś, co jest policyjne; dotycząca czegoś związanego z policją jako służbą państwową}
\Clue{4}{}{Acanthiza murina - gatunek ptaka z rodziny buszówkowatych (Acanthizidae) występujący w Australii, Nowej Gwinei i Tasmanii}
\Clue{5}{}{ktoś niemrawy}
\Clue{6}{}{historyczne urządzenie do zrywki drewna}
\Clue{8}{}{wypowiedź anegdotyczna, zabawna, żartobliwa, zachęcająca do śmiechu}
\Clue{10}{}{zbiór norm postępowania ustalonych dla zakonników przez założyciela zakonu i potwierdzonych przez papieża lub biskupa}
\Clue{12}{}{czerwono-czarny pluskwiak różnoskrzydły, wszystkożerny pospolity wczesną wiosną}
\Clue{16}{}{cząstka elementarna należąca do silnie oddziałujących femionów oraz hadronów; charakteryzuje się spinem połówkowym}
\Clue{18}{}{Ophiacodontidae, Ophiacodontia - grupa pelykozaurów obejmująca najwcześniejsze znane synapsydy; pojawiły się we wczesnym karbonie}
\Clue{19}{}{partia utworu pod względem wysokości pomiędzy tenorem a sopranem, śpiewana przez osobę o altowym głosie}
\Clue{20}{}{Żaglowiec posiadający główny maszt w pobliżu środka wyporu jednostki, oraz dodatkowo szczątkowy, bardzo mały maszt wysokości ok. 1/3 głównego, posadowiony daleko na rufie}
\Clue{21}{}{u roślin, głównie u gatunków z rodziny wiechlinowatych, niewielki, łódeczkowatego kształtu, zielony lub bezzieleniowy liść (podsadka) wyrastający u podstawy kłosków}
\Clue{22}{}{część męskiej bielizny, spodenki z wydłużonymi nogawkami noszone pod spodniami}
\Clue{23}{}{męski dodatek do stroju noszony zamiast muszki lub krawatu; ozdoba w kształcie przypominająca krótki krawat, łatwa w noszeniu, bo niewymagająca wiązania}
\Clue{24}{}{Tubaria (W.G. Sm.) Gillet - rodzaj grzybów należący do rodziny strzępiakowatych (Inocybaceae); rosną na próchniejącym drewnie lub na ziemi, na leżących lub zagrzebanych kawałkach próchniejącego drewna}
\Clue{25}{}{ryba z rodziny dorszowatych jako smaczna potrawa}
\Clue{26}{}{metalowy pręt zgięty na końcu, o różnych kształtach i różnorakim zastosowaniu}
\Clue{27}{}{zakład gastronomiczny pełniący rolę stołówki}
\Clue{28}{}{wnęka w ścianie; najczęściej wstawia się w nią okna lub drzwi}
\Clue{30}{}{dwójka koni zaprzęgniętych do cugu}
\Clue{31}{}{dzień pracy giełdy; określony przedział czasu, w którym odbywają się transakcje giełdowe}
\Clue{32}{}{gwiazdozbiór okołobiegunowy nieba północnego}
\Clue{34}{}{małe pole; zdrobniale lub pieszczotliwie o polu}\end{PuzzleClues}\newpage\section*{Krzyżówka 148}

\noindent\begin{Puzzle}{25}{20}|*	|*	|*	|*	|*	|*	|*	|[1][S]\drarr	|k	|o	|n	|w	|e	|r	|s	|j	|a	|[][,]{ }	|g	|e	|n	|ó	|w	|*	|*	|*	|.
|*	|*	|*	|[2][S]\drarr	|p	|ł	|o	|s	|k	|u	|r	|n	|i	|c	|a	|*	|*	|*	|*	|*	|*	|*	|*	|*	|*	|*	|.
|*	|*	|*	|p	|*	|*	|[3][S]\rarr	|k	|o	|c	|i	|a	|[][,]{ }	|m	|u	|z	|y	|k	|a	|*	|*	|*	|[4][S]\darr	|*	|*	|*	|.
|*	|*	|*	|l	|*	|[5][S]\rarr	|k	|o	|m	|u	|t	|a	|t	|o	|r	|*	|*	|[6][S]\rarr	|k	|l	|a	|p	|s	|*	|[7][S]\darr	|*	|.
|*	|*	|*	|a	|[8][S]\drarr	|w	|y	|c	|h	|ó	|d	|*	|*	|*	|*	|*	|*	|*	|[9][S]\drarr	|e	|*	|*	|c	|*	|z	|*	|.
|*	|*	|*	|s	|c	|[10][S]\drarr	|c	|z	|ę	|ś	|ć	|*	|*	|*	|*	|*	|*	|*	|ś	|*	|*	|*	|h	|*	|e	|*	|.
|*	|*	|*	|t	|e	|w	|*	|n	|*	|*	|*	|*	|[11][S]\rarr	|a	|z	|d	|a	|r	|c	|h	|y	|*	|m	|*	|z	|*	|.
|*	|*	|*	|r	|n	|a	|*	|i	|*	|*	|*	|*	|*	|*	|[12][S]\drarr	|d	|ź	|w	|i	|g	|*	|*	|i	|*	|n	|*	|.
|*	|*	|*	|o	|t	|z	|*	|a	|*	|[13][S]\rarr	|ł	|o	|p	|a	|t	|k	|a	|*	|s	|*	|*	|*	|d	|*	|a	|*	|.
|*	|*	|*	|n	|u	|a	|*	|[][,]{ }	|*	|[14][S]\drarr	|g	|e	|n	|i	|u	|s	|z	|e	|k	|*	|*	|*	|t	|[15][S]\darr	|w	|*	|.
|*	|*	|*	|*	|r	|*	|*	|n	|[16][S]\rarr	|p	|a	|b	|i	|a	|n	|i	|c	|z	|a	|n	|i	|n	|*	|r	|a	|*	|.
|*	|*	|*	|[17][S]\darr	|i	|[18][S]\rarr	|p	|o	|d	|l	|e	|ś	|*	|*	|g	|*	|*	|[19][S]\drarr	|c	|y	|c	|e	|r	|o	|n	|*	|.
|*	|*	|[20][S]\rarr	|m	|a	|t	|e	|r	|i	|a	|l	|i	|s	|t	|a	|*	|*	|b	|z	|[21][S]\darr	|[22][S]\darr	|*	|*	|p	|i	|*	|.
|*	|*	|*	|k	|*	|*	|*	|m	|*	|n	|[23][S]\rarr	|m	|a	|k	|r	|e	|l	|a	|*	|w	|k	|[24][S]\darr	|*	|n	|e	|*	|.
|*	|*	|[25][S]\drarr	|l	|a	|l	|k	|a	|r	|s	|t	|w	|o	|*	|*	|*	|*	|b	|*	|ą	|e	|p	|[26][S]\darr	|i	|*	|*	|.
|*	|*	|p	|i	|*	|*	|*	|l	|*	|z	|*	|*	|*	|*	|*	|*	|[27][S]\rarr	|e	|s	|t	|r	|a	|d	|a	|*	|*	|.
|*	|*	|ł	|k	|[28][S]\rarr	|m	|i	|n	|i	|a	|n	|k	|i	|e	|t	|a	|*	|c	|*	|e	|n	|s	|y	|k	|*	|*	|.
|*	|*	|o	|*	|*	|*	|*	|a	|*	|*	|[29][S]\rarr	|m	|ł	|o	|t	|e	|c	|z	|e	|k	|*	|s	|t	|*	|*	|*	|.
|[30][S]\rarr	|s	|z	|a	|j	|n	|a	|*	|[31][S]\rarr	|p	|a	|s	|[][,]{ }	|s	|ł	|u	|c	|k	|i	|*	|*	|u	|k	|*	|*	|*	|.
|*	|[32][S]\rarr	|a	|m	|a	|l	|g	|a	|m	|a	|c	|j	|a	|*	|*	|[33][S]\rarr	|n	|a	|d	|n	|o	|s	|i	|e	|*	|*	|.
|*	|*	|*	|[34][S]\rarr	|p	|o	|d	|e	|j	|r	|z	|l	|i	|w	|i	|e	|c	|*	|*	|*	|*	|*	|*	|*	|*	|*	|.\end{Puzzle}

\newpage

\begin{PuzzleClues}{\textbf{Poziome}\\}\Clue{1}{}{w genetyce zmiana sekwencji w zrekombinowanych cząstkach, wynikająca z naprawy niedopasowania, które mogło powstać podczas rekombinacji}
\Clue{2}{}{Triticum monococcum - gatunek zbóż z rodziny wiechlinowatych}
\Clue{3}{}{hałaśliwa i jazgotliwa muzyka}
\Clue{5}{}{element przełączający obwody elektryczne}
\Clue{6}{}{uderzenie w czyjeś pośladki, zwłaszcza ręką}
\Clue{8}{}{wydatek; suma, która została wydana lub będzie wydana na coś}
\Clue{9}{}{liczba niewymierna, będąca podstawą logarytmu naturalnego; można ją definiować na kilka różnych sposobów}
\Clue{10}{}{to, co wyróżnia się w wyniku podziału większej całości bądź operacji myślowych dokonywanych na większej całości}
\Clue{11}{}{Azhdarchidae - rodzina pterozaurów, żyjąca w późnej kredzie na terenie dzisiejszej Azji, Ameryki Północnej oraz Afryki; zaliczały się do niej jedne z największych latających stworzeń, jakie kiedykolwiek zamieszkiwały naszą planetę}
\Clue{12}{}{dźwignica}
\Clue{13}{}{narzędzie kuchenne podobne do łyżki o płaskiej szerokiej końcówce (w gruncie rzeczy nazywa się tak kilka modeli narzędzi), które może służyć do rozsmarowywania czegoś, odskrobywania, transportowania i serwowania (zwykle czegoś pieczonego)}
\Clue{14}{}{mały lub miły geniusz - ktoś bardzo zdolny}
\Clue{16}{}{mieszkaniec Pabianic}
\Clue{18}{}{ur. w 1952 r., śpiewaczka (mezzosopran koloraturowy) laureatka wielu międzynarodowych konkursów śpiewaczki}
\Clue{19}{}{twórczość Cycerona, jego styl, teksty}
\Clue{20}{}{o człowieku, który jest przywiązany do rzeczy materialnych, przypisuje im największą wagę}
\Clue{23}{}{SKUMBRIA; ławicowa, ważna gospodarczo ryba}
\Clue{25}{}{odmiana teatru zajmująca się lalkami, które występują w sztuce zamiast aktorów, i widowiskami z udziałem tych lalek}
\Clue{27}{}{podwyższenie na sali, wolnym powietrzu, na którym odbywają się występy artystyczne}
\Clue{28}{}{krótka, szybka ankieta}
\Clue{29}{}{ruchomy styl przerywacza w obwodzie elektrycznym}
\Clue{30}{}{gimnastyk, medalista mistrzostw świata i Europy w latach siedemdziesiątych}
\Clue{31}{}{pas kontuszowy wyrabiany w Słucku}
\Clue{32}{}{amalgamowanie - sposób uzyskiwania metali szlachetnych z rud przy użyciu rtęci}
\Clue{33}{}{górny koniec szwu międzynosowego, stykający się z kością czołową}
\Clue{34}{}{człowiek, który jest podejrzliwy, nieufny}\end{PuzzleClues}

\begin{PuzzleClues}{\textbf{Pionowe}\\}\Clue{1}{}{typ skoczni o punkcie konstrukcyjnym usytuowanym między 75. (włącznie) a 100. metrem (bez tej odległości) zeskoku}
\Clue{2}{}{blacha kirysu osłaniająca piersi}
\Clue{4}{}{niemiecki rzeźbiarz, malarz i grafik (1887-1955) rzeźby figuralne, popiersia portretowe, akwaforty, ilustracje książkowe}
\Clue{7}{}{oficjalne składanie zeznań}
\Clue{8}{}{najmniejsza jednostka w armii rzymskiej licząca początkowo 100 a później 60 żołnierzy}
\Clue{9}{}{osoba, która ściska dłonie; dawniej był to zawód, obecnie raczej w użyciu pogardliwym}
\Clue{10}{}{naczynie stołowe do podawania zupy}
\Clue{12}{}{gazotron z katodą torowaną wypełnioną argonem}
\Clue{14}{}{miejsce przeznaczone do walk szermierczych}
\Clue{15}{}{nagromadzenie ropy w naturalnych jamach ciała; w przeciwieństwie do ropnia, ropniak nie lokalizuje się bezpośrednio w tkankach}
\Clue{17}{}{MLIK, MOLIK}
\Clue{19}{}{ciastko pieczone w niewielkiej okrągłej formie}
\Clue{21}{}{element struktury tkaniny, jeden z dwóch układów nitek tworzących tkaninę}
\Clue{22}{}{jeden z czołowych kompozytorów amerykańskiej muzyki rozrywkowej (1885-1945); musicale muzyka do filmów}
\Clue{24}{}{historyczna rzymska miara długości, tzw. podwójny krok, sążeń, ok. 1,48 metra}
\Clue{25}{}{element sprzętów lub pojazdów, poruszających się po śniegu, umożliwiających im ślizg}
\Clue{26}{}{o pieniądzach}\end{PuzzleClues}\newpage\section*{Krzyżówka 149}

\noindent\begin{Puzzle}{19}{26}|*	|*	|*	|*	|[1][S]\drarr	|g	|r	|a	|c	|i	|l	|i	|z	|u	|c	|h	|*	|*	|*	|*	|.
|*	|*	|[2][S]\rarr	|a	|n	|a	|n	|a	|s	|ó	|w	|k	|a	|*	|[3][S]\drarr	|m	|a	|ł	|y	|*	|.
|[4][S]\drarr	|j	|a	|p	|o	|n	|i	|e	|c	|*	|*	|*	|[5][S]\drarr	|k	|o	|w	|a	|r	|*	|*	|.
|h	|[6][S]\rarr	|w	|y	|w	|r	|o	|t	|e	|k	|*	|*	|i	|*	|b	|*	|[7][S]\darr	|*	|*	|*	|.
|a	|*	|[8][S]\rarr	|n	|i	|e	|s	|t	|a	|c	|j	|o	|n	|a	|r	|n	|o	|ś	|ć	|*	|.
|t	|[9][S]\darr	|*	|*	|n	|*	|[10][S]\rarr	|a	|r	|a	|m	|i	|d	|*	|ó	|*	|p	|*	|*	|*	|.
|a	|i	|*	|*	|i	|*	|*	|*	|[11][S]\drarr	|r	|e	|z	|y	|s	|t	|o	|r	|*	|*	|*	|.
|*	|n	|[12][S]\rarr	|k	|a	|r	|a	|i	|b	|*	|*	|*	|k	|*	|[][,]{ }	|*	|*	|*	|*	|*	|.
|[13][S]\rarr	|d	|i	|k	|r	|e	|o	|z	|a	|u	|r	|y	|*	|*	|c	|*	|*	|*	|*	|*	|.
|*	|i	|[14][S]\rarr	|d	|z	|i	|k	|ó	|w	|*	|*	|[15][S]\rarr	|s	|z	|y	|k	|a	|n	|a	|*	|.
|*	|a	|*	|*	|*	|*	|*	|*	|i	|[16][S]\rarr	|t	|y	|ł	|ó	|w	|k	|a	|*	|*	|*	|.
|[17][S]\drarr	|n	|a	|j	|e	|m	|n	|i	|c	|t	|w	|o	|*	|*	|i	|*	|*	|*	|*	|*	|.
|z	|a	|*	|*	|*	|*	|*	|*	|i	|[18][S]\drarr	|t	|y	|t	|u	|l	|i	|k	|*	|[19][S]\darr	|*	|.
|n	|*	|*	|*	|[20][S]\rarr	|r	|o	|j	|e	|k	|*	|*	|[21][S]\darr	|[22][S]\darr	|n	|[23][S]\darr	|[24][S]\darr	|*	|u	|*	|.
|a	|*	|*	|[25][S]\rarr	|c	|z	|a	|p	|l	|a	|[][,]{ }	|r	|a	|f	|o	|w	|a	|*	|p	|*	|.
|k	|*	|[26][S]\drarr	|g	|w	|i	|z	|d	|*	|w	|*	|*	|n	|e	|p	|a	|k	|[27][S]\darr	|r	|*	|.
|[][,]{ }	|*	|d	|[28][S]\rarr	|a	|g	|r	|a	|f	|a	|*	|*	|g	|s	|r	|z	|s	|c	|z	|*	|.
|t	|[29][S]\rarr	|r	|e	|w	|e	|r	|s	|*	|ł	|*	|*	|i	|t	|a	|k	|a	|z	|e	|*	|.
|o	|*	|u	|[30][S]\rarr	|p	|ł	|ó	|c	|i	|e	|n	|k	|o	|*	|w	|a	|m	|e	|j	|*	|.
|n	|*	|g	|[31][S]\rarr	|i	|l	|n	|i	|c	|k	|a	|*	|t	|*	|n	|*	|i	|r	|m	|*	|.
|a	|*	|i	|[32][S]\rarr	|l	|u	|n	|i	|t	|*	|*	|*	|e	|*	|y	|*	|t	|s	|o	|*	|.
|ż	|*	|[][,]{ }	|*	|[33][S]\rarr	|p	|r	|o	|l	|a	|m	|i	|n	|a	|*	|[34][S]\rarr	|e	|k	|s	|*	|.
|o	|*	|b	|*	|*	|[35][S]\rarr	|c	|u	|m	|i	|n	|g	|s	|*	|*	|*	|k	|*	|t	|*	|.
|w	|*	|i	|*	|*	|[36][S]\rarr	|p	|r	|z	|y	|g	|r	|y	|w	|k	|a	|*	|*	|k	|*	|.
|y	|*	|e	|*	|*	|*	|*	|[37][S]\rarr	|w	|y	|c	|i	|n	|e	|k	|*	|*	|*	|a	|*	|.
|*	|*	|g	|*	|*	|*	|*	|*	|*	|[38][S]\rarr	|s	|t	|a	|t	|u	|s	|*	|*	|*	|*	|.
|*	|*	|*	|[39][S]\rarr	|i	|m	|m	|e	|l	|m	|a	|n	|*	|*	|*	|*	|*	|*	|*	|*	|.\end{Puzzle}

\newpage

\begin{PuzzleClues}{\textbf{Poziome}\\}\Clue{1}{}{Gracilisuchus - rodzaj niewielkiego archozaura z kladu Suchia, żyjącego w środkowym triasie na terenie dzisiejszej Ameryki Południowej; jego szczątki odnaleziono w formacji Chanares w Argentynie}
\Clue{2}{}{wódka o smaku ananasa}
\Clue{3}{}{dziecko, chłopczyk, zazwyczaj w wieku mniej niż 6 lat}
\Clue{4}{}{o mieszkańcu Japonii, człowieku pochodzenia japońskiego, z lekceważeniem}
\Clue{5}{}{stop zawierający przeciętnie 30\% niklu, 20\% kobaltu i 50\% żelaza}
\Clue{6}{}{but szyty na lewej stronie, a noszony po wywróceniu na prawą}
\Clue{8}{}{możliwość przenoszenia z miejsca na miejsce}
\Clue{10}{}{polimer, poliamid włóknotwórczy o dużej wytrzymałości}
\Clue{11}{}{najprostszy element bierny obwodu elektrycznego wykonany z materiału stawiającego opór przepływowi prądu, wykorzystywany do ograniczenia prądu w nim płynącego}
\Clue{12}{}{członek indiańskiego plemienia Karaibów}
\Clue{13}{}{Dicraeosauridae - rodzina dinozaurów z grupy zauropodów}
\Clue{14}{}{część Tarnobrzega, dawniej osobny obszar słynący z Pałacu Tarnowskich wraz z dworem i kompleksem parkowo-ogrodowym}
\Clue{15}{}{utrudnienie na torze wyścigowym}
\Clue{16}{}{końcowa, prezentująca napisy sekwencja filmu, odcinka serialu lub innego programu}
\Clue{17}{}{wynajmowanie się do pracy, paranie się najemnictwem}
\Clue{18}{}{nazwa publikacji}
\Clue{20}{}{wir wodny}
\Clue{25}{}{Egretta gularis gularis - nominatywny podgatunek czapli rafowej (Egretta gularis)}
\Clue{26}{}{śpiew niektórych ptaków}
\Clue{28}{}{architektoniczno-dekoracyjny zwornik, ozdobna klamra, która bywa umieszczana w punktach łączenia poszczególnych elementów konstrukcyjnych, np. jako zwieńczenie archiwolty}
\Clue{29}{}{lewa strona, np. haftowanej tkaniny, obrazu, skrzydeł ikony}
\Clue{30}{}{zdrobniale o płótnie - tkaninie}
\Clue{31}{}{pisarka i publicystka (1865-96), redaktorka „Bluszcz”, „Ilustrowany skarbczyk Polski”, poezje, powieści, komedie}
\Clue{32}{}{grunt pochodzący z Księżyca}
\Clue{33}{}{białko globularne o dużej zawartości reszt kwasu glutaminowego i proliny}
\Clue{34}{}{były chłopak, partner, mąż; rzeczownik rodzaju męskiego, nieodmienny}
\Clue{35}{}{(1894-1962), amerykański poeta, prozaik i malarz, przedstawiciel amerykańskiej awangardy}
\Clue{36}{}{instrumentalny wstęp do utworu}
\Clue{37}{}{fragment kartki z informacją wycięty z prasy}
\Clue{38}{}{pozycja lub funkcja czegoś w obrębie jakiejś większej całości}
\Clue{39}{}{zawrót}\end{PuzzleClues}

\begin{PuzzleClues}{\textbf{Pionowe}\\}\Clue{1}{}{osoba, która chętnie przejmuje i wprowadza nowe poglądy, idee, sądy, działania, rzeczy, nowinki}
\Clue{3}{}{wszystkie zawierane w drodze czynności prawnych stosunki i umowy cywilnoprawne, w szczególności dotyczące wymiany dóbr i usług}
\Clue{4}{}{mikrobiolog japoński (1873-1938); lek przeciw kile; salwarsan}
\Clue{5}{}{duży ptak pochodzący z Ameryki Płn.; samiec z wachlarzowatym ogonem oraz podgardlem ze zwisającym płatem skóry; hodowany dla smacznego mięsa}
\Clue{7}{}{ostre pouczenie, reprymenda}
\Clue{9}{}{stan środkowowsch. części USA, obszar 94 tyś. km2, stolica Indianapolis}
\Clue{11}{}{osoba bawiąca innych, rozweselająca publiczność, najczęściej zawodowo}
\Clue{17}{}{oznaczenie statku informujące o dopuszczalnym zanurzeniu statku w wodzie morskiej i słodkiej}
\Clue{18}{}{niewielka część czegoś, odcinek, fragment, urywek}
\Clue{19}{}{drobna uprzejmość}
\Clue{21}{}{hormon peptydowy, którego zadaniem jest kontrola stężenia jonów sodowych i potasowych w organizmie}
\Clue{22}{}{podniosłe obchody jakiegoś wydarzenia}
\Clue{23}{}{zawartość wazki, małej, ozdobnej wazy}
\Clue{24}{}{aksamitka, Tagetes - rodzaj roślin jednorocznych należący do rodziny astrowatych}
\Clue{26}{}{drugie z kolei przełożenie skrzyni biegów w pojeździe mechanicznym regulujące stopień prędkości pojazdu}
\Clue{27}{}{miasto w południowej części województwa pomorskiego, w powiecie chojnickim, siedziba gminy miejsko-wiejskiej Czersk}\end{PuzzleClues}\newpage\section*{Krzyżówka 150}

\noindent\begin{Puzzle}{25}{19}|*	|*	|*	|*	|*	|*	|*	|*	|*	|*	|*	|*	|*	|[1][S]\drarr	|b	|u	|c	|*	|[2][S]\drarr	|b	|ę	|b	|e	|n	|*	|*	|.
|*	|*	|*	|*	|*	|*	|*	|*	|*	|*	|*	|[3][S]\drarr	|h	|a	|n	|c	|o	|c	|k	|*	|*	|*	|[4][S]\darr	|*	|*	|*	|.
|*	|*	|*	|*	|*	|*	|*	|*	|*	|[5][S]\darr	|*	|s	|*	|v	|[6][S]\rarr	|s	|p	|r	|a	|w	|a	|*	|ś	|*	|*	|*	|.
|*	|*	|[7][S]\drarr	|o	|s	|p	|r	|z	|ę	|t	|*	|z	|*	|i	|*	|*	|*	|*	|c	|*	|[8][S]\drarr	|l	|w	|i	|a	|*	|.
|*	|*	|p	|*	|[9][S]\rarr	|b	|o	|i	|l	|e	|a	|u	|*	|z	|[10][S]\rarr	|l	|i	|c	|z	|e	|b	|n	|i	|k	|*	|*	|.
|*	|*	|a	|[11][S]\rarr	|p	|o	|w	|t	|ó	|r	|k	|a	|*	|o	|*	|[12][S]\darr	|[13][S]\darr	|*	|k	|*	|u	|*	|d	|*	|*	|*	|.
|*	|*	|s	|[14][S]\rarr	|b	|i	|s	|u	|r	|m	|a	|n	|*	|*	|[15][S]\darr	|o	|d	|[16][S]\darr	|a	|*	|j	|[17][S]\darr	|w	|*	|*	|*	|.
|*	|*	|t	|[18][S]\rarr	|ś	|w	|i	|n	|i	|a	|r	|e	|k	|*	|p	|b	|e	|r	|[][,]{ }	|[19][S]\darr	|a	|p	|a	|*	|*	|*	|.
|*	|[20][S]\drarr	|o	|d	|c	|i	|e	|k	|*	|k	|[21][S]\rarr	|r	|y	|j	|o	|s	|k	|o	|c	|z	|k	|i	|*	|*	|*	|*	|.
|*	|w	|f	|*	|*	|*	|[22][S]\darr	|*	|*	|a	|*	|i	|*	|[23][S]\darr	|d	|e	|a	|z	|h	|a	|*	|a	|[24][S]\darr	|[25][S]\darr	|*	|*	|.
|*	|i	|o	|*	|*	|*	|s	|*	|*	|d	|*	|a	|*	|j	|p	|r	|l	|e	|i	|j	|*	|s	|t	|c	|*	|*	|.
|*	|e	|r	|[26][S]\rarr	|k	|a	|p	|o	|t	|a	|ż	|*	|*	|a	|o	|w	|k	|t	|ń	|ą	|[27][S]\darr	|e	|u	|e	|*	|*	|.
|*	|s	|i	|*	|*	|*	|ó	|*	|*	|n	|*	|*	|*	|w	|r	|a	|o	|a	|s	|c	|k	|c	|b	|l	|*	|*	|.
|*	|z	|u	|[28][S]\rarr	|k	|o	|l	|o	|r	|*	|*	|*	|*	|a	|a	|t	|m	|*	|k	|z	|ą	|k	|m	|o	|*	|*	|.
|*	|a	|m	|*	|[29][S]\rarr	|ś	|n	|i	|e	|ż	|y	|c	|a	|*	|*	|o	|a	|*	|a	|e	|p	|i	|a	|z	|*	|*	|.
|*	|k	|*	|[30][S]\rarr	|ś	|m	|i	|e	|t	|a	|n	|k	|a	|*	|*	|r	|n	|*	|*	|k	|i	|*	|r	|j	|*	|*	|.
|*	|*	|*	|*	|*	|[31][S]\rarr	|k	|o	|l	|u	|b	|r	|y	|n	|a	|*	|i	|*	|*	|*	|e	|*	|y	|a	|*	|*	|.
|[32][S]\rarr	|k	|r	|o	|w	|a	|*	|*	|*	|*	|*	|*	|*	|*	|[33][S]\rarr	|m	|a	|r	|c	|e	|l	|i	|n	|*	|*	|*	|.
|*	|*	|*	|*	|*	|*	|*	|*	|*	|[34][S]\rarr	|p	|o	|w	|i	|d	|z	|*	|*	|*	|*	|*	|*	|a	|*	|*	|*	|.
|*	|*	|*	|*	|*	|[35][S]\rarr	|p	|o	|k	|e	|r	|[][,]{ }	|r	|o	|z	|b	|i	|e	|r	|a	|n	|y	|*	|*	|*	|*	|.\end{Puzzle}

\newpage

\begin{PuzzleClues}{\textbf{Poziome}\\}\Clue{1}{}{człowiek zarozumiały, mający o sobie wysokie mniemanie, w rzeczywistości mało inteligentny}
\Clue{2}{}{muzyczny instrument perkusyjny}
\Clue{3}{}{amerykański pianista i kompozytor jazzowy ur. w 1940 r.; pionier jazzu fusion}
\Clue{6}{}{materia, zespół określonych okoliczności, ważnych dla kogoś lub czegoś}
\Clue{7}{}{urządzenia zamontowane na statku}
\Clue{8}{}{zatoka Morza Śródziemnego, u wybrzeży Francji, głębokość do 1000 m}
\Clue{9}{}{(1636-1711), francuski poeta i krytyk literacki, historiograf Ludwika XIV; „Sztuka rymotwórcza”}
\Clue{10}{}{wyraz, którego prymarną funkcją jest określanie liczby bądź kolejności bytów czy sytuacji}
\Clue{11}{}{czynność polegająca na powtórnym wykonaniu czegoś}
\Clue{14}{}{staropolska nazwa muzułmanina}
\Clue{18}{}{pomocnik świniarza, pasący świnie}
\Clue{20}{}{to, że jakaś ciecz odciekła}
\Clue{21}{}{ryjoskoczkowate, Macroscelididae - jedyna rodzina w obrębie rzędu ryjkonosów; należy do niej kilkanaście współcześnie żyjących oraz kilka wymarłych gatunków niewielkich afrykańskich zwierząt, wcześniej zaliczanych do owadożernych}
\Clue{26}{}{przewrócenie się samolotu na grzbiet podczas lądowania lub startu}
\Clue{28}{}{żywa, kontrastowa barwa, taka, która się odróżnia, stanowi barwny akcent}
\Clue{29}{}{Leucojum - rodzaj roślin z rodziny amarylkowatych}
\Clue{30}{}{słodka śmietana, często o wysokiej zawartości tłuszczu}
\Clue{31}{}{wężownica, serpentyna odprzodowe działo o długiej lufie, używane w XVI/XVII w}
\Clue{32}{}{dorosła samica niektórych innych - niezaliczających się do bydła - ssaków (przeżuwaczy, parzystokopytnych)}
\Clue{33}{}{część Poznania znajdująca się po wschodniej stronie osiedla Ławica}
\Clue{34}{}{wieś w Wielkopolsce, siedziba gminy o tej samej nazwie}
\Clue{35}{}{odmiana pokera, w której za przegraną nie płaci się pieniędzmi, lecz ściągnięciem z siebie jakiegoś elementu garderoby}\end{PuzzleClues}

\begin{PuzzleClues}{\textbf{Pionowe}\\}\Clue{1}{}{zawiadomienie o nadejściu przesyłki pocztowej, której nie można doręczyć adresatowi bezpośrednio, informujące adresata o możliwości odbioru przesyłki we wskazanym urzędzie pocztowym (lub innej instytucji) oraz do kiedy będzie to możliwe; jest to niepoprawny ortograficznie zapis tego rzeczownika}
\Clue{2}{}{Anas poecilorhyncha zonorhyncha - podgatunek ptaka wyróżniony w obrębie gatunku kaczka pstrodzioba (Anas poecilorhyncha); status taksonu sporny, przez niektórych autorów uznawany za odrębny gatunek (Anas zonorhyncha)}
\Clue{3}{}{określenie grup chłopstwa francuskiego, biorącego udział w powstaniach w okresie rewolucji francuskiej w Bretanni i departamencie Mayenne}
\Clue{4}{}{DEREŃ}
\Clue{5}{}{dolna warstwa nawierzchni drogowej}
\Clue{7}{}{w kościołach wczesnochrześcijańskich: pomieszczenie w prezbiterium przylegające symetrycznie do boków apsydy; stanowi zakrystię, zazwyczaj występuje podwójnie, po obu stronach ołtarza jako diakonikon i prothesis}
\Clue{8}{}{zderzak pojazdu szynowego}
\Clue{12}{}{wysłannik państwa lub instytucji, który ma obserwować, badać coś, raportować}
\Clue{13}{}{zdobienie wykonane techniką dekalkomanii}
\Clue{15}{}{FILAR  }
\Clue{16}{}{ornament architektoniczny w kształcie rozwiniętej róży}
\Clue{17}{}{kolarz, zwycięzca Wyścigu Pokoju w 1985 r., później przeszedł na zawodowstwo}
\Clue{19}{}{świecki symbol świąt Wielkiejnocy}
\Clue{20}{}{wysoka i chuda osoba}
\Clue{22}{}{partner w biznesie, który wnosi część wkładu w inwestycję}
\Clue{23}{}{wyspa w Azji Południowo-Wschodniej, jedna z Wielkich Wysp Sundajskich}
\Clue{24}{}{dawny instrument smyczkowy o długim korpusie rezonansowym z jedną struną}
\Clue{25}{}{GRZEBIONATKA roślina zielna, krzew lub półkrzew z szarłowatych uprawiana jako ozdobna}
\Clue{27}{}{poddawanie wyrobów obróbce poprzez wystawienie ich na równomierne działanie jakiegoś czynnika czy składnika}\end{PuzzleClues}\newpage\section*{Krzyżówka 151}

\noindent\begin{Puzzle}{23}{22}|*	|[1][S]\darr	|*	|[2][S]\darr	|*	|*	|*	|*	|*	|*	|*	|*	|*	|*	|[3][S]\drarr	|g	|ę	|ś	|*	|[4][S]\darr	|[5][S]\darr	|*	|*	|[6][S]\darr	|.
|*	|d	|*	|t	|*	|*	|*	|*	|*	|*	|*	|[7][S]\drarr	|r	|e	|l	|i	|n	|g	|*	|k	|r	|*	|*	|m	|.
|*	|z	|*	|l	|*	|[8][S]\rarr	|b	|o	|ż	|o	|g	|r	|o	|b	|i	|e	|c	|*	|*	|o	|o	|*	|*	|n	|.
|*	|i	|*	|e	|*	|[9][S]\rarr	|s	|e	|m	|a	|f	|o	|r	|*	|g	|[10][S]\drarr	|o	|l	|b	|r	|z	|y	|m	|*	|.
|[11][S]\drarr	|d	|o	|n	|[][,]{ }	|q	|u	|i	|c	|h	|o	|t	|e	|*	|a	|ś	|*	|*	|[12][S]\darr	|e	|r	|*	|*	|*	|.
|k	|a	|[13][S]\darr	|*	|[14][S]\drarr	|s	|ł	|u	|p	|i	|c	|a	|*	|*	|*	|n	|[15][S]\darr	|*	|i	|l	|u	|*	|[16][S]\darr	|[17][S]\darr	|.
|o	|*	|r	|[18][S]\darr	|o	|*	|*	|*	|*	|[19][S]\darr	|*	|t	|[20][S]\darr	|[21][S]\darr	|[22][S]\darr	|i	|w	|*	|n	|a	|s	|[23][S]\darr	|k	|s	|.
|p	|[24][S]\darr	|a	|n	|d	|[25][S]\darr	|[26][S]\darr	|[27][S]\darr	|*	|r	|[28][S]\darr	|o	|p	|p	|m	|e	|y	|*	|f	|c	|z	|b	|i	|z	|.
|*	|o	|d	|i	|p	|p	|b	|w	|*	|o	|p	|r	|i	|i	|o	|ż	|w	|*	|o	|j	|n	|u	|n	|a	|.
|[29][S]\drarr	|p	|i	|e	|r	|s	|i	|ó	|w	|k	|a	|*	|e	|ę	|d	|y	|i	|*	|r	|a	|i	|d	|o	|b	|.
|m	|e	|o	|z	|a	|a	|a	|ł	|*	|*	|w	|*	|p	|k	|e	|n	|a	|[30][S]\darr	|m	|[][,]{ }	|k	|z	|[][,]{ }	|e	|.
|u	|r	|m	|a	|w	|m	|ł	|*	|*	|*	|i	|*	|r	|n	|r	|k	|d	|k	|a	|c	|*	|i	|f	|r	|.
|s	|a	|e	|w	|a	|m	|y	|*	|[31][S]\rarr	|m	|a	|s	|z	|y	|n	|a	|[][,]{ }	|u	|c	|z	|ą	|c	|a	|*	|.
|t	|[][,]{ }	|t	|o	|[][,]{ }	|o	|*	|*	|*	|*	|n	|[32][S]\darr	|[][,]{ }	|[][,]{ }	|i	|*	|s	|b	|j	|ą	|*	|z	|m	|*	|.
|e	|m	|r	|d	|c	|f	|[33][S]\drarr	|m	|m	|o	|*	|b	|z	|j	|z	|[34][S]\darr	|k	|i	|a	|s	|*	|*	|i	|*	|.
|l	|y	|i	|o	|z	|i	|z	|[35][S]\darr	|*	|[36][S]\darr	|*	|u	|i	|a	|m	|r	|a	|k	|*	|t	|*	|*	|l	|*	|.
|[][,]{ }	|d	|a	|w	|a	|t	|n	|r	|[37][S]\darr	|n	|[38][S]\darr	|f	|e	|ś	|*	|y	|r	|u	|*	|k	|*	|*	|i	|*	|.
|p	|l	|*	|i	|s	|*	|*	|a	|p	|a	|s	|o	|l	|*	|*	|k	|b	|l	|*	|o	|*	|*	|j	|*	|.
|s	|a	|*	|e	|o	|*	|[39][S]\drarr	|p	|r	|z	|e	|r	|o	|s	|t	|*	|o	|u	|*	|w	|*	|*	|n	|*	|.
|i	|n	|*	|c	|w	|*	|f	|e	|a	|g	|n	|*	|n	|*	|*	|*	|w	|m	|*	|a	|*	|*	|e	|*	|.
|*	|a	|*	|*	|a	|*	|i	|ć	|w	|u	|a	|*	|y	|*	|*	|*	|y	|*	|*	|*	|*	|*	|*	|*	|.
|*	|*	|*	|*	|*	|*	|ś	|*	|o	|l	|t	|*	|*	|[40][S]\rarr	|h	|o	|*	|*	|*	|*	|*	|*	|*	|*	|.
|*	|*	|*	|*	|*	|*	|*	|*	|*	|*	|*	|*	|*	|*	|*	|*	|*	|*	|*	|*	|*	|*	|*	|*	|.\end{Puzzle}

\newpage

\begin{PuzzleClues}{\textbf{Poziome}\\}\Clue{3}{}{ptak wodny; poszczególne gatunki tego ptaka klasyfikowane są w taksonomii biologicznej w obrębie podrodziny gęsi (Anserinae), w rodzinie kaczkowatych (Anatidae)}
\Clue{7}{}{rodzaj półki, która podwieszana jest nad kuchnią lub stołemi; przeznaczona na kuchenne drobiazgi}
\Clue{8}{}{członek średniowiecznego zakonu rycerskiego}
\Clue{9}{}{urządzenie brzegowe wskazujące siłę i kierunek wiatru}
\Clue{10}{}{gwiazda o dużym promieniu i jasności kilkaset razy większej od Słońca}
\Clue{11}{}{fikcyjna postać z powieści Miguela de Cervantesa; szlachcic, który pod wpływem literatury o czynach rycerskich postanawia zostać błędnym rycerzem}
\Clue{14}{}{część składowa pługa łacząca jego korpus z grządzielem lub ramą}
\Clue{29}{}{zawartość piersiówki, czyli niewielkiej i płaskiej, szklanej lub metalowej buteleczki przeznaczonej do przechowywania i przenoszenia alkoholu}
\Clue{31}{}{dydaktyczna}
\Clue{33}{}{rodzaj sieciowych gier komputerowych RPG, w których duża liczba graczy może grać ze sobą w wirtualnym świecie}
\Clue{39}{}{przerost komórek albo narządów ponad miarę}
\Clue{40}{}{w chemii: symbol holmu}\end{PuzzleClues}

\begin{PuzzleClues}{\textbf{Pionowe}\\}\Clue{1}{}{długa prosta broń drzewcowa zakończona ostrym grotem}
\Clue{2}{}{O -  pierwiastek chemiczny o liczbie atomowej 8, niemetal z grupy tlenowców w układzie okresowym}
\Clue{3}{}{klasa (wartość) ludzi, człowieka; środowisko, z jakiego się ktoś wywodzi, wrażenie jakie sprawia}
\Clue{4}{}{miara zależności zmiennych losowych przy usuniętym wpływie innych zmiennych losowych z ustalonego zbioru}
\Clue{5}{}{rozrusznik silnika spalinowego; urządzenie do uruchamiania silnika spalinowego}
\Clue{6}{}{w chemii: symbol manganu}
\Clue{7}{}{namagnesowana wkładka ferrytowa w falowodzie}
\Clue{10}{}{różnokształtny rozgałęziony kryształ lodu, dendryt}
\Clue{11}{}{przykre doświadczenie, mające pozytywne konsekwencje, motywujące do działania}
\Clue{12}{}{punkt, w którym można się czegoś dowiedzieć}
\Clue{13}{}{dział fizyki i metrologii zajmujący się ilościowymi pomiarami energii promieniowania i wielkości fizycznych z nią związanych}
\Clue{14}{}{odprawa celna, która pozwala na wwiezienie do obszaru celnego towarów spoza tego obszaru przy całkowitym lub częściowym zwolnieniu z cła na określony czas (24 miesiące) ze skutkiem konieczności wywiezienia tych towarów po tym czasie lub uzyskania dla nich innego statusu (np. przeznaczenie do obrotu na danym obszarze celnym)}
\Clue{15}{}{urząd zajmujący się pozyskiwaniem, gromadzeniem, przetwarzaniem i wykorzystywaniem informacji o dochodach, obrotach, rzeczach i prawach majątkowych podmiotów podlegających kontroli skarbowej}
\Clue{16}{}{filmy, które mogą oglądać zarówno dorośli, jak i dzieci; najczęściej są to filmy przygodowe, animowane}
\Clue{17}{}{proceder przestępczy, polegający na grabieży mienia pozostającego bez opieki w wyniku klęsk żywiołowych, przewrotów społecznych lub wojen; szabrownictwo}
\Clue{18}{}{człowiek robiący coś niezawodowo, bez umiejętności w danej dziedzinie}
\Clue{19}{}{jednostka czasu powiązana z obiegiem Ziemi wokół Słońca}
\Clue{20}{}{odmiana pieprzu; niedojrzałe, zielone nasiona pieprzu czarnego marynowane w kwasie octowym lub mlekowym albo konserwowane solanką}
\Clue{21}{}{najczęściej uprawiana i bardzo znana odmiana wśród odmian warzywnych fasoli wielokwiatowej}
\Clue{22}{}{w architekturze odchodzenie od historyzmu, dążenie do prostoty i funkcjonalności}
\Clue{23}{}{kajakarz, dwukrotny mistrz świata z 1977 r}
\Clue{24}{}{serial przewidziany na dużą liczbę odcinków o niewielkich walorach artystycznych, przedstawiający losy grupy osób powiązanych ze sobą pokrewieństwem lub wspólnym miejscem pracy}
\Clue{25}{}{roślina rosnąca w środowisku piaszczystym, głównie na wydmach}
\Clue{26}{}{członek grupy politycznej lub zbrojnej zwalczającej bolszewików w czasie wojny domowej w latach 1917-1923, zwolennik rosyjskiego nacjonalizmu i caratu}
\Clue{27}{}{przenośnie: człowiek przeznaczony do wykonywania ciężkich prac lub taki, który po prostu ciężko i dużo pracuje (jak wół)}
\Clue{28}{}{PAPIO; afrykańska, wąskonosa małpa roślinożerna}
\Clue{29}{}{Mustelus canis - gatunek drapieżnej ryby chrzęstnoszkieletowej z rodziny mustelowatych (Triakidae)}
\Clue{30}{}{CUBICULUM, sypialnia w starorzymskim domu}
\Clue{32}{}{roztwór buforowy}
\Clue{33}{}{w chemii: symbol cynku}
\Clue{34}{}{gwałtowny wybuch śmiechu}
\Clue{35}{}{rzemień, taśma do podtrzymywania szabli przy pasie}
\Clue{36}{}{upiór Pierścienia; upiorny sługa Saurona w literackim legendarium, wykreowanym przez J. R. R. Tolkiena}
\Clue{37}{}{prawica, prawa strona sceny politycznej i poglądy dla niej charakterystyczne}
\Clue{38}{}{jedna z najważniejszych i najtrwalszych instytucji politycznych starożytnego Rzymu}
\Clue{39}{}{fioł, zapamiętanie w czymś, mania}\end{PuzzleClues}\newpage\section*{Krzyżówka 152}

\noindent\begin{Puzzle}{22}{27}|*	|*	|*	|*	|*	|*	|*	|*	|[1][S]\drarr	|p	|i	|a	|s	|k	|o	|w	|n	|i	|c	|a	|*	|*	|[2][S]\darr	|.
|*	|*	|[3][S]\drarr	|k	|o	|r	|z	|e	|n	|i	|e	|*	|*	|*	|*	|*	|[4][S]\darr	|*	|*	|[5][S]\darr	|*	|*	|n	|.
|*	|*	|b	|*	|*	|[6][S]\rarr	|t	|r	|u	|f	|l	|a	|*	|[7][S]\rarr	|c	|a	|p	|e	|l	|l	|a	|*	|ó	|.
|*	|*	|r	|*	|*	|[8][S]\rarr	|f	|o	|t	|o	|a	|u	|t	|o	|t	|r	|o	|f	|*	|i	|*	|*	|ż	|.
|*	|*	|z	|*	|[9][S]\drarr	|m	|e	|g	|a	|t	|o	|n	|a	|*	|[10][S]\drarr	|t	|r	|ó	|j	|n	|ó	|g	|*	|.
|*	|*	|e	|[11][S]\darr	|c	|*	|[12][S]\darr	|*	|*	|[13][S]\drarr	|h	|a	|r	|a	|d	|*	|z	|*	|*	|i	|*	|*	|*	|.
|*	|*	|z	|m	|ó	|*	|v	|*	|*	|g	|*	|[14][S]\drarr	|ź	|d	|z	|i	|e	|b	|l	|a	|r	|z	|*	|.
|[15][S]\drarr	|t	|i	|u	|r	|n	|i	|u	|r	|a	|*	|e	|*	|*	|i	|[16][S]\darr	|c	|*	|[17][S]\darr	|[][,]{ }	|*	|*	|[18][S]\darr	|.
|k	|*	|n	|n	|e	|[19][S]\darr	|t	|*	|*	|l	|*	|s	|*	|*	|o	|p	|z	|[20][S]\darr	|s	|k	|*	|*	|ś	|.
|r	|[21][S]\darr	|y	|i	|ń	|k	|a	|[22][S]\drarr	|g	|a	|r	|a	|*	|*	|b	|u	|k	|z	|l	|r	|*	|*	|w	|.
|e	|o	|*	|c	|k	|o	|r	|c	|*	|g	|*	|u	|[23][S]\rarr	|f	|a	|l	|a	|i	|s	|e	|*	|*	|i	|.
|t	|s	|*	|y	|a	|m	|a	|z	|*	|o	|[24][S]\darr	|ł	|*	|*	|k	|*	|*	|e	|h	|d	|[25][S]\darr	|*	|d	|.
|*	|u	|[26][S]\darr	|p	|*	|p	|*	|a	|*	|[][,]{ }	|k	|*	|*	|[27][S]\darr	|[][,]{ }	|[28][S]\darr	|*	|m	|*	|y	|d	|*	|w	|.
|[29][S]\drarr	|t	|r	|i	|n	|a	|k	|s	|o	|d	|o	|n	|*	|s	|d	|d	|*	|i	|[30][S]\darr	|t	|o	|*	|a	|.
|k	|k	|y	|u	|*	|k	|[31][S]\darr	|o	|[32][S]\rarr	|e	|l	|e	|k	|t	|r	|o	|n	|o	|w	|o	|l	|t	|*	|.
|o	|a	|b	|m	|*	|t	|ś	|m	|*	|m	|o	|*	|*	|o	|u	|w	|*	|m	|i	|w	|n	|*	|*	|.
|f	|[][,]{ }	|i	|*	|*	|o	|c	|i	|*	|i	|k	|*	|*	|ł	|g	|o	|*	|o	|d	|a	|o	|[33][S]\darr	|[34][S]\darr	|.
|e	|s	|c	|*	|*	|r	|i	|e	|[35][S]\darr	|d	|a	|*	|*	|o	|i	|d	|*	|r	|e	|*	|s	|d	|d	|.
|r	|o	|k	|*	|*	|*	|e	|r	|n	|o	|t	|*	|*	|w	|*	|z	|*	|z	|o	|*	|a	|r	|o	|.
|d	|s	|i	|*	|*	|*	|g	|z	|a	|f	|o	|*	|*	|y	|*	|i	|*	|e	|f	|*	|k	|o	|r	|.
|a	|e	|*	|*	|*	|*	|[][,]{ }	|*	|s	|f	|r	|*	|*	|*	|*	|k	|*	|*	|o	|*	|s	|b	|a	|.
|m	|n	|[36][S]\darr	|[37][S]\drarr	|g	|e	|r	|e	|z	|a	|*	|*	|*	|*	|*	|*	|*	|*	|n	|*	|o	|n	|d	|.
|*	|*	|p	|k	|*	|*	|y	|*	|y	|*	|[38][S]\rarr	|ś	|c	|i	|e	|r	|w	|o	|*	|*	|ń	|i	|c	|.
|[39][S]\rarr	|b	|l	|i	|k	|*	|ż	|[40][S]\drarr	|w	|i	|e	|r	|t	|n	|i	|c	|a	|*	|*	|*	|s	|c	|a	|.
|[41][S]\drarr	|p	|a	|c	|h	|n	|o	|t	|k	|a	|[][,]{ }	|b	|r	|a	|z	|y	|l	|i	|j	|s	|k	|a	|*	|.
|a	|*	|m	|z	|*	|*	|w	|o	|a	|*	|*	|*	|*	|[42][S]\rarr	|p	|i	|e	|r	|o	|g	|i	|*	|*	|.
|s	|*	|a	|*	|*	|*	|y	|n	|*	|*	|*	|[43][S]\rarr	|p	|a	|s	|y	|w	|i	|z	|m	|*	|*	|*	|.
|*	|*	|*	|*	|*	|*	|*	|*	|*	|*	|*	|*	|*	|*	|*	|*	|*	|*	|*	|*	|*	|*	|*	|.\end{Puzzle}

\newpage

\begin{PuzzleClues}{\textbf{Poziome}\\}\Clue{1}{}{Ammophila - rodzaj roślin należących do rodziny wiechlinowatych}
\Clue{3}{}{to, skąd coś lub ktoś się wywodzi}
\Clue{6}{}{czekoladka, zwykle kulista bądź o nieregularnym kształcie, wykonana z czekolady i masła lub śmietany}
\Clue{7}{}{KAPELLA; KOZA}
\Clue{8}{}{autotroficzny organizm pozyskujący pożywienie na drodze fotosyntezy}
\Clue{9}{}{wielokrotność tony, oznaczająca milion ton}
\Clue{10}{}{sprzęt wyposażony w trzy nogi i służący jako podstawa pod coś}
\Clue{13}{}{kraina ze stworzonej przez J.R.R. Tolkiena mitologii Śródziemia, której północną granicę stanowią rzeka Harnen i łańcuch Ephel Dúath, zamieszkała przez Haradrimów}
\Clue{14}{}{błonkówka z rodziny rośliniarek, wyrządza szkody w oziminach, larwy , żyją w źdźbłach traw i zbóż}
\Clue{15}{}{suknia z dołączonym elementem podkreślający tył bioder i nadający kobiecej sylwetce kształt litery S}
\Clue{22}{}{bieg statku gdy wiatr dmie z boku}
\Clue{23}{}{miasto w płn. Francji (Normandia), zwycięska bitwa wojsk alianckich z udziałem dywizji gen. Maczka nad Niemcami w 1944 r}
\Clue{29}{}{Trinaksodon - wymarły gad z grupy cynodontów, zamieszkujący południowe lasy około 240 mln lat temu; zamieszkiwał południową część Pangei, natomiast skamieniałości pochodzą głównie z Afryki}
\Clue{32}{}{jednostka energii stosowana w fizyce, opisująca energię, jaką uzyskuje bądź traci elektron, który przemieścił się w polu elektrycznym o różnicy potencjałów równej 1 woltowi}
\Clue{37}{}{afryk. małpa wąskonosa średniej wielkości, roślinożerna, żyje gromadnie}
\Clue{38}{}{martwe zwierzę}
\Clue{39}{}{w fotografii silne, utrwalone na kliszy odbicie światła}
\Clue{40}{}{zespół powiąznych ze sobą urządzeń mechanicznych niezbędnych do wykonania otworu wiertniczego w skale, gruncie lub betonie}
\Clue{41}{}{pachnotka zwyczajna, pachnotka uprawna, perilla zwyczajna, Perilla frutescens - gatunek rośliny jednorocznej z rodziny jasnotowatych; pochodzi z obszaru Indii, Chin, Japonii, Półwyspu Indyjskiego i Indochin; jako uciekinier z uprawy lub gatunek zawleczony rozprzestrzeniła się również gdzieniegdzie poza tymi rejonami; jest uprawiana w wielu krajach świata}
\Clue{42}{}{potrawa z ciasta z farszem mięsnym lub kapustą podawana z masłem lub sosem}
\Clue{43}{}{orientacja polityczna, w myśl której należy utrzymywać współpracę i dobre stosunki z Rosją, natomiast unikać nawiązywania tychże z Austrią i Niemcami; nurt popularny szczególnie w Polsce w czasach I wojny światowej}\end{PuzzleClues}

\begin{PuzzleClues}{\textbf{Pionowe}\\}\Clue{1}{}{nuta, składowa część dźwięku}
\Clue{2}{}{nóż stołowy - sztuciec służący do krojenia}
\Clue{3}{}{Brzeziny - wieś}
\Clue{4}{}{Ribes - rodzaj krzewu z rodziny agrestowatych (Grossulariaceae)}
\Clue{5}{}{limit w rachunku bankowym, do jakiego kredytobiorca może się zadłużyć w okresie określonym umową}
\Clue{9}{}{zdrobniale - córeczka, córka}
\Clue{10}{}{przedłużenie bukrszprytu ku przodowi}
\Clue{11}{}{zarząd miejski w starożytnym Rzymie}
\Clue{12}{}{suzuki z modelu Vitara}
\Clue{13}{}{Galago demidoff - gatunek owadożernej małpiatki z rodziny galagowatych; zamieszkuje lasy galeriowe Afryki zachodniej i centralnej}
\Clue{14}{}{zastępca atamana ASAUŁ}
\Clue{15}{}{owadożerny ssak o podziemnym trybie życia}
\Clue{16}{}{jednostka zdawkowa w Afganistanie; 1/100 afgani}
\Clue{17}{}{kod ISO 4217 szylinga Somalilandu}
\Clue{18}{}{DEREŃ}
\Clue{19}{}{program, aplikacja komputerowa, która służy do pakowania i kompresowania (zmniejszania rozmiaru) plików i folderów}
\Clue{20}{}{fikcyjny świat z cyklu powieści fantasy stworzonego przez Ursulę K. Le Guin}
\Clue{21}{}{choroba drzew iglastych, wywoływana przez grzyby niedoskonałe i workowce, objawiająca się przebarwieniami igieł, mogąca występować pod jedną z kilku postaci, m.in. osutki wiosennej, osutki jesiennej, osutki północnej czy osutki śnieżnej}
\Clue{22}{}{przyrząd do pomiaru czasu, zarówno mierzący upływ czasu w sposób ciągły (jak zegar), jak również mierzący czas tylko podczas pomiaru (jak stoper)}
\Clue{24}{}{wierzyciel, którego wierzytelność zabezpieczona jest nieruchomością dłużnika}
\Clue{25}{}{język germański blisko spokrewniony z językami niderlandzkim, afrykanerskim i niemieckim, często uznawany również za dialekt tego ostatniego; język używany głównie przez mieszkańców Niemiec północnych i Holandii}
\Clue{26}{}{kompozytor i dyrygent (1899-1978); utwory orkiestrowe, fortepianowe, chóralne, pieśni, opera}
\Clue{27}{}{pokój stołowy, jadalnia}
\Clue{28}{}{żartobliwie o dowodzie osobistym}
\Clue{29}{}{pusty, szczelny przedział na statku wodnym}
\Clue{30}{}{urządzenie umożliwiające nie tylko komunikację słowną z osobą, która chce wejść do budynku (jak to ma miejsce w domofonie), lecz umożliwiające również obejrzenie tej osoby}
\Clue{31}{}{ścieg tworzący wzór przypominający ułożone obok siebie ziarenka ryżu}
\Clue{33}{}{przedmioty niewielkich rozmiarów lub małe istoty}
\Clue{34}{}{czynnik, który wpływa na człowieka}
\Clue{35}{}{kawałek materiału lub skórki, zazwyczaj z jakimś znaczkiem, logo lub napisem, który naszywa się na inny materiał}
\Clue{36}{}{coś, co kogoś kompromituje, zaważa o jego negatywnej ocenie}
\Clue{37}{}{dzieło o niskiej wartości artystycznej i estetycznej}
\Clue{40}{}{biała glina stosowana do bielenia sufitów}
\Clue{41}{}{mistrz, tuz, znakomitość}\end{PuzzleClues}\newpage\section*{Krzyżówka 153}

\noindent\begin{Puzzle}{19}{26}|*	|*	|*	|*	|*	|*	|*	|[1][S]\darr	|[2][S]\drarr	|i	|s	|e	|o	|*	|[3][S]\darr	|[4][S]\drarr	|u	|s	|d	|*	|.
|*	|*	|*	|*	|*	|*	|*	|d	|b	|[5][S]\darr	|*	|[6][S]\darr	|*	|*	|b	|d	|*	|*	|*	|*	|.
|*	|*	|*	|*	|*	|*	|*	|w	|r	|p	|*	|c	|[7][S]\darr	|*	|o	|r	|*	|[8][S]\darr	|*	|*	|.
|*	|*	|*	|*	|*	|*	|*	|u	|a	|r	|*	|i	|p	|[9][S]\darr	|s	|o	|[10][S]\darr	|d	|[11][S]\darr	|*	|.
|*	|*	|*	|*	|*	|*	|*	|n	|n	|z	|[12][S]\darr	|l	|r	|k	|c	|h	|s	|z	|w	|*	|.
|*	|*	|*	|*	|*	|*	|*	|a	|d	|e	|l	|a	|z	|o	|h	|o	|ł	|i	|o	|[13][S]\darr	|.
|*	|*	|*	|*	|*	|*	|[14][S]\drarr	|s	|z	|p	|i	|c	|e	|l	|*	|b	|o	|e	|d	|i	|.
|*	|*	|*	|[15][S]\darr	|*	|*	|a	|t	|e	|ę	|s	|a	|k	|e	|[16][S]\darr	|y	|d	|r	|o	|l	|.
|*	|*	|*	|s	|*	|*	|b	|o	|l	|d	|t	|p	|s	|o	|n	|c	|y	|z	|l	|o	|.
|*	|[17][S]\darr	|*	|o	|[18][S]\darr	|*	|r	|k	|*	|*	|[][,]{ }	|*	|z	|r	|a	|z	|s	|b	|o	|r	|.
|*	|s	|[19][S]\darr	|p	|d	|*	|a	|r	|*	|*	|p	|[20][S]\darr	|t	|y	|s	|a	|z	|i	|t	|a	|.
|*	|ą	|m	|l	|r	|*	|k	|o	|[21][S]\drarr	|m	|a	|g	|a	|z	|y	|n	|e	|k	|*	|z	|.
|*	|d	|o	|i	|z	|*	|a	|t	|w	|*	|s	|a	|ł	|a	|c	|k	|k	|[][,]{ }	|*	|[][,]{ }	|.
|*	|[][,]{ }	|w	|c	|e	|*	|d	|n	|i	|*	|t	|n	|c	|*	|a	|a	|*	|z	|*	|r	|.
|*	|s	|a	|a	|w	|*	|a	|o	|d	|[22][S]\darr	|e	|s	|e	|*	|n	|*	|*	|i	|*	|ó	|.
|*	|k	|[][,]{ }	|[][,]{ }	|i	|*	|b	|ś	|o	|d	|r	|u	|n	|*	|i	|*	|*	|e	|*	|ż	|.
|*	|o	|p	|p	|a	|[23][S]\darr	|r	|ć	|w	|o	|s	|*	|i	|*	|e	|*	|*	|l	|*	|n	|.
|*	|r	|o	|o	|k	|s	|a	|*	|i	|r	|k	|*	|e	|*	|*	|*	|*	|o	|*	|i	|.
|*	|u	|g	|ł	|[][,]{ }	|m	|*	|*	|s	|k	|i	|*	|[][,]{ }	|*	|*	|*	|*	|n	|*	|c	|.
|[24][S]\drarr	|p	|r	|u	|s	|a	|c	|z	|k	|a	|*	|*	|r	|*	|*	|*	|*	|y	|*	|o	|.
|s	|k	|z	|d	|z	|l	|[25][S]\drarr	|p	|o	|s	|k	|r	|z	|y	|p	|k	|a	|*	|*	|w	|.
|k	|o	|e	|n	|a	|e	|s	|*	|*	|*	|[26][S]\rarr	|t	|u	|m	|a	|k	|i	|*	|*	|y	|.
|u	|w	|b	|i	|r	|c	|m	|[27][S]\rarr	|ś	|w	|i	|a	|t	|[][,]{ }	|d	|y	|s	|k	|u	|*	|.
|n	|y	|o	|o	|y	|*	|*	|*	|[28][S]\rarr	|z	|a	|k	|o	|p	|i	|a	|n	|k	|a	|*	|.
|k	|*	|w	|w	|*	|*	|[29][S]\rarr	|k	|r	|a	|k	|o	|w	|i	|a	|k	|*	|*	|*	|*	|.
|s	|*	|a	|a	|*	|*	|*	|[30][S]\rarr	|p	|r	|o	|t	|e	|k	|t	|o	|r	|k	|a	|*	|.
|*	|*	|*	|*	|[31][S]\rarr	|c	|o	|l	|l	|a	|g	|e	|*	|*	|*	|*	|*	|*	|*	|*	|.\end{Puzzle}

\newpage

\begin{PuzzleClues}{\textbf{Poziome}\\}\Clue{2}{}{jezioro we Włoszech, w Alpach Lombardzkich, powierzchnia 65 km2, głębokość do 251 m, przez Iseo przepływa rzeka Oglio}
\Clue{4}{}{kod ISO 4217 dolara amerykańskiego}
\Clue{14}{}{pogardliwie o osobie, która szpieguje i donosi w celu uzyskania jakichś korzyści}
\Clue{21}{}{małe pomieszczenie, które służy do przechowywania rzeczy}
\Clue{24}{}{mieszkanka Prus - dawnego państwa, jednego z kilku państw (o różnych ustrojach i nazwach na przestrzeni wieków) istniejących również na historycznych terenach Prus w okresie od 1226 do 1947; państwa, wokół którego odbyło się zjednoczenie Niemiec w XIX wieku}
\Clue{25}{}{chrząszcz z rodziny stonkowatych}
\Clue{26}{}{futro ze skóry tych zwierząt}
\Clue{27}{}{fikcyjny świat w kształcie dysku, w którym dzieje się akcja wieloksiągu fantasy autorstwa Terry'ego Pratchetta}
\Clue{28}{}{droga krajowa z Krakowa do Zakopanego, której długość wynosi 102 km, słynąca z dużego poziomu zatłoczenia i dużej liczby wypadków}
\Clue{29}{}{melodia, do której można tańczyć krakowiaka}
\Clue{30}{}{opiekunka, obrończyni}
\Clue{31}{}{malarz włoski (1240-przed 1313); obrazy, freski religijne}\end{PuzzleClues}

\begin{PuzzleClues}{\textbf{Pionowe}\\}\Clue{1}{}{właściwość czegoś, co występuje lub jest pomnożone 12 razy}
\Clue{2}{}{cienka, elastyczna skóra, służąca do okładania podeszwy wewnątrz obuwia i łącząca spód obuwia z jego wierzchem}
\Clue{3}{}{popiersie; rzeźba lub płaskorzeźba przedstawiająca górną część postaci}
\Clue{4}{}{mieszkanka Drohobycza}
\Clue{5}{}{pędzenie bydła, koni z danego miejsca na inne przez jakiś teren}
\Clue{6}{}{miasto i port w Indonezji na płd. wybrzeżu Jawy}
\Clue{7}{}{funkcja wzajemnie jednoznaczna przeprowadzająca przestrzeń rzutową na siebie i zachowująca współliniowość punktów}
\Clue{8}{}{Telophorus viridis - gatunek ptaka  z rodziny dzierzbików (Malaconotidae)}
\Clue{9}{}{pochwa osłaniająca korzeń pierwotny zarodka traw}
\Clue{10}{}{chrząszcz z rodziny łyszczynkowatych}
\Clue{11}{}{hydropłat; statek wodny z płatami nośnymi pod kadłubem; osiąga znaczne szybkości}
\Clue{12}{}{publiczna odezwa skierowana przez biskupa lub arcybiskupa do wiernych z nauką albo istotną informacją}
\Clue{13}{}{wielkość opisująca przyrost funkcji na danym przedziale}
\Clue{14}{}{sprawa, rzecz, która jest niezrozumiała, zawiła, trudna do pojęcia, zrozumienia, abstrakcyjna}
\Clue{15}{}{agatis nowozelandzki, soplica kauri, soplica australijska, Agathis australis - gatunek drzewa z rodziny araukariowatych; występuje na Wyspie Północnej w Nowej Zelandii}
\Clue{16}{}{przenikanie, wypełnianie czegoś,  np. przestrzeni, powietrza,  czymś, np. zapachem, wilgocią}
\Clue{17}{}{praktyka polityczna w starożytnej Grecji, rodzaj tajnego głosowania, podczas którego wolni obywatele typowali osoby podejrzane o dążenie do tyranii i zasługujące na wygnanie z miasta na 10 lat}
\Clue{18}{}{Dendrolagus inustus - gatunek ssaka z rodziny kangurowatych; zamieszkuje północną i zachodnią Nową Gwineę (od Vogelkop i półwyspu Fak Fak do północnego wybrzeża Papui-Nowej Gwinei)}
\Clue{19}{}{uroczysta wypowiedź pożegnalna i chwalebna, wygłoszona na pogrzebie pod adresem osoby zmarłej}
\Clue{20}{}{KANSU; prowincja w północnych Chinach, główne miasto Lanzhou, 20 2 mln mieszkańców, powierzchnia 530 tyś. km2}
\Clue{21}{}{jakieś zdarzenie, które odbywa się w obecności świadków, zwykle wstydliwe lub śmieszne}
\Clue{22}{}{antylopa zaliczana do gazeli; Afryka, Syria, płw. Arabski}
\Clue{23}{}{tłuszcz wytopiony z mięsa (np. gęsiego czy słoniny) z dodatkiem przypraw oraz często skwarków czy cebuli; używany do smarowania pieczywa}
\Clue{24}{}{śmierdziel; ameryk, drapieżnik z łasicowatych; cenne futro}
\Clue{25}{}{w chemii: symbol samaru}\end{PuzzleClues}\newpage\section*{Krzyżówka 154}

\noindent\begin{Puzzle}{23}{29}|*	|*	|*	|*	|*	|[1][S]\darr	|*	|*	|*	|*	|*	|*	|[2][S]\darr	|*	|[3][S]\drarr	|o	|t	|ą	|g	|*	|[4][S]\darr	|[5][S]\darr	|[6][S]\darr	|*	|.
|*	|[7][S]\rarr	|r	|a	|p	|p	|e	|*	|[8][S]\rarr	|t	|a	|r	|l	|a	|t	|a	|n	|*	|[9][S]\darr	|*	|b	|n	|p	|*	|.
|*	|*	|*	|*	|*	|r	|[10][S]\rarr	|c	|h	|o	|c	|h	|o	|ł	|e	|k	|*	|*	|g	|*	|a	|i	|o	|*	|.
|[11][S]\rarr	|o	|l	|b	|r	|z	|y	|m	|y	|*	|[12][S]\darr	|*	|k	|[13][S]\darr	|s	|[14][S]\drarr	|a	|r	|b	|i	|t	|e	|r	|*	|.
|*	|[15][S]\darr	|*	|[16][S]\darr	|*	|e	|*	|*	|*	|*	|w	|*	|a	|j	|t	|p	|*	|*	|u	|*	|a	|o	|t	|*	|.
|[17][S]\drarr	|b	|r	|z	|u	|s	|i	|e	|c	|*	|y	|*	|c	|ę	|o	|o	|*	|*	|r	|*	|l	|k	|u	|[18][S]\darr	|.
|k	|i	|*	|a	|[19][S]\rarr	|t	|y	|n	|i	|e	|c	|*	|j	|z	|w	|r	|[20][S]\darr	|*	|*	|*	|i	|r	|g	|k	|.
|r	|e	|[21][S]\darr	|i	|*	|r	|*	|*	|*	|[22][S]\darr	|i	|*	|a	|y	|a	|a	|z	|*	|[23][S]\darr	|*	|a	|e	|a	|a	|.
|e	|l	|d	|m	|*	|z	|*	|*	|*	|d	|r	|[24][S]\darr	|*	|k	|n	|ż	|i	|*	|d	|[25][S]\darr	|*	|ś	|l	|c	|.
|i	|i	|e	|e	|[26][S]\darr	|e	|*	|*	|*	|e	|u	|n	|*	|[][,]{ }	|i	|e	|e	|*	|z	|b	|[27][S]\darr	|l	|c	|z	|.
|s	|s	|t	|k	|k	|g	|*	|*	|*	|l	|c	|a	|*	|u	|e	|n	|l	|[28][S]\darr	|i	|r	|f	|o	|z	|k	|.
|l	|t	|e	|[][,]{ }	|a	|a	|*	|*	|[29][S]\darr	|i	|h	|l	|*	|r	|[][,]{ }	|i	|o	|a	|e	|u	|i	|n	|y	|a	|.
|e	|k	|k	|p	|t	|c	|*	|*	|s	|k	|*	|e	|*	|a	|w	|e	|n	|z	|ń	|t	|ń	|o	|k	|[][,]{ }	|.
|r	|a	|t	|r	|a	|z	|*	|*	|z	|w	|*	|ż	|*	|l	|z	|[][,]{ }	|y	|y	|[][,]{ }	|a	|s	|ś	|*	|k	|.
|*	|*	|o	|z	|f	|*	|*	|*	|y	|e	|*	|n	|*	|s	|o	|m	|[][,]{ }	|d	|ś	|l	|k	|ć	|*	|r	|.
|*	|*	|r	|y	|o	|*	|*	|[30][S]\darr	|b	|n	|*	|o	|[31][S]\rarr	|k	|r	|ó	|l	|e	|w	|n	|a	|*	|*	|z	|.
|*	|*	|[][,]{ }	|s	|r	|*	|*	|w	|l	|t	|*	|ś	|*	|i	|c	|z	|u	|k	|i	|o	|*	|*	|*	|y	|.
|*	|*	|a	|ł	|e	|*	|*	|i	|a	|k	|*	|ć	|*	|*	|o	|g	|d	|*	|ę	|ś	|[32][S]\darr	|*	|*	|ż	|.
|*	|[33][S]\rarr	|k	|o	|z	|i	|c	|z	|k	|a	|*	|[][,]{ }	|*	|*	|w	|o	|z	|*	|t	|ć	|u	|*	|*	|ó	|.
|*	|*	|t	|w	|a	|*	|*	|j	|*	|*	|*	|l	|*	|*	|e	|w	|i	|*	|y	|*	|s	|*	|*	|w	|.
|*	|*	|y	|n	|*	|*	|[34][S]\rarr	|a	|l	|u	|w	|i	|u	|m	|*	|e	|k	|[35][S]\darr	|*	|*	|t	|*	|*	|k	|.
|*	|*	|w	|y	|*	|[36][S]\darr	|*	|[][,]{ }	|*	|*	|*	|c	|*	|*	|*	|*	|*	|s	|*	|*	|r	|*	|*	|a	|.
|*	|*	|a	|*	|*	|d	|*	|l	|[37][S]\drarr	|n	|i	|e	|k	|o	|n	|k	|r	|e	|t	|n	|o	|ś	|ć	|*	|.
|*	|*	|c	|[38][S]\rarr	|t	|u	|t	|o	|r	|*	|*	|n	|*	|[39][S]\rarr	|m	|o	|s	|t	|*	|[40][S]\darr	|ń	|[41][S]\darr	|[42][S]\darr	|*	|.
|*	|*	|y	|*	|*	|s	|[43][S]\rarr	|k	|i	|e	|l	|c	|e	|*	|*	|*	|*	|o	|*	|d	|*	|f	|w	|*	|.
|*	|*	|j	|*	|*	|z	|*	|a	|f	|[44][S]\rarr	|r	|y	|s	|u	|n	|e	|k	|*	|*	|r	|*	|r	|u	|*	|.
|*	|*	|n	|*	|*	|a	|*	|l	|f	|*	|*	|j	|*	|*	|*	|*	|*	|*	|*	|y	|*	|y	|r	|*	|.
|*	|*	|y	|*	|*	|*	|*	|n	|*	|*	|*	|n	|*	|[45][S]\rarr	|m	|u	|t	|u	|a	|l	|i	|z	|m	|*	|.
|*	|*	|*	|*	|*	|[46][S]\rarr	|p	|a	|r	|a	|m	|a	|g	|n	|e	|t	|y	|z	|m	|*	|*	|*	|*	|*	|.
|*	|[47][S]\rarr	|w	|y	|ł	|ó	|w	|*	|*	|*	|*	|*	|*	|*	|*	|*	|*	|*	|*	|*	|*	|*	|*	|*	|.\end{Puzzle}

\newpage

\begin{PuzzleClues}{\textbf{Poziome}\\}\Clue{3}{}{CEREUS}
\Clue{7}{}{ur. w 1952 r., śpiewaczka (alt); wykonawczyni muzyki oratoryjnej i pieśni}
\Clue{8}{}{sztywna tkanina bawełniana lub z włókien sztucznych, bielona, barwiona lub drukowana, mocno krochmalona używana na teatralne kostiumy}
\Clue{10}{}{mały snopek, wiązka czegoś (najczęściej sznurków, gałązek, trawek), która przypomina kształtem chochoły do ochrony przed wilgocią i zimnem}
\Clue{11}{}{gwiazdy o dużych promieniach i jasnościach - kilkaset razy jaśniejsze i wielokrotnie większe od Słońca}
\Clue{14}{}{sędzia sportowy}
\Clue{17}{}{skupienie włókien mięśniowych, które tworzy typowe mięśnie szkieletowe wraz ze ścięgnami}
\Clue{19}{}{opactwo benedyktynów w Tyńcu w południowo-zachodniej części Krakowa}
\Clue{31}{}{córka króla}
\Clue{33}{}{młode kozicy}
\Clue{34}{}{osady powstające w procesie akumulacji na skutek działalności wód płynących}
\Clue{37}{}{cecha człowieka, który jest chaotyczny, nie podejmuje zdecydowanych działań, jest uległy}
\Clue{38}{}{mentor, opiekun przypisany do konkretnej osoby, np. w dużej firmie lub organizacji}
\Clue{39}{}{uczęszczana trasa, np. w handlu}
\Clue{43}{}{miasto na prawach powiatu w środkowo-wschodniej Polsce, stolica województwa świętokrzyskiego}
\Clue{44}{}{praca plastyczna, która została narysowana za pomocą narzędzi, które służą do rysowania}
\Clue{45}{}{doktryna złagodzonego socjalizmu, oparta na zasadzie wzajemności w przeciwstawianiu do walki konkurencyjnej}
\Clue{46}{}{zjawisko magnesowania się makroskopowego ciała w zewnętrznym polu magnetycznym w kierunku zgodnym z kierunkiem pola zewnętrznego}
\Clue{47}{}{rezultat czynności wyławiania}\end{PuzzleClues}

\begin{PuzzleClues}{\textbf{Pionowe}\\}\Clue{1}{}{człowiek, który przestrzega, pilnuje}
\Clue{2}{}{jakiś wyodrębniony obszar w niektórych grach komputerowych, część wykreowanej rzeczywistości}
\Clue{3}{}{test wydajności systemu komputerowego: sprzętu lub oprogramowania}
\Clue{4}{}{szyk bojowy piechoty w XV i XVI wieku}
\Clue{5}{}{to, że coś jest nieokreślone - trudne do opisania, nazwania, zinterpretowania}
\Clue{6}{}{mieszkaniec Portugalii, człowiek pochodzenia portugalskiego}
\Clue{9}{}{niemiły w obejściu człowiek, naburmuszony, aspołeczny, często też zarozumiały}
\Clue{12}{}{pogardliwie o prostytutce - kobiecie uprawiającej prostytucję, odbywającej stosunki płciowe w celach zarobkowych}
\Clue{13}{}{język z rodziny języków uralskich}
\Clue{14}{}{niepostępujący i nieustępujący zespół objawów ze strony ośrodkowego układu nerwowego, powodujący różnorodne zaburzenia psychomotoryczne oraz upośledzenie rozwoju umysłowego; przyczyną jest najczęściej niedotlenienie mózgu powstałe na skutek nieprawidłowości w czasie ciąży, przedwczesnego porodu, patologicznego przebiegu porodu i inne}
\Clue{15}{}{modrzaczek, Leucobryum - rodzaj mchów z rodziny bielistkowatych}
\Clue{16}{}{zaimek zastępujący przysłówek}
\Clue{17}{}{austriacki skrzypek wirtuoz i kompozytor (1875-1962); znane miniatury skrzypcowe}
\Clue{18}{}{krzyżówka, Anas platyrhynchos - gatunek dużego ptaka wodnego z rodziny kaczkowatych (Anatidae); zasiedla większą część półkuli północnej, poza tym została introdukowana do południowo-wschodniej Australii oraz na Nową Zelandię}
\Clue{20}{}{ufoludek, istota z innej planety}
\Clue{21}{}{rodzaj detektora promieniowania służący do pomiaru strumienia neutronów poprzez pomiar promieniotwórczości wywołanej aktywacją materii przez te neutrony}
\Clue{22}{}{interesantka}
\Clue{23}{}{niedziela}
\Clue{24}{}{każdy rodzaj opłaty za wykorzystywanie lub za możliwość wykorzystywania wszelkich praw autorskich do dzieła literackiego, artystycznego lub naukowego, także za korzystanie z doświadczenia zawodowego w przemyśle, handlu lub nauce}
\Clue{25}{}{cecha człowieka: to, że ktoś jest agresywny, dopuszcza się przemocy}
\Clue{26}{}{metoda pokrywania powierzchni, na przykład malowania lub zabezpieczania przed korozją, oparta na technologii elektrochemicznej, polegająca na katodowym lakierowaniu powierzchni metalowych w zanurzeniu w specjalnym roztworze przy jednoczesnym podłączeniu napięcia}
\Clue{27}{}{zatoka Morza Bałtyckiego, między Finlandią a Estonią, głębokość do 123 m, główne porty: Petersburg, Tallin, Helsinki}
\Clue{28}{}{sól lub związek organiczny zawierający grupę N3}
\Clue{29}{}{zarośla krzewów kolczastych w strefie przyśródziemnomorskiej, głównie Bułgarii i Jugosławii}
\Clue{30}{}{urzędowe oględziny terenu, lokalu itp. dla zbadania okoliczności zdarzenia lub stwierdzenia czegoś}
\Clue{32}{}{poetycko o miejscu na uboczu, z dala od ludzi, w ustroniu}
\Clue{35}{}{miasto w Japonii (środkowe Honsiu) na wsch. od Nagoi, ośrodek produkcji porcelany}
\Clue{36}{}{włókno; rdzeń w łodygach roślin i drzew}
\Clue{37}{}{w muzyce jazzowej: krótka, prosta, rytmiczna fraza melodyczna służąca za tło do improwizacji solowej}
\Clue{40}{}{Mandrillus leucophaeus - gatunek małpy wąskonosej z rodziny makakowatych, blisko spokrewniony z mandrylem, od którego różni się głównie jednolitym ubarwieniem twarzy; zamieszkuje zalesione tereny w Nigerii i Kamerunie oraz wyspę Bioko}
\Clue{41}{}{środkowy, poziomy człon belkowania z reguły leżący między architrawem i gzymsem, często zdobiony rzeźbami, jeden z bardziej ozdobnych elementów antycznych budowli}
\Clue{42}{}{STARNBERG}\end{PuzzleClues}\newpage\section*{Krzyżówka 155}

\noindent\begin{Puzzle}{19}{22}|*	|*	|*	|[1][S]\darr	|*	|*	|*	|*	|*	|*	|*	|*	|*	|*	|*	|[2][S]\darr	|*	|*	|[3][S]\darr	|*	|.
|*	|*	|*	|ż	|[4][S]\darr	|*	|*	|*	|*	|*	|*	|*	|*	|[5][S]\darr	|*	|r	|[6][S]\darr	|*	|p	|*	|.
|*	|[7][S]\rarr	|t	|a	|p	|e	|t	|a	|*	|*	|[8][S]\darr	|*	|*	|f	|*	|u	|a	|*	|l	|*	|.
|*	|[9][S]\drarr	|w	|ł	|o	|s	|i	|e	|ń	|*	|v	|*	|*	|t	|*	|c	|s	|*	|u	|*	|.
|*	|k	|*	|o	|s	|[10][S]\rarr	|t	|e	|r	|m	|i	|t	|*	|y	|*	|h	|e	|*	|w	|*	|.
|*	|l	|[11][S]\darr	|b	|z	|[12][S]\drarr	|r	|a	|t	|y	|s	|z	|c	|z	|e	|*	|s	|[13][S]\darr	|i	|*	|.
|*	|a	|o	|a	|y	|s	|[14][S]\rarr	|t	|u	|n	|e	|l	|*	|j	|[15][S]\darr	|*	|o	|w	|o	|*	|.
|*	|s	|g	|*	|c	|z	|[16][S]\rarr	|a	|r	|g	|u	|s	|*	|o	|h	|[17][S]\darr	|r	|e	|g	|[18][S]\darr	|.
|*	|z	|r	|*	|i	|y	|[19][S]\drarr	|s	|ą	|d	|*	|*	|*	|l	|o	|z	|[][,]{ }	|k	|r	|m	|.
|*	|t	|o	|[20][S]\rarr	|e	|k	|s	|h	|i	|b	|i	|c	|j	|o	|n	|i	|s	|t	|a	|*	|.
|*	|o	|d	|*	|*	|*	|t	|*	|*	|*	|*	|*	|[21][S]\darr	|g	|g	|e	|ą	|o	|m	|[22][S]\darr	|.
|*	|r	|n	|*	|[23][S]\rarr	|z	|e	|s	|t	|a	|w	|i	|k	|*	|z	|l	|d	|r	|*	|i	|.
|*	|*	|i	|[24][S]\rarr	|t	|e	|r	|y	|l	|e	|n	|*	|u	|*	|e	|e	|o	|*	|[25][S]\darr	|l	|.
|*	|[26][S]\drarr	|k	|o	|k	|i	|e	|t	|n	|i	|k	|*	|f	|*	|*	|n	|w	|*	|ł	|e	|.
|*	|e	|*	|[27][S]\rarr	|s	|p	|o	|d	|e	|k	|*	|*	|f	|*	|*	|i	|y	|[28][S]\darr	|a	|b	|.
|[29][S]\drarr	|k	|r	|o	|t	|n	|i	|c	|a	|*	|*	|*	|*	|*	|*	|c	|*	|p	|w	|o	|.
|l	|s	|*	|*	|*	|*	|z	|[30][S]\rarr	|i	|k	|o	|n	|o	|g	|r	|a	|f	|i	|a	|*	|.
|u	|c	|*	|*	|[31][S]\rarr	|k	|o	|n	|t	|y	|n	|g	|e	|n	|t	|*	|*	|e	|*	|*	|.
|n	|e	|*	|[32][S]\rarr	|s	|a	|m	|o	|t	|n	|i	|c	|z	|o	|ś	|ć	|*	|r	|*	|*	|.
|e	|s	|*	|[33][S]\rarr	|n	|i	|e	|p	|r	|e	|c	|y	|z	|y	|j	|n	|o	|ś	|ć	|*	|.
|t	|y	|*	|*	|[34][S]\rarr	|p	|r	|ó	|g	|[][,]{ }	|r	|z	|e	|c	|z	|n	|y	|*	|*	|*	|.
|a	|*	|[35][S]\rarr	|k	|o	|t	|*	|*	|*	|*	|*	|*	|*	|*	|*	|*	|*	|*	|*	|*	|.
|*	|*	|*	|*	|*	|*	|*	|*	|*	|*	|*	|*	|*	|*	|*	|*	|*	|*	|*	|*	|.\end{Puzzle}

\newpage

\begin{PuzzleClues}{\textbf{Poziome}\\}\Clue{7}{}{papierowe, dekoracyjne obicie ścian}
\Clue{9}{}{TRYCHINA; wicień pasożytujący w jelicie człowieka, świni gryzoni}
\Clue{10}{}{owad z rzędu termitów}
\Clue{12}{}{WŁÓCZNIA, PIKA}
\Clue{14}{}{rodzaj kolczyka, który wymaga dość szerokiej dziurki w ciele (zwykle w uchu) - wkłada się go do takiej dziurki, dzięki czemu skóra nie jest wiotka i uzyskuje się efekt interesująco wyglądającej, odpowiednio zaprezentowanej dziury na przestrzał w człowieku}
\Clue{16}{}{ptak z rodziny bażantów o długich lotkach i barwnym upierzeniu; występuje na Płw. Indochińskim oraz na Sumatrze i Borneo}
\Clue{19}{}{zdanie, opinia o czymś, ocena czegoś, pogląd, przekonanie}
\Clue{20}{}{człowiek, który obnaża się w miejscach publicznych}
\Clue{23}{}{zespół urządzeń (elementów), które powinny działać razem dla osiągnięcia efektu - często taki zestaw jest postrzegany jako całość, jedno urządzenie}
\Clue{24}{}{angielskie włókno poliestrowe, bardzo mocne, odporne na działanie temperatury, światła, kwasów, rozpuszczalników; używane do wyrobu tkanin odzieżowych i bieliźnianych oraz technicznych}
\Clue{26}{}{Phyllocladus - rodzaj drzew i krzewów z rodziny zastrzalinowatych (Podocarpaceae); obejmuje cztery gatunki występujące na obszarze od Filipin i Indonezji po Tasmanię i Nową Zelandię}
\Clue{27}{}{podstawka pod szklankę, filiżankę}
\Clue{29}{}{urządzenie teletransmisyjne umożliwiające jednoczesne przesyłanie wielu sygnałów jednym torem telekomunikacyjnym}
\Clue{30}{}{rzeźbiarz ur. w 1927 r., profesor ASP w Warszawie, rzeźba portretowa, pomniki, medale}
\Clue{31}{}{określona ilość czegoś (np. żywności), która przysługuje komuś}
\Clue{32}{}{stan opuszczenia, osamotnienia, izolacji}
\Clue{33}{}{brak precyzji, dokładności}
\Clue{34}{}{stopień skalny położony poprzecznie w korycie rzeki powstały w wyniku różnic w erozji skał, przez które przepływa rzeka}
\Clue{35}{}{żartobliwa nazwa kulki umieszczonej w niektórych klawiaturach pod najniższym rzędem klawiszy (która może też być osobnym urządzeniem) - pełni ona podobne funkcje jak myszka}\end{PuzzleClues}

\begin{PuzzleClues}{\textbf{Pionowe}\\}\Clue{1}{}{symbol żałoby, ciemny strój lub czarna wstążka przyszyta do ubrania}
\Clue{2}{}{zmiana położenia punktu materialnego w stosunku do innych punktów dokonująca się w czasie}
\Clue{3}{}{zapis pomiaru wykonanego pluwiografem}
\Clue{4}{}{pokrycie szkieletu samolotu wykonane z różnorakich materiałów}
\Clue{5}{}{lekarz specjalizujący się w dziedzinie chorób gruźliczych}
\Clue{6}{}{pracownik sądu, który zdał egzamin sędziowski, ale nie ma praktyki, pozwalającej na nominację sędziowską}
\Clue{8}{}{miasto w Portugalii, ośrodek administracyjny dystryktu Viseu; ośrodek turystyczny}
\Clue{9}{}{dom, w którym mieszkają zakonnicy lub zakonnice}
\Clue{11}{}{osoba, która uprawia ogród - sadzi i dba o warzywa, drzewa owocowe i rośliny ozdobne}
\Clue{12}{}{ustalony porządek ustawienia ludzi, zwierząt, pojazdów, wojsk}
\Clue{13}{}{wektorowa wielkość fizyczna - kierunkowa wielkość fizyczna, opisywana przez podanie bezwzględnej wartości liczbowej (modułu), kierunku, zwrotu i punktu przyłożenia}
\Clue{15}{}{jezioro w Chinach na Nizinie Chińskiej}
\Clue{17}{}{typ słodkowodnego glonu o dużym znaczeniu gospodarczym; pożywienie, surowiec do otrzymywania jodu itp}
\Clue{18}{}{skrót, symbol jednostki - metra}
\Clue{19}{}{izomer przestrzenny, związek chemiczny, w którym atomy połączone są w tych samych sekwencjach co w innym, ale różnią się jedynie ułożeniem przestrzennym}
\Clue{21}{}{używany od starożytności do czasów współczesnych okrągły pleciony z trzciny kosz, uszczelniany asfaltem lub smołą; służył m.in. do transportu na rzekach Eufrat i Tygrys}
\Clue{22}{}{miasto w Zairze, port nad rzeką Kasai}
\Clue{25}{}{sposób walki w szyku konnym przejęty od Tatarów przez jazdę polską}
\Clue{26}{}{ekstrawagancje, szum, pompa, które czemuś towarzyszą, cuda wianki}
\Clue{28}{}{gruczoł mlekowy u ludzi, trzeciorzędowa cecha płciowa, parzysty organ, który mają kobiety}
\Clue{29}{}{w obrabiarce: rodzaj podpórki używanej przy obróbce długich przedmiotów}\end{PuzzleClues}\newpage\section*{Krzyżówka 156}

\noindent\begin{Puzzle}{23}{28}|*	|*	|*	|*	|*	|*	|*	|*	|[1][S]\drarr	|s	|ę	|d	|z	|i	|a	|[][,]{ }	|k	|a	|l	|o	|s	|z	|*	|*	|.
|*	|[2][S]\rarr	|g	|r	|o	|c	|h	|o	|w	|a	|l	|s	|k	|*	|*	|*	|*	|*	|*	|[3][S]\darr	|*	|[4][S]\darr	|*	|*	|.
|*	|*	|*	|*	|*	|*	|[5][S]\rarr	|s	|a	|m	|o	|k	|r	|y	|t	|y	|c	|y	|z	|m	|*	|j	|*	|[6][S]\darr	|.
|*	|[7][S]\drarr	|k	|o	|d	|i	|e	|u	|m	|[][,]{ }	|p	|s	|t	|r	|e	|*	|*	|*	|*	|a	|*	|a	|*	|k	|.
|*	|s	|[8][S]\drarr	|g	|i	|m	|b	|o	|p	|a	|t	|r	|i	|o	|t	|a	|*	|*	|*	|ź	|*	|r	|[9][S]\darr	|e	|.
|*	|e	|r	|*	|*	|*	|*	|*	|i	|*	|*	|*	|*	|*	|*	|*	|[10][S]\darr	|*	|*	|n	|[11][S]\darr	|o	|l	|y	|.
|*	|k	|e	|[12][S]\rarr	|ż	|m	|i	|g	|r	|o	|d	|z	|i	|a	|n	|k	|a	|*	|[13][S]\darr	|i	|a	|s	|o	|b	|.
|*	|t	|m	|[14][S]\rarr	|n	|i	|e	|s	|z	|p	|u	|ł	|k	|a	|*	|*	|g	|*	|t	|c	|n	|z	|d	|o	|.
|*	|o	|i	|*	|*	|[15][S]\darr	|*	|*	|y	|*	|*	|[16][S]\darr	|*	|*	|*	|*	|e	|*	|o	|a	|o	|e	|o	|r	|.
|*	|r	|z	|*	|*	|g	|*	|*	|c	|*	|[17][S]\rarr	|p	|r	|ą	|d	|*	|n	|*	|m	|*	|n	|n	|w	|d	|.
|*	|*	|*	|*	|*	|o	|[18][S]\rarr	|p	|a	|n	|g	|o	|l	|i	|n	|[][,]{ }	|c	|h	|i	|ń	|s	|k	|i	|*	|.
|*	|[19][S]\drarr	|w	|i	|o	|s	|k	|a	|*	|*	|*	|s	|*	|*	|*	|*	|j	|*	|n	|*	|*	|o	|e	|*	|.
|*	|f	|[20][S]\rarr	|c	|a	|p	|*	|*	|*	|[21][S]\rarr	|i	|z	|o	|c	|y	|j	|a	|n	|i	|a	|n	|*	|c	|*	|.
|*	|l	|[22][S]\rarr	|s	|t	|o	|j	|k	|o	|*	|*	|u	|*	|*	|*	|*	|*	|*	|*	|*	|*	|*	|[][,]{ }	|*	|.
|*	|e	|*	|*	|*	|d	|*	|*	|*	|[23][S]\rarr	|k	|r	|y	|s	|z	|t	|a	|ł	|*	|*	|*	|*	|k	|*	|.
|[24][S]\drarr	|b	|o	|m	|b	|a	|r	|d	|i	|e	|r	|*	|*	|*	|*	|*	|*	|*	|*	|[25][S]\darr	|*	|*	|a	|*	|.
|m	|o	|[26][S]\rarr	|p	|o	|r	|u	|c	|z	|n	|i	|k	|o	|w	|a	|*	|*	|*	|*	|b	|*	|*	|r	|*	|.
|x	|g	|*	|*	|[27][S]\rarr	|s	|c	|e	|n	|a	|[][,]{ }	|p	|u	|d	|e	|ł	|k	|o	|w	|a	|*	|*	|o	|*	|.
|*	|r	|[28][S]\rarr	|m	|o	|t	|y	|w	|i	|k	|*	|*	|*	|*	|[29][S]\rarr	|p	|e	|s	|z	|k	|a	|*	|w	|*	|.
|*	|a	|*	|[30][S]\rarr	|e	|w	|a	|n	|g	|e	|l	|i	|a	|*	|*	|*	|*	|*	|*	|a	|*	|*	|y	|*	|.
|[31][S]\rarr	|m	|i	|k	|r	|o	|c	|j	|a	|*	|[32][S]\rarr	|c	|z	|y	|s	|z	|c	|z	|e	|n	|i	|e	|*	|*	|.
|*	|*	|*	|*	|*	|[][,]{ }	|*	|*	|*	|*	|*	|*	|*	|*	|*	|*	|*	|*	|*	|*	|*	|*	|*	|*	|.
|*	|[33][S]\darr	|*	|*	|[34][S]\rarr	|l	|a	|n	|d	|r	|y	|n	|k	|a	|*	|*	|*	|*	|*	|*	|*	|*	|*	|*	|.
|[35][S]\drarr	|z	|w	|ę	|ż	|e	|n	|i	|e	|[][,]{ }	|d	|w	|u	|n	|a	|s	|t	|n	|i	|c	|y	|*	|*	|*	|.
|d	|r	|*	|[36][S]\rarr	|o	|ś	|r	|o	|d	|e	|k	|[][,]{ }	|k	|u	|r	|a	|t	|o	|r	|s	|k	|i	|*	|*	|.
|u	|o	|*	|*	|[37][S]\rarr	|n	|e	|u	|r	|o	|c	|y	|b	|e	|r	|n	|e	|t	|y	|k	|*	|*	|*	|*	|.
|m	|s	|*	|[38][S]\rarr	|d	|e	|l	|i	|m	|i	|t	|a	|c	|j	|a	|*	|*	|*	|*	|*	|*	|*	|*	|*	|.
|a	|t	|*	|*	|*	|*	|*	|*	|*	|*	|*	|*	|*	|*	|*	|*	|*	|*	|*	|*	|*	|*	|*	|*	|.
|*	|*	|*	|*	|*	|*	|*	|*	|*	|*	|*	|*	|*	|*	|*	|*	|*	|*	|*	|*	|*	|*	|*	|*	|.\end{Puzzle}

\newpage

\begin{PuzzleClues}{\textbf{Poziome}\\}\Clue{1}{}{określenie sędziego, którego decyzje nie podobają się kibicom}
\Clue{2}{}{wieś w Polsce położona w województwie kujawsko-pomorskim, w powiecie lipnowskim, w gminie Dobrzyń nad Wisłą}
\Clue{5}{}{postawa krytyczna wobec samego siebie}
\Clue{7}{}{trójskrzyn pstry, kroton, Codiaeum variegatum - gatunek z rodziny wilczomleczowatych; popularna roślina doniczkowa}
\Clue{8}{}{człowiek, zwykle młody, który uznaje siebie samego za patriotę, intensywnie interesuje się pewnymi aspektami historii Polski i jednocześnie ma tendencję do zarzucania braku patriotyzmu wielu innym osobom; zazwyczaj ksenofob, mitoman, który jest dość agresywny w manifestowaniu swojego przywiązania do ojczyzny, podczas gdy jego wiedza o świecie jest powierzchowna}
\Clue{12}{}{mieszkanka Żmigrodu, kobieta pochodząca ze Żmigrodu}
\Clue{14}{}{krzew lub drzewo z rodziny różowatych, owoce brunatne, jadalne po przemarznięciu}
\Clue{17}{}{ruch cieczy, zwłaszcza wody}
\Clue{18}{}{pangolin pięciopalczasty, łuskowiec chiński, Manis pentadactyla - ssak łożyskowy zaliczany do łuskowców; żyje w lasach południowych Chin, Nepalu, Birmy, Laosu, Wietnamu i północnych Indii}
\Clue{19}{}{jednostka osadnicza o w miarę zwartej zabudowie, której mieszkańcy (znaczna ich część) trudnią się rolnictwem; skupisko ludzkie nieposiadające praw miejskich}
\Clue{20}{}{obleśny stary mężczyzna}
\Clue{21}{}{związek organiczny zawierający ugrupowanie -N=C=O}
\Clue{22}{}{kanadyjski, łyżwiarz figurowy, wicemistrz olimpijski z Lillehammer}
\Clue{23}{}{kryształ górski - bezbarwny, przezroczysty minerał}
\Clue{24}{}{strzelec bramki}
\Clue{26}{}{żona porucznika}
\Clue{27}{}{scena tworzona przez zamkniętą przestrzeń i oddzielona od widowni kurtyną}
\Clue{28}{}{zdrobniale: motyw - w języku młodzieżowym: sytuacja, zdarzenie}
\Clue{29}{}{malarz (1767-1831); portrety, kompozycje historyczne, krajobrazy}
\Clue{30}{}{przen. zespół nienaruszalnych zasad}
\Clue{31}{}{microtia - wada wrodzona, niecałkowity brak małżowiny usznej}
\Clue{32}{}{czynność prowadząca do usunięcia wszelkiego rodzaju zanieczyszczeń (m.in. brudu, odpadków)}
\Clue{34}{}{kobieta ubierająca się w róż i robiąca różowy makijaż}
\Clue{35}{}{grupa wad wrodzonych polegających na częściowym lub całkowitym zwężeniu światła dwunastnicy}
\Clue{36}{}{placówka o charakterze otwartym, która prowadzi w miejscu zamieszkania nieletnich działalność profilaktyczną, opiekuńczo-wychowawczą i resocjalizacyjno-terapeutyczną}
\Clue{37}{}{naukowiec, specjalista w dziedzinie neurocybernetyki}
\Clue{38}{}{rozpoznanie i oddzielenie jednych elementów wypowiedzi od drugich}\end{PuzzleClues}

\begin{PuzzleClues}{\textbf{Pionowe}\\}\Clue{1}{}{istota fantastyczna, żywiąca się ludzką krwią, prawie nieśmiertelna, mająca postać kobiety}
\Clue{3}{}{tyle, ile zmieści się w maźnicy - naczyniu do przechowywania smaru, smoły lub dziegciu}
\Clue{4}{}{ruch artystyczny z ok. 1895-1905; termin używany na określenie niemieckiej i austriackiej secesji}
\Clue{6}{}{elektryczny lub elektroniczny instrument klawiszowy, który pozwala grającemu na wykonywanie utworów z automatycznym akompaniamentem; zapis uproszczony nazwy instrumentu, spotykany często}
\Clue{7}{}{w informatyce: najmniejsza fizyczna jednostka zapisu danych na dyskach twardych, dyskietkach i innych nośnikach danych naśladujących dyski, której wielkość (ze względów historycznych) wynosi 512 bajtów; zapisywany i czytany zawsze w całości}
\Clue{8}{}{ptak zaroślowy podmokłych terenów, z rzędu wróblowatych, gniazdo workowate, wiszące; Europa, zach. Azja - chroniony}
\Clue{9}{}{lodowiec górski, który zajmuje tylko obszar karu}
\Clue{10}{}{instytucja rządowa lub pożytku publicznego działająca jako podmiot gospodarczy}
\Clue{11}{}{zapowiedź przybycia}
\Clue{13}{}{zatoka u wybrzeży Celebesu}
\Clue{15}{}{obszar lasu wraz z przynależnymi do niego urządzeniami komunikacyjnymi, transportowymi, technicznymi, zabudowaniami i innymi środkami produkcji przeznaczonymi do wykonywania w nim i za jego pośrednictwem działalności w dziedzinie produkcji leśnej}
\Clue{16}{}{podwodna część budowli wodnej}
\Clue{19}{}{zdjęcie rentgenowskie wykonane techniką flebografii, czyli z kontrastem}
\Clue{24}{}{jednostka strumienia indukcji magnetycznej w układzie CGS; 1 Ms = 1 Gs * cm2 = 10E-8 Wb = 10E-8 V*s}
\Clue{25}{}{BAKEN; pływający zakotwiczony znak nawigacyjny na wodach śródlądowych}
\Clue{33}{}{coś, co powstało z połączenia czegoś, np. kilku elementów, części itp}
\Clue{35}{}{coś, co napawa dumą}\end{PuzzleClues}\newpage\section*{Krzyżówka 157}

\noindent\begin{Puzzle}{24}{25}|*	|*	|*	|*	|*	|*	|*	|*	|*	|*	|*	|*	|*	|*	|[1][S]\darr	|*	|*	|[2][S]\darr	|*	|[3][S]\darr	|*	|[4][S]\darr	|*	|*	|*	|.
|*	|*	|*	|*	|*	|[5][S]\darr	|*	|*	|*	|*	|*	|[6][S]\darr	|*	|*	|s	|*	|*	|p	|*	|h	|*	|u	|*	|*	|*	|.
|*	|[7][S]\darr	|*	|*	|[8][S]\darr	|i	|*	|*	|[9][S]\rarr	|w	|ę	|g	|o	|r	|z	|[][,]{ }	|j	|a	|p	|o	|ń	|s	|k	|i	|*	|.
|*	|p	|*	|[10][S]\darr	|s	|n	|*	|[11][S]\rarr	|o	|r	|l	|e	|ń	|[][,]{ }	|p	|o	|s	|p	|o	|l	|i	|t	|y	|*	|*	|.
|*	|a	|[12][S]\rarr	|p	|i	|e	|c	|y	|k	|*	|*	|*	|*	|[13][S]\darr	|a	|[14][S]\darr	|*	|r	|*	|o	|*	|e	|[15][S]\darr	|*	|*	|.
|*	|r	|[16][S]\darr	|r	|m	|r	|[17][S]\darr	|*	|*	|*	|*	|*	|*	|m	|c	|t	|[18][S]\darr	|y	|*	|c	|*	|r	|s	|*	|*	|.
|*	|n	|ś	|z	|a	|t	|d	|*	|*	|*	|*	|[19][S]\darr	|*	|r	|l	|r	|p	|k	|*	|e	|*	|z	|i	|*	|*	|.
|*	|i	|p	|y	|*	|*	|r	|*	|*	|*	|*	|k	|*	|ó	|e	|z	|o	|a	|[20][S]\darr	|n	|*	|e	|ł	|*	|*	|.
|*	|k	|i	|n	|[21][S]\rarr	|k	|a	|r	|a	|t	|*	|o	|*	|w	|*	|a	|p	|r	|k	|*	|*	|n	|a	|*	|*	|.
|*	|[][,]{ }	|ą	|i	|*	|[22][S]\darr	|c	|*	|*	|*	|*	|p	|*	|k	|*	|s	|r	|z	|o	|*	|*	|i	|[][,]{ }	|*	|*	|.
|*	|w	|c	|e	|[23][S]\drarr	|c	|h	|o	|d	|z	|i	|e	|ż	|a	|n	|k	|a	|*	|r	|*	|*	|e	|o	|*	|*	|.
|[24][S]\rarr	|s	|y	|s	|t	|e	|m	|o	|w	|o	|ś	|ć	|*	|[][,]{ }	|*	|a	|w	|*	|o	|*	|*	|[][,]{ }	|d	|*	|*	|.
|*	|p	|[][,]{ }	|i	|e	|w	|a	|[25][S]\drarr	|c	|z	|k	|*	|*	|z	|*	|w	|n	|[26][S]\darr	|ł	|*	|[27][S]\darr	|m	|ś	|*	|*	|.
|*	|a	|k	|e	|r	|k	|*	|m	|*	|*	|[28][S]\rarr	|w	|ó	|ł	|*	|i	|o	|p	|a	|[29][S]\drarr	|f	|o	|r	|*	|*	|.
|*	|n	|r	|n	|a	|a	|*	|o	|*	|*	|*	|*	|[30][S]\darr	|o	|*	|c	|ś	|r	|z	|f	|i	|t	|o	|*	|*	|.
|*	|i	|ó	|i	|p	|*	|*	|n	|[31][S]\darr	|*	|*	|*	|d	|d	|*	|a	|ć	|o	|[][,]{ }	|i	|t	|y	|d	|*	|*	|.
|*	|a	|l	|e	|i	|*	|*	|t	|d	|*	|*	|*	|ź	|z	|*	|*	|[][,]{ }	|s	|r	|l	|o	|l	|k	|*	|*	|.
|*	|ł	|e	|*	|a	|*	|*	|e	|a	|[32][S]\rarr	|r	|e	|w	|i	|a	|*	|j	|e	|u	|a	|h	|k	|o	|*	|*	|.
|*	|y	|w	|*	|*	|*	|[33][S]\rarr	|u	|c	|z	|u	|c	|i	|e	|*	|*	|ę	|k	|d	|r	|o	|o	|w	|*	|*	|.
|*	|*	|i	|*	|*	|*	|*	|x	|h	|*	|*	|*	|g	|j	|*	|*	|z	|t	|o	|o	|r	|w	|a	|*	|*	|.
|*	|*	|c	|[34][S]\rarr	|k	|i	|l	|*	|*	|*	|*	|*	|n	|k	|*	|*	|y	|o	|b	|w	|m	|e	|*	|*	|*	|.
|*	|*	|z	|[35][S]\rarr	|f	|a	|k	|t	|o	|l	|o	|g	|i	|a	|*	|*	|k	|r	|r	|i	|o	|*	|*	|*	|*	|.
|*	|*	|*	|[36][S]\rarr	|e	|k	|s	|p	|l	|o	|z	|j	|a	|*	|*	|*	|o	|i	|e	|e	|n	|*	|*	|*	|*	|.
|*	|[37][S]\rarr	|t	|r	|a	|n	|s	|m	|i	|t	|e	|r	|*	|*	|*	|*	|w	|u	|w	|c	|*	|*	|*	|*	|*	|.
|[38][S]\rarr	|ż	|ó	|ł	|t	|o	|o	|k	|[][,]{ }	|z	|ł	|o	|t	|o	|p	|l	|a	|m	|y	|*	|*	|*	|*	|*	|*	|.
|*	|*	|*	|[39][S]\rarr	|k	|o	|n	|t	|o	|[][,]{ }	|w	|i	|n	|i	|e	|n	|*	|*	|*	|*	|*	|*	|*	|*	|*	|.\end{Puzzle}

\newpage

\begin{PuzzleClues}{\textbf{Poziome}\\}\Clue{9}{}{Anguilla japonica - ryba z rodziny węgorzowatych (Anguillidae); występuje w zlewisku Morza Japońskiego: Japonii, Tajwanie, Korei, Chinach, a także północnych Filipinach}
\Clue{11}{}{skrzydlak pospolity, orleń, skrzydlak, Myliobatis aquila - gatunek ryby chrzęstnoszkieletowej z rodziny orleniowatych (Myliobatidae); ryba ta występuje we wschodniej części Atlantyku, w Morzu Śródziemnym, u brzegów północno-zachodniej Francji, w pobliżu Anglii, Szkocji i u wybrzeży południowej Norwegii, także w Oceanie Indyjskim w pobliżu Afryki Południowej}
\Clue{12}{}{elektryczny przyrząd grzejny do ogrzewania pomieszczeń}
\Clue{21}{}{jednostka masy używana w jubilerstwie do określania wielkości kamieni i pereł; 1 karat = 0,2 g = 2 · 10-4 kg}
\Clue{23}{}{mieszkanka Chodzieży, kobieta pochodząca z Chodzieży}
\Clue{24}{}{zgodność z regułami systemu, zależność od zasad funkcjonowania systemu}
\Clue{25}{}{kod ISO 4217 korony czeskiej}
\Clue{28}{}{wytrzebiony samiec bydła domowego}
\Clue{29}{}{w piłce nożnej: płaskie, prostopadłe podanie do biegnącego partnera}
\Clue{32}{}{przedstawienie, pokaz np. tańca, jazdy na lodzie, piosenki, często z bardzo bogatą oprawą muzyczną, artystyczną czy dekoracjami, mające zapewnić rozrywkę widzom}
\Clue{33}{}{miłość, gorąca sympatia do kogoś}
\Clue{34}{}{stępka}
\Clue{35}{}{nauka zajmująca się faktami, badająca naturę faktów, pewników}
\Clue{36}{}{dużo silnych, nagłych wrażeń}
\Clue{37}{}{przekaźnik, który transmituje sygnały (np. obraz, dźwięk) z wykorzystaniem fal elektromagnetycznych}
\Clue{38}{}{Metriopelia aymara - gatunek ptaka z rodziny gołębiowatych (Columbidae)}
\Clue{39}{}{lewa strona konta księgowego}\end{PuzzleClues}

\begin{PuzzleClues}{\textbf{Pionowe}\\}\Clue{1}{}{niewielkie kluseczki przygotowywane z mąki i jajek}
\Clue{2}{}{porcja paprykarza, tj. paprykowego pasztetu na kanapkę; określona ilość tego produktu, zazwyczaj puszka lub plastikowy kubeczek}
\Clue{3}{}{najmłodsza epoka geologiczna, która trwa do dziś}
\Clue{4}{}{rodzaj niekonwencjonalnego usterzenia samolotu, w którym tradycyjny poziomy lewy i prawy ster wysokości oraz statecznik pionowy ze sterem kierunku zastąpione są przez tylko dwa stateczniki z powierzchniami sterowymi, ustawione pod kątem w stosunku do  siebie, w kształcie literyV patrząc z przodu samolotu}
\Clue{5}{}{obojętny składnik układu nie biorący udziału w reakcji chemicznej}
\Clue{6}{}{w chemii: symbol germanu}
\Clue{7}{}{Disporum sessile - gatunek rośliny zielnej z rodziny zimowitowatych}
\Clue{8}{}{MAZU; gatunek łososia pacyficznego o długości do 80 cm}
\Clue{10}{}{niosąc dostarczyć coś do określonego miejsca}
\Clue{13}{}{Solenopsis fugax - gatunek mrówki z podrodziny wścieklic; jest małą mrówką, często mieszkającą blisko lub wewnątrz gniazd innych gatunków mrówek i stosującą lestobiozę; w przypadku budowy samodzielnego gniazda pożywienie dostarczają jej mszyce korzeniowe}
\Clue{14}{}{błyskawica i grzmot}
\Clue{15}{}{jedna z sił bezwładności występująca w obracających się układach odniesienia}
\Clue{16}{}{mężczyzna, który jest mało energiczny i powolny}
\Clue{17}{}{waluta Grecji przed wprowadzeniem euro}
\Clue{18}{}{zgodność tekstu z normą językową}
\Clue{19}{}{biolog (1888-1941); prace zakresu genetyki; zamordowany przez hitlerowców}
\Clue{20}{}{Climacteris erythrops - gatunek ptaka z rodziny korołazów (Climacteridae) występujący w Australii}
\Clue{22}{}{organ, struktura anatomiczna, przewód, który jest zazwyczaj  wypełniony płynem}
\Clue{23}{}{działanie medyczne lub paramedyczne, mające na celu leczenie, pomoc choremu}
\Clue{25}{}{dyrygent francuski (1875-1964) prowadził m.in. Metropolitan Opera, orkiestry symfoniczne w Bostonie, Paryżu}
\Clue{26}{}{oddział szpitala, uczelni medycznej lub weterynaryjnej albo instytutu badawczego (z dziedziny nauk biomedycznych), w którym dokonywane są sekcje zwłok zmarłych osób lub zwierząt}
\Clue{27}{}{hormon roślinny, bioregulator roślinny - związek organiczny należący do regulatorów wzrostu i rozwoju roślin}
\Clue{29}{}{górnik, który pracuje przy filarach, nietkniętych częściach górotworu, podtrzymujących strop}
\Clue{30}{}{czynnik sprawczy, coś, co inicjuje lub popycha jakiś proces do przodu, wywiera wpływ}
\Clue{31}{}{górna, najwyższa część budynku, mająca za zadanie przykrycie i osłanianie go przed opadami atmosferycznymi}\end{PuzzleClues}\newpage\section*{Krzyżówka 158}

\noindent\begin{Puzzle}{20}{22}|*	|*	|*	|[1][S]\drarr	|w	|i	|d	|z	|e	|n	|i	|e	|[][,]{ }	|b	|a	|r	|w	|n	|e	|*	|*	|.
|*	|[2][S]\rarr	|p	|o	|m	|a	|r	|a	|ń	|c	|z	|ó	|w	|k	|a	|*	|[3][S]\drarr	|m	|i	|t	|*	|.
|*	|*	|*	|l	|*	|*	|*	|*	|[4][S]\rarr	|a	|n	|g	|s	|t	|r	|o	|m	|*	|*	|[5][S]\darr	|*	|.
|*	|*	|*	|s	|[6][S]\drarr	|k	|a	|n	|d	|y	|*	|[7][S]\rarr	|g	|a	|ś	|n	|i	|k	|*	|w	|[8][S]\darr	|.
|*	|*	|[9][S]\drarr	|z	|d	|a	|n	|i	|e	|*	|*	|[10][S]\rarr	|t	|r	|a	|s	|e	|r	|*	|i	|c	|.
|*	|*	|w	|y	|i	|*	|*	|*	|[11][S]\darr	|*	|[12][S]\drarr	|ł	|a	|ń	|c	|u	|c	|h	|*	|m	|z	|.
|*	|*	|a	|n	|a	|[13][S]\rarr	|t	|u	|s	|z	|k	|a	|*	|*	|*	|*	|h	|*	|*	|p	|a	|.
|*	|*	|ł	|a	|l	|[14][S]\drarr	|c	|z	|e	|r	|w	|o	|n	|i	|e	|c	|*	|[15][S]\darr	|*	|e	|s	|.
|*	|[16][S]\darr	|[][,]{ }	|*	|o	|ł	|*	|*	|u	|[17][S]\rarr	|a	|m	|p	|e	|l	|o	|z	|a	|u	|r	|*	|.
|*	|p	|b	|[18][S]\darr	|g	|u	|[19][S]\rarr	|g	|l	|o	|s	|a	|l	|i	|a	|*	|*	|d	|*	|g	|*	|.
|*	|r	|r	|h	|i	|s	|*	|*	|*	|*	|[][,]{ }	|*	|*	|[20][S]\rarr	|k	|i	|n	|i	|n	|a	|*	|.
|*	|ó	|z	|a	|z	|k	|*	|*	|[21][S]\darr	|[22][S]\rarr	|p	|*	|*	|[23][S]\rarr	|u	|t	|r	|u	|d	|*	|*	|.
|[24][S]\drarr	|g	|e	|r	|m	|a	|n	|i	|s	|t	|a	|*	|[25][S]\darr	|[26][S]\darr	|[27][S]\darr	|*	|*	|n	|[28][S]\darr	|*	|*	|.
|d	|*	|g	|n	|*	|*	|*	|*	|*	|*	|n	|*	|g	|g	|s	|*	|*	|k	|r	|*	|*	|.
|*	|*	|o	|e	|[29][S]\rarr	|b	|e	|ł	|k	|o	|t	|*	|a	|a	|u	|*	|*	|t	|u	|*	|*	|.
|[30][S]\rarr	|z	|w	|i	|e	|r	|a	|c	|z	|*	|o	|*	|z	|l	|d	|*	|[31][S]\darr	|*	|l	|*	|*	|.
|*	|[32][S]\darr	|y	|t	|*	|*	|[33][S]\rarr	|c	|e	|n	|t	|r	|a	|l	|a	|*	|o	|[34][S]\darr	|a	|*	|*	|.
|*	|o	|*	|*	|*	|[35][S]\rarr	|s	|z	|u	|t	|e	|r	|*	|e	|ń	|*	|l	|r	|d	|*	|*	|.
|*	|b	|*	|[36][S]\rarr	|n	|a	|m	|i	|o	|t	|n	|i	|k	|*	|c	|*	|i	|e	|a	|*	|*	|.
|[37][S]\rarr	|a	|m	|n	|e	|z	|j	|a	|[][,]{ }	|p	|o	|u	|r	|a	|z	|o	|w	|a	|*	|*	|*	|.
|[38][S]\rarr	|n	|a	|w	|a	|*	|[39][S]\rarr	|p	|r	|a	|w	|o	|*	|*	|y	|*	|k	|l	|*	|*	|*	|.
|*	|*	|[40][S]\rarr	|f	|a	|n	|t	|a	|s	|t	|y	|k	|a	|*	|k	|*	|a	|*	|*	|*	|*	|.
|*	|*	|[41][S]\rarr	|p	|o	|m	|y	|ł	|k	|a	|*	|*	|*	|*	|*	|*	|*	|*	|*	|*	|*	|.\end{Puzzle}

\newpage

\begin{PuzzleClues}{\textbf{Poziome}\\}\Clue{1}{}{zdolność organizmu lub maszyny do rozróżniania przedmiotów oparta na wrażliwości na długość fali (lub częstotliwość) światła, które przedmioty te odbijają, emitują lub przepuszczają}
\Clue{2}{}{nalewka na pomarańczach}
\Clue{3}{}{symbol, narracja funkcjonujące w świadomości danej społeczności; znaczenie przypisywane czemuś przez daną społeczność, szerzone np. przez środki masowego przekazu}
\Clue{4}{}{Anders, ur w 1814r. fizyk i astronom szwedzki; jeden z twórców astrofizyki, prace dotyczące analizy widmowej}
\Clue{6}{}{miasto w Sri Lance nad rzeką Mahaweli, ośrodek adm. Prowincji Centralnej}
\Clue{7}{}{przyrząd służący do gaszenia i zapalania świec; ma postać drewnianego kija odpowiedniej długości, na którego szczycie znajduje się stożkowy metalowy kołpak}
\Clue{9}{}{uporządkowany ciąg wyrazów służący zazwyczaj do zakomunikowania jakiejś treści}
\Clue{10}{}{robotnik kreślący na prefabrykatach wyrobów linie, wzdłuż których ma być prowadzona obróbka, cięcie i wiercenie otworów na nity oraz nitowanie}
\Clue{12}{}{ciąg co najmniej dwóch połączonych ze sobą atomów, które tworzą związek chemiczny lub część związku}
\Clue{13}{}{sztuka ubitego drobiu (rzadziej: innej upolowanej zwierzyny drobnej, np. królika lub zająca)}
\Clue{14}{}{radziecki banknot o wartości 10 rubli}
\Clue{17}{}{Ampelosaurus - rodzaj opancerzonego zauropoda z rodziny tytanozaurów; żył w epoce późnej kredy na terenach współczesnej Europy}
\Clue{19}{}{bezsensowne skojarzenia w przyśpiewkach}
\Clue{20}{}{małocząsteczkowy polipeptyd, hormon tkankowy}
\Clue{22}{}{puaz - jednostka lepkości dynamicznej w układzie jednostek miar CGS, nazwana na cześć francuskiego fizyka i lekarza Jeana L. M. Poiseuille'a. 1 P = 1 dyn·s/cm2 = 1 g·/(cm·s)}
\Clue{23}{}{utrudzenie}
\Clue{24}{}{ekspert w dziedzinie germanistyki}
\Clue{29}{}{niewyraźna mowa spowodowana jakimś utrudnieniem w wypowiadaniu słów}
\Clue{30}{}{mięsień, który tworzy pierścień mięśniowy dookoła tkanek}
\Clue{33}{}{siedziba centrali, główny ośrodek czegoś}
\Clue{35}{}{tłuczeń drogowy   }
\Clue{36}{}{TASIK; drobny motyl nocny, gąsienica żeruje na jabłoni, szkodnik}
\Clue{37}{}{zaburzenie pamięci powstałe w wyniku uszkodzenia mózgu, trwające od momentu urazu (utraty przytomności) do chwili świadomego zapamiętywania zdarzeń bieżących}
\Clue{38}{}{część kościoła przeznaczona dla wiernych zawarta między prezbiterium a kruchtą}
\Clue{39}{}{norma prawna, zasada}
\Clue{40}{}{dziedzina literatury, którą cechuje duża swoboda w kreowaniu świata przedstawionego}
\Clue{41}{}{postępowanie o negatywnych konsekwencjach, np. zły wybór, niewłaściwa decyzja}\end{PuzzleClues}

\begin{PuzzleClues}{\textbf{Pionowe}\\}\Clue{1}{}{las olchowy (olszowy) porastający żyzne, bagienne siedliska, o wysokim poziomie wody stojącej}
\Clue{3}{}{składające się ścianki aparatu lub obiektywu, najczęściej w starych aparatach fotograficznych}
\Clue{5}{}{w architekturze: dekoracyjne wykończenie w kształcie wysokiego trójkąta wieńczące szczyt portalu albo ostrołuk okna, które jest charakterystyczne dla architektury gotyckiej oraz stosowane w zdobnictwie neogotyckim}
\Clue{6}{}{figura retoryczna polegająca na wpleceniu w wypowiedź cytatu,   krótkiego dialogu, sentencji lub czyichś myśli}
\Clue{8}{}{wolne chwile, często kiedy nie ma się zaplanowanych innych zajęć, czas do dyspozycji, czas jako własność człowieka}
\Clue{9}{}{niski wał ograniczający koryto rzeki i równoległy do jej przebiegu, położony nieco wyżej niż równina zlewowa; powstaje podczas powodzi w wyniku sedymentacji grubego materiału niesionego przez rzekę}
\Clue{11}{}{jezioro w Kanadzie w prowincji Ontario}
\Clue{12}{}{organiczny związek chemiczny, amid kwasu pantoinowego i ß-alaniny}
\Clue{14}{}{kostna lub rogowa płytka wraz z innymi łuskami tworząca pokrywę ciała wielu zwierząt i pełniąca zazwyczaj funkcje obronne; występuje u ryb, gadów, ptaków i niektórych bezkręgowców (np. u motyli)}
\Clue{15}{}{tytuł zawodowy nadawany pracownikom służby bibliotecznej, leśnictwa i muzealnictwa}
\Clue{16}{}{przenośnie coś, co stanowi przeszkodę, symbolicznie dzieli}
\Clue{18}{}{torowiec kanadyjski, brązowy medalista z Atlanty w sprincie}
\Clue{21}{}{w chemii: symbol siarki}
\Clue{24}{}{jeden z języków programowania}
\Clue{25}{}{chłonny materiał opatrunkowy, produkowany najczęściej z wybielonej bawełny, w postaci rzadkiej tkaniny, następnie zwykle wyjałowiony}
\Clue{26}{}{astronom niemiecki (1812-1855); odkrył Neptuna}
\Clue{27}{}{mieszkaniec Sudanu, człowiek pochodzenia sudańskiego}
\Clue{28}{}{chwyt w zapasach}
\Clue{31}{}{drewno pozyskiwane z oliwki europejskiej}
\Clue{32}{}{miasto w płn. Szkocji nad Zatoką Firth of Lorn}
\Clue{34}{}{historyczna moneta (głównie srebrna) bita od XIV wieku w Hiszpanii, a następnie w Portugalii}\end{PuzzleClues}\newpage\section*{Krzyżówka 159}

\noindent\begin{Puzzle}{21}{25}|*	|*	|*	|*	|[1][S]\darr	|*	|[2][S]\darr	|[3][S]\darr	|[4][S]\darr	|*	|*	|*	|*	|*	|*	|*	|[5][S]\darr	|*	|*	|*	|[6][S]\darr	|*	|.
|*	|*	|[7][S]\darr	|*	|m	|*	|k	|p	|ś	|*	|*	|*	|*	|*	|*	|*	|p	|*	|[8][S]\darr	|*	|k	|*	|.
|*	|[9][S]\rarr	|p	|i	|a	|*	|a	|r	|w	|[10][S]\drarr	|m	|e	|z	|o	|t	|e	|r	|a	|p	|i	|a	|*	|.
|*	|*	|e	|*	|j	|[11][S]\darr	|t	|e	|i	|t	|[12][S]\drarr	|k	|ó	|ł	|e	|c	|z	|k	|o	|*	|u	|*	|.
|*	|*	|r	|*	|*	|k	|a	|z	|a	|o	|g	|*	|[13][S]\darr	|*	|*	|*	|y	|[14][S]\darr	|p	|*	|c	|*	|.
|*	|[15][S]\darr	|ł	|*	|*	|a	|t	|e	|t	|r	|ę	|*	|o	|[16][S]\rarr	|j	|a	|s	|s	|y	|*	|z	|*	|.
|*	|k	|o	|*	|*	|r	|o	|n	|ł	|o	|ś	|*	|s	|*	|[17][S]\darr	|[18][S]\darr	|m	|z	|t	|[19][S]\darr	|u	|*	|.
|*	|a	|r	|[20][S]\drarr	|f	|a	|n	|t	|o	|m	|*	|*	|c	|*	|t	|m	|a	|k	|[][,]{ }	|a	|k	|*	|.
|*	|r	|ó	|m	|*	|b	|i	|a	|ś	|i	|*	|*	|h	|*	|u	|o	|k	|o	|i	|n	|o	|*	|.
|*	|l	|d	|y	|*	|o	|a	|*	|ć	|s	|[21][S]\rarr	|ł	|a	|w	|r	|a	|*	|ł	|n	|a	|w	|*	|.
|*	|i	|k	|s	|[22][S]\darr	|n	|*	|*	|*	|t	|[23][S]\drarr	|g	|i	|l	|a	|k	|*	|a	|w	|c	|i	|*	|.
|*	|k	|o	|z	|s	|*	|[24][S]\darr	|*	|*	|r	|z	|*	|z	|*	|w	|[][,]{ }	|*	|[][,]{ }	|e	|h	|e	|*	|.
|*	|[][,]{ }	|w	|a	|z	|[25][S]\rarr	|p	|ę	|c	|z	|a	|k	|*	|*	|a	|c	|*	|ś	|s	|r	|c	|*	|.
|[26][S]\drarr	|w	|a	|k	|a	|y	|a	|m	|a	|*	|m	|*	|*	|*	|*	|i	|*	|r	|t	|o	|*	|*	|.
|h	|i	|t	|[][,]{ }	|l	|*	|d	|*	|*	|[27][S]\rarr	|s	|ł	|ó	|j	|*	|ę	|*	|e	|y	|n	|*	|*	|.
|i	|ę	|e	|l	|*	|*	|o	|[28][S]\rarr	|u	|r	|z	|ą	|d	|*	|*	|ż	|*	|d	|c	|i	|*	|*	|.
|p	|k	|*	|e	|[29][S]\drarr	|s	|k	|r	|a	|j	|*	|[30][S]\rarr	|d	|i	|a	|k	|*	|n	|y	|c	|*	|*	|.
|s	|s	|*	|ś	|d	|[31][S]\darr	|*	|*	|*	|*	|*	|[32][S]\darr	|*	|*	|*	|i	|*	|i	|j	|z	|*	|*	|.
|o	|z	|[33][S]\darr	|n	|i	|w	|*	|[34][S]\drarr	|n	|i	|l	|s	|s	|o	|n	|*	|*	|a	|n	|n	|*	|*	|.
|m	|y	|d	|y	|s	|s	|*	|b	|[35][S]\rarr	|d	|i	|u	|g	|o	|n	|i	|e	|*	|y	|o	|*	|*	|.
|e	|*	|u	|*	|n	|a	|*	|*	|*	|*	|[36][S]\rarr	|m	|u	|s	|z	|t	|r	|a	|*	|ś	|*	|*	|.
|t	|[37][S]\drarr	|n	|i	|e	|d	|y	|p	|l	|o	|m	|a	|t	|y	|c	|z	|n	|o	|ś	|ć	|*	|*	|.
|r	|y	|g	|*	|y	|*	|*	|[38][S]\rarr	|w	|a	|l	|t	|o	|r	|n	|i	|a	|*	|*	|*	|*	|*	|.
|i	|b	|*	|*	|*	|[39][S]\rarr	|d	|r	|z	|e	|w	|o	|[][,]{ }	|ś	|w	|i	|a	|t	|a	|*	|*	|*	|.
|a	|*	|*	|*	|*	|*	|*	|*	|[40][S]\rarr	|m	|e	|r	|e	|n	|g	|a	|*	|*	|*	|*	|*	|*	|.
|*	|[41][S]\rarr	|s	|t	|r	|o	|j	|n	|i	|c	|a	|*	|*	|*	|*	|*	|*	|*	|*	|*	|*	|*	|.\end{Puzzle}

\newpage

\begin{PuzzleClues}{\textbf{Poziome}\\}\Clue{9}{}{jednostka zdawkowa w Mjanmie; 1/100 kyata}
\Clue{10}{}{działający miejscowo niechirurgiczny zabieg medyczny polegający na dostarczeniu bezpośrednio do skóry właściwej substancji leczniczych, regenerujących lub odżywczych}
\Clue{12}{}{małe kółko - coś w kształcie koła (rzecz, plama)}
\Clue{16}{}{lasi}
\Clue{20}{}{model części ciała, zwykle naturalnej wielkości, używany do demonstracji anatomii lub do ćwiczeń w wykonywaniu zabiegów chirurgicznych, położniczych i w udzielaniu pierwszej pomocy}
\Clue{21}{}{zespół prawosławnych zabudowań klasztornych}
\Clue{23}{}{ptak z rodziny ziarnojadów występujący na obszarze Azji i płn. Afryki}
\Clue{25}{}{PĘCAK; kasza z całych ziaren jęczmienia lub pszenicy}
\Clue{26}{}{prefektura w środkowym Honsiu (Japonia)}
\Clue{27}{}{opakowanie o pojemności co najmniej 50 cm3}
\Clue{28}{}{instytucja, organ władzy publicznej}
\Clue{29}{}{końcowa część jakiejś powierzchni}
\Clue{30}{}{psalmista}
\Clue{34}{}{Birgit, ur. w 1918 r., śpiewaczka szwedzka (sopran); uznana odtwórczyni dzieł R. Wagnera}
\Clue{35}{}{diugoniowate, Dugongidae - rodzina morskich ssaków łożyskowych z rzędu syren; jedyny żyjący współcześnie gatunek tej rodziny - diugoń (Dugong dugon) - występuje w wodach tropikalnych u wybrzeża wschodniej Afryki, Azji, Australii i Nowej Gwinei, natomiast krowa morska (†Hydrodamalis gigas) została wytępiona w XVIII wieku}
\Clue{36}{}{ćwiczenia wojskowe mające na celu sprawne wykonywanie rozkazów przez żołnierzy}
\Clue{37}{}{to, że ktoś jest nietaktowny, nie jest dobrym dyplomatą, nie potrafi się odpowiednio zachować, wypowiadać}
\Clue{38}{}{RÓG; instrument dęty, blaszany, mający kształt długiej, wąskiej rury zwiniętej w trzy kręgi}
\Clue{39}{}{drzewo obecne w rozmaitych wierzeniach, które ma być filarem świata i symbolem jego równowagi}
\Clue{40}{}{ciastko z białka ubitego na pianę i syropu, przełożone kremem lub konfiturą}
\Clue{41}{}{część ramy w harfie, na której znajdują się kołki przytrzymujące struny}\end{PuzzleClues}

\begin{PuzzleClues}{\textbf{Pionowe}\\}\Clue{1}{}{według kalendarza gregoriańskiego - piąty miesiąc roku; maj ma 31 dni}
\Clue{2}{}{występowanie zwiększonej lub zmniejszonej aktywności ruchowej}
\Clue{3}{}{w prawie kanonicznym}
\Clue{4}{}{promieniowanie widzialne, odbierane przez oko człowieka, a wydzielane np. przez słońcę, urządzenia elektryczne, ogień; światło, jasność}
\Clue{5}{}{coś nadzwyczaj dobrego, smacznego, coś, co ktoś szczególnie lubi jeść}
\Clue{6}{}{Hevea Aubl. - wiecznie zielone drzewo z rodziny wilczomleczowatych, z którego soków (lateksu) wytwarza się kauczuk naturalny}
\Clue{7}{}{rodzina słodkowodnych małży z rzędu Palaeoheterodonta}
\Clue{8}{}{wielkość zamierzonych wydatków poniesionych przez przedsiębiorstwa na powiększenie stanu kapitału rzeczowego oraz stanu zapasów}
\Clue{10}{}{robotnik, który czuwa nad stanem torów kolejowych na określonym odcinku i nadzoruje pracę pracowników torowych}
\Clue{11}{}{zaprzęgowy pojazd wieloosobowy przeznaczony do dalekich podróży, z  krytym nadwoziem, używany w Polsce przez szlachtę w XVII-XVIII w}
\Clue{12}{}{potrawa z mięsa gęsiego}
\Clue{13}{}{miasto w Niemczech (Saksonia) w pobliżu Lipska}
\Clue{14}{}{szkoła, po której ukończeniu otrzymuje się świadectwo dojrzałości oraz prawo do studiowania na uczelni wyższej}
\Clue{15}{}{Pipistrellus nathusii - gatunek ssaka z rzędu nietoperzy; występuje w całej Europie z wyjątkiem Półwyspu Iberyjskiego, na północ sięgając Wysp Brytyjskich i południowej części Skandynawii, w Polsce rozmieszczony jest bardzo nieregularnie}
\Clue{17}{}{wieś w województwie opolskim (powiat opolski), położona na północ od Opola, nad Małą Panwią i Jeziorem Turawskim}
\Clue{18}{}{Eurapteryx gravis - gatunek wymarłego ptaka nielota z rodziny moaków (Emeidae)}
\Clue{19}{}{niewłaściwość chronologiczna, niezgodność z rzeczywistymi stosunkami czasowymi}
\Clue{20}{}{Peromyscus maniculatus - ssak z rodziny chomikowatych, występujący w Ameryce Północnej od Alaski po Meksyk}
\Clue{22}{}{pas materiału, który nosi się na szyi, głowie lub ramionach}
\Clue{23}{}{tkanina (zwykle wełniana, choć zdarzają się też sztuczne) imitująca zamsz, posiadająca lekki meszek}
\Clue{24}{}{boisko do gry w rugby}
\Clue{26}{}{dział geodezji, zajmujący się pomiarami wysokości terenu i przedstawianiem wyników w postaci map, wykresów, profilów lub modeli}
\Clue{29}{}{cały dorobek (czyli gł. kinematografia) wytwórni Disney}
\Clue{31}{}{zagranie w koszykówce polegające na umieszczeniu piłki w koszu poprzez włożenie jej do niego, poprzedzone wyskokiem}
\Clue{32}{}{cyfrowy układ kombinacyjny, który wykonuje operacje dodawania dwóch (lub więcej) liczb dwójkowych}
\Clue{33}{}{obrzędowa trąba tybetańska}
\Clue{34}{}{bajt - najmniejsza adresowalna jednostka informacji pamięci komputerowej, składająca się z bitów}
\Clue{37}{}{jednostka informacji w systemie dwójkowym, oznaczająca 2\textasciicircum80 bajtów}\end{PuzzleClues}\newpage\section*{Krzyżówka 160}

\noindent\begin{Puzzle}{23}{26}|*	|*	|*	|*	|*	|*	|*	|*	|*	|*	|*	|*	|*	|*	|*	|*	|*	|*	|*	|*	|*	|[1][S]\darr	|[2][S]\darr	|*	|.
|*	|*	|*	|[3][S]\drarr	|o	|b	|s	|e	|s	|j	|o	|n	|i	|s	|t	|k	|a	|*	|*	|*	|*	|b	|ł	|*	|.
|*	|*	|*	|e	|[4][S]\darr	|*	|*	|*	|[5][S]\darr	|*	|*	|*	|*	|*	|*	|*	|*	|*	|*	|[6][S]\darr	|*	|i	|a	|*	|.
|*	|*	|*	|g	|m	|*	|[7][S]\rarr	|z	|e	|m	|s	|t	|a	|[][,]{ }	|f	|a	|r	|a	|o	|n	|a	|*	|n	|*	|.
|*	|*	|*	|z	|e	|*	|[8][S]\rarr	|l	|u	|b	|c	|z	|y	|k	|*	|*	|*	|[9][S]\rarr	|k	|i	|w	|i	|*	|*	|.
|*	|*	|*	|e	|m	|[10][S]\darr	|*	|*	|*	|*	|*	|*	|*	|*	|[11][S]\rarr	|ł	|u	|p	|i	|e	|ń	|*	|*	|*	|.
|*	|*	|*	|m	|o	|j	|[12][S]\rarr	|g	|a	|ł	|ę	|z	|i	|a	|k	|[][,]{ }	|z	|b	|i	|t	|y	|*	|*	|*	|.
|*	|*	|[13][S]\drarr	|p	|r	|a	|g	|m	|a	|t	|y	|k	|a	|[][,]{ }	|s	|ł	|u	|ż	|b	|o	|w	|a	|*	|*	|.
|*	|[14][S]\darr	|b	|l	|a	|ł	|[15][S]\darr	|*	|*	|*	|*	|[16][S]\darr	|*	|[17][S]\darr	|*	|*	|*	|*	|*	|p	|[18][S]\darr	|*	|*	|*	|.
|*	|d	|a	|a	|n	|o	|d	|*	|*	|[19][S]\rarr	|o	|s	|ł	|o	|m	|u	|ł	|*	|*	|e	|m	|*	|*	|*	|.
|*	|r	|j	|r	|d	|w	|r	|[20][S]\rarr	|w	|e	|r	|k	|*	|d	|*	|*	|[21][S]\darr	|[22][S]\rarr	|b	|r	|a	|s	|*	|*	|.
|*	|z	|o	|z	|u	|i	|o	|[23][S]\rarr	|w	|z	|ó	|r	|[][,]{ }	|c	|h	|e	|m	|i	|c	|z	|n	|y	|*	|*	|.
|*	|e	|s	|[][,]{ }	|m	|e	|ż	|*	|*	|[24][S]\darr	|*	|ó	|*	|i	|[25][S]\drarr	|m	|o	|s	|t	|e	|k	|*	|*	|*	|.
|*	|w	|*	|p	|*	|c	|d	|*	|*	|c	|[26][S]\darr	|c	|[27][S]\darr	|ą	|ż	|*	|r	|[28][S]\darr	|*	|[][,]{ }	|i	|*	|*	|*	|.
|*	|o	|[29][S]\darr	|r	|*	|[][,]{ }	|ż	|[30][S]\darr	|[31][S]\darr	|z	|c	|e	|g	|g	|e	|*	|g	|m	|[32][S]\darr	|o	|e	|[33][S]\darr	|*	|*	|.
|*	|r	|t	|ó	|[34][S]\rarr	|s	|a	|m	|f	|a	|i	|n	|a	|*	|g	|*	|a	|e	|c	|w	|t	|p	|[35][S]\darr	|[36][S]\darr	|.
|*	|y	|h	|b	|*	|k	|k	|a	|a	|b	|s	|i	|j	|*	|l	|*	|*	|d	|y	|o	|*	|r	|a	|a	|.
|[37][S]\drarr	|t	|o	|n	|g	|a	|*	|f	|w	|a	|*	|e	|d	|*	|u	|*	|*	|i	|k	|c	|[38][S]\darr	|o	|l	|m	|.
|s	|n	|m	|y	|*	|l	|*	|i	|o	|n	|*	|*	|y	|[39][S]\darr	|g	|*	|*	|a	|l	|o	|w	|m	|g	|p	|.
|y	|i	|a	|*	|*	|n	|[40][S]\darr	|a	|r	|*	|*	|*	|*	|a	|a	|*	|*	|t	|i	|ż	|a	|i	|i	|l	|.
|b	|a	|s	|*	|*	|y	|g	|*	|e	|[41][S]\rarr	|m	|e	|s	|z	|*	|*	|*	|e	|z	|e	|g	|n	|e	|a	|.
|i	|*	|*	|*	|*	|*	|a	|[42][S]\rarr	|k	|r	|o	|s	|n	|o	|*	|*	|[43][S]\drarr	|k	|a	|r	|i	|e	|r	|*	|.
|r	|[44][S]\rarr	|w	|o	|m	|e	|l	|a	|*	|[45][S]\rarr	|g	|ł	|o	|w	|a	|*	|b	|a	|c	|n	|n	|n	|k	|*	|.
|a	|*	|*	|[46][S]\rarr	|u	|l	|e	|g	|ł	|o	|ś	|ć	|*	|*	|*	|*	|*	|*	|j	|e	|a	|t	|a	|*	|.
|k	|*	|[47][S]\rarr	|r	|o	|s	|a	|*	|[48][S]\rarr	|w	|i	|n	|k	|u	|l	|a	|c	|j	|a	|*	|*	|*	|*	|*	|.
|*	|*	|[49][S]\rarr	|n	|i	|e	|s	|z	|a	|b	|l	|o	|n	|o	|w	|o	|ś	|ć	|*	|*	|*	|*	|*	|*	|.
|*	|[50][S]\rarr	|w	|i	|ą	|z	|*	|*	|*	|*	|*	|*	|*	|*	|*	|*	|*	|*	|*	|*	|*	|*	|*	|*	|.\end{Puzzle}

\newpage

\begin{PuzzleClues}{\textbf{Poziome}\\}\Clue{3}{}{kobieta, która ma na jakimś tle obsesję, niebędącą jednak jednostką chorobową}
\Clue{7}{}{dolegliwości ze strony układu pokarmowego, nazywane tak szczególnie wtedy, gdy towarzyszą pobytowi w Afryce, najczęściej w Egipcie}
\Clue{8}{}{znana przyprawa (afrodyzjak); natka lubczyku}
\Clue{9}{}{inna nazwa dolara nowozelandzkiego}
\Clue{11}{}{bicie kogoś; obecne tylko we frazeologii}
\Clue{12}{}{Ramaria stricta (Pers.) Quél. - gatunek grzyba z rodziny siatkolistowatych (Gomphaceae); w Europie uważana jest za grzyb niejadalny, jednak w Chinach, Meksyku i na Madagaskarze jest grzybem jadalnym}
\Clue{13}{}{każdy akt prawny regulujący kompetencje, zasady hierarchii służbowej a także prawa i obowiązki pracowników administracji publicznej}
\Clue{19}{}{oślik - mieszaniec międzygatunkowy ogiera konia domowego z klaczą osła, całkowicie bezpłodny; niższy i słabszy od muła i dlatego dużo rzadziej hodowany}
\Clue{20}{}{mechanizm zegarka}
\Clue{22}{}{linfa biegnąca od końca rei ku rufie w celu manewrowania żaglami w poziomie}
\Clue{23}{}{zapis składu lub budowy pierwiastków cząsteczek bądź związków chemicznych za pomocą symboli stosowanych w chemii}
\Clue{25}{}{część oprawki okularów, łącząca szkła}
\Clue{34}{}{gęsta katalońska potrawa, przygotowywana z pomidorów, bakłażanów, papryki, cukinii, cebuli i czosnku duszonych na oliwie}
\Clue{37}{}{państwo leżące na archipelagu o tej samej nazwie, w Polinezji na południowym Pacyfiku, położone w jednej trzeciej drogi pomiędzy Nową Zelandią a Hawajami, na południe od Samoa, na wschód od Fidżi}
\Clue{41}{}{dietetyczna karma dla koni, zwłaszcza wyścigowych}
\Clue{42}{}{warsztat tkacki}
\Clue{43}{}{najszybszy bieg konia, odmiana galopu}
\Clue{44}{}{(1873-1911), pisarz i krytyk literacki}
\Clue{45}{}{zdolność organizmu człowieka do przyjęcia dużej dawki alkoholu i zachowania przytomności}
\Clue{46}{}{cecha czegoś, w czym objawia się to, że ktoś jest uległy}
\Clue{47}{}{włoski malarz i grafik (1615-73) batalistyka, pejzaże}
\Clue{48}{}{zabezpieczenie należności banku na towarze znajdującym się w drodze}
\Clue{49}{}{to, że ktoś/coś jest nieszablonowe, nieschematyczne}
\Clue{50}{}{drzewo lub krzew strefy umiarkowanej o cennym drewnie}\end{PuzzleClues}

\begin{PuzzleClues}{\textbf{Pionowe}\\}\Clue{1}{}{proces przekształcania danych w informacje, a informacji w wiedzę, która może być wykorzystana do zwiększenia konkurencyjności przedsiębiorstwa}
\Clue{2}{}{dawna jednostka podziału pól wspólnoty w średniowiecznej Europie Zachodniej, służąca pomiarom powierzchni i długości ziemi przeznaczonej pod zasiewy}
\Clue{3}{}{pierwszy egzemplarz danej publkacji wysyłany przez drukarnię, aby redakcja mogła go sprawdzić i ostatecznie zatwierdzić do druku}
\Clue{4}{}{pismo dyplomatyczne dołączane do noty lub wręczane osobiście, stanowiące najczęściej poglądy rządu na omawiany temat}
\Clue{5}{}{w chemii: symbol europu}
\Clue{6}{}{Megachiroptera - podrząd dużych nietoperzy; występują głównie w tropikalnej strefie Afryki i Azji, a także w Australii i na wielu wyspach oceanicznych}
\Clue{10}{}{Juniperus scopulorum - gatunek z rodziny cyprysowatych}
\Clue{13}{}{w geologii: drugi wiek (w geochronologii) lub drugie piętro (w chronostratygrafii) środkowej jury}
\Clue{14}{}{pracownia, w której tworzone są drzeworyty}
\Clue{15}{}{grzyb należący do drożdży, wywołujący choroby grzybicze}
\Clue{16}{}{fakt skrócenia czegoś}
\Clue{17}{}{pręt, drut lub lina stosowana w celu usztywnienia słupa, wysięgnika itd}
\Clue{18}{}{nakładka na wiośle w miejscu styku z dulką}
\Clue{21}{}{MÓRG, JUTRZYNA}
\Clue{24}{}{pasterz owiec lub wołów w Rumunii, Mołdawii, Węgrzech oraz części Ukrainy, a także u ludów Kaukazu i Azji Środkowej}
\Clue{25}{}{przewóz towarów i ludzi statkami wodnymi}
\Clue{26}{}{Taxus - rodzaj roślin z rodziny cisowatych (Taxaceae), których przedstawiciele występują w Europie, Azji północnej, wschodniej i południowo-wschodniej, północno-zachodniej Afryce oraz Ameryce Północnej}
\Clue{27}{}{instrument muzyczny dęty drewniany z grupy dud popularny w rejonie Beskidu Śląskiego, pompowany dymlokiem bez konieczności dmuchania ustami}
\Clue{28}{}{zbiór książek, pism, materiałów multimedialnych, przechowywany gdzieś i udostępniany zainteresowanym}
\Clue{29}{}{Hugh, angielski chirurg ortopeda (1833-91); wprowadził szynę do unieruchamiania kończyn}
\Clue{30}{}{o grupie ludzi połączonych niejasnymi, egoistycznymi interesami, popierających i broniących się wzajemnie, których celem jest trwanie na różnych poziomach i wymiarach władzy}
\Clue{31}{}{cienkie chrupkie ciastko w kształcie kokardki, smażone w głębokim tłuszczu i posypywane cukrem pudrem, najczęściej przygotowywane w czasie karnawału, na tłusty czwartek lub na ostatki}
\Clue{32}{}{zmiana budowy łańcuchowej związku w budowę cykliczną (pierścieniową)}
\Clue{33}{}{ktoś ważny, liczący się}
\Clue{35}{}{mieszkanka Algierii, kobieta pochodzenia algierskiego}
\Clue{36}{}{ozdobna flaszeczka na leki i wonności popularna w starożytności i średniowieczu}
\Clue{37}{}{mieszkaniec Syberii, człowiek pochodzenia syberyjskiego}
\Clue{38}{}{końcowy odcinek żeńskiego układu rozrodczego różnych zwierząt}
\Clue{39}{}{miasto w Federacji Rosyjskiej, port w pobliżu ujścia Donu do Morza Azowskiego,80 tys. mieszkańców (1986)}
\Clue{40}{}{mały żaglowiec towarowy używany na przełomie XIX i XX w}
\Clue{43}{}{symbol bara, jednostki miary ciśnienia w układzie jednostek CGS określonej jako 10E+6 dyn/cm2 = 10E+6 b}\end{PuzzleClues}\newpage\section*{Krzyżówka 161}

\noindent\begin{Puzzle}{17}{19}|*	|[1][S]\drarr	|r	|ó	|w	|[][,]{ }	|i	|r	|y	|g	|a	|c	|y	|j	|n	|y	|*	|*	|.
|[2][S]\rarr	|p	|a	|d	|w	|a	|n	|*	|*	|*	|*	|*	|*	|*	|*	|*	|*	|[3][S]\darr	|.
|*	|u	|[4][S]\drarr	|h	|i	|t	|*	|*	|*	|*	|*	|*	|*	|*	|*	|*	|*	|b	|.
|*	|n	|k	|*	|*	|*	|*	|*	|[5][S]\drarr	|s	|z	|e	|r	|p	|a	|*	|*	|a	|.
|[6][S]\rarr	|k	|o	|ń	|[][,]{ }	|b	|e	|l	|g	|i	|j	|s	|k	|i	|*	|[7][S]\darr	|*	|ż	|.
|*	|t	|c	|[8][S]\rarr	|m	|e	|z	|z	|a	|n	|i	|n	|o	|*	|*	|c	|[9][S]\darr	|a	|.
|*	|[][,]{ }	|i	|[10][S]\rarr	|g	|i	|e	|r	|u	|l	|a	|*	|[11][S]\darr	|*	|*	|e	|d	|n	|.
|*	|z	|a	|*	|*	|[12][S]\rarr	|v	|o	|l	|t	|a	|*	|p	|*	|*	|g	|i	|t	|.
|*	|l	|[][,]{ }	|*	|*	|*	|*	|*	|e	|*	|[13][S]\rarr	|l	|o	|g	|g	|i	|a	|*	|.
|*	|e	|m	|*	|*	|[14][S]\drarr	|v	|r	|i	|e	|s	|*	|ś	|[15][S]\darr	|*	|e	|l	|*	|.
|*	|w	|u	|[16][S]\rarr	|k	|a	|p	|i	|t	|u	|l	|a	|r	|z	|*	|ł	|e	|*	|.
|*	|n	|z	|[17][S]\rarr	|p	|r	|o	|c	|e	|s	|*	|*	|ó	|n	|[18][S]\darr	|k	|k	|*	|.
|*	|y	|y	|*	|*	|u	|*	|[19][S]\rarr	|r	|y	|k	|*	|d	|a	|s	|a	|t	|*	|.
|*	|*	|k	|[20][S]\rarr	|w	|i	|e	|k	|*	|*	|[21][S]\darr	|[22][S]\darr	|e	|j	|ł	|*	|y	|*	|.
|[23][S]\rarr	|m	|a	|c	|a	|*	|*	|*	|*	|*	|i	|o	|k	|d	|u	|[24][S]\darr	|k	|*	|.
|*	|*	|*	|*	|*	|[25][S]\drarr	|r	|e	|w	|e	|r	|s	|*	|k	|ż	|g	|*	|*	|.
|[26][S]\rarr	|b	|l	|a	|c	|h	|o	|w	|n	|i	|c	|a	|*	|a	|ą	|a	|*	|*	|.
|*	|[27][S]\rarr	|k	|o	|b	|i	|e	|t	|a	|*	|*	|*	|*	|*	|c	|y	|*	|*	|.
|*	|*	|*	|*	|[28][S]\rarr	|n	|o	|c	|e	|k	|[][,]{ }	|r	|u	|d	|y	|*	|*	|*	|.
|[29][S]\rarr	|m	|r	|o	|k	|*	|*	|*	|*	|*	|*	|*	|*	|*	|*	|*	|*	|*	|.\end{Puzzle}

\newpage

\begin{PuzzleClues}{\textbf{Poziome}\\}\Clue{1}{}{ręcznie lub mechanicznie wykonane podłużne zagłębienie w ziemi służące do regulowania gospodarki wodnej upraw rolnych}
\Clue{2}{}{PADOVANA - utwór muzyczny (taniec) wschodzący w skład barokowej suity}
\Clue{4}{}{coś przebojowego, najczęściej nowego, co robi karierę, ktoś to dobrze ocenia}
\Clue{5}{}{członek tybetańskiego plemienia Szerpów, zamieszkującego Himalaje na terenie Indii i Nepalu}
\Clue{6}{}{belg, koń barabancki - najstarsza zimnokrwista rasa konia domowego, wyhodowana w Belgii w wyniku krzyżowania koni miejscowych z dawnymi rycerskimi końmi flamandzkimi, masywnymi brabanosami i ardenami}
\Clue{8}{}{MEZANIN, PÓŁPIĘTRO, ANTRESOLA}
\Clue{10}{}{fizyk (1917-75); specjalista w dziedzinie fizyki jądrowej wysokich energii}
\Clue{12}{}{fizyk i fizjolog włoski (1745-1827); wynalazł elektrofor, odkrył gaz błotny, zbudował kondensator i ogniwo galwaniczne}
\Clue{13}{}{balkon wnękowy}
\Clue{14}{}{holenderski botanik i genetyk (1848-1935); twórca teorii zmienności mutacyjnej organizmu}
\Clue{16}{}{jedno z pomieszczeń klasztornych, służące zakonnikom do zebrań, także sala zebrań kapituły kanoników}
\Clue{17}{}{przebieg następujących po sobie i powiązanych przyczynowo zmian}
\Clue{19}{}{donośny gardłowy głos zwierzęcia}
\Clue{20}{}{długość życia lub istnienia czegoś, czas, który upłynął od momentu czyichś urodzin lub od momentu zaistnienia czegoś}
\Clue{23}{}{rodzaj pieczywa chrupkiego podobnego do żydowskiej macy}
\Clue{25}{}{dokument poświadczający istnienie stosunku dłużnego, np. stanowiący pokwitowanie za otrzymany na przechowanie przedmiot lub pieniądze, które należy zwrócić}
\Clue{26}{}{konstrukcja dachowa stosowana przy budowie mostów i dźwignic}
\Clue{27}{}{dorosły człowiek płci żeńskiej}
\Clue{28}{}{nocek nadwodny, nocek Daubentona, Myotis daubentonii - gatunek ssaka z rzędu nietoperzy; pospolity w całej Polsce, z wyjątkiem terenów ubogich w zbiorniki i cieki wodne}
\Clue{29}{}{ciemność, cień, brak wystarczającej ilości światła}\end{PuzzleClues}

\begin{PuzzleClues}{\textbf{Pionowe}\\}\Clue{1}{}{miejsce, do którego zlewa się płynne nieczystości produkowane na danym obszarze}
\Clue{3}{}{znany pięknie upierzony ptak łowny pochodzenia azjatyckiego z rodzaju Phasianus z rodziny kurowatych lub z kliku innych pokrewnych rodzajów (np. Lophura, Lophoporus), których przedstawiciele mają podobne cechy wyglądu}
\Clue{4}{}{hałaśliwa i jazgotliwa muzyka}
\Clue{5}{}{paramilitarny tytuł, noszony przez przywódcę NSDAP w danym okręgu partyjnym}
\Clue{7}{}{wkład finansowy}
\Clue{9}{}{człowiek biegły w dyskutowaniu}
\Clue{11}{}{skorupiak z rzędu pośródków}
\Clue{14}{}{owca grzywiasta z płn. wsch. Afryki, rzadka ceniona}
\Clue{15}{}{znajdek - osoba, najczęściej dziecko, znalezione przez obcych ludzi, płci żeńskiej}
\Clue{18}{}{pracownik (zwykle fizyczny), który pracuje (a bardzo często także mieszka) w domu (zwykle majętnego) pracodawcy}
\Clue{21}{}{jedna ze starszych usług sieciowych umożliwiająca rozmowę na tematycznych lub towarzyskich kanałach komunikacyjnych, jak również prywatną z inną podłączoną aktualnie osobą}
\Clue{22}{}{owad społeczny z podrodziny os właściwych, przeważnie niewielkich rozmiarów, najczęściej z rodzaju Vespula lub Dolichovespula, którego nazwa gatunkowa w jezyku polskim zawiera członosa}
\Clue{24}{}{malarz rosyjski (1831-94); obrazy religijne, historyczne, portrety}
\Clue{25}{}{starożytna hebrajska miara objętości płynów, równa około 7 lub 3,7 litra}\end{PuzzleClues}\newpage\section*{Krzyżówka 162}

\noindent\begin{Puzzle}{22}{33}|*	|*	|*	|*	|*	|*	|*	|*	|*	|*	|*	|*	|*	|[1][S]\darr	|*	|*	|*	|*	|*	|*	|*	|*	|*	|.
|*	|*	|*	|*	|*	|*	|*	|*	|*	|*	|*	|*	|*	|d	|[2][S]\darr	|*	|*	|*	|[3][S]\darr	|*	|*	|*	|*	|.
|*	|[4][S]\darr	|*	|*	|*	|*	|*	|[5][S]\drarr	|s	|ł	|a	|w	|i	|a	|n	|k	|a	|*	|d	|*	|[6][S]\darr	|*	|*	|.
|*	|m	|*	|*	|*	|*	|*	|k	|*	|*	|*	|*	|[7][S]\darr	|w	|a	|*	|[8][S]\darr	|*	|ż	|*	|p	|*	|*	|.
|*	|i	|*	|*	|*	|*	|*	|a	|*	|*	|*	|[9][S]\darr	|m	|e	|r	|*	|d	|*	|e	|*	|r	|[10][S]\darr	|*	|.
|*	|e	|[11][S]\rarr	|u	|k	|o	|ś	|n	|i	|k	|*	|k	|o	|s	|c	|*	|r	|*	|l	|*	|o	|k	|*	|.
|*	|s	|*	|*	|*	|*	|*	|c	|[12][S]\darr	|*	|*	|r	|t	|*	|y	|*	|z	|*	|a	|*	|t	|o	|*	|.
|*	|z	|*	|*	|*	|*	|*	|l	|a	|*	|*	|z	|o	|*	|z	|*	|e	|[13][S]\darr	|d	|*	|o	|m	|*	|.
|*	|a	|*	|*	|*	|*	|*	|e	|u	|*	|[14][S]\darr	|e	|c	|*	|[][,]{ }	|*	|w	|b	|a	|[15][S]\darr	|k	|o	|*	|.
|*	|n	|[16][S]\rarr	|a	|c	|h	|a	|r	|d	|*	|z	|w	|r	|*	|ł	|*	|i	|i	|[][,]{ }	|s	|l	|r	|*	|.
|*	|a	|*	|*	|[17][S]\darr	|*	|*	|z	|y	|*	|e	|[][,]{ }	|o	|*	|u	|*	|a	|g	|b	|t	|e	|y	|*	|.
|*	|[][,]{ }	|*	|*	|r	|*	|*	|a	|c	|*	|s	|k	|s	|*	|s	|*	|k	|o	|r	|e	|p	|j	|*	|.
|*	|e	|*	|*	|ó	|*	|*	|n	|j	|[18][S]\darr	|k	|a	|s	|[19][S]\darr	|k	|*	|[][,]{ }	|s	|u	|r	|s	|c	|*	|.
|*	|l	|*	|*	|ż	|*	|*	|k	|a	|p	|r	|w	|o	|e	|o	|*	|c	|[][,]{ }	|n	|n	|y	|z	|*	|.
|*	|a	|*	|*	|[][,]{ }	|*	|*	|a	|*	|o	|o	|o	|w	|m	|w	|*	|i	|s	|a	|i	|d	|y	|*	|.
|*	|s	|*	|*	|i	|*	|*	|*	|*	|l	|b	|w	|i	|e	|a	|*	|e	|i	|t	|c	|r	|k	|[20][S]\darr	|.
|*	|t	|*	|*	|n	|*	|*	|*	|[21][S]\rarr	|d	|i	|y	|e	|r	|t	|i	|m	|e	|n	|t	|o	|*	|t	|.
|*	|y	|*	|[22][S]\darr	|d	|[23][S]\drarr	|b	|ę	|b	|e	|n	|*	|c	|y	|y	|*	|n	|c	|a	|w	|p	|*	|a	|.
|*	|c	|[24][S]\drarr	|s	|y	|k	|o	|m	|o	|r	|a	|*	|*	|t	|*	|[25][S]\darr	|y	|i	|*	|o	|s	|*	|f	|.
|[26][S]\drarr	|z	|n	|a	|j	|o	|m	|e	|k	|*	|*	|*	|*	|u	|*	|g	|*	|e	|*	|*	|*	|*	|l	|.
|a	|n	|i	|b	|s	|p	|[27][S]\darr	|*	|*	|*	|[28][S]\rarr	|n	|e	|r	|k	|a	|*	|b	|*	|*	|*	|*	|a	|.
|n	|o	|e	|a	|k	|i	|p	|*	|*	|*	|*	|[29][S]\drarr	|b	|a	|ś	|n	|i	|o	|w	|o	|ś	|ć	|*	|.
|t	|ś	|p	|n	|i	|a	|p	|*	|*	|*	|*	|c	|*	|[][,]{ }	|*	|g	|*	|r	|*	|*	|*	|*	|*	|.
|y	|ć	|o	|i	|*	|*	|r	|*	|*	|*	|*	|h	|*	|p	|*	|t	|*	|z	|*	|*	|*	|*	|*	|.
|s	|[][,]{ }	|r	|s	|*	|*	|[][S]-	|*	|[30][S]\drarr	|d	|r	|a	|k	|o	|n	|o	|w	|a	|t	|e	|*	|*	|*	|.
|e	|p	|z	|*	|*	|*	|o	|*	|s	|*	|*	|m	|*	|m	|*	|k	|*	|ń	|[31][S]\darr	|*	|*	|*	|*	|.
|p	|o	|ą	|*	|*	|*	|w	|*	|o	|*	|*	|*	|*	|o	|*	|*	|*	|s	|b	|*	|*	|*	|*	|.
|t	|p	|d	|*	|*	|*	|i	|*	|r	|*	|[32][S]\rarr	|p	|a	|s	|t	|y	|l	|k	|a	|*	|*	|*	|*	|.
|y	|y	|n	|[33][S]\rarr	|s	|i	|e	|d	|e	|m	|n	|a	|s	|t	|k	|a	|*	|i	|r	|*	|*	|*	|*	|.
|k	|t	|o	|*	|*	|*	|c	|[34][S]\rarr	|k	|a	|p	|a	|r	|o	|w	|c	|e	|*	|g	|*	|*	|*	|*	|.
|*	|u	|ś	|*	|*	|*	|*	|*	|*	|*	|*	|*	|*	|w	|*	|*	|*	|*	|i	|*	|*	|*	|*	|.
|*	|*	|ć	|*	|*	|*	|*	|*	|*	|[35][S]\rarr	|p	|o	|ż	|a	|r	|e	|k	|*	|e	|*	|*	|*	|*	|.
|*	|*	|*	|*	|*	|[36][S]\rarr	|s	|e	|g	|o	|v	|i	|a	|*	|*	|*	|*	|*	|l	|*	|*	|*	|*	|.
|*	|*	|*	|*	|*	|*	|*	|*	|*	|*	|*	|*	|*	|*	|*	|*	|*	|*	|*	|*	|*	|*	|*	|.\end{Puzzle}

\newpage

\begin{PuzzleClues}{\textbf{Poziome}\\}\Clue{5}{}{Słowianka - kobieta z jednego z ludów indoeuropejskich, które posługują się językami słowiańskimi, mają podobne zwyczaje, obrzędy i wierzenia}
\Clue{11}{}{znak pisarski, który ma kształt ukośnej kreski}
\Clue{16}{}{pisarz francuski (1899-1974), komedie bulwarowe, dramaty psychologiczne; „Idiotka”}
\Clue{21}{}{instrumentalna forma muzyki popularnej w XVIII w}
\Clue{23}{}{z dezaprobatą lub sympatią o dziecku}
\Clue{24}{}{ośla figa - gatunek figowca o ciężkostrawnych owocach i cennym drewnie}
\Clue{26}{}{poufałe określenie znajomego mężczyzny}
\Clue{28}{}{nerka zwierzęcia postrzegana jako mięso, element podrobów}
\Clue{29}{}{bajkowość, bajeczność, magiczność, niezwykłość; cecha czegoś, co się odznacza niezwykłą, specyficzną atmosferą, która przypomina baśń}
\Clue{30}{}{Rhinochimaeridae - rodzina morskich ryb chrzęstnoszkieletowych z rzędu chimerokształtnych (Chimaeriformes); w zapisie kopalnym są one znane z jury}
\Clue{32}{}{pigułka, tabletka - sproszkowany lek uformowany w niewielki krążek}
\Clue{33}{}{przyjęcie z okazji czyichś siedemnastych urodzin}
\Clue{34}{}{Capparales - rząd roślin wyróżniany w licznych XX-wiecznych systemach klasyfikacyjnych okrytonasiennych obejmujący kapustowate, kaparowate i rezedowate; w systemach APG, w tym ostatnim z 2009, rośliny tu klasyfikowane włączane są do rzędu kapustowców (Brassicales)}
\Clue{35}{}{zdrobniale: pożar - niekontrolowany ogień, niebezpieczna sytuacja, w której coś się pali}
\Clue{36}{}{miasto w Hiszpanii (Stara Kastylia) u podnóży Sierra de Guadarrama, ośrodek administracyjny prowincji Segovia}\end{PuzzleClues}

\begin{PuzzleClues}{\textbf{Pionowe}\\}\Clue{1}{}{amerykański finansista i polityk (1865-1951); pokojowa nagroda Nobla}
\Clue{2}{}{Narcissus bulbocodium - gatunek roślin należący do rodziny amarylkowatych}
\Clue{3}{}{dżelada, Theropithecus gelada - gatunek małpy wąskonosej z rodziny makakowatych (Cercopithecidae), jedyny przedstawiciel rodzaju Theropithecus; zamieszkuje wyżyny i góry północno-zachodniej Etiopii i Erytrei,  najliczniej występuje w Parku Narodowym Semien w Etiopii}
\Clue{4}{}{wskaźnik ukazujący zmianę popytu na dane dobro w stosunku do innego dobra}
\Clue{5}{}{córka kanclerza}
\Clue{6}{}{Protoclepsydrops) - rodzaj owodniowca o niepewnej pozycji filogenetycznej, żyjącego w późnym w późnym karbonie (środkowy pensylwan) na terenie dzisiejszej Ameryki Północnej}
\Clue{7}{}{osoba biorąca udział w wyścigach motocrossowych, uprawiająca motocross}
\Clue{8}{}{kangur drzewny, Dendrolagus ursinus - gatunek torbacza z rodziny kangurowatych; występuje na terenach północno-zachodniej Nowej Gwinei}
\Clue{9}{}{gatunek roślin z rodziny marzanowatych}
\Clue{10}{}{mieszkaniec Komorów, człowiek pochodzenia komoryjskiego}
\Clue{12}{}{rodzaj wewnętrznego popisu w szkołach muzycznych}
\Clue{13}{}{odmiana bigosu występująca w okolicach Szprotawy, charakteryzująca się obecnością nietypowych odmian suszonych grzybów, owoców leśnych oraz mięsa z dzikiej zwierzyny}
\Clue{14}{}{zeskrobany fragment materiału biologicznego (skóry, zaschniętej krwi, narządu itp.), użyty do badań diagnostycznych}
\Clue{15}{}{zajęcie sternika}
\Clue{17}{}{ciepły, delikatny róż wpadający w pomarańcz}
\Clue{18}{}{płaski, osuszony teren depresyjny}
\Clue{19}{}{świadczenie pieniężne przysługujące niektórym pracownikom, wykonującym pracę w szczególnych warunkach lub o szczególnym charakterze, spełniającym dodatkowo określone wymagania, ustające w momencie nabycia przez uprawnionego prawa do emerytury, osiągnięcia określonego wieku lub jego śmierci}
\Clue{20}{}{płyta z równego, gładkiego materiału}
\Clue{22}{}{sztangista grecki, srebrny medalista z Atlanty w wadze do 59 kg}
\Clue{23}{}{skopiowana z negatywu, naświetlona i obrobiona taśma filmowa (dziś może to być też inny niż taśma nośnik), przeznaczona do wielokrotnego odtwarzania}
\Clue{24}{}{cecha jakiejś rzeczy, przedmiotu: kiepskie wykonanie}
\Clue{25}{}{miasto w Indiach, stolica stanu Sikkim; instytut tybetologii}
\Clue{26}{}{środek dezynfekcyjny, środek antyseptyczny - substancja, która niszczy drobnoustroje i ich przetrwalniki}
\Clue{27}{}{członek PPR - Polskiej Partii Robotniczej}
\Clue{29}{}{pogardliwie: chłop}
\Clue{30}{}{ryjówka, Sorex - ssak owadożerny z rodziny sorków; występuje na wszystkich kontynentyach oprócz  Ameryki Południowej, Antarktydy i Australii}
\Clue{31}{}{KOWALIK; ptak leśny z rzędu wróblowatych, szaroniebieski, spodem rudy, owadożerny, chroniony}\end{PuzzleClues}\newpage\section*{Krzyżówka 163}

\noindent\begin{Puzzle}{16}{33}|*	|*	|*	|*	|*	|*	|*	|*	|*	|[1][S]\darr	|*	|*	|*	|*	|*	|*	|*	|.
|*	|*	|*	|*	|[2][S]\darr	|*	|*	|*	|*	|ś	|*	|*	|*	|*	|*	|*	|*	|.
|*	|*	|*	|*	|ł	|*	|*	|*	|*	|r	|*	|*	|*	|*	|*	|*	|[3][S]\darr	|.
|*	|*	|*	|*	|u	|*	|*	|*	|*	|o	|*	|*	|*	|*	|*	|*	|s	|.
|*	|*	|*	|*	|p	|*	|*	|*	|*	|d	|*	|*	|*	|*	|*	|*	|z	|.
|*	|*	|*	|*	|e	|*	|*	|*	|*	|o	|*	|*	|*	|*	|*	|*	|c	|.
|*	|*	|*	|*	|k	|*	|*	|*	|*	|w	|*	|*	|*	|*	|*	|*	|z	|.
|*	|*	|*	|*	|[][,]{ }	|*	|*	|*	|[4][S]\darr	|i	|*	|*	|*	|*	|*	|*	|ę	|.
|*	|*	|*	|*	|o	|*	|*	|*	|d	|s	|*	|*	|*	|*	|*	|*	|t	|.
|*	|*	|*	|*	|s	|*	|*	|*	|o	|k	|*	|*	|*	|*	|*	|[5][S]\darr	|k	|.
|*	|*	|*	|*	|a	|*	|*	|*	|d	|o	|*	|*	|*	|*	|*	|j	|i	|.
|*	|[6][S]\rarr	|e	|n	|d	|y	|w	|i	|a	|[][,]{ }	|l	|e	|t	|n	|i	|a	|*	|.
|*	|*	|[7][S]\rarr	|f	|o	|s	|f	|a	|t	|a	|z	|a	|*	|*	|*	|ł	|*	|.
|*	|*	|*	|*	|w	|*	|*	|*	|e	|s	|*	|*	|*	|*	|*	|o	|*	|.
|*	|*	|*	|*	|y	|[8][S]\darr	|*	|*	|k	|t	|*	|*	|*	|*	|*	|w	|*	|.
|*	|*	|*	|*	|*	|o	|*	|*	|[][,]{ }	|a	|*	|*	|*	|*	|*	|i	|*	|.
|*	|*	|*	|*	|*	|l	|*	|*	|o	|t	|*	|*	|*	|*	|*	|e	|*	|.
|*	|*	|*	|*	|*	|e	|*	|[9][S]\darr	|ł	|y	|*	|*	|*	|*	|*	|c	|*	|.
|*	|*	|*	|*	|*	|j	|*	|ś	|ó	|c	|*	|*	|*	|*	|*	|[][,]{ }	|*	|.
|*	|*	|*	|*	|*	|e	|*	|l	|w	|z	|*	|*	|*	|*	|*	|p	|*	|.
|*	|*	|*	|*	|*	|k	|*	|u	|k	|n	|*	|*	|*	|*	|*	|e	|*	|.
|*	|*	|*	|*	|*	|[][,]{ }	|*	|z	|o	|e	|*	|*	|*	|*	|*	|s	|[10][S]\darr	|.
|*	|*	|*	|*	|*	|b	|*	|i	|w	|*	|*	|*	|*	|*	|*	|t	|s	|.
|*	|*	|*	|*	|*	|e	|*	|c	|y	|*	|*	|*	|[11][S]\darr	|*	|*	|k	|z	|.
|*	|*	|[12][S]\rarr	|h	|u	|r	|m	|a	|*	|*	|*	|*	|k	|*	|*	|o	|y	|.
|*	|*	|*	|*	|*	|g	|*	|*	|*	|*	|*	|*	|a	|*	|*	|w	|b	|.
|[13][S]\rarr	|z	|i	|f	|t	|a	|*	|*	|*	|*	|[14][S]\rarr	|g	|r	|u	|d	|a	|*	|.
|*	|*	|*	|*	|*	|m	|*	|*	|*	|*	|*	|*	|t	|*	|*	|t	|*	|.
|*	|*	|*	|*	|*	|o	|*	|*	|*	|*	|*	|*	|u	|*	|*	|y	|*	|.
|*	|*	|*	|*	|*	|t	|*	|*	|*	|*	|[15][S]\rarr	|o	|z	|ó	|r	|*	|*	|.
|*	|*	|*	|*	|*	|o	|*	|*	|*	|*	|*	|*	|*	|*	|*	|*	|*	|.
|*	|*	|*	|*	|*	|w	|*	|*	|*	|*	|*	|*	|*	|*	|*	|*	|*	|.
|*	|*	|*	|*	|*	|y	|*	|*	|*	|*	|*	|*	|*	|*	|*	|*	|*	|.
|*	|*	|*	|*	|*	|*	|*	|*	|*	|*	|*	|*	|*	|*	|*	|*	|*	|.\end{Puzzle}

\newpage

\begin{PuzzleClues}{\textbf{Poziome}\\}\Clue{6}{}{uprawna odmiana sałaty o długich i mięsistych liściach}
\Clue{7}{}{enzym należący do hydrolaz; hydrolizuje on wiązania fosforanomonoestrowe, w efekcie czego następuje defosforylacja cząsteczki}
\Clue{12}{}{drzewo lub krzew międzyzwrotnikowy uprawiany dla słodkich, jadalnych owoców}
\Clue{13}{}{miasto w Egipcie nad rzeką Damietta}
\Clue{14}{}{sklejona bryła czegoś, często ziemi}
\Clue{15}{}{cielęcy w galarecie}\end{PuzzleClues}

\begin{PuzzleClues}{\textbf{Pionowe}\\}\Clue{1}{}{środowisko, najczęściej wodne, w którym warunki życia ulegają gwałtownym zmianom}
\Clue{2}{}{skała osadowa o zróżnicowanym składzie, najczęściej zawierająca znaczne ilości kaolinitu oraz drobnego kwarcu, wykazuje się dobrą łupkowatością}
\Clue{3}{}{Euphausiacea - rząd morskich pancerzowców liczący 86 gatunków; w odróżnieniu od innych skorupiaków szczętki posiadają skrzela zewnętrzne}
\Clue{4}{}{świadczenie socjalne przysługujące pracownikom, których dzieci idą do szkoły}
\Clue{5}{}{Juniperus drupacea - gatunek z rodziny cyprysowatych}
\Clue{8}{}{olejek eteryczny pozyskiwany ze skórek owoców pomarańczy bergamota}
\Clue{9}{}{z gromady krągłoustych; drapieżny o nagiej skórze wydzielającej śliz; bez oczu}
\Clue{10}{}{w górnictwie - wyrobisko}
\Clue{11}{}{członek Zakonu Kartuzów}\end{PuzzleClues}\newpage\section*{Krzyżówka 164}

\noindent\begin{Puzzle}{19}{28}|*	|*	|*	|*	|[1][S]\drarr	|c	|y	|f	|r	|a	|[][,]{ }	|r	|z	|y	|m	|s	|k	|a	|*	|[2][S]\darr	|.
|*	|*	|[3][S]\darr	|[4][S]\darr	|n	|[5][S]\darr	|[6][S]\darr	|*	|[7][S]\darr	|*	|*	|*	|*	|*	|[8][S]\darr	|*	|*	|[9][S]\darr	|[10][S]\darr	|m	|.
|*	|*	|s	|a	|i	|r	|p	|[11][S]\rarr	|s	|a	|r	|n	|a	|*	|p	|*	|*	|s	|g	|e	|.
|*	|[12][S]\darr	|t	|p	|e	|e	|ł	|[13][S]\rarr	|u	|s	|t	|k	|a	|*	|o	|[14][S]\darr	|[15][S]\darr	|i	|e	|r	|.
|[16][S]\drarr	|k	|r	|a	|s	|n	|o	|d	|r	|z	|e	|w	|*	|*	|d	|a	|s	|h	|n	|t	|.
|r	|l	|ę	|r	|t	|a	|n	|*	|i	|*	|*	|*	|[17][S]\darr	|*	|s	|k	|t	|h	|e	|o	|.
|e	|a	|c	|a	|a	|u	|i	|*	|g	|*	|*	|*	|s	|[18][S]\darr	|a	|r	|a	|i	|r	|n	|.
|s	|n	|z	|t	|c	|l	|w	|*	|a	|*	|[19][S]\drarr	|h	|u	|r	|d	|y	|c	|j	|a	|*	|.
|t	|g	|y	|[][,]{ }	|j	|t	|o	|*	|o	|*	|n	|*	|k	|o	|n	|l	|j	|c	|ł	|*	|.
|y	|*	|c	|g	|o	|*	|w	|*	|*	|*	|e	|*	|n	|b	|i	|*	|a	|z	|*	|*	|.
|t	|*	|i	|o	|n	|[20][S]\drarr	|a	|n	|t	|y	|k	|*	|o	|u	|k	|*	|*	|y	|*	|*	|.
|u	|[21][S]\drarr	|e	|l	|a	|s	|t	|o	|m	|e	|r	|y	|*	|s	|[][,]{ }	|[22][S]\darr	|[23][S]\darr	|c	|*	|[24][S]\darr	|.
|c	|ł	|l	|g	|r	|i	|e	|*	|*	|*	|o	|*	|*	|t	|k	|m	|ć	|y	|*	|k	|.
|j	|ó	|s	|i	|n	|t	|*	|*	|*	|*	|p	|*	|*	|a	|u	|o	|w	|*	|*	|o	|.
|a	|d	|t	|e	|o	|o	|[25][S]\drarr	|h	|i	|r	|o	|l	|a	|*	|l	|n	|i	|*	|[26][S]\darr	|m	|.
|[][,]{ }	|ź	|w	|g	|ś	|*	|a	|*	|[27][S]\darr	|*	|l	|*	|*	|*	|i	|o	|e	|*	|c	|e	|.
|g	|[][,]{ }	|o	|o	|ć	|*	|i	|*	|z	|*	|i	|*	|*	|*	|s	|g	|k	|*	|o	|d	|.
|a	|p	|*	|*	|*	|[28][S]\rarr	|s	|k	|a	|ł	|a	|[][,]{ }	|l	|i	|t	|a	|*	|*	|n	|i	|.
|t	|e	|*	|*	|*	|*	|*	|*	|t	|*	|*	|*	|*	|*	|y	|m	|*	|*	|c	|a	|.
|u	|ł	|*	|*	|*	|[29][S]\rarr	|f	|o	|r	|n	|a	|l	|k	|a	|*	|i	|*	|*	|e	|n	|.
|n	|n	|*	|*	|[30][S]\rarr	|e	|s	|c	|u	|d	|e	|l	|l	|a	|*	|s	|*	|*	|p	|t	|.
|k	|o	|*	|*	|[31][S]\rarr	|m	|o	|r	|d	|e	|r	|c	|a	|*	|*	|t	|*	|*	|t	|k	|.
|u	|m	|*	|*	|*	|*	|*	|*	|n	|[32][S]\rarr	|z	|a	|g	|a	|d	|k	|a	|*	|[][,]{ }	|a	|.
|*	|o	|*	|*	|*	|*	|*	|*	|i	|*	|[33][S]\rarr	|m	|u	|n	|d	|a	|n	|i	|a	|*	|.
|*	|r	|*	|[34][S]\rarr	|s	|u	|k	|c	|e	|s	|o	|r	|k	|a	|*	|*	|*	|*	|r	|*	|.
|*	|s	|*	|*	|[35][S]\rarr	|m	|a	|ź	|n	|i	|c	|a	|*	|*	|[36][S]\rarr	|r	|o	|n	|t	|*	|.
|*	|k	|*	|[37][S]\rarr	|c	|y	|k	|l	|i	|s	|t	|ó	|w	|k	|a	|*	|*	|*	|*	|*	|.
|*	|a	|*	|*	|*	|*	|[38][S]\rarr	|c	|e	|b	|u	|l	|k	|a	|*	|*	|*	|*	|*	|*	|.
|*	|*	|*	|[39][S]\rarr	|c	|a	|l	|i	|*	|*	|*	|*	|*	|*	|*	|*	|*	|*	|*	|*	|.\end{Puzzle}

\newpage

\begin{PuzzleClues}{\textbf{Poziome}\\}\Clue{1}{}{element systemu notacji liczb pochodzenia etruskiego}
\Clue{11}{}{euroazjatycki ssak z rodziny jeleniowatych - łowna}
\Clue{13}{}{poetycko o usteczkach}
\Clue{16}{}{andyjski krzew uprawiany w strefie tropikalnej ze względu na liście zawierające kokainę - krzew kokainowy}
\Clue{19}{}{wystający na zewnątrz murów obronnych ganek z otworami strzelniczymi w podłodze, z którego wyrzucano pociski i wylewano wrzątek na nieprzyjaciela}
\Clue{20}{}{starożytność; okres w historii Bliskiego Wschodu, Europy i Afryki Północnej obejmujący dzieje tych regionów od powstania pierwszych cywilizacji do około V wieku n.e}
\Clue{21}{}{tworzywa naturalne i sztuczne o właściwościach kauczuku, odznaczające się dużą sprężystością i rozciągliwością}
\Clue{25}{}{antylopa Huntera, Beatragus hunteri - gatunek ssaka z rodziny krętorogich, zamieszkujący sawanny w Kenii i Somalii; introdukowany w Tsavo National Park w Kenii}
\Clue{28}{}{skała, która ma zwartą budowę, jest zwięzła i twarda}
\Clue{29}{}{żona fornala}
\Clue{30}{}{zupa z kilku rodzajów mięsa i warzyw uważana za katalońskie danie narodowe}
\Clue{31}{}{osoba, która zabija, morduje innego człowieka; przestępca}
\Clue{32}{}{niejasna, tajemnicza sprawa}
\Clue{33}{}{kraina zwana także Przyziemiem, której mieszkańcy nie wierzą w magię; miejsce wymyślone przez Piersa Anthony'ego, autora powieści z cykluXanth}
\Clue{34}{}{następczyni, kontynuatorka}
\Clue{35}{}{naczynie, które służy do przechowywania mazi, smaru, smoły}
\Clue{36}{}{w dawnej Polsce patrol wojskowy kontrolujący w nocy warty i posterunki}
\Clue{37}{}{okrągła i płaska sportowa czapka z daszkiem}
\Clue{38}{}{przekształcony pęd podziemny o funkcji spichrzowej i przetrwalnikowej}
\Clue{39}{}{miasto w Kolumbii, w Andacii, ośrodek administracyjny departamentu Valle del Cauca, 1,3 mln. mieszkańców (1985)}\end{PuzzleClues}

\begin{PuzzleClues}{\textbf{Pionowe}\\}\Clue{1}{}{cecha czegoś, co robi się zaocznie, nie na miejscu}
\Clue{2}{}{socjolog amerykański ur. w 1910 r., przedstawiciel funkcjonalizmu}
\Clue{3}{}{przestępstwo, które polega na nakłanianiu osoby do uprawiania nierządu}
\Clue{4}{}{organellum w komórkach eukariotycznych przeznaczone do modyfikacji i dystrybucji substancji tworzonych przez komórki}
\Clue{5}{}{marka samochodu; istniejący od 1899 r. we Francji producent samochodów, będący zarazem jedną z najstarszych firm samochodowych}
\Clue{6}{}{Pottiaceae - rodzina mchów z rzędu płoniwowców}
\Clue{7}{}{miasto w Filipinach na wyspie Mindanao; ośrodek handlowy regionu górniczego, rybołówstwo}
\Clue{8}{}{Splachnum sphaericum - gatunek mchu należący do rodziny podsadnikowatych; rzadki gatunek wysokogórski, rośnie w luźnych lub nieco skupionych darniach koloru żółtozielonego, roślina objęta ścisłą ochroną gatunkową w Polsce}
\Clue{9}{}{plemię Lordów Sihhów, którzy od urodzenia posiadali zdolność używania magii; plemię fikcyjne z cyklu powieściForteca C. J. Cherryh}
\Clue{10}{}{oficerski stopień wyższy od pułkownika}
\Clue{12}{}{uderzenie w dzwon okrętowy}
\Clue{14}{}{obraz namalowany farbami akrylowymi}
\Clue{15}{}{zakład, którego działalność związana jest z danym terenem}
\Clue{16}{}{przywrócenie istnienia gatunku lub populacji zagrożonej wyginięciem}
\Clue{17}{}{gruba i szorstka wełniana tkanina}
\Clue{18}{}{porcja robusty, popularnego gatunku kawy; określoną ilość robusty, zazwyczaj słoik lub specjalna torebka}
\Clue{19}{}{eufemistycznie o cmentarzu, zwłaszcza otoczonym szczególnym kultem czy choćby sentymentem}
\Clue{20}{}{ocenianie kogoś lub czegoś w celu odrzucenia osób lub rzeczy, które nie spełniają określonych wymagań}
\Clue{21}{}{łódź przystosowana do użytku na pełnym morzu}
\Clue{22}{}{kobieta, która uznaje i praktykuje pozostawanie w relacji uczuciowej i seksualnej tylko z jednym partnerem na raz}
\Clue{23}{}{rodzaj pasmanterii, metalowy kolec służący do ozdabiania ubrań}
\Clue{24}{}{aktorka, osoba, która w dawnych czasach zapewniała rozrywkę odtwarzając scenki, wcielając się w inne postaci, oceniana jako ekscentryczka, wolny duch}
\Clue{25}{}{dźwięk „a” podwyższony o pół tonu}
\Clue{26}{}{ilustracja, w której głównym celem jest wyrażenie własnej prezentacji projektu, idei, nastroju, dla użycia w filmach, grach komputerowych lub komiksach}
\Clue{27}{}{liczba osób zatrudnionych lub liczba miejsc pracy w danym przedsiębiorstwie, sektorze lub kraju}\end{PuzzleClues}\newpage\section*{Krzyżówka 165}

\noindent\begin{Puzzle}{19}{21}|*	|*	|*	|*	|*	|*	|*	|*	|*	|*	|*	|*	|*	|[1][S]\drarr	|g	|l	|e	|b	|a	|*	|.
|*	|*	|[2][S]\darr	|*	|[3][S]\darr	|[4][S]\drarr	|r	|a	|m	|a	|*	|[5][S]\darr	|[6][S]\drarr	|c	|e	|l	|l	|o	|*	|*	|.
|*	|*	|p	|*	|ł	|f	|[7][S]\rarr	|k	|a	|r	|a	|f	|k	|a	|*	|[8][S]\darr	|[9][S]\darr	|*	|*	|*	|.
|*	|*	|ł	|*	|a	|l	|*	|[10][S]\darr	|[11][S]\darr	|*	|*	|e	|l	|p	|*	|s	|m	|*	|[12][S]\darr	|*	|.
|*	|*	|ó	|[13][S]\drarr	|p	|o	|l	|s	|k	|i	|*	|z	|i	|*	|*	|t	|i	|[14][S]\darr	|t	|*	|.
|*	|[15][S]\rarr	|t	|w	|a	|r	|d	|z	|i	|e	|l	|*	|s	|*	|*	|r	|ó	|z	|v	|*	|.
|*	|[16][S]\darr	|n	|e	|d	|e	|*	|k	|j	|[17][S]\drarr	|s	|z	|t	|u	|f	|a	|d	|a	|*	|*	|.
|*	|d	|o	|r	|ł	|t	|*	|o	|[][,]{ }	|b	|*	|[18][S]\darr	|e	|*	|[19][S]\darr	|t	|[][,]{ }	|z	|*	|*	|.
|*	|w	|*	|k	|o	|*	|*	|p	|h	|u	|[20][S]\darr	|w	|r	|[21][S]\darr	|p	|u	|p	|u	|[22][S]\darr	|*	|.
|*	|o	|[23][S]\darr	|*	|*	|*	|*	|*	|y	|f	|t	|a	|*	|z	|a	|s	|s	|l	|p	|*	|.
|*	|j	|s	|*	|*	|*	|[24][S]\darr	|[25][S]\drarr	|b	|e	|r	|g	|m	|a	|n	|*	|z	|e	|u	|*	|.
|*	|a	|ł	|*	|*	|*	|b	|e	|r	|t	|y	|o	|*	|p	|c	|*	|c	|ń	|r	|*	|.
|[26][S]\rarr	|r	|o	|k	|o	|k	|o	|*	|y	|*	|p	|n	|*	|y	|e	|*	|z	|k	|p	|*	|.
|*	|k	|w	|*	|*	|[27][S]\darr	|b	|*	|d	|*	|l	|i	|*	|l	|r	|[28][S]\darr	|e	|a	|u	|*	|.
|*	|i	|o	|*	|[29][S]\rarr	|k	|o	|l	|o	|*	|e	|k	|*	|e	|n	|m	|l	|*	|r	|*	|.
|*	|*	|*	|*	|*	|*	|w	|*	|w	|*	|t	|*	|*	|n	|i	|o	|i	|[30][S]\darr	|a	|*	|.
|[31][S]\rarr	|k	|a	|m	|p	|e	|s	|z	|y	|n	|*	|*	|*	|i	|*	|y	|*	|d	|t	|*	|.
|[32][S]\rarr	|m	|a	|j	|a	|n	|k	|a	|*	|*	|*	|[33][S]\rarr	|d	|e	|s	|z	|c	|z	|*	|*	|.
|*	|*	|*	|*	|*	|*	|i	|[34][S]\rarr	|b	|a	|r	|s	|s	|*	|*	|e	|*	|i	|*	|*	|.
|[35][S]\rarr	|t	|y	|t	|a	|n	|*	|*	|[36][S]\rarr	|a	|n	|d	|r	|i	|e	|s	|s	|e	|n	|*	|.
|*	|*	|*	|*	|[37][S]\rarr	|b	|ł	|o	|t	|n	|i	|a	|r	|k	|a	|*	|*	|ń	|*	|*	|.
|*	|*	|*	|*	|*	|*	|*	|*	|*	|*	|*	|*	|*	|*	|*	|*	|*	|*	|*	|*	|.\end{Puzzle}

\newpage

\begin{PuzzleClues}{\textbf{Poziome}\\}\Clue{1}{}{upadek; niespodziewany, niezamierzony ruch polegający na zmianie pozycji z pionowej na poziomą}
\Clue{4}{}{element konstrukcyjny pojazdu, podstawowa część tworząca jego szkielet, wyznaczająca jego oś, zapewniająca nośność i odpowiednią sztywność podwozia}
\Clue{6}{}{wiolonczela}
\Clue{7}{}{zawartość karafki; tyle, ile się mieści w karafce}
\Clue{13}{}{język należący do grupy języków zachodniosłowiańskich}
\Clue{15}{}{osoba niezłomna, odporna na trudy}
\Clue{17}{}{potrawa z mięsa wołowego bez kości naszpikowanego słoniną, obsmażonego i duszonego z przyprawami}
\Clue{25}{}{Bo, (1869-1957), poeta szwedzki, krytyk faszyzmu}
\Clue{26}{}{ostatnia faza baroku w sztuce i architekturze}
\Clue{29}{}{jakiś mężczyzna lub chłopak, zwłaszcza obcy lub taki, względem którego zachowujemy dystans; gościu, koleś}
\Clue{31}{}{amerykańskie drzewo uprawiane głównie w Indonezji dla barwnika i cennego drewna}
\Clue{32}{}{przedstawicielka historycznego ludu Majów}
\Clue{33}{}{substancja, jaka spada podczas opadów deszczu}
\Clue{34}{}{prawnik warszawski z okresu Sejmu Czteroletniego (1760-1812); współpracownik Dekerta}
\Clue{35}{}{TITAN}
\Clue{36}{}{kompozytor holenderski ur. w 1939 r}
\Clue{37}{}{duża barka służąca do przewozu urobku wydobytego przez pogłębiarki}\end{PuzzleClues}

\begin{PuzzleClues}{\textbf{Pionowe}\\}\Clue{1}{}{łow. samiec kozicy}
\Clue{2}{}{zwyczajowa nazwa tkaniny wykonanej w splocie płóciennym o wielorakim zastosowaniu}
\Clue{3}{}{w górnictwie: urządzenie zatrzymujące klatkę szybową w razie zerwania się liny nośnej}
\Clue{4}{}{lekka broń szermierza z gałką na końcu}
\Clue{5}{}{nakrycie głowy mężczyzn w krajach islamskich, z czerwonego filcu, w kształcie ściętego stożka}
\Clue{6}{}{smar do nart}
\Clue{8}{}{niska chmura warstwowa}
\Clue{9}{}{słodki produkt spożywczy, w warunkach naturalnych wytwarzany głównie przez pszczoły właściwe, poprzez przetwarzanie nektaru kwiatowego roślin miododajnych, a także niektórych wydzielin występujących na liściach drzew iglastych}
\Clue{10}{}{pogardliwie o Niemcu}
\Clue{11}{}{kij golfowy, łączący zalety ironów i woodów}
\Clue{12}{}{telewizja, dział telekomunikacji zajmujący się przekazywaniem ruchomego obrazu oraz dźwięku na odległość}
\Clue{13}{}{mechanizm zegarka}
\Clue{14}{}{zdrobniale o zazuli - kukułce określanej regionalnie}
\Clue{16}{}{Ceraphronoidea - nadrodzina błonkówek z grupy owadziarek}
\Clue{17}{}{niewielki bar miejsce zakupu i spożywania posiłków}
\Clue{18}{}{wagon do przewozu ludzi, wchodzący w skład niekolejowego systemu transportu, np.: wagonik tramwajowy, wagonik kolejki linowej lub górskiej}
\Clue{19}{}{w Polsce od XVII w. średniozbrojna jazda zwana kozacką}
\Clue{20}{}{w fotografice: obiektyw anastygmatyczny złożony z trzech członów}
\Clue{21}{}{wykonanie czynności fizjologicznej; u roślin kwiatowych, przeniesienie ziarna pyłku na znamię słupka}
\Clue{22}{}{KARDYNAŁ}
\Clue{23}{}{słowo honoru - obietnica, której można zaufać}
\Clue{24}{}{kompozytor, pisarz i orientalista (1600-1672); zbiór tureckich pieśni ludowych}
\Clue{25}{}{liczba niewymierna, będąca podstawą logarytmu naturalnego; można ją definiować na kilka różnych sposobów}
\Clue{27}{}{kelwin - jednostka temperatury w układzie SI równa 1/273,16 temperatury termodynamicznej punktu potrójnego wody, oznaczana K}
\Clue{28}{}{słowacki kompozytor i pedagog (1906-1984); utwory symfoniczne, kameralne, kantaty, pieśni}
\Clue{30}{}{określony, wyznaczony termin, data}\end{PuzzleClues}\newpage\section*{Krzyżówka 166}

\noindent\begin{Puzzle}{21}{20}|*	|[1][S]\drarr	|k	|u	|l	|a	|s	|*	|*	|[2][S]\drarr	|s	|a	|l	|a	|*	|*	|[3][S]\drarr	|s	|i	|a	|l	|*	|.
|[4][S]\rarr	|f	|u	|n	|d	|a	|m	|e	|n	|t	|*	|*	|*	|[5][S]\rarr	|w	|a	|r	|*	|*	|*	|*	|*	|.
|*	|e	|*	|*	|[6][S]\rarr	|c	|h	|e	|z	|y	|*	|*	|*	|*	|[7][S]\drarr	|r	|e	|n	|i	|*	|*	|*	|.
|*	|e	|[8][S]\rarr	|l	|o	|d	|o	|ł	|a	|m	|a	|c	|z	|*	|k	|[9][S]\drarr	|k	|ł	|ą	|b	|*	|*	|.
|*	|r	|*	|*	|[10][S]\rarr	|p	|r	|z	|e	|b	|i	|e	|g	|*	|o	|t	|u	|[11][S]\darr	|*	|[12][S]\darr	|*	|*	|.
|*	|i	|*	|*	|[13][S]\rarr	|d	|w	|u	|p	|a	|r	|c	|e	|*	|t	|y	|r	|w	|*	|a	|*	|*	|.
|*	|a	|*	|*	|*	|*	|[14][S]\drarr	|w	|a	|r	|g	|a	|*	|*	|w	|n	|s	|y	|*	|l	|*	|*	|.
|*	|*	|*	|*	|*	|*	|f	|*	|*	|k	|*	|*	|[15][S]\rarr	|h	|i	|t	|*	|r	|*	|b	|*	|*	|.
|*	|[16][S]\rarr	|b	|r	|y	|k	|l	|a	|*	|*	|[17][S]\rarr	|m	|a	|n	|c	|a	|*	|ó	|*	|e	|*	|*	|.
|[18][S]\rarr	|w	|e	|j	|ś	|c	|i	|e	|*	|[19][S]\drarr	|o	|l	|c	|h	|a	|*	|*	|b	|[20][S]\darr	|r	|*	|*	|.
|*	|*	|*	|*	|*	|[21][S]\rarr	|p	|i	|e	|s	|z	|n	|i	|a	|*	|*	|*	|[][,]{ }	|t	|t	|[22][S]\darr	|*	|.
|[23][S]\rarr	|p	|a	|z	|i	|k	|*	|[24][S]\rarr	|a	|z	|g	|h	|a	|d	|i	|*	|[25][S]\darr	|a	|h	|i	|b	|*	|.
|*	|*	|*	|*	|*	|*	|*	|*	|[26][S]\rarr	|t	|r	|z	|o	|n	|k	|ó	|w	|k	|i	|*	|a	|*	|.
|*	|*	|*	|*	|*	|*	|[27][S]\rarr	|g	|a	|u	|g	|u	|i	|n	|*	|*	|e	|c	|n	|*	|k	|*	|.
|*	|*	|*	|*	|*	|*	|[28][S]\rarr	|k	|o	|b	|r	|a	|*	|*	|*	|*	|n	|y	|k	|[29][S]\darr	|b	|*	|.
|[30][S]\rarr	|p	|o	|d	|s	|t	|r	|o	|n	|a	|*	|*	|*	|*	|*	|*	|e	|z	|[][,]{ }	|s	|o	|*	|.
|*	|[31][S]\rarr	|c	|i	|ą	|g	|n	|i	|k	|*	|*	|*	|*	|*	|*	|*	|t	|o	|t	|z	|r	|*	|.
|*	|*	|*	|*	|*	|*	|[32][S]\rarr	|o	|b	|l	|i	|c	|ó	|w	|k	|a	|*	|w	|a	|r	|t	|*	|.
|*	|*	|*	|*	|*	|*	|*	|*	|*	|*	|*	|[33][S]\rarr	|s	|t	|e	|a	|r	|y	|n	|a	|*	|*	|.
|*	|*	|[34][S]\rarr	|w	|a	|r	|a	|n	|[][,]{ }	|b	|ł	|ę	|k	|i	|t	|n	|y	|*	|k	|f	|*	|*	|.
|[35][S]\rarr	|p	|a	|r	|t	|i	|a	|[][,]{ }	|k	|a	|t	|a	|l	|o	|ń	|s	|k	|a	|*	|*	|*	|*	|.\end{Puzzle}

\newpage

\begin{PuzzleClues}{\textbf{Poziome}\\}\Clue{1}{}{noga człowieka określana z niechęcią}
\Clue{2}{}{duży, obszerny pokój}
\Clue{3}{}{zewnętrzna warstwa ziemi, litosfery}
\Clue{4}{}{osadzona w gruncie dolna część budowli lub podstawa konstrukcji}
\Clue{5}{}{ciecz podgrzana do temperatury, w której zaczyna wrzeć; ciecz bardzo gorąca, powodująca oparzenia w kontakcie ze skórą}
\Clue{6}{}{francuski matematyk i inżynier (1718-98); twórca hydrauliki}
\Clue{7}{}{malarz włoski, przedstawiciel wczesnego baroku (1575-1642), obrazy, freski religijne i mityczne}
\Clue{8}{}{statek o specjalnej konstrukcji przystosowany do łamania pokrywy lodowej}
\Clue{9}{}{część urwanej liny wiertniczej}
\Clue{10}{}{zachodzenie, dokonywanie się czegoś, zwykle jakiegoś procesu}
\Clue{13}{}{krocionogi; roślinożerne stawonogi z gromady wijów, częste w ściółce leśnej}
\Clue{14}{}{część korony kwiatu}
\Clue{15}{}{coś przebojowego, najczęściej nowego, co robi karierę, ktoś to dobrze ocenia}
\Clue{16}{}{stalowa lub fiszbinowa listewka w gorsecie}
\Clue{17}{}{sieć zastawiona używana przez rybaków bałtyckich do połowu śledzi lub szprot}
\Clue{18}{}{początek, pierwszy etap procesu przetwarzania danych}
\Clue{19}{}{drewno z drzewa o tej samej nazwie}
\Clue{21}{}{rodzaj żelaznego dłuta służącego do żłobienia w drewnie}
\Clue{23}{}{paź}
\Clue{24}{}{zapaśnik irański w stylu wolnym, złoty medalista z Atlanty w wadze do 90 kg}
\Clue{26}{}{Apocrita - podrząd owadów z rzędu błonkówek; trzonkówki charakteryzują się przewężeniem między tułowiem a odwłokiem, przechodzą przeobrażenie zupełne}
\Clue{27}{}{samodzielne muzeum malarstwa, rzeźby lub salon wystawowy połączony ze sprzedażą dzieł sztuki}
\Clue{28}{}{jadowity wąż z jednego z kilku rodzajów zaliczanych do rodziny zdradnicowatych (Elapidae), występujący w Azji i Afryce, mający charakterystyczny zwyczaj unoszenia ciała i rozszerzania odcinka szyjnego}
\Clue{30}{}{szczegółowa, poświęcona węższemu obszarowi informacji część strony internetowej, tworzona w kodzie za pomocą poprzedzenia znakami/}
\Clue{31}{}{traktor; pojazd mechaniczny przystosowany do ciągnięcia pojazdów lub urządzeń nie posiadających własnego napędu}
\Clue{32}{}{okładzina na zewnątrz ściany budynku, wykonana może być z różnych rodzajów materiałów}
\Clue{33}{}{mieszanina kwasów: palmitynowego i stearynowego, używana jako surowiec do produkcji świec}
\Clue{34}{}{Varanus macraei - gatunek gada z rodziny waranowatych, występujący w Indonezji}
\Clue{35}{}{otwarcie szachowe zaczynające się od posunięć: 1. d4 Sf6, 2. c4 e6, 3. g3 d5, 4. Gg2}\end{PuzzleClues}

\begin{PuzzleClues}{\textbf{Pionowe}\\}\Clue{1}{}{rodzaj widowiska teatralnego lub filmowego, w którym podstawowym elementem i chwytem artystycznym są efekty świetlne i dźwiękowe, kreujące fikcyjny, fantastyczny świat przedstawiony}
\Clue{2}{}{porcja, zwykle butelka, także szklanka napoju tymbark (firmy Tymbark)}
\Clue{3}{}{apelacja - odwołanie się od wydanego wyroku sądowego}
\Clue{7}{}{konstrukcja stalowa do zmniejszenia prędkości bądź utrzymania statku w miejscu; rzucona na linie lub łańcuchu}
\Clue{9}{}{dobór barw w obrazie}
\Clue{11}{}{wyrób objęty podatkiem akcyzowym}
\Clue{12}{}{architekt włoski (1404-72), teoretyk sztuki epoki renesansu}
\Clue{14}{}{figura w jeździe figurowej na łyżwach, polegająca na skoku tyłem z wewnętrznej krawędzi lewej łyżwy}
\Clue{19}{}{szkoła - instytucja zajmująca się kształceniem dzieci i młodzieży}
\Clue{20}{}{organizacja pozarządowa zajmująca się badaniami i analizami dotyczącymi spraw publicznych}
\Clue{22}{}{określenie lewej burty statku}
\Clue{25}{}{przedstawiciel ludu celtyckiego zamieszkującego w starożytności Bretanię}
\Clue{29}{}{technika rysowania polegająca na kreśleniu kresek równoległych lub przecinających się, w celu wydobycia bryłowatości rysowanego przedmiotu}\end{PuzzleClues}\newpage\section*{Krzyżówka 167}

\noindent\begin{Puzzle}{24}{27}|*	|*	|*	|*	|*	|*	|[1][S]\darr	|*	|*	|*	|*	|*	|*	|*	|*	|*	|*	|*	|*	|*	|*	|*	|*	|*	|[2][S]\darr	|.
|*	|*	|*	|*	|*	|[3][S]\rarr	|t	|a	|c	|h	|o	|g	|r	|a	|f	|*	|*	|*	|*	|*	|[4][S]\darr	|*	|*	|*	|b	|.
|[5][S]\rarr	|m	|a	|r	|k	|e	|r	|[][,]{ }	|g	|e	|n	|e	|t	|y	|c	|z	|n	|y	|*	|*	|p	|[6][S]\darr	|*	|*	|ł	|.
|[7][S]\rarr	|a	|n	|a	|l	|i	|z	|a	|[][,]{ }	|f	|r	|a	|z	|o	|w	|a	|*	|*	|[8][S]\darr	|*	|a	|r	|*	|[9][S]\darr	|ę	|.
|[10][S]\rarr	|s	|i	|e	|d	|e	|m	|d	|z	|i	|e	|s	|i	|ą	|t	|k	|a	|*	|ł	|*	|p	|z	|*	|o	|d	|.
|*	|*	|[11][S]\rarr	|m	|a	|g	|i	|e	|r	|k	|a	|*	|*	|*	|*	|*	|*	|*	|u	|*	|a	|y	|*	|d	|n	|.
|*	|*	|*	|[12][S]\darr	|[13][S]\darr	|[14][S]\drarr	|e	|s	|t	|e	|y	|a	|*	|[15][S]\drarr	|a	|s	|p	|i	|s	|*	|d	|m	|*	|j	|i	|.
|[16][S]\rarr	|s	|a	|r	|d	|e	|l	|a	|[][,]{ }	|p	|e	|r	|u	|w	|i	|a	|ń	|s	|k	|a	|*	|i	|*	|a	|k	|.
|*	|[17][S]\darr	|*	|e	|z	|c	|[][,]{ }	|[18][S]\rarr	|m	|a	|g	|a	|z	|y	|n	|*	|[19][S]\rarr	|w	|a	|r	|g	|a	|c	|z	|*	|.
|*	|m	|[20][S]\darr	|j	|i	|c	|p	|[21][S]\rarr	|m	|e	|l	|o	|d	|r	|a	|m	|a	|t	|*	|*	|*	|n	|[22][S]\darr	|d	|*	|.
|*	|i	|w	|e	|a	|h	|a	|*	|*	|*	|*	|[23][S]\rarr	|k	|a	|r	|t	|o	|f	|l	|a	|n	|k	|a	|*	|[24][S]\darr	|.
|*	|o	|e	|s	|ł	|i	|r	|*	|*	|*	|*	|[25][S]\drarr	|d	|ż	|i	|h	|a	|d	|*	|*	|*	|a	|p	|*	|w	|.
|*	|d	|k	|t	|k	|*	|k	|[26][S]\drarr	|p	|r	|o	|b	|l	|e	|m	|o	|w	|o	|ś	|ć	|*	|*	|i	|[27][S]\darr	|e	|.
|*	|o	|a	|r	|a	|*	|o	|f	|[28][S]\rarr	|ś	|c	|i	|a	|n	|a	|*	|*	|*	|*	|*	|*	|[29][S]\darr	|o	|p	|l	|.
|*	|j	|[][,]{ }	|a	|*	|*	|w	|l	|*	|[30][S]\drarr	|p	|o	|l	|i	|f	|a	|g	|i	|*	|*	|*	|m	|n	|ę	|i	|.
|*	|a	|p	|c	|[31][S]\darr	|*	|y	|a	|*	|g	|[32][S]\rarr	|p	|r	|e	|s	|k	|r	|y	|p	|c	|j	|a	|*	|c	|t	|.
|*	|d	|o	|j	|p	|[33][S]\darr	|*	|m	|[34][S]\rarr	|o	|k	|a	|p	|*	|*	|*	|*	|[35][S]\darr	|*	|*	|*	|n	|[36][S]\darr	|z	|a	|.
|*	|[][,]{ }	|ł	|a	|ó	|b	|[37][S]\darr	|b	|[38][S]\rarr	|r	|e	|l	|a	|c	|j	|a	|[][,]{ }	|t	|o	|t	|a	|l	|n	|a	|*	|.
|*	|w	|u	|*	|ł	|i	|k	|i	|*	|s	|*	|i	|[39][S]\rarr	|p	|i	|s	|m	|o	|*	|*	|[40][S]\darr	|i	|e	|r	|*	|.
|*	|y	|d	|[41][S]\darr	|s	|m	|a	|r	|*	|*	|*	|w	|*	|*	|[42][S]\rarr	|k	|a	|n	|d	|*	|i	|c	|r	|s	|*	|.
|*	|s	|n	|r	|e	|b	|p	|o	|[43][S]\rarr	|l	|h	|o	|t	|k	|a	|*	|*	|g	|*	|*	|n	|h	|u	|k	|*	|.
|[44][S]\drarr	|p	|i	|o	|t	|r	|o	|w	|s	|k	|i	|*	|*	|*	|*	|*	|*	|h	|*	|*	|t	|e	|d	|i	|*	|.
|h	|o	|o	|m	|e	|o	|t	|a	|[45][S]\rarr	|g	|a	|z	|e	|l	|a	|[][,]{ }	|c	|u	|v	|i	|e	|r	|a	|*	|*	|.
|a	|w	|w	|n	|k	|w	|*	|n	|[46][S]\rarr	|p	|s	|a	|l	|m	|o	|d	|i	|a	|*	|*	|r	|*	|*	|*	|*	|.
|k	|y	|a	|y	|*	|n	|*	|i	|*	|*	|*	|[47][S]\rarr	|d	|e	|o	|n	|i	|*	|*	|*	|i	|*	|*	|*	|*	|.
|*	|*	|*	|*	|*	|i	|*	|e	|*	|[48][S]\rarr	|b	|e	|z	|p	|a	|n	|c	|e	|r	|z	|o	|w	|c	|e	|*	|.
|[49][S]\rarr	|w	|i	|d	|ł	|a	|k	|*	|*	|*	|*	|*	|*	|*	|*	|[50][S]\rarr	|l	|a	|b	|o	|r	|a	|n	|t	|*	|.
|[51][S]\rarr	|w	|i	|l	|k	|*	|[52][S]\rarr	|a	|n	|t	|r	|o	|p	|o	|l	|o	|g	|i	|z	|m	|*	|*	|*	|*	|*	|.\end{Puzzle}

\newpage

\begin{PuzzleClues}{\textbf{Poziome}\\}\Clue{3}{}{urządzenie pomiarowe łączące w sobie funkcje prędkościomierza i zegara, które rejestruje w funkcji czasu: przejechaną przez pojazd drogę, jego chwilową prędkość, jak również tzw. aktywność kierowcy, czyli okresy jego pracy, dyżuru przy pojeździe czy odpoczynku}
\Clue{5}{}{charakterystyczna właściwość organizmu wykorzystywana do określenia jego genotypu}
\Clue{7}{}{model analizy języka naturalnego, w którym struktura zdania jest przedstawiona za pomocą zagnieżdżonych składników, oznaczonych przez swoją syntaktyczną kategorię}
\Clue{10}{}{grupa obiektów, która składa się z 70 elementów}
\Clue{11}{}{czapka węgierska o płaskiej, okrągłej lub czworokątnej główce, ozdobiona piórem lub kitą}
\Clue{14}{}{amerykańskie organy}
\Clue{15}{}{jeleń z Półwyspu indyjskiego}
\Clue{16}{}{Sardela chilijska, Engraulis ringens - gatunek morskiej ryby z rodziny sardelowatych (Engraulidae)}
\Clue{18}{}{czasopismo (najczęściej tygodnik), będące uzupełnieniem codziennych wydań określonej gazety, wydawane przez tego samego wydawcę co gazeta, często drukowane na lepszym papierze i o bogatszej zawartości graficznej niż wydania codzienne}
\Clue{19}{}{czarny niedźwiedź z małym półksiężycem na piersi - Płw. Indyjski, Cejlon}
\Clue{21}{}{jakieś dramatyczne wydarzenie w życiu, przedstawione w sposób przejaskrawiony}
\Clue{23}{}{Allium cepa var. aggregatum - znana (zwłaszcza w Europie Południowej) odmiana cebuli zwyczajnej o czerwonawych i lekko spłaszczonych cebulkach}
\Clue{25}{}{walka zbrojna, którą - w celu szerzenia lub obrony wiary - toczą muzułmanie z nieprzyjaciółmi islamu; czasem za głównego wroga islamu uznaje się Zachód}
\Clue{26}{}{bycie problemowym}
\Clue{28}{}{wall - profil kogoś/czegoś na facebooku}
\Clue{30}{}{zwierzęta wszystkożerne}
\Clue{32}{}{recepta lekarska, porada, przepis}
\Clue{34}{}{dolna, pozioma krawędź dachu, zwykle wysunięta przed płaszczyznę elewacji}
\Clue{38}{}{relacja obejmująca wszystkie elementy zbioru na którym jest rozpatrywana}
\Clue{39}{}{umiejętność wyrażania myśli znakami graficznymi, pisanie}
\Clue{42}{}{ciasto miodowo-cukrowe podawane pszczelim matkom}
\Clue{43}{}{kompozytor i dyrygent jugosłowiański zm. w 1962 r}
\Clue{44}{}{Stefan, ur. w 1910r. astronom; prace z astrofizyki}
\Clue{45}{}{gazela edmi, Gazella cuvieri - ssak kopytny z rodziny krętorogich; zamieszkuje północną Afrykę}
\Clue{46}{}{melodia śpiewanego psalmu, śpiewanie psalmów}
\Clue{47}{}{rasa zebu}
\Clue{48}{}{Anostraca - skorupiaki występujące na wszystkich kontynentach, we wszystkich strefach zoogeograficznych; dotychczas opisano ponad 300 gatunków, jednakże około 1/4 z nich znana jest tylko z locus typicus lub mniej niż 3 lokacji; najliczniejszą w gatunki jest Palearktyka z centrum bioróżnorodności na Bałkanach; zwyczajowo zaliczane do sztucznej grupy dużych skrzelonogów}
\Clue{49}{}{Lycopodium - rodzaj roślin zaliczanych do rodziny widłakowatych}
\Clue{50}{}{jeden z pracowników apteki, farmaceuta zajmujący się m.in. przygotowywaniem lekarstw}
\Clue{51}{}{inna nazwa tocznia - przewlekłej choroby autoimmunologicznej}
\Clue{52}{}{nurt filozoficzny, w myśl którego człowiek (rozumiany jako najwyższy poziom w hierarchii wytworów przyrody) rozwija się nie dzięki powiązaniom międzyludzkim i relacjom ze środowiskiem, ale dzięki posiadanym cechom biologicznym}\end{PuzzleClues}

\begin{PuzzleClues}{\textbf{Pionowe}\\}\Clue{1}{}{trzmiel drzewny, Bombus hypnorum - gatunek trzmiela występujący w Europie i w pewnych częściach Azji; niedawno zasięg występowania trzemila drzewnego rozszerzył się również na Wielką Brytanię i Islandię}
\Clue{2}{}{LABIRYNT}
\Clue{4}{}{cienki i chrupki indyjski naleśnik, zazwyczaj robiony z mąki z soczewicy, ciecierzycy, odmiany czarnej fasoli zwanej urad lub mąki ryżowej}
\Clue{6}{}{mieszkanka starożytnego państwa rzymskiego}
\Clue{8}{}{część naboju służąca do umieszczenia w niej ładunku miotającego i spłonki}
\Clue{9}{}{ruszenie w drogę jakimś pojazdem}
\Clue{12}{}{biuro, w którym dokonuje się rejstracji, wpisu, zgłoszeń}
\Clue{13}{}{część kielicha kwiatu}
\Clue{14}{}{gatunek mangi, obfitujący w podteksty seksualne i często prezentujący kobiety w bieliźnie albo nagie, jednak bez pokazywania detali anatomicznych; termin ecchi stosowany jest na całym świecie przez fanów mangi i anime do określenia tematów lub podtekstów seksualnych}
\Clue{15}{}{wyraz lub grupa wyrazów połączonych ze sobą zgodnie z regułami językowymi i coś opisujących, określających, wyrażających}
\Clue{17}{}{Meliphaga vicina - gatunek ptaka z rodziny miodojadów (Meliphagidae), który zamieszkuje wschodnie wybrzeże Australii}
\Clue{20}{}{Gallirallus australis australis - podgatunek nielotnego ptaka wyróżniony w obrębie gatunku weka (Gallirallus australis)}
\Clue{22}{}{pędruś}
\Clue{24}{}{najmłodszy i najmniej doświadczony żołnierz wchodzący w skład legionu}
\Clue{25}{}{paliwo powstałe z przetwórstwa biomasy — produktów organizmów żywych np. roślinnych, zwierzęcych czy mikroorganizmów}
\Clue{26}{}{proces polegający na podpaleniu potrawy w trakcie jej przygotowywania}
\Clue{27}{}{malarz, sztycharz (1804-62) sceny rodzajowo-obyczajowe, obrazy religijne}
\Clue{29}{}{austriacki karabin powtarzalny o dwutaktowym zamku}
\Clue{30}{}{przednia część koszuli męskiej}
\Clue{31}{}{miara długości wynosząca 50 łokci}
\Clue{33}{}{miejsce, gdzie pędzi się bimber}
\Clue{35}{}{miasto w Chinach (Jilin); wydobycie węgla kamiennego, rud miedzi; przemysł metalowy maszynowy, chemiczny}
\Clue{36}{}{Pablo (1904-73), poeta chilijski; „Winogrona i wiatr”, „Sława i śmierć Joachima Muriety”, „Wyznaję, że żyłem” - Leninowska nagroda pokoju, Nobel 1971}
\Clue{37}{}{osłona wystających elementów płatowca zmniejszająca współczynnik oporu aerodynamicznego}
\Clue{40}{}{słabo zagospodarowana część kontynentu lub państwa, oddalona od wybrzeża morskiego oraz ośrodków przemysłowych}
\Clue{41}{}{miasto na Ukrainie nad Sułą}
\Clue{44}{}{metalowy pręt zgięty na końcu, o różnych kształtach i różnorakim zastosowaniu}\end{PuzzleClues}\newpage\section*{Krzyżówka 168}

\noindent\begin{Puzzle}{19}{23}|*	|*	|*	|[1][S]\darr	|[2][S]\drarr	|m	|a	|e	|b	|a	|s	|h	|i	|*	|[3][S]\darr	|[4][S]\drarr	|l	|u	|t	|*	|.
|*	|*	|*	|k	|e	|[5][S]\darr	|[6][S]\drarr	|a	|l	|l	|a	|i	|n	|*	|k	|p	|*	|*	|*	|[7][S]\darr	|.
|*	|[8][S]\darr	|*	|i	|n	|p	|s	|*	|*	|*	|[9][S]\darr	|[10][S]\darr	|*	|*	|a	|i	|*	|*	|*	|p	|.
|*	|r	|*	|m	|g	|l	|z	|*	|*	|*	|h	|h	|*	|*	|t	|e	|*	|*	|*	|y	|.
|*	|e	|*	|o	|a	|e	|e	|*	|*	|*	|e	|o	|*	|*	|a	|p	|[11][S]\darr	|[12][S]\darr	|[13][S]\darr	|ł	|.
|*	|p	|*	|g	|g	|m	|s	|*	|*	|*	|r	|r	|[14][S]\darr	|*	|r	|r	|w	|k	|k	|ó	|.
|*	|r	|[15][S]\darr	|r	|e	|i	|n	|*	|*	|[16][S]\darr	|t	|r	|g	|[17][S]\darr	|c	|z	|ą	|r	|o	|w	|.
|*	|o	|p	|a	|m	|e	|a	|*	|*	|m	|z	|o	|a	|d	|z	|[][,]{ }	|s	|o	|n	|k	|.
|[18][S]\drarr	|d	|e	|f	|e	|n	|s	|y	|w	|a	|*	|r	|r	|r	|y	|k	|a	|k	|t	|a	|.
|d	|u	|r	|*	|n	|n	|t	|[19][S]\darr	|[20][S]\rarr	|r	|e	|e	|d	|u	|k	|a	|t	|o	|r	|*	|.
|e	|k	|f	|*	|t	|i	|a	|p	|*	|u	|*	|k	|z	|k	|*	|j	|k	|d	|m	|[21][S]\darr	|.
|k	|c	|o	|*	|*	|k	|*	|i	|*	|d	|*	|*	|i	|*	|*	|e	|a	|y	|a	|k	|.
|o	|j	|r	|*	|[22][S]\darr	|*	|*	|z	|*	|e	|*	|[23][S]\rarr	|c	|i	|e	|ń	|*	|l	|n	|o	|.
|m	|a	|a	|[24][S]\drarr	|f	|o	|n	|o	|g	|r	|a	|f	|i	|a	|*	|s	|[25][S]\darr	|[][,]{ }	|i	|p	|.
|u	|[][,]{ }	|c	|s	|o	|[26][S]\drarr	|g	|i	|p	|s	|*	|*	|e	|[27][S]\darr	|*	|k	|d	|r	|f	|i	|.
|n	|p	|j	|ę	|n	|s	|*	|d	|*	|t	|*	|*	|l	|z	|*	|i	|ż	|ó	|e	|e	|.
|i	|r	|a	|p	|*	|p	|*	|*	|*	|w	|*	|*	|*	|ł	|*	|*	|e	|ż	|s	|c	|.
|z	|o	|*	|*	|*	|ł	|*	|*	|*	|o	|*	|[28][S]\rarr	|a	|u	|t	|o	|m	|a	|t	|*	|.
|a	|s	|*	|[29][S]\rarr	|s	|y	|j	|o	|n	|*	|[30][S]\rarr	|r	|a	|d	|a	|*	|*	|ń	|a	|*	|.
|t	|t	|[31][S]\rarr	|a	|k	|w	|a	|t	|i	|n	|t	|a	|*	|a	|*	|*	|*	|c	|c	|*	|.
|o	|a	|*	|*	|*	|*	|*	|*	|*	|*	|*	|*	|*	|*	|*	|*	|*	|o	|j	|*	|.
|r	|*	|[32][S]\rarr	|d	|r	|u	|ż	|y	|n	|a	|[][,]{ }	|s	|k	|a	|u	|t	|o	|w	|a	|*	|.
|*	|*	|*	|*	|[33][S]\rarr	|b	|i	|s	|k	|u	|p	|[][,]{ }	|p	|o	|l	|o	|w	|y	|*	|*	|.
|*	|*	|*	|*	|*	|[34][S]\rarr	|d	|z	|i	|k	|a	|[][,]{ }	|k	|a	|r	|t	|a	|*	|*	|*	|.\end{Puzzle}

\newpage

\begin{PuzzleClues}{\textbf{Poziome}\\}\Clue{2}{}{miasto w Japonii (środkowe Honsiu), ośrodek administracyjny prefektury Gumma}
\Clue{4}{}{stop do lutowania}
\Clue{6}{}{organistka francuska ur. w 1926 r., znana wykonawczyni utworów Bacha i kompozytorów francuskich}
\Clue{18}{}{odpieranie ataku, bronienie się}
\Clue{20}{}{człowiek, który pomaga w resocjalizacji m.in. przestępców i ludzi uzależnionych}
\Clue{23}{}{odrobina, ślad}
\Clue{24}{}{utrwalanie dźwięków na płytach gramofonowych i taśmach magnetycznych}
\Clue{26}{}{trudna, kłopotliwa i nieprzyjemna sytuacja, problem}
\Clue{28}{}{ciąg zdarzeń, kiedy coś się dzieje automatycznie, jeden fakt pociąga za sobą drugi}
\Clue{29}{}{Ziemia Obiecana, biblijne miejsce, do którego Mojżesz prowadził Izraelitów po wyprowadzeniu ich z Egiptu i przeprowadzeniu przez Morze Czerwone}
\Clue{30}{}{grupa ludzi, wybrana, żeby radzić, rządzić (np. rada starszych)}
\Clue{31}{}{odmiana techniki druku wklęsłego zbliżona do akwaforty, niegdyś stosowana jako jedna z metod odtwarzania obrazów i rysunków, dzisiaj wykorzystywana już tylko jako technika artystyczna, a więc zaliczana do grafiki warsztatowej}
\Clue{32}{}{zorganizowana grupa harcerzy}
\Clue{33}{}{biskup wojskowy}
\Clue{34}{}{dopuszczenie do zawodów, zwykle rangi mistrzowskiej, drużyny lub zawodnika niespełniającego formalnych warunków uczestnictwa}\end{PuzzleClues}

\begin{PuzzleClues}{\textbf{Pionowe}\\}\Clue{1}{}{urządzenie służące do rejestracji a następnie przedstawiania graficznego w postaci kimogramu (specjalnego wykresu) różnych rodzajów procesów fizjologicznych (np. bicie serca, ruchy mięśni itp.) i fizycznych (np. zmian ciśnienia atmosferycznego)}
\Clue{2}{}{W środowisku artystycznym, mi.n wśród aktorów: angaż, umowa. (Wymowa wyrazu - francuska)}
\Clue{3}{}{mieszkaniec Kataru, człowiek pochodzenia katarskiego}
\Clue{4}{}{przyprawa, rodzaj pieprzu, wyrabiany z bardzo ostrej papryki Cayenne}
\Clue{5}{}{członek plemienia}
\Clue{6}{}{godzina - czwarta po południu}
\Clue{7}{}{lawina ze świeżo spadłego, sypkiego śniegu}
\Clue{8}{}{jeden z typów reprodukcji, kiedy liczba dzieci równa jest liczbie rodziców}
\Clue{9}{}{fizyk niemiecki (1857-94); twórca podstaw radiokomunikacji}
\Clue{10}{}{horror literacki, utwór utrzymany w konwencji fantastyki grozy}
\Clue{11}{}{Penstemon - odmiana pszenicy o kłosie z długimi wąsami}
\Clue{12}{}{Crocodylus porosus - gatunek gada z rodziny krokodylowatych, największy współcześnie żyjący gad, obszar jego występowania obejmuje Indie, Sri Lankę, Azję Południowo-Wschodnią, Filipiny, północną Australię, Cejlon, Sumatrę, Jawę, Borneo, Fidżi}
\Clue{13}{}{grupa ludzi, którzy biorą udział w kontrmanifestacji - manifestacji, która jest formą wyrazu sprzeciwu dla innej manifestacji}
\Clue{14}{}{człowiek, którego charakteryzuje pogarda dla osób, stanów, cech, sytuacji itp}
\Clue{15}{}{wykonywanie w jakimś materialne serii regularnych otworów w celu realizacji określonej funkcji, np. ułatwienia przesuwania, odrywania, wpinania w segregator, mocowania, przepuszczania jakichś substancji, zasiewu, programowania procesów obliczeniowych itp}
\Clue{16}{}{to, że ktoś marudzi i narzeka, ociągając się i odwlekając przez to zrobienie czegoś}
\Clue{17}{}{krój liter w druku, kształt czcionki}
\Clue{18}{}{osoba przeprowadzająca dekomunizację}
\Clue{19}{}{twór skalny o kształcie kuli, o średnicy powyżej 2 mm, powstający w wyniku chemicznego wytrącania się minerałów wokół okruchowego jądra}
\Clue{21}{}{wzniesienie terenu, często utworzone sztucznie przez człowieka (choć niekoniecznie)}
\Clue{22}{}{jednostka poziomu głośności dźwięku}
\Clue{24}{}{ŚCIERWNIK; drapieżny ptak o nagiej głowie i szyi; padlinożerny o mocnych, długich skrzydłach, niektóre gatunki chronione}
\Clue{25}{}{JAM}
\Clue{26}{}{przepłynięcie turystycznego szlaku wodnego}
\Clue{27}{}{urojenie, żywione przekonanie o czymś albo o kimś, które nie ma potwierdzenia w rzeczywistości}\end{PuzzleClues}\newpage\section*{Krzyżówka 169}

\noindent\begin{Puzzle}{22}{33}|*	|*	|*	|*	|*	|*	|*	|[1][S]\drarr	|d	|ł	|u	|g	|[][,]{ }	|p	|u	|b	|l	|i	|c	|z	|n	|y	|*	|.
|*	|*	|[2][S]\drarr	|m	|i	|n	|i	|a	|t	|u	|r	|k	|a	|*	|*	|*	|[3][S]\darr	|*	|*	|*	|*	|*	|*	|.
|[4][S]\drarr	|f	|e	|l	|d	|m	|a	|r	|s	|z	|a	|ł	|e	|k	|*	|*	|b	|*	|*	|*	|*	|*	|[5][S]\darr	|.
|c	|*	|g	|*	|*	|*	|*	|c	|*	|*	|[6][S]\drarr	|g	|r	|z	|y	|b	|i	|e	|ń	|*	|*	|[7][S]\darr	|r	|.
|h	|*	|z	|*	|*	|*	|*	|h	|*	|[8][S]\drarr	|p	|t	|a	|s	|z	|y	|n	|i	|e	|c	|*	|z	|o	|.
|o	|*	|o	|[9][S]\darr	|[10][S]\rarr	|n	|i	|e	|u	|m	|i	|e	|j	|ę	|t	|n	|o	|ś	|ć	|*	|*	|w	|z	|.
|r	|*	|r	|j	|*	|[11][S]\darr	|*	|o	|*	|a	|e	|*	|[12][S]\rarr	|s	|i	|t	|k	|a	|*	|[13][S]\darr	|*	|ł	|m	|.
|o	|[14][S]\drarr	|c	|a	|r	|b	|a	|l	|*	|r	|s	|*	|*	|[15][S]\darr	|*	|*	|l	|*	|[16][S]\darr	|f	|*	|ó	|ó	|.
|b	|p	|y	|r	|*	|i	|*	|o	|[17][S]\darr	|c	|[][,]{ }	|*	|*	|f	|[18][S]\darr	|*	|e	|*	|k	|o	|*	|k	|w	|.
|a	|e	|z	|z	|*	|c	|*	|g	|k	|o	|c	|*	|*	|i	|r	|*	|*	|*	|o	|s	|*	|n	|k	|.
|[][,]{ }	|s	|m	|ą	|*	|a	|*	|i	|o	|n	|i	|*	|*	|l	|ą	|*	|*	|*	|n	|f	|[19][S]\darr	|i	|a	|.
|u	|z	|*	|b	|[20][S]\rarr	|z	|n	|a	|m	|i	|ę	|[][,]{ }	|n	|a	|c	|z	|y	|n	|i	|o	|w	|e	|*	|.
|n	|t	|*	|[][,]{ }	|*	|*	|*	|[][,]{ }	|e	|*	|t	|*	|*	|k	|z	|*	|*	|[21][S]\darr	|c	|r	|r	|n	|*	|.
|v	|*	|*	|g	|*	|*	|*	|p	|n	|*	|y	|*	|*	|t	|ę	|[22][S]\darr	|*	|l	|z	|[][,]{ }	|ą	|i	|*	|.
|e	|[23][S]\drarr	|p	|r	|z	|y	|g	|o	|d	|a	|*	|*	|*	|e	|t	|n	|*	|a	|e	|b	|b	|e	|*	|.
|r	|ż	|*	|e	|*	|*	|*	|d	|a	|*	|[24][S]\drarr	|u	|k	|r	|a	|i	|ń	|s	|k	|i	|*	|*	|*	|.
|r	|e	|*	|c	|*	|*	|*	|w	|*	|*	|a	|*	|*	|i	|*	|e	|*	|e	|*	|a	|*	|*	|*	|.
|i	|l	|*	|k	|*	|[25][S]\drarr	|p	|o	|p	|y	|c	|h	|l	|e	|*	|k	|*	|c	|*	|ł	|*	|*	|*	|.
|c	|a	|*	|i	|[26][S]\darr	|w	|*	|d	|*	|[27][S]\darr	|c	|*	|*	|*	|*	|o	|*	|z	|*	|y	|*	|*	|*	|.
|h	|z	|*	|*	|t	|z	|*	|n	|[28][S]\rarr	|l	|e	|p	|i	|ę	|ż	|n	|i	|k	|*	|*	|*	|*	|*	|.
|t	|o	|*	|[29][S]\drarr	|e	|d	|e	|a	|*	|i	|l	|[30][S]\drarr	|o	|m	|a	|s	|t	|a	|*	|*	|*	|*	|*	|.
|a	|w	|[31][S]\drarr	|k	|o	|ł	|o	|*	|*	|c	|e	|p	|*	|*	|*	|t	|*	|[][,]{ }	|*	|*	|*	|[32][S]\darr	|*	|.
|[][S]-	|i	|h	|a	|r	|u	|*	|*	|*	|e	|r	|r	|*	|*	|*	|y	|*	|w	|*	|*	|*	|g	|*	|.
|l	|e	|a	|r	|b	|ż	|*	|*	|*	|n	|a	|a	|*	|*	|*	|t	|*	|ą	|[33][S]\darr	|[34][S]\darr	|*	|e	|*	|.
|u	|c	|r	|e	|a	|n	|*	|*	|*	|c	|n	|w	|*	|*	|*	|u	|*	|g	|p	|m	|*	|n	|*	|.
|n	|*	|d	|l	|n	|i	|*	|*	|*	|j	|d	|o	|*	|*	|*	|c	|*	|l	|a	|u	|*	|e	|*	|.
|d	|*	|[][,]{ }	|k	|i	|k	|*	|*	|*	|a	|o	|z	|*	|*	|*	|y	|*	|i	|z	|ż	|*	|r	|*	|.
|b	|*	|c	|a	|s	|*	|*	|*	|*	|*	|*	|n	|*	|*	|*	|j	|[35][S]\rarr	|k	|i	|o	|g	|a	|*	|.
|o	|*	|o	|*	|t	|*	|*	|*	|*	|*	|*	|a	|*	|*	|*	|n	|*	|a	|o	|c	|*	|ł	|*	|.
|r	|*	|r	|*	|a	|*	|*	|*	|*	|*	|*	|w	|*	|*	|*	|o	|*	|*	|w	|j	|*	|*	|*	|.
|g	|*	|e	|*	|*	|*	|*	|*	|*	|*	|*	|c	|*	|*	|*	|ś	|*	|*	|a	|a	|*	|*	|*	|.
|a	|*	|*	|*	|*	|*	|*	|*	|*	|*	|*	|a	|*	|*	|*	|ć	|*	|*	|t	|*	|*	|*	|*	|.
|*	|*	|*	|*	|*	|*	|*	|*	|*	|*	|*	|*	|*	|*	|*	|*	|*	|*	|e	|*	|*	|*	|*	|.
|*	|*	|*	|*	|*	|*	|*	|*	|*	|*	|*	|*	|*	|*	|*	|*	|*	|*	|*	|*	|*	|*	|*	|.\end{Puzzle}

\newpage

\begin{PuzzleClues}{\textbf{Poziome}\\}\Clue{1}{}{nominalne zadłużenie podmiotów sektora finansów publicznych zaciągnięte z tytułu: papierów wartościowych, pożyczek, kredytów, przykętych depozytów i zobowiązań wymagalnych}
\Clue{2}{}{zdrobniale: miniatura: bardzo małych rozmiarów obraz, często portret, wykonany na pergaminie, kości słoniowej, metalu czy porcelanie}
\Clue{4}{}{osoba nosząca tytuł lub stopień feldmarszałka}
\Clue{6}{}{lilia wodna - wodna bylina o dużych kwiatach i tarczowatych liściach}
\Clue{8}{}{kleszcz z rzędu żukowców}
\Clue{10}{}{cecha człowieka nieporadnego, niezdarnego, niezaradnego}
\Clue{12}{}{miasto w USA (Alaska) na Wyspie Baranowa, port rybacki}
\Clue{14}{}{żeglarz portugalski (1468-1520); odkrył wschodnie wybrzeża Brazylii}
\Clue{20}{}{uszkodzenie lub zmiana w drobnych naczyniach krwionośnych, które powstaje najczęściej jako efekt występowania problemów hormonalnych, nadciśnienia tętniczego i problemów z krążeniem; może mieć charakter zarówno wrodzony, jak i nabyty}
\Clue{23}{}{fascynacja, coś, czym człowiek się interesuje i zajmuje, często związane z przeżyciami}
\Clue{24}{}{przedmiot szkolny lub uczony w ramach kursu, na którym opanowuje się podstawy języka ukraińskiego}
\Clue{25}{}{popychadło, osoba mało zaradna i lekceważona przez innych}
\Clue{28}{}{Petasites - rodzaj bylin z rodziny astrowatych rosnący w strefie umiarkowanej}
\Clue{29}{}{miasto w płd.-zach. Kamerunie nad rzeką Sanaca; huta aluminium, elektrownia wodna}
\Clue{30}{}{OKRASA}
\Clue{31}{}{kolokwium, sprawdzian na studiach}
\Clue{35}{}{jezioro w Ugandzie, na wysokości 1033 m, powierzchnia w zależności od pory roku 2590-6000 km , przez to jezioro przepływa Nil Wiktorii}\end{PuzzleClues}

\begin{PuzzleClues}{\textbf{Pionowe}\\}\Clue{1}{}{odłam archeologii, który zajmuje się badaniami archeologicznymi prowadzonymi w środowisku wodnym}
\Clue{2}{}{zaklęcie, które ma na celu uwolnienie człowieka, zwierzęcia, miejsca lub przedmiotu od wpływu groźnych sił nadprzyrodzonych}
\Clue{3}{}{odmiana okularów trzymających się na nosie dzięki specjalnie wygiętemu sprężynującemu elementowi łączącemu oba szkła}
\Clue{4}{}{choroba genetyczna objawiająca się atakami padaczkowymi, nagłymi skurczami poszczególnych grup mięśniowych i otępieniem}
\Clue{5}{}{luźna rozmowa}
\Clue{6}{}{pies myśliwski dobrze atakujący zwierzynę}
\Clue{7}{}{zwłókniała tkanka łączna, nadmiernie rozrośnięta wskutek stanu zapalnego, urazu, zakażenia, zaburzeń ukrwienia}
\Clue{8}{}{Henryk (1792-1863), polski architekt włoskiego pochodzenia - Hotel Europejski}
\Clue{9}{}{gatunek rośliny z rodziny różowatych. Według nowszych ujęć taksonomicznych jest to synonim gatunku Sorbus umbellata FritschA. Kerner, Sched}
\Clue{11}{}{miasto w Rumunii, w okręgu Neamt w Karpatach Wsch}
\Clue{13}{}{najaktywniejsza odmiana alotropowa fosforu}
\Clue{14}{}{węgierskie miasto, które od XIX w. pełniło funkcję stolicy kraju; obecnie część Budapesztu}
\Clue{15}{}{pergaminowe zwitki z cytatami z Pięcioksięgu, które wyznawcy judaizmu umieszczają na lewym ramieniu i na czole podczas modlitwy}
\Clue{16}{}{pieszczotliwie o koniu - zwierzęciu}
\Clue{17}{}{słowny rozkaz; także siedziba dowództwa}
\Clue{18}{}{pieszczotliwie o małych rękach, dłoniach, przede wszystkim dziecięcych}
\Clue{19}{}{wąska szczelina wycięta w caliźnie celem ułatwienia urabiania kopaliny, np. węgla}
\Clue{21}{}{Bacillus anthracis - Gram-dodatnia bakteria, która wywołuje wąglika}
\Clue{22}{}{niezgodność z konstytucją, sprzeczność z zapisami konstytucji}
\Clue{23}{}{pierwiastek należący do ósmej grupy układu okresowego pierwiastków}
\Clue{24}{}{określenie wykonawcze; stopniowo przyspieszając coraz szybciej}
\Clue{25}{}{wiązanie wzdłużne szkieletu statku}
\Clue{26}{}{muzykant, instumentalista grający na teorbanie}
\Clue{27}{}{urzędowe pozwolenie na wykonywanie jakichś czynności, jakiegoś zawodu}
\Clue{29}{}{przedstawicielka ugrofińskiego ludu Karelów, którzy zamieszkują Karelię na pograniczu Finlandii i Rosji}
\Clue{30}{}{naukowiec zajmujący się naukami prawnymi}
\Clue{31}{}{nurt w muzyce punkowej charakteryzujący się ekstremalnie szybkim tempem i ekspresyjnymi tekstami (często dotyczącymi polityki i ideologii)}
\Clue{32}{}{główny przełożony zakonu wybierany przez kapitułę na kilka lat lub dożywotnio}
\Clue{33}{}{rodzina dużych motyli dziennych}
\Clue{34}{}{przedstawiciel glonów z gromady zielenic (Chlorophyta) rodzaju Mougeotia występujący w wodach słodkich i słonawych Eurazji, Afryki, Ameryki i Australii}\end{PuzzleClues}\newpage\section*{Krzyżówka 170}

\noindent\begin{Puzzle}{17}{33}|*	|*	|*	|[1][S]\darr	|*	|*	|*	|*	|*	|*	|*	|*	|*	|*	|*	|*	|*	|*	|.
|*	|*	|*	|c	|*	|*	|[2][S]\darr	|*	|[3][S]\darr	|*	|*	|*	|*	|*	|*	|*	|[4][S]\darr	|[5][S]\darr	|.
|*	|[6][S]\rarr	|r	|i	|v	|a	|s	|*	|a	|[7][S]\drarr	|k	|a	|n	|i	|a	|*	|d	|p	|.
|*	|*	|*	|ą	|*	|*	|a	|*	|p	|c	|*	|*	|*	|[8][S]\darr	|*	|*	|o	|r	|.
|*	|*	|*	|g	|*	|*	|m	|*	|l	|h	|*	|*	|*	|s	|*	|*	|j	|z	|.
|*	|*	|*	|n	|*	|*	|o	|*	|i	|o	|*	|*	|*	|c	|*	|*	|e	|e	|.
|*	|*	|[9][S]\darr	|i	|*	|*	|t	|*	|k	|ł	|*	|*	|*	|h	|*	|*	|ż	|g	|.
|*	|*	|a	|k	|*	|*	|n	|*	|a	|o	|*	|*	|*	|r	|*	|*	|d	|u	|.
|*	|*	|r	|*	|*	|[10][S]\darr	|i	|*	|n	|d	|*	|*	|*	|o	|*	|*	|ż	|b	|.
|*	|*	|a	|*	|*	|e	|c	|[11][S]\drarr	|t	|e	|n	|d	|e	|n	|c	|j	|a	|*	|.
|*	|*	|u	|*	|*	|k	|z	|k	|*	|ć	|[12][S]\darr	|*	|[13][S]\darr	|i	|[14][S]\darr	|*	|c	|*	|.
|*	|*	|k	|*	|[15][S]\rarr	|s	|o	|w	|a	|*	|s	|*	|r	|s	|p	|[16][S]\darr	|z	|*	|.
|*	|*	|a	|*	|*	|p	|ś	|a	|[17][S]\drarr	|w	|a	|t	|y	|k	|a	|n	|*	|*	|.
|*	|*	|r	|*	|*	|o	|ć	|s	|k	|*	|j	|*	|b	|o	|n	|i	|*	|*	|.
|*	|*	|i	|*	|*	|r	|*	|*	|a	|*	|k	|*	|o	|[][,]{ }	|c	|e	|*	|*	|.
|*	|[18][S]\darr	|a	|*	|[19][S]\darr	|t	|[20][S]\drarr	|i	|n	|b	|a	|*	|ł	|t	|e	|d	|*	|*	|.
|*	|j	|[][,]{ }	|[21][S]\rarr	|m	|a	|j	|d	|a	|n	|*	|*	|ó	|u	|r	|o	|*	|*	|.
|*	|e	|c	|[22][S]\darr	|e	|c	|a	|*	|d	|*	|*	|*	|w	|r	|z	|k	|*	|*	|.
|*	|ż	|h	|k	|t	|j	|k	|*	|a	|*	|*	|[23][S]\darr	|*	|y	|[][,]{ }	|ł	|[24][S]\darr	|*	|.
|*	|[][,]{ }	|i	|r	|o	|a	|*	|*	|*	|[25][S]\darr	|*	|r	|*	|s	|b	|a	|m	|*	|.
|[26][S]\drarr	|b	|l	|e	|d	|*	|*	|*	|*	|b	|*	|ó	|[27][S]\darr	|t	|u	|d	|e	|*	|.
|k	|i	|i	|w	|a	|*	|*	|*	|*	|ł	|*	|ż	|ś	|y	|r	|n	|z	|*	|.
|r	|a	|j	|[][,]{ }	|[][,]{ }	|*	|*	|*	|*	|o	|*	|a	|c	|c	|t	|o	|o	|*	|.
|o	|ł	|s	|u	|e	|*	|*	|*	|*	|c	|*	|n	|i	|z	|o	|ś	|s	|*	|.
|k	|o	|k	|t	|d	|*	|*	|*	|*	|h	|*	|i	|ą	|n	|w	|ć	|f	|*	|.
|[][,]{ }	|b	|a	|l	|u	|*	|*	|*	|*	|i	|*	|e	|g	|e	|y	|*	|e	|*	|.
|ł	|r	|*	|e	|k	|[28][S]\rarr	|k	|u	|ź	|n	|i	|c	|a	|*	|*	|*	|r	|*	|.
|y	|z	|*	|n	|a	|*	|*	|*	|*	|*	|*	|*	|r	|*	|*	|*	|a	|*	|.
|ż	|u	|*	|o	|c	|[29][S]\rarr	|a	|f	|g	|a	|ń	|s	|k	|o	|ś	|ć	|*	|*	|.
|w	|c	|[30][S]\rarr	|w	|y	|z	|w	|a	|n	|i	|e	|*	|a	|*	|*	|*	|*	|*	|.
|o	|h	|*	|a	|j	|*	|*	|*	|*	|*	|*	|*	|*	|*	|*	|*	|*	|*	|.
|w	|y	|*	|n	|n	|*	|*	|*	|*	|*	|*	|*	|*	|*	|*	|*	|*	|*	|.
|y	|*	|*	|a	|a	|*	|*	|*	|*	|*	|*	|*	|*	|*	|*	|*	|*	|*	|.
|*	|*	|*	|*	|*	|*	|*	|*	|*	|*	|*	|*	|*	|*	|*	|*	|*	|*	|.\end{Puzzle}

\newpage

\begin{PuzzleClues}{\textbf{Poziome}\\}\Clue{6}{}{(1791-1865), poeta hiszpański, książę; „Don Alwaro”}
\Clue{7}{}{ludowy poeta z Opolszczyzny (1872-1957), uczestnik III powstania śląskiego; „Wiersze śląskie”}
\Clue{11}{}{potencjalna właściwość}
\Clue{15}{}{Macrolepiota procera (Scop.) Singer - gatunek grzyba należący do rodziny pieczarkowatych (Agaricaceae); dość pospolity na terenie całej Polski, gdzie występuje od lata do późnej jesieni, rosnąć na brzegach lasów liściastych i iglastych, na polanach leśnych i zrębach, na łąkach, w parkach, na poboczach szos, na cmentarzach}
\Clue{17}{}{państwo kościelne w płn.- zach. części Rzymu}
\Clue{20}{}{impreza, biba}
\Clue{21}{}{Plac Niepodległości w Kijowie}
\Clue{26}{}{miasto w Słowenii w Alpach nad jeziorem Bled, ośrodek sportów wodnych i zimowych}
\Clue{28}{}{Kuźnica Grabowska, wieś w województwie wielkopolskim}
\Clue{29}{}{bycie Afgańczykiem, przynależność do narodu afgańskiego}
\Clue{30}{}{zwymyślanie kogoś, nie przebierając w słowach; zmuszenie kogoś do udziału w czymś, np. pojedynku, wykorzystując jako argument najczęściej honor lub czyjeś ambicje}\end{PuzzleClues}

\begin{PuzzleClues}{\textbf{Pionowe}\\}\Clue{1}{}{pojazd mechaniczny przystosowany do ciągnięcia pojazdów bądź urządzeń nie posiadających własnego napędu, np. maszyn rolniczych, przyczep, naczep, dział; może być wyposażony w osprzęt do robót ziemnych (spycharkowy, zrywarkowy, ładowarkowy, koparkowy, chwytakowy, hakowy)}
\Clue{2}{}{stan opuszczenia, osamotnienia, izolacji}
\Clue{3}{}{osoba odbywająca praktyki w ramach aplikacji prawniczej}
\Clue{4}{}{jeździec, który w czasie polowania goni z psami zwierzynę}
\Clue{5}{}{połączenie dwóch elementów umożliwiające ich względny obrót}
\Clue{7}{}{ukraińska potrawa, wywar z gotowanej cielęciny, wieprzowiny lub mięsa drobiowego, gotowany z czosnkiem, cebulą, liściem laurowym, pieprzem i podawany na zimno}
\Clue{8}{}{obiekt hotelarski zlokalizowany poza obszarami zabudowanymi, przy szlakach turystycznych, świadczący minimalny zakres usług związanych z pobytem klientów: przeznaczony dla potrzeb obsługi ruchu turystycznego, zapewniający miejsce odpoczynku, schronienia przed wpływem niekorzystnych warunków atmosferycznych}
\Clue{9}{}{igława chilijska, Araucaria araucana - gatunek drzewa nagonasiennego z rodziny araukariowatych}
\Clue{10}{}{wyprowadzenie zwłok na miejsce, gdzie pozostają do pogrzebu}
\Clue{11}{}{sytuacja, kiedy dobra atmosfera pryska i robi się drętwo}
\Clue{12}{}{ryba z dorszowatych, zamieszkuje wody mórz polarnych, pokarm dla polarnych ssaków i większych ryb}
\Clue{13}{}{rybołów zwyczajny, Pandion haliaetus - gatunek dużego, wędrownego ptaka drapieżnego z rodziny rybołowów (Pandionidae)}
\Clue{14}{}{element opancerzenia okrętów wojennych, chroniący burtę}
\Clue{16}{}{to, że coś jest niedokładne, jest zrobione, wykonane, przeprowadzone z małą dokładnością, z niewystarczającą starannością i precyzją}
\Clue{17}{}{robocze komando w obozie koncentracyjnym Auschwitz, zajmujące się sortowaniem odzieży}
\Clue{18}{}{Atelerix albiventris - gatunek ssaka łożyskowego z rodziny jeżowatych, występujący w centralnej i wschodniej Afryce}
\Clue{19}{}{metoda, wykorzystywana w nauczaniu; niekiedy utożsamiana z 'techniką dydaktyczną'}
\Clue{20}{}{typ radzieckich samolotów i śmigłowców, dzieło Jakowlewa}
\Clue{22}{}{krew utlenowana, zawierająca hemoglobinę wysyconą tlenem w ponad 97\%, płynąca tętnicami obiegu dużego i żyłami obiegu małego}
\Clue{23}{}{modlitwa maryjna duchowo jednocząca chrześcijanina z osobą Matki Jezusa i kontemplująca tajemnice zbawienia}
\Clue{24}{}{warstwa atmosfery ziemskiej na wysokości od około 45 do 85 km, w której temperatura maleje wraz z wysokością}
\Clue{25}{}{piłkarz radziecki, napastnik Dynama Kijów, najlepszy piłkarz Europy w 1975 r}
\Clue{26}{}{krok narciarski, który polega na silnym odpychaniu się raz jedną, raz drugą nogą w bok}
\Clue{27}{}{program umożliwiający pobieranie plików z sieci}\end{PuzzleClues}\newpage\section*{Krzyżówka 171}

\noindent\begin{Puzzle}{22}{25}|*	|[1][S]\darr	|*	|*	|*	|*	|*	|*	|*	|*	|*	|[2][S]\drarr	|p	|r	|z	|e	|j	|ę	|c	|i	|e	|*	|[3][S]\darr	|.
|*	|s	|[4][S]\drarr	|a	|n	|a	|l	|i	|z	|a	|[][,]{ }	|b	|i	|l	|a	|n	|s	|u	|*	|[5][S]\darr	|*	|[6][S]\darr	|d	|.
|[7][S]\drarr	|k	|r	|y	|z	|a	|*	|*	|[8][S]\darr	|*	|*	|e	|*	|[9][S]\darr	|*	|*	|*	|[10][S]\darr	|[11][S]\darr	|p	|[12][S]\darr	|r	|u	|.
|k	|a	|o	|*	|*	|*	|*	|[13][S]\darr	|s	|[14][S]\darr	|[15][S]\rarr	|r	|m	|b	|*	|*	|*	|e	|b	|o	|r	|o	|x	|.
|u	|ł	|ż	|*	|*	|*	|*	|t	|c	|p	|*	|n	|*	|e	|*	|*	|[16][S]\drarr	|u	|r	|s	|o	|n	|*	|.
|s	|k	|e	|*	|*	|[17][S]\darr	|*	|a	|h	|o	|*	|i	|*	|z	|*	|[18][S]\darr	|o	|f	|u	|m	|ś	|n	|*	|.
|a	|a	|k	|*	|*	|b	|[19][S]\rarr	|t	|a	|c	|z	|n	|i	|k	|*	|c	|w	|o	|d	|a	|l	|e	|*	|.
|c	|*	|*	|*	|[20][S]\darr	|o	|*	|a	|t	|i	|[21][S]\darr	|i	|*	|i	|*	|z	|c	|n	|o	|k	|i	|*	|*	|.
|z	|*	|[22][S]\drarr	|u	|n	|g	|a	|r	|*	|ą	|p	|*	|*	|e	|[23][S]\darr	|y	|a	|i	|w	|*	|n	|*	|*	|.
|[][,]{ }	|[24][S]\darr	|j	|[25][S]\darr	|ó	|o	|*	|a	|*	|g	|ł	|*	|[26][S]\darr	|r	|k	|t	|*	|a	|n	|[27][S]\darr	|y	|*	|[28][S]\darr	|.
|p	|p	|a	|o	|ż	|r	|[29][S]\darr	|k	|*	|ł	|e	|[30][S]\drarr	|k	|u	|r	|e	|ń	|*	|i	|m	|[][,]{ }	|*	|o	|.
|o	|r	|r	|r	|[][,]{ }	|*	|g	|o	|*	|o	|ć	|f	|o	|n	|z	|l	|*	|*	|k	|y	|r	|*	|b	|.
|p	|z	|z	|l	|d	|[31][S]\darr	|o	|w	|*	|ś	|[][,]{ }	|o	|k	|k	|y	|n	|*	|*	|*	|r	|u	|*	|t	|.
|i	|e	|y	|e	|o	|s	|n	|a	|*	|ć	|b	|r	|i	|o	|ż	|i	|*	|*	|[32][S]\darr	|m	|d	|*	|u	|.
|e	|k	|n	|a	|[][,]{ }	|t	|i	|t	|*	|*	|r	|m	|l	|w	|[][,]{ }	|c	|*	|*	|n	|e	|e	|[33][S]\darr	|r	|.
|l	|u	|k	|n	|s	|o	|o	|e	|*	|*	|z	|a	|a	|o	|p	|t	|*	|*	|i	|k	|r	|ś	|a	|.
|a	|p	|a	|i	|e	|p	|m	|*	|*	|*	|y	|c	|r	|ś	|a	|w	|[34][S]\darr	|*	|e	|o	|a	|w	|c	|.
|t	|s	|*	|z	|r	|i	|e	|*	|*	|*	|d	|j	|z	|ć	|p	|o	|b	|[35][S]\darr	|l	|f	|l	|i	|j	|.
|y	|t	|[36][S]\drarr	|m	|a	|n	|t	|u	|a	|n	|k	|a	|*	|*	|i	|*	|i	|b	|a	|i	|n	|a	|a	|.
|*	|w	|o	|*	|*	|a	|r	|[37][S]\rarr	|i	|w	|a	|*	|*	|*	|e	|[38][S]\rarr	|d	|u	|b	|l	|e	|t	|*	|.
|[39][S]\rarr	|o	|b	|c	|y	|*	|i	|*	|*	|*	|*	|[40][S]\rarr	|r	|o	|s	|z	|a	|d	|a	|*	|*	|ł	|*	|.
|*	|*	|l	|*	|[41][S]\rarr	|r	|a	|k	|*	|[42][S]\rarr	|i	|n	|d	|e	|k	|s	|*	|u	|*	|*	|*	|a	|*	|.
|[43][S]\drarr	|p	|a	|m	|i	|r	|*	|*	|[44][S]\rarr	|s	|p	|ł	|a	|w	|i	|k	|*	|l	|*	|*	|*	|*	|*	|.
|a	|[45][S]\drarr	|t	|a	|m	|b	|u	|r	|a	|*	|[46][S]\drarr	|y	|d	|[][S]2	|*	|[47][S]\rarr	|l	|e	|n	|t	|o	|*	|*	|.
|u	|d	|*	|[48][S]\rarr	|e	|k	|l	|e	|k	|t	|y	|z	|m	|*	|[49][S]\rarr	|k	|o	|c	|z	|*	|*	|*	|*	|.
|*	|*	|*	|*	|*	|*	|*	|*	|*	|*	|*	|*	|[50][S]\rarr	|t	|o	|k	|i	|*	|*	|*	|*	|*	|*	|.\end{Puzzle}

\newpage

\begin{PuzzleClues}{\textbf{Poziome}\\}\Clue{2}{}{utrata kontroli nad przedsiębiorstwem przez jedną grupę na rzecz innej grupy}
\Clue{4}{}{stan faktyczny danej firmy, który jest określony w postaci liczb bilansowych}
\Clue{7}{}{względnie szeroki brzeg tylnej części głowy gadów, który może posiadać szkielet kostny (jak u dinozaurów z grupy marginocefali) lub chrzęstny (jak u agamy kołnierzastej, australijskiej jaszczurki, u której na rusztowaniu chrzęstnym rozpościerają się płaty skórne)}
\Clue{15}{}{skrót/symbol yuana}
\Clue{16}{}{północnoamerykański. jeżozwierz nadrzewny, aktywny nocą}
\Clue{19}{}{górnik, który pracuje przy taczkach; określenie środowiskowe górnicze}
\Clue{22}{}{pianista węgierski (1909-1972); laureat Konkursu im. F. Chopina w 1932 r}
\Clue{30}{}{kozacki oddział pod wodzą atamana na Ukrainie Zaporoskiej}
\Clue{36}{}{mieszkanka Mantui (miasta we Włoszech)}
\Clue{37}{}{eurazjatycki gatunek wierzby, w Polsce pospolita, srebrzyste bazie, pędy na obręcze do beczek}
\Clue{38}{}{dawny ubiór męski}
\Clue{39}{}{przybysz z kosmosu}
\Clue{40}{}{zamiana pracowników na stanowiskach w zakładzie pracy}
\Clue{41}{}{polski pistolet maszynowy, kaliber 9mm}
\Clue{42}{}{wskaźnik}
\Clue{43}{}{górska kraina w Azji Środkowej, głównie w Tadżykistanie}
\Clue{44}{}{kawałek materiału (najczęściej drewna bądź tworzywa sztucznego) pływającego na powierzchni wody, służący do sygnalizowania pochwycenia przynęty przez rybę}
\Clue{45}{}{instrument strunowy szarpany z pudłem rezonansowym, gryfem i progami na podstrunnicy; wykonywany jest z drewna, ma zwykle cztery struny, obecnie najczęściej metalowe}
\Clue{46}{}{jednostka powierzchni równa polu kwadratu o bokach długości jarda}
\Clue{47}{}{określenie wykonawcze: powoli}
\Clue{48}{}{nurt polegający na twórczości kompilacyjnej, łączącej różne elementy i treści z różnych stylów, epok i kierunków artystycznych, nieprowadzący do nowej syntezy, nieoryginalny}
\Clue{49}{}{czterokołowy pojazd z nadwoziem zawieszonym na łańcuchach lub pasach, wprowadzony w XVI w. na Węgrzech}
\Clue{50}{}{zaloty samców niektórych gatunków ptaków w obecności samic, w porze godów; walki między samcami, charakterystyczne 'śpiew', 'tańce' itp}\end{PuzzleClues}

\begin{PuzzleClues}{\textbf{Pionowe}\\}\Clue{1}{}{Kościół św. Michała Archanioła i św. Stanisława Biskupa i Męczennika oraz klasztor paulinów, kompleks sakralny, znajdujący się w Krakowie przy ulicy Skałecznej 15}
\Clue{2}{}{włoski architekt i rzeźbiarz (1598-1680), przedstawiciel baroku}
\Clue{3}{}{temat, czyli jeden głos rozpoczynający fugę do którego  dostosowane są ściśle inne, występujące po nim głosy}
\Clue{4}{}{rodzaj nadziewanego rogalika}
\Clue{5}{}{domieszka jakiegoś smaku, także: cień, wspomnienie smaku czegoś, co się niedawno zjadło}
\Clue{6}{}{amerykański badacz Antarktydy ur. w 1899 r.; dowiódł że Ziemia Aleksandra jest wyspą}
\Clue{7}{}{Crypturellus cinereus - gatunek ptaka z rodziny kusaczy (Tinamidae)}
\Clue{8}{}{kompozytor holenderski ur. w 1935r., utwory orkiestrowe, kameralne, wokalno-instrumentowe; opera 'Het Labyrinth'}
\Clue{9}{}{negatywna cecha czegoś, co nie daje określonego celu czy efektu, na przykład bezkierunkowość niektórych zajęć w szkołach, które nie uczą niczego konkretnego, ważnego ani potrzebnego w późniejszej pracy zawodowej}
\Clue{10}{}{dział poetyki obejmujący wiedzę o harmonijnym doborze dźwięków mowy}
\Clue{11}{}{pomieszczenie w szpitalu służące do przechowywania brudnej pościeli i śmieci do momentu ich wyprania i wywiezienia, a także do magazynowania środków i narzędzi wykorzystywanych do sprzątania}
\Clue{12}{}{roślina zasiedlająca podłoża zmienione przez człowieka, szczególnie środowiska miejskie np. budynki i ich sąsiedztwa, drogi i przydroża, tereny kolejowe, parkingi i place, porty, wysypiska odpadów, hałdy i tereny przemysłowe}
\Clue{13}{}{Acoraceae - rodzina roślin zielnych z monotypowego rzędu tatarakowców}
\Clue{14}{}{to, że coś jest pociągłe, podłużne; kształt twarzy lub jej elementów (np. policzków)}
\Clue{16}{}{ssak z rodziny krętorogich hodowany dla mięsa i wełny}
\Clue{17}{}{miasto w Indonezji, na Jawie, w pobliżu Dżakarty, ośrodek turystyczny i wypoczynkowy}
\Clue{18}{}{zjawisko społeczne, działalność człowieka związana z czytaniem książek}
\Clue{20}{}{ząbkowany nóż z otworami w ostrzu służący do krojenia sera}
\Clue{21}{}{żartobliwie o płci męskiej}
\Clue{22}{}{zdrobniale o jarzynie - jadalnej części jakiejś rośliny}
\Clue{23}{}{krzyż z trzema poprzeczkami umieszczany w herbach papieskich}
\Clue{24}{}{przestępstwo polegające na wręczaniu, braniu lub żądaniu korzyści majątkowej lub osobistej (łapówki)}
\Clue{25}{}{doktryna polityczna ukształtowana w latach 1830-1848, której celem było dążenie do ograniczenia władzy absolutystycznej}
\Clue{26}{}{pracownik, który zajmuje się odlewami w kokilach}
\Clue{27}{}{zwierzę, najczęściej stawonóg, które stale lub przejściowo żyje w gnieździe mrówek lub termitów i korzysta z ich pokarmu; jest to albo pasożyt, drapieżca zjadający mrówki, termity i ich larwy, albo symbiont mrówek lub termitów}
\Clue{28}{}{zwężenie struktury anatomicznej posiadającej światło}
\Clue{29}{}{dział trygonometrii zajmujący się badaniem elementarnych własności funkcji trygonometrycznych}
\Clue{30}{}{zespół ludzi, którzy są częścią składową formacji, oddziału wojskowego}
\Clue{31}{}{sznur nasycony masą z prochu i gumy arabskiej używany dawniej jako lont}
\Clue{32}{}{szermierz, mistrz i wicemistrz świata z 1963 i 70 r., brązowy medalista z Meksyku w turnieju drużynowym szpady}
\Clue{33}{}{miejsce na drodze bądź na skrzyżowaniu, w którym znajduje się sygnalizacja świetlna}
\Clue{34}{}{wprowadzenie obcych elementów do islamu}
\Clue{35}{}{integralny składnik jakiejś większej całości}
\Clue{36}{}{dziecko, które przeznaczono do stanu zakonnego}
\Clue{43}{}{w chemii: symbol złota}
\Clue{45}{}{litera alfabetu}
\Clue{46}{}{w chemii: symbol itru}\end{PuzzleClues}\newpage\section*{Krzyżówka 172}

\noindent\begin{Puzzle}{22}{33}|*	|*	|*	|*	|*	|*	|*	|*	|*	|*	|*	|*	|*	|*	|*	|*	|*	|*	|*	|*	|*	|[1][S]\darr	|*	|.
|*	|*	|*	|*	|[2][S]\drarr	|p	|o	|s	|t	|e	|r	|u	|n	|k	|o	|w	|y	|*	|*	|*	|*	|d	|*	|.
|*	|*	|*	|*	|t	|*	|[3][S]\drarr	|a	|u	|t	|o	|p	|o	|r	|t	|r	|e	|t	|*	|*	|*	|i	|*	|.
|*	|[4][S]\drarr	|o	|b	|r	|ą	|c	|z	|k	|a	|*	|*	|[5][S]\drarr	|p	|i	|l	|o	|t	|ó	|w	|k	|a	|*	|.
|*	|s	|*	|*	|z	|*	|h	|*	|*	|[6][S]\darr	|*	|*	|ś	|*	|*	|*	|*	|*	|*	|*	|*	|b	|*	|.
|*	|r	|*	|*	|ó	|*	|e	|*	|*	|e	|*	|[7][S]\rarr	|w	|i	|k	|t	|o	|r	|i	|a	|*	|e	|*	|.
|*	|e	|*	|[8][S]\darr	|s	|*	|m	|*	|*	|u	|*	|*	|i	|*	|*	|*	|*	|*	|*	|*	|*	|ł	|*	|.
|*	|b	|*	|h	|ł	|*	|o	|*	|*	|l	|[9][S]\rarr	|r	|e	|i	|d	|*	|*	|[10][S]\darr	|*	|*	|[11][S]\darr	|*	|[12][S]\darr	|.
|*	|r	|*	|a	|o	|*	|t	|*	|[13][S]\drarr	|e	|p	|i	|t	|l	|e	|n	|e	|k	|*	|*	|d	|*	|h	|.
|*	|n	|*	|r	|*	|*	|r	|*	|b	|r	|*	|*	|l	|*	|*	|[14][S]\darr	|*	|a	|*	|*	|r	|*	|u	|.
|*	|a	|*	|b	|[15][S]\rarr	|w	|o	|d	|a	|*	|*	|*	|i	|*	|*	|i	|*	|l	|*	|[16][S]\darr	|z	|[17][S]\darr	|l	|.
|*	|[][,]{ }	|*	|a	|*	|*	|p	|*	|j	|*	|*	|*	|c	|[18][S]\darr	|*	|p	|*	|u	|[19][S]\darr	|p	|e	|ż	|l	|.
|*	|p	|[20][S]\drarr	|j	|e	|z	|i	|o	|r	|o	|[][,]{ }	|z	|a	|s	|t	|o	|i	|s	|k	|o	|w	|e	|*	|.
|[21][S]\drarr	|a	|p	|t	|e	|c	|z	|k	|a	|*	|*	|*	|*	|p	|*	|d	|*	|*	|i	|r	|c	|l	|*	|.
|k	|p	|a	|e	|*	|*	|m	|*	|m	|*	|*	|*	|*	|ó	|*	|[][,]{ }	|*	|*	|n	|z	|ó	|a	|*	|.
|s	|r	|p	|l	|*	|*	|[][,]{ }	|*	|*	|*	|*	|*	|*	|ł	|*	|v	|*	|*	|g	|ą	|w	|z	|*	|.
|i	|o	|r	|*	|*	|*	|d	|*	|*	|*	|*	|*	|*	|g	|*	|i	|*	|*	|d	|d	|k	|o	|*	|.
|ą	|ć	|o	|*	|*	|*	|o	|*	|*	|*	|*	|[22][S]\darr	|*	|ł	|*	|d	|*	|*	|o	|e	|a	|n	|*	|.
|ż	|*	|t	|*	|*	|*	|d	|*	|*	|*	|[23][S]\darr	|f	|*	|o	|*	|e	|*	|*	|m	|k	|*	|i	|*	|.
|k	|*	|n	|*	|*	|*	|a	|*	|*	|[24][S]\drarr	|d	|u	|ń	|s	|k	|o	|ś	|ć	|*	|[][,]{ }	|*	|k	|*	|.
|a	|*	|i	|*	|*	|*	|t	|*	|*	|e	|z	|n	|*	|k	|[25][S]\darr	|*	|*	|*	|*	|s	|*	|i	|*	|.
|[][,]{ }	|*	|k	|*	|*	|*	|n	|*	|*	|w	|i	|t	|*	|a	|k	|*	|*	|*	|*	|p	|*	|e	|*	|.
|k	|*	|[][,]{ }	|*	|*	|*	|i	|*	|*	|o	|e	|*	|*	|[][,]{ }	|o	|*	|[26][S]\darr	|*	|*	|i	|*	|l	|*	|.
|u	|*	|s	|*	|*	|*	|*	|*	|*	|l	|w	|*	|*	|w	|n	|*	|g	|*	|*	|ę	|*	|*	|*	|.
|c	|*	|i	|*	|*	|*	|*	|*	|*	|u	|i	|*	|*	|a	|i	|*	|w	|*	|*	|t	|*	|*	|*	|.
|h	|*	|e	|*	|*	|*	|*	|*	|*	|c	|ą	|*	|[27][S]\rarr	|r	|a	|m	|a	|*	|*	|r	|*	|*	|*	|.
|a	|*	|r	|*	|*	|*	|*	|*	|*	|j	|t	|*	|*	|g	|k	|*	|t	|*	|*	|z	|*	|*	|*	|.
|r	|*	|p	|*	|*	|*	|*	|*	|*	|o	|k	|*	|*	|o	|ó	|*	|e	|*	|*	|o	|*	|*	|*	|.
|s	|*	|o	|*	|*	|*	|*	|*	|*	|n	|a	|*	|*	|w	|w	|*	|m	|*	|*	|n	|*	|*	|*	|.
|k	|*	|w	|*	|*	|*	|*	|*	|*	|i	|*	|*	|*	|a	|k	|*	|a	|*	|*	|y	|*	|*	|*	|.
|a	|*	|a	|*	|*	|*	|*	|[28][S]\rarr	|i	|z	|m	|i	|t	|*	|a	|*	|l	|*	|*	|*	|*	|*	|*	|.
|*	|*	|t	|*	|[29][S]\rarr	|s	|n	|e	|e	|m	|*	|*	|*	|*	|*	|*	|a	|*	|*	|*	|*	|*	|*	|.
|[30][S]\rarr	|w	|y	|p	|ł	|a	|w	|e	|k	|*	|*	|*	|*	|*	|*	|*	|*	|*	|*	|*	|*	|*	|*	|.
|*	|*	|*	|*	|*	|*	|*	|*	|*	|*	|*	|*	|*	|*	|*	|*	|*	|*	|*	|*	|*	|*	|*	|.\end{Puzzle}

\newpage

\begin{PuzzleClues}{\textbf{Poziome}\\}\Clue{2}{}{stopień w policji, odpowiadający stopniowi sierżanta w wojsku}
\Clue{3}{}{utwór literacki, w którym umieszczony został wizerunek twórcy utworu; tekst, w którym autor pisze o sobie}
\Clue{4}{}{obrączka z tworzyw sztucznych, często z wbudowanym nadajnikiem, zakładana na nogę ptakom w celu monitorowania ich liczebności i kierunków przemieszczania}
\Clue{5}{}{łódź motorowa wykorzystywana przez portowych pilotów}
\Clue{7}{}{wodna bylina o kolistych, pływających liściach, dorzecze Amazonki}
\Clue{9}{}{pisarz angielski (1818-83), przygodowe powieści dla młodzieży; „Łowcy skalpów”, „Jeździec bez głowy”}
\Clue{13}{}{epoksyd - organiczny związek chemiczny zawierający trójczłonowe ugrupowanie cykliczne złożone z atomu tlenu i dwóch atomów węgla}
\Clue{15}{}{zbiornik wodny lub ciek występujący naturalnie lub utworzony sztucznie w przyrodzie}
\Clue{20}{}{jezioro lodowcowe utworzone na przedpolu lodowca w wyniku zatamowania naturalnego odpływu wód lodowcowych przez jęzor lub czoło lodowca}
\Clue{21}{}{szafka, w której znajdują się akcesoria potrzebne do udzielenia pierwszej pomocy przy drobnych obrażeniach}
\Clue{24}{}{zespół cech właściwych Duńczykom, Danii lub czemuś duńskiemu}
\Clue{27}{}{w budownictwie: prętowy układ konstrukcyjny, płaski lub przestrzenny, obciążony siłami skupionymi, momentami lub obciążeniem rozłożonym (równomiernie bądź też nierównomiernie)}
\Clue{28}{}{miasto w płn.-zach. Turcji, ośrodek administracyjny ilu Kocaeli, port nad morzem Marmara}
\Clue{29}{}{miasto w Irlandii nad Zatoką Kenmare}
\Clue{30}{}{bezkręgowiec zaliczany do typu płazińców, rzędu wirków}\end{PuzzleClues}

\begin{PuzzleClues}{\textbf{Pionowe}\\}\Clue{1}{}{z podziwem o kimś bystrym, łebskim, sprytnym}
\Clue{2}{}{część pługa do odcinania skiby w pionie}
\Clue{3}{}{pozytywna reakcja ruchowa roślin w odpowiedzi na czynnik chemiczny, polegająca na zwracaniu się rośliny w stronę bodźca}
\Clue{4}{}{Cyathea dealbata - nazwa zwyczajowa gatunku paproci drzewiastej z rzędu olbrzymkowców, rosnący w Nowej Zelandii; charakteryzuje się biało-srebrnym spodem liści; dorasta do 10 m wysokości; jest nieoficjalnym symbolem Nowej Zelandii i Nowozelandczyków, występuje m.in. w herbie państwa, a także w logo Nowozelandzkiego Komitetu Olimpjskiego, a także reprezentacji piłki nożnej i rugby union}
\Clue{5}{}{pomieszczenie w szkole, pełniące funkcję świetlicy}
\Clue{6}{}{fizjolog szwedzki (1905-83); Nobel za odkrycie dotyczące mediatorów uwalnianych na zakończeniach nerwowych}
\Clue{8}{}{pęk włosów z tyłu peruki}
\Clue{10}{}{tkanka miękiszowa powstająca na zranionej powierzchni narządów roślin}
\Clue{11}{}{broń, w której istnienie drzewca determinuje możliwość jej skutecznego użycia oraz w której drzewce są znacznie dłuższe, niż wynikałoby to z roli uchwytu}
\Clue{12}{}{miasto w Anglii nad estuarium Humber, ważny port handlowy i rybacki}
\Clue{13}{}{przyswojone w polszczyźnie z języka tureckiego określenie dnia świątecznego, które występuje w nazwach niektórych muzułmańskich świąt}
\Clue{14}{}{iPod umożliwiający oglądanie filmów}
\Clue{16}{}{spiętrzenie porządków architektonicznych, wprowadzenie w jednym obiekcie, na kilku kondygnacjach, porządków reprezentujących kolejno: porządek dorycki, joński, koryncki}
\Clue{17}{}{stop ferromagnetyczny złożony z żelaza i niklu}
\Clue{18}{}{spółgłoska wymawiana poprzez zbliżenie warg do siebie lub górnych zębów do dolnej wargi}
\Clue{19}{}{lekkoatleta amerykański, dwukrotny mistrz olimpijski w biegu na 110 m przez płotki z Los Angeles i Seulu}
\Clue{20}{}{Polystichum falcatum - gatunek paproci; pochodzi z Azji Południowo-Wschodniej, introdukowany w Europie i Ameryce Północnej}
\Clue{21}{}{książka z przepisami kulinarnymi, która może oprócz nich zawierać ogólne porady i wskazówki dotyczące gotowania}
\Clue{22}{}{bardzo rozpowszechniona nazwa waluty funkcjonującej w wielu krajach}
\Clue{23}{}{karta z cyfrą 9}
\Clue{24}{}{koncepcja filozoficzna wyjaśniająca aktualny stan rzeczywistości jako efekt ewolucji stanów poprzednich}
\Clue{25}{}{kieliszek do koniaku}
\Clue{26}{}{państwo w Ameryce Środkowej, położone nad Oceanem Atlantyckim i Oceanem Spokojnym}\end{PuzzleClues}\newpage\section*{Krzyżówka 173}

\noindent\begin{Puzzle}{24}{32}|*	|*	|*	|*	|*	|*	|*	|*	|*	|*	|*	|*	|*	|*	|*	|*	|*	|*	|*	|*	|*	|*	|*	|[1][S]\darr	|*	|.
|*	|*	|*	|*	|*	|[2][S]\drarr	|m	|o	|n	|i	|e	|r	|*	|[3][S]\drarr	|k	|r	|y	|m	|i	|n	|a	|ł	|e	|k	|*	|.
|*	|[4][S]\rarr	|t	|e	|u	|t	|o	|n	|*	|*	|*	|[5][S]\rarr	|s	|c	|j	|e	|n	|t	|o	|l	|o	|g	|i	|a	|*	|.
|*	|[6][S]\drarr	|k	|u	|b	|e	|k	|*	|[7][S]\drarr	|z	|a	|r	|a	|z	|e	|k	|*	|*	|*	|*	|*	|*	|*	|r	|*	|.
|*	|n	|*	|*	|*	|g	|*	|[8][S]\rarr	|p	|r	|z	|e	|b	|a	|r	|w	|i	|e	|n	|i	|e	|*	|*	|a	|*	|.
|*	|a	|[9][S]\darr	|[10][S]\drarr	|k	|o	|z	|i	|o	|ł	|e	|k	|*	|j	|*	|*	|*	|*	|*	|*	|[11][S]\darr	|*	|*	|c	|*	|.
|[12][S]\drarr	|s	|p	|e	|r	|r	|y	|*	|d	|[13][S]\rarr	|w	|o	|r	|k	|o	|l	|o	|t	|*	|*	|k	|*	|*	|e	|*	|.
|p	|t	|r	|r	|[14][S]\drarr	|o	|d	|p	|a	|d	|y	|[][,]{ }	|k	|o	|m	|u	|n	|a	|l	|n	|e	|*	|[15][S]\darr	|n	|[16][S]\darr	|.
|a	|r	|z	|a	|m	|c	|*	|*	|t	|[17][S]\darr	|*	|[18][S]\darr	|[19][S]\drarr	|w	|ł	|o	|ś	|ć	|*	|*	|t	|*	|m	|a	|l	|.
|n	|ó	|e	|t	|r	|z	|*	|[20][S]\darr	|e	|k	|[21][S]\darr	|b	|d	|s	|*	|[22][S]\darr	|[23][S]\darr	|[24][S]\rarr	|h	|a	|o	|m	|a	|*	|i	|.
|i	|j	|s	|o	|ó	|n	|*	|r	|k	|o	|t	|e	|u	|k	|*	|t	|c	|[25][S]\rarr	|u	|s	|t	|ę	|p	|*	|n	|.
|c	|*	|t	|s	|w	|o	|*	|o	|[][,]{ }	|s	|w	|n	|w	|i	|*	|u	|h	|[26][S]\darr	|*	|[27][S]\darr	|r	|*	|n	|*	|i	|.
|z	|*	|r	|t	|k	|ś	|[28][S]\drarr	|n	|o	|m	|a	|d	|a	|*	|*	|j	|r	|n	|*	|p	|i	|*	|i	|*	|a	|.
|ą	|*	|z	|e	|o	|ć	|s	|i	|d	|e	|r	|e	|j	|[29][S]\rarr	|g	|a	|z	|a	|*	|o	|o	|*	|k	|[30][S]\darr	|[][,]{ }	|.
|t	|[31][S]\darr	|e	|n	|j	|*	|b	|n	|[][,]{ }	|t	|ó	|l	|h	|*	|[32][S]\darr	|n	|e	|s	|*	|d	|z	|*	|[][,]{ }	|z	|p	|.
|k	|p	|ń	|e	|a	|*	|*	|*	|l	|y	|g	|*	|i	|*	|k	|*	|s	|i	|*	|l	|a	|[33][S]\darr	|p	|a	|r	|.
|o	|r	|[][,]{ }	|s	|d	|[34][S]\rarr	|k	|l	|u	|k	|*	|*	|n	|*	|r	|*	|t	|o	|*	|o	|*	|n	|i	|t	|z	|.
|*	|a	|p	|*	|*	|*	|*	|*	|k	|*	|[35][S]\darr	|*	|*	|*	|z	|*	|*	|n	|[36][S]\darr	|t	|*	|e	|ł	|o	|e	|.
|[37][S]\rarr	|w	|o	|j	|s	|k	|o	|[][,]{ }	|s	|u	|p	|l	|e	|m	|e	|n	|t	|o	|w	|e	|*	|g	|o	|k	|m	|.
|[38][S]\drarr	|d	|z	|i	|e	|ń	|[][,]{ }	|j	|u	|t	|r	|z	|e	|j	|s	|z	|y	|*	|i	|k	|*	|a	|g	|a	|i	|.
|g	|a	|a	|*	|*	|[39][S]\drarr	|i	|n	|s	|p	|e	|k	|t	|*	|a	|[40][S]\drarr	|c	|a	|l	|*	|*	|c	|r	|[][,]{ }	|a	|.
|r	|*	|z	|[41][S]\darr	|*	|w	|*	|*	|u	|*	|t	|*	|*	|*	|k	|l	|*	|*	|g	|[42][S]\darr	|[43][S]\darr	|j	|z	|w	|n	|.
|o	|*	|i	|b	|*	|ą	|*	|*	|*	|*	|o	|*	|*	|*	|*	|i	|[44][S]\darr	|*	|e	|p	|d	|o	|b	|i	|y	|.
|c	|*	|e	|r	|*	|ż	|[45][S]\rarr	|n	|e	|u	|r	|o	|z	|w	|y	|r	|o	|d	|n	|i	|e	|n	|i	|e	|*	|.
|h	|*	|m	|y	|*	|*	|[46][S]\rarr	|h	|o	|t	|[][,]{ }	|d	|o	|g	|*	|a	|a	|[47][S]\darr	|a	|e	|p	|i	|e	|ń	|[48][S]\darr	|.
|[][,]{ }	|*	|s	|f	|[49][S]\rarr	|b	|a	|r	|r	|a	|m	|u	|n	|d	|a	|*	|z	|a	|*	|r	|r	|z	|t	|c	|g	|.
|w	|[50][S]\rarr	|k	|o	|l	|u	|m	|b	|*	|*	|i	|*	|*	|*	|*	|*	|i	|s	|*	|z	|a	|m	|y	|o	|a	|.
|ł	|*	|a	|k	|[51][S]\rarr	|j	|u	|t	|r	|z	|e	|n	|k	|a	|*	|*	|k	|t	|*	|g	|w	|*	|*	|w	|j	|.
|o	|*	|*	|*	|[52][S]\rarr	|s	|o	|ł	|o	|w	|j	|o	|w	|*	|*	|*	|*	|i	|*	|a	|a	|*	|*	|a	|c	|.
|s	|*	|*	|*	|*	|[53][S]\rarr	|t	|o	|m	|a	|s	|z	|o	|w	|i	|a	|n	|k	|a	|*	|c	|*	|*	|*	|y	|.
|k	|*	|*	|*	|[54][S]\rarr	|c	|z	|u	|b	|a	|k	|[][,]{ }	|a	|u	|s	|t	|r	|a	|l	|i	|j	|s	|k	|i	|*	|.
|i	|*	|[55][S]\rarr	|p	|o	|d	|s	|a	|d	|n	|i	|k	|o	|w	|c	|e	|*	|*	|[56][S]\rarr	|h	|a	|l	|b	|a	|*	|.
|*	|[57][S]\rarr	|z	|i	|e	|l	|o	|n	|k	|a	|*	|*	|*	|*	|*	|*	|*	|*	|*	|*	|*	|*	|*	|*	|*	|.\end{Puzzle}

\newpage

\begin{PuzzleClues}{\textbf{Poziome}\\}\Clue{2}{}{wynalazca francuski (1823-1906); uchodzi za wynalazcę żelbetu}
\Clue{3}{}{zdrobniale: kryminał - film kryminalny}
\Clue{4}{}{krzyżak - członek Zakonu Krzyżackiego}
\Clue{5}{}{ideologia, nawiązująca do psychologii, nauki, religii i filozofii}
\Clue{6}{}{naczynie, najczęściej ceramiczne (ale również metalowe, szklane, plastikowe itd.), służące do picia napojów, przeważnie ciepłych}
\Clue{7}{}{czynnik (drobnoustrój) powodujący chorobę zakaźną}
\Clue{8}{}{zmiana koloru skóry powstała wskutek działania czynnika zewnętrznego lub w rezultacie schorzenia organizmu}
\Clue{10}{}{na Kujawach: mątewka - przyrząd kuchenny służący do mieszania}
\Clue{12}{}{ur. w 1913 r., amerykański psychobiolog i badacz mózgu, nagroda Nobla za badania w zakresie funkcji mózgu związanych z procesem widzenia}
\Clue{13}{}{LOTOPAŁANKA; australijski torbacz wielkości wiewiórki}
\Clue{14}{}{odpady związane z nieprzemysłową działalnością człowieka}
\Clue{19}{}{rozległa własność ziemska}
\Clue{24}{}{mityczna roślina, z której wyrabiano sok o tej samej nazwie pity podczas religijnych rytuałów irańskich}
\Clue{25}{}{rodzaj jednostki redakcyjnej tekstu prawnego}
\Clue{28}{}{człowiek, który nie zagrzewa długo miejsca, nie tylko w wymiarze przestrzennym}
\Clue{29}{}{miasto w płd.-zach. Palestynie, na pograniczu Egiptu i Izraela, w 1967 r. zajęta i okupowana przez Izrael}
\Clue{34}{}{pilot szybowcowy, brązowy medalista Mistrzostw Świata w 1972 r}
\Clue{37}{}{grupa żołnierzy zawodowych powołana w momentach poważnego zagrożenia}
\Clue{38}{}{dzień, który będzie po dzisiaj; określenie używane w żargonie urzędniczym, w ogólnej polszczyźnie uznane za niepoprawne}
\Clue{39}{}{pojemnik w kształcie skrzynki bez dna, który służy do uprawy roślin}
\Clue{40}{}{jednostka miary długości stosowana w krajach anglosaskich, równa 2,54 cm}
\Clue{45}{}{uszkodzenie i zanik komórek nerwowych}
\Clue{46}{}{rodzaj kanapki z gorącą parówką lub kiełbaską z zimnym sosem (majonez, ketchup)}
\Clue{49}{}{ROGOZĄB}
\Clue{50}{}{(1451-1506) włoski żeglarz, odkrywca Ameryki}
\Clue{51}{}{przede wszystkim w kulturze ludowej: planeta Wenus, widoczna na niebie przed wschodem Słońca}
\Clue{52}{}{kosmonauta radziecki: pierwszy w historii przelot międzyorbitalny od stacji Mir do stacji Salut 7}
\Clue{53}{}{mieszkanka Tomaszowa Mazowieckiego}
\Clue{54}{}{Aviceda subcristata subcristata - nominatywny podgatunek ptaka wyróżniony w obrębie gatunku czubak australijski (Aviceda subcristata)}
\Clue{55}{}{Splachnales - rząd mchów należący do podklasy Bryidae, którą wyróżnia perystom podwójny, z naprzemiennie ustawionymi ząbkami i segmentami}
\Clue{56}{}{kufel o pojemności pół kwarty lub jednego litra}
\Clue{57}{}{Tricholoma equestre - gatunek grzyba należący do rodziny gąskowatych}\end{PuzzleClues}

\begin{PuzzleClues}{\textbf{Pionowe}\\}\Clue{1}{}{oryginalna polska zbroja łuskowa produkowana od XVI do XVIII wieku; składała się ze skórzanego kaftana pokrytego metalowymi płytkami w kształcie łusek nachodzących na siebie}
\Clue{2}{}{czas tego, co wydarza się w tym roku}
\Clue{3}{}{wybitny kompozytor rosyjski (1840-1893); opery; 'Dama Pikowa', balety; 'Jezioro łabędzie', symfonie utwory orkiestrowe, kameralne, fortepianowe, koncerty, pieśni}
\Clue{6}{}{atmosfera miejsca, zdarzenia, grupy osób}
\Clue{7}{}{popularna nazwa podatku płaconego od posiadanych dóbr majątkowych o wysokiej wartości}
\Clue{9}{}{coś poza światem, przestrzeń poza światem, ziemią}
\Clue{10}{}{grecki astronom, filozof, matematyk (275-194 p.n.e.), wyznaczył kąt nachylenia ekliptyki do równika niebieskiego}
\Clue{11}{}{trioza należąca do ketoz}
\Clue{12}{}{pogardliwie: rozpieszczone dziecko, takie, które trzeba wyręczać w codziennych, przyziemnych sprawach, któremu trzeba usługiwać, takie, które uważa się za pępek świata}
\Clue{14}{}{nazwa ssaka z rzędu szczerbaków}
\Clue{15}{}{Graptemys oculifera - gatunek gada z rodziny żółwi błotnych o charakterystycznym ubarwieniu: na każdej wewnętrznej blaszce karapaksu znajdują się żółte bądź pomarańczowe kółka, a na zewnętrznej półokręgi tej samej barwy}
\Clue{16}{}{w szachach, linia 1 lub 8, gdzie pionek przeciwnika może się zmienić w dowolną figurę}
\Clue{17}{}{produkt służący do pielęgnacji i upiększania ciała, zwierząt lub rzeczy}
\Clue{18}{}{stan w płd-zach Nigerii, główne miasto Benin}
\Clue{19}{}{część Zatoki Perskiej u wybrzeży Zjedn. Emiratów Arabskich i Kataru}
\Clue{20}{}{w feudalnej Japonii: chłop pańszczyźniany}
\Clue{21}{}{porcja twarogu; określona ilość tego produktu, zazwyczaj opakowanie, plastikowy pojemni, kubeczek}
\Clue{22}{}{organiczny związek chemiczny, terpen}
\Clue{23}{}{jeden z chrześcijańskich sakramentów; nadanie imienia}
\Clue{26}{}{organ roślin nasiennych powstający z zapłodnionego zalążka i składający się z zarodka otoczonego tkanką zapasową i osłoniętego łupiną nasienną}
\Clue{27}{}{młody chłopiec, nastolatek}
\Clue{28}{}{w chemii: symbol antymonu}
\Clue{30}{}{największa żyła serca, uchodzi do prawego jego przedsionka}
\Clue{31}{}{treść zgodna z rzeczywistością}
\Clue{32}{}{nóż ogrodniczy}
\Clue{33}{}{pogląd, zakładający, że powszechnie przyjęta interpretacja holocaustu jest albo w dużym stopniu przesadzona, albo całkowicie zafałszowana}
\Clue{35}{}{starorzymski urzędnik, którego funkcją była jurysdykcja cywilna na terenie miasta Rzymu}
\Clue{36}{}{miasto w Australii (Australia Płd.)}
\Clue{38}{}{warzywo strączkowe; ziarno ciecierzycy pospolitej}
\Clue{39}{}{giętki przewód rurowy}
\Clue{40}{}{Callionymus lyra - gatunek morskiej ryby okoniokształtnej z rodziny lirowatych}
\Clue{41}{}{prostokątny lub trapezowy żagiel rejowy podnoszony na półmaszcie szkunera}
\Clue{42}{}{łatwostrawny pokarm pszczół i starszych larw}
\Clue{43}{}{powodowanie odchodzenia przez kogoś od wartości moralnych}
\Clue{44}{}{Rhodopis vesper - gatunek ptaka z rzędu jerzykowych (Apodiformes), z rodziny kolibrów (Trochilidae), z podrodziny kolibrów (Trochilinae)}
\Clue{47}{}{szkoła filozoficzna akcentująca autorytet Wed}
\Clue{48}{}{poeta (1924-44), zginął w powstaniu warszawskim; „Homer i Orchidea”}\end{PuzzleClues}\newpage\section*{Krzyżówka 174}

\noindent\begin{Puzzle}{25}{20}|*	|*	|*	|*	|*	|*	|*	|*	|*	|*	|*	|*	|*	|*	|*	|*	|*	|*	|[1][S]\darr	|*	|*	|*	|*	|*	|*	|*	|.
|*	|*	|*	|*	|*	|*	|*	|*	|*	|*	|*	|*	|*	|*	|*	|*	|*	|*	|m	|*	|*	|*	|*	|*	|*	|*	|.
|*	|*	|*	|*	|*	|*	|*	|*	|*	|*	|*	|*	|*	|*	|*	|[2][S]\darr	|*	|*	|e	|*	|*	|*	|*	|*	|*	|*	|.
|*	|*	|*	|*	|*	|*	|*	|*	|*	|*	|*	|*	|*	|*	|[3][S]\darr	|w	|*	|*	|l	|*	|*	|*	|*	|*	|*	|[4][S]\darr	|.
|*	|*	|*	|*	|*	|*	|*	|*	|*	|*	|*	|*	|*	|*	|u	|e	|*	|[5][S]\darr	|o	|*	|*	|*	|*	|*	|*	|c	|.
|*	|*	|*	|*	|*	|*	|*	|*	|*	|[6][S]\darr	|*	|*	|*	|*	|l	|c	|*	|p	|d	|*	|*	|*	|*	|*	|*	|h	|.
|*	|*	|*	|*	|*	|*	|*	|*	|*	|k	|*	|*	|*	|*	|t	|k	|*	|e	|r	|*	|*	|*	|*	|*	|*	|l	|.
|*	|*	|*	|*	|*	|*	|[7][S]\drarr	|u	|n	|a	|j	|z	|a	|u	|r	|*	|*	|n	|a	|*	|*	|*	|*	|*	|*	|e	|.
|*	|*	|*	|*	|*	|*	|k	|*	|*	|w	|*	|[8][S]\drarr	|k	|n	|a	|g	|a	|*	|m	|*	|*	|*	|[9][S]\darr	|*	|*	|b	|.
|*	|*	|*	|[10][S]\rarr	|k	|o	|r	|o	|n	|a	|*	|i	|[11][S]\rarr	|k	|r	|o	|p	|k	|a	|[][,]{ }	|n	|a	|d	|[][,]{ }	|i	|*	|.
|*	|*	|*	|*	|*	|*	|o	|*	|*	|[][,]{ }	|*	|r	|*	|*	|o	|*	|*	|*	|t	|*	|*	|*	|y	|*	|*	|*	|.
|*	|*	|*	|*	|*	|*	|k	|*	|*	|m	|*	|o	|*	|*	|j	|*	|*	|*	|*	|*	|*	|*	|n	|*	|*	|*	|.
|*	|[12][S]\rarr	|k	|o	|p	|i	|s	|t	|k	|a	|*	|n	|[13][S]\rarr	|k	|a	|l	|m	|u	|s	|ó	|w	|k	|a	|*	|*	|*	|.
|*	|*	|*	|*	|*	|*	|z	|*	|*	|c	|*	|i	|*	|*	|l	|*	|*	|*	|*	|*	|*	|*	|m	|*	|*	|*	|.
|*	|*	|*	|*	|*	|*	|t	|*	|*	|c	|*	|c	|*	|*	|i	|*	|*	|*	|*	|*	|*	|*	|i	|*	|*	|*	|.
|*	|*	|*	|*	|*	|*	|y	|*	|*	|h	|*	|z	|*	|*	|s	|*	|*	|*	|*	|*	|*	|*	|k	|*	|*	|*	|.
|*	|*	|*	|*	|*	|*	|n	|*	|*	|i	|*	|n	|*	|*	|t	|*	|*	|*	|*	|*	|*	|*	|a	|*	|*	|*	|.
|*	|*	|*	|*	|*	|*	|*	|*	|*	|a	|*	|o	|*	|*	|a	|*	|*	|*	|*	|*	|*	|*	|*	|*	|*	|*	|.
|*	|*	|*	|*	|*	|*	|*	|*	|*	|t	|*	|ś	|*	|*	|*	|*	|*	|*	|*	|*	|*	|*	|*	|*	|*	|*	|.
|[14][S]\rarr	|z	|j	|e	|ł	|c	|z	|a	|ł	|o	|ś	|ć	|*	|*	|*	|*	|*	|*	|*	|*	|*	|*	|*	|*	|*	|*	|.
|*	|*	|*	|*	|*	|*	|*	|*	|*	|*	|*	|*	|*	|*	|*	|*	|*	|*	|*	|*	|*	|*	|*	|*	|*	|*	|.\end{Puzzle}

\newpage

\begin{PuzzleClues}{\textbf{Poziome}\\}\Clue{7}{}{Unaysaurus - rodzaj bazalnego zauropodomorfa; żył w okresie triasu na terenach Ameryki Południowej}
\Clue{8}{}{duża kość}
\Clue{10}{}{Fritillaria - rodzaj rośliny z rodziny liliowatych}
\Clue{11}{}{mocny, wyrazisty akcent, często na końcu wypowiedzi, dookreślenie, wyraźne, dosadne podsumowanie, puenta}
\Clue{12}{}{kobieta, która sporządza kopie dla archiwum}
\Clue{13}{}{nalewka sporządzona na korzeniach tataraku}
\Clue{14}{}{cecha jedzenia, które zjełczało}\end{PuzzleClues}

\begin{PuzzleClues}{\textbf{Pionowe}\\}\Clue{1}{}{jakieś dramatyczne wydarzenie w życiu, przedstawione w sposób przejaskrawiony}
\Clue{2}{}{szklany słoik z takąż pokrywką zamykany za pomocą gumowej uszczelki i sprężyny służący do przechowywania przetworów owocowych, jarzynowych i mięsnych}
\Clue{3}{}{skrajny rojalista}
\Clue{4}{}{pieczywo z mąki pszennej, żytniej lub mieszanej, ewent. z domieszką kukurydzianej, jęczmiennej lub owsianej, z wody, soli i czynników spulchniających (drożdże, zakwas) wypiekane w temp. ok. 250 oC}
\Clue{5}{}{kod ISO 4217 waluty sol (nowy sol)}
\Clue{6}{}{espresso z niewielką ilością mleka}
\Clue{7}{}{zakończenie belki stropowej, wysunięte przed lico muru}
\Clue{8}{}{cecha sytuacji, życia, zdarzenia - złośliwość losu występująca pod przykrywką innych, pozytywnych zdarzeń}
\Clue{9}{}{cecha czegoś, co nie jest czynnością, ruchem, ale sprawia wrażenie dynamicznego, pełnego ruchu, życia, koloru}\end{PuzzleClues}\newpage\section*{Krzyżówka 175}

\noindent\begin{Puzzle}{24}{32}|*	|*	|*	|*	|*	|*	|*	|*	|*	|[1][S]\drarr	|k	|r	|o	|ś	|n	|i	|c	|a	|*	|*	|[2][S]\drarr	|b	|e	|z	|*	|.
|*	|*	|*	|[3][S]\drarr	|t	|e	|s	|t	|[][,]{ }	|p	|s	|y	|c	|h	|o	|l	|o	|g	|i	|c	|z	|n	|y	|*	|[4][S]\darr	|.
|*	|*	|*	|w	|*	|*	|*	|[5][S]\rarr	|h	|a	|f	|t	|[][,]{ }	|k	|o	|s	|z	|y	|k	|o	|w	|y	|*	|*	|m	|.
|*	|[6][S]\darr	|[7][S]\drarr	|p	|e	|r	|f	|i	|d	|n	|o	|ś	|ć	|*	|*	|*	|*	|*	|*	|*	|i	|*	|[8][S]\darr	|*	|a	|.
|*	|d	|z	|r	|[9][S]\darr	|*	|*	|*	|*	|t	|*	|*	|[10][S]\darr	|[11][S]\darr	|[12][S]\darr	|*	|*	|[13][S]\darr	|*	|*	|ą	|*	|f	|*	|r	|.
|*	|u	|a	|a	|h	|*	|[14][S]\rarr	|d	|ż	|e	|t	|*	|p	|p	|b	|*	|*	|ż	|*	|*	|z	|*	|r	|*	|c	|.
|*	|s	|p	|w	|u	|*	|*	|*	|*	|r	|*	|*	|o	|ó	|i	|*	|[15][S]\darr	|a	|[16][S]\drarr	|w	|e	|d	|e	|l	|*	|.
|[17][S]\rarr	|z	|a	|k	|r	|ę	|t	|*	|*	|k	|*	|*	|k	|ł	|o	|*	|w	|g	|g	|[18][S]\darr	|k	|[19][S]\darr	|s	|*	|*	|.
|*	|a	|c	|a	|m	|[20][S]\rarr	|o	|d	|p	|a	|d	|*	|r	|a	|f	|*	|e	|i	|o	|n	|[][,]{ }	|d	|c	|*	|*	|.
|*	|*	|h	|*	|a	|*	|*	|*	|*	|*	|*	|*	|y	|r	|l	|*	|i	|e	|l	|a	|m	|o	|o	|[21][S]\darr	|*	|.
|*	|*	|*	|*	|*	|*	|*	|*	|*	|[22][S]\darr	|*	|*	|w	|k	|a	|[23][S]\drarr	|f	|l	|i	|s	|a	|k	|*	|b	|*	|.
|[24][S]\rarr	|p	|o	|d	|b	|r	|ó	|d	|e	|k	|*	|*	|a	|u	|w	|s	|a	|e	|z	|y	|ł	|t	|*	|r	|*	|.
|[25][S]\drarr	|z	|a	|w	|a	|l	|i	|d	|r	|o	|g	|a	|*	|s	|o	|z	|n	|k	|n	|c	|ż	|o	|[26][S]\darr	|o	|*	|.
|i	|[27][S]\darr	|*	|[28][S]\rarr	|c	|e	|l	|o	|w	|n	|i	|k	|*	|z	|n	|a	|g	|*	|a	|e	|e	|r	|b	|ń	|*	|.
|f	|k	|*	|*	|*	|[29][S]\drarr	|d	|e	|m	|a	|g	|o	|g	|*	|o	|r	|*	|*	|*	|n	|ń	|*	|r	|[][,]{ }	|[30][S]\darr	|.
|e	|a	|*	|*	|*	|k	|*	|[31][S]\darr	|*	|r	|*	|[32][S]\darr	|*	|*	|i	|z	|*	|*	|*	|i	|s	|*	|y	|s	|p	|.
|*	|n	|*	|*	|*	|u	|*	|s	|*	|*	|*	|t	|*	|*	|d	|y	|[33][S]\darr	|*	|*	|e	|k	|[34][S]\darr	|z	|a	|u	|.
|[35][S]\drarr	|i	|n	|f	|o	|r	|m	|a	|t	|o	|r	|i	|u	|m	|*	|z	|m	|[36][S]\darr	|*	|*	|i	|w	|g	|m	|l	|.
|w	|u	|[37][S]\drarr	|g	|o	|t	|y	|k	|[][,]{ }	|p	|ł	|o	|m	|i	|e	|n	|i	|s	|t	|y	|*	|i	|u	|o	|m	|.
|i	|k	|l	|*	|*	|y	|[38][S]\darr	|i	|[39][S]\drarr	|h	|a	|s	|i	|m	|*	|a	|z	|p	|*	|[40][S]\darr	|*	|e	|n	|c	|a	|.
|e	|[][,]{ }	|i	|*	|*	|n	|p	|*	|p	|*	|*	|ó	|*	|*	|*	|*	|e	|o	|[41][S]\darr	|m	|*	|l	|[][,]{ }	|z	|n	|.
|k	|z	|l	|*	|[42][S]\rarr	|a	|e	|r	|o	|z	|o	|l	|[][,]{ }	|s	|i	|a	|r	|c	|z	|a	|n	|o	|w	|y	|*	|.
|[][,]{ }	|w	|a	|*	|*	|*	|r	|[43][S]\rarr	|p	|a	|w	|*	|*	|*	|*	|*	|k	|z	|g	|r	|[44][S]\darr	|p	|i	|n	|*	|.
|c	|y	|k	|[45][S]\rarr	|t	|a	|l	|e	|r	|z	|e	|*	|*	|*	|*	|*	|a	|n	|o	|k	|s	|e	|e	|n	|*	|.
|h	|c	|*	|*	|[46][S]\rarr	|d	|u	|s	|z	|y	|c	|z	|k	|a	|*	|*	|*	|i	|r	|o	|k	|s	|r	|a	|*	|.
|ł	|z	|[47][S]\rarr	|b	|o	|l	|s	|z	|e	|w	|i	|z	|a	|c	|j	|a	|*	|k	|z	|t	|a	|t	|z	|*	|*	|.
|o	|a	|[48][S]\drarr	|t	|e	|s	|t	|[][,]{ }	|c	|i	|ą	|ż	|o	|w	|y	|*	|*	|*	|e	|n	|n	|k	|b	|*	|*	|.
|p	|j	|m	|[49][S]\drarr	|m	|a	|r	|s	|z	|a	|ł	|k	|ó	|w	|n	|a	|*	|*	|l	|o	|e	|o	|o	|*	|*	|.
|i	|n	|e	|k	|*	|*	|a	|[50][S]\rarr	|k	|o	|r	|p	|o	|r	|a	|c	|j	|a	|*	|ś	|r	|w	|w	|*	|*	|.
|ę	|y	|s	|a	|*	|*	|c	|[51][S]\rarr	|a	|r	|b	|i	|t	|r	|a	|l	|n	|o	|ś	|ć	|*	|i	|i	|*	|*	|.
|c	|*	|t	|p	|*	|*	|j	|*	|*	|*	|*	|*	|*	|*	|*	|*	|*	|*	|*	|*	|*	|e	|e	|*	|*	|.
|y	|*	|i	|a	|[52][S]\rarr	|m	|a	|ł	|p	|e	|c	|z	|k	|i	|*	|*	|*	|*	|*	|*	|*	|c	|c	|*	|*	|.
|*	|*	|*	|*	|*	|*	|*	|*	|*	|*	|*	|*	|*	|*	|*	|*	|*	|*	|*	|*	|*	|*	|*	|*	|*	|.\end{Puzzle}

\newpage

\begin{PuzzleClues}{\textbf{Poziome}\\}\Clue{1}{}{wieś w Polsce położona w województwie opolskim, w powiecie strzeleckim, w gminie Izbicko}
\Clue{2}{}{Syringa, lilak - niepoprawna (z botanicznego punktu widzenia), często używana nazwa roślin z rodziny oliwkowatych}
\Clue{3}{}{narzędzie badawcze w psychologii pozwalające na uzyskanie takiej reprezentatywnej próbki zachowań, o których można przyjąć (np. na podstawie założeń teoretycznych lub związków empirycznych), że są one wskaźnikami interesującej nas cechy psychologicznej}
\Clue{5}{}{haft tworzony poprzez przeplatanie ściegów poziomych z pionowymi}
\Clue{7}{}{perfidia - cecha człowieka, polegająca na podstępności połączonej ze złośliwością}
\Clue{14}{}{bitumiczna odmiana węgla brunatnego o zbitej, jednorodnej budowie}
\Clue{16}{}{technika zjazdu na nartach w oparciu o wykonywanie skrętów na krótkich odcinkach poprzez pochylanie nart na boki, ku krawędziom}
\Clue{17}{}{okres w życiu lub innym ciągu wydarzeń, który jest odbierany przez uczestników wydarzeń jako trudny, ale po którym zwykle następuje jakaś (często pozytywna) zmiana}
\Clue{20}{}{pozostałość po czymś, uboczny rezultat jakiegoś działania}
\Clue{23}{}{osoba zawodowo zajmująca się przewozem turystów na rzece Dunajec}
\Clue{24}{}{część skrzypiec, na której grający opiera podbródek}
\Clue{25}{}{zawada; coś, co zawadza}
\Clue{28}{}{część broni służąca do wyznaczania elementom broni odpowiadającym za początkowy kierunek lotu pocisku takiego położenia, aby tor pocisku przechodził przez cel}
\Clue{29}{}{pogardliwie o człowieku, który zdobywa sobie poparcie za pomocą chwytliwych haseł}
\Clue{35}{}{punkt, w którym udzielane są informacje, szczególnie w bibliotekach i czytelniach na temat zasad i sposobów udostępniania zbiorów}
\Clue{37}{}{późna odmiana gotyku, nacechowana m.in. zwiększoną jasnością wnętrz i ubogaconą dekoracją sklepień}
\Clue{39}{}{poeta turecki (1884-1933), twórczość pod wpływem symbolizmu tureckiego, zbiory poezji, szkice literackie}
\Clue{42}{}{aerozole atmosferyczne zawierające siarczany lub kwas siarkowy}
\Clue{43}{}{ptak z rodziny kurowatych (Phasianidae)}
\Clue{45}{}{żele, czynele}
\Clue{46}{}{jednostka, byt}
\Clue{47}{}{narzucanie rządów i doktryny marksistowsko-leninowskiej, często wbrew woli zainteresowanych}
\Clue{48}{}{płytka pozwalająca wykryć wzrost stężenia gonadotropiny kosmówkowej (hCG) w moczu kobiety}
\Clue{49}{}{córka marszałka}
\Clue{50}{}{rodzaj organizacji społecznej, zazwyczaj posiadającej osobowość prawną, której istotnym substratem są jej członkowie (korporanci)}
\Clue{51}{}{cecha tego, co arbitralne: niepodlegające dyskusji, odgórnie ustalone, uchodzące za takie}
\Clue{52}{}{saki, Pitheciinae - podrodzina małp szerokonosych z rodziny Pitheciidae obejmująca m.in. właściwe saki (saki białolica, saki szara), uakari oraz szatankę; występują w lasach północnej i centralnej Ameryki Południowej}\end{PuzzleClues}

\begin{PuzzleClues}{\textbf{Pionowe}\\}\Clue{1}{}{Chromileptes altivelis - gatunek ryby okoniokształtnej z rodziny strzępielowatych}
\Clue{2}{}{związek kobiety i mężczyzny usankcjonowany prawnie lub religijnie}
\Clue{3}{}{utwór do ćwiczenia techniki gry na czymś lub techniki wokalnej}
\Clue{4}{}{niemiecki malarz i grafik (1880-1916) kompozycje oparte na tematach animalistycznych}
\Clue{6}{}{efemistyczne określenie dupy}
\Clue{7}{}{olejek aromatyczny o swoistym zapachu, używany jako dodatek do ciast; ekstrakt roślinny lub jego syntetyczny odpowiednik}
\Clue{8}{}{ubraniowa tkanina wełniana o splocie płóciennym i ziarnistej powierzchni, używana do czycia cieńszych wiosenno-letnich garniturów}
\Clue{9}{}{drzewo lub krzew międzyzwrotnikowy uprawiany dla słodkich, jadalnych owoców}
\Clue{10}{}{odejmowana lub odchylana część obudowy maszyny}
\Clue{11}{}{jednostka miary stanowiąca połowę arkusza}
\Clue{12}{}{flawonoid, związek flawonowy - organiczny związek chemiczny występujący w roślinach, spełniający funkcję barwnika, przeciwutleniacza i naturalnego insektycyda oraz fungicyda, chroniących przed atakiem ze strony owadów i grzybów; jest barwnikiem znajdującycm się w powierzchniowych warstwach tkanek roślinnych, nadającym im intensywny kolor i ograniczającym szkodliwy wpływ promieniowania ultrafioletowego; często mówi się tak o flawonoidzie pozyskanym z owocu cytrusowego, ale też niejednokrotnie pojęcie to jest używane na określenie dowolnego flawonoidu}
\Clue{13}{}{zdrobniale - żagiel: rodzaj pędnika wiatrowego stosowanego do napędów żaglowców, jachtów, bojerów, żaglowozów itd}
\Clue{15}{}{Wejfang, miasto w Chinach (Shandong); węzeł drogowy}
\Clue{16}{}{przenośnie o biedzie}
\Clue{18}{}{stan czegoś wynikający z dokonanej czynności nasycania (cieczą itp.)}
\Clue{19}{}{lekarz - osoba, która zawodowo leczy ludzi albo zwierzęta; człowiek, który posiada właściwe kwalifikacje, uprawnienia do udzielania świadczeń zdrowotnych}
\Clue{21}{}{rodzaj broni automatycznej wyposażonej w mechanizm spustowy umożliwiający prowadzenie wyłącznie ognia ciągłego (seriami)}
\Clue{22}{}{(1862-1940), pisarz, powieści obyczajowe, nowele, komedie, aforyzmy; „Bankruci”, „W syrenim grodzie”}
\Clue{23}{}{brak światła, jasności}
\Clue{25}{}{miasto w płd.-zach. Nigerii, duchowa stolica ludu Jorubu, 214,5 tys. mieszkańców (1983), ośrodek handlu i rzemiosła}
\Clue{26}{}{Cimbex luteus - owad z rodziny bryzgunowatych}
\Clue{27}{}{Elanus caeruleus - gatunek średniego ptaka drapieżnego z rodziny jastrzębiowatych (Accipitridae), z podrodziny kaniuków (Elaninae)}
\Clue{29}{}{część fortyfikacji o narysie bastionowym, prosty odcinek wału ziemnego, często umocnionego murem z cegły lub kamienia, cofnięty względem bastionów, łączy poszczególne bastiony ze sobą}
\Clue{30}{}{wagon pasażerski, który ma długi, wąski korytarz, do którego prowadzą wejścia do przedziałów}
\Clue{31}{}{miasto na Ukrainie nad jeziorem Saki}
\Clue{32}{}{siarkosól - sól siarkowego odpowiednika kwasu karboksylowego, w którego cząsteczkach 1 lub 2 atomy tlenu zostały zastąpione atomami siarki}
\Clue{33}{}{dawna nazwa wybijanki, odmiany warcab}
\Clue{34}{}{w botanice: rodzaj owocu zbiorowego, który powstaje przez zrośnięcie się licznych drobnych pestkowców osadzonych na wspólnym zmięśniałym dnie kwiatowym}
\Clue{35}{}{najwcześniejsze lata życia mężczyzny}
\Clue{36}{}{podest; element poziomy schodów}
\Clue{37}{}{BEZ}
\Clue{38}{}{kontrola czegoś, przegląd}
\Clue{39}{}{przenośnie: pewnien zestaw wymagań, który gdzieś istnieje, który ktoś chce spełnić, żeby coś osiągnąć}
\Clue{40}{}{stan psychiczny człowieka przygnębionego, będącego w złym humorze}
\Clue{41}{}{rodzaj nekrozy w organizmie roślinnym, prowadzący do zamierania całych organów}
\Clue{44}{}{urządzenie służące do przebiegowego odczytywania: obrazu, kodu paskowego lub magnetycznego, fal radiowych itp. do formy elektronicznej (najczęściej cyfrowej)}
\Clue{48}{}{miasto w Grecji, węzeł kolejowy i drogowy}
\Clue{49}{}{okrycie dla konia, często zdobione}\end{PuzzleClues}\newpage\section*{Krzyżówka 176}

\noindent\begin{Puzzle}{18}{27}|*	|*	|[1][S]\darr	|[2][S]\drarr	|p	|a	|s	|e	|k	|*	|[3][S]\drarr	|p	|ę	|t	|ó	|w	|k	|a	|*	|.
|*	|[4][S]\rarr	|k	|l	|e	|j	|n	|o	|t	|*	|k	|[5][S]\darr	|[6][S]\darr	|*	|*	|*	|*	|*	|*	|.
|*	|*	|i	|o	|*	|[7][S]\drarr	|c	|y	|b	|e	|r	|p	|u	|n	|k	|*	|*	|*	|[8][S]\darr	|.
|*	|[9][S]\darr	|e	|s	|[10][S]\drarr	|g	|ó	|r	|a	|*	|a	|b	|g	|*	|*	|*	|[11][S]\darr	|*	|r	|.
|*	|j	|ł	|i	|e	|e	|[12][S]\drarr	|z	|b	|*	|ś	|i	|r	|*	|[13][S]\darr	|[14][S]\darr	|j	|[15][S]\darr	|o	|.
|*	|u	|*	|l	|g	|i	|p	|*	|[16][S]\darr	|*	|n	|t	|u	|*	|k	|m	|a	|a	|s	|.
|[17][S]\drarr	|t	|e	|l	|e	|g	|r	|a	|m	|*	|i	|*	|p	|*	|a	|a	|ł	|n	|a	|.
|m	|a	|[18][S]\darr	|a	|r	|e	|z	|[19][S]\darr	|a	|[20][S]\darr	|k	|[21][S]\drarr	|o	|l	|b	|r	|o	|t	|*	|.
|i	|*	|b	|z	|i	|r	|ę	|l	|g	|b	|[][,]{ }	|b	|w	|[22][S]\darr	|i	|y	|w	|y	|*	|.
|ę	|*	|e	|a	|a	|*	|s	|o	|a	|i	|p	|e	|a	|e	|n	|n	|i	|f	|*	|.
|t	|*	|z	|u	|*	|*	|ł	|t	|z	|l	|i	|l	|n	|s	|a	|i	|e	|r	|*	|.
|a	|*	|p	|r	|[23][S]\darr	|*	|o	|n	|y	|l	|ę	|l	|i	|c	|*	|s	|c	|a	|*	|.
|[][,]{ }	|*	|r	|*	|t	|[24][S]\darr	|*	|i	|n	|i	|c	|o	|e	|o	|[25][S]\darr	|t	|[][,]{ }	|z	|*	|.
|p	|[26][S]\drarr	|o	|l	|e	|i	|n	|a	|*	|n	|i	|*	|*	|r	|e	|a	|p	|a	|*	|.
|i	|g	|b	|*	|r	|b	|[27][S]\rarr	|r	|o	|g	|o	|ź	|n	|i	|k	|*	|o	|*	|*	|.
|e	|o	|l	|*	|m	|i	|*	|z	|*	|*	|p	|*	|*	|a	|s	|*	|ś	|*	|*	|.
|p	|ł	|e	|*	|o	|z	|*	|[][,]{ }	|*	|*	|l	|*	|*	|l	|t	|[28][S]\darr	|r	|*	|*	|.
|r	|ą	|m	|[29][S]\drarr	|p	|a	|p	|r	|o	|c	|a	|n	|y	|*	|a	|s	|e	|*	|*	|.
|z	|b	|o	|p	|l	|*	|*	|d	|*	|*	|m	|*	|*	|*	|z	|c	|d	|*	|*	|.
|o	|[][,]{ }	|w	|a	|a	|*	|*	|z	|[30][S]\darr	|[31][S]\darr	|e	|[32][S]\rarr	|s	|h	|a	|h	|n	|*	|*	|.
|w	|d	|o	|r	|s	|[33][S]\drarr	|w	|a	|b	|i	|k	|*	|*	|*	|*	|e	|i	|*	|*	|.
|a	|o	|ś	|a	|t	|m	|*	|w	|i	|ż	|*	|*	|*	|*	|*	|r	|*	|*	|*	|.
|*	|m	|ć	|f	|y	|e	|*	|o	|g	|e	|[34][S]\drarr	|p	|a	|r	|t	|c	|h	|*	|*	|.
|*	|o	|*	|r	|k	|s	|[35][S]\drarr	|s	|a	|w	|a	|n	|n	|a	|*	|h	|*	|*	|*	|.
|*	|w	|*	|a	|a	|s	|m	|z	|*	|s	|n	|*	|*	|*	|*	|e	|*	|*	|*	|.
|*	|y	|*	|z	|*	|e	|o	|y	|*	|k	|g	|*	|[36][S]\rarr	|b	|o	|n	|e	|*	|*	|.
|*	|*	|*	|a	|*	|l	|a	|i	|*	|*	|e	|*	|*	|*	|*	|*	|*	|*	|*	|.
|*	|*	|*	|*	|*	|*	|*	|*	|*	|*	|*	|*	|*	|*	|*	|*	|*	|*	|*	|.\end{Puzzle}

\newpage

\begin{PuzzleClues}{\textbf{Poziome}\\}\Clue{2}{}{związek, w którym świadczy się jakieś usługi, ktoś działa na rzecz kogoś lub czegoś}
\Clue{3}{}{mało znana, dawna nazwa ziemniaka - jadalnej bulwy rośliny nazywanej tak samo (etymologia tej nazwy wiąże się prawdopodobnie z faktem, iż ziemniaki to warzywa, które przybyły do Polski z dalekich krajów - źródłosłowem może być staropolski wyrazpąć, który oznacza drogę)}
\Clue{4}{}{drogi przedmiot wykonany z użyciem metali i kamieni szlachetnych i półszlachetnych}
\Clue{7}{}{nurt w literaturze science-fiction, który skupia się na ludziach i otaczającej ich zaawansowanej technologii komputerowej i informacyjnej, niekiedy połączony z różnego rodzaju zamieszaniem w społeczeństwie (np. stan wojenny)}
\Clue{10}{}{piętro, strych, poddasze}
\Clue{12}{}{odpowiednik zettabajta w systemie dwójkowym, równy 2\textasciicircum70 bajtów}
\Clue{17}{}{tekst przesłany początkowo przez telegraf, później przez dalekopis, współcześnie przez telefon, faks, e-mail lub SMS}
\Clue{21}{}{spermacet; wydzielina otrzymywana z tłuszczu kaszalota, przemysł farmaceutyczny i kosmetyczny}
\Clue{26}{}{ester glicerolu i kwasu oleinowego}
\Clue{27}{}{wieś w Polsce położona w Zagłębiu Dąbrowskim, w województwie śląskim, w powiecie będzińskim, w gminie Bobrowniki}
\Clue{29}{}{dzielnica Tychów położona w południowej części miasta}
\Clue{32}{}{malarz, grafik, ilustrator ur. w 1846, od 1977 r. w USA; scenograf}
\Clue{33}{}{coś (przyrząd, substancja, pożywienie), co służy do wabienia zwierząt}
\Clue{34}{}{amerykański kompozytor (1904-1974); konstruktor wielu oryginalnych instrumentów muzycznych}
\Clue{35}{}{jedna z wielu formacji trawiastych strefy międzyzwrotnikowej o klimacie gorącym z wyraźnie zaznaczoną porą suchą, trwającą od 6 do 9 miesięcy oraz porą deszczową}
\Clue{36}{}{zatoka u wybrzeży Celebesu (Indonezja)}\end{PuzzleClues}

\begin{PuzzleClues}{\textbf{Pionowe}\\}\Clue{1}{}{krótki występ na czołowej powierzchni elementu np. tarczy, sprzęgła}
\Clue{2}{}{Losillasaurus - rodzaj dużego zauropoda z grupy turiazaurów; żył na przełomie jury i kredy na terenach współczesnej Europy}
\Clue{3}{}{Zygaena trifolii - gatunek motyla z rodziny kraśnikowatych; występuje na większości obszaru Europy; brak go w Norwegii, Finlandii, Irlandii i na Bałkanach}
\Clue{5}{}{binarny odpowiednik pebitu równy 2\textasciicircum50 = 1024\textasciicircum5 bitów}
\Clue{6}{}{zgrupowanie oddziałów pod wspólnym dowództwem, zawiązane doraźnie w trakcie walk w wyniku określonej sytuacji taktycznej lub operacyjnej}
\Clue{7}{}{fizyk niemiecki (1882-1945); badacz promieniotwórczości, skonstruował licznik do rejestracji cząstek naładowanych}
\Clue{8}{}{osad atmosferyczny w postaci kropel wody powstających na powierzchni skał, roślin i innych przedmiotów w wyniku skraplania się pary wodnej zawartej w powietrzu}
\Clue{9}{}{tkanina o splocie płóciennym wytwarzana z włókna o tej samej nazwie}
\Clue{10}{}{w mitologii rzymskiej kamena (nimfa), wieszczka}
\Clue{11}{}{Juniperius media - gatunek z rodziny cyprysowatych; mieszaniec jałowca chińskiego i jałowca sabińskiego}
\Clue{12}{}{rama toru}
\Clue{13}{}{KAJUTA; mieszkalne pomieszczenie na statku}
\Clue{14}{}{twórca różnorakich dzieł o tematyce morskiej}
\Clue{15}{}{litota}
\Clue{16}{}{czasopismo (najczęściej tygodnik), będące uzupełnieniem codziennych wydań określonej gazety, wydawane przez tego samego wydawcę co gazeta, często drukowane na lepszym papierze i o bogatszej zawartości graficznej niż wydania codzienne}
\Clue{17}{}{Mentha ×piperita - gatunek rośliny należący do rodziny jasnotowatych; spontaniczny mieszaniec międzygatunkowy mięty nadwodnej i mięty zielonej}
\Clue{18}{}{cecha osoby, która nie tworzy lub nie wyolbrzymia problemów}
\Clue{19}{}{Streptoprocne rutila - gatunek ptaka z rzędu jerzykowych (Apodiformes), z rodziny jerzykowatych (Apodidae)}
\Clue{20}{}{rachunek szczegółowy, zestawienie wszystkich opłat za połączenia i usługi dodane, jakie abonent przeprowadził w danym okresie rozliczeniowym}
\Clue{21}{}{miasto w Kolumbii w Andach; przemysł włókienniczy i spożywczy}
\Clue{22}{}{miejscowość w Hiszpanii (Nowa Kastylia), w pobliżu Madrytu; późnorenesansowy zespół architektoniczny z XVI w}
\Clue{23}{}{wytwarzanie przedmiotów z termoplastu}
\Clue{24}{}{seat z modelu Ibiza}
\Clue{25}{}{stan świadomości, uniesienie duchowe, któremu towarzyszy koncentracja na przedmiocie pragnień i zmniejszenie wrażliwości na bodźce zewnętrzne}
\Clue{26}{}{Columba livia domestica - udomowiona forma gołębia skalnego (Columba livia)}
\Clue{28}{}{dyrygent niemiecki (1891-1966); osiadł w Szwajcarii, propagator muzyki współczesnej}
\Clue{29}{}{utwór instrumentalny, często o charakterze wirtuozowskim w formie swobodnej fantazji}
\Clue{30}{}{wóz o dwóch kołach bez kozła; bieda, biedka, dwukółka}
\Clue{31}{}{miasto w europejskiej części Federacji Rosyjskiej, stolica Udmurcji, położone nad rzeką Iż, dopływem Kamy}
\Clue{33}{}{architekt niemiecki (1853-1909), biurowce, domy mieszkalne i towarowe}
\Clue{34}{}{szwedzkie miasto nad rzeką Ljungan}
\Clue{35}{}{m. we wsch. Kubie, ośrodek wydobycia rud niklu i kobaltu}\end{PuzzleClues}\newpage\section*{Krzyżówka 177}

\noindent\begin{Puzzle}{20}{30}|*	|*	|*	|*	|[1][S]\drarr	|z	|a	|r	|o	|b	|e	|k	|*	|*	|*	|[2][S]\drarr	|f	|l	|i	|p	|*	|.
|*	|*	|*	|[3][S]\darr	|c	|[4][S]\darr	|[5][S]\darr	|[6][S]\rarr	|b	|o	|ż	|a	|[][,]{ }	|r	|ę	|k	|a	|*	|*	|*	|[7][S]\darr	|.
|*	|[8][S]\darr	|[9][S]\drarr	|g	|o	|s	|p	|o	|d	|a	|r	|k	|a	|[][,]{ }	|w	|o	|d	|n	|a	|*	|d	|.
|[10][S]\drarr	|b	|a	|r	|d	|z	|o	|[][,]{ }	|d	|o	|b	|r	|y	|*	|*	|ł	|*	|*	|*	|[11][S]\darr	|a	|.
|s	|i	|k	|i	|a	|t	|d	|[12][S]\rarr	|o	|d	|d	|y	|c	|h	|a	|n	|i	|e	|*	|m	|m	|.
|k	|m	|s	|e	|*	|u	|s	|*	|*	|*	|[13][S]\drarr	|p	|o	|t	|n	|i	|k	|*	|*	|e	|s	|.
|l	|a	|u	|k	|*	|c	|ą	|*	|[14][S]\rarr	|o	|p	|o	|n	|a	|*	|e	|*	|*	|*	|c	|e	|.
|e	|r	|m	|o	|*	|a	|d	|*	|*	|[15][S]\rarr	|r	|a	|m	|f	|o	|r	|y	|n	|c	|h	|*	|.
|p	|e	|*	|w	|*	|*	|n	|*	|*	|*	|z	|*	|[16][S]\rarr	|a	|l	|z	|a	|c	|j	|a	|*	|.
|i	|s	|*	|*	|*	|[17][S]\darr	|o	|*	|[18][S]\rarr	|v	|e	|g	|a	|*	|*	|[][,]{ }	|[19][S]\darr	|[20][S]\darr	|*	|n	|*	|.
|e	|t	|*	|*	|*	|w	|ś	|*	|*	|[21][S]\drarr	|s	|t	|ó	|ł	|*	|s	|n	|s	|*	|i	|*	|.
|n	|a	|*	|*	|[22][S]\darr	|o	|ć	|[23][S]\darr	|*	|f	|u	|*	|*	|*	|[24][S]\darr	|z	|i	|z	|*	|z	|*	|.
|i	|n	|[25][S]\rarr	|p	|a	|s	|*	|m	|*	|r	|w	|*	|*	|[26][S]\darr	|l	|a	|e	|t	|*	|m	|*	|.
|e	|*	|*	|[27][S]\darr	|s	|k	|*	|i	|*	|e	|n	|*	|*	|p	|i	|l	|m	|o	|*	|[][,]{ }	|*	|.
|[][,]{ }	|*	|*	|z	|s	|o	|*	|g	|*	|g	|i	|*	|[28][S]\darr	|e	|n	|o	|i	|s	|*	|p	|*	|.
|t	|*	|*	|w	|u	|w	|*	|d	|[29][S]\darr	|a	|k	|[30][S]\rarr	|g	|ł	|o	|w	|a	|*	|*	|o	|*	|.
|r	|*	|*	|i	|n	|o	|[31][S]\drarr	|a	|s	|t	|*	|*	|r	|n	|w	|y	|s	|*	|*	|d	|*	|.
|ó	|*	|*	|ą	|t	|ś	|w	|ł	|c	|a	|*	|*	|i	|i	|i	|*	|z	|[32][S]\darr	|*	|n	|*	|.
|j	|*	|*	|z	|a	|ć	|r	|e	|e	|[][,]{ }	|*	|*	|n	|a	|e	|*	|e	|n	|*	|i	|*	|.
|d	|*	|*	|e	|*	|*	|ó	|k	|n	|r	|*	|*	|*	|*	|c	|*	|k	|e	|*	|e	|*	|.
|z	|*	|*	|k	|*	|*	|b	|[][,]{ }	|o	|a	|*	|*	|*	|*	|*	|*	|*	|k	|*	|s	|*	|.
|i	|*	|*	|[][,]{ }	|*	|*	|l	|g	|p	|k	|*	|*	|*	|*	|*	|*	|*	|t	|*	|i	|*	|.
|e	|*	|*	|r	|*	|[33][S]\darr	|o	|a	|i	|i	|*	|*	|[34][S]\drarr	|k	|r	|z	|t	|a	|*	|e	|*	|.
|l	|*	|*	|z	|*	|u	|w	|r	|s	|e	|*	|*	|g	|*	|*	|*	|*	|r	|[35][S]\darr	|n	|*	|.
|n	|*	|*	|ą	|*	|k	|e	|d	|a	|t	|[36][S]\rarr	|s	|a	|f	|a	|r	|i	|*	|o	|i	|*	|.
|e	|*	|[37][S]\rarr	|d	|ó	|ł	|*	|ł	|r	|o	|*	|[38][S]\rarr	|b	|y	|ł	|y	|*	|*	|t	|o	|*	|.
|*	|[39][S]\rarr	|t	|u	|b	|a	|*	|o	|z	|w	|[40][S]\rarr	|k	|a	|n	|t	|o	|r	|*	|u	|w	|*	|.
|*	|*	|*	|*	|*	|d	|*	|w	|*	|a	|[41][S]\rarr	|k	|r	|k	|*	|*	|*	|*	|n	|y	|*	|.
|*	|*	|*	|*	|*	|*	|*	|y	|*	|*	|[42][S]\rarr	|m	|i	|s	|s	|o	|u	|r	|i	|*	|*	|.
|*	|[43][S]\rarr	|h	|u	|l	|o	|k	|*	|*	|*	|*	|*	|*	|*	|*	|*	|*	|*	|t	|*	|*	|.
|*	|*	|*	|*	|*	|*	|*	|*	|*	|*	|*	|*	|*	|*	|*	|*	|*	|*	|*	|*	|*	|.\end{Puzzle}

\newpage

\begin{PuzzleClues}{\textbf{Poziome}\\}\Clue{1}{}{praca, możliwość zarobienia pieniędzy}
\Clue{2}{}{figura w jeździe figurowej na łyżwach, polegająca na skoku tyłem z wewnętrznej krawędzi lewej łyżwy}
\Clue{6}{}{siła, która jest postrzegana jako mająca władzę nad człowiekiem, dająca mu coś lub zabierająca}
\Clue{9}{}{dział gospodarki obejmujący działania z zakresu planowania, rozwijania, dystrybuowania i zarządzania optymalnym zużyciem wody}
\Clue{10}{}{jedna z najwyższych ocen szkolnych, piątka}
\Clue{12}{}{proces życiowy związany z uzyskiwaniem przez organizmy energii użytecznej biologicznie}
\Clue{13}{}{kawałek materiału, umieszczany pod elementem garderoby, żeby absorbował pot}
\Clue{14}{}{(kabla) giętka powłoka z gumy wulkanicznej lub tworzywa termoplastycznego zabezpieczająca żyły kabla}
\Clue{15}{}{Rhamphorhynchus) - rodzaj niewielkiego pterozaura z rodziny Rhamphorhynchidae; dobrze zachowane skamieniałości tego pterozaura, znalezione w osadach w Solnhofen w Niemczech pokazują nawet mikrostrukturę jego skrzydeł}
\Clue{16}{}{kraina historyczna we wsch. Francji, między Wogezami a Renem, główne miasta Miluza i Strasburg}
\Clue{18}{}{poeta hiszpański (1503-36), wybitny humanista - sonety, kancony, elegie}
\Clue{21}{}{(wiertniczy) element wiertnicy umieszczony nad otworem wiertniczym, przeznaczony do obracania przewodu wiertniczego}
\Clue{25}{}{kształt, linia, graficznie wyodrębniony fragment powierzchni}
\Clue{30}{}{siedlisko rozumu, myśli}
\Clue{31}{}{amerykański staffordshire terier, rodzinny pies z tyou bullowatych}
\Clue{34}{}{odrobina, kapka, ociupina, mała ilość}
\Clue{36}{}{fason damskich płaszczy i kostiumów sportowych szytych na wzór munduru lub stroju myśliwego używanego w Afryce}
\Clue{37}{}{złe samopoczucie, przygnębienie, smutek, trudne położenie}
\Clue{38}{}{były partner - osoba, która była kiedyś w związku, lecz teraz już nie jest}
\Clue{39}{}{rodzaj ochronnego opakowania na rysunki, mapy, szkice, plakaty wykonywanego najszęściej z tektury lub plastiku}
\Clue{40}{}{w kościołach - osoba odpowiedzialna za oprawę muzyczną mszy bądź nabożeństwa (przewodnik chóru, organista lub solista)}
\Clue{41}{}{skrót pochodzący od nazwy Kościół rzymskokatolicki}
\Clue{42}{}{stan w środkowej części USA, pow. 180,5 tyś. km2, stolica Jefferson City}
\Clue{43}{}{małpa wąskonosa zaliczana do gibbonów}\end{PuzzleClues}

\begin{PuzzleClues}{\textbf{Pionowe}\\}\Clue{1}{}{koda}
\Clue{2}{}{kołnierz, który jest długi i zwęża się ku obu końcom}
\Clue{3}{}{dział sztuk plastycznych obejmujący dzieła wykonane techniką powielania odbitek na dowolnym podłożu lub dzieło wykonane tą technika}
\Clue{4}{}{używana dawniej przez piłkarzy pończocha bez stopy}
\Clue{5}{}{bycie w stanie podlegania sądowi, bycie oskarżonym}
\Clue{7}{}{kompozytor, śpiewak i aktor teatrów warszawskich (1789-1852); komedioopery}
\Clue{8}{}{BIMARCHAN, MARISTAN, w architekturze muzułmańskiej bogato zdobiony szpital}
\Clue{9}{}{miasto w płn. Etiopii na wysokości 2300 m, miasto koronacyjne królów etiopskich}
\Clue{10}{}{sklepienie żebrowe występujące w okresie wczesnego gotyku na terenie Dolnego Śląska, Małopolski}
\Clue{11}{}{mechanizm służący do naprowadzania pionowego działa}
\Clue{13}{}{to, co się przesuwa; część maszyny, sprzętu}
\Clue{17}{}{podobieństwo do wosku pod względem struktury, koloru, wyglądu}
\Clue{19}{}{poufale, żartobliwie lub lekceważąco o Niemcu}
\Clue{20}{}{hit, przebój, wypas, coś świetnego, kozackiego}
\Clue{21}{}{średniej wielkości współczesny okręt o uzbrojeniu, w skład którego wchodzą wyrzutnie rakietowe}
\Clue{22}{}{z Aten, rzeźbiarz grecki z VI w p. n. e, twórca pierwszego pomnika historycznego, posagu 'Tyranobójców'}
\Clue{23}{}{skupisko tkanki chłonnej w gardle, element pierścienia Waldeyera}
\Clue{24}{}{budynek wykonany za pomocą specjalnej technologii: najpierw podnosi się na linach gotowe stropy i do nich dostawia się ściany zewnętrzne oraz wewnętrzne}
\Clue{26}{}{stan, gdy czegoś jest pod dostatkiem, gdy można korzystać z pełnego zasobu czegoś; pełny zakres zazwyczaj jakiejś cechy lub stanu}
\Clue{27}{}{relacja między powiązanymi elementami zdania, w której jeden musi mieć pewną z góry ustaloną formę, niezależnie od formy drugiego}
\Clue{28}{}{właściwie Griniewski (1880-1932), pisarz rosyjski polskiego pochodzenia; „Szkarłatne żagle”}
\Clue{29}{}{dramaturg}
\Clue{31}{}{Passeriformes - rząd ptaków z podgromady Neornithes}
\Clue{32}{}{w mitologii greckiej: napój bogów}
\Clue{33}{}{struktura, całość z powiązanych wzajemnie elementów}
\Clue{34}{}{model okrętu naturalnej wielkości}
\Clue{35}{}{rzadki minerał z gromady minerałów uranylu, występuje tylko w niektórych rejonach Ziemi}\end{PuzzleClues}\newpage\section*{Krzyżówka 178}

\noindent\begin{Puzzle}{21}{26}|*	|*	|*	|*	|*	|*	|*	|*	|*	|*	|[1][S]\drarr	|o	|g	|r	|z	|e	|w	|n	|i	|k	|*	|*	|.
|*	|*	|*	|*	|*	|*	|*	|[2][S]\rarr	|l	|e	|s	|k	|o	|w	|*	|[3][S]\drarr	|a	|m	|b	|e	|r	|*	|.
|*	|[4][S]\drarr	|p	|r	|z	|e	|w	|ó	|d	|[][,]{ }	|o	|d	|g	|r	|o	|m	|o	|w	|y	|*	|*	|[5][S]\darr	|.
|*	|c	|*	|*	|*	|[6][S]\rarr	|a	|n	|t	|e	|n	|a	|*	|[7][S]\rarr	|d	|e	|j	|w	|u	|d	|*	|a	|.
|[8][S]\drarr	|h	|e	|r	|m	|a	|*	|*	|*	|*	|g	|[9][S]\drarr	|s	|z	|a	|l	|o	|t	|k	|a	|*	|b	|.
|p	|o	|*	|[10][S]\drarr	|k	|l	|i	|w	|e	|r	|*	|k	|*	|*	|*	|o	|[11][S]\drarr	|t	|a	|r	|*	|u	|.
|l	|r	|[12][S]\rarr	|p	|ó	|ł	|s	|k	|ó	|r	|e	|k	|*	|[13][S]\rarr	|s	|d	|g	|*	|*	|*	|*	|j	|.
|z	|o	|*	|i	|*	|*	|*	|*	|*	|*	|*	|*	|*	|*	|*	|i	|ó	|*	|*	|*	|*	|a	|.
|*	|b	|*	|ę	|*	|[14][S]\darr	|*	|*	|*	|*	|*	|*	|*	|[15][S]\rarr	|s	|a	|r	|d	|a	|n	|a	|*	|.
|[16][S]\drarr	|a	|l	|t	|*	|w	|[17][S]\drarr	|b	|a	|s	|k	|i	|n	|k	|a	|*	|n	|*	|*	|*	|*	|*	|.
|ś	|[][,]{ }	|*	|r	|*	|y	|o	|[18][S]\rarr	|s	|u	|p	|e	|r	|s	|a	|m	|o	|c	|h	|ó	|d	|*	|.
|c	|u	|[19][S]\darr	|o	|*	|l	|r	|*	|[20][S]\rarr	|h	|a	|n	|d	|i	|c	|a	|p	|*	|*	|*	|[21][S]\darr	|*	|.
|i	|h	|a	|[][,]{ }	|*	|e	|l	|*	|*	|*	|*	|*	|[22][S]\rarr	|s	|z	|a	|ł	|*	|[23][S]\darr	|*	|m	|*	|.
|e	|l	|b	|h	|[24][S]\drarr	|w	|y	|p	|i	|e	|r	|d	|e	|k	|[][,]{ }	|m	|a	|m	|u	|t	|a	|*	|.
|g	|a	|s	|a	|h	|*	|*	|[25][S]\drarr	|p	|i	|e	|r	|w	|i	|a	|s	|t	|e	|k	|*	|t	|*	|.
|[][,]{ }	|*	|o	|l	|a	|[26][S]\rarr	|a	|s	|p	|i	|r	|a	|n	|t	|k	|a	|*	|[27][S]\darr	|o	|*	|u	|*	|.
|g	|*	|r	|n	|y	|*	|*	|t	|*	|[28][S]\darr	|*	|*	|*	|*	|*	|*	|*	|w	|p	|*	|r	|*	|.
|a	|[29][S]\drarr	|b	|e	|d	|ł	|k	|a	|[][,]{ }	|f	|i	|o	|l	|e	|t	|o	|w	|a	|*	|*	|k	|*	|.
|ł	|d	|e	|*	|n	|*	|*	|r	|*	|u	|*	|*	|*	|[30][S]\rarr	|d	|r	|u	|ż	|y	|n	|a	|*	|.
|ą	|e	|n	|*	|*	|*	|*	|y	|[31][S]\rarr	|l	|u	|b	|l	|i	|n	|i	|a	|n	|i	|n	|*	|*	|.
|z	|s	|t	|[32][S]\rarr	|ć	|m	|a	|[][,]{ }	|b	|a	|r	|o	|w	|a	|*	|[33][S]\drarr	|t	|o	|p	|ó	|r	|*	|.
|k	|e	|*	|*	|[34][S]\rarr	|l	|e	|w	|*	|r	|*	|*	|[35][S]\rarr	|m	|i	|ł	|o	|ś	|ć	|*	|*	|*	|.
|o	|r	|*	|*	|*	|[36][S]\rarr	|o	|y	|o	|*	|*	|*	|*	|*	|*	|a	|*	|ć	|*	|*	|*	|*	|.
|w	|e	|*	|*	|*	|*	|[37][S]\rarr	|g	|o	|d	|z	|i	|n	|a	|*	|j	|*	|*	|*	|*	|*	|*	|.
|y	|k	|[38][S]\rarr	|p	|r	|e	|r	|a	|f	|a	|e	|l	|i	|t	|y	|z	|m	|*	|*	|*	|*	|*	|.
|*	|*	|*	|[39][S]\rarr	|b	|ó	|l	|*	|*	|[40][S]\rarr	|g	|r	|z	|y	|w	|a	|*	|*	|*	|*	|*	|*	|.
|*	|*	|*	|*	|*	|*	|*	|*	|*	|*	|*	|*	|*	|*	|*	|*	|*	|*	|*	|*	|*	|*	|.\end{Puzzle}

\newpage

\begin{PuzzleClues}{\textbf{Poziome}\\}\Clue{1}{}{specjalista zajmujący się ogrzewnictwem}
\Clue{2}{}{(1831-95), pisarz rosyjski; „Powiatowa lady Macbeth”}
\Clue{3}{}{bursztyn; kopalna żywica drzew iglastych}
\Clue{4}{}{uziemiony przewód, którego zadaniem jest ochrona od bezpośrednich wyładowań atmosferycznych oraz przepięć łączeniowych}
\Clue{6}{}{urządzenie zamieniające fale elektromagnetyczne na sygnał elektryczny i odwrotnie; jest niezbędnym elementem składowym każdego systemu radiokomunikacji}
\Clue{7}{}{wzmocnienie steru ze stępką}
\Clue{8}{}{relikwiarz metalowy w kształcie popiersia charakterystyczny dla sztuki średniowiecznej}
\Clue{9}{}{Allium ascalonicum - gatunek rośliny uprawnej należący do rodziny czosnkowatych}
\Clue{10}{}{trójkątny żagiel podnoszony na sztagu łączącym przedni maszt z bukszprytem}
\Clue{11}{}{TING; łowny ssak z rodziny krętorogich, pokrojem zbliżony do kozy}
\Clue{12}{}{oprawa książki, w której grzbiet i rogi okładki pokrywa skóra, a jej resztę papier lub płótno}
\Clue{13}{}{kod ISO 4217 funta sudańskiego}
\Clue{15}{}{narodowy taniec kataloński, symbol solidarności i jedności Katalończyków}
\Clue{16}{}{partia utworu pod względem wysokości pomiędzy tenorem a sopranem, śpiewana przez osobę o altowym głosie}
\Clue{17}{}{falbanka w sukni lub bluzce, zaczynająca się w pasie i kończąca w okolicy uda, rozkloszowana; wygląda jak krótka spódniczka nałożona na bluzkę lub sukienkę}
\Clue{18}{}{typ samochodu sportowego, którego opływowo - aerodynamiczna konstrukcja oraz silnik o mocy ponad 500 KM, pozwalają na uzyskanie prędkości przekraczających 300 km/h}
\Clue{20}{}{niepełnosprawność, przeszkoda; długotrwały stan występowania pewnych ograniczeń w prawidłowym funkcjonowaniu człowieka}
\Clue{22}{}{szaleństwo, amok - stan psychiczny będący skutkiem silnych: namiętności, gniewu, radości, stan wielkiego podniecenia}
\Clue{24}{}{ktoś mały, słaby, nisko oceniany przez silniejsze otoczenie}
\Clue{25}{}{liczba, która gdy zostanie podniesiona do potęgi, da określoną liczbę podpierwiastkową}
\Clue{26}{}{kandydat; osoba pretendująca do czegoś}
\Clue{29}{}{Laccaria amethystina - gatunek grzybów z rodziny piestróweczkowatych; grzyb jadalny, jednak niezbyt smaczny, poza tym o małych rozmiarach, z tego też powodu bez praktycznego znaczenia spożywczego}
\Clue{30}{}{podstawowa jednostka w strukturze organizacji harcerskich, skupiająca harcerzy i harcerki}
\Clue{31}{}{mieszkaniec Lublina}
\Clue{32}{}{człowiek wędrujący od baru do baru; określenie znane z prozy Charlesa Bukowskiego}
\Clue{33}{}{broń obuchowa, rodzaj ciężkiej siekiery, służył do rzucania i walki wręcz}
\Clue{34}{}{osoba spod znaku Lwa}
\Clue{35}{}{uczucie żywione do drugiej osoby, które zwykle wiąże się z pożądaniem}
\Clue{36}{}{miasto w płn.-zach Nigerii i ośrodek handlowy}
\Clue{37}{}{jakiś okres, termin, wyznaczona pora}
\Clue{38}{}{nurt w sztuce II połowy XIX wieku nawiązujący do twórczości malarzy wczesnego renesansu włoskiego}
\Clue{39}{}{silne nieprzyjemne odczucie fizyczne, które jest następstwem podrażnienia nerwów czuciowych}
\Clue{40}{}{grzbiet fali}\end{PuzzleClues}

\begin{PuzzleClues}{\textbf{Pionowe}\\}\Clue{1}{}{utwór wykonywany podczas występu estradowego, zawiera elementy satyry społecznej lub politycznej, dotyczącej aktualnej sytuacji państwa}
\Clue{3}{}{następstwo dźwięków uporządkowanych wg zasad tonalnych, rytmicznych i formalnych}
\Clue{4}{}{rzadka wada wrodzona serca charakteryzująca się częściowym lub całkowitym brakiem mięśnia sercowego prawej komory}
\Clue{5}{}{miasto w środkowej Nigerii w Stołecznym Terytorium Federalnym}
\Clue{8}{}{kod ISO 4217 dla waluty polskiej - złotego - sprzed denominacji z 1 stycznia 1995}
\Clue{9}{}{akt normatywny stanowiący zbiór przepisów regulujących odpowiedzialność karną obywateli danego państwa, zawierający definicję przestępstwa, zasady odpowiedzialności za przestępstwo, zasady przedawnienia odpowiedzialności karnej oraz spis kar i reguły ich stosowania}
\Clue{10}{}{obszar występowania łąk wysokogórskich zwanych halami}
\Clue{11}{}{jednopłat o płacie nośnym umieszczonym nad kadłubem}
\Clue{14}{}{krwiak, siniak - lokalne nagromadzenie się krwi, opuszczającej uszkodzone naczynia krwionośne, w obrębie różnych struktur tkankowych}
\Clue{16}{}{odmiana ściegu dzierganego w kształcie stylizowanej gałązki mająca zastosowanie w haftach ozdobnych}
\Clue{17}{}{znane lotnisko paryskie}
\Clue{19}{}{substancja, będąca sorbentem w reakcji absorpcji, czyli reakcji sorpcji wgłębnej}
\Clue{21}{}{arkusz maturalny rozwiązywany podczas pisemnego egzaminu maturalnego z danego przedmiotu lub praca maturalna}
\Clue{23}{}{pomocniczy wykop w celu uzyskania ziemi na budowlę ziemną}
\Clue{24}{}{kompozytor austriacki (1732-1809); najstarszy z klasyków wiedeńskich; ustalił skład klasycznej orkiestry symfonicznej; symfonie, kwartety, koncerty, sonaty. opery, kantaty}
\Clue{25}{}{osoba doświadczona, której nie da się oszukać}
\Clue{27}{}{to, w jakim stopniu coś jest ważne}
\Clue{28}{}{cienka; miękka tkanina jedwabna używana na suknie damskie, bieliznę, szale}
\Clue{29}{}{coś, co stanowi pozytywnie nacechowane zwieńczenie jakiegoś działania}
\Clue{33}{}{człowiek z marginesu społecznego, menel}\end{PuzzleClues}\newpage\section*{Krzyżówka 179}

\noindent\begin{Puzzle}{19}{32}|*	|*	|*	|*	|*	|*	|*	|*	|*	|*	|*	|*	|*	|*	|[1][S]\drarr	|h	|a	|h	|n	|*	|.
|*	|*	|*	|*	|*	|*	|*	|*	|*	|*	|*	|*	|[2][S]\darr	|*	|m	|*	|*	|[3][S]\darr	|*	|*	|.
|*	|*	|*	|*	|[4][S]\rarr	|ł	|u	|k	|[][,]{ }	|g	|r	|e	|c	|k	|i	|*	|*	|a	|[5][S]\darr	|*	|.
|*	|*	|*	|*	|[6][S]\darr	|*	|*	|*	|[7][S]\rarr	|l	|a	|k	|h	|n	|a	|u	|*	|n	|e	|*	|.
|*	|[8][S]\darr	|[9][S]\drarr	|o	|p	|a	|ł	|*	|*	|*	|*	|*	|a	|[10][S]\darr	|ł	|*	|[11][S]\darr	|a	|k	|*	|.
|[12][S]\drarr	|p	|o	|ł	|a	|b	|s	|z	|c	|z	|y	|z	|n	|a	|*	|*	|b	|l	|s	|*	|.
|t	|o	|b	|[13][S]\rarr	|n	|a	|c	|i	|s	|k	|*	|*	|g	|l	|*	|*	|i	|i	|t	|*	|.
|r	|d	|r	|[14][S]\rarr	|g	|a	|b	|i	|n	|e	|t	|*	|c	|p	|*	|*	|l	|z	|e	|*	|.
|ó	|k	|a	|*	|o	|*	|*	|*	|[15][S]\rarr	|b	|l	|a	|h	|a	|*	|[16][S]\darr	|b	|a	|r	|*	|.
|j	|o	|z	|*	|l	|*	|[17][S]\darr	|*	|*	|[18][S]\darr	|*	|*	|e	|k	|[19][S]\darr	|b	|e	|[][,]{ }	|n	|*	|.
|e	|m	|o	|*	|i	|*	|ż	|*	|*	|s	|*	|[20][S]\drarr	|n	|a	|t	|o	|r	|p	|*	|*	|.
|c	|o	|b	|*	|n	|*	|y	|*	|*	|ł	|*	|n	|g	|*	|r	|t	|g	|o	|*	|*	|.
|z	|r	|u	|*	|[][,]{ }	|*	|d	|*	|[21][S]\darr	|o	|*	|i	|o	|*	|ą	|t	|i	|r	|*	|*	|.
|k	|z	|r	|*	|o	|*	|o	|*	|u	|w	|*	|p	|p	|*	|b	|o	|a	|t	|*	|*	|.
|a	|y	|c	|[22][S]\darr	|l	|*	|s	|*	|m	|n	|*	|k	|t	|*	|k	|m	|[][,]{ }	|f	|*	|[23][S]\darr	|.
|*	|*	|a	|e	|b	|*	|t	|*	|b	|i	|[24][S]\drarr	|o	|e	|*	|a	|l	|z	|e	|*	|k	|.
|*	|*	|*	|k	|r	|*	|w	|*	|r	|k	|k	|w	|r	|*	|*	|e	|w	|l	|*	|a	|.
|*	|*	|[25][S]\drarr	|s	|z	|k	|o	|ł	|a	|*	|o	|*	|u	|*	|*	|s	|i	|o	|*	|l	|.
|*	|*	|p	|t	|y	|*	|*	|*	|*	|*	|z	|*	|s	|[26][S]\darr	|*	|s	|s	|w	|[27][S]\darr	|o	|.
|*	|*	|u	|r	|m	|[28][S]\rarr	|t	|ę	|ż	|n	|i	|k	|*	|p	|*	|*	|ł	|a	|p	|r	|.
|*	|*	|d	|a	|i	|[29][S]\rarr	|s	|o	|s	|n	|a	|[][,]{ }	|l	|i	|m	|b	|a	|*	|t	|m	|.
|*	|*	|e	|l	|*	|*	|*	|*	|*	|[30][S]\darr	|[][,]{ }	|[31][S]\darr	|*	|ę	|[32][S]\darr	|*	|*	|*	|a	|e	|.
|*	|[33][S]\drarr	|r	|i	|e	|p	|i	|n	|*	|b	|b	|a	|[34][S]\darr	|t	|w	|[35][S]\darr	|*	|[36][S]\darr	|s	|n	|.
|*	|n	|*	|g	|*	|[37][S]\drarr	|i	|n	|t	|e	|r	|n	|u	|n	|c	|j	|u	|s	|z	|*	|.
|*	|a	|*	|a	|*	|s	|*	|*	|*	|t	|ó	|g	|p	|o	|i	|a	|[38][S]\darr	|t	|n	|*	|.
|*	|s	|*	|*	|*	|*	|*	|*	|*	|o	|d	|o	|u	|*	|e	|e	|b	|r	|i	|*	|.
|[39][S]\rarr	|t	|ę	|c	|z	|a	|*	|*	|*	|n	|k	|l	|s	|*	|r	|n	|o	|z	|k	|*	|.
|*	|ę	|[40][S]\rarr	|i	|z	|o	|n	|o	|m	|i	|a	|*	|t	|*	|k	|*	|ż	|e	|o	|*	|.
|[41][S]\drarr	|p	|a	|t	|e	|l	|n	|i	|c	|a	|*	|*	|*	|*	|a	|*	|e	|l	|w	|*	|.
|b	|*	|[42][S]\rarr	|w	|a	|c	|h	|l	|a	|r	|z	|*	|*	|*	|*	|*	|k	|e	|a	|*	|.
|*	|*	|*	|*	|*	|*	|[43][S]\rarr	|t	|e	|k	|s	|t	|y	|l	|i	|a	|*	|c	|t	|*	|.
|[44][S]\rarr	|f	|o	|s	|f	|a	|t	|a	|z	|a	|[][,]{ }	|k	|w	|a	|ś	|n	|a	|*	|e	|*	|.
|*	|*	|*	|*	|*	|[45][S]\rarr	|t	|y	|p	|*	|*	|[46][S]\rarr	|b	|a	|r	|y	|ł	|a	|*	|*	|.\end{Puzzle}

\newpage

\begin{PuzzleClues}{\textbf{Poziome}\\}\Clue{1}{}{chemik, profesor Uniwersytetu Łódzkiego (1913-85); prace dotyczące syntezy nowych leków i barwników}
\Clue{4}{}{rodzaj broni stosowanej w starożytnej Grecji}
\Clue{7}{}{miasto w Indiach, stolica stanu Uttar Prades nad rzeką Gomati, ważny węzeł komunikacyjny}
\Clue{9}{}{materiał służący do opalania}
\Clue{12}{}{kraina historyczna zamieszkana niegdyś przez słowian połabskich; obejmująca ziemie między Morzem Bałtyckim, Łabą, Hawelą i Odrą}
\Clue{13}{}{wpływ (słowny lub poprzez działanie, również wpływ zewnętrznego czynnika, np. czasu) wywierany na osobę lub grupę w celu przyspieszenia jakichś działań, podjęcia decyzji}
\Clue{14}{}{specjalne wydzielone i odosobnione pomieszczenie w restauracji}
\Clue{15}{}{astronauta amerykański, odbył lot na pokładzie Discovery w 1989r}
\Clue{20}{}{filozof niemiecki (1854-1924); jeden z głównych przedstawicieli tzw. marburskiej szkoły neokantyzmu}
\Clue{24}{}{jednostka natężenia pola magnetycznego w układzie CGS; 1 Oe = 79,577 A/m}
\Clue{25}{}{instytucja zajmująca się kształceniem dzieci i młodzieży}
\Clue{28}{}{element usztywniający konstrukcję nośną}
\Clue{29}{}{bardzo miękkie drewno pozyskiwane z wielu gatunków drzewa o tej samej nazwie}
\Clue{33}{}{najwybitniejszy malarz rosyjski 91844-1930) przedstawiciel realizmu; 'Burłacy na Wołdze', 'Aresztowanie agitatora'}
\Clue{37}{}{dyplomata papieski w małym państwie lub mieście, które nie jest stolicą}
\Clue{39}{}{zjawisko optyczne i meteorologiczne, występujące w postaci charakterystycznego wielobarwnego łuku, widocznego, gdy Słońce oświetla krople wody w atmosferze ziemskiej}
\Clue{40}{}{równość wobec prawa, co oznacza, że ogólnie sformułowane prawo dotyczy wszystkich obywateli}
\Clue{41}{}{Taeniura grabata - gatunek ryb chrzęstnoszkieletowych z rodziny ogończowatych (Dasyatidae); patelnica zamieszkuje wschodni Ocean Atlantycki i południową część Morza Śródziemnego}
\Clue{42}{}{w gwarze łowieckiej; ogon głuszca}
\Clue{43}{}{materiały, wyroby włókiennicze}
\Clue{44}{}{enzym z klasy hydrolaz o aktywności fosfatazy, zajmujący się katalizowaniem defosforylacji różnych estrów fosforanowych}
\Clue{45}{}{w biologii: jedna z podstawowych kategorii systematycznych stosowanych w systematyce organizmów, niższa od królestwa, a wyższa od gromady (classis) w systematyce zwierząt lub klasy (classis) w botanice}
\Clue{46}{}{rodzaj dużej beczki}\end{PuzzleClues}

\begin{PuzzleClues}{\textbf{Pionowe}\\}\Clue{1}{}{substancja, która jest drobno roztarta, bardzo pokruszona}
\Clue{2}{}{Changchengopterus - rodzaj pterozaura z rodziny Wukongopteridae; żył w jurze na terenach współczesnych Chin}
\Clue{3}{}{analiza, której celem jest dobór optymalnych akcji do portfela inwestora giełdowego, co ma zapewnić możliwie najwyższy zysk przy najmniejszym możliwym ryzyku}
\Clue{5}{}{uczeń, który nie mieszka w internacie}
\Clue{6}{}{Manis gigantea, Smutsia gigantea - ssak łożyskowy zaliczany do łuskowców; zamieszkuje tropikalne lasy i sawanny od zachodniej Afryki do Ugandy}
\Clue{8}{}{urzędnik dworski w dawnej Polsce odpowiedzialny za siedzibę monarchy}
\Clue{9}{}{IKONOKLASTA, IKONOBURCA}
\Clue{10}{}{Vicugna pacos - południowoamerykański, trawożerny ssak parzystokopytny z rodziny wielbłądowatych, podobny do lamy i do owcy}
\Clue{11}{}{Billbergia nutans - gatunek roślin należący do rodziny bromeliowatych}
\Clue{12}{}{zdrobniale o trójce - czymś oznaczanym numerem 3, trójką}
\Clue{16}{}{strój bez dolnej części bielizny lub całkowity brak stroju}
\Clue{17}{}{Żydzi, ludność żydowska}
\Clue{18}{}{zbiór słów lub wyrażeń ułożonych i opracowanych według określonej zasady}
\Clue{19}{}{niewielkie urządzenie wzmacniające dźwięk}
\Clue{20}{}{inżynier niemiecki (1860-1940); jeden z pionierów telewizji}
\Clue{21}{}{wewnętrzny cień Ziemi obserwowany na powierzchni Księżyca}
\Clue{22}{}{najwyższego stopnia rozgrywki ligowe w jakiejś dyscyplinie sportu odbywające się w danym kraju}
\Clue{23}{}{fantastyczna kraina stworzona przez brytyjskiego pisarza C.S. Lewisa, w której rozgrywa się część akcji powieści Koń i jego chłopiec z cyklu Opowieści z Narnii}
\Clue{24}{}{zarost o niewielkiej długości porastający podbródek}
\Clue{25}{}{środek kosmetyczny służący do ochrony, wysuszania lub natłuszczania i upiększania skóry}
\Clue{26}{}{blizna, znamię, miejsce na skórze odróżniające się od reszty}
\Clue{27}{}{Theraphosidae syn. Aviculariidae - rodzina pająków z podrzędu Opisthothela, sekcji Orthognatha (syn. Mygalomorpha); obejmuje ok. 850 gatunków najczęściej dużych pająków o bardzo różnorodnym ubarwieniu}
\Clue{30}{}{samochód ciężarowy przeznaczony do przewożenia gotowego betonu}
\Clue{31}{}{miasto w środk. Chile w rejonie La Araucania}
\Clue{32}{}{preparat do wcierania w skórę lub we włosy, kosmetyk}
\Clue{33}{}{dźwignia nożna wprawiająca urządzenie w ruch}
\Clue{34}{}{obniżka ceny}
\Clue{35}{}{miasto w Hiszpanii (Andaluzja) u podnóża Gór Betyckich 102,67 tys. mieszk.(1986)}
\Clue{36}{}{gwiazdozbiór zodiakalny nieba południowego, również znak Zodiaku}
\Clue{37}{}{w chemii: symbol siarki}
\Clue{38}{}{czeski mechanik i wynalazca (1782-1835); skonstruował zegar astronomiczny, łódź i pojazd o napędzie parowym}
\Clue{41}{}{bajt - najmniejsza adresowalna jednostka informacji pamięci komputerowej, składająca się z bitów}\end{PuzzleClues}\newpage\section*{Krzyżówka 180}

\noindent\begin{Puzzle}{24}{33}|*	|*	|*	|*	|*	|*	|*	|*	|*	|*	|*	|*	|*	|*	|[1][S]\darr	|*	|*	|*	|*	|*	|*	|*	|*	|*	|*	|.
|*	|*	|*	|*	|*	|*	|*	|*	|*	|*	|*	|*	|*	|*	|e	|*	|*	|*	|*	|*	|*	|[2][S]\darr	|*	|*	|*	|.
|*	|*	|*	|*	|*	|*	|*	|*	|*	|*	|*	|*	|*	|*	|k	|*	|*	|*	|*	|*	|*	|c	|*	|*	|*	|.
|*	|*	|*	|*	|*	|*	|*	|*	|*	|*	|*	|*	|[3][S]\rarr	|p	|r	|o	|ś	|c	|i	|u	|c	|h	|*	|*	|*	|.
|*	|*	|*	|*	|*	|*	|*	|*	|*	|*	|*	|*	|*	|*	|a	|*	|*	|*	|*	|*	|*	|o	|*	|[4][S]\darr	|*	|.
|*	|*	|*	|*	|*	|*	|*	|*	|*	|*	|*	|*	|*	|*	|n	|*	|*	|*	|[5][S]\darr	|*	|*	|w	|*	|h	|*	|.
|*	|*	|*	|*	|*	|*	|*	|*	|*	|*	|*	|*	|*	|*	|*	|*	|*	|*	|c	|*	|*	|d	|*	|i	|*	|.
|*	|*	|*	|*	|*	|[6][S]\rarr	|s	|y	|n	|e	|c	|z	|e	|k	|[][,]{ }	|m	|a	|m	|u	|n	|i	|*	|*	|s	|*	|.
|*	|*	|*	|*	|*	|*	|*	|*	|*	|*	|[7][S]\darr	|*	|*	|*	|*	|*	|*	|*	|k	|*	|*	|*	|*	|p	|*	|.
|*	|*	|*	|*	|*	|*	|*	|*	|*	|*	|z	|*	|*	|*	|[8][S]\rarr	|l	|a	|f	|i	|r	|y	|n	|d	|a	|*	|.
|*	|[9][S]\rarr	|ż	|ó	|ł	|t	|o	|[][S]-	|b	|i	|a	|ł	|y	|[][,]{ }	|k	|a	|r	|z	|e	|ł	|*	|*	|*	|n	|*	|.
|*	|*	|*	|*	|*	|*	|*	|*	|*	|*	|r	|*	|*	|*	|*	|[10][S]\rarr	|s	|y	|r	|t	|a	|*	|*	|i	|*	|.
|*	|*	|*	|*	|*	|*	|*	|*	|*	|*	|a	|*	|*	|*	|*	|*	|*	|*	|[][,]{ }	|*	|*	|*	|*	|s	|*	|.
|*	|*	|*	|*	|*	|*	|*	|*	|*	|*	|s	|*	|*	|*	|*	|*	|*	|*	|z	|*	|*	|*	|*	|t	|*	|.
|*	|*	|*	|*	|*	|*	|*	|*	|*	|*	|a	|*	|*	|*	|*	|*	|*	|*	|ł	|*	|*	|*	|*	|y	|*	|.
|*	|*	|*	|*	|*	|*	|*	|*	|*	|*	|i	|*	|*	|*	|*	|*	|*	|*	|o	|*	|*	|*	|*	|k	|*	|.
|*	|*	|*	|*	|*	|*	|*	|*	|*	|*	|*	|*	|*	|*	|*	|*	|*	|*	|ż	|*	|*	|*	|*	|a	|*	|.
|[11][S]\rarr	|z	|a	|d	|r	|z	|e	|c	|h	|n	|i	|a	|[][,]{ }	|c	|z	|a	|r	|n	|o	|r	|o	|g	|a	|*	|*	|.
|*	|*	|*	|*	|*	|*	|*	|*	|*	|*	|*	|*	|*	|*	|*	|*	|*	|*	|n	|*	|*	|*	|*	|*	|*	|.
|*	|*	|*	|*	|*	|*	|*	|*	|*	|*	|*	|*	|*	|[12][S]\darr	|*	|*	|*	|*	|y	|*	|*	|*	|*	|*	|*	|.
|*	|*	|*	|*	|*	|*	|*	|*	|*	|*	|*	|*	|[13][S]\drarr	|s	|p	|l	|i	|t	|*	|*	|*	|*	|*	|*	|*	|.
|*	|*	|*	|[14][S]\rarr	|k	|u	|k	|l	|i	|k	|[][,]{ }	|r	|o	|z	|e	|s	|ł	|a	|n	|y	|*	|*	|*	|*	|*	|.
|*	|*	|*	|*	|*	|*	|*	|*	|*	|*	|*	|*	|k	|c	|*	|*	|*	|*	|*	|*	|*	|*	|*	|*	|*	|.
|*	|*	|*	|*	|*	|*	|*	|*	|*	|*	|*	|*	|a	|z	|*	|*	|*	|*	|*	|*	|*	|*	|*	|*	|*	|.
|*	|*	|*	|*	|*	|*	|*	|*	|*	|*	|*	|*	|z	|e	|*	|*	|*	|*	|*	|*	|*	|*	|*	|*	|*	|.
|*	|*	|*	|*	|*	|*	|*	|*	|*	|*	|*	|*	|j	|n	|*	|*	|*	|*	|*	|*	|*	|*	|*	|*	|*	|.
|*	|*	|*	|*	|*	|*	|*	|*	|*	|*	|*	|*	|o	|i	|*	|*	|*	|*	|*	|*	|*	|*	|*	|*	|*	|.
|*	|*	|*	|*	|*	|*	|*	|*	|*	|*	|*	|*	|n	|a	|*	|*	|*	|*	|*	|*	|*	|*	|*	|*	|*	|.
|*	|*	|*	|*	|*	|*	|*	|*	|*	|*	|*	|*	|a	|c	|*	|*	|*	|*	|*	|*	|*	|*	|*	|*	|*	|.
|*	|*	|*	|*	|*	|*	|*	|*	|*	|*	|*	|*	|l	|t	|*	|*	|*	|*	|*	|*	|*	|*	|*	|*	|*	|.
|*	|*	|*	|*	|*	|*	|*	|*	|*	|*	|*	|*	|i	|w	|*	|*	|*	|*	|*	|*	|*	|*	|*	|*	|*	|.
|*	|*	|*	|*	|*	|*	|*	|*	|*	|*	|*	|*	|z	|o	|*	|*	|*	|*	|*	|*	|*	|*	|*	|*	|*	|.
|*	|*	|*	|*	|*	|*	|*	|*	|*	|*	|*	|*	|m	|*	|*	|*	|*	|*	|*	|*	|*	|*	|*	|*	|*	|.
|*	|*	|*	|*	|*	|*	|*	|*	|*	|*	|*	|*	|*	|*	|*	|*	|*	|*	|*	|*	|*	|*	|*	|*	|*	|.\end{Puzzle}

\newpage

\begin{PuzzleClues}{\textbf{Poziome}\\}\Clue{3}{}{zgrubiale, z oburzeniem: człowiek prosty, nieokrzesany, bez manier}
\Clue{6}{}{mężczyzna niesamodzielny, silnie związany z matką}
\Clue{8}{}{prostytutka - kobieta uprawiająca seks za pieniądze}
\Clue{9}{}{gwiazda ciągu głównego o barwie żółto-białej, posiadającej typ widmowy F}
\Clue{10}{}{Mała, zatoka Morza Śródziemnego u wybrzeży Tunezji, głębokość 20-40 m}
\Clue{11}{}{Xylocopa valga - gatunek dużej pszczoły samotnicy powszechny w Europie zachodniej, centralnej i południowej}
\Clue{13}{}{miasto w Chorwacji nad Morzem Adriatyckim w Dalmacji, duży ośrodek turystyczny}
\Clue{14}{}{Geum reptans - gatunek rośliny należący do rodziny różowatych; występuje w Europie: w Karpatach i Alpach oraz w górach Półwyspu Bałkańskiego}\end{PuzzleClues}

\begin{PuzzleClues}{\textbf{Pionowe}\\}\Clue{1}{}{rodzaj zasłony (z tkaniny, skóry, żelaza), która bywa ustawiana przed kominkiem}
\Clue{2}{}{KOBDO}
\Clue{4}{}{nauka o kulturze i języku Hiszpanii i krajów hiszpańskojęzycznych}
\Clue{5}{}{polisacharyd, wielocukier - węglowodan, biopolimer składający się z merów będących cukrami prostymi połączonych wiązaniami glikozydowymi}
\Clue{7}{}{miasto we wsch. Litwie w pobliżu granicy z Łotwą}
\Clue{12}{}{zachowanie niedojrzałe, właściwe szczeniakowi - młodemu człowiekowi; oznaka braku wychowania}
\Clue{13}{}{wyrażenie języka, które uzyskuje różne znaczenia w zależności od kontekstu użycia}\end{PuzzleClues}\newpage\section*{Krzyżówka 181}

\noindent\begin{Puzzle}{26}{26}|*	|*	|*	|*	|*	|*	|*	|*	|*	|*	|*	|*	|*	|*	|*	|*	|*	|[1][S]\darr	|*	|*	|*	|*	|*	|*	|*	|*	|*	|.
|*	|*	|*	|*	|*	|*	|*	|*	|*	|*	|*	|*	|*	|*	|[2][S]\darr	|*	|*	|s	|*	|*	|*	|*	|[3][S]\darr	|*	|*	|*	|*	|.
|*	|*	|*	|*	|*	|*	|*	|*	|*	|*	|*	|*	|*	|*	|z	|*	|*	|z	|*	|*	|*	|*	|a	|*	|*	|*	|*	|.
|*	|*	|*	|*	|*	|*	|*	|*	|*	|*	|*	|*	|*	|*	|y	|*	|*	|a	|*	|*	|*	|*	|m	|*	|*	|*	|*	|.
|*	|*	|*	|*	|*	|*	|*	|*	|*	|*	|*	|*	|*	|*	|g	|*	|*	|r	|*	|*	|*	|*	|b	|[4][S]\darr	|*	|*	|*	|.
|*	|*	|*	|*	|*	|*	|*	|*	|*	|*	|*	|*	|*	|*	|m	|*	|*	|a	|*	|*	|*	|*	|y	|p	|*	|*	|*	|.
|*	|*	|*	|*	|*	|*	|*	|*	|*	|*	|*	|*	|*	|*	|u	|*	|*	|d	|*	|*	|*	|*	|s	|o	|*	|*	|*	|.
|*	|*	|*	|*	|*	|*	|*	|*	|*	|*	|*	|*	|*	|*	|n	|*	|*	|a	|*	|*	|*	|*	|t	|ł	|*	|*	|*	|.
|*	|*	|*	|*	|*	|*	|*	|*	|*	|*	|*	|[5][S]\rarr	|f	|o	|t	|o	|n	|*	|*	|*	|*	|*	|o	|ą	|*	|*	|*	|.
|*	|*	|*	|*	|*	|*	|*	|*	|*	|*	|*	|*	|*	|*	|[][,]{ }	|*	|*	|*	|*	|*	|*	|*	|m	|c	|*	|*	|*	|.
|*	|*	|*	|*	|*	|*	|*	|*	|*	|*	|*	|*	|*	|*	|i	|*	|*	|*	|*	|*	|*	|[6][S]\darr	|a	|z	|*	|*	|*	|.
|*	|*	|*	|*	|*	|*	|*	|*	|*	|*	|*	|*	|*	|*	|i	|*	|[7][S]\darr	|*	|*	|*	|*	|d	|[][,]{ }	|e	|*	|*	|*	|.
|*	|*	|*	|*	|*	|*	|*	|*	|*	|*	|*	|*	|*	|*	|i	|*	|w	|*	|*	|[8][S]\rarr	|ż	|u	|p	|n	|i	|k	|*	|.
|*	|*	|*	|*	|*	|*	|*	|*	|*	|*	|*	|[9][S]\darr	|*	|*	|[][,]{ }	|*	|a	|*	|*	|*	|*	|r	|l	|i	|*	|*	|*	|.
|*	|*	|*	|*	|*	|*	|*	|*	|*	|[10][S]\darr	|*	|k	|*	|*	|w	|*	|b	|*	|*	|*	|*	|g	|a	|e	|*	|*	|*	|.
|*	|*	|*	|*	|*	|*	|*	|*	|*	|ż	|*	|ą	|[11][S]\rarr	|h	|a	|m	|i	|l	|t	|o	|n	|*	|m	|[][,]{ }	|*	|*	|*	|.
|*	|*	|*	|*	|*	|*	|*	|*	|*	|y	|*	|p	|*	|*	|z	|*	|e	|*	|*	|*	|*	|*	|i	|n	|*	|*	|*	|.
|*	|*	|*	|*	|*	|*	|[12][S]\rarr	|m	|o	|d	|l	|i	|t	|w	|a	|*	|c	|*	|*	|*	|*	|*	|s	|e	|*	|*	|*	|.
|*	|*	|[13][S]\rarr	|s	|z	|y	|n	|e	|l	|*	|*	|e	|*	|*	|*	|*	|*	|*	|*	|*	|*	|*	|t	|r	|*	|*	|*	|.
|*	|*	|*	|*	|*	|*	|*	|*	|*	|*	|*	|l	|*	|*	|*	|*	|*	|*	|*	|*	|*	|*	|a	|w	|*	|*	|*	|.
|[14][S]\rarr	|c	|y	|r	|a	|n	|e	|c	|z	|k	|a	|[][,]{ }	|a	|n	|d	|a	|m	|a	|ń	|s	|k	|a	|*	|o	|*	|*	|*	|.
|*	|*	|*	|*	|*	|*	|*	|*	|*	|*	|*	|s	|*	|*	|*	|*	|*	|*	|*	|*	|*	|*	|*	|w	|*	|*	|*	|.
|*	|*	|[15][S]\rarr	|k	|r	|e	|d	|y	|t	|[][,]{ }	|g	|o	|t	|ó	|w	|k	|o	|w	|y	|*	|*	|*	|*	|e	|*	|*	|*	|.
|*	|*	|*	|*	|*	|*	|*	|*	|*	|*	|*	|l	|*	|*	|*	|*	|*	|*	|*	|*	|*	|*	|*	|*	|*	|*	|*	|.
|*	|*	|*	|*	|*	|*	|*	|*	|*	|*	|[16][S]\rarr	|n	|o	|t	|a	|c	|j	|a	|[][,]{ }	|d	|i	|r	|a	|c	|a	|*	|*	|.
|*	|*	|*	|*	|*	|*	|*	|*	|*	|*	|*	|a	|*	|*	|*	|*	|*	|*	|*	|*	|*	|*	|*	|*	|*	|*	|*	|.
|*	|*	|*	|*	|*	|*	|[17][S]\rarr	|s	|l	|o	|t	|*	|*	|*	|*	|*	|*	|*	|*	|*	|*	|*	|*	|*	|*	|*	|*	|.\end{Puzzle}

\newpage

\begin{PuzzleClues}{\textbf{Poziome}\\}\Clue{5}{}{kwant energii pola elektrycznego}
\Clue{8}{}{zarządca żupy królewskiej, dzierżawca}
\Clue{11}{}{amerykański łyżwiarz figurowy, mistrz olimpijski z Sarajewa, czterokrotny mistrz świata}
\Clue{12}{}{czynność związana z kultem, polegająca na skupieniu swoich myśli w kierunku istot uznanych za święte}
\Clue{13}{}{męski, wełniany płaszcz o ściśle określonym kroju, ozdobiony patkami, często metalowymi guzikami, noszony przez rosyjskich wojskowych i urzędników państwowych w XIX i na początku XX w}
\Clue{14}{}{Anas gibberifrons albogularis - podgatunek ptaka wyróżniony w obrębie gatunku cyraneczka szara (Anas gibberifrons); status taksonu sporny, bywa klasyfikowany jako odrębny gatunek (Anas albogularis)}
\Clue{15}{}{kredyt bankowy udzielany w gotówce na dowolny cel}
\Clue{16}{}{sposób zapisu wprowadzony w 1939 przez Paula Diraca do mechaniki kwantowej, służący do zapisywania stanów kwantowych}
\Clue{17}{}{skrzydełko na przedniej krawędzi skrzydła samolotu}\end{PuzzleClues}

\begin{PuzzleClues}{\textbf{Pionowe}\\}\Clue{1}{}{zagadka w formie wiersza}
\Clue{2}{}{król Polski i wielki książę litewski (1587-1632), król Szwecji (1592-1599, jako Sigismund), tytularny król Szwecji (1599-1632) z dynastii Wazów}
\Clue{3}{}{Ambystoma maculatum - gatunek płaza ogoniastago z rodziny ambystomowatych, występujący we wschodniej części Ameryki Północnej od Nowej Szkocji po Teksas}
\Clue{4}{}{połączenie między dwiema komórkami nerwowymi}
\Clue{6}{}{miasto w Indiach (Madhja Prades) ważny ośrodek hutnictwa żelaza, 490,2 tys. mieszkańców (1991)}
\Clue{7}{}{imitacja ptaków w łowiectwie}
\Clue{9}{}{używany w celach przemysłowych roztwór chlorku sodu o różnych stężeniach}
\Clue{10}{}{człowiek, który wyznaje judaizm}\end{PuzzleClues}\newpage\section*{Krzyżówka 182}

\noindent\begin{Puzzle}{19}{20}|*	|[1][S]\darr	|*	|*	|*	|[2][S]\drarr	|h	|e	|p	|a	|t	|o	|l	|o	|g	|i	|a	|*	|*	|*	|.
|*	|k	|[3][S]\rarr	|m	|o	|s	|t	|[][,]{ }	|ł	|u	|k	|o	|w	|y	|*	|*	|[4][S]\darr	|*	|*	|*	|.
|*	|o	|[5][S]\drarr	|s	|t	|a	|c	|j	|a	|[][,]{ }	|k	|l	|i	|e	|n	|c	|k	|a	|*	|*	|.
|*	|m	|p	|*	|[6][S]\rarr	|m	|y	|l	|a	|i	|*	|*	|[7][S]\darr	|[8][S]\drarr	|c	|h	|o	|d	|y	|*	|.
|*	|ó	|o	|[9][S]\rarr	|t	|o	|p	|o	|s	|*	|[10][S]\darr	|*	|b	|a	|[11][S]\darr	|*	|p	|*	|*	|*	|.
|*	|r	|i	|*	|*	|g	|*	|*	|*	|*	|m	|*	|o	|l	|m	|*	|i	|[12][S]\darr	|*	|*	|.
|*	|k	|n	|[13][S]\rarr	|a	|r	|k	|t	|y	|k	|a	|*	|c	|b	|i	|[14][S]\darr	|d	|z	|*	|*	|.
|*	|a	|t	|*	|[15][S]\darr	|a	|*	|[16][S]\darr	|*	|[17][S]\darr	|g	|[18][S]\darr	|z	|e	|k	|k	|ó	|w	|*	|*	|.
|*	|[][,]{ }	|[][,]{ }	|[19][S]\darr	|z	|j	|*	|b	|*	|b	|n	|p	|e	|r	|r	|o	|ł	|o	|*	|*	|.
|*	|r	|a	|k	|a	|*	|*	|e	|[20][S]\drarr	|r	|e	|a	|k	|t	|o	|r	|*	|l	|*	|*	|.
|*	|o	|n	|ł	|m	|[21][S]\darr	|*	|a	|s	|z	|z	|g	|*	|y	|m	|y	|[22][S]\darr	|n	|*	|*	|.
|[23][S]\drarr	|z	|d	|a	|r	|z	|e	|n	|i	|e	|*	|o	|*	|n	|e	|t	|g	|i	|*	|*	|.
|s	|r	|[][,]{ }	|m	|ó	|g	|[24][S]\darr	|*	|n	|z	|*	|n	|[25][S]\darr	|i	|r	|k	|i	|e	|*	|*	|.
|u	|o	|c	|c	|z	|o	|k	|[26][S]\darr	|u	|i	|*	|*	|k	|*	|*	|o	|e	|n	|*	|*	|.
|n	|d	|l	|z	|*	|d	|o	|r	|s	|n	|[27][S]\rarr	|l	|o	|t	|i	|*	|r	|i	|*	|*	|.
|n	|c	|i	|u	|*	|a	|t	|e	|*	|i	|*	|*	|z	|*	|*	|*	|y	|e	|*	|*	|.
|i	|z	|c	|c	|*	|*	|l	|c	|[28][S]\rarr	|a	|r	|i	|a	|n	|i	|z	|m	|*	|*	|*	|.
|t	|a	|k	|h	|[29][S]\rarr	|m	|i	|j	|a	|n	|k	|a	|*	|*	|*	|*	|s	|*	|*	|*	|.
|a	|*	|*	|*	|*	|*	|k	|a	|[30][S]\rarr	|k	|o	|n	|o	|t	|a	|t	|k	|a	|*	|*	|.
|*	|*	|*	|*	|*	|*	|*	|*	|[31][S]\rarr	|a	|t	|e	|n	|y	|*	|*	|i	|*	|*	|*	|.
|[32][S]\rarr	|s	|a	|l	|w	|i	|n	|i	|a	|*	|*	|*	|*	|*	|*	|*	|*	|*	|*	|*	|.\end{Puzzle}

\newpage

\begin{PuzzleClues}{\textbf{Poziome}\\}\Clue{2}{}{dziedzina medycyny zajmująca się schorzeniami, budową i funkcjonowaniem wątroby oraz pęcherzyka żółciowego i dróg żółciowych}
\Clue{3}{}{most, którego przęsła mają kształt łuków}
\Clue{5}{}{urządzenie elektroniczne korzystające z usług zapewnianych przez serwer}
\Clue{6}{}{(MY LAI) wieś w płd. Wietnamie; w 1968r wymordowanie ludności przez oddział amerykańskich żołnierzy}
\Clue{8}{}{dojście do czegoś przez znajomości, protekcja, fory, np. mieć chody}
\Clue{9}{}{powtarzający się motyw, który często występuje w obrębie literatury i sztuki danej kultury, cywilizacji}
\Clue{13}{}{obszar polarny wokół bieguna płn, obszar 21 min km2}
\Clue{20}{}{urządzenie do wyzwalania energii jądrowej w wyniku łańcuchowej reakcji rozszczepiania jąder atomowych substancji rozszczepialnej}
\Clue{23}{}{w informatyce: zaistniała w systemie operacyjnym okoliczność, która powoduje zmianę w zakresie jakiegoś procesu}
\Clue{27}{}{właściwie Viaud - pisarz francuski (1850-1923), powieści z życia marynarzy i rybaków; „Rybak islandzki”}
\Clue{28}{}{protestancki prąd religijny, wyodrębniony w XVI wieku z kalwinizmu}
\Clue{29}{}{stacja mająca rozgałęzienie torów przeznaczone jedynie do krzyżowania i wyprzedzania pociągów}
\Clue{30}{}{coś zapisanego, zazwyczaj w celu zapamiętania lub podkreślenia czegoś}
\Clue{31}{}{starożytne państwo}
\Clue{32}{}{Salvinia - rodzaj paproci wodnych, należący do rodziny salwiniowatych (Salviniaceae); należy do niego ok. 10 gatunków; gatunkiem typowym jest Salvinia natans (L.) Allioni}\end{PuzzleClues}

\begin{PuzzleClues}{\textbf{Pionowe}\\}\Clue{1}{}{komórka służąca do rozmnażania płciowego}
\Clue{2}{}{instrument odtwarzający dźwięki utrwalone na specjalnych nośnikach, np. na wałkach ze sztyftami, perforowanej taśmie lub płycie}
\Clue{4}{}{grabarz}
\Clue{5}{}{typ komputerowej gry przygodowej, w którym postacią steruje się za pomocą wskaźnika, zwykle kierowanego przy użyciu myszy}
\Clue{7}{}{w edytorstwie: krótkie, hasłowe określenia, pojedyncze wyrazy lub zwroty, czasem powtórzony tytuł rozdziału, które znajdują się na zewnętrznych marginesach i informują o treści poszczególnych fragmentów tekstu}
\Clue{8}{}{zakon polski o surowej regule założony w Krakowie w 1888 r.; przytułki dla bezdomnych, starców ubogich}
\Clue{10}{}{stop metali zawierający magnez, wykorzystywany w przemyśle}
\Clue{11}{}{mały blastomer powstający na skutek nierówności segmentacji jaja}
\Clue{12}{}{umorzenie jakiejś opłaty, np. zwolnienie z podatku}
\Clue{14}{}{pieszczotliwie o porcji jedzenia, sowitego posiłku (zwykle o talerzu lub misce)}
\Clue{15}{}{szadź, sadź - osad lodowy tworzący się w wyniku odwilży}
\Clue{16}{}{przenośnie: głupek, prostak, frajer (człowiek o cechach przypisywanych stereotypowo nowemu lub nawet nie nowemu uczniakowi)}
\Clue{17}{}{mieszkanka Brzezin}
\Clue{18}{}{patka na ramieniu ubrania wojskowego lub cywilnego}
\Clue{19}{}{osoba, która często kłamie, mówi nieprawdę}
\Clue{20}{}{jedna z funkcji trygonometrycznych - w jednostkowym kole o środku w punkcie (0,0) definiowana jako rzędna punktu (x,y) leżącego na okręgu i wyznaczającego koniec promienia koła, interesujący nas kąt zawarty jest pomiędzy tym właśnie promieniem a dodatnią półosią OX}
\Clue{21}{}{stan, w którym ludzie zgadzają się ze sobą, kiedy tak samo myślą, czują, uważają}
\Clue{22}{}{Maksymilian (1846-74) malarz; pejzaże sceny z powstań. sceny myśliwskie 'Patrol powstańczy', 'Wyjazd na polowanie'}
\Clue{23}{}{członek ortodoksyjnego odłamu muzułmanów przeciwstawnego szyitom i skupiającego większość wyznawców islamu}
\Clue{24}{}{NIEDŹWIEDŹ MORSKI; gatunek foki}
\Clue{25}{}{pomieszczenie w dawnej szkole, w którym zamykano uczniów za karę, zostawiano po lekcjach}
\Clue{26}{}{w starożytności prowincja rzymska na obecnych terenach wschodniej Szwajcarii, Tyrolu i płn. Bawarii}\end{PuzzleClues}\newpage\section*{Krzyżówka 183}

\noindent\begin{Puzzle}{21}{25}|*	|*	|*	|*	|*	|*	|[1][S]\drarr	|f	|i	|s	|z	|b	|i	|n	|o	|w	|c	|e	|*	|*	|*	|*	|.
|*	|*	|*	|[2][S]\drarr	|e	|n	|d	|o	|s	|z	|k	|i	|e	|l	|e	|t	|*	|*	|*	|*	|*	|*	|.
|*	|*	|[3][S]\darr	|r	|[4][S]\rarr	|o	|r	|n	|a	|t	|*	|[5][S]\drarr	|p	|a	|p	|r	|o	|t	|n	|i	|k	|*	|.
|*	|*	|f	|o	|*	|[6][S]\rarr	|u	|n	|d	|e	|a	|d	|*	|*	|*	|*	|*	|[7][S]\darr	|*	|*	|*	|[8][S]\darr	|.
|[9][S]\drarr	|ł	|o	|z	|ó	|w	|k	|a	|*	|*	|[10][S]\darr	|e	|*	|[11][S]\darr	|*	|*	|*	|b	|*	|[12][S]\darr	|*	|o	|.
|k	|*	|w	|m	|*	|*	|[][,]{ }	|[13][S]\darr	|*	|*	|d	|n	|*	|a	|*	|*	|*	|ł	|*	|b	|[14][S]\darr	|d	|.
|a	|[15][S]\drarr	|l	|i	|n	|i	|s	|k	|o	|*	|r	|i	|*	|t	|*	|*	|[16][S]\darr	|ę	|[17][S]\darr	|a	|k	|w	|.
|t	|g	|e	|a	|*	|*	|e	|o	|[18][S]\rarr	|l	|o	|e	|w	|e	|*	|[19][S]\darr	|ś	|k	|w	|r	|u	|i	|.
|a	|ł	|r	|r	|*	|*	|n	|s	|[20][S]\darr	|[21][S]\darr	|g	|r	|*	|n	|*	|s	|w	|i	|i	|t	|k	|e	|.
|r	|u	|*	|*	|[22][S]\rarr	|l	|a	|m	|p	|k	|a	|*	|[23][S]\darr	|k	|*	|z	|i	|t	|w	|ł	|u	|c	|.
|y	|c	|*	|*	|*	|*	|c	|o	|e	|o	|[][,]{ }	|[24][S]\darr	|a	|a	|*	|c	|n	|n	|i	|o	|r	|z	|.
|n	|h	|*	|*	|*	|*	|k	|f	|r	|ń	|k	|k	|n	|*	|*	|z	|i	|y	|s	|m	|y	|e	|.
|k	|ó	|*	|*	|*	|*	|i	|i	|m	|c	|o	|w	|o	|*	|*	|e	|a	|[][,]{ }	|e	|i	|d	|r	|.
|a	|w	|[25][S]\drarr	|ł	|u	|g	|*	|z	|*	|ó	|l	|a	|n	|*	|[26][S]\darr	|c	|[][,]{ }	|n	|k	|e	|z	|z	|.
|*	|*	|k	|[27][S]\drarr	|f	|r	|*	|y	|[28][S]\rarr	|w	|e	|r	|s	|y	|f	|i	|k	|a	|c	|j	|a	|*	|.
|[29][S]\drarr	|t	|r	|a	|s	|a	|*	|k	|*	|k	|j	|t	|*	|*	|r	|o	|a	|d	|j	|c	|*	|*	|.
|b	|[30][S]\drarr	|a	|l	|d	|o	|z	|a	|*	|a	|o	|a	|*	|[31][S]\darr	|y	|s	|r	|o	|a	|z	|*	|*	|.
|r	|p	|s	|t	|*	|*	|*	|*	|*	|*	|w	|l	|*	|k	|z	|z	|ł	|l	|*	|y	|*	|*	|.
|z	|e	|p	|d	|*	|*	|*	|[32][S]\darr	|[33][S]\rarr	|m	|a	|n	|g	|o	|*	|c	|o	|b	|*	|k	|*	|*	|.
|e	|g	|e	|o	|*	|[34][S]\rarr	|e	|m	|a	|n	|*	|i	|*	|t	|*	|z	|w	|r	|*	|*	|*	|*	|.
|z	|m	|d	|r	|*	|*	|*	|i	|[35][S]\rarr	|s	|o	|k	|*	|i	|*	|ę	|a	|z	|*	|*	|*	|*	|.
|i	|a	|o	|f	|*	|[36][S]\rarr	|o	|t	|o	|k	|*	|*	|*	|k	|*	|k	|t	|y	|*	|*	|*	|*	|.
|n	|t	|d	|e	|[37][S]\rarr	|b	|i	|o	|n	|i	|k	|a	|*	|*	|*	|i	|a	|m	|*	|*	|*	|*	|.
|k	|y	|o	|r	|[38][S]\rarr	|p	|r	|z	|e	|s	|i	|l	|a	|n	|i	|e	|*	|*	|*	|*	|*	|*	|.
|a	|t	|n	|*	|*	|[39][S]\rarr	|n	|a	|u	|s	|z	|n	|i	|k	|i	|*	|*	|*	|*	|*	|*	|*	|.
|*	|*	|*	|*	|*	|*	|*	|*	|*	|*	|*	|*	|*	|*	|*	|*	|*	|*	|*	|*	|*	|*	|.\end{Puzzle}

\newpage

\begin{PuzzleClues}{\textbf{Poziome}\\}\Clue{1}{}{Mysticeti - jeden z dwóch podrzędów waleni, charakteryzujący się obecnością fiszbinów, czyli rogowych płyt w szczękach, służących do odfiltrowania planktonu z wody; do fiszbinowców zalicza się największe ssaki morskie, np. płetwala błękitnego, który jest największym zwierzęciem, jakie żyło kiedykolwiek na Ziemi}
\Clue{2}{}{szkielet znajdujący się wewnątrz ciała, otoczony innymi tkankami}
\Clue{4}{}{wierzchnia szata używana przez księdza podczas mszy haftowana z drogich tkanin}
\Clue{5}{}{zarodnikowa roślina naczyniowa np. paproć, skrzyp, widłak itp}
\Clue{6}{}{kategoria istot żywych w różnych mitologiach (np. w kulcie voodoo), przeniesiona do kultury popularnej i rozwinięta w fantasy; stwór, który jest ciałem niegdyś zmarłej istoty, której proces śmierci został zakłócony lub która została przywrócona do życia zwykle za pomocą czarów, rytuałów lub magicznych mikstur}
\Clue{9}{}{chroniony ptak zaroślowy z rzędu wróblowatych, oliwkowo-szara, owadożerna; Eurazja, środkowa Afryka}
\Clue{15}{}{rodzaj stalowej liny dwuskrętnej mającej rdzenie z włókna roślinnego}
\Clue{18}{}{amerykański kompozytor i pianista (1904-1988); 'My Fair Lady'}
\Clue{22}{}{mała lampa w różnej formie: elektryczna, gazowa, oliwna itd}
\Clue{25}{}{miejsce, obszar, rejon, który jest podmokły lub położony w pobliżu zbiornika wodnego i który cechuje się właściwą dla dużej wilotności fauną i florą}
\Clue{27}{}{w chemii: symbol fransu}
\Clue{28}{}{dział poetyki zajmujący się badaniem budowy wiersza, czyli jego formą, metryką wersów a także kompozycją wiersza}
\Clue{29}{}{droga do przebycia, także: droga wytyczona dla zawodników sportowych}
\Clue{30}{}{cukier prosty, który zawiera grupę aldehydową}
\Clue{33}{}{drewno pozyskiwane z mangowca}
\Clue{34}{}{jednostka stężenia roztworów substancji promieniotwórczych}
\Clue{35}{}{produkt otrzymany ze zdrowych, dojrzałych, świeżych lub przechowywanych owoców lub warzyw}
\Clue{36}{}{w numizmatyce: obwódka monety}
\Clue{37}{}{interdyscyplinarna nauka badająca budowę i zasady działania organizmów oraz ich adaptowanie w technice (zwłaszcza w automatyce) i budowie urządzeń technicznych na wzór organizmu}
\Clue{38}{}{SOLSTYCJUM}
\Clue{39}{}{element garderoby}\end{PuzzleClues}

\begin{PuzzleClues}{\textbf{Pionowe}\\}\Clue{1}{}{dokument używany na posiedzeniach senatu, zawierający najczęściej teksty ustaw}
\Clue{2}{}{cecha czegoś, co istnieje fizycznie, wielkość, parametr informujący o tym, czy coś jest duże, czy małe w zakresie jednego wymiaru, kilku lub wszystkich}
\Clue{3}{}{John (1826-64); angielski wynalazca i przemysłowiec, zbudował pług do drenowania i pług z napędem parowym}
\Clue{5}{}{jednostka gęstości liniowej włókien syntetycznych; włókno o długości 9000 metrów i masie 1 grama ma gęstość 1 deniera, 1 den = 1g / 9000m}
\Clue{7}{}{nadolbrzym nalężący do I klasy jasności}
\Clue{8}{}{posiłek popołudniowy, wieczorny; wieczerza}
\Clue{9}{}{mechaniczny instrument muzyczny}
\Clue{10}{}{nawierzchnia kolejowa wraz z podtorzem i budowlami inżynieryjnymi oraz gruntem, na którym jest usytuowana}
\Clue{11}{}{mieszkanka starożytnego państwa ateńskiego}
\Clue{12}{}{wieloobrazowe malarstwo na ścianach, stropach, skupieniach budowli}
\Clue{13}{}{dział fizyki zajmujący się problemami związanymi z przestrzenią kosmiczną}
\Clue{14}{}{jedno ziarno kukurydzy (częsta synekdocha); rodzaj zboża}
\Clue{15}{}{wieś sołecka w Polsce położona w województwie świętokrzyskim, w powiecie kazimierskim, w gminie Kazimierza Wielka}
\Clue{16}{}{Porcula salvania, Sus salvanius - ssak z rodziny świniowatych zagrożony wyginięciem; zamieszkuje południowe Himalaje - Indie, Nepal, Bhutan}
\Clue{17}{}{wnikliwa, szczegółowa analiza czegoś}
\Clue{19}{}{szczecioszczęki, Chaetognatha - typ drobnych morskich, drapieżnych zwierząt bezkręgowych o strzałkowato wydłużonym, dwubocznie symetrycznym ciele z otworem gębowym otoczonym cedzącymi szczeciami i chitynowymi kolcami; szczecioszczękie kształtem ciała przypominają ryby}
\Clue{20}{}{kraina historyczna w północnej Rosji przeduralskiej, między Peczorą, Wyczegdą i Kamą a Uralem}
\Clue{21}{}{końcowa część czegoś, zakończenie}
\Clue{23}{}{krótkie ogłoszenie opisujące np. towar na sprzedaż lub do kupna}
\Clue{24}{}{srebrna moneta bita w średniowieczu (od końca XIII w.) na Śląsku, w Wielkopolsce, w Królestwie Polskim, w państwie krzyżackim i na Pomorzu Zachodnim}
\Clue{25}{}{Craspedodon - rodzaj dinozaura znany jedynie na podstawie trzech zębów datowanych na górną kredę, znalezionych w okolicach wsi Lonzée w Belgii}
\Clue{26}{}{deska z drewna drzewa liściastego używana do wyrobu boazerii i desek parkietowych}
\Clue{27}{}{rysownik i ilustrator (1836-93) ilustracje do 'Pana Tadeusza'}
\Clue{29}{}{Brzezinka - wieś}
\Clue{30}{}{rodzaj skały magmowej charakteryzującej się szczególną mega- lub gigantokrystaliczną teksturą, wzbogaconej w pierwiastki niekompatybilne oraz minerały zawierające składniki lotne}
\Clue{31}{}{morski ssak z rodziny uchatek; uchatka niedźwiedziowata}
\Clue{32}{}{proces podziału pośredniego jądra komórkowego, któremu towarzyszy precyzyjne rozdzielenie chromosomów do dwóch komórek potomnych}\end{PuzzleClues}\newpage\section*{Krzyżówka 184}

\noindent\begin{Puzzle}{23}{29}|*	|*	|*	|*	|*	|*	|*	|*	|*	|*	|*	|*	|*	|*	|*	|[1][S]\darr	|*	|*	|*	|*	|[2][S]\darr	|*	|*	|*	|.
|*	|*	|*	|*	|*	|*	|*	|[3][S]\drarr	|w	|i	|e	|l	|k	|a	|[][,]{ }	|b	|r	|y	|t	|a	|n	|i	|a	|*	|.
|*	|*	|*	|[4][S]\rarr	|p	|a	|t	|o	|l	|o	|g	|i	|a	|[][,]{ }	|n	|a	|r	|z	|ą	|d	|o	|w	|a	|*	|.
|*	|*	|[5][S]\rarr	|s	|ł	|u	|ż	|b	|a	|[][,]{ }	|b	|o	|ż	|a	|*	|r	|*	|*	|[6][S]\darr	|*	|ś	|*	|*	|*	|.
|*	|[7][S]\rarr	|j	|a	|s	|t	|a	|r	|n	|i	|a	|*	|[8][S]\rarr	|g	|r	|y	|f	|*	|i	|[9][S]\darr	|n	|*	|*	|*	|.
|[10][S]\rarr	|s	|i	|e	|r	|m	|i	|ę	|ż	|k	|a	|*	|*	|*	|*	|*	|*	|*	|m	|p	|i	|*	|*	|*	|.
|[11][S]\rarr	|ł	|u	|t	|[][,]{ }	|s	|z	|c	|z	|ę	|ś	|c	|i	|a	|*	|[12][S]\rarr	|a	|l	|m	|i	|k	|*	|[13][S]\darr	|*	|.
|[14][S]\rarr	|p	|a	|l	|e	|o	|a	|z	|j	|a	|c	|i	|*	|*	|*	|*	|*	|*	|e	|k	|[][,]{ }	|*	|t	|*	|.
|[15][S]\drarr	|b	|i	|l	|a	|n	|s	|[][,]{ }	|p	|ł	|a	|t	|n	|i	|c	|z	|y	|*	|l	|t	|f	|*	|r	|*	|.
|g	|*	|[16][S]\drarr	|c	|l	|*	|[17][S]\rarr	|b	|r	|e	|k	|i	|n	|i	|a	|*	|*	|*	|m	|*	|u	|*	|i	|*	|.
|m	|[18][S]\rarr	|p	|a	|r	|t	|i	|a	|[][,]{ }	|h	|i	|s	|z	|p	|a	|ń	|s	|k	|a	|*	|n	|*	|u	|*	|.
|d	|*	|r	|*	|[19][S]\rarr	|w	|k	|r	|ę	|t	|k	|a	|*	|*	|[20][S]\rarr	|m	|a	|p	|n	|i	|k	|*	|m	|*	|.
|*	|[21][S]\rarr	|z	|ł	|o	|c	|i	|k	|[][,]{ }	|s	|z	|a	|f	|i	|r	|o	|w	|y	|*	|[22][S]\darr	|c	|*	|f	|*	|.
|[23][S]\rarr	|n	|e	|u	|r	|o	|h	|o	|r	|m	|o	|n	|*	|*	|*	|*	|*	|[24][S]\darr	|*	|d	|j	|[25][S]\darr	|[][,]{ }	|*	|.
|*	|[26][S]\drarr	|p	|o	|s	|t	|a	|w	|a	|n	|g	|a	|r	|d	|a	|*	|*	|w	|*	|o	|i	|w	|p	|*	|.
|[27][S]\drarr	|k	|o	|m	|i	|s	|j	|a	|[][,]{ }	|r	|o	|z	|j	|e	|m	|c	|z	|a	|*	|p	|*	|i	|a	|*	|.
|w	|r	|c	|*	|*	|*	|[28][S]\darr	|*	|*	|*	|*	|*	|*	|*	|[29][S]\rarr	|p	|y	|ł	|*	|o	|*	|e	|c	|*	|.
|i	|a	|z	|*	|*	|*	|p	|*	|[30][S]\drarr	|m	|i	|c	|z	|m	|a	|n	|*	|ę	|*	|w	|*	|c	|k	|*	|.
|n	|j	|w	|[31][S]\rarr	|m	|a	|r	|a	|b	|u	|t	|[][,]{ }	|j	|a	|w	|a	|j	|s	|k	|i	|*	|h	|h	|*	|.
|c	|n	|a	|[32][S]\drarr	|o	|k	|o	|l	|e	|*	|*	|*	|*	|*	|*	|[33][S]\rarr	|w	|a	|ł	|e	|k	|*	|a	|*	|.
|e	|i	|r	|c	|*	|[34][S]\drarr	|p	|r	|a	|d	|z	|i	|e	|j	|e	|*	|*	|k	|[35][S]\darr	|d	|*	|[36][S]\darr	|m	|*	|.
|r	|k	|z	|y	|*	|g	|i	|*	|u	|*	|[37][S]\darr	|[38][S]\rarr	|s	|k	|i	|b	|a	|*	|g	|z	|*	|b	|a	|*	|.
|a	|*	|e	|f	|*	|ł	|n	|[39][S]\rarr	|m	|u	|f	|k	|a	|*	|*	|[40][S]\rarr	|t	|u	|l	|e	|j	|a	|*	|*	|.
|d	|*	|n	|e	|*	|o	|a	|[41][S]\rarr	|o	|b	|r	|o	|t	|ó	|w	|k	|a	|*	|i	|n	|[42][S]\darr	|l	|*	|*	|.
|a	|*	|i	|r	|*	|w	|c	|*	|n	|*	|y	|*	|*	|*	|*	|[43][S]\rarr	|g	|a	|s	|i	|w	|o	|*	|*	|.
|*	|*	|e	|k	|*	|n	|j	|*	|t	|[44][S]\rarr	|z	|i	|o	|m	|e	|k	|*	|*	|s	|e	|i	|n	|*	|*	|.
|*	|*	|*	|a	|*	|i	|a	|*	|*	|*	|*	|*	|*	|[45][S]\rarr	|p	|u	|d	|ł	|o	|*	|t	|*	|*	|*	|.
|*	|*	|*	|*	|*	|a	|*	|*	|*	|[46][S]\rarr	|ś	|l	|ą	|s	|k	|i	|e	|[][,]{ }	|n	|i	|e	|b	|o	|*	|.
|*	|*	|*	|*	|*	|*	|*	|*	|*	|*	|*	|*	|*	|*	|*	|*	|*	|*	|*	|*	|k	|*	|*	|*	|.
|*	|*	|*	|*	|*	|*	|*	|*	|*	|*	|*	|*	|*	|*	|*	|*	|*	|*	|*	|*	|*	|*	|*	|*	|.\end{Puzzle}

\newpage

\begin{PuzzleClues}{\textbf{Poziome}\\}\Clue{3}{}{unitarne państwo wyspiarskie położone w Europie Zachodniej}
\Clue{4}{}{dział medycyny, nauka o chorobach narządów wewnętrznych}
\Clue{5}{}{zorganizowana działalność społeczna kościołów polegająca na głoszeniu zasad wiary i celebrowaniu liturgii w różnych grupach, nad którymi opiekę ze strony kościołów sprawuje duszpasterz}
\Clue{7}{}{miasto w województwie pomorskim, w powiecie puckim, w gminie Jastarnia, na Mierzei Helskiej}
\Clue{8}{}{Polski stawiacz min z okresu II wojny światowej}
\Clue{10}{}{zdrobnienie słowa siermięga}
\Clue{11}{}{przypadek, który sprawia, że coś się udaje, odrobina powodzenia}
\Clue{12}{}{myszoryjek}
\Clue{14}{}{kilkanaście niewielkich grup narodowych, stanowiących pozostałości dawnego zaludnienia północno-wschodniej Azji}
\Clue{15}{}{zestawienie wszystkich transakcji dokonanych między rezydentami (gospodarką krajową) a nierezydentami (zagranicą) w danym okresie}
\Clue{16}{}{w chemii: symbol chloru}
\Clue{17}{}{JARZĄB, BRZĘK}
\Clue{18}{}{otwarcie szachowe, które charakteryzuje się posunięciami: 1. e4 e5, 2. Sf3 Sc6, 3. Gb5}
\Clue{19}{}{uczucie, stan intensywnej fascynacji jakimś zjawiskiem, zaangażowania się w jakieś zjawisko czy działanie, zafiksowania się na jakimś zjawisku; bycie wkręconym}
\Clue{20}{}{torba na mapy}
\Clue{21}{}{Chlorestes notata - gatunek ptaka z rzędu jerzykowych (Apodiformes), z rodziny kolibrów (Trochilidae), z podrodziny kolibrów (Trochilinae)}
\Clue{23}{}{hormon wytwarzany przez tkankę nerwową, głównie przez komórki nerwowe podwzgórza i przysadki, zakończenia włókien nerwowych}
\Clue{26}{}{tendencja w sztuce w latach 70. i 80. XX wieku szukająca nowych form wyrazu w stosunku do awangardy}
\Clue{27}{}{komisja, która rozstrzyga spory między pracownikami a pracodawcami}
\Clue{29}{}{maleńkie drobiny substancji wszelkiego rodzaju}
\Clue{30}{}{najniższy stopień oficerski w marynarce wojennej w Rosji carskiej, w radzieckiej zaś najwyższy stopień podoficerski}
\Clue{31}{}{Leptoptilos javanicus - gatunek ptaka z rodziny bocianowatych (Ciconiidae); zamieszkuje południową Azję, od Indii do południowych Chin i Jawy}
\Clue{32}{}{przestrzeń okalajaca coś, zwykle w układzie (kształcie) kolistym, półkolistym lub łukowo wygiętym}
\Clue{33}{}{fałd skóry, który często wypełniony jest tłuszczem}
\Clue{34}{}{dawne czasy, okres prehistoryczny}
\Clue{38}{}{wygięcie warstw skalnych, które straciło swoją pierwotną strukturę}
\Clue{39}{}{kawałek futra lub materiału podbity watoliną, zeszyty w ten sposób żeby można było włożyć doń dłonie dla ochrony przed zimnem - noszona od XVI do XX w}
\Clue{40}{}{pojemnik lub opakowanie w kształcie rurki}
\Clue{41}{}{księga handlowa, w której zawarte są zestawienia obrotów, sprawozdania i bilanse za krótki okres}
\Clue{43}{}{materiał do gaszenia łuku elektrycznego w łączniku}
\Clue{44}{}{osoba, która jest członkiem ziomkostwa}
\Clue{45}{}{odsiadka, kara więzienia}
\Clue{46}{}{mięsna potrawa regionalna kuchni śląskiej przygotowywana przede wszystkim na świąteczne okazje}\end{PuzzleClues}

\begin{PuzzleClues}{\textbf{Pionowe}\\}\Clue{1}{}{botanik niemiecki (1831-88); twórca podstaw mikologii}
\Clue{2}{}{domknięcie zbioru argumentów funkcji, dla których ma ona wartość różną od zera}
\Clue{3}{}{u kręgowców - część szkieletu podtrzymująca kończyny przednie (przednie płetwy)}
\Clue{6}{}{rodzaj figury akrobacji lotniczej: pół pętli z przewrotem}
\Clue{9}{}{przedstawiciel jednego z plemion zamieszkujących tereny współczesnej Szkocji w czasie, gdy starożytni Rzymianie przybyli do północnej Brytanii}
\Clue{13}{}{Pyrus communis 'Triumf Packhama' - odmiana uprawna gruszy pospolitej}
\Clue{15}{}{kod ISO 4217 waluty dalasi}
\Clue{16}{}{to, że coś się przepoczwarzyło; o owadach}
\Clue{22}{}{figura retoryczna, która polega na użyciu dwóch słów (zwykle rzeczowników) w tej samej formie fleksyjnej}
\Clue{24}{}{Pardosa - rodzaj pająka z rodziny pogońcowatych; niektóre gatunki wałęsaków są jadowite i groźne dla człowieka}
\Clue{25}{}{Wiechecki - (1896-1979), pisarz, humorysta i satyryk; „Znakiem tego”, „Wiadomo - stolica”, „Cafe pod Minoga”, „Ksiuty z Melpomeną”}
\Clue{26}{}{osoba będąca zwierzchnikiem osad lokowanych na prawie wołoskim znajdującym się w obrębie średniowiecznego starostwa}
\Clue{27}{}{nieprzemakalny płaszcz noszony przez marynarzy}
\Clue{28}{}{wyłączne prawo właściciela dóbr ziemskich do produkcji i sprzedaży piwa, gorzałki i miodu w obrębie jego dóbr oraz przywilej do sprowadzania tych wyrobów z innych miast i czerpania z tego tytułu dochodów}
\Clue{30}{}{miasto w USA (Teksas) port połączony kanałem z Zatoką Meksykańska; ośrodek rafinacji ropy naftowej}
\Clue{32}{}{mała cyfra}
\Clue{34}{}{jedna z dwu podstawowych części każdej broni białej, drugą częścią jest rękojeść; jest zasadniczą (główną - stąd etymologia), roboczą częścią broni tego typu}
\Clue{35}{}{angielski anatom i fizjolog (1597-1677); autor jednej z pierwszych prac o krzywicy}
\Clue{36}{}{hala pneumatyczna, stanowiąca poszycie obiektów na świeżym powietrzu, najczęściej sportowych}
\Clue{37}{}{poziomy pas dekoracyjny stosowany w architekturze}
\Clue{42}{}{pilot szybowcowy, mistrz świata w klasie standard w 1958 r}\end{PuzzleClues}\newpage\section*{Krzyżówka 185}

\noindent\begin{Puzzle}{23}{31}|*	|*	|*	|[1][S]\drarr	|d	|a	|n	|i	|e	|l	|[][,]{ }	|d	|u	|b	|i	|c	|k	|i	|*	|[2][S]\darr	|[3][S]\darr	|[4][S]\darr	|[5][S]\darr	|[6][S]\darr	|.
|[7][S]\rarr	|f	|r	|a	|n	|c	|i	|s	|*	|*	|*	|*	|*	|*	|*	|*	|*	|*	|*	|ł	|i	|z	|k	|s	|.
|*	|*	|[8][S]\rarr	|l	|i	|p	|o	|p	|r	|o	|t	|e	|i	|d	|*	|*	|*	|*	|*	|a	|z	|r	|w	|t	|.
|*	|*	|[9][S]\darr	|m	|[10][S]\darr	|[11][S]\darr	|*	|*	|*	|[12][S]\drarr	|s	|y	|s	|t	|e	|m	|i	|k	|*	|t	|o	|o	|e	|a	|.
|*	|[13][S]\drarr	|p	|a	|r	|z	|y	|s	|t	|o	|k	|o	|p	|y	|t	|n	|e	|*	|[14][S]\darr	|a	|n	|s	|s	|w	|.
|*	|a	|u	|n	|z	|i	|*	|*	|[15][S]\rarr	|p	|r	|z	|e	|d	|z	|i	|m	|e	|k	|*	|i	|t	|t	|a	|.
|*	|p	|m	|a	|e	|m	|*	|[16][S]\drarr	|r	|o	|z	|u	|m	|*	|[17][S]\rarr	|h	|i	|n	|a	|j	|a	|n	|a	|*	|.
|*	|r	|a	|c	|p	|n	|*	|s	|[18][S]\darr	|ń	|*	|[19][S]\rarr	|b	|a	|u	|m	|a	|n	|n	|*	|z	|i	|*	|*	|.
|*	|e	|*	|h	|*	|a	|*	|o	|k	|c	|*	|*	|[20][S]\drarr	|o	|n	|d	|u	|l	|a	|*	|y	|c	|[21][S]\darr	|*	|.
|*	|t	|[22][S]\darr	|*	|*	|[][,]{ }	|[23][S]\drarr	|f	|a	|z	|e	|n	|d	|a	|*	|[24][S]\drarr	|s	|ó	|l	|*	|d	|o	|d	|*	|.
|*	|u	|k	|*	|[25][S]\darr	|r	|b	|c	|m	|y	|[26][S]\rarr	|c	|z	|e	|r	|e	|ś	|n	|i	|a	|*	|w	|e	|*	|.
|*	|r	|o	|*	|e	|y	|r	|i	|i	|k	|*	|*	|i	|[27][S]\rarr	|p	|r	|ó	|b	|k	|a	|*	|a	|d	|*	|.
|*	|o	|l	|[28][S]\darr	|k	|b	|a	|k	|e	|o	|[29][S]\rarr	|m	|e	|z	|o	|p	|o	|r	|*	|*	|*	|t	|e	|*	|.
|*	|w	|c	|c	|o	|a	|h	|*	|ń	|w	|*	|*	|d	|*	|*	|e	|*	|[30][S]\drarr	|s	|e	|l	|e	|r	|*	|.
|*	|a	|o	|z	|g	|*	|e	|*	|s	|a	|*	|*	|z	|*	|*	|g	|*	|p	|*	|[31][S]\darr	|*	|*	|k	|*	|.
|*	|n	|l	|a	|r	|*	|*	|*	|k	|t	|[32][S]\rarr	|d	|i	|o	|l	|*	|*	|a	|*	|k	|*	|*	|o	|*	|.
|*	|i	|i	|s	|o	|[33][S]\darr	|[34][S]\drarr	|n	|i	|e	|w	|y	|c	|z	|u	|w	|a	|l	|n	|o	|ś	|ć	|*	|*	|.
|*	|e	|c	|[][,]{ }	|s	|d	|s	|*	|*	|*	|*	|*	|z	|*	|*	|[35][S]\darr	|*	|m	|*	|o	|*	|*	|[36][S]\darr	|*	|.
|*	|*	|z	|u	|z	|u	|t	|*	|*	|*	|[37][S]\drarr	|r	|e	|l	|i	|k	|w	|i	|a	|r	|z	|*	|k	|*	|.
|[38][S]\drarr	|g	|e	|n	|e	|r	|a	|ł	|*	|*	|h	|*	|n	|*	|*	|a	|[39][S]\darr	|*	|[40][S]\darr	|d	|[41][S]\darr	|*	|a	|*	|.
|g	|*	|k	|i	|k	|g	|n	|*	|[42][S]\rarr	|r	|y	|b	|i	|k	|*	|s	|s	|*	|s	|y	|h	|*	|c	|*	|.
|o	|*	|*	|w	|*	|*	|d	|[43][S]\rarr	|b	|y	|d	|l	|e	|ń	|*	|e	|z	|*	|k	|n	|i	|[44][S]\darr	|z	|*	|.
|g	|*	|*	|e	|[45][S]\rarr	|h	|e	|r	|b	|e	|r	|t	|*	|*	|*	|t	|w	|[46][S]\darr	|r	|a	|p	|a	|y	|*	|.
|o	|[47][S]\rarr	|k	|r	|ą	|g	|*	|[48][S]\rarr	|b	|l	|o	|k	|*	|[49][S]\darr	|*	|a	|a	|k	|z	|t	|e	|u	|[][,]{ }	|*	|.
|l	|[50][S]\darr	|[51][S]\rarr	|s	|e	|k	|s	|a	|p	|i	|l	|*	|*	|p	|*	|*	|r	|u	|y	|a	|r	|t	|d	|*	|.
|*	|ż	|[52][S]\rarr	|a	|t	|*	|[53][S]\rarr	|p	|o	|g	|o	|n	|i	|e	|c	|*	|c	|d	|d	|*	|t	|o	|ó	|*	|.
|[54][S]\drarr	|m	|i	|l	|a	|[][,]{ }	|m	|o	|r	|s	|k	|a	|*	|p	|*	|*	|o	|e	|ł	|*	|e	|p	|ł	|*	|.
|a	|u	|*	|n	|[55][S]\rarr	|p	|i	|e	|c	|z	|a	|r	|n	|i	|k	|[][,]{ }	|w	|ł	|o	|s	|k	|i	|*	|*	|.
|j	|d	|*	|y	|*	|[56][S]\rarr	|p	|i	|r	|a	|t	|*	|*	|t	|*	|*	|n	|k	|*	|*	|s	|l	|*	|*	|.
|a	|a	|*	|*	|[57][S]\rarr	|k	|l	|i	|n	|g	|o	|ń	|s	|k	|i	|*	|i	|i	|*	|*	|t	|o	|*	|*	|.
|*	|*	|*	|*	|*	|*	|*	|*	|[58][S]\rarr	|g	|r	|a	|m	|a	|t	|y	|k	|*	|*	|*	|*	|t	|*	|*	|.
|[59][S]\rarr	|o	|ż	|w	|i	|o	|w	|a	|t	|e	|*	|*	|*	|*	|*	|*	|*	|[60][S]\rarr	|k	|ą	|t	|*	|*	|*	|.\end{Puzzle}

\newpage

\begin{PuzzleClues}{\textbf{Poziome}\\}\Clue{1}{}{polski piłkarz, występujący na pozycji pomocnika lub napastnika, oraz trener}
\Clue{7}{}{malarz francuski (1420-przed 1481) kontynuator paryskiej szkoły iluminatorskiej; miniatury, portrety}
\Clue{8}{}{białko złożone, nośnik cholesterolu, wchodzące w skład błony komórkowej}
\Clue{12}{}{ustrój społeczno-polityczny}
\Clue{13}{}{Artiodactyla - rząd ssaków łożyskowych, których naturalnym obszarem występowania są wszystkie kontynenty poza Antarktydą i Australią; zajmują różne środowiska - lasy, sawanny, pustynie, góry, a nawet zbiorniki wodne}
\Clue{15}{}{PIĘDZIK}
\Clue{16}{}{rozsądność, zmyślność}
\Clue{17}{}{jeden z dwu głównych kierunków buddyzmu, propaguje ideał arhanta, własnym wysiłkiem osiągającego doskonałość i nirwanę}
\Clue{19}{}{lekkoatleta niemiecki, mistrz olimpijski z Barcelony w biegu na 5000 m}
\Clue{20}{}{tkanina z wełny czesankowej, z lekko falowanymi poprzecznymi prążkami, podobna do rypsu}
\Clue{23}{}{dom na dużej plantacji, szczególnie plantacji kawy}
\Clue{24}{}{w chemii: związek chemiczny powstały w wyniku całkowitego lub częściowego zastąpienia w kwasach atomów wodoru innymi atomami bądź grupami o właściwościach elektrofilowych}
\Clue{26}{}{Prunus avium - gatunek rośliny należący do rodziny różowatych; niewielkie drzewo, które owocuje małymi czerwonawymi lub bordowymi pestkowcami, lekko cierpkimi, ale i słodkimi w smaku}
\Clue{27}{}{niewielka ilość substancji lub wyrobu, przeznaczona do badań, doświadczeń lub wykorzystywana do pokazywania, jako całość}
\Clue{29}{}{por o rozmiarze od 2 do 50 nm (według IUPAC)}
\Clue{30}{}{roślina warzywna o pierzastych liściach i jadalnym korzeniu}
\Clue{32}{}{alkohol zawierający dwie grupy hydroksylowe}
\Clue{34}{}{to, że coś jest niewyczuwalne fizycznie, nie można tego wyczuć, dostrzec zmysłami}
\Clue{37}{}{w liturgii katolickiej: uroczyste nabożeństwo wielkanocne z procesją odprawiane w Wielką Sobotę lub pierwszy dzień Wielkanocy}
\Clue{38}{}{oficer wojskowy wysokiego stopnia}
\Clue{42}{}{owad bezskrzydły okryty metalicznie lśniącymi łuskami, szkodnik w magazynach i spiżarniach}
\Clue{43}{}{giez bydlęcy, duża, owłosiona muchówka, której larwy rozwijają się w organizmie przeżuwaczy}
\Clue{45}{}{Zbigniew Herbert - polski poeta, eseista, dramaturg, twórca słynnego cyklu poetyckiegoPan Cogito, autor słuchowisk; kawaler Orderu Orła Białego; z wykształcenia ekonomista, prawnik i filozof}
\Clue{47}{}{pierścień zwinięty z wielu zwojów walcówki, taśmy, drutu itp}
\Clue{48}{}{przeszkoda w przewodzeniu bodźców nerwowych, zwłaszcza w układzie przedsionkowo-komorowym serca}
\Clue{51}{}{atrakcyjność fizyczna, energia seksualna}
\Clue{52}{}{jednostka zdawkowa w Laosie; 1/100 kipa}
\Clue{53}{}{mały pająk, który żyje w ściółce leśnej i niedaleko wody, każdy pająk z rodziny pogońcowatych}
\Clue{54}{}{jednostka odległości stosowana w nawigacji morskiej oraz lotnictwie; jednej mili morskiej odpowiadają 1852 metry, czyli uśredniona długość łuku południka ziemskiego odpowiadająca jednej minucie kątowej koła wielkiego}
\Clue{55}{}{Hydromantes italicus - gatunek płaza ogoniastego z rodziny bezpłucnikowatych, występujący w północnych i środkowych Włoszech oraz na bardzo małym obszarze w południowo-wschodniej Francji}
\Clue{56}{}{klasa jachtu; mała jednomasztowa drewniana żaglówka}
\Clue{57}{}{sztuczny język artystyczny wymyślony przez dra Marca Okranda dla rasy Klingonów na potrzeby serialu Star Trek}
\Clue{58}{}{osoba specjalizująca się w dziedzinie gramatyki}
\Clue{59}{}{Loasaceae - rodzina roślin z rzędu dereniowców; obejmuje 20 rodzajów z około 330 gatunkami, największe ich zróżnicowanie występuje w zachodniej części Stanów Zjednoczonych i w Meksyku, nieco mniejsze w Andach w Ameryce Południowej}
\Clue{60}{}{przestrzeń (róg) między dwiema stykającymi się powierzchniami}\end{PuzzleClues}

\begin{PuzzleClues}{\textbf{Pionowe}\\}\Clue{1}{}{antologia zawierająca utwory różnych autorów, które powstały w tym samym czasie}
\Clue{2}{}{deska do sprawdzania prostoliniowości wznoszonego muru}
\Clue{3}{}{lek przeciwbakteryjny, stosowany w leczeniu gruźlicy oraz niektórych innych mykobakterioz}
\Clue{4}{}{Zygnemaceae - rodzina glonów z grupy zielenic (Chlorophyta), z gromady Charophyceae}
\Clue{5}{}{zbieranie datków pieniężnych w celach charytatywnych przez osoby do tego uprawnione, tzw. kwestarzy}
\Clue{6}{}{ustawiony na lądzie stały znak nawigacyjny}
\Clue{9}{}{KUGUAR; amerykański drapieżnik z rodziny kotów o dł. 2 m}
\Clue{10}{}{potoczne określenie materiału o nazwie handlowej Velcro; wynalazek mający postać taśmy pokrytej małymi haczykami, które wczepiają się w drugą pokrytą mechatą tkaniną taśmę, tworząc razem połączenie łatwe do rozpinania i zapinania, wykorzystywane w produkcji odzieży, sprzętu sportowego i innych przedmiotów powszechnego użytku}
\Clue{11}{}{osoba niewrażliwa, nieumiejąca okazywać uczuć}
\Clue{12}{}{Encalyptaceae - rodzina mchów z rzędu opończykowców}
\Clue{13}{}{uszlachetnianie, wykańczanie materiału nadające mu nowych lub lepszych cech i właściwości}
\Clue{14}{}{rów, który jest mały i wąski}
\Clue{16}{}{film lub inne medium zawierające treści o charakterze miękkiej pornografii (soft porno)}
\Clue{18}{}{Maciej (1734-1821); kompozytor twórca pierwszej zachowanej opery polskiej; 'Nędza uszczęśliwiona'}
\Clue{20}{}{mechanizm współdzielenia funkcjonalności między klasami w programowaniu obiektowym}
\Clue{21}{}{fotografik, inżynier elektryk (1880-1965); twórca nowoczesnego portretu w fotografice polskiej}
\Clue{22}{}{Acanthagenys rufogularis - gatunek ptaka z rodziny miodojadów (Meliphagidae)}
\Clue{23}{}{astronom duński (1546-1601), supernowa gwiazda w gwiazdozbiorze Kasjopei}
\Clue{24}{}{ręczny granatnik przeciwpancery RPG-7}
\Clue{25}{}{nazwa handlowa określająca paliwo stałe produkowane z węgla kamiennego lub brunatnego przeznaczone do produkcji energii cieplnej w kotłach retortowych}
\Clue{28}{}{astronomiczny czas słoneczny średni na południku zerowym, za który przyjęto południk przechodzący przez obserwatorium astronomiczne w miejscowości Greenwich}
\Clue{30}{}{miasto we Włoszech (Kalabria) u wybrzeży Morza Tyrreńskiego}
\Clue{31}{}{liczba, która opisuje położenie punktu w jakimś układzie odniesienia, współrzędnych (często w liczbie mnogiej)}
\Clue{33}{}{miasto w Indiach (Madhja Prades) ważny ośrodek hutnictwa żelaza, 490,2 tys. mieszkańców (1991)}
\Clue{34}{}{(1897-1939), poeta, działacz KPP; „Trzy salwy” (z Broniewskim i Wandurskim)}
\Clue{35}{}{opakowanie, pudełko służące do przechowywania drobnych przedmiotów}
\Clue{36}{}{dół, najczęściej z wodą (lub pozostałością po niej, np. mułem), dziki i zaniedbany, porośnięty roślinnością}
\Clue{37}{}{AZDYK, ECHOSONDA}
\Clue{38}{}{rosyjski pisarz, poeta, dramaturg i publicysta; klasyk literatury rosyjskiej}
\Clue{39}{}{(środ.?, reg.?) osoba zajmująca się przemytem, któraszwarcuje}
\Clue{40}{}{część kompozycji artystycznej, która może być połączona z pozostałymi częściami w sposób ruchomy lub nieruchomy}
\Clue{41}{}{tekst w postaci elektronicznej, cechuje go nielinearność i możliwość kontekstowego przemieszczania się w nim}
\Clue{44}{}{samoczynny sternik}
\Clue{46}{}{zdrobniale o kudłach jako włosach}
\Clue{49}{}{klasyczny deseń w formie nieco przekrzywionej kratki}
\Clue{50}{}{znój, trud; żmudna, ciężka praca}
\Clue{54}{}{werset Koranu}\end{PuzzleClues}\newpage\section*{Krzyżówka 186}

\noindent\begin{Puzzle}{25}{24}|*	|*	|*	|*	|*	|*	|*	|*	|*	|*	|*	|*	|*	|*	|*	|*	|*	|*	|[1][S]\darr	|*	|*	|*	|*	|*	|*	|*	|.
|*	|*	|*	|*	|*	|*	|*	|*	|*	|*	|*	|*	|*	|*	|*	|*	|*	|*	|k	|*	|*	|*	|*	|*	|*	|*	|.
|*	|*	|*	|*	|*	|*	|*	|*	|*	|*	|*	|*	|*	|*	|*	|*	|*	|*	|r	|*	|*	|*	|*	|*	|*	|*	|.
|*	|*	|*	|*	|*	|*	|*	|*	|*	|[2][S]\darr	|*	|*	|*	|[3][S]\rarr	|d	|o	|k	|ł	|a	|d	|n	|o	|ś	|ć	|*	|*	|.
|*	|*	|*	|*	|*	|*	|*	|*	|*	|k	|*	|*	|*	|*	|*	|*	|*	|*	|j	|*	|[4][S]\darr	|*	|*	|*	|*	|*	|.
|[5][S]\rarr	|o	|l	|e	|j	|e	|k	|[][,]{ }	|k	|a	|m	|f	|o	|r	|o	|w	|y	|*	|o	|*	|j	|*	|*	|*	|*	|*	|.
|*	|[6][S]\drarr	|k	|u	|l	|i	|k	|[][,]{ }	|b	|r	|ą	|z	|o	|w	|o	|g	|r	|z	|b	|i	|e	|t	|y	|*	|*	|*	|.
|*	|a	|*	|*	|*	|*	|*	|*	|*	|i	|*	|*	|*	|*	|*	|*	|*	|*	|r	|*	|l	|*	|*	|*	|*	|*	|.
|*	|n	|*	|*	|[7][S]\rarr	|h	|i	|s	|p	|a	|n	|o	|*	|*	|*	|*	|*	|*	|a	|*	|e	|*	|*	|*	|*	|*	|.
|*	|a	|*	|*	|*	|*	|*	|*	|*	|m	|*	|*	|*	|*	|*	|*	|*	|*	|z	|*	|ń	|*	|*	|*	|*	|*	|.
|*	|l	|*	|*	|*	|[8][S]\drarr	|n	|o	|t	|a	|*	|*	|*	|*	|*	|*	|*	|*	|[][,]{ }	|*	|*	|*	|*	|*	|*	|*	|.
|*	|i	|[9][S]\rarr	|s	|a	|r	|d	|o	|u	|*	|*	|*	|*	|*	|*	|*	|*	|*	|p	|*	|*	|*	|*	|*	|*	|*	|.
|*	|z	|*	|*	|*	|e	|*	|*	|*	|*	|*	|*	|*	|*	|*	|*	|*	|*	|e	|*	|*	|*	|*	|*	|*	|*	|.
|*	|a	|*	|*	|[10][S]\darr	|s	|*	|*	|*	|*	|*	|*	|*	|*	|*	|[11][S]\darr	|*	|*	|r	|*	|*	|*	|*	|*	|*	|*	|.
|*	|[][,]{ }	|*	|*	|r	|o	|*	|*	|*	|*	|*	|*	|*	|*	|*	|d	|*	|*	|y	|*	|*	|*	|*	|*	|*	|*	|.
|[12][S]\rarr	|c	|z	|t	|e	|r	|d	|z	|i	|e	|s	|t	|ó	|w	|k	|a	|*	|*	|g	|*	|*	|*	|*	|*	|*	|*	|.
|*	|h	|*	|*	|z	|*	|*	|*	|*	|*	|*	|*	|*	|*	|*	|r	|*	|*	|l	|*	|*	|*	|*	|*	|*	|*	|.
|*	|e	|*	|*	|o	|*	|*	|*	|*	|*	|*	|*	|*	|*	|*	|v	|*	|*	|a	|*	|*	|*	|*	|*	|*	|*	|.
|*	|m	|*	|*	|n	|*	|*	|*	|*	|*	|*	|*	|*	|*	|*	|a	|*	|*	|c	|*	|*	|*	|*	|*	|*	|*	|.
|*	|i	|*	|*	|a	|*	|*	|*	|*	|*	|*	|*	|*	|*	|*	|s	|*	|*	|j	|*	|*	|*	|*	|*	|*	|*	|.
|*	|c	|*	|*	|t	|*	|*	|*	|*	|*	|*	|*	|*	|*	|*	|*	|*	|*	|a	|*	|*	|*	|*	|*	|*	|*	|.
|*	|z	|*	|*	|o	|*	|*	|*	|*	|*	|*	|*	|*	|*	|*	|*	|*	|*	|l	|*	|*	|*	|*	|*	|*	|*	|.
|*	|n	|*	|*	|r	|*	|*	|*	|*	|*	|*	|*	|*	|*	|*	|*	|*	|*	|n	|*	|*	|*	|*	|*	|*	|*	|.
|*	|a	|*	|*	|*	|*	|*	|*	|[13][S]\rarr	|n	|i	|e	|a	|u	|t	|e	|n	|t	|y	|c	|z	|n	|o	|ś	|ć	|*	|.
|*	|*	|*	|*	|*	|*	|*	|*	|*	|*	|*	|*	|*	|*	|*	|*	|*	|*	|*	|*	|*	|*	|*	|*	|*	|*	|.\end{Puzzle}

\newpage

\begin{PuzzleClues}{\textbf{Poziome}\\}\Clue{3}{}{cecha jakiegoś działania; to, że coś jest wykonywane szczegółowo i precyzyjnie}
\Clue{5}{}{olejek eteryczny, pozyskiwany z korzeni i drewna cynamonowca kamforowego, składnikiem którego jest m.in. kamfora}
\Clue{6}{}{Numenius phaeopus variegatus - podgatunek ptaka wyróżniony w obrębie gatunku kulik mniejszy (Numenius phaeopus); obszar występowania obejmuje północno-wschodnią Syberię po dorzecza Kołyki i Anadyru}
\Clue{7}{}{angloarab hiszpański - rasa koni wywodząca się z Hiszpanii; powstała poprzez krzyżowanie klaczy o andaluzyjskim czy też arabskim dziedzictwie z folblutami}
\Clue{8}{}{niewielkich rozmiarów tekst prasowy}
\Clue{9}{}{(1831-1908), dramatopisarz francuski; „Madame Sans-Gene”}
\Clue{12}{}{czterdziesta rocznica}
\Clue{13}{}{cecha czegoś, co nie jest tym, za co ktoś to podaje}\end{PuzzleClues}

\begin{PuzzleClues}{\textbf{Pionowe}\\}\Clue{1}{}{krajobraz, który powstał na skutek działania procesów transportu i akumulacji zniszczonego i rozdrobnionego materiału skalnego, występuje w dużych odległościach od przedpola lądolodu}
\Clue{2}{}{południowoamerykański ptak z rodziny dwuczubów}
\Clue{4}{}{jeleń szlachetny, Cervus elaphus - gatunek dużego ssaka lądowego z rodziny jeleniowatych, rozpowszechnionego na całym świecie}
\Clue{6}{}{badanie jakościowego (analiza jakościowa) i ilościowego (analiza ilościowa) składu chemicznego substancji}
\Clue{8}{}{element sprężynujący, łagodzący wstrząsy w czasie jazdy pojazdów}
\Clue{10}{}{część niektórych instrumentów muzycznych, która współgra z wibratorem (np. ze strunami w gitarze, skrzypcach) i powoduje wzmocnienie dźwięku}
\Clue{11}{}{węgierski pisarz i działacz polityczny (1912-73), członek Światowej Rady Pokoju; „Czarny chleb”, powieści, dramaty}\end{PuzzleClues}\newpage\section*{Krzyżówka 187}

\noindent\begin{Puzzle}{22}{22}|*	|*	|*	|*	|*	|*	|*	|*	|*	|*	|*	|*	|*	|*	|*	|[1][S]\drarr	|a	|u	|g	|i	|e	|r	|*	|.
|*	|*	|*	|*	|*	|*	|[2][S]\drarr	|m	|o	|t	|o	|r	|*	|[3][S]\drarr	|k	|u	|c	|z	|b	|a	|j	|a	|*	|.
|*	|*	|[4][S]\rarr	|f	|a	|r	|b	|i	|a	|r	|z	|*	|*	|l	|[5][S]\rarr	|d	|a	|m	|a	|n	|*	|*	|*	|.
|*	|*	|[6][S]\rarr	|m	|i	|k	|a	|s	|a	|*	|[7][S]\rarr	|m	|r	|o	|k	|a	|w	|k	|i	|*	|*	|*	|*	|.
|*	|[8][S]\rarr	|a	|f	|f	|e	|t	|t	|u	|s	|o	|*	|[9][S]\drarr	|g	|e	|r	|a	|*	|[10][S]\darr	|*	|*	|[11][S]\darr	|*	|.
|*	|[12][S]\rarr	|b	|e	|j	|d	|e	|w	|i	|n	|d	|*	|s	|g	|[13][S]\darr	|[][,]{ }	|*	|[14][S]\darr	|t	|*	|[15][S]\darr	|a	|*	|.
|*	|[16][S]\rarr	|p	|o	|n	|u	|r	|a	|k	|*	|*	|*	|i	|i	|s	|m	|[17][S]\darr	|s	|e	|*	|w	|n	|*	|.
|*	|[18][S]\rarr	|o	|r	|c	|h	|i	|d	|e	|a	|*	|*	|e	|a	|ę	|ó	|d	|p	|r	|[19][S]\darr	|ó	|s	|*	|.
|*	|*	|[20][S]\rarr	|t	|e	|t	|a	|n	|u	|r	|*	|*	|w	|*	|k	|z	|o	|l	|e	|b	|ł	|e	|*	|.
|*	|*	|*	|*	|*	|*	|[][,]{ }	|[21][S]\drarr	|a	|z	|o	|t	|e	|k	|[][,]{ }	|g	|l	|i	|n	|u	|*	|r	|[22][S]\darr	|.
|*	|*	|*	|[23][S]\darr	|*	|[24][S]\darr	|g	|g	|[25][S]\darr	|*	|*	|[26][S]\darr	|c	|*	|o	|o	|e	|t	|*	|t	|*	|i	|p	|.
|*	|*	|*	|g	|[27][S]\rarr	|b	|a	|r	|a	|n	|*	|a	|z	|*	|t	|w	|*	|*	|[28][S]\darr	|*	|[29][S]\darr	|m	|a	|.
|*	|*	|[30][S]\darr	|a	|*	|a	|l	|u	|l	|*	|*	|l	|k	|*	|w	|y	|[31][S]\darr	|[32][S]\darr	|d	|*	|k	|i	|n	|.
|*	|*	|j	|z	|*	|c	|w	|b	|l	|*	|[33][S]\drarr	|b	|a	|k	|a	|*	|c	|m	|e	|*	|l	|m	|c	|.
|*	|*	|o	|i	|*	|h	|a	|o	|*	|*	|l	|e	|*	|[34][S]\drarr	|r	|e	|z	|e	|r	|w	|a	|*	|e	|.
|*	|*	|g	|a	|*	|a	|n	|ś	|*	|*	|i	|r	|*	|g	|t	|*	|u	|m	|y	|*	|p	|[35][S]\darr	|r	|.
|[36][S]\rarr	|b	|a	|n	|*	|t	|i	|ć	|*	|*	|a	|t	|*	|e	|y	|*	|k	|b	|w	|*	|s	|g	|n	|.
|*	|*	|*	|t	|*	|a	|c	|*	|*	|*	|o	|y	|*	|t	|*	|*	|a	|r	|a	|*	|*	|r	|i	|.
|[37][S]\drarr	|w	|i	|e	|ś	|*	|z	|*	|*	|*	|n	|n	|*	|r	|[38][S]\rarr	|d	|r	|a	|c	|h	|m	|a	|*	|.
|b	|*	|[39][S]\rarr	|p	|ł	|o	|n	|n	|i	|k	|i	|*	|*	|y	|*	|*	|i	|n	|j	|*	|*	|v	|*	|.
|i	|*	|*	|*	|[40][S]\rarr	|g	|a	|i	|r	|d	|n	|e	|r	|*	|*	|*	|n	|a	|a	|*	|*	|e	|*	|.
|*	|[41][S]\rarr	|l	|e	|e	|r	|*	|*	|*	|[42][S]\rarr	|g	|a	|w	|ę	|d	|a	|*	|*	|*	|*	|*	|s	|*	|.
|*	|[43][S]\rarr	|k	|o	|l	|e	|n	|d	|e	|r	|*	|*	|*	|*	|*	|[44][S]\rarr	|o	|k	|r	|e	|s	|*	|*	|.\end{Puzzle}

\newpage

\begin{PuzzleClues}{\textbf{Poziome}\\}\Clue{1}{}{dramatopisarz francuski (1820-89), komedie obyczajowe i społeczne; „Zięć pana Poirier”}
\Clue{2}{}{źródło, przyczyna, pobudka, czynnik decydujący o takim, a nie innym przebiegu jakiegoś procesu}
\Clue{3}{}{kosmata, ciepła tkanina wełniana, która za czasów Sobieskiego była popularną tkaniną na docieplenia i podszewki płaszczy (zwłaszcza w swojej czerwonej odmianie kolorystycznej), oprócz tego była rozpowszechniona jako tkanina na ciepłe okrycia wierzchnie dla mniej zamożnej, prowincjonalnej szlachty}
\Clue{4}{}{posiadacz farbiarni}
\Clue{5}{}{jest jednym z dwóch dystryktów wchodzących w skład indyjskiego terytorium związkowego Daman i Diu; leży na zachodnim wybrzeżu Indii, nad Morzem Arabskim, graniczy tylko ze stanem Gujarat}
\Clue{6}{}{japoński okręt flagowy w bitwie morskiej pod Cuszima}
\Clue{7}{}{muchy pasożytujące na skórze nietoperza}
\Clue{8}{}{określenie wykonawcze; czule, tkliwie, namiętnie}
\Clue{9}{}{miasto w Niemczech (Turyngia), nad Białą Elsterą, ważny węzeł kolejowy}
\Clue{12}{}{wiatr wiejący skośnie od dziobu}
\Clue{16}{}{ktoś ponury, nadmiernie poważny, nieposiadający poczucia humoru}
\Clue{18}{}{nazwa roślin z rodziny storczykowatych, w szczególności stosowana w odniesieniu do roślin o efektownych kwiatach, uprawianych w warunkach szklarniowych}
\Clue{20}{}{przedstawiciel grupy tetanurów; zwierzę mięsożerne}
\Clue{21}{}{nieorganiczny związek glinu i azotu, półprzewodnik}
\Clue{27}{}{osoba spod znaku zodiaku Barana}
\Clue{33}{}{OKA; japońska 'żywa torpeda' podwieszana do samolotu}
\Clue{34}{}{powściągliwość, ostrożność w ocenie czegoś, ograniczone zaufanie do czegoś, nieufność}
\Clue{36}{}{w nowych mediach: czasowa blokada użytkownika w internecie}
\Clue{37}{}{tereny pozamiejskie, o rzadszej zabudowie, zielone, rolnicze lub letniskowe}
\Clue{38}{}{w metrologii: jednostka masy lub objętości (kojarzona z anglosaskim, amerykańskim kręgiem kulturowym; jednostka o takiej nazwie była też dawniej używana w Polsce w aptekarstwie)}
\Clue{39}{}{Polytrichopsida - klasa roślin należąca do gromady mchów, obejmująca gatunki charakteryzujące się wysokim stopniem organizacji gametofitów}
\Clue{40}{}{bezodpływowe, słone jezioro w południowej Australii, powierzchnia zmienna około 5 tyś. km2}
\Clue{41}{}{miasto w Niemczech (Dolna Saksonia) port nad rzeką Ems}
\Clue{42}{}{utwór literacki w formie opowiadania charakteryzujący się otwartą kompozycją, swobodą w prowadzeniu wątków, powtórzeniami, licznymi zwrotami do słuchaczy}
\Clue{43}{}{kolendra siewna, Coriandrum sativum - gatunek rośliny z rodziny selerowatych (Apiaceae)}
\Clue{44}{}{zdanie wieloczłonowe}\end{PuzzleClues}

\begin{PuzzleClues}{\textbf{Pionowe}\\}\Clue{1}{}{zespół objawów klinicznych związanych z nagłym wystąpieniem ogniskowego lub uogólnionego zaburzenia czynności mózgu, powstały w wyniku zaburzenia krążenia mózgowego i utrzymujący się ponad 24 godziny}
\Clue{2}{}{układ ogniw składających się z dwóch elektrod zanurzonych w elektrolicie, służący do zasilania urządzeń energią}
\Clue{3}{}{odrębny budynek lub pomieszczenie otwierające się arkadami na zewnątrz}
\Clue{9}{}{dżdżownik - nadwodny ptak z rzędu mew - siewek, żywi się ślimakami i skorupiakami}
\Clue{10}{}{przenośnie obszar, dziedzina działania}
\Clue{11}{}{Anserimimus - rodzaj wszystkożernego, dwunożnego teropoda z rodziny ornitomimów; żył w okresie późnej kredy na terenach centralnej Azji}
\Clue{13}{}{wada drewna powstała w wyniku niezarośnięcia resztek odpadłych lub uciętych gałęzi drzewa}
\Clue{14}{}{w żargonie IRC - zerwanie komunikacji pomiędzy serwerami sieci}
\Clue{15}{}{wykastrowany samiec bydła domowego z rodziny parzystokopytnych, trzymany w gospodarstwach i używany przez ludzi przy różnych pracach ze względu na zwoją znaczną wielkość i siłę}
\Clue{17}{}{miasto we Francji (Burgundia) nad rzeką Doubs}
\Clue{19}{}{okrycie nogi, niekoniecznie ludzkiej, jedna sztuka z pary butów}
\Clue{21}{}{cecha dźwięku}
\Clue{22}{}{w Polsce od XVII w. średniozbrojna jazda zwana kozacką}
\Clue{23}{}{II w południowej Turcji, ośrodek administracyjny Gaziantep}
\Clue{24}{}{muzyka z Dominikany, do której tańczy się bachatę - nazwa jednego utworu, ale równie często używana jako nazwa gatunku muzycznego}
\Clue{25}{}{kod ISO 4217 waluty lek}
\Clue{26}{}{zakonnik należący do Zgromadzenia Braci Albertynów}
\Clue{28}{}{znoszenie statku powietrznego}
\Clue{29}{}{uderzenie w czyjeś pośladki, zwłaszcza ręką}
\Clue{30}{}{system ćwiczeń}
\Clue{31}{}{gimnastyk radziecki, siedmiokrotny mistrz i trzykrotny wicemistrz olimpijski z Helsinek i Melbourne}
\Clue{32}{}{naturalna lub synetetyczne sprężysta cienka przegroda oddzielająca dwa ośrodki ciekłe lub gazowe}
\Clue{33}{}{prowincja w płn-wsch. Chinach, powierzchnia 151 tyś. km2, ośrodek administracyjny Shenyang}
\Clue{34}{}{dawniej (głównie w XIX w. i na początku XX w.) często spotykane cholewki z płótna, filcu lub sukna, wkładane zwykle na obuwie jako zabezpieczenie przed chłodem, dziś stosowane w odzieży stylizowanej na taką z dawnej epoki lub przy obuwiu dla współczesnych dandysów (robionym na zamówienie)}
\Clue{35}{}{ur. 1895r, angielski poeta i prozaik; „Ja Klaudiusz”, „Córka Homera”}
\Clue{37}{}{w chemii: symbol bizmutu}\end{PuzzleClues}\newpage\section*{Krzyżówka 188}

\noindent\begin{Puzzle}{23}{27}|*	|*	|*	|*	|*	|*	|[1][S]\drarr	|m	|a	|s	|k	|a	|[][,]{ }	|p	|o	|ś	|m	|i	|e	|r	|t	|n	|a	|*	|.
|*	|*	|*	|[2][S]\rarr	|t	|y	|g	|r	|y	|s	|*	|*	|*	|*	|*	|*	|[3][S]\drarr	|n	|a	|w	|a	|g	|a	|*	|.
|*	|[4][S]\darr	|[5][S]\darr	|*	|*	|*	|ą	|[6][S]\rarr	|p	|r	|o	|s	|t	|o	|m	|y	|ś	|l	|n	|o	|ś	|ć	|*	|*	|.
|*	|p	|t	|*	|*	|*	|g	|*	|[7][S]\rarr	|g	|a	|z	|[][,]{ }	|ł	|z	|a	|w	|i	|ą	|c	|y	|*	|*	|*	|.
|*	|o	|u	|[8][S]\rarr	|b	|l	|o	|k	|[][,]{ }	|w	|s	|c	|h	|o	|d	|n	|i	|*	|*	|*	|*	|*	|*	|*	|.
|[9][S]\drarr	|s	|p	|ó	|ł	|g	|ł	|o	|s	|k	|a	|[][,]{ }	|s	|z	|c	|z	|e	|l	|i	|n	|o	|w	|a	|*	|.
|d	|k	|e	|*	|*	|*	|*	|[10][S]\rarr	|g	|a	|d	|ż	|e	|c	|i	|a	|r	|s	|t	|w	|o	|*	|[11][S]\darr	|*	|.
|y	|o	|t	|*	|*	|*	|*	|[12][S]\rarr	|l	|i	|c	|z	|b	|a	|[][,]{ }	|m	|a	|s	|o	|w	|a	|*	|n	|*	|.
|r	|c	|*	|*	|*	|*	|*	|*	|[13][S]\rarr	|ż	|y	|ł	|a	|[][,]{ }	|j	|ą	|d	|r	|o	|w	|a	|*	|o	|[14][S]\darr	|.
|e	|z	|*	|[15][S]\darr	|[16][S]\rarr	|g	|a	|z	|[][,]{ }	|f	|e	|r	|m	|i	|o	|n	|ó	|w	|*	|*	|*	|*	|w	|a	|.
|k	|*	|*	|z	|[17][S]\rarr	|f	|o	|r	|m	|a	|[][,]{ }	|d	|r	|u	|k	|o	|w	|a	|*	|*	|*	|*	|i	|e	|.
|t	|*	|[18][S]\rarr	|a	|n	|a	|l	|f	|a	|b	|e	|t	|k	|a	|*	|*	|[][,]{ }	|*	|*	|*	|*	|*	|k	|r	|.
|o	|*	|*	|m	|*	|*	|*	|*	|[19][S]\drarr	|g	|w	|ó	|ź	|d	|ź	|*	|z	|*	|*	|*	|*	|[20][S]\darr	|o	|o	|.
|r	|[21][S]\drarr	|j	|ę	|z	|y	|k	|[][,]{ }	|j	|e	|l	|e	|n	|i	|*	|[22][S]\rarr	|d	|y	|b	|y	|*	|p	|w	|b	|.
|i	|r	|*	|ś	|[23][S]\darr	|[24][S]\drarr	|z	|d	|a	|n	|i	|e	|[][,]{ }	|p	|o	|d	|r	|z	|ę	|d	|n	|e	|*	|i	|.
|a	|o	|*	|c	|n	|g	|*	|[25][S]\darr	|w	|[26][S]\darr	|*	|*	|[27][S]\drarr	|k	|i	|j	|ó	|w	|*	|*	|*	|d	|*	|c	|.
|t	|m	|*	|i	|i	|ą	|*	|j	|o	|ł	|*	|*	|m	|*	|[28][S]\darr	|[29][S]\rarr	|j	|e	|k	|i	|m	|o	|w	|*	|.
|*	|e	|[30][S]\drarr	|e	|k	|s	|p	|i	|r	|a	|c	|j	|a	|*	|p	|*	|*	|*	|[31][S]\rarr	|h	|a	|l	|l	|*	|.
|*	|r	|d	|*	|o	|i	|*	|n	|z	|d	|[32][S]\rarr	|a	|n	|t	|i	|o	|q	|u	|i	|a	|*	|o	|*	|[33][S]\darr	|.
|*	|*	|r	|[34][S]\darr	|l	|e	|*	|j	|e	|o	|*	|*	|i	|[35][S]\rarr	|k	|o	|l	|ę	|d	|a	|*	|g	|*	|m	|.
|[36][S]\drarr	|f	|a	|l	|a	|n	|g	|a	|*	|w	|*	|*	|l	|[37][S]\rarr	|l	|i	|s	|*	|*	|*	|*	|i	|*	|o	|.
|o	|*	|j	|o	|i	|i	|*	|*	|*	|n	|[38][S]\rarr	|b	|a	|w	|e	|ł	|n	|i	|a	|n	|k	|a	|*	|t	|.
|c	|*	|r	|p	|c	|c	|*	|[39][S]\rarr	|k	|i	|w	|i	|*	|*	|*	|*	|*	|[40][S]\darr	|*	|*	|*	|*	|[41][S]\darr	|y	|.
|h	|*	|e	|o	|i	|z	|[42][S]\rarr	|j	|a	|c	|o	|b	|s	|e	|n	|*	|*	|f	|*	|*	|*	|*	|k	|l	|.
|m	|*	|p	|l	|*	|n	|[43][S]\rarr	|p	|r	|z	|ę	|d	|z	|a	|[][,]{ }	|z	|g	|r	|z	|e	|b	|n	|a	|*	|.
|a	|*	|*	|i	|*	|i	|*	|[44][S]\rarr	|w	|y	|t	|w	|ó	|r	|c	|a	|*	|f	|*	|*	|*	|*	|z	|*	|.
|n	|*	|*	|t	|[45][S]\rarr	|k	|u	|p	|a	|*	|*	|*	|*	|*	|*	|*	|*	|*	|*	|*	|*	|*	|o	|*	|.
|*	|*	|*	|*	|*	|*	|*	|*	|*	|*	|*	|*	|*	|*	|*	|*	|*	|*	|*	|*	|*	|*	|*	|*	|.\end{Puzzle}

\newpage

\begin{PuzzleClues}{\textbf{Poziome}\\}\Clue{1}{}{gipsowy lub woskowy odlew twarzy osoby zmarłej, wykonany w celu utrwalenia wizerunku tejże osoby}
\Clue{2}{}{wilk workowaty, tygrys tasmański, wilk tasmański, fałszywy tygrys, Tassie, Tazzy, Thylacinus cynocephalus - gatunek ssaka z rodziny wilków workowatych, największy drapieżny torbacz czasów współczesnych; pierwotnie występował na terenach Australii i Nowej Gwinei, w czasach historycznych został wyparty wyłącznie do terenów Tasmanii, gdzie wyginął w XX wieku}
\Clue{3}{}{ryba z rodziny dorszowatych występująca w morzach arktycznych}
\Clue{6}{}{o kimś prostomyślnym; to, że ktoś jest prostomyślny; myśli prosto, naiwnie, prostodusznie}
\Clue{7}{}{drażniący bojowy środek trujący przeznaczony do skażania powietrza, który powoduje silne podrażnienie nerwów czuciowych spojówki i rogówki oraz błon śluzowych oczu, czego wynikiem jest silne łzawienie}
\Clue{8}{}{kraje Europy Środkowej i Wschodniej, które po II Wojnie Światowej były pod władzą ZSRR}
\Clue{9}{}{spółgłoska, która powstaje, gdy narządy mowy w czasie artykulacji tworzą dostatecznie wąską szczelinę, by powstał szum, tarcie}
\Clue{10}{}{zamiłowanie do gadżetów - nowych, uchodzących za bajeranckie, szpanerskie przedmiotów, produktów, szczególnie elektronicznych}
\Clue{12}{}{wartość opisująca liczbę nukleonów (czyli protonów i neutronów) w jądrze atomu (nuklidzie) danego izotopu danego pierwiastka}
\Clue{13}{}{jedna z żył tworzących splot wiciowaty, przechodząca przez jądro i pachwinę}
\Clue{16}{}{model opisujący idealny gaz kwantowy nieoddziałujących fermionów}
\Clue{17}{}{element urządzenia drukującego przyjmujący farbę drukową (lub inną nadrukowywaną substancję, np. lakier, klej) w miejscach obrazu drukowego w celu przekazania tej farby dalej, na podłoże drukowe}
\Clue{18}{}{lekceważąco o kobiecie, która nie ma podstawowej wiedzy w jakiejś dziedzinie, o ignorantce}
\Clue{19}{}{wykonany z twardego metalu trzpień zaostrzony z jednej strony, z drugiej zakończony płaskim łbem}
\Clue{21}{}{Phyllitis scolopendrium - zwyczajowa nazwa języcznika zwyczajnego}
\Clue{22}{}{element konstrukcyjny statku żaglowego łączący dwie części masztu}
\Clue{24}{}{zdanie, będące częścią zdania podrzędnie złożonego, które zastępuje rozwija jedną z części zdania zdania nadrzędnego}
\Clue{27}{}{stolica i największe miasto Ukrainy, nad rzeką Dniepr}
\Clue{29}{}{rosyjski kolarz zawodowy, członek grupy Novell}
\Clue{30}{}{w językoznawstwie - wydech; podczas tego procesu są wytwarzane dźwięki mowy}
\Clue{31}{}{miasto w Austrii (Tyrol) w dolinie rzeki Inn}
\Clue{32}{}{departament w Kolumbii, ośrodek administracyjny Meddelin}
\Clue{35}{}{wizyta duszpasterska, odwiedziny księży u parafian w okresie Bożego Narodzenia}
\Clue{36}{}{organizacja faszystowska lub nacjonalistyczna}
\Clue{37}{}{drapieżnik z rodziny psów łowny: cenne futro}
\Clue{38}{}{określenie grubej tkaniny bawełnianej o nienajlepszej jakości}
\Clue{39}{}{inna nazwa dolara nowozelandzkiego}
\Clue{42}{}{rzeźbiarz duński, samouk ur. w 1912 r.; kompozycje w metalu}
\Clue{43}{}{przędza wytwarzana w sposób zgrzebny}
\Clue{44}{}{osoba lub przedsiębiorstwo, które coś wytwarza}
\Clue{45}{}{całość, w której są zgromadzone różne rzeczy, zjawiska materialne lub abstrakcyjne}\end{PuzzleClues}

\begin{PuzzleClues}{\textbf{Pionowe}\\}\Clue{1}{}{gatunek nurkującej kaczki o długości do 50 cm, wszystkożerna, łowna zamieszkuje obszar płn. Eurazji i Ameryki Płn}
\Clue{3}{}{miasto uzdrowiskowe w województwie dolnośląskim, w powiecie lubańskim}
\Clue{4}{}{Eresus - rodzaj pająka z niewielkiej rodziny poskoczowatych}
\Clue{5}{}{przebojowość granicząca z bezczelnością, zuchwałość}
\Clue{9}{}{organ władzy w organizacji}
\Clue{11}{}{kompozytor radziecki (1896-1984); autor 'Hymnu młodzieży demokratycznej świata'; pieśni chóralne, masowe}
\Clue{14}{}{słowo używane w zapisie, czyt. aerobik}
\Clue{15}{}{ślub kobiety; zamążpójście}
\Clue{19}{}{wieś w Polsce położona w województwie zachodniopomorskim, w powiecie drawskim, w gminie Kalisz Pomorski}
\Clue{20}{}{dział psychologii, nauka o biologicznym rozwoju dzieci}
\Clue{21}{}{geograf (1871-1954); twórca nowoczesnej kartografii polskiej}
\Clue{23}{}{sekta głosząca swobodę obyczajów, która ma nie wpływać na możność przystępowania do sakramentów religijnych}
\Clue{24}{}{ptak z rzędu wróblowatych; Afryka, Azja, Australia}
\Clue{25}{}{miasto w płd.-wsch. Ugandzie port nad Jeziorem Wiktorii}
\Clue{26}{}{żołnierz, wojskowy, który podczas obsługi artyleryjskiego środka bojowego odpowiada za wprowadzanie pocisku do komory nabojowej; najczęściej: członek załogi czołgu, którego zadaniem jest ładowanie armaty i karabinu maszynowego}
\Clue{27}{}{ABAKA}
\Clue{28}{}{ostre marynaty warzywne}
\Clue{30}{}{linka lub łańcuch służący do podnoszenia rei}
\Clue{33}{}{łuskoskrzydły owad odżywiający się nektarem kwiatów i sokami roślinnymi; zapyla kwiaty}
\Clue{34}{}{jest to proces polegający na wchodzeniu magmy w skorupę ziemską, bez wychodzenia na jej powierzchnię; wypukła część komory wypełnionej magmą skierowana jest ku dołowi}
\Clue{36}{}{śpiewak operowy (tenor) ur. w 1937 r., solista Teatru Wielkiego w Warszawie}
\Clue{40}{}{kod ISO 4217 franka francuskiego}
\Clue{41}{}{miasto w Japonii położone na płn. od Tokio}\end{PuzzleClues}\newpage\section*{Krzyżówka 189}

\noindent\begin{Puzzle}{22}{20}|*	|*	|*	|*	|*	|[1][S]\darr	|[2][S]\drarr	|l	|v	|l	|*	|[3][S]\drarr	|s	|i	|ł	|o	|w	|n	|i	|a	|*	|[4][S]\darr	|*	|.
|*	|*	|*	|*	|*	|b	|l	|[5][S]\rarr	|p	|r	|e	|c	|y	|p	|i	|t	|a	|c	|j	|a	|*	|e	|*	|.
|*	|*	|*	|*	|[6][S]\darr	|o	|u	|[7][S]\drarr	|p	|r	|z	|e	|c	|i	|w	|n	|i	|k	|*	|*	|*	|g	|*	|.
|*	|*	|*	|[8][S]\darr	|p	|r	|c	|k	|[9][S]\drarr	|r	|e	|f	|e	|r	|e	|n	|d	|a	|r	|z	|*	|e	|*	|.
|*	|*	|[10][S]\drarr	|g	|r	|z	|y	|w	|a	|[][,]{ }	|f	|a	|l	|i	|*	|*	|[11][S]\darr	|*	|*	|[12][S]\darr	|*	|r	|*	|.
|*	|*	|b	|w	|o	|e	|f	|a	|r	|*	|*	|l	|[13][S]\drarr	|s	|z	|p	|r	|y	|c	|k	|a	|*	|*	|.
|*	|[14][S]\darr	|r	|*	|c	|ś	|e	|ś	|a	|*	|*	|o	|k	|*	|*	|*	|y	|[15][S]\darr	|[16][S]\drarr	|w	|i	|j	|*	|.
|*	|o	|y	|*	|e	|l	|r	|n	|n	|*	|[17][S]\rarr	|s	|e	|*	|[18][S]\darr	|[19][S]\darr	|c	|d	|ś	|a	|*	|*	|*	|.
|*	|w	|ł	|*	|s	|a	|a	|i	|ż	|*	|*	|p	|l	|*	|s	|f	|i	|i	|r	|t	|*	|*	|*	|.
|*	|a	|o	|*	|[][,]{ }	|d	|z	|c	|e	|*	|*	|o	|l	|*	|t	|i	|n	|u	|e	|e	|*	|*	|*	|.
|*	|d	|w	|*	|e	|[][,]{ }	|a	|a	|r	|[20][S]\rarr	|f	|r	|e	|g	|a	|t	|a	|*	|d	|r	|*	|*	|*	|.
|*	|o	|a	|*	|o	|c	|*	|*	|*	|*	|*	|y	|r	|*	|n	|n	|*	|*	|n	|k	|*	|*	|*	|.
|*	|p	|t	|[21][S]\rarr	|l	|i	|n	|k	|e	|*	|*	|n	|*	|[22][S]\darr	|*	|e	|*	|*	|i	|a	|[23][S]\darr	|*	|*	|.
|*	|y	|o	|[24][S]\darr	|i	|e	|*	|*	|*	|[25][S]\rarr	|m	|a	|l	|g	|a	|s	|z	|k	|a	|*	|s	|*	|*	|.
|*	|l	|ś	|d	|c	|l	|[26][S]\rarr	|r	|u	|r	|a	|*	|*	|h	|*	|s	|*	|*	|*	|*	|z	|*	|*	|.
|*	|n	|ć	|e	|z	|i	|*	|*	|[27][S]\rarr	|p	|a	|w	|i	|a	|n	|[][,]{ }	|ż	|o	|ł	|t	|y	|*	|*	|.
|*	|o	|*	|c	|n	|s	|*	|*	|*	|*	|*	|*	|*	|s	|[28][S]\rarr	|k	|u	|l	|a	|*	|n	|*	|*	|.
|*	|ś	|*	|h	|y	|t	|*	|*	|*	|*	|*	|[29][S]\rarr	|s	|t	|a	|l	|l	|a	|*	|*	|i	|*	|*	|.
|[30][S]\rarr	|ć	|m	|a	|*	|y	|*	|*	|*	|*	|*	|*	|*	|*	|*	|u	|[31][S]\rarr	|s	|z	|k	|o	|p	|*	|.
|*	|*	|*	|*	|*	|*	|*	|*	|*	|*	|[32][S]\rarr	|t	|a	|r	|a	|b	|a	|n	|*	|*	|n	|*	|*	|.
|*	|*	|[33][S]\rarr	|t	|c	|h	|ó	|r	|z	|l	|i	|w	|o	|ś	|ć	|*	|*	|*	|*	|*	|*	|*	|*	|.\end{Puzzle}

\newpage

\begin{PuzzleClues}{\textbf{Poziome}\\}\Clue{2}{}{kod ISO 4217 łata}
\Clue{3}{}{sala zaopatrzona w specjalistyczny sprzęt sportowy do treningu mięśni}
\Clue{5}{}{wydzielanie z roztworu substancji chemicznej w postaci stałego, bardzo trudno rozpuszczalnego jej związku (tworzącego osad) przez dodanie odpowiedniego odczynnika, lub w wyniku elektrolizy}
\Clue{7}{}{ktoś, kto występuje przeciwko czemuś lub komuś}
\Clue{9}{}{stanowisko urzędnicze w administracji}
\Clue{10}{}{załamująca się część fali, która przyjmuje postać piany}
\Clue{13}{}{zdrobniale: szpryca - rozpryskiwacz}
\Clue{16}{}{stawonóg z podtypu wijów}
\Clue{17}{}{w chemii: symbol selenu}
\Clue{20}{}{dobrze uzbrojony okręt wojenny lub eskortowy}
\Clue{21}{}{pedagog austriacki (1884-1938); twórca tzw. nauczania łącznego}
\Clue{25}{}{rdzenna mieszkanka Madagaskaru}
\Clue{26}{}{kobieta, postrzegana pod względem seksualnym, zwłaszcza rozwiązła, mająca wielu partnerów}
\Clue{27}{}{pawian zielony, pawian masajski, babuin, Papio cynocephalus - ssak z rzędu naczelnych, zamieszkujący tereny południowej, środkowej i wschodniej Afryki}
\Clue{28}{}{przedmiot o kształcie kulistym}
\Clue{29}{}{drewniana lub kamienna ławka ustawiona w prezbiterium; zgodnie z normą tylko w liczbie mnogiej}
\Clue{30}{}{motyl prowadzący nocny tryb życia}
\Clue{31}{}{pogardliwie o Niemcu}
\Clue{32}{}{bęben w kształcie wydłużonego walca, używany w dawnym wojsku}
\Clue{33}{}{cecha człowieka: to, że ktoś ma skłonność do tchórzostwa, brakuje mu odwagi i kieruje się strachem o samego siebie}\end{PuzzleClues}

\begin{PuzzleClues}{\textbf{Pionowe}\\}\Clue{1}{}{Pohlia melanodon - gatunek mchu z rodziny prątnikowatych}
\Clue{2}{}{enzym katalizujący utlenianie lucyferyny}
\Clue{3}{}{półsyntetyczny antybiotyk ß-laktamowy o szerokim spektrum działania bakteriobójczego}
\Clue{4}{}{miasto w płn. Węgrzech, ośrodek administracyjny komitatu Heves, u podnóża Gór Bukowych}
\Clue{6}{}{określenie wszelkiej działalności wiatru na rzeźbę terenu}
\Clue{7}{}{inna nazwa berberysu}
\Clue{8}{}{jednostka mocy równa 1000000000 watów}
\Clue{9}{}{dokonuje aranżacji utworu muzycznego}
\Clue{10}{}{cecha czegoś, co ma strukturę niejednorodną, ma w strukturze bryły}
\Clue{11}{}{odbitka graficzna uzyskana przez odbicie na papierze rysunku wykonanego na płycie graficznej}
\Clue{12}{}{naczynie o pojemności kwaterki (1/4 litra)}
\Clue{13}{}{Helen (1880-1968), pisarka amerykańska, pozbawiona wzroku, słuchu i częściowo mowy; „Historia mego życia”}
\Clue{14}{}{zapylanie kwiatów pyłkiem przenoszonym przez owady}
\Clue{15}{}{jest jednym z dwóch dystryktów wchodzących w skład indyjskiego terytorium związkowego Daman i Diu; zajmuje przybrzeżną wyspę Diu, na której znajduje się jego stolica - miasto Diu - oraz niewielki półwysep Ghoghla}
\Clue{16}{}{statystyka (funkcja) stosowana jako tzw. miara tendencji centralnej, tzn. wskaźnik pokazujący w jakiś sposóbśrodek rozkładu;środek można zdefiniować na wiele sposobów, istnieje też wiele rodzajów średniej}
\Clue{18}{}{sytuacja, całokształt okoliczności}
\Clue{19}{}{klub wyposażony w sprzęt do uprawiania fitnessu}
\Clue{22}{}{nieumarły, stworzenie podobne do ghoula, mające także niektóre cechy ducha}
\Clue{23}{}{wysoki czepek kobiecy z białego płótna, zdobiony na czubku koronkami i wstążkami noszony w Polsce za panowania Augusta II}
\Clue{24}{}{dziewczyna lub kobieta o bardzo małych piersiach}\end{PuzzleClues}\newpage\section*{Krzyżówka 190}

\noindent\begin{Puzzle}{21}{33}|*	|*	|*	|*	|*	|*	|*	|[1][S]\drarr	|e	|l	|e	|k	|t	|r	|o	|ś	|m	|i	|e	|c	|i	|*	|.
|*	|*	|*	|[2][S]\darr	|*	|*	|[3][S]\rarr	|t	|a	|n	|i	|e	|c	|*	|*	|*	|*	|*	|*	|*	|[4][S]\darr	|[5][S]\darr	|.
|*	|*	|*	|c	|*	|*	|*	|o	|*	|*	|*	|*	|*	|*	|*	|[6][S]\drarr	|p	|ę	|d	|*	|g	|ż	|.
|*	|*	|*	|h	|[7][S]\rarr	|p	|a	|p	|a	|d	|o	|p	|o	|u	|l	|o	|s	|*	|*	|*	|r	|y	|.
|*	|[8][S]\rarr	|t	|o	|n	|a	|ż	|*	|[9][S]\rarr	|f	|r	|a	|n	|k	|i	|s	|t	|a	|*	|*	|z	|d	|.
|*	|[10][S]\rarr	|p	|r	|a	|w	|o	|s	|k	|r	|ę	|t	|*	|[11][S]\rarr	|h	|a	|c	|k	|n	|e	|y	|*	|.
|*	|*	|*	|o	|*	|[12][S]\darr	|*	|*	|*	|*	|*	|[13][S]\darr	|[14][S]\rarr	|p	|o	|d	|l	|a	|s	|*	|b	|*	|.
|*	|*	|[15][S]\darr	|b	|*	|m	|*	|*	|*	|*	|*	|ś	|*	|[16][S]\darr	|*	|n	|*	|*	|*	|[17][S]\darr	|[][,]{ }	|*	|.
|*	|[18][S]\drarr	|p	|a	|l	|i	|s	|a	|n	|d	|e	|r	|*	|s	|*	|i	|*	|*	|*	|w	|p	|*	|.
|*	|g	|i	|[][,]{ }	|*	|e	|[19][S]\darr	|*	|*	|*	|[20][S]\rarr	|o	|r	|o	|z	|c	|o	|*	|[21][S]\darr	|o	|r	|*	|.
|*	|a	|e	|d	|[22][S]\darr	|d	|d	|*	|[23][S]\darr	|*	|*	|d	|*	|s	|*	|z	|[24][S]\darr	|[25][S]\darr	|c	|k	|a	|*	|.
|*	|z	|r	|y	|c	|z	|z	|[26][S]\drarr	|d	|o	|p	|e	|ł	|n	|i	|e	|n	|i	|e	|*	|w	|*	|.
|*	|[][,]{ }	|d	|p	|z	|i	|i	|n	|i	|*	|*	|k	|*	|a	|[27][S]\darr	|k	|a	|l	|d	|*	|d	|*	|.
|[28][S]\drarr	|k	|o	|l	|o	|k	|w	|i	|n	|t	|a	|*	|*	|*	|p	|[][,]{ }	|r	|d	|r	|*	|z	|*	|.
|k	|o	|ł	|o	|s	|[][,]{ }	|n	|e	|g	|*	|*	|[29][S]\darr	|*	|*	|o	|g	|z	|e	|[][,]{ }	|*	|i	|*	|.
|o	|p	|a	|m	|n	|o	|y	|m	|*	|*	|[30][S]\darr	|p	|[31][S]\darr	|*	|c	|o	|ą	|f	|c	|*	|w	|*	|.
|m	|a	|*	|a	|e	|b	|[][,]{ }	|o	|*	|*	|ł	|o	|j	|[32][S]\darr	|i	|ł	|d	|r	|z	|*	|y	|*	|.
|p	|l	|*	|t	|k	|r	|a	|ż	|*	|*	|y	|z	|e	|ż	|s	|y	|[][,]{ }	|a	|e	|*	|*	|*	|.
|u	|n	|*	|y	|[][,]{ }	|z	|t	|e	|*	|[33][S]\drarr	|ż	|y	|l	|a	|k	|*	|k	|n	|r	|*	|*	|*	|.
|t	|i	|[34][S]\darr	|c	|n	|e	|r	|b	|*	|z	|k	|c	|e	|b	|[][,]{ }	|*	|r	|s	|w	|*	|*	|*	|.
|e	|a	|p	|z	|e	|ż	|a	|n	|*	|w	|a	|j	|ń	|y	|o	|*	|y	|y	|o	|*	|*	|*	|.
|r	|n	|i	|n	|a	|o	|k	|o	|*	|i	|*	|a	|[][,]{ }	|[][,]{ }	|d	|*	|t	|*	|n	|*	|*	|*	|.
|[][,]{ }	|y	|n	|a	|p	|n	|t	|ś	|[35][S]\darr	|n	|*	|[][,]{ }	|p	|p	|r	|*	|y	|*	|y	|*	|*	|*	|.
|a	|*	|g	|*	|o	|y	|o	|ć	|l	|g	|*	|r	|a	|o	|z	|*	|c	|*	|*	|*	|*	|*	|.
|n	|*	|w	|*	|l	|*	|r	|*	|i	|l	|*	|y	|m	|ł	|u	|*	|z	|*	|*	|*	|*	|*	|.
|a	|*	|i	|*	|i	|*	|*	|[36][S]\darr	|s	|i	|*	|g	|p	|u	|t	|*	|n	|*	|*	|*	|*	|*	|.
|l	|*	|n	|*	|t	|*	|*	|n	|e	|a	|*	|l	|a	|d	|o	|*	|y	|*	|*	|*	|*	|*	|.
|o	|*	|[][,]{ }	|*	|a	|*	|*	|a	|n	|n	|*	|o	|s	|n	|w	|*	|*	|*	|*	|*	|*	|*	|.
|g	|*	|p	|*	|ń	|*	|*	|f	|t	|i	|*	|w	|o	|i	|y	|*	|*	|*	|*	|*	|*	|*	|.
|o	|*	|a	|*	|s	|*	|*	|t	|e	|z	|*	|a	|w	|o	|*	|*	|*	|*	|*	|*	|*	|*	|.
|w	|*	|p	|[37][S]\rarr	|k	|a	|w	|a	|*	|m	|*	|*	|y	|w	|*	|*	|*	|*	|*	|*	|*	|*	|.
|y	|[38][S]\rarr	|u	|p	|i	|e	|k	|*	|*	|*	|*	|*	|*	|e	|*	|*	|*	|*	|*	|*	|*	|*	|.
|*	|*	|a	|*	|*	|*	|*	|*	|*	|*	|*	|*	|*	|*	|*	|*	|*	|*	|*	|*	|*	|*	|.
|*	|*	|*	|*	|*	|*	|*	|*	|*	|*	|*	|*	|*	|*	|*	|*	|*	|*	|*	|*	|*	|*	|.\end{Puzzle}

\newpage

\begin{PuzzleClues}{\textbf{Poziome}\\}\Clue{1}{}{odpady, w skład których wchodzi zużyty sprzęt elektroniczny albo jego części}
\Clue{3}{}{rytmiczne ruchy ciała zazwyczaj w takt muzyki; głęboko zakorzenione w kulturze zachowanie o charakterze rozrywkowym, artystycznym lub rytualnym - taniec w tym znaczeniu jest konceptualizowany na wiele sposobów}
\Clue{6}{}{bardzo szybki ruch}
\Clue{7}{}{ur. w 1904 r., kompozytor grecki, od 1927 r.  w Paryżu; utwory orkiestrowe. kameralne, fortepianowe}
\Clue{8}{}{wielkość statku, okrętu, grupy statków lub całej floty podana w jednostkach objętościowych, określanych jako tony rejestrowe}
\Clue{9}{}{zwolennik generała Francisco Franco, przywódcy antyrepublikańskiego przewrotu (1936-1939) w Hiszpanii}
\Clue{10}{}{miejsce na jezdni, gdzie się skręca w prawo, pas do wykonywania prawoskrętu}
\Clue{11}{}{kuc Hackney - rasa prawdziwych kuców z typowym dla nich charakterem, dzieło hodowcy Christophera Wilsona z Kirkby Lonsdale w Kumbrii; dzięki ciekawemu i bardzo charakterystycznemu sposobowi ruchu można je obecnie często spotkać na różnego rodzaju pokazach}
\Clue{14}{}{biegacz, członek 'srebrnej' sztafety 4x400 z Montrealu}
\Clue{18}{}{drewno pozyskiwane z drzew tropikalnych z rodzaju Dalbergia}
\Clue{20}{}{meksykański malarz i grafik (1883-1949) twórczość związana z tradycjami sztuki prekolumbijskiej i ludowej}
\Clue{26}{}{część zdania oznaczająca pasywny przedmiot czynności wyrażonej orzeczeniem zdania w stronie czynnej}
\Clue{28}{}{płożąca się roślina zielna z rodziny dyniowatych, żółte owoce wielkości pomarańczy silnie przeczyszczające}
\Clue{33}{}{Phlebia - rodzaj grzybów z rodziny strocznikowatych; grzyby o rozpostartym owocniku i hymenoforze zazwyczaj pomarszczonym i promieniście żyłkowanym}
\Clue{37}{}{nasiona kawowca, które (po obróbce: paleniu, suszeniu, mieleniu) są wykorzystywane do przygotowywania parzonego napoju - kawy}
\Clue{38}{}{wytwór pieczenia; tyle, ile zostało upieczone}\end{PuzzleClues}

\begin{PuzzleClues}{\textbf{Pionowe}\\}\Clue{1}{}{elita, grupa osób szalenie popularnych, o wysokiej pozycji społecznej i z wieloma sukcesami na koncie}
\Clue{2}{}{choroba symulowana  w celu uniknięcia wypełniania obowiązków}
\Clue{4}{}{Boletus edulis Bull. - gatunek grzybów z rodziny borowikowatych, jeden z najbardziej poszukiwanych grzybów jadalnych, wysoko ceniony ze względu na walory smakowe}
\Clue{5}{}{osoba chciwa, skąpa}
\Clue{6}{}{Discelium nudum - gatunek mchu z rodziny osadniczkowatych}
\Clue{12}{}{Hasemania nana - endemiczny gatunek słodkowodnej ryby z rodziny kąsaczowatych (Characidae)}
\Clue{13}{}{wnętrze czegoś, co, co nie na zewnątrz}
\Clue{15}{}{coś mało ważnego}
\Clue{16}{}{Pinus - rodzaj roślin z rodziny sosnowatych (Pinaceae Lindl.) obejmujący niemal 115 gatunków drzew i krzewów; występuje przeważnie w strefie klimatu umiarkowanego półkuli północnej, choć niektóre gatunki rosną również w strefach cieplejszych (zwykle w górach)}
\Clue{17}{}{rodzaj patelni o wysokich brzegach; wykorzystywany m.in. w kuchni chińskiej, wietnamskiej, tajskiej}
\Clue{18}{}{mieszanina gazów, której głównym składnikiem jest metan, obecna w kopalniach i jaskiniach}
\Clue{19}{}{atraktor, który jest fraktalem}
\Clue{21}{}{drewno pozyskiwane z czerwonego cedru}
\Clue{22}{}{Allium neapolitanum - gatunek rośliny z rodziny amarylkowatych}
\Clue{23}{}{rzecz, która ma określone zastsowanie w danej sytuacji, jest w tej sytuacji potrzebna}
\Clue{24}{}{narząd szczególnie wrażliwy na promieniowanie danego rodzaju, zewnętrzne lub wewnętrzne w stosunku do organizmu}
\Clue{25}{}{francuska rasa owiec hodowanych głównie dla mięsa}
\Clue{26}{}{niemożliwość, zauważalny brak działania lub działanie w jakiś (określony) sposób niepełne, cecha czegoś, co jest niemożebne - niemożliwe}
\Clue{27}{}{pocisk napędzany silnikiem odrzutowym, przeznaczony do niszczenia celów za pomocą przenoszonego ładunku bojowego}
\Clue{28}{}{komputer przetwarzający sygnał ciągły (analogowy) przeważnie elektryczny}
\Clue{29}{}{pozycja obronna rozbudowana między pozycjami zasadniczymi skośnie do przedniego skraju obrony, ma za zadanie rozcinać lub kanalizować ruch nacierającego nieprzyjaciela}
\Clue{30}{}{czerpadło; narzędzie wiertnicze w postaci rury z zaworem przeznaczone do usuwania zwiercin z otworu wiertniczego}
\Clue{31}{}{Ozotoceros bezoarticus - średniej wielkości ssak parzystokopytny z rodziny jeleniowatych, jedyny przedstawiciel rodzaju Ozotoceros; zamieszkuje otwarte i suche tereny trawiaste Ameryki Południowej}
\Clue{32}{}{świstkowate, Leptodactylidae - rodzina płazów z rzędu płazów bezogonowych obejmująca 12 rodzajów; płazy te żyją w Ameryce Południowej, Środkowej, na południu Ameryki Północnej i na wyspach Morza Karaibskiego. }
\Clue{33}{}{nauka teologii protestanckiej odrzucająca kult świętych, rozumienie Eucharystii jako ofiary, celibat, zakony i władzę papieża}
\Clue{34}{}{pingwin białobrewy, Pygoscelis papua - gatunek dużego nielotnego ptaka wodnego z rodziny pingwinów (Spheniscidae), zamieszkującego chłodne oceany półkuli południowej; gnieździ się na Półwyspie Antarktycznym, Falklandach, Południowej Georgii, Wyspach Kerguelena, Wyspach Heard i McDonalda, Orkadach Południowych, Macquarie, Wyspach Crozeta, Wyspach Księcia Edwarda i na Sandwichu Południowym}
\Clue{35}{}{jednostka zdawkowa w Lesotho; 1/100 loti}
\Clue{36}{}{ciekła frakcja ropy naftowej, wrząca w przedziale temperatur 170-250 stopni Celsjusza}\end{PuzzleClues}\newpage\section*{Krzyżówka 191}

\noindent\begin{Puzzle}{19}{27}|*	|*	|*	|*	|*	|*	|*	|*	|*	|*	|*	|*	|*	|[1][S]\drarr	|k	|u	|m	|a	|k	|*	|.
|*	|*	|*	|*	|*	|*	|*	|*	|*	|*	|*	|*	|*	|g	|*	|*	|*	|[2][S]\darr	|[3][S]\darr	|*	|.
|*	|*	|*	|*	|*	|*	|*	|*	|*	|*	|*	|*	|*	|r	|*	|[4][S]\darr	|*	|z	|m	|*	|.
|*	|*	|*	|[5][S]\darr	|*	|*	|*	|*	|[6][S]\darr	|*	|*	|*	|*	|z	|[7][S]\darr	|o	|[8][S]\darr	|a	|i	|*	|.
|*	|[9][S]\darr	|*	|t	|*	|*	|*	|*	|t	|*	|[10][S]\drarr	|c	|*	|e	|t	|w	|g	|p	|s	|*	|.
|*	|b	|*	|u	|*	|*	|*	|*	|e	|*	|m	|*	|*	|c	|e	|e	|r	|a	|o	|*	|.
|*	|i	|*	|r	|*	|*	|*	|*	|r	|[11][S]\darr	|e	|[12][S]\rarr	|c	|h	|o	|r	|a	|ł	|*	|*	|.
|*	|l	|*	|b	|*	|[13][S]\darr	|*	|*	|m	|m	|c	|*	|*	|*	|r	|o	|f	|*	|*	|*	|.
|*	|a	|*	|i	|*	|b	|*	|*	|o	|a	|h	|[14][S]\darr	|[15][S]\rarr	|w	|i	|l	|i	|a	|*	|*	|.
|*	|r	|*	|n	|*	|o	|*	|[16][S]\drarr	|g	|r	|a	|s	|i	|c	|a	|*	|t	|*	|*	|*	|.
|*	|d	|[17][S]\darr	|a	|[18][S]\darr	|k	|*	|c	|r	|k	|n	|ł	|*	|[19][S]\darr	|[][,]{ }	|*	|y	|*	|*	|*	|.
|*	|[][,]{ }	|n	|[][,]{ }	|d	|o	|[20][S]\darr	|z	|a	|i	|i	|o	|*	|a	|w	|*	|z	|*	|[21][S]\darr	|*	|.
|*	|a	|a	|w	|i	|w	|k	|ł	|f	|z	|k	|w	|*	|l	|z	|*	|a	|[22][S]\darr	|b	|*	|.
|*	|r	|n	|o	|e	|i	|a	|a	|*	|e	|a	|e	|*	|i	|g	|*	|c	|m	|r	|*	|.
|*	|t	|o	|d	|s	|s	|p	|p	|*	|t	|[][,]{ }	|n	|*	|k	|l	|*	|j	|o	|o	|*	|.
|*	|y	|c	|n	|e	|k	|i	|a	|*	|a	|m	|k	|[23][S]\darr	|a	|ę	|*	|a	|s	|ń	|[24][S]\darr	|.
|[25][S]\drarr	|s	|z	|a	|l	|o	|t	|k	|a	|*	|o	|a	|p	|n	|d	|*	|*	|z	|[][,]{ }	|ł	|.
|m	|t	|ą	|*	|*	|*	|a	|*	|*	|*	|l	|*	|r	|t	|n	|*	|*	|c	|ś	|ą	|.
|o	|y	|s	|*	|[26][S]\drarr	|p	|ł	|a	|t	|n	|e	|r	|z	|*	|o	|[27][S]\darr	|[28][S]\darr	|z	|r	|c	|.
|n	|c	|t	|*	|h	|*	|*	|*	|*	|*	|k	|*	|e	|*	|ś	|s	|a	|[][,]{ }	|u	|z	|.
|o	|z	|k	|*	|y	|*	|*	|*	|*	|*	|u	|*	|b	|*	|c	|p	|r	|w	|t	|n	|.
|p	|n	|a	|[29][S]\darr	|d	|*	|*	|*	|*	|*	|l	|*	|i	|*	|i	|o	|a	|i	|o	|i	|.
|t	|y	|*	|w	|r	|*	|[30][S]\rarr	|k	|o	|p	|a	|*	|e	|[31][S]\darr	|*	|c	|r	|n	|w	|k	|.
|e	|*	|*	|a	|a	|[32][S]\rarr	|a	|m	|p	|e	|r	|o	|g	|o	|d	|z	|i	|n	|a	|*	|.
|r	|*	|*	|p	|z	|[33][S]\rarr	|b	|u	|s	|o	|n	|i	|*	|*	|*	|n	|*	|y	|*	|*	|.
|e	|*	|*	|n	|y	|*	|*	|*	|[34][S]\rarr	|z	|a	|c	|i	|e	|n	|i	|e	|*	|*	|*	|.
|s	|*	|[35][S]\rarr	|o	|d	|l	|o	|t	|*	|*	|*	|[36][S]\rarr	|s	|a	|m	|k	|a	|*	|*	|*	|.
|*	|*	|*	|*	|*	|[37][S]\rarr	|t	|o	|p	|i	|e	|l	|i	|c	|a	|*	|*	|*	|*	|*	|.\end{Puzzle}

\newpage

\begin{PuzzleClues}{\textbf{Poziome}\\}\Clue{1}{}{płaz bezogonowych z rodziny ropuszek o ciemnym grzbiecie, chroniona}
\Clue{10}{}{symbol stopni Celsjusza}
\Clue{12}{}{uroczysty śpiew kościelny}
\Clue{15}{}{miasto w Grecji (Attyka)}
\Clue{16}{}{gruczoł dokrewny kręgowców}
\Clue{25}{}{rodzaj małej cebuli, warzywa, która jest ceniona w kulinariach dzięki swojemu dość delikatnemu, lekko czosnkowemu smakowi; cebulka rośliny nazywanej tak samo}
\Clue{26}{}{rzemieślnik wyrabiający zbroje, tarcze i hełmy a także białą broń}
\Clue{30}{}{60 sztuk}
\Clue{32}{}{ładunek elektryczny przepływający w czasie 1 godziny przez przekrój poprzeczny przewodnika, gdy natężenie prądu elektrycznego płynącego przez tę powierzchnię wynosi 1 amper}
\Clue{33}{}{włoski kompozytor i pianista (1866-1924); propagator neoklasycyzmu; utwory fortepianowe, orkiestrowe, opery, pieśni}
\Clue{34}{}{miejsce zacienione}
\Clue{35}{}{ruch ptaków}
\Clue{36}{}{pogardliwie o kobiecie}
\Clue{37}{}{zmarła, która się utopiła}\end{PuzzleClues}

\begin{PuzzleClues}{\textbf{Pionowe}\\}\Clue{1}{}{zły czyn}
\Clue{2}{}{chęć do działania}
\Clue{3}{}{tradycyjna i bardzo pożywna gęsta pasta japońska, produkowana ze sfermentowanej soi, najczęściej z dodatkiem ryżu lub jęczmienia, soli oraz drożdży}
\Clue{4}{}{kombinezon ochronny używany przez pracowników technicznych i sportowców; także strój damski mający ten fason}
\Clue{5}{}{wirnikowy silnik wodny przetwarzający energię mechaniczną płynącej wody}
\Clue{6}{}{przyrząd mierzący temperaturę i rejestrujący jej przebieg w funkcji czasu}
\Clue{7}{}{teoria stworzona przez Alberta Einsteina celem usunięcia niezgodności między zasadą względności Galileusza (głoszącą, że prędkość jest względna) a prawami elektrodynamiki}
\Clue{8}{}{wyżarzanie grafityzujące; stosuje się je w stosunku do żeliwa białego w celu uzyskania żeliwa ciągliwego}
\Clue{9}{}{rodzaj bilardu polegający na pokazie trików bilardowych}
\Clue{10}{}{metoda chemii obliczeniowej wykorzystującą mechanikę klasyczną do modelowania układów molekularnych}
\Clue{11}{}{przezroczysta, cienka i lekka, dość ekskluzywna tkanina bawełniana, używana głównie na firanki i zasłony}
\Clue{13}{}{wybrzeże, zbocze}
\Clue{14}{}{mieszkanka Słowenii, kobieta pochodzenia słoweńskiego}
\Clue{16}{}{człowiek, który wolno się porusza}
\Clue{17}{}{cząstka o rozmiarach nanometrycznych}
\Clue{18}{}{konstruktor niemiecki (1858-1913); pierwszy zbudował wysokoprężny silnik spalinowy z samoczynnym zapłonem}
\Clue{19}{}{dawne określenie czerwonego wina produkowanego w Hiszpanii}
\Clue{20}{}{każdy zasób, który może być przeliczony na coś innego (np. na wartość pieniężną, społeczną, emocjonalną)}
\Clue{21}{}{rodzaj broni gładkolufowej powstałej w XVIII wieku}
\Clue{22}{}{młode wino, o niskiej zawartości alkoholu}
\Clue{23}{}{praca wykonana przez maszynę}
\Clue{24}{}{zwornik}
\Clue{25}{}{w architekturze antycznej okrągła budowla bez ścian, której dach wspiera się na kolumnach}
\Clue{26}{}{organiczny związek chemiczny, otrzymywany w reakcji kwasów karboksylowych i ich pochodnych z hydrazyną}
\Clue{27}{}{podest; element poziomy schodów}
\Clue{28}{}{jezioro w Brazylii}
\Clue{29}{}{spoiwo mineralne powietrzne (suchowiążące)}
\Clue{31}{}{w chemii: symbol tlenu}\end{PuzzleClues}\newpage\section*{Krzyżówka 192}

\noindent\begin{Puzzle}{21}{31}|*	|*	|*	|*	|*	|*	|*	|*	|*	|*	|*	|*	|*	|*	|*	|*	|*	|*	|*	|*	|[1][S]\darr	|*	|.
|*	|*	|*	|*	|*	|*	|*	|*	|[2][S]\drarr	|b	|l	|o	|k	|*	|*	|[3][S]\drarr	|c	|y	|c	|e	|k	|*	|.
|*	|[4][S]\drarr	|n	|p	|r	|*	|*	|[5][S]\drarr	|h	|a	|i	|t	|i	|*	|[6][S]\rarr	|w	|o	|l	|s	|k	|i	|*	|.
|*	|b	|[7][S]\rarr	|s	|t	|a	|d	|n	|i	|a	|k	|i	|*	|*	|*	|y	|[8][S]\drarr	|a	|b	|e	|l	|*	|.
|*	|r	|[9][S]\darr	|*	|*	|[10][S]\drarr	|k	|a	|d	|ź	|*	|*	|*	|[11][S]\drarr	|m	|s	|z	|a	|ł	|*	|o	|*	|.
|*	|a	|t	|[12][S]\darr	|[13][S]\darr	|k	|*	|g	|ż	|[14][S]\rarr	|h	|u	|l	|l	|*	|t	|b	|*	|*	|*	|f	|*	|.
|[15][S]\drarr	|m	|u	|c	|h	|o	|m	|o	|r	|[][,]{ }	|m	|g	|l	|e	|j	|a	|r	|k	|a	|*	|*	|[16][S]\darr	|.
|z	|a	|s	|h	|i	|l	|*	|ś	|a	|*	|*	|*	|*	|h	|*	|w	|o	|[17][S]\darr	|[18][S]\darr	|[19][S]\darr	|*	|a	|.
|m	|[][,]{ }	|z	|ł	|l	|i	|*	|ć	|*	|*	|*	|*	|*	|t	|*	|a	|j	|p	|o	|b	|*	|e	|.
|ę	|t	|ó	|o	|d	|m	|*	|*	|*	|[20][S]\rarr	|i	|l	|j	|i	|n	|*	|n	|y	|w	|i	|[21][S]\darr	|r	|.
|c	|r	|w	|d	|e	|a	|[22][S]\rarr	|c	|o	|l	|t	|r	|a	|n	|e	|*	|i	|t	|u	|a	|k	|o	|.
|z	|i	|[][,]{ }	|z	|n	|c	|*	|*	|*	|*	|*	|*	|[23][S]\darr	|e	|*	|*	|k	|o	|l	|ł	|ó	|p	|.
|e	|u	|n	|i	|b	|j	|*	|*	|*	|*	|*	|[24][S]\darr	|ł	|n	|*	|*	|*	|n	|a	|a	|ł	|o	|.
|n	|m	|a	|a	|r	|a	|[25][S]\rarr	|d	|z	|i	|u	|r	|a	|*	|*	|*	|*	|[][,]{ }	|c	|[][,]{ }	|e	|r	|.
|i	|f	|r	|r	|a	|*	|*	|*	|*	|*	|*	|e	|z	|*	|*	|*	|*	|s	|j	|a	|c	|t	|.
|e	|a	|o	|k	|n	|*	|*	|*	|*	|[26][S]\rarr	|d	|z	|i	|e	|n	|n	|i	|k	|a	|r	|z	|*	|.
|[][,]{ }	|l	|d	|a	|d	|*	|[27][S]\rarr	|ś	|r	|u	|b	|o	|k	|r	|ę	|t	|*	|a	|*	|m	|k	|*	|.
|i	|n	|o	|[][,]{ }	|i	|[28][S]\rarr	|g	|a	|l	|t	|o	|n	|*	|*	|*	|*	|*	|l	|[29][S]\darr	|i	|o	|*	|.
|n	|a	|w	|a	|a	|*	|*	|[30][S]\rarr	|l	|i	|m	|a	|n	|o	|w	|i	|a	|n	|k	|a	|*	|*	|.
|d	|*	|y	|b	|[][,]{ }	|[31][S]\darr	|*	|*	|[32][S]\rarr	|p	|a	|t	|o	|k	|a	|*	|*	|y	|o	|*	|[33][S]\darr	|*	|.
|u	|*	|*	|s	|r	|r	|[34][S]\rarr	|z	|a	|c	|h	|o	|w	|a	|n	|i	|e	|*	|l	|*	|h	|[35][S]\darr	|.
|s	|*	|[36][S]\rarr	|o	|z	|ó	|r	|*	|*	|[37][S]\drarr	|p	|r	|z	|e	|p	|i	|s	|*	|o	|*	|u	|g	|.
|t	|*	|[38][S]\rarr	|r	|e	|w	|i	|a	|*	|s	|*	|[][,]{ }	|*	|*	|*	|*	|*	|*	|*	|*	|f	|m	|.
|r	|*	|*	|p	|c	|n	|*	|*	|*	|t	|[39][S]\rarr	|k	|o	|p	|r	|o	|w	|i	|n	|a	|*	|t	|.
|i	|*	|*	|c	|z	|i	|*	|[40][S]\rarr	|d	|o	|ś	|w	|i	|a	|d	|c	|z	|e	|n	|i	|e	|*	|.
|a	|*	|*	|y	|n	|k	|*	|*	|*	|p	|[41][S]\rarr	|a	|k	|a	|t	|y	|z	|j	|a	|*	|*	|*	|.
|l	|*	|*	|j	|a	|*	|[42][S]\rarr	|i	|n	|k	|o	|r	|p	|o	|r	|a	|c	|j	|a	|*	|*	|*	|.
|n	|*	|*	|n	|*	|*	|*	|*	|*	|a	|[43][S]\rarr	|c	|a	|u	|s	|a	|*	|*	|*	|*	|*	|*	|.
|e	|[44][S]\rarr	|g	|a	|s	|k	|e	|l	|l	|*	|[45][S]\rarr	|o	|b	|o	|d	|r	|y	|t	|*	|*	|*	|*	|.
|*	|*	|*	|*	|[46][S]\rarr	|z	|ł	|o	|t	|a	|[][,]{ }	|w	|o	|l	|n	|o	|ś	|ć	|*	|*	|*	|*	|.
|*	|*	|[47][S]\rarr	|k	|a	|w	|a	|l	|e	|r	|z	|y	|s	|t	|a	|*	|*	|*	|*	|*	|*	|*	|.
|*	|*	|*	|*	|[48][S]\rarr	|z	|e	|s	|p	|ó	|ł	|*	|*	|*	|*	|*	|*	|*	|*	|*	|*	|*	|.\end{Puzzle}

\newpage

\begin{PuzzleClues}{\textbf{Poziome}\\}\Clue{2}{}{blok startowy}
\Clue{3}{}{pierś - parzysty organ, który mają kobiety}
\Clue{4}{}{kod ISO 4217 rupii nepalskiej}
\Clue{5}{}{państwo w Ameryce Środkowej, zajmujące część wyspy Haiti na Morzu Karaibskim w archipelagu Wielkich Antyli i graniczące z Republiką Dominikany}
\Clue{6}{}{Włodzimierz, poeta, członek Cyganerii Warszawskiej (1824-82), „Halka”, „Domek przy ulicy Głębokiej”, libretta do oper „Halka” i „Hrabina”}
\Clue{7}{}{Pomatostomidae - rodzina ptaków z rzędu wróblowych (Passeriformes), do której należy 5 gatunków; występują w Australii i Nowej Gwinei}
\Clue{8}{}{matematyk norweski (1802-29); współtwórca funkcji eliptycznych i hiperbolicznych}
\Clue{10}{}{duże naczynie gospodarskie w kształcie ściętego stożka}
\Clue{11}{}{księga liturgiczna w Kościele katolickim zawierająca tekst mszy i przepisy ich odprawiania}
\Clue{14}{}{miasto w Kanadzie (Quebec) w zespole miejskim Ottawy; ośr. przemysłu drzewnego}
\Clue{15}{}{Amanita fulva -  gatunek grzybów należący do rodziny muchomorowatych}
\Clue{20}{}{Marszak - pisarz rosyjski (1896-1953), opowiadania i powieści dla młodzieży; „Czarno na białym”, „Jak człowiek stał się olbrzymem”}
\Clue{22}{}{amerykański saksofonista i kompozytor jazzowy (1926-1967); przedstawiciel hard popu}
\Clue{25}{}{zwykle mała miejscowość, która jest postrzegana jako niezbyt interesująca i zapóźniona cywilizacyjnie}
\Clue{26}{}{pracownik redakcji, który zajmuje się zbieraniem informacji, pisaniem artykułów, przeprowadzaniem wywiadów lub prezentowaniem wiadomości}
\Clue{27}{}{screwdriver - popularny koktajl alkoholowy bazujący na niedrogich składnikach: wódce i soku pomarańczowym}
\Clue{28}{}{angielski przewodnik i antropolog (1822-1911 ); twórca zasad eugeniki, stworzył podstawy daktyloskopii}
\Clue{30}{}{mieszkanka Limanowej}
\Clue{32}{}{miód pszczeli nieskrystalizowany}
\Clue{34}{}{podporządkownie się nakazom, wymogom}
\Clue{36}{}{cielęcy w galarecie}
\Clue{37}{}{wskazówka, zwykle pisana, jak coś zrobić; zwłaszcza przepis na wykonanie potrawy}
\Clue{38}{}{pokaz mody, prezentacja modelek i modeli w nowych strojach wymyślonych przez projektanta}
\Clue{39}{}{drobny miedziany pieniążek (lub jakaś ilość tych pieniędzy)}
\Clue{40}{}{wiedza w jakiejś dziedzinie zdobyta na drodze samodzielnych działań; praktyka}
\Clue{41}{}{zespół objawów polegający na występowaniu: pobudzenia ruchowego, lęku, rozdrażnienia, trudnego do zniesienia niepokoju}
\Clue{42}{}{zjawisko polegające na tworzeniu wyrazów złożonych zawierających orzeczenie i dopełnienie lub okolicznik}
\Clue{43}{}{przyczyna, pozwalająca wszcząć czynności prawne}
\Clue{44}{}{pisarka angielska (1810-65), szkice i powieści obyczajowo-społeczne, „Panie z Cranford”}
\Clue{45}{}{przedstawiciel słowiańskiego plemienia Obodrytów}
\Clue{46}{}{swobody, prawa i przywileje, przysługujące szlachcie w Rzeczypospolitej Obojga Narodów}
\Clue{47}{}{żołnierz służący w kawalerii - oddziale wojska walczącego konno}
\Clue{48}{}{zbiór objawów lub zaburzeń związanych z jakąś chorobą}\end{PuzzleClues}

\begin{PuzzleClues}{\textbf{Pionowe}\\}\Clue{1}{}{narzędzie w postaci zakrzywionego ostrza stalowego osadzonego na drewnianym stylisku}
\Clue{2}{}{ucieczka Mahometa z Mekki do Medyny w 622 r. uznana za początek ery muzułmańskiej}
\Clue{3}{}{zbiór przedmiotów przeznaczonych do zaprezentowania publiczności}
\Clue{4}{}{budowla w kształcie łuku mająca na celu upamiętnienie osoby lub wydarzenia}
\Clue{5}{}{ludzkie genitalia - męskie lub żeńskie, zwłaszcza nieubrane (co czasem może ich właściciela krępować)}
\Clue{8}{}{podłużna płytka wykonywana z żelaza, brązu, skóry lub rogu, stanowiąca łuskę zbroi lamelkowej}
\Clue{9}{}{wieś w Polsce położona w województwie podkarpackim, w powiecie mieleckim, w gminie Tuszów Narodowy}
\Clue{10}{}{zgodność linii prowadzącej od obserwatora do obiektu obserwowanego oraz osi przyrządu pomiarowego}
\Clue{11}{}{długodystansowiec fiński, mistrz i wicemistrz olimpijski z Los Angeles (1932) i Berlina}
\Clue{12}{}{chłodziarka, w której czynnik chłodzący najpierw jest pochłaniany przez absorber, potem wydziela się z roztworu}
\Clue{13}{}{Hildenbrandia rivularis - gatunek słodkowodnego krasnorostu tworzącego skorupiaste plechy na kamieniach zanurzonych w wodzie}
\Clue{15}{}{rodzaj zmęczenia, polegający na szkodliwym oddziaływaniu techniki na człowieka, jego sferę psychiczną i fizyczną}
\Clue{16}{}{przestarzale port lotniczy}
\Clue{17}{}{Python sebae - gatunek gada z rodziny pytonów, podrzędu węży, występujący w całej Afryce Środkowej i Południowej na południe od Sahary}
\Clue{18}{}{jeden z elementów cyklu miesiączkowego (efektem owulacji jest uwolnienie gamet żeńskich czyli komórek jajowych)}
\Clue{19}{}{zbiorcze określenie ruchów politycznych i sił zbrojnych działających podczas wojny domowej w Rosji w latach 1917-1923, walczących z komunistami (bolszewikami) i ruchami ich wspierającymi}
\Clue{21}{}{małe kółko w mechanizmie}
\Clue{23}{}{pojazd i urządzenie badawcze jednocześnie, stosowany do eksploracji kosmosu}
\Clue{24}{}{element elektroniczny, którego zasada działania oparta jest na zjawisku piezoelektrycznym w krysztale kwarcu}
\Clue{29}{}{jakiś mężczyzna lub chłopak, zwłaszcza obcy lub taki, względem którego zachowujemy dystans; gościu, koleś}
\Clue{31}{}{(niebieski) wielkie koło będące przecięciem sfery niebieskiej płaszczyzną prostopadła do osi świata, przechodząca przez środek sfery niebieskiej}
\Clue{33}{}{HUFIEC}
\Clue{35}{}{czas uniwersalny, UT - astronomiczny czas słoneczny średni na południku zerowym, za który przyjęto południk przechodzący przez obserwatorium astronomiczne w miejscowości Greenwich (obecnie jest to dzielnica Londynu w Wielkiej Brytanii)}
\Clue{37}{}{krótka skarpetka, która przykrywa jedynie stopę}\end{PuzzleClues}\newpage\section*{Krzyżówka 193}

\noindent\begin{Puzzle}{22}{32}|*	|*	|*	|*	|*	|*	|*	|*	|*	|*	|*	|*	|*	|*	|*	|*	|*	|*	|*	|*	|*	|*	|[1][S]\darr	|.
|*	|*	|*	|[2][S]\darr	|*	|[3][S]\darr	|[4][S]\drarr	|r	|a	|m	|o	|n	|e	|s	|k	|a	|*	|*	|*	|*	|[5][S]\darr	|*	|k	|.
|*	|*	|[6][S]\darr	|b	|[7][S]\rarr	|o	|m	|f	|a	|l	|o	|m	|a	|n	|c	|j	|a	|*	|*	|[8][S]\darr	|m	|*	|i	|.
|*	|*	|f	|i	|[9][S]\rarr	|k	|a	|z	|a	|n	|i	|e	|*	|*	|*	|[10][S]\darr	|*	|[11][S]\darr	|*	|z	|u	|[12][S]\darr	|p	|.
|*	|[13][S]\drarr	|r	|e	|t	|o	|r	|s	|j	|a	|*	|*	|*	|*	|[14][S]\darr	|w	|[15][S]\darr	|a	|[16][S]\darr	|a	|s	|w	|e	|.
|*	|k	|y	|g	|*	|l	|l	|*	|*	|*	|*	|*	|[17][S]\darr	|*	|l	|a	|b	|k	|ż	|m	|i	|a	|r	|.
|*	|l	|s	|*	|[18][S]\darr	|i	|a	|[19][S]\rarr	|g	|n	|i	|o	|t	|*	|o	|r	|u	|a	|u	|s	|c	|ż	|y	|.
|*	|o	|*	|*	|p	|c	|*	|*	|*	|*	|*	|*	|o	|[20][S]\rarr	|t	|e	|n	|d	|r	|z	|a	|k	|*	|.
|*	|c	|[21][S]\rarr	|p	|r	|z	|y	|m	|i	|o	|t	|n	|o	|*	|[][,]{ }	|g	|c	|e	|a	|*	|l	|a	|[22][S]\darr	|.
|*	|*	|*	|*	|o	|n	|*	|[23][S]\rarr	|i	|n	|t	|e	|l	|*	|ś	|*	|h	|m	|w	|[24][S]\darr	|*	|*	|p	|.
|*	|[25][S]\drarr	|s	|e	|g	|o	|u	|*	|*	|[26][S]\rarr	|r	|o	|b	|a	|l	|*	|e	|i	|*	|ś	|*	|*	|a	|.
|*	|l	|*	|*	|r	|ś	|*	|*	|[27][S]\drarr	|o	|n	|t	|a	|r	|i	|o	|*	|k	|*	|l	|*	|*	|t	|.
|*	|a	|*	|*	|a	|ć	|*	|*	|k	|[28][S]\drarr	|w	|y	|r	|s	|z	|e	|c	|*	|*	|e	|[29][S]\darr	|*	|i	|.
|*	|w	|[30][S]\darr	|*	|m	|[][,]{ }	|*	|[31][S]\darr	|s	|p	|*	|*	|*	|*	|g	|[32][S]\darr	|*	|[33][S]\darr	|*	|d	|ż	|[34][S]\darr	|s	|.
|*	|*	|l	|*	|[][,]{ }	|ł	|*	|a	|y	|r	|*	|[35][S]\darr	|*	|*	|o	|j	|[36][S]\darr	|r	|[37][S]\darr	|ź	|ó	|w	|o	|.
|*	|[38][S]\darr	|a	|[39][S]\darr	|n	|a	|*	|b	|k	|z	|*	|b	|*	|*	|w	|u	|s	|o	|b	|[][,]{ }	|ł	|ł	|n	|.
|*	|u	|b	|w	|a	|g	|[40][S]\darr	|s	|*	|e	|[41][S]\drarr	|r	|e	|c	|y	|d	|y	|w	|i	|s	|t	|a	|*	|.
|[42][S]\drarr	|n	|i	|e	|p	|o	|r	|z	|ą	|d	|n	|o	|ś	|ć	|*	|e	|l	|e	|a	|u	|a	|d	|*	|.
|i	|i	|r	|n	|r	|d	|y	|t	|*	|w	|i	|ń	|*	|*	|*	|o	|w	|r	|ł	|w	|c	|z	|*	|.
|z	|o	|y	|e	|a	|z	|b	|y	|*	|i	|e	|[][,]{ }	|*	|*	|*	|c	|e	|[][,]{ }	|o	|o	|z	|u	|*	|.
|b	|n	|n	|z	|w	|ą	|a	|f	|[43][S]\darr	|o	|u	|a	|*	|*	|*	|h	|t	|g	|ś	|r	|e	|c	|*	|.
|a	|i	|t	|u	|c	|c	|[][,]{ }	|i	|k	|ś	|f	|u	|*	|*	|*	|r	|a	|ó	|l	|o	|k	|h	|*	|.
|[][,]{ }	|s	|o	|e	|z	|a	|d	|k	|a	|n	|n	|t	|*	|*	|*	|z	|*	|r	|i	|w	|[][,]{ }	|n	|*	|.
|c	|t	|w	|l	|y	|*	|r	|a	|r	|i	|o	|o	|*	|*	|*	|e	|*	|s	|w	|a	|i	|a	|*	|.
|h	|a	|a	|c	|*	|*	|a	|n	|a	|e	|ś	|m	|*	|*	|*	|ś	|*	|k	|i	|*	|n	|*	|*	|.
|o	|*	|t	|z	|*	|[44][S]\darr	|p	|t	|m	|*	|ć	|a	|*	|*	|*	|c	|*	|i	|e	|*	|d	|*	|*	|.
|r	|*	|e	|y	|*	|s	|i	|*	|b	|*	|*	|t	|*	|*	|*	|i	|*	|*	|*	|*	|y	|*	|*	|.
|y	|*	|*	|k	|[45][S]\drarr	|z	|e	|r	|o	|*	|*	|y	|*	|*	|*	|j	|*	|*	|*	|*	|j	|*	|*	|.
|c	|*	|*	|*	|g	|c	|ż	|*	|l	|*	|[46][S]\rarr	|c	|h	|e	|l	|a	|t	|o	|r	|*	|s	|*	|*	|.
|h	|*	|*	|*	|u	|z	|n	|*	|*	|*	|*	|z	|[47][S]\rarr	|t	|e	|n	|o	|r	|*	|*	|k	|*	|*	|.
|*	|*	|*	|*	|l	|u	|a	|*	|[48][S]\rarr	|m	|o	|n	|a	|c	|h	|i	|j	|k	|a	|*	|i	|*	|*	|.
|*	|*	|*	|*	|a	|r	|*	|*	|*	|[49][S]\rarr	|m	|a	|r	|g	|i	|n	|e	|s	|*	|*	|*	|*	|*	|.
|*	|*	|*	|*	|*	|*	|*	|*	|*	|*	|*	|*	|*	|*	|*	|*	|*	|*	|*	|*	|*	|*	|*	|.\end{Puzzle}

\newpage

\begin{PuzzleClues}{\textbf{Poziome}\\}\Clue{4}{}{skórzona kurtka, która zawsze ma podwójne klapy z ćwiekami, przesunięty w bok zamek błyskawiczny i przyszyty pas do ściągania pół}
\Clue{7}{}{wróżenie z pępka lub pępowiny}
\Clue{9}{}{przemowa duchownego podczas nabożeństwa, tłumacząca zgromadzonym zasady wiary}
\Clue{13}{}{środki odwetowe zastosowane przez jedno państwo w odpowiedzi na godzące w jego interes akty innego państwa}
\Clue{19}{}{ubytek przekroju ciągnionego lub walcowanego materiału}
\Clue{20}{}{parowóz ze zbiornikiem wodnym i skrzynią węglową, pozbawiony tendra, czyli specjalnego wagonu do przewozu węgla i wody dla parowozu}
\Clue{21}{}{roślina zielna z rodziny złożonych, w Polsce pospolity chwast polny i roślina ruderalna}
\Clue{23}{}{największy na świecie producent układów scalonych oraz twórca mikroprocesorów z rodziny x86, które znajdują się w większości komputerów osobistych}
\Clue{25}{}{miasto w zach. Mali nad Nigerem; ośrodek handlowy}
\Clue{26}{}{wielki, brzydki robak}
\Clue{27}{}{miasto w USA, w stanie Kalifornia}
\Clue{28}{}{miasto w Bułgarii (okręg Michajłowgrad) na przedgórzu Starej Płaniny}
\Clue{41}{}{wielokrotny przestępca, chwytany na popełnianiu tego samego przestępstwa}
\Clue{42}{}{cecha jakiejś rzeczy, przedmiotu: kiepskie wykonanie}
\Clue{45}{}{punkt na skali, od którego rozpoczyna się mierzenie}
\Clue{46}{}{czynnik chelatujący, czyli taki, który w procesie chelatacji dostarcza strukturze chelatu pierścieniowego wielopodstawnego ligandu}
\Clue{47}{}{najwyższy głos męski: śpiewak o takim głosie}
\Clue{48}{}{mieszkanka Monachium}
\Clue{49}{}{niezadrukowany, niezapisany brzeg strony}\end{PuzzleClues}

\begin{PuzzleClues}{\textbf{Pionowe}\\}\Clue{1}{}{śledzie rozpołowione, lekko solone i wędzone, spożywane w krajach anglosaskich po usmażeniu jako popularne danie śniadaniowe}
\Clue{2}{}{pojedynczy wyścig}
\Clue{3}{}{fakt, który jest ujawniony w toku procesu sądowego i wpływa łagodząco na wyrok}
\Clue{4}{}{cienka rzadka tkanina bawełniana; gaza}
\Clue{5}{}{gatunek filmowy, w którym scenom mówionym towarzyszą sceny muzyczne i taneczne}
\Clue{6}{}{rodzaj flaneli w kratkę, noszonej czasem jako dolna warstwa odzieży}
\Clue{8}{}{wyprawiona skóra niektórych zwierząt, pozbawiona warstwy zewnętrznej; miękka ciągliwa, używana do wyrobu rękawiczek, obuwia i odzieży}
\Clue{10}{}{osoba z grupy skandynawskich wikingów, którzy w VIII i IX wieku działali w rejonach obecnej Ukrainy i Rosji, tworząc tam w późniejszym okresie m.in. pierwszy organizm państwowy, nazwany później Rusią Kijowską}
\Clue{11}{}{student}
\Clue{12}{}{owad uskrzydlony z rzędu ważek}
\Clue{13}{}{kał, odchody}
\Clue{14}{}{rodzaj lotu, w którym poruszający się obiekt nie jest w jego trakcie napędzany}
\Clue{15}{}{amerykański socjolog (1904-1971); laureat pokojowej nagrody Nobla}
\Clue{16}{}{GRUS; gwiazdozbiór nieba południowego}
\Clue{17}{}{jeden z elementów okna (podobnie jak pasek menu) używany w GUI}
\Clue{18}{}{plan działań mających na celu poprawienie efektywności poprzez analizowanie słabych i mocnych stron ocenianego podmiotu}
\Clue{22}{}{owoc (nibyjagoda) rośliny o tej samej nazwie}
\Clue{24}{}{Alosa suworowi - gatunek ryby z rodziny śledziowatych (Clupeidae)}
\Clue{25}{}{szkocki finansista i ekonomista (1671-1728); generalny kontroler finansów we Francji}
\Clue{27}{}{syk węża}
\Clue{28}{}{jedna z dwóch uzupełniających pór roku w przyrodzie, w strefie klimatu umiarkowanego (drugą uzupełniającą porą jest przedzimie)}
\Clue{29}{}{gatunek ryby z rodziny pielęgnicowatych}
\Clue{30}{}{Anabantidae - rodzina słodkowodnych ryb okoniokształtnych zaliczana do błędnikowców}
\Clue{31}{}{zalotnik, niekiedy niechciany (przez pannę lub jej rodziców) i nieudolny; słowo używane ironicznie lub żartobliwie}
\Clue{32}{}{członek grupy istniejącej we wczesnej historii chrześcijaństwa (tuż po śmierci Jezusa z Nazaretu), tworzącej Kościół jerozolimski}
\Clue{33}{}{rodzaj roweru przeznaczony do jazdy w urozmaiconym terenie}
\Clue{34}{}{ironicznie, pobłażliwie o władzy - uprawnieniu do zwierzchnictwa nad jakąś grupą, możliwości sprawowania rządów}
\Clue{35}{}{rodzaj broni palnej, w której kolejne fazy cyklu pracy broni (poza wprowadzeniem pierwszego naboju do komory nabojowej i uruchomieniem mechanizmu spustowego) są wykonywane bez użycia energii strzelającego}
\Clue{36}{}{kształt postaci lub obiektu rysujący się na kontrastowym tle}
\Clue{37}{}{wieś w Polsce położona w województwie wielkopolskim, w powiecie pilskim, w gminie Białośliwie, nad Notecią, przy drodze wojewódzkiej nr 190}
\Clue{38}{}{człowiek, który jest zwolennikiem unii, zjednoczenia, najczęściej o charakterze politycznym}
\Clue{39}{}{mieszkaniec Wenezueli, człowiek pochodzenia wenezuelskiego}
\Clue{40}{}{zwierzę kręgowe żyjące w wodzie, oddychające skrzelami i poruszające się za pomocą płetw, żywiące się mięsem innych zwierząt}
\Clue{41}{}{cecha człowieka: to, że ktoś żywi podejrzenia}
\Clue{42}{}{punkt zamkniętej opieki zdrowotnej w zakładzie pracy, jednostce wojskowej, internacie, przeznaczony dla lżej chorych i tworzony w sytuacji utrudnionego dostępu do szpitala}
\Clue{43}{}{gra bilardowa, najmniej popularna odmiana bilardu}
\Clue{44}{}{ktoś chudy, zazwyczaj o zapadniętych policzkach, które upodobniają twarz do szczurzego pyszczka}
\Clue{45}{}{fragment powierzchni grubszy i zwykle twardszy niż jej reszta, zgrubienie}\end{PuzzleClues}\newpage\section*{Krzyżówka 194}

\noindent\begin{Puzzle}{18}{30}|*	|*	|[1][S]\darr	|*	|[2][S]\darr	|[3][S]\darr	|*	|*	|*	|*	|[4][S]\darr	|*	|*	|*	|*	|*	|*	|[5][S]\darr	|*	|.
|*	|*	|c	|*	|h	|p	|[6][S]\drarr	|g	|r	|o	|d	|z	|i	|c	|a	|*	|*	|ć	|*	|.
|*	|*	|e	|[7][S]\darr	|a	|o	|i	|[8][S]\rarr	|m	|r	|o	|k	|*	|*	|*	|[9][S]\darr	|[10][S]\darr	|w	|*	|.
|*	|[11][S]\drarr	|p	|u	|n	|k	|t	|*	|[12][S]\darr	|[13][S]\darr	|k	|*	|*	|*	|*	|t	|j	|i	|*	|.
|*	|ł	|e	|l	|o	|u	|u	|*	|m	|s	|[][,]{ }	|*	|*	|*	|[14][S]\darr	|r	|o	|e	|*	|.
|*	|u	|l	|*	|*	|ś	|r	|*	|a	|k	|p	|*	|*	|[15][S]\darr	|l	|z	|d	|k	|*	|.
|*	|b	|i	|*	|[16][S]\darr	|n	|b	|*	|r	|u	|ł	|*	|[17][S]\drarr	|d	|u	|m	|a	|*	|*	|.
|*	|*	|a	|[18][S]\darr	|s	|i	|i	|*	|s	|p	|y	|[19][S]\darr	|b	|e	|k	|i	|n	|*	|*	|.
|*	|[20][S]\darr	|d	|f	|t	|k	|*	|[21][S]\darr	|y	|o	|w	|w	|o	|l	|s	|e	|[][S](	|*	|*	|.
|[22][S]\drarr	|f	|a	|l	|a	|*	|*	|m	|l	|w	|a	|i	|x	|t	|e	|l	|v	|*	|*	|.
|s	|a	|*	|u	|ń	|[23][S]\darr	|[24][S]\darr	|a	|i	|a	|j	|t	|e	|a	|m	|e	|[][S])	|*	|*	|.
|z	|r	|*	|i	|c	|j	|p	|r	|a	|n	|ą	|w	|r	|[][,]{ }	|b	|c	|*	|[25][S]\darr	|*	|.
|y	|m	|[26][S]\drarr	|d	|z	|i	|e	|l	|n	|i	|c	|a	|*	|w	|u	|[][,]{ }	|*	|s	|*	|.
|m	|e	|g	|*	|y	|g	|l	|i	|k	|e	|y	|*	|[27][S]\darr	|s	|r	|p	|*	|t	|*	|.
|a	|r	|r	|*	|k	|*	|o	|n	|a	|*	|*	|*	|s	|t	|k	|i	|*	|r	|*	|.
|n	|k	|z	|*	|*	|*	|n	|*	|*	|*	|*	|*	|ł	|e	|a	|r	|[28][S]\darr	|o	|*	|.
|o	|i	|e	|[29][S]\rarr	|k	|i	|e	|r	|k	|u	|t	|*	|u	|c	|*	|e	|k	|j	|*	|.
|w	|*	|b	|*	|[30][S]\rarr	|f	|u	|t	|r	|o	|*	|*	|p	|z	|[31][S]\darr	|n	|u	|n	|*	|.
|s	|*	|i	|[32][S]\drarr	|h	|a	|s	|e	|ł	|k	|o	|*	|e	|n	|a	|e	|p	|i	|*	|.
|k	|*	|e	|n	|*	|*	|t	|*	|*	|*	|*	|*	|k	|a	|l	|j	|k	|ś	|[33][S]\darr	|.
|i	|*	|ń	|a	|[34][S]\drarr	|g	|e	|n	|e	|r	|a	|ł	|*	|*	|p	|s	|ó	|[][,]{ }	|e	|.
|*	|[35][S]\darr	|*	|w	|a	|[36][S]\rarr	|s	|c	|a	|f	|a	|t	|i	|*	|i	|k	|w	|n	|k	|.
|[37][S]\drarr	|s	|w	|i	|n	|g	|*	|*	|*	|*	|*	|*	|*	|*	|n	|i	|k	|a	|s	|.
|g	|o	|*	|s	|a	|*	|*	|[38][S]\rarr	|a	|m	|y	|r	|y	|n	|a	|*	|a	|d	|p	|.
|a	|l	|*	|*	|g	|[39][S]\rarr	|k	|a	|p	|i	|t	|u	|l	|a	|r	|z	|*	|o	|r	|.
|l	|ó	|[40][S]\drarr	|j	|a	|d	|[][,]{ }	|p	|s	|z	|c	|z	|e	|l	|i	|*	|*	|b	|e	|.
|a	|w	|f	|*	|*	|*	|*	|*	|*	|*	|[41][S]\rarr	|k	|o	|g	|u	|t	|*	|n	|s	|.
|*	|k	|a	|*	|[42][S]\rarr	|r	|d	|z	|e	|ń	|[][,]{ }	|a	|t	|o	|m	|o	|w	|y	|*	|.
|[43][S]\drarr	|a	|n	|t	|y	|d	|o	|g	|m	|a	|t	|y	|z	|m	|*	|*	|*	|*	|*	|.
|p	|*	|o	|*	|*	|*	|*	|*	|*	|*	|*	|*	|*	|*	|*	|*	|*	|*	|*	|.
|*	|*	|*	|*	|*	|*	|*	|*	|*	|*	|*	|*	|*	|*	|*	|*	|*	|*	|*	|.\end{Puzzle}

\newpage

\begin{PuzzleClues}{\textbf{Poziome}\\}\Clue{6}{}{pal stalowy stosowany przy robotach ziemnych}
\Clue{8}{}{przenośnie: beznadzieja, smutek, żal, zło, czas, okoliczności, w których nie ma radości, jest tylko przygnębienie}
\Clue{11}{}{określone miejsce, bardzo mała część powierzchni}
\Clue{17}{}{niższa izba parlamentu rosyjskiego}
\Clue{22}{}{przenośnie: tłum ludzi}
\Clue{26}{}{część Polski po podziale dokonanym przez Bolesława Krzywoustego w 1138}
\Clue{29}{}{KIRKUT}
\Clue{30}{}{skóra niektórych zwierząt (pokryta włosami), która jest po wyprawieniu używana do wyrobu różnych przedmiotów, głównie odzieży, galanterii i elementów wystroju wnętrz}
\Clue{32}{}{motto}
\Clue{34}{}{oficerski stopień wyższy od pułkownika}
\Clue{36}{}{miasto we Włoszech (Kampania), przy linii kolejowej Neapol Brindisi}
\Clue{37}{}{styl jazzowy będący stadium przejściowym między jazzem historycznym a nowoczesnym}
\Clue{38}{}{związek organiczny z grupy terpenów, pentacykliczny triterpen; nienasycony alkohol}
\Clue{39}{}{pomieszczenie w kościele przeznaczone na zebrania}
\Clue{40}{}{wydzielina gruczołu jadowego pszczoły robotnicy lub matki pszczelej}
\Clue{41}{}{sterczące, stojące dęba włosy (mogą sterczeć samoistnie lub być fryzurą celowo postawioną)}
\Clue{42}{}{jądro atomowe z elektronami niewalencyjnymi}
\Clue{43}{}{podejście, nastawienie człowieka, który nie uznaje dogmatów bezkrytycznie, podważa je lub w ogóle ich nie uznaje}\end{PuzzleClues}

\begin{PuzzleClues}{\textbf{Pionowe}\\}\Clue{1}{}{impreza, na której są prezentowane i sprzedawane wyroby ludowe, rękodzielnicze}
\Clue{2}{}{zatoka Morza Bałtyckiego u wybrzeży Szwecji}
\Clue{3}{}{mężczyzna, który kusi, uwodzi}
\Clue{4}{}{jednostka pływająca o konstrukcji pontonowej (najczęściej stalowej), o przekroju w kształcie litery U, przeznaczona do wynoszenia ponad poziom wody innych jednostek}
\Clue{5}{}{gwóźdź z wypukłą główką}
\Clue{6}{}{hiszpański pianista, kompozytor i dyrygent (1895-1980); wybitny wykonawca utworów Debussyego, Ravela. Albeniza}
\Clue{7}{}{w rzemiośle pszczelarskim, konstrukcja, najczęściej drewniana, używana do hodowli pszczół}
\Clue{9}{}{Bombus (Psithyrus) quadricolor - gatunek owada z rodziny pszczołowatych}
\Clue{10}{}{jodan - sól kwasu jodowego (HIO3), zawierająca anion jodanowy}
\Clue{11}{}{kora drzewna zdejmowana w okresie krążenia soków z wiązu, lipy, sosny}
\Clue{12}{}{francuska pieśń rewolucyjna, od 1975r hymn państwowy Republiki Francuskiej}
\Clue{13}{}{to, że ktoś skupuje}
\Clue{14}{}{mieszkanka Luksemburga - stolicy państwa Luksemburg}
\Clue{15}{}{delta narastająca w kierunku przeciwnym do spływu wód rzecznych, co jest skutkiem akumulacji wód rzecznych po cofkach}
\Clue{16}{}{członek konserwatywnego ugrupowania politycznego, działającego w Galicji od lat 60. XIX }
\Clue{17}{}{model samochodu dostawczego klasy midi i maxi produkowany przez Peugeota w firmie Sevel Sud}
\Clue{18}{}{według spirytystów energia, prąd psychiczny wydzielający się z ludzkiego ciała; myśli skierowane na jakąś osobę, wydarzenie, sytuację}
\Clue{19}{}{wierzba wiciowa - euroazjatycki krzew o długich, dekoracyjnych liściach}
\Clue{20}{}{spodnie z grubego płótna lub drelichu - DŻINSY, TEKSASY}
\Clue{21}{}{ciepłowodna ryba morska z rodziny żaglic o wadze do 1000 kg}
\Clue{22}{}{pianista i kompozytor (1789-1831); wirtuoz o europejskiej sławie; utwory fortepianowe, kameralne, pieśni}
\Clue{23}{}{utwór muzyczny stylizowany na taniec jig; często wykorzystywany jest jako część suity klasycznej}
\Clue{24}{}{Peloneustes - nazwa rodzajowa drapieżnego pliozaura, żyjącego pod koniec jury; jego szczątki odkryto na terenie Anglii oraz Rosji, co wskazuje na fakt, że zwierzę to występowało szeroko w morzach oblewających obszary Europy}
\Clue{25}{}{Philaeus chrysops - gatunek średniej wielkości pająka z rodziny skakunowatych (Salticidae)}
\Clue{26}{}{ciastko z ciasta francuskiego przełożonego dżemem lub marmoladą uformowane w kształt przypominający grzebień do czesania}
\Clue{27}{}{żeński element płciowy u kwiatów okrytonasiennych}
\Clue{28}{}{RŻNIĄCZKA cenna pastewna trawa zbitokępkowa o szerokich liściach}
\Clue{31}{}{ogród skalny, w którym uprawiane są rośliny górskie dla celów dydaktycznych i naukowych}
\Clue{32}{}{coś (nagromadzonego), co nad czymś wisi, z czegoś zwisa}
\Clue{33}{}{goniec, umyślny; pracownik odpowiedzialny za roznoszenie korespondencji}
\Clue{34}{}{gruzińskie wino}
\Clue{35}{}{część utworu, wstawka solowa jakiegoś artysty}
\Clue{37}{}{jezioro w Grecji połączone z rzeką Marica}
\Clue{40}{}{miasto we Włoszech (Marche) nad Morzem Adriatyckim}
\Clue{43}{}{siła, z jaką Ziemia przyciąga masę 1 g w miejscu; 1 gf = 1 p = 0,0980665 N}\end{PuzzleClues}\newpage\section*{Krzyżówka 195}

\noindent\begin{Puzzle}{24}{26}|*	|*	|*	|*	|*	|*	|*	|*	|*	|*	|[1][S]\drarr	|b	|a	|r	|a	|n	|*	|[2][S]\drarr	|m	|a	|p	|a	|*	|[3][S]\darr	|*	|.
|*	|*	|[4][S]\darr	|*	|[5][S]\darr	|*	|[6][S]\drarr	|w	|i	|e	|l	|k	|a	|[][,]{ }	|s	|t	|o	|p	|a	|*	|*	|*	|*	|p	|*	|.
|*	|*	|p	|*	|s	|[7][S]\drarr	|s	|e	|k	|t	|o	|r	|[][,]{ }	|p	|r	|y	|w	|a	|t	|n	|y	|*	|*	|r	|*	|.
|*	|*	|u	|*	|ó	|b	|ł	|*	|*	|*	|g	|[8][S]\rarr	|o	|ś	|[][,]{ }	|p	|o	|r	|t	|a	|l	|o	|w	|a	|*	|.
|*	|*	|z	|[9][S]\darr	|l	|i	|ó	|[10][S]\rarr	|p	|s	|i	|z	|ą	|b	|[][,]{ }	|b	|i	|a	|ł	|a	|w	|y	|*	|d	|*	|.
|*	|*	|a	|k	|[][,]{ }	|e	|d	|*	|[11][S]\rarr	|s	|k	|o	|k	|*	|*	|*	|[12][S]\darr	|d	|*	|[13][S]\darr	|[14][S]\darr	|*	|*	|z	|*	|.
|*	|*	|n	|o	|o	|r	|*	|[15][S]\rarr	|i	|z	|a	|b	|a	|l	|*	|*	|s	|y	|*	|i	|s	|*	|*	|i	|*	|.
|*	|*	|e	|r	|r	|u	|*	|*	|*	|*	|*	|*	|*	|*	|[16][S]\darr	|*	|k	|g	|*	|n	|e	|*	|*	|a	|*	|.
|*	|*	|k	|n	|g	|t	|[17][S]\drarr	|k	|r	|z	|y	|ż	|*	|*	|k	|*	|o	|m	|*	|t	|n	|*	|*	|d	|*	|.
|*	|*	|[][,]{ }	|u	|a	|*	|s	|*	|*	|*	|[18][S]\darr	|*	|*	|[19][S]\rarr	|a	|l	|t	|a	|i	|r	|*	|*	|[20][S]\rarr	|e	|*	|.
|*	|*	|p	|t	|n	|*	|t	|*	|*	|*	|d	|*	|*	|*	|n	|*	|n	|t	|*	|y	|*	|*	|*	|k	|*	|.
|*	|[21][S]\darr	|a	|k	|i	|*	|r	|*	|*	|*	|u	|*	|*	|*	|t	|*	|i	|*	|[22][S]\rarr	|g	|ł	|ó	|g	|*	|*	|.
|[23][S]\drarr	|b	|l	|a	|c	|h	|a	|[][,]{ }	|z	|i	|m	|n	|o	|w	|a	|l	|c	|o	|w	|a	|n	|a	|*	|[24][S]\darr	|*	|.
|r	|r	|e	|[][,]{ }	|z	|*	|u	|*	|[25][S]\drarr	|c	|a	|r	|b	|a	|l	|*	|k	|*	|*	|*	|*	|[26][S]\darr	|*	|d	|*	|.
|e	|o	|o	|k	|n	|*	|s	|*	|k	|*	|*	|*	|[27][S]\drarr	|k	|a	|p	|i	|t	|a	|n	|*	|w	|*	|e	|*	|.
|g	|m	|s	|o	|a	|[28][S]\drarr	|s	|p	|r	|z	|ę	|ż	|a	|j	|*	|*	|*	|*	|*	|*	|*	|i	|*	|k	|*	|.
|i	|e	|t	|n	|*	|a	|*	|*	|z	|*	|*	|*	|n	|[29][S]\rarr	|r	|e	|e	|d	|u	|k	|a	|c	|j	|a	|*	|.
|m	|k	|o	|i	|*	|l	|*	|*	|y	|*	|[30][S]\rarr	|m	|g	|ł	|a	|w	|i	|c	|a	|*	|*	|e	|*	|*	|*	|.
|e	|[][,]{ }	|m	|c	|*	|a	|*	|*	|w	|[31][S]\rarr	|d	|e	|l	|f	|i	|n	|[][,]{ }	|g	|a	|r	|b	|a	|t	|y	|*	|.
|n	|m	|s	|z	|*	|m	|[32][S]\drarr	|p	|i	|e	|l	|m	|i	|e	|n	|i	|*	|[33][S]\rarr	|f	|l	|ą	|d	|r	|a	|*	|.
|t	|e	|k	|y	|*	|o	|c	|*	|c	|*	|*	|[34][S]\drarr	|k	|o	|n	|i	|n	|c	|k	|*	|*	|m	|*	|*	|*	|.
|a	|t	|i	|n	|*	|z	|o	|[35][S]\rarr	|a	|z	|u	|l	|*	|[36][S]\rarr	|u	|s	|t	|o	|n	|o	|g	|i	|*	|*	|*	|.
|r	|y	|*	|o	|*	|a	|n	|*	|*	|*	|*	|o	|*	|*	|[37][S]\rarr	|s	|y	|n	|a	|n	|t	|r	|o	|p	|*	|.
|z	|l	|*	|w	|*	|u	|*	|*	|*	|*	|*	|m	|*	|*	|[38][S]\rarr	|t	|e	|m	|p	|e	|r	|a	|*	|*	|*	|.
|*	|u	|[39][S]\rarr	|a	|g	|r	|e	|s	|j	|a	|*	|m	|*	|*	|[40][S]\rarr	|c	|h	|o	|c	|h	|o	|ł	|e	|k	|*	|.
|*	|*	|*	|*	|*	|*	|[41][S]\rarr	|p	|o	|p	|r	|a	|d	|*	|*	|*	|*	|*	|*	|*	|*	|*	|*	|*	|*	|.
|*	|*	|*	|*	|*	|*	|*	|*	|*	|*	|*	|*	|*	|*	|*	|*	|*	|*	|*	|*	|*	|*	|*	|*	|*	|.\end{Puzzle}

\newpage

\begin{PuzzleClues}{\textbf{Poziome}\\}\Clue{1}{}{osoba spod znaku zodiaku Barana}
\Clue{2}{}{układ punktów, np. mapa ciała}
\Clue{6}{}{mityczne zwierzę, które według niepotwierdzonych (kilkudziesięciu tysięcy) relacji żyje w Górach Skalistych i w sąsiednich regionach (USA i Kanada)}
\Clue{7}{}{część gospodarki narodowej, która nie jest finansowana ani kapitałem państwowym ani samorządowym}
\Clue{8}{}{stosowana w autobusach niskopodłogowych oś, w której poprzeczna belka znajduje się niżej niż piasty kół, przez co podłoga może być obniżona}
\Clue{10}{}{Erythronium albidum - gatunek roślin z rodziny liliowatych}
\Clue{11}{}{napad najczęściej o celu rabunkowym}
\Clue{15}{}{największe jezioro w Gwatemali, powierzchnia 730 km2, głębokość d 18 m}
\Clue{17}{}{najprostszy zespół obudowy górniczej złożony ze stojaka z podłożoną stropnicą}
\Clue{19}{}{najjaśniejsza gwiazda w gwiazdozbiorze Oriona}
\Clue{20}{}{liczba niewymierna, będąca podstawą logarytmu naturalnego; można ją definiować na kilka różnych sposobów}
\Clue{22}{}{Crataegus - rodzaj roślin należący do rodziny różowatych (Rosaceae)}
\Clue{23}{}{cienki wyrób hutniczy, któremu nadaje się pożądany kształt poprzez spłaszczanie walcem w niskich temperaturach}
\Clue{25}{}{żeglarz portugalski (1468-1520); odkrył wschodnie wybrzeża Brazylii}
\Clue{27}{}{najwyższy z korpusu oficerów młodszych}
\Clue{28}{}{zwierzę pociągowe wraz z osprzętem w gospodarstwie rolnym}
\Clue{29}{}{ponowne kształcenie osoby, która utraciła wcześniej posiadane wiadomości lub umiejętności}
\Clue{30}{}{obłok gazu i pyłu międzygwiezdnego, rozległa otoczka gwiazdy}
\Clue{31}{}{Sousa chinensis - gatunek walenia z rodziny delfinowatych; żyje w wodach Oceanu Inyjskiego i Spokojnego}
\Clue{32}{}{rodzaj pierogów z dość luźnym farszem mięsnym, potrawa kuchni rosyjskiej}
\Clue{33}{}{bałtycka ryba z rzędu płastug}
\Clue{34}{}{barwy dominujące w obrazie}
\Clue{35}{}{miasto w Argentynie (Pampa); węzeł kolejowy}
\Clue{36}{}{Stomatopoda - jest to mały, ale bardzo specyficzny rząd morskich skorupiaków z gromady pancerzowców, o stosunkowo prymitywnej budowie z bardzo słabo wykształconym pancerzem, który to jest słabo zesklerotyzowany; ciało wydłużone do długości 34 cm, z bardzo długim odwłokiem; dorosłe skorupiaki prowadzą najczęściej ryjący tryb życia na dnie morza, a larwy spotykane są w planktonie; dotychczas poznano około 170 gatunków}
\Clue{37}{}{zwierzę związane z człowiekiem i jego działalnością, ale nie hodowane przez niego np. mysz, szczur}
\Clue{38}{}{rzeźba świętego wykonana przez ludowego artystę}
\Clue{39}{}{zbrojna napaść jednego państwa na drugie}
\Clue{40}{}{niewielki chochoł lub normalnych rozmiarów chochoł chroniący przed wilgocią, ale określany z pozytywnym nacechowaniem}
\Clue{41}{}{rzeka we wschodniej Słowacji i w południowo-wschodniej Polsce, w dorzeczu Wisły}\end{PuzzleClues}

\begin{PuzzleClues}{\textbf{Pionowe}\\}\Clue{1}{}{logiczne myślenie, poprawny sposób rozumowania}
\Clue{2}{}{zbiór wszystkich form fleksyjnych danego wyrazu}
\Clue{3}{}{ojciec babci lub dziadka}
\Clue{4}{}{Alosa caspia palaeostomi - podgatunek Puzanka kaspijskiego (Alosa caspia)}
\Clue{5}{}{związek chemiczny będący solą organicznego kwasu lub zasady}
\Clue{6}{}{skiełkowane ziarno zbóż używane w piwowarstwie}
\Clue{7}{}{okres w historii Polski, kiedy rządy, jako prezydent RP, przewodniczący KC PZPR i prezes Rady Ministrów PRL, sprawował Bolesław Bierut}
\Clue{9}{}{kornutka długoczułkowa, Eucera longicornis - gatunek owada z rodziny pszczołowatych}
\Clue{12}{}{malarz, grafik (1876-1868) pejzaże, sceny rodzajowe, kompozycje fantastyczne, sztuka użytkowa}
\Clue{13}{}{podstęp mający zaszkodzić konkretnym osobom czy instytucjom, zwykle przez potejemne wywołanie jakiegoś konfliktu}
\Clue{14}{}{jednostka zdawkowa w kilku krajach Azji (sen to zniekształcone brzmienie anglojęzycznegocent)}
\Clue{16}{}{twarde włókno, pozyskiwane z agawy kantalowej}
\Clue{17}{}{Richard (1864-1949); kompozytor niemiecki, neuromantyk, mistrz instrumentacji; poematy symfoniczne, opery; 'Salome', 'Elektra'}
\Clue{18}{}{ludowa ukraińska pieśń epiczna - DUMKA}
\Clue{21}{}{nieorganiczny związek chemiczny z grupy bromków stosowany do ochorny roślin oraz konserwacji drewna}
\Clue{23}{}{w Polsce w XVII-XVIII wieku - zastępca hetmana}
\Clue{24}{}{przedrostek jednostki miary oznaczający mnożnik 10}
\Clue{25}{}{choroba dziecięca spowodowana niedoborem witaminy D}
\Clue{26}{}{wojskowy stopień oficerski w polskiej Marynarce Wojennej, odpowiadający generałowi dywizji w Wojskach Lądowych i Siłach Powietrznych}
\Clue{27}{}{FOLBLUT}
\Clue{28}{}{Alamosaurus - rodzaj zauropoda z grupy tytanozaurów; jego nazwa oznaczajaszczur z Alamo; żył u schyłku późnej kredy, około 71-65,5 mln lat temu na terenach Ameryki Północnej}
\Clue{32}{}{określenie wykonawcze}
\Clue{34}{}{miasto w Szwecji na płn. od Malme, nad Sundem}\end{PuzzleClues}\newpage\section*{Krzyżówka 196}

\noindent\begin{Puzzle}{23}{32}|*	|*	|*	|*	|[1][S]\drarr	|s	|z	|y	|d	|ł	|o	|s	|z	|[][,]{ }	|w	|ł	|o	|s	|i	|s	|t	|y	|*	|*	|.
|[2][S]\rarr	|e	|p	|i	|l	|i	|t	|o	|n	|*	|*	|*	|*	|[3][S]\darr	|[4][S]\darr	|*	|[5][S]\darr	|*	|[6][S]\drarr	|p	|u	|c	|h	|*	|.
|*	|*	|*	|*	|a	|[7][S]\drarr	|c	|e	|l	|e	|b	|r	|a	|n	|t	|*	|p	|*	|r	|[8][S]\darr	|*	|*	|*	|*	|.
|[9][S]\drarr	|p	|a	|n	|i	|c	|z	|ą	|t	|k	|o	|*	|*	|i	|y	|*	|t	|*	|o	|k	|[10][S]\darr	|*	|[11][S]\darr	|[12][S]\darr	|.
|e	|*	|[13][S]\drarr	|o	|k	|u	|l	|a	|r	|k	|i	|*	|*	|e	|m	|*	|a	|*	|ś	|a	|s	|*	|l	|k	|.
|n	|*	|r	|*	|a	|d	|*	|*	|[14][S]\darr	|*	|[15][S]\darr	|*	|*	|j	|o	|[16][S]\darr	|s	|*	|l	|r	|t	|*	|u	|o	|.
|d	|*	|o	|*	|t	|z	|*	|*	|w	|[17][S]\darr	|f	|*	|*	|a	|t	|s	|z	|*	|i	|e	|r	|[18][S]\darr	|t	|k	|.
|o	|[19][S]\darr	|z	|*	|*	|o	|*	|*	|y	|c	|r	|[20][S]\darr	|*	|s	|k	|u	|n	|[21][S]\darr	|n	|n	|z	|p	|ó	|s	|.
|k	|g	|s	|*	|*	|z	|*	|[22][S]\drarr	|g	|e	|o	|d	|y	|n	|a	|m	|i	|k	|a	|*	|e	|o	|w	|a	|.
|r	|r	|n	|*	|[23][S]\darr	|i	|*	|h	|ł	|d	|s	|*	|*	|o	|[][,]{ }	|g	|k	|w	|[][,]{ }	|*	|m	|p	|k	|[][,]{ }	|.
|y	|u	|u	|*	|n	|e	|*	|n	|u	|r	|t	|*	|[24][S]\darr	|ś	|k	|a	|*	|a	|k	|[25][S]\darr	|i	|y	|a	|p	|.
|n	|n	|w	|[26][S]\darr	|a	|m	|[27][S]\darr	|l	|p	|o	|y	|*	|p	|ć	|o	|i	|[28][S]\drarr	|z	|w	|r	|o	|t	|*	|o	|.
|o	|t	|a	|h	|ś	|k	|m	|*	|*	|w	|i	|*	|e	|*	|l	|t	|t	|i	|a	|ż	|n	|[][,]{ }	|*	|m	|.
|l	|[][,]{ }	|c	|i	|l	|a	|a	|[29][S]\drarr	|w	|i	|k	|u	|n	|i	|a	|*	|o	|k	|s	|n	|k	|e	|*	|a	|.
|o	|p	|z	|p	|a	|*	|t	|p	|*	|n	|*	|*	|d	|*	|n	|[30][S]\darr	|a	|r	|o	|i	|o	|l	|*	|r	|.
|g	|o	|[][,]{ }	|e	|d	|[31][S]\darr	|e	|o	|*	|a	|*	|*	|r	|[32][S]\darr	|k	|k	|l	|y	|l	|ą	|*	|a	|[33][S]\darr	|a	|.
|i	|d	|p	|r	|o	|f	|r	|w	|[34][S]\darr	|*	|*	|*	|i	|b	|o	|w	|e	|s	|u	|c	|*	|s	|c	|ń	|.
|a	|[][,]{ }	|l	|t	|w	|u	|i	|o	|l	|*	|[35][S]\darr	|[36][S]\darr	|v	|a	|w	|a	|t	|z	|b	|z	|[37][S]\darr	|t	|h	|c	|.
|*	|n	|u	|e	|n	|z	|a	|j	|i	|*	|r	|w	|e	|l	|a	|c	|a	|t	|n	|k	|t	|y	|l	|z	|.
|*	|o	|j	|n	|i	|j	|ł	|o	|l	|*	|u	|i	|*	|o	|t	|z	|*	|a	|a	|a	|e	|c	|e	|o	|.
|[38][S]\drarr	|g	|ą	|s	|k	|a	|[][,]{ }	|w	|i	|o	|s	|e	|n	|n	|a	|*	|[39][S]\darr	|ł	|*	|*	|r	|z	|b	|w	|.
|r	|a	|c	|j	|o	|*	|o	|a	|p	|*	|*	|k	|*	|*	|*	|*	|z	|*	|*	|*	|a	|n	|[][,]{ }	|a	|.
|ó	|m	|y	|o	|w	|[40][S]\darr	|p	|t	|u	|*	|*	|[][,]{ }	|*	|*	|[41][S]\rarr	|ż	|a	|b	|a	|[][,]{ }	|b	|y	|k	|*	|.
|w	|i	|*	|l	|a	|t	|a	|e	|t	|[42][S]\rarr	|f	|r	|e	|i	|k	|o	|r	|p	|s	|*	|i	|*	|l	|*	|.
|n	|*	|[43][S]\darr	|o	|t	|e	|t	|*	|k	|*	|*	|o	|[44][S]\darr	|*	|*	|*	|u	|[45][S]\darr	|*	|*	|t	|*	|a	|*	|.
|o	|*	|n	|g	|e	|r	|r	|*	|a	|[46][S]\rarr	|c	|z	|a	|r	|d	|a	|s	|z	|*	|*	|*	|*	|s	|*	|.
|n	|*	|i	|i	|*	|m	|u	|[47][S]\darr	|*	|[48][S]\rarr	|t	|r	|u	|t	|ó	|w	|k	|a	|*	|[49][S]\rarr	|t	|u	|z	|*	|.
|o	|[50][S]\drarr	|p	|a	|g	|a	|n	|i	|n	|i	|*	|o	|t	|*	|*	|*	|i	|p	|*	|*	|*	|*	|t	|*	|.
|g	|g	|i	|*	|*	|j	|k	|ł	|*	|*	|*	|d	|o	|*	|*	|*	|*	|r	|*	|*	|*	|*	|o	|*	|.
|i	|n	|g	|*	|*	|s	|o	|*	|*	|*	|*	|c	|m	|[51][S]\rarr	|m	|i	|n	|a	|*	|[52][S]\rarr	|m	|i	|r	|*	|.
|*	|ó	|o	|*	|*	|k	|w	|*	|[53][S]\rarr	|m	|u	|z	|a	|*	|*	|[54][S]\rarr	|z	|w	|i	|s	|*	|*	|n	|*	|.
|*	|j	|n	|*	|*	|a	|y	|*	|*	|*	|*	|y	|t	|*	|*	|[55][S]\rarr	|c	|a	|p	|*	|*	|*	|y	|*	|.
|*	|*	|*	|*	|*	|*	|*	|[56][S]\rarr	|b	|z	|d	|*	|*	|*	|*	|*	|*	|*	|*	|*	|*	|*	|*	|*	|.\end{Puzzle}

\newpage

\begin{PuzzleClues}{\textbf{Poziome}\\}\Clue{1}{}{szydłosz włoskowy, Cirriphyllum piliferum - gatunek mchu z rodziny krótkoszowatych; tworzy luźno skupione jasno- lub ciemnozielone darnie wysokości do 10 cm; łodyżki równomiernie pierzasto rozgałęzione, liście łodygowe jajowate, na szczycie zaokrąglone i nagle zwężone w długi kończyk; rośnie na ziemi, w wilgotnych, trawiastych miejscach, w widnych lasach liściastych}
\Clue{2}{}{glony rosnące na powierzchni skał}
\Clue{6}{}{miękka i delikatna naturalna okrywa włosowa ciał zwierząt (ssaków i ptaków) i rzadziej: elementów roślin}
\Clue{7}{}{duchowny odprawiający liturgię}
\Clue{9}{}{małe dziecko, syn pana - posiadacza majątku, właściciela lub władcy czegoś}
\Clue{13}{}{małe okulary, najczęściej noszone przez dzieci}
\Clue{22}{}{kierunek badawczy z pogranicza geologii i geofizyki zajmujący się analizą sił działających we wnętrzu Ziemi i ich przejawów na powierzchni}
\Clue{28}{}{skręt, obrót, zwrócenie się}
\Clue{29}{}{WIGOŃ}
\Clue{38}{}{gatunek grzyba jadalnego, którego owocniki zaczynają pojawiać się na wiosnę}
\Clue{41}{}{żaba rycząca, żaba wół, Lithobates catesbeianus, Rana catesbeiana - kosmopolityczny gatunek dużego płaza bezogonowego z rodziny żabowatych, pochodzący z Ameryki Północnej}
\Clue{42}{}{niemieckie oddziały używane do tłumienia ruchów rewolucyjnych oraz przeciw powstańcom polskim}
\Clue{46}{}{utwór muzyczny (taniec) stylizowany na czardasza}
\Clue{48}{}{pszczoła robotnica składająca jaja prawie wyłącznie do trutowych komórek plastra}
\Clue{49}{}{ktoś wpływowy}
\Clue{50}{}{legendarny skrzypek i kompozytor (1782-1840); rozwinął fakturę skrzypcową}
\Clue{51}{}{jednostka wagi (50 sykli) i jednostka pieniężna (100 drachm) w starożytnej Grecji}
\Clue{52}{}{radziecka stacja kosmiczna umożliwiająca przyłączanie do niej innych statków}
\Clue{53}{}{rodzaj twórczości artystycznej}
\Clue{54}{}{przechył samolotu}
\Clue{55}{}{rodzaj haka na linie używany do opuszczania i podnoszenia aparatu wiertniczego}
\Clue{56}{}{kod ISO 4217 dolara Belize}\end{PuzzleClues}

\begin{PuzzleClues}{\textbf{Pionowe}\\}\Clue{1}{}{ogół wiernych świeckich}
\Clue{3}{}{coś niejasnego}
\Clue{4}{}{Phleum hubbardii - gatunek rośliny należący do rodziny wiechlinowatych}
\Clue{5}{}{myśliwy polujący na ptaki}
\Clue{6}{}{acydofit, roślina acydofilna - roślina, która optymalne warunki do swojego rozwoju znajduje na podłożu kwaśnym, którego pH  7}
\Clue{7}{}{kobieta nieposiadająca obywatelstwa kraju, w którym przebywa}
\Clue{8}{}{KOTULEJ}
\Clue{9}{}{nauka o wydzielaniu wewnętrznym, gruczołach, hormonach zwierzęcych i ich działaniu}
\Clue{10}{}{mała kość słuchowa, wchodząca w skład ucha środkowego}
\Clue{11}{}{płyn, który ułatwia płynięcie lutu}
\Clue{12}{}{popularna angielska odmiana jabłoni}
\Clue{13}{}{Scytodes thoracica - gatunek pająka z rodzaju rozsnuwacza; jedyny gatunek z tej rodziny, występujący w Polsce}
\Clue{14}{}{osoba skłonna do błazeństw}
\Clue{15}{}{jezioro w Szwecji, w pobliżu granicy z Norwegią (Góry Skandynawskie)}
\Clue{16}{}{miasto w Azerbejdżanie nad Morzem Kaspijskim; przemysł chemiczny, huta aluminium, walcownia rur}
\Clue{17}{}{drewno uzyskiwane z cedru}
\Clue{18}{}{popyt, który znacznie się zmienia w wyniku niewielkich zmian czynnika kształtującego}
\Clue{19}{}{coś stałego, pewnego, co stanowi dla kogoś punkt oparcia}
\Clue{20}{}{litera oznaczająca wymiar}
\Clue{21}{}{szczególna forma ciała stałego, w której atomy układają się w pozornie regularną, jednak nie w powtarzającą się strukturę, co uniemożliwia wyróżnienie ich komórek elementarnych}
\Clue{22}{}{kod ISO 4217 lempiry}
\Clue{23}{}{Mimetidae - rodzina pająków z podrzędu Opisthothela; zarówno polską, jak i łacińską nazwę naukową zawdzięczają zdolnościom mimetycznym; w Polsce występują 4 gatunki: guzoń pajęczarz (Ero furcata), Ero aphana, Ero cambridgei, Ero tuberculata}
\Clue{24}{}{pamięć przenośna oparta na złączu USB i technologii flash memory}
\Clue{25}{}{KUPKÓWKA pospolita - cenna pastewna trawa zbitokępkowa o szerokich liściach}
\Clue{26}{}{dziedzina medycyny zajmująca się przebiegiem i leczeniem nadciśnienia tętniczego}
\Clue{27}{}{materiał posiadający właściwości ochronne i lecznicze, przeznaczone do pokrywania ran lub zmienionej powierzchni skóry, wchłaniania wydzielin z tkanek, zabezpieczania przed zainfekowaniem}
\Clue{28}{}{elegancki, zwykle damski strój}
\Clue{29}{}{Convolvulaceae - rodzina roślin zielnych z rzędu psiankowców obejmująca ponad 1650 gatunków w 57 rodzajach}
\Clue{30}{}{posążek wotywny przedstawiający nagiego młodzieńca, charakterystyczny dla sztuki starożytnej Grecji}
\Clue{31}{}{długa broń palna myśliwska, używana także jak broń bojowa}
\Clue{32}{}{przedmiot z elastycznej gumy, lateksu, czasami z folii, wypełniony gazem i służący jako dekoracja lub zabawka}
\Clue{33}{}{chleb żytni wypiekany bez drożdży tradycyjnymi metodami, duży i ciężki, pierwotnie robiony przez średniowieczne klasztory w dniu świąt ich patronów}
\Clue{34}{}{kura o mniejszych rozmiarach niż przeciętnie (może to być ptak będący przedstawicielem rasy liliput polski, ale zdaje się, że nie musi)}
\Clue{35}{}{rodzaj gry karcianej}
\Clue{36}{}{wiek, jaki musi osiągnąć organizm, żeby móc się rozmnażać}
\Clue{37}{}{jednostka informacji, równa 10\textasciicircum12 bitów}
\Clue{38}{}{Isopoda - rząd skorupiaków z gromady pancerzowców, obejmujący około 4000 gatunków zwierząt; przedstawiciele żyją głównie w morzach, czasem na lądzie lub w wodach słodkich}
\Clue{39}{}{żeglarz, taternik, generał, pisarz (1867-1941); jeden z pionierów polskiego narciarstwa i żeglarstwa}
\Clue{40}{}{zatoka Morza Egejskiego między Płw Chalcydyckim a wybrzeżem Grecji, główny port Saloniki}
\Clue{43}{}{jezioro w Kanadzie, powierzchnia 4,8 tyś. km2, głębokość do 165 m, rzeką Nipign połączone z Jeziorem Górnym}
\Clue{44}{}{pralka automatyczna}
\Clue{45}{}{słoik z przetworami, które przygotowuje się z owoców i warzyw na zimę (zazwyczaj nazwę tę stosuje się do przetworów robionych w domu lub tych, które tradycyjnie pochodzą/pochodziły z produkcji hobbystycznej)}
\Clue{47}{}{skała osadowa, bardzo drobnoziarnista, zbudowana jest głównie z minerałów ilastych (najczęściej kaolinit i illit) z domieszką łyszczyków oraz pyłu kwarcowego}
\Clue{50}{}{zanieczyszczenie, metaforycznie o błocie, brudzie}\end{PuzzleClues}\newpage\section*{Krzyżówka 197}

\noindent\begin{Puzzle}{24}{33}|*	|[1][S]\drarr	|t	|o	|p	|o	|r	|n	|i	|c	|a	|[][,]{ }	|w	|i	|e	|l	|k	|a	|*	|*	|*	|*	|*	|*	|*	|.
|[2][S]\rarr	|k	|o	|n	|c	|e	|p	|c	|i	|k	|*	|*	|*	|*	|[3][S]\drarr	|p	|a	|r	|t	|n	|e	|r	|k	|a	|*	|.
|[4][S]\rarr	|o	|j	|c	|a	|s	|z	|e	|k	|*	|*	|*	|*	|[5][S]\rarr	|p	|r	|o	|f	|e	|s	|o	|r	|e	|k	|*	|.
|[6][S]\rarr	|z	|o	|o	|*	|*	|*	|*	|*	|*	|*	|*	|*	|[7][S]\rarr	|r	|a	|n	|d	|e	|r	|s	|*	|*	|*	|*	|.
|*	|i	|[8][S]\rarr	|l	|o	|t	|n	|i	|c	|t	|w	|o	|[][,]{ }	|s	|z	|t	|u	|r	|m	|o	|w	|e	|*	|*	|[9][S]\darr	|.
|*	|b	|*	|[10][S]\darr	|*	|[11][S]\darr	|[12][S]\darr	|*	|*	|[13][S]\rarr	|m	|a	|u	|r	|e	|s	|k	|a	|*	|*	|*	|*	|*	|*	|t	|.
|*	|r	|[14][S]\rarr	|k	|o	|s	|m	|o	|s	|*	|[15][S]\rarr	|s	|a	|r	|d	|y	|n	|e	|l	|e	|*	|*	|[16][S]\darr	|*	|e	|.
|*	|ó	|*	|r	|[17][S]\rarr	|r	|e	|a	|l	|i	|s	|t	|y	|c	|z	|n	|o	|ś	|ć	|*	|*	|*	|r	|*	|n	|.
|*	|d	|[18][S]\darr	|a	|*	|e	|t	|[19][S]\darr	|*	|[20][S]\drarr	|d	|r	|u	|c	|i	|e	|ń	|c	|e	|*	|*	|*	|u	|[21][S]\darr	|n	|.
|*	|[][,]{ }	|k	|s	|[22][S]\drarr	|b	|o	|m	|b	|a	|[][,]{ }	|k	|o	|b	|a	|l	|t	|o	|w	|a	|*	|*	|s	|p	|e	|.
|*	|p	|o	|n	|m	|r	|d	|i	|[23][S]\darr	|u	|[24][S]\drarr	|p	|u	|d	|ł	|o	|*	|[25][S]\darr	|*	|[26][S]\darr	|*	|*	|y	|r	|s	|.
|*	|a	|s	|o	|e	|n	|a	|t	|d	|t	|a	|*	|*	|*	|*	|*	|*	|z	|*	|s	|*	|*	|c	|e	|s	|.
|*	|j	|z	|g	|g	|y	|[][,]{ }	|*	|a	|o	|r	|*	|[27][S]\drarr	|l	|e	|c	|*	|b	|*	|n	|[28][S]\darr	|[29][S]\darr	|y	|s	|e	|.
|*	|ę	|y	|ł	|a	|[][,]{ }	|s	|*	|r	|m	|a	|[30][S]\drarr	|r	|o	|s	|s	|*	|l	|[31][S]\darr	|i	|p	|u	|s	|j	|e	|.
|*	|c	|k	|ó	|l	|m	|i	|[32][S]\rarr	|j	|a	|b	|ł	|o	|ń	|[][,]{ }	|k	|w	|i	|e	|c	|i	|s	|t	|a	|*	|.
|*	|z	|[][,]{ }	|w	|a	|e	|m	|[33][S]\darr	|e	|t	|i	|a	|k	|*	|*	|*	|*	|ż	|l	|k	|ł	|ł	|y	|[][,]{ }	|*	|.
|*	|y	|i	|[][,]{ }	|n	|d	|p	|c	|e	|y	|k	|m	|[][,]{ }	|*	|*	|*	|*	|e	|e	|e	|o	|u	|k	|i	|*	|.
|*	|n	|n	|b	|k	|a	|s	|y	|l	|k	|a	|a	|k	|*	|*	|*	|*	|n	|k	|r	|n	|g	|a	|n	|*	|.
|[34][S]\rarr	|o	|f	|i	|o	|l	|o	|g	|i	|a	|*	|c	|o	|[35][S]\darr	|*	|*	|[36][S]\darr	|i	|t	|s	|o	|a	|*	|f	|*	|.
|*	|w	|l	|a	|z	|i	|n	|a	|n	|[][,]{ }	|*	|z	|ś	|r	|*	|[37][S]\darr	|k	|e	|r	|*	|s	|[][,]{ }	|*	|l	|*	|.
|*	|a	|a	|ł	|a	|s	|a	|n	|g	|p	|[38][S]\darr	|*	|c	|o	|*	|d	|i	|*	|o	|*	|o	|i	|*	|a	|*	|.
|*	|t	|c	|o	|u	|t	|*	|e	|*	|r	|o	|*	|i	|g	|*	|e	|r	|*	|d	|*	|k	|n	|*	|c	|*	|.
|*	|y	|y	|g	|r	|a	|*	|c	|[39][S]\darr	|z	|l	|[40][S]\darr	|e	|o	|[41][S]\darr	|l	|y	|*	|y	|*	|s	|t	|*	|y	|*	|.
|*	|*	|j	|a	|*	|*	|*	|z	|g	|e	|e	|l	|l	|ź	|b	|i	|s	|*	|n	|*	|z	|e	|*	|j	|*	|.
|*	|*	|n	|r	|[42][S]\darr	|*	|[43][S]\rarr	|k	|a	|m	|i	|e	|n	|n	|i	|k	|*	|*	|a	|*	|t	|r	|*	|n	|*	|.
|*	|*	|y	|d	|l	|*	|*	|a	|r	|y	|s	|r	|y	|i	|g	|a	|*	|*	|m	|*	|a	|n	|*	|a	|*	|.
|*	|*	|*	|ł	|b	|*	|*	|*	|d	|s	|t	|w	|*	|c	|[][S]-	|t	|*	|*	|o	|*	|ł	|e	|*	|*	|*	|.
|*	|[44][S]\rarr	|w	|y	|p	|a	|ł	|*	|e	|ł	|o	|i	|*	|a	|b	|n	|*	|*	|m	|*	|t	|t	|*	|*	|*	|.
|*	|*	|*	|*	|*	|*	|*	|*	|r	|o	|ś	|c	|*	|*	|e	|o	|*	|*	|e	|*	|n	|o	|*	|*	|*	|.
|*	|*	|*	|*	|*	|*	|*	|*	|o	|w	|ć	|k	|*	|*	|a	|ś	|*	|*	|t	|*	|e	|w	|*	|*	|*	|.
|*	|*	|*	|[45][S]\drarr	|l	|i	|c	|z	|b	|a	|*	|*	|*	|*	|t	|ć	|*	|*	|r	|*	|*	|a	|*	|*	|*	|.
|[46][S]\rarr	|t	|ł	|u	|s	|z	|c	|z	|a	|*	|*	|*	|*	|*	|*	|*	|*	|*	|*	|*	|*	|*	|*	|*	|*	|.
|[47][S]\rarr	|s	|e	|l	|s	|k	|i	|n	|*	|*	|*	|*	|*	|*	|*	|*	|*	|*	|*	|*	|*	|*	|*	|*	|*	|.
|*	|*	|*	|*	|*	|*	|*	|*	|*	|*	|*	|*	|*	|*	|*	|*	|*	|*	|*	|*	|*	|*	|*	|*	|*	|.\end{Puzzle}

\newpage

\begin{PuzzleClues}{\textbf{Poziome}\\}\Clue{1}{}{Thoracocharax stellatus - gatunek ryby z rodziny pstrążeniowatych (Gasteropelecidae); występuje w wodach słodkich Ameryki Południowej: w dorzeczach Parany, Amazonki i Orinoko}
\Clue{2}{}{zdrobniale: koncept - pomysł}
\Clue{3}{}{kobieta, która pozostaje z kimś w stałym lub przelotnym związku; również taka, z którą odbyło się stosunek seksualny}
\Clue{4}{}{żartobliwie o zakonniku}
\Clue{5}{}{człowiek, który się wymądrza, jest przemądrzały, nadmiernie chwali się swoją wiedzą i zdolnościami (często bezpodstawnie)}
\Clue{6}{}{teren udostępniony odwiedzającym, na którym hodowane są zwierzęta, najczęściej pochodzące z różnych obszarów geograficznych}
\Clue{7}{}{miasto w Danii na Płw. Jutlandzkim, port przy ujściu rzeki Gudena do zatoki Randers Fjord}
\Clue{8}{}{rodzaj lotnictwa bojowego, przeznaczony do bezpośredniego wsparcia wojsk lądowych, uzbrojony w opancerzone samoloty szturmowe}
\Clue{13}{}{MORESKA; ornament w formie splecionych wici roślinnych}
\Clue{14}{}{WSZECHŚWIAT}
\Clue{15}{}{zwyczajowa nazwa ryb z rodziny śledziowatych zaliczanych do sardynek}
\Clue{17}{}{to, że coś stara się dobrze, prawdziwie oddawać realia}
\Clue{20}{}{NITNIKOWCE; gromada słodkowodnych obleńców okrytych grubym oskórkiem}
\Clue{22}{}{urządzenie do teleterapii, stosowana w lecznictwie do zwalczania chorób nowotworowych, w defektoskopii, do sterylizacji żywności oraz w chemii radiacyjnej}
\Clue{24}{}{więzienie}
\Clue{27}{}{(1909-66) poeta i satyryk; „Myśli nieuczesane”, „Kpię i drogę pytam”, „Do Abla i Kaina”}
\Clue{30}{}{sir Alec (1800-1862) angielski żeglarz, odkrywca Ziemi Wiktorii}
\Clue{32}{}{Malus floribunda - gatunek niskiego drzewa lub krzewu owocowego z rodziny różowatych}
\Clue{34}{}{dział zoologii, nauka o wężach}
\Clue{43}{}{działo burzące, ładowane kamieniami}
\Clue{44}{}{ubytek materiału w metalu poddawanemu obróbce cieplnej}
\Clue{45}{}{abstrakcyjne pojęcie, pierwotnie służące do porównywania wielkości zbiorów, a później wielkości ciągłych, obecnie rozważane w matematyce w oderwaniu od ewentualnych fizycznych zastosowań}
\Clue{46}{}{pospólstwo, hołota - z niechęcią o tłumie ludzi}
\Clue{47}{}{skóra z foki lub z młodych niedźwiedzi morskich, zabarwiona na czarno}\end{PuzzleClues}

\begin{PuzzleClues}{\textbf{Pionowe}\\}\Clue{1}{}{Tragopogon floccosus - gatunek roślin należący do rodziny astrowatych}
\Clue{3}{}{pomieszczenie, najczęściej w wagonie kolejowym lub na statku, wydzielone z większej całości, o sprecyzowanej funkcji}
\Clue{9}{}{stan w płd-wsch. części USA, powierzchnia 109 tyś. km2, główne miasta: Memphis, Chattanooga, Nashville (stolica)}
\Clue{10}{}{Actenoides monachus  - gatunek leśnego ptaka z rzędu kraskowych (Coraciiformes), z rodziny zimorodkowatych (Alcedinidae), z podrodziny łowców (Halcyoninae)}
\Clue{11}{}{zawodnik, który zdobył srebrny medal}
\Clue{12}{}{jedna z metod przybliżania wartości całki oznaczonej funkcji rzeczywistej}
\Clue{16}{}{filologia rosyjska; nauka zajmująca się językiem i literaturą rosyjską}
\Clue{18}{}{system wag wykorzystywany do obliczeń wskaźnika cen towarów i usług konsumpcyjnych}
\Clue{19}{}{częściowo lub całkowicie wymyślona i fałszywa historia na temat kogoś lub czegoś, często bardzo ubarwiona}
\Clue{20}{}{dział automatyki zajmujacy się automatyzacją procesów wytwarzania i procesów technologicznych}
\Clue{21}{}{szybki wzrost bazy monetarnej, którego źródłem są nadwyżki bilansu płatniczego}
\Clue{22}{}{Megalancosaurus - nazwa rodzajowa diapsyda, żyjącego w górnym triasie na terenie obecnych północnych Włoch}
\Clue{23}{}{wspólna nazwa dla gatunków herbaty, przeważnie czarnej, pochodzących z prowincji Dardżyling (ang. Darjeeling) i okolic}
\Clue{24}{}{porcja kawy, napoju, zaparzonego z arabiki (popularnego gatunku kawy)}
\Clue{25}{}{stosunek seksualny, współżycie}
\Clue{26}{}{bardzo słodkie ciasto z karmelem, orzechami i czekoladą (przywodzące na myśl znany baton o tej samej nazwie)}
\Clue{27}{}{roczny cykl świąt, który rozpoczyna się w Kościele rzymskokatolickim od pierwszej niedzieli adwentu, w Kościele wschodnim od 1 września}
\Clue{28}{}{Pristiophoriformes - rząd ryb morskich z podgromady spodoustych; współcześnie grupa tych ryb zamieszkuje Ocean Indyjski i Ocean Spokojny}
\Clue{29}{}{realizowana programistycznie usługa świadczona poprzez sieć telekomunikacyjną, a w tym sieć komputerową, w szczególności przez Internet}
\Clue{30}{}{w górnictwie: stojak stalowy w szeregu wyznaczającym linię zawału}
\Clue{31}{}{przyrząd, złożony z dwóch oddziałujących na siebie przewodników, służący do pomiaru natężenia prądu}
\Clue{33}{}{młoda lub mała Cyganka}
\Clue{35}{}{wieś w Polsce położona w województwie dolnośląskim, w powiecie świdnickim, w gminie Strzegom}
\Clue{36}{}{zbroja, której korpus składająca się z dwóch części: ochraniającego pierś napierśnika i osłaniającego plecy naplecznika, połączonych za pomocą rzemieni w pasie i na ramionach}
\Clue{37}{}{mały stopień nasilenia czegoś; subtelność, wrażliwość}
\Clue{38}{}{tłustość, mętność}
\Clue{39}{}{małe pomieszczenie w domu lub mieszkaniu, gdzie są przechowywane ubrania}
\Clue{40}{}{miasto w Szkocji, główne miasto Szetlandów, na wyspie Mainland}
\Clue{41}{}{funkcjonujące w Polsce i sąsiednich krajach określenie rock and rolla}
\Clue{42}{}{kod ISO 4217 funta libańskiego}
\Clue{45}{}{w rzemiośle pszczelarskim, konstrukcja, najczęściej drewniana, używana do hodowli pszczół}\end{PuzzleClues}\newpage\section*{Krzyżówka 198}

\noindent\begin{Puzzle}{20}{31}|*	|*	|*	|[1][S]\darr	|*	|*	|*	|*	|*	|*	|*	|*	|*	|*	|*	|*	|*	|*	|*	|[2][S]\darr	|*	|.
|*	|*	|[3][S]\rarr	|l	|a	|s	|o	|n	|ó	|g	|[][,]{ }	|w	|i	|e	|l	|k	|i	|*	|*	|p	|*	|.
|[4][S]\drarr	|p	|i	|a	|s	|e	|k	|[][,]{ }	|z	|w	|a	|ł	|o	|w	|y	|*	|*	|*	|*	|e	|[5][S]\darr	|.
|g	|*	|*	|w	|[6][S]\darr	|*	|*	|*	|*	|*	|*	|*	|*	|*	|*	|[7][S]\darr	|*	|*	|*	|r	|i	|.
|r	|*	|[8][S]\rarr	|a	|p	|t	|*	|[9][S]\rarr	|p	|r	|z	|e	|d	|m	|i	|o	|t	|*	|*	|e	|e	|.
|a	|[10][S]\darr	|*	|[][,]{ }	|l	|*	|*	|*	|*	|*	|*	|[11][S]\drarr	|s	|i	|e	|w	|c	|a	|*	|ł	|s	|.
|n	|s	|[12][S]\darr	|p	|a	|[13][S]\rarr	|s	|p	|r	|z	|ę	|ż	|a	|j	|*	|s	|*	|*	|[14][S]\darr	|k	|i	|.
|d	|p	|z	|o	|n	|*	|*	|[15][S]\rarr	|l	|i	|m	|a	|n	|o	|w	|i	|a	|n	|k	|a	|*	|.
|a	|ó	|w	|d	|d	|*	|*	|*	|*	|*	|[16][S]\rarr	|b	|y	|c	|z	|k	|i	|*	|o	|*	|*	|.
|*	|ł	|i	|u	|e	|[17][S]\rarr	|l	|e	|p	|t	|o	|k	|l	|i	|d	|*	|*	|[18][S]\darr	|w	|*	|*	|.
|[19][S]\drarr	|g	|ą	|s	|k	|a	|[][,]{ }	|ż	|ó	|ł	|t	|a	|*	|*	|*	|*	|*	|h	|a	|*	|*	|.
|o	|ł	|z	|z	|a	|[20][S]\darr	|*	|*	|*	|*	|*	|[][,]{ }	|*	|*	|[21][S]\darr	|*	|*	|y	|l	|*	|*	|.
|f	|o	|e	|k	|*	|k	|*	|*	|*	|*	|[22][S]\rarr	|k	|u	|r	|w	|a	|*	|d	|i	|[23][S]\darr	|*	|.
|e	|s	|k	|o	|*	|a	|[24][S]\rarr	|k	|a	|n	|t	|a	|r	|y	|d	|a	|*	|r	|k	|t	|*	|.
|r	|k	|[][,]{ }	|w	|[25][S]\rarr	|p	|i	|e	|r	|w	|o	|r	|o	|d	|z	|t	|w	|o	|*	|e	|*	|.
|t	|a	|h	|a	|*	|e	|[26][S]\rarr	|s	|ł	|o	|w	|o	|*	|*	|i	|[27][S]\darr	|*	|f	|*	|m	|*	|.
|a	|[][,]{ }	|e	|*	|*	|l	|*	|*	|[28][S]\darr	|*	|[29][S]\darr	|l	|*	|*	|ę	|p	|*	|o	|*	|p	|*	|.
|[][,]{ }	|s	|t	|*	|*	|u	|*	|*	|a	|*	|k	|i	|[30][S]\darr	|[31][S]\darr	|k	|ł	|[32][S]\darr	|b	|*	|e	|*	|.
|w	|z	|e	|*	|*	|s	|*	|*	|s	|*	|ó	|ń	|b	|n	|[][,]{ }	|a	|s	|i	|*	|r	|*	|.
|a	|c	|r	|*	|*	|z	|*	|[33][S]\drarr	|p	|ó	|ł	|s	|i	|o	|s	|t	|r	|a	|*	|a	|*	|.
|r	|z	|o	|*	|*	|[][,]{ }	|*	|n	|i	|*	|e	|k	|e	|w	|ł	|e	|e	|*	|*	|t	|[34][S]\darr	|.
|i	|e	|c	|*	|*	|s	|*	|a	|r	|[35][S]\darr	|c	|a	|d	|o	|o	|c	|b	|*	|[36][S]\darr	|u	|c	|.
|a	|l	|y	|*	|*	|t	|*	|r	|a	|k	|z	|*	|a	|h	|n	|z	|r	|[37][S]\darr	|k	|r	|h	|.
|n	|i	|k	|*	|*	|o	|*	|o	|c	|o	|k	|*	|*	|u	|i	|e	|n	|k	|u	|a	|e	|.
|t	|n	|l	|*	|*	|s	|*	|s	|j	|t	|o	|*	|*	|c	|a	|k	|y	|o	|l	|[][,]{ }	|l	|.
|o	|o	|i	|*	|[38][S]\darr	|o	|[39][S]\rarr	|t	|a	|o	|*	|*	|*	|i	|*	|*	|[][,]{ }	|r	|e	|u	|i	|.
|w	|w	|c	|*	|k	|w	|*	|*	|*	|r	|*	|*	|*	|a	|*	|*	|e	|d	|b	|p	|o	|.
|a	|a	|z	|[40][S]\rarr	|m	|a	|s	|k	|a	|*	|[41][S]\rarr	|h	|e	|n	|r	|y	|k	|*	|i	|a	|s	|.
|*	|*	|n	|*	|[][S]2	|n	|*	|*	|*	|[42][S]\rarr	|n	|a	|s	|i	|ę	|ż	|r	|z	|a	|ł	|*	|.
|[43][S]\rarr	|t	|y	|p	|*	|y	|*	|[44][S]\rarr	|k	|o	|n	|w	|e	|n	|c	|j	|a	|*	|k	|u	|*	|.
|*	|*	|*	|*	|*	|*	|[45][S]\rarr	|o	|b	|ł	|o	|ś	|ć	|*	|*	|*	|n	|*	|*	|*	|*	|.
|*	|*	|*	|*	|*	|*	|*	|[46][S]\rarr	|f	|i	|n	|i	|t	|y	|z	|m	|*	|*	|*	|*	|*	|.\end{Puzzle}

\newpage

\begin{PuzzleClues}{\textbf{Poziome}\\}\Clue{3}{}{Mysis mixta - największy polski gatunek skorupiaka z rzędu lasonogów}
\Clue{4}{}{piasek pochodzenia polodowcowego o niejednolitej strukturze, niewypłukany i nieprzesiany przez wody lodowcowe}
\Clue{8}{}{astronauta amerykański w wyprawie Atlantisa w 1991 r}
\Clue{9}{}{to, czego się uczy w szkole, nauka, która została objęta programem nauczania}
\Clue{11}{}{rolnik}
\Clue{13}{}{konie lub bydło zaprzęgane do wozu albo pługa}
\Clue{15}{}{mieszkanka Limanowej}
\Clue{16}{}{konserwa rybna w ostrej zalewie pomidorowej}
\Clue{17}{}{Leptocleidus - nazwa rodzajowa plezjozaura żyjącego we wczesnej kredzie}
\Clue{19}{}{gatunek grzybów należący do rodziny gąskowatych}
\Clue{22}{}{prostytutka - kobieta odbywająca stosunki płciowe w celach zarobkowych}
\Clue{24}{}{MAJKA LEKARSKA; metalicznie zielony chrząszcz}
\Clue{25}{}{przenoście: geneza, pierwsze wystąpienie, pojawienie się czegoś, oryginalne pochodzenie}
\Clue{26}{}{podstawowa porcja informacji, na której operuje system komputerowy}
\Clue{33}{}{klacz tej samej matki, ale innego ojca}
\Clue{39}{}{w taoizmie: zasada leżąca u podstawy wszechświata, rządząca przyrodą wraz ze społeczeństwem jako jej częścią}
\Clue{40}{}{twarz o nieruchomym wyrazie}
\Clue{41}{}{imię średniowiecznych piastowskich książąt śląskich z tzw. monarchii Henryków Śląskich}
\Clue{42}{}{gatunek paproci o pojedynczych liściach}
\Clue{43}{}{rodzaj, odmiana}
\Clue{44}{}{spotkanie - najczęściej jakiegoś ugrupowania politycznego - w celu określenia najważniejszych celów i sposobów ich realizacji}
\Clue{45}{}{cecha tego, co jest obłe}
\Clue{46}{}{założenie w dowodzeniu naukowym mówiące, że każdy proces jest skończony}\end{PuzzleClues}

\begin{PuzzleClues}{\textbf{Pionowe}\\}\Clue{1}{}{lawa powstała w wyniku podwodnej erupcji, bardzo szybko stygnąca i dzieląca się na elipsoidalne, zwykle spłaszczone buły przypominające bochenki lub poduszki}
\Clue{2}{}{mała perła - wytwór małży, zbudowany z masy perłowej, wykorzystywany w jubilerstwie}
\Clue{4}{}{zdarzenie negatywnie lub niejednoznacznie oceniane społecznie, o potencjale medialnym i sile oddziaływania społecznego}
\Clue{5}{}{miasto we Włoszech (Marche); 41,8 tys. mieszkańców (1982)}
\Clue{6}{}{płachta materiału o stosunkowo dużych rozmiarach chroniąca towar przed przemoknięciem}
\Clue{7}{}{1) wieloletnia trawa luźnokępkowa - rajgras wyniosły 2) luźnokępkowa trawa pastewna - konietlica łąkowa}
\Clue{10}{}{spółgłoska, która powstaje, gdy narządy mowy w czasie artykulacji tworzą dostatecznie wąską szczelinę, by powstał szum, tarcie}
\Clue{11}{}{Gastrophryne carolinensis - gatunek płaza bezogonowego z rodziny wąskopyskowatych, występujący na południowo-wschodnich obszarach Ameryki Północnej, od Marylandu do Florydy na południu i do Teksasu na zachodzie}
\Clue{12}{}{związek należący do grupy pierścieniowych związków chemicznych, w którym co najmniej jeden układ cykliczny zawiera jeden lub więcej atomów pierwiastków innych niż węgiel}
\Clue{14}{}{BARGIEL}
\Clue{18}{}{lęk przed kontaktem z wodą}
\Clue{19}{}{oferta - składana jako propozycja realizacji zamówienia publicznego - która przewiduje, zgodnie z warunkami określonymi w specyfikacji istotnych warunków zamówienia, odmienny niż określony przez zamawiającego sposób wykonania zamówienia publicznego}
\Clue{20}{}{rodzaj kapelusza noszonego w wojsku i marynarce}
\Clue{21}{}{cecha człowieka o niezgrabnych i niezdarnych ruchach}
\Clue{23}{}{temperatura, w której dany materiał odkształca się pod wpływem nagrzanego powietrza}
\Clue{27}{}{zdrobniale: płatek - w botanice: listek korony kwiatowej}
\Clue{28}{}{wentylacja mechaniczna stosowana w przemyśle, mająca na celu jednoczesne odpylanie, usuwanie pary wodnej, chłodzenie surowca, maszyn, urządzeń i pomieszczeń}
\Clue{29}{}{małe kółko w zabawie dziecięcej}
\Clue{30}{}{biga. biedka, dwukółka}
\Clue{31}{}{mieszkaniec Nowej Huty}
\Clue{32}{}{telewizor, urządzenie elektroniczne przeznaczone do zdalnego odbioru ruchomego obrazu, który jest nadawany przez telewizję}
\Clue{33}{}{to, co narosło, nagromadziło się przez osadzanie się jakiejś materii na innej}
\Clue{34}{}{hokeista, obrońca Chicago, jeden z lepszych obrońców w NHL}
\Clue{35}{}{miasto i port w Jugosławii (Czarnogóra) nad Zatoką Kotorską kąpielisko i ośrodek turystyczny}
\Clue{36}{}{duży pieczony pieróg nadziewany siekanym mięsem, rybą, kapustą lub grzybami}
\Clue{37}{}{sztruks - rodzaj prążkowanej tkaniny na odzież roboczą}
\Clue{38}{}{jednostka wielokrotna jednostki pola powierzchni - metra kwadratowego}\end{PuzzleClues}\newpage\section*{Krzyżówka 199}

\noindent\begin{Puzzle}{19}{23}|*	|[1][S]\darr	|*	|*	|*	|*	|*	|*	|[2][S]\darr	|*	|*	|*	|*	|*	|*	|*	|*	|[3][S]\darr	|*	|[4][S]\darr	|.
|*	|s	|*	|*	|[5][S]\rarr	|s	|k	|o	|w	|r	|o	|n	|e	|k	|*	|*	|[6][S]\darr	|m	|[7][S]\darr	|z	|.
|[8][S]\rarr	|z	|n	|a	|k	|[][,]{ }	|z	|a	|p	|y	|t	|a	|n	|i	|a	|*	|c	|e	|p	|a	|.
|*	|e	|*	|*	|*	|*	|[9][S]\drarr	|p	|l	|a	|s	|t	|r	|o	|n	|*	|z	|t	|u	|r	|.
|*	|n	|[10][S]\darr	|[11][S]\rarr	|e	|m	|b	|l	|e	|m	|a	|t	|*	|*	|*	|*	|y	|a	|n	|a	|.
|*	|i	|w	|[12][S]\rarr	|t	|r	|a	|n	|s	|e	|p	|t	|*	|*	|*	|*	|t	|l	|k	|d	|.
|*	|c	|a	|[13][S]\darr	|*	|*	|s	|[14][S]\rarr	|z	|u	|b	|o	|ż	|e	|n	|i	|e	|*	|t	|n	|.
|*	|*	|r	|f	|[15][S]\darr	|*	|a	|*	|c	|*	|*	|*	|[16][S]\darr	|*	|*	|*	|l	|*	|[][,]{ }	|o	|.
|*	|[17][S]\drarr	|k	|u	|k	|u	|ł	|c	|z	|e	|[][,]{ }	|j	|a	|j	|o	|*	|n	|[18][S]\darr	|b	|ś	|.
|*	|p	|a	|n	|t	|*	|y	|*	|*	|*	|*	|*	|n	|*	|*	|*	|i	|u	|a	|ć	|.
|*	|r	|*	|d	|o	|*	|k	|*	|*	|*	|*	|[19][S]\rarr	|t	|u	|t	|k	|a	|r	|z	|*	|.
|*	|z	|*	|u	|ś	|*	|*	|[20][S]\rarr	|k	|u	|r	|z	|y	|s	|k	|o	|*	|a	|o	|*	|.
|*	|e	|*	|m	|*	|*	|[21][S]\darr	|*	|[22][S]\rarr	|k	|w	|i	|l	|a	|j	|a	|*	|n	|w	|*	|.
|*	|s	|[23][S]\darr	|*	|*	|*	|g	|*	|*	|*	|[24][S]\rarr	|b	|o	|e	|i	|n	|g	|*	|y	|*	|.
|*	|u	|a	|*	|*	|*	|a	|[25][S]\rarr	|c	|h	|r	|u	|p	|k	|o	|ś	|ć	|*	|*	|*	|.
|*	|w	|s	|*	|*	|[26][S]\darr	|j	|*	|*	|*	|[27][S]\rarr	|l	|a	|w	|e	|t	|a	|*	|*	|*	|.
|*	|a	|s	|*	|[28][S]\rarr	|b	|o	|c	|z	|n	|i	|k	|*	|*	|*	|*	|*	|*	|*	|*	|.
|*	|l	|e	|*	|*	|i	|w	|*	|*	|*	|*	|*	|*	|*	|*	|*	|*	|*	|*	|*	|.
|*	|n	|m	|*	|[29][S]\rarr	|n	|i	|e	|z	|g	|ł	|ę	|b	|i	|o	|n	|o	|ś	|ć	|*	|.
|*	|o	|b	|*	|[30][S]\drarr	|d	|e	|l	|i	|k	|a	|t	|n	|o	|ś	|ć	|*	|*	|*	|*	|.
|[31][S]\rarr	|ś	|l	|i	|m	|a	|c	|z	|n	|i	|c	|a	|*	|*	|*	|*	|*	|*	|*	|*	|.
|*	|ć	|e	|*	|a	|*	|*	|*	|*	|*	|*	|*	|*	|*	|*	|*	|*	|*	|*	|*	|.
|*	|*	|r	|[32][S]\rarr	|c	|a	|ł	|k	|a	|[][,]{ }	|p	|i	|e	|r	|w	|s	|z	|a	|*	|*	|.
|*	|*	|*	|*	|*	|*	|*	|*	|*	|*	|*	|*	|*	|*	|*	|*	|*	|*	|*	|*	|.\end{Puzzle}

\newpage

\begin{PuzzleClues}{\textbf{Poziome}\\}\Clue{5}{}{ptak z rodziny skowronków (Alaudidae)}
\Clue{8}{}{znak interpunkcyjny, który umieszcza się na końcu zdania pytającego}
\Clue{9}{}{usztywniony przód koszuli frakowej noszony w XIX w}
\Clue{11}{}{rodzaj gestu, którego znaczenie jest jednoznaczne i właściwe dla danej kultury, tzw. gest kulturowy}
\Clue{12}{}{nawa prostopadła do osi kościoła, położona pomiędzy prezbiterium, a resztą jego budynku}
\Clue{14}{}{stanie się mniej bogatym, różnorodnym; też: pogorszenie się, utracenie jakości}
\Clue{17}{}{zrzucany na kogoś kłopot lub problematyczny obowiązek}
\Clue{19}{}{chrząszcz, którego larwy żyją i żerują w liściach}
\Clue{20}{}{kurz - kiedy jest go dużo}
\Clue{22}{}{mydłokrzew, mydłodrzew, mydłoka, Quillaja - drzewo z rodziny mydłokrzewowatych; pochodzi z Ameryki Południowej}
\Clue{24}{}{typ amerykańskiego samolotu wojskowego, pasażerskiego}
\Clue{25}{}{cecha jedzenia, które podczas gryzienia chrupie, jest kruche lub jędrne}
\Clue{27}{}{podstawa broni palnej służąca do mocowania lufy i manewrowania działem}
\Clue{28}{}{strug do obróbki trudno dostępnych miejsc}
\Clue{29}{}{cecha czegoś, co jest niezbadane, nieznane, nieprzeniknione}
\Clue{30}{}{cecha fizyczna czegoś, co jest łatwo zniszczyć}
\Clue{31}{}{woluta - ornament w kształcie zbliżonym do litery S zakończony spiralami na obu końcach}
\Clue{32}{}{funkcja przyjmująca stałą wartość na trajektoriach rozwiązań równania różniczkowego (lub układu równań różniczkowych)}\end{PuzzleClues}

\begin{PuzzleClues}{\textbf{Pionowe}\\}\Clue{1}{}{(1904-87), pisarz, publicysta i prawnik; „Pitawal warszawski”, „Maria Kalergis”, „Bratanek ostatniego króla”}
\Clue{2}{}{NARZĘPIK}
\Clue{3}{}{osoba, która słucha metalu, członek subkultury (“nie jestem metalem, punkiem ani nawet discopolową cizią”)}
\Clue{4}{}{cecha człowieka, który jest obrotny, przedsiębiorczy i  potrafi radzić sobie w życiu}
\Clue{6}{}{pomieszczenie w obrębie biblioteki lub innej placówki kultury, przeznaczone do czytania}
\Clue{7}{}{jedna setna punktu procentowego}
\Clue{9}{}{ślad, siniec, pręga pozostała po uderzeniu np. rózgą}
\Clue{10}{}{proces produkcji piwa}
\Clue{13}{}{główny folwark wśród tych, które były własnością jednej osoby, także obszar wokół niego}
\Clue{15}{}{osobistość, osoba ważna, licząca się}
\Clue{16}{}{ssak z rodziny krętorogich; stepy i sawanny Afryki, żyje także z Azji i w Europie}
\Clue{17}{}{cecha czegoś, co jest przesuwalne, co można przesuwać, co daje się przesuwać}
\Clue{18}{}{siódma według oddalenia od Słońca planeta Układu Słonecznego (Herschel)}
\Clue{21}{}{Galeobdolon - rodzaj roślin z rodziny jasnotowatych}
\Clue{23}{}{program tworzący kod maszynowy na podstawie kodu źródłowego (tzw. asemblacja) wykonanego w niskopoziomowym języku programowania bazującym na podstawowych operacjach procesora zwanym językiem asemblera}
\Clue{26}{}{kobieca opaska na głowę, aksamitna lub jedwabna, rodzaj diademu do przystrajania fryzury}
\Clue{30}{}{firma kosmetyczna założona w 1984}\end{PuzzleClues}\newpage\section*{Krzyżówka 200}

\noindent\begin{Puzzle}{16}{32}|*	|*	|*	|*	|*	|[1][S]\drarr	|k	|u	|w	|e	|j	|t	|c	|z	|y	|k	|*	|.
|*	|*	|*	|*	|[2][S]\rarr	|k	|o	|ź	|l	|a	|r	|e	|k	|*	|*	|*	|*	|.
|*	|*	|[3][S]\rarr	|n	|a	|w	|ó	|j	|*	|[4][S]\drarr	|f	|t	|a	|l	|a	|n	|*	|.
|*	|[5][S]\rarr	|g	|ó	|r	|a	|l	|s	|k	|o	|ś	|ć	|*	|[6][S]\darr	|*	|[7][S]\darr	|*	|.
|*	|*	|[8][S]\darr	|[9][S]\darr	|[10][S]\darr	|d	|*	|[11][S]\darr	|[12][S]\darr	|m	|*	|[13][S]\drarr	|c	|i	|o	|s	|*	|.
|*	|[14][S]\drarr	|s	|u	|b	|r	|e	|t	|k	|a	|*	|k	|*	|s	|*	|ł	|*	|.
|*	|h	|z	|n	|o	|a	|*	|r	|ą	|t	|[15][S]\darr	|ą	|*	|s	|*	|o	|*	|.
|*	|y	|e	|i	|k	|t	|*	|o	|t	|n	|p	|d	|*	|o	|*	|d	|*	|.
|*	|l	|a	|w	|o	|*	|*	|c	|*	|i	|o	|z	|*	|u	|[16][S]\darr	|y	|*	|.
|*	|e	|t	|e	|c	|*	|*	|h	|*	|k	|s	|i	|*	|d	|g	|c	|*	|.
|*	|m	|*	|r	|h	|*	|*	|u	|[17][S]\darr	|o	|i	|o	|*	|u	|a	|z	|*	|.
|*	|o	|*	|e	|ó	|*	|*	|s	|s	|w	|e	|ł	|*	|n	|r	|*	|*	|.
|*	|r	|*	|k	|d	|*	|*	|*	|i	|a	|l	|e	|*	|*	|n	|*	|*	|.
|*	|f	|*	|*	|*	|*	|*	|*	|e	|t	|e	|k	|*	|*	|u	|*	|[18][S]\darr	|.
|*	|i	|*	|[19][S]\rarr	|m	|c	|l	|a	|r	|e	|n	|*	|*	|*	|s	|*	|p	|.
|*	|z	|[20][S]\rarr	|d	|z	|w	|o	|n	|o	|*	|i	|*	|*	|*	|z	|*	|y	|.
|*	|m	|*	|*	|*	|*	|*	|[21][S]\darr	|w	|*	|e	|*	|*	|*	|e	|*	|l	|.
|*	|*	|[22][S]\rarr	|w	|y	|p	|i	|s	|*	|*	|*	|*	|*	|*	|k	|*	|o	|.
|*	|[23][S]\drarr	|s	|t	|r	|e	|e	|t	|w	|o	|r	|k	|e	|r	|*	|*	|n	|.
|*	|b	|[24][S]\drarr	|b	|u	|ł	|k	|a	|[][,]{ }	|m	|a	|ś	|l	|a	|n	|a	|*	|.
|*	|a	|z	|[25][S]\drarr	|s	|t	|o	|r	|y	|t	|e	|l	|l	|i	|n	|g	|*	|.
|*	|r	|g	|i	|*	|[26][S]\rarr	|m	|o	|c	|*	|[27][S]\darr	|*	|*	|*	|*	|*	|*	|.
|*	|*	|u	|n	|*	|[28][S]\rarr	|p	|r	|z	|e	|d	|d	|z	|i	|e	|ń	|*	|.
|*	|[29][S]\darr	|b	|k	|[30][S]\drarr	|p	|l	|a	|j	|t	|a	|*	|*	|*	|*	|*	|*	|.
|*	|m	|i	|l	|m	|*	|*	|k	|[31][S]\rarr	|k	|r	|a	|j	|c	|a	|r	|*	|.
|*	|o	|c	|i	|a	|[32][S]\rarr	|s	|i	|m	|e	|o	|n	|i	|*	|*	|*	|*	|.
|[33][S]\drarr	|s	|i	|n	|g	|e	|l	|*	|*	|[34][S]\rarr	|c	|e	|z	|a	|*	|*	|*	|.
|e	|k	|e	|a	|i	|[35][S]\rarr	|z	|o	|o	|c	|h	|o	|r	|i	|a	|*	|*	|.
|s	|i	|l	|t	|e	|*	|[36][S]\drarr	|h	|i	|m	|a	|l	|a	|j	|e	|*	|*	|.
|p	|t	|*	|o	|r	|*	|r	|*	|*	|*	|*	|*	|*	|*	|*	|*	|*	|.
|*	|ó	|[37][S]\rarr	|r	|a	|k	|a	|s	|*	|*	|*	|*	|*	|*	|*	|*	|*	|.
|*	|w	|*	|*	|*	|[38][S]\rarr	|k	|l	|i	|p	|e	|r	|*	|*	|*	|*	|*	|.
|*	|*	|*	|*	|*	|*	|*	|*	|*	|*	|*	|*	|*	|*	|*	|*	|*	|.\end{Puzzle}

\newpage

\begin{PuzzleClues}{\textbf{Poziome}\\}\Clue{1}{}{mieszkaniec Kuwejtu, człowiek pochodzenia kuwejckiego}
\Clue{2}{}{inaczej koźlarz - rodzaj grzybów z rodziny borowikowatych; grzyb kapeluszowy o dużych lub średnich owocnikach}
\Clue{3}{}{element maszyny, wał, na który coś jest nawijane lub z którego coś jest rozwijane}
\Clue{4}{}{ester kwasu ftalowego}
\Clue{5}{}{zespół cech typowych dla górali}
\Clue{13}{}{stale rosnący siekacz słonia; kieł}
\Clue{14}{}{postać z operetki lub opery komicznej, wywodząca się z komedii dell'arte; sprytna pokojówka, służąca posługująca się intrygami}
\Clue{19}{}{marka samochodu; założona w 1989 spółka zajmująca się projektowaniem i produkcją seryjnych samochodów opartych na technologii Formuły 1}
\Clue{20}{}{w górnictwie: warstwa wybieranej soli przy wybieraniu dwuwarstwowym}
\Clue{22}{}{wyimek wypisany z jakiegoś tekstu}
\Clue{23}{}{pracownik socjalny, który prowadzi działalność w środowisku osób znajdujących się poza marginesem społecznym, często po prostu na ulicy}
\Clue{24}{}{rodzaj bułki}
\Clue{25}{}{metoda wizualizacji w komunikacji, która może być wykorzystywana jako metoda badawcza lub strategia komunikacyjna}
\Clue{26}{}{prawomocność - stan prawny odnoszący się do rozstrzygnięć organów administracyjnych oraz do orzeczeń sądowych, mający miejsce wówczas, gdy orzeczenie sądowe lub rozstrzygnięcie organu administracji nie może być zmienione czy uchylone poprzez wniesienie środka odwoławczego lub zaskarżenia}
\Clue{28}{}{dzień bezpośrednio poprzedzający jakiś inny, ważny dzień}
\Clue{30}{}{przenośnie: niepowodzenie przedsięwzięcia}
\Clue{31}{}{moneta srebrna bita od XIII wieku, początkowo w Tyrolu}
\Clue{32}{}{lekkoatletka włoska, mistrzyni olimpijska z Moskwy w skoku wzwyż}
\Clue{33}{}{często mniejsza rozmiarem płyta, która zawiera jeden utwór, służy do promocji nowej, większej płyty}
\Clue{34}{}{sieć, niewód do połowu ryb dennych}
\Clue{35}{}{rozsiewanie nasion lub owoców przez zwierzęta}
\Clue{36}{}{przenośnie: szczyt, wyżyny czegoś}
\Clue{37}{}{święte jezioro buddystów w zachodnich Himalajach (Chiny)}
\Clue{38}{}{żaglowiec o smukłym kadłubie i bardzo wysokich masztach służący w XIX w. do przewozu herbat}\end{PuzzleClues}

\begin{PuzzleClues}{\textbf{Pionowe}\\}\Clue{1}{}{prostokąt, który ma wszystkie boki równe}
\Clue{4}{}{Theridiidae - rodzina pająków z podrzędu Opisthothela, obejmująca ponad 2350 gatunków podzielonych na ok. 120 rodzajów; w Polsce występuje ponad 60 gatunków, głównie z rodzajów: Achaearanea, Dipoena, Steatoda, Theridion}
\Clue{6}{}{miasto w środkowej Francji nad płd.-zach. od miasto Bourges; przemysł metalowy,}
\Clue{7}{}{cecha kogoś lub czegoś, kto/co jest słodki, słodko wygląda}
\Clue{8}{}{gwiazda w gwiazdozbiorze Pegaza}
\Clue{9}{}{studenckie określenie uniwersytetu - siedziby uczelni, miejsca, gdzie m.in. odbywają się zajęcia dydaktyczne}
\Clue{10}{}{rodzaj pająka z rodziny ukośnikowatych}
\Clue{11}{}{ślimak morski o dekoracyjnej muszli}
\Clue{12}{}{przestrzeń (róg) między dwiema stykającymi się powierzchniami}
\Clue{13}{}{brodawka przędna pająka}
\Clue{14}{}{pogląd filozoficzny, hilemorfizm}
\Clue{15}{}{osiedlenie (przymusowe) na wschodzie Rosji (zwykle na Syberii)}
\Clue{16}{}{kubek, naczynie z uchem, z którego można pić}
\Clue{17}{}{Władymir (1910-68) malarz radziecki, reprezentant realizmu socjalistycznego, tematyka historyczno-rewolucyjna}
\Clue{18}{}{jedna z podpór w konstrukcji mostu}
\Clue{21}{}{Merostomata - gromada szczękoczułkowców obejmująca 4 gatunki żyjące współcześnie i około 500 gatunków kopalnych; zalicza się do nich stawonogi z 6-7 parami odnóży na głowotułowiu, blaszkowatymi odnóżami na odwłoku, na których znajdują się wyrostki skrzelowe, a wreszcie szpiczasto zakończonym telsonem}
\Clue{23}{}{miasto położone na Ukrainie, nad rzeką Rów, na Podolu, w obwodzie winnickim, siedziba władz rejonu barskiego}
\Clue{24}{}{człowiek, który doprowadza kogoś do zguby}
\Clue{25}{}{przyrząd do korygowania dewiacji magnetycznych busoli}
\Clue{27}{}{pilot polski, drużynowy mistrz świata w lataniu precyzyjnym oraz indywidualny z 1994 r}
\Clue{29}{}{zatoka Morza Karaibskiego w północnych wybrzeży Panamy}
\Clue{30}{}{okrągła lub czworokątna czapka pozbawiona daszka, o główce płaskiej, okrągłej bądź kwadratowej wykonana z grubego sukna i obszyta tzw.barankiem}
\Clue{33}{}{kod ISO 4217 pesety}
\Clue{36}{}{skorupiak z rzędu dziesięcionogów}\end{PuzzleClues}


\end{document}
